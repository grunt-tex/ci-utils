% \iffalse meta-comment
%
% Copyright 1993-2014
% The LaTeX3 Project and any individual authors listed elsewhere
% in this file.
%
% This file is part of the LaTeX base system.
% -------------------------------------------
%
% It may be distributed and/or modified under the
% conditions of the LaTeX Project Public License, either version 1.3c
% of this license or (at your option) any later version.
% The latest version of this license is in
%    http://www.latex-project.org/lppl.txt
% and version 1.3c or later is part of all distributions of LaTeX
% version 2005/12/01 or later.
%
% This file has the LPPL maintenance status "maintained".
%
% The list of all files belonging to the LaTeX base distribution is
% given in the file `manifest.txt'. See also `legal.txt' for additional
% information.
%
% The list of derived (unpacked) files belonging to the distribution
% and covered by LPPL is defined by the unpacking scripts (with
% extension .ins) which are part of the distribution.
%
% \fi
%
% \iffalse
%%% From File: ltfinal.dtx
%
%<*driver>
% \fi
\ProvidesFile{ltfinal.dtx}
             [2014/09/29 v1.1e LaTeX Kernel (Final Settings)]
% \iffalse
\documentclass{ltxdoc}
\GetFileInfo{ltfinal.dtx}
\title{\filename}
\date{\filedate}
\author{%
  Johannes Braams\and
  David Carlisle\and
  Alan Jeffrey\and
  Leslie Lamport\and
  Frank Mittelbach\and
  Chris Rowley\and
  Rainer Sch\"opf}
\begin{document}
\maketitle
 \DocInput{ltfinal.dtx}
\end{document}
%</driver>
% \fi
%
% \CheckSum{472}
%
% \section{Final settings}
% This section contains the final settings for \LaTeX.  It initialises
% some debugging and typesetting parameters, sets the default
% |\catcode|s and uc/lc codes, and inputs the hyphenation file.
%
% \StopEventually{}
%
% \changes{v0.1a}{1994/03/07}{Initial version, split from latex.dtx}
% \changes{v0.1a}{1994/03/07}{Remove oldcomments environment}
% \changes{v0.1c}{1994/04/21}{Added comments, set the catcodes of
%    128--255.}
% \changes{v0.1d}{1994/04/23}{Check that \cs{font@submax} is still zero}
% \changes{v0.1e}{1994/05/02}{Set all the catcodes}
% \changes{v0.1f}{1994/05/03}{Set the catcode of control-J to be
%    `other', for use in messages.}
% \changes{v0.1g}{1994/05/05}{Added empty errhelp.}
% \changes{v0.1h}{1994/05/13}{Added package ot1enc, and defined
%    \cs{@acci}, \cs{@accii} and \cs{@acciii}.}
% \changes{v0.1j}{1994/05/18}{Corrected the lccode for d-bar.}
% \changes{v0.1k}{1994/05/19}{Removed \cs{makeat...}}
% \changes{v1.0n}{1994/05/31}{Renamed lthyphen.* to lthyphen.*.}
% \changes{v1.0o}{1994/11/17}
%         {\cs{@tempa} to \cs{reserved@a}}
% \changes{v1.0p}{1994/12/01}
%         {Renamed lthyphen.* to hyphen.*.}
% \changes{v1.0r}{1995/06/05}
%         {Added \cs{MakeUppercase} and \cs{MakeLowercase}.}
% \changes{v1.0s}{1995/06/06}
%         {Made \cs{MakeUppercase} and \cs{MakeLowercase} brace their
%         argument.}
%
% \subsection{Debugging}
%
% By default, \LaTeX{} shows statistics:
%    \begin{macrocode}
%<*2ekernel>
\tracingstats1
%    \end{macrocode}
%
% \subsection{Typesetting parameters}
%
% \begin{macro}{\@lowpenalty}
% \begin{macro}{\@medpenalty}
% \begin{macro}{\@highpenalty}
%    These are penalties used internally.
%    \begin{macrocode}
\newcount\@lowpenalty
\newcount\@medpenalty
\newcount\@highpenalty
%    \end{macrocode}
% \end{macro}
% \end{macro}
% \end{macro}
% The default values of the picture and |\fbox| parameters:
%    \begin{macrocode}
\unitlength = 1pt
\fboxsep = 3pt
\fboxrule = .4pt
%    \end{macrocode}
% The saved value of \TeX's |\maxdepth|:
%    \begin{macrocode}
\@maxdepth       = \maxdepth
%    \end{macrocode}
% |\vsize| initialized because a |\clearpage| with |\vsize < \topskip|
%  causes trouble.
% |\@colroom| and |\@colht| also initialized because |\vsize| may be
%  set to them if a |\clearpage| is done before the |\begin{document}|
%
%    \begin{macrocode}
\vsize = 1000pt
\@colroom = \vsize
\@colht = \vsize
%    \end{macrocode}
% Initialise |\textheight| |\textwidth| and page style, to avoid
% internal errors if they are not set by the class.
% \changes{v0.1b}{1994/04/18}
%         {Initialise \cs{textheight}, \cs{textwidth} and page style}
%    \begin{macrocode}
\textheight=.5\maxdimen
\textwidth=\textheight
\ps@empty
%    \end{macrocode}
%
% \subsection{Lccodes for hyphenation}
%
% \changes{v1.1b}{1998/05/20}{Set up lccodes before loading
%    hyphenation files: pr/2639}
%    We set things up so that hyphenation files can assume that the
%    default (T1) lccodes are in use (at present this also sets up the
%    uccodes).
%    We temporarily define |\reserved@a| to apply |\reserved@c| to
%    all the numbers in the range of its arguments.
%    \begin{macrocode}
\def\reserved@a#1#2{%
   \@tempcnta#1\relax
   \@tempcntb#2\relax
   \reserved@b
}
\def\reserved@b{%
   \ifnum\@tempcnta>\@tempcntb\else
      \reserved@c\@tempcnta
      \advance\@tempcnta\@ne
      \expandafter\reserved@b
   \fi
}
%    \end{macrocode}
%    Depending on the \TeX{} version, we might not be allowed to do
%    this for non-ASCII characters.
% \changes{v1.0n}{1994/06/09}{For \TeX2, do not set codes for higher
%                   half of character table.}
%    \begin{macrocode}
\def\reserved@c#1{%
   \count@=#1\advance\count@ by -"20
   \uccode#1=\count@
   \lccode#1=#1
}
\reserved@a{`\a}{`\z}
\ifnum\inputlineno=\m@ne\else
  \reserved@a{"A0}{"BC}
  \reserved@a{"E0}{"FF}
\fi
%    \end{macrocode}
% The upper case characters need their |\uccode| and |\lccode| values
% set, and their |\sfcode| set to 999.
%    \begin{macrocode}
\def\reserved@c#1{%
   \count@=#1\advance\count@ by "20
   \uccode#1=#1
   \lccode#1=\count@
   \sfcode#1=999
}
\reserved@a{`\A}{`\Z}
\ifnum\inputlineno=\m@ne\else
  \reserved@a{"80}{"9C}
  \reserved@a{"C0}{"DF}
\fi
%    \end{macrocode}
% Well, it would be nice if that were correct, but unfortunately, the
% Cork encoding contains some odd slots whose uccode or lccode isn't
% quite what you'd expect.
%    \begin{macrocode}
\uccode`\^^Y=`\I     % dotless i
\lccode`\^^Y=`\^^Y   % dotless i
\uccode`\^^Z=`\J     % dotless j, ae in OT1
\lccode`\^^Z=`\^^Z   % dotless j, ae in OT1
\ifnum\inputlineno=\m@ne\else
  \lccode`\^^9d=`\i    % dotted I
  \uccode`\^^9d=`\^^9d % dotted I
  \lccode`\^^9e=`\^^9e % d-bar
  \uccode`\^^9e=`\^^d0 % d-bar
\fi
%    \end{macrocode}
% Finally here is one that helps hyphenation in the OT1 encoding.
% \changes{v1.0z}{1996/10/31}
%    {Added extra \cs{lcode}, hoping it does no harm in T1 (pr/1969)}
%    \begin{macrocode}
\lccode`\^^[=`\^^[   % oe in OT1
%    \end{macrocode}
%
% And we also set the |\lccode| of |\-| and |\textcompwordmark| so
% that they do not prevent hyphenation in the remainder of the word
% (as suggested by Lars Helstr\"om).
% \changes{v1.1e}{2003/10/13}
%    {Added extra \cs{lccode} for \cs{-} and \cs{textcompwordmark}}
%    \begin{macrocode}
\lccode`\- =`\-   % default hyphen char
\lccode 127=127   % alternate hyphen char
\lccode 23 =23    % textcompwordmark in T1
%    \end{macrocode}
%
% \subsection{Hyphenation}
%
% \changes{v0.1a}{1994/03/07}{move code here from lhyphen.dtx}
% \changes{v0.1a}{1994/03/07}
%         {use \cs{InputIfFileExists} not \cs{IfFileExists}}
% \changes{v1.0x}{1995/11/01}
%      {(DPC) Switch meaning of \cs{@addtofilelist} for cfg files}%
% The following code will be compiled into the format file. It checks
% for the existence of \texttt{hyphen.cfg} in inputs that file if
% found. Otherwise it inputs \texttt{hyphen.ltx}.  Note that these
% are loaded in \emph{before} the |\catcode|s are set, so local
% hyphenation files can use 8-bit input.
%
% We try to load the customized hyphenation description file.
%    \begin{macrocode}
\InputIfFileExists{hyphen.cfg}
           {\typeout{===========================================^^J%
                      Local configuration file hyphen.cfg used^^J%
                     ===========================================}%
             \def\@addtofilelist##1{\xdef\@filelist{\@filelist,##1}}%
           }
           {% The Plain TeX hyphenation tables [NOT TO BE CHANGED IN ANY WAY!]
% Unlimited copying and redistribution of this file are permitted as long
% as this file is not modified. Modifications are permitted, but only if
% the resulting file is not named hyphen.tex.
\patterns{ % just type <return> if you're not using INITEX
.ach4
.ad4der
.af1t
.al3t
.am5at
.an5c
.ang4
.ani5m
.ant4
.an3te
.anti5s
.ar5s
.ar4tie
.ar4ty
.as3c
.as1p
.as1s
.aster5
.atom5
.au1d
.av4i
.awn4
.ba4g
.ba5na
.bas4e
.ber4
.be5ra
.be3sm
.be5sto
.bri2
.but4ti
.cam4pe
.can5c
.capa5b
.car5ol
.ca4t
.ce4la
.ch4
.chill5i
.ci2
.cit5r
.co3e
.co4r
.cor5ner
.de4moi
.de3o
.de3ra
.de3ri
.des4c
.dictio5
.do4t
.du4c
.dumb5
.earth5
.eas3i
.eb4
.eer4
.eg2
.el5d
.el3em
.enam3
.en3g
.en3s
.eq5ui5t
.er4ri
.es3
.eu3
.eye5
.fes3
.for5mer
.ga2
.ge2
.gen3t4
.ge5og
.gi5a
.gi4b
.go4r
.hand5i
.han5k
.he2
.hero5i
.hes3
.het3
.hi3b
.hi3er
.hon5ey
.hon3o
.hov5
.id4l
.idol3
.im3m
.im5pin
.in1
.in3ci
.ine2
.in2k
.in3s
.ir5r
.is4i
.ju3r
.la4cy
.la4m
.lat5er
.lath5
.le2
.leg5e
.len4
.lep5
.lev1
.li4g
.lig5a
.li2n
.li3o
.li4t
.mag5a5
.mal5o
.man5a
.mar5ti
.me2
.mer3c
.me5ter
.mis1
.mist5i
.mon3e
.mo3ro
.mu5ta
.muta5b
.ni4c
.od2
.odd5
.of5te
.or5ato
.or3c
.or1d
.or3t
.os3
.os4tl
.oth3
.out3
.ped5al
.pe5te
.pe5tit
.pi4e
.pio5n
.pi2t
.pre3m
.ra4c
.ran4t
.ratio5na
.ree2
.re5mit
.res2
.re5stat
.ri4g
.rit5u
.ro4q
.ros5t
.row5d
.ru4d
.sci3e
.self5
.sell5
.se2n
.se5rie
.sh2
.si2
.sing4
.st4
.sta5bl
.sy2
.ta4
.te4
.ten5an
.th2
.ti2
.til4
.tim5o5
.ting4
.tin5k
.ton4a
.to4p
.top5i
.tou5s
.trib5ut
.un1a
.un3ce
.under5
.un1e
.un5k
.un5o
.un3u
.up3
.ure3
.us5a
.ven4de
.ve5ra
.wil5i
.ye4
4ab.
a5bal
a5ban
abe2
ab5erd
abi5a
ab5it5ab
ab5lat
ab5o5liz
4abr
ab5rog
ab3ul
a4car
ac5ard
ac5aro
a5ceou
ac1er
a5chet
4a2ci
a3cie
ac1in
a3cio
ac5rob
act5if
ac3ul
ac4um
a2d
ad4din
ad5er.
2adi
a3dia
ad3ica
adi4er
a3dio
a3dit
a5diu
ad4le
ad3ow
ad5ran
ad4su
4adu
a3duc
ad5um
ae4r
aeri4e
a2f
aff4
a4gab
aga4n
ag5ell
age4o
4ageu
ag1i
4ag4l
ag1n
a2go
3agog
ag3oni
a5guer
ag5ul
a4gy
a3ha
a3he
ah4l
a3ho
ai2
a5ia
a3ic.
ai5ly
a4i4n
ain5in
ain5o
ait5en
a1j
ak1en
al5ab
al3ad
a4lar
4aldi
2ale
al3end
a4lenti
a5le5o
al1i
al4ia.
ali4e
al5lev
4allic
4alm
a5log.
a4ly.
4alys
5a5lyst
5alyt
3alyz
4ama
am5ab
am3ag
ama5ra
am5asc
a4matis
a4m5ato
am5era
am3ic
am5if
am5ily
am1in
ami4no
a2mo
a5mon
amor5i
amp5en
a2n
an3age
3analy
a3nar
an3arc
anar4i
a3nati
4and
ande4s
an3dis
an1dl
an4dow
a5nee
a3nen
an5est.
a3neu
2ang
ang5ie
an1gl
a4n1ic
a3nies
an3i3f
an4ime
a5nimi
a5nine
an3io
a3nip
an3ish
an3it
a3niu
an4kli
5anniz
ano4
an5ot
anoth5
an2sa
an4sco
an4sn
an2sp
ans3po
an4st
an4sur
antal4
an4tie
4anto
an2tr
an4tw
an3ua
an3ul
a5nur
4ao
apar4
ap5at
ap5ero
a3pher
4aphi
a4pilla
ap5illar
ap3in
ap3ita
a3pitu
a2pl
apoc5
ap5ola
apor5i
apos3t
aps5es
a3pu
aque5
2a2r
ar3act
a5rade
ar5adis
ar3al
a5ramete
aran4g
ara3p
ar4at
a5ratio
ar5ativ
a5rau
ar5av4
araw4
arbal4
ar4chan
ar5dine
ar4dr
ar5eas
a3ree
ar3ent
a5ress
ar4fi
ar4fl
ar1i
ar5ial
ar3ian
a3riet
ar4im
ar5inat
ar3io
ar2iz
ar2mi
ar5o5d
a5roni
a3roo
ar2p
ar3q
arre4
ar4sa
ar2sh
4as.
as4ab
as3ant
ashi4
a5sia.
a3sib
a3sic
5a5si4t
ask3i
as4l
a4soc
as5ph
as4sh
as3ten
as1tr
asur5a
a2ta
at3abl
at5ac
at3alo
at5ap
ate5c
at5ech
at3ego
at3en.
at3era
ater5n
a5terna
at3est
at5ev
4ath
ath5em
a5then
at4ho
ath5om
4ati.
a5tia
at5i5b
at1ic
at3if
ation5ar
at3itu
a4tog
a2tom
at5omiz
a4top
a4tos
a1tr
at5rop
at4sk
at4tag
at5te
at4th
a2tu
at5ua
at5ue
at3ul
at3ura
a2ty
au4b
augh3
au3gu
au4l2
aun5d
au3r
au5sib
aut5en
au1th
a2va
av3ag
a5van
ave4no
av3era
av5ern
av5ery
av1i
avi4er
av3ig
av5oc
a1vor
3away
aw3i
aw4ly
aws4
ax4ic
ax4id
ay5al
aye4
ays4
azi4er
azz5i
5ba.
bad5ger
ba4ge
bal1a
ban5dag
ban4e
ban3i
barbi5
bari4a
bas4si
1bat
ba4z
2b1b
b2be
b3ber
bbi4na
4b1d
4be.
beak4
beat3
4be2d
be3da
be3de
be3di
be3gi
be5gu
1bel
be1li
be3lo
4be5m
be5nig
be5nu
4bes4
be3sp
be5str
3bet
bet5iz
be5tr
be3tw
be3w
be5yo
2bf
4b3h
bi2b
bi4d
3bie
bi5en
bi4er
2b3if
1bil
bi3liz
bina5r4
bin4d
bi5net
bi3ogr
bi5ou
bi2t
3bi3tio
bi3tr
3bit5ua
b5itz
b1j
bk4
b2l2
blath5
b4le.
blen4
5blesp
b3lis
b4lo
blun4t
4b1m
4b3n
bne5g
3bod
bod3i
bo4e
bol3ic
bom4bi
bon4a
bon5at
3boo
5bor.
4b1ora
bor5d
5bore
5bori
5bos4
b5ota
both5
bo4to
bound3
4bp
4brit
broth3
2b5s2
bsor4
2bt
bt4l
b4to
b3tr
buf4fer
bu4ga
bu3li
bumi4
bu4n
bunt4i
bu3re
bus5ie
buss4e
5bust
4buta
3butio
b5uto
b1v
4b5w
5by.
bys4
1ca
cab3in
ca1bl
cach4
ca5den
4cag4
2c5ah
ca3lat
cal4la
call5in
4calo
can5d
can4e
can4ic
can5is
can3iz
can4ty
cany4
ca5per
car5om
cast5er
cas5tig
4casy
ca4th
4cativ
cav5al
c3c
ccha5
cci4a
ccompa5
ccon4
ccou3t
2ce.
4ced.
4ceden
3cei
5cel.
3cell
1cen
3cenc
2cen4e
4ceni
3cent
3cep
ce5ram
4cesa
3cessi
ces5si5b
ces5t
cet4
c5e4ta
cew4
2ch
4ch.
4ch3ab
5chanic
ch5a5nis
che2
cheap3
4ched
che5lo
3chemi
ch5ene
ch3er.
ch3ers
4ch1in
5chine.
ch5iness
5chini
5chio
3chit
chi2z
3cho2
ch4ti
1ci
3cia
ci2a5b
cia5r
ci5c
4cier
5cific.
4cii
ci4la
3cili
2cim
2cin
c4ina
3cinat
cin3em
c1ing
c5ing.
5cino
cion4
4cipe
ci3ph
4cipic
4cista
4cisti
2c1it
cit3iz
5ciz
ck1
ck3i
1c4l4
4clar
c5laratio
5clare
cle4m
4clic
clim4
cly4
c5n
1co
co5ag
coe2
2cog
co4gr
coi4
co3inc
col5i
5colo
col3or
com5er
con4a
c4one
con3g
con5t
co3pa
cop3ic
co4pl
4corb
coro3n
cos4e
cov1
cove4
cow5a
coz5e
co5zi
c1q
cras5t
5crat.
5cratic
cre3at
5cred
4c3reta
cre4v
cri2
cri5f
c4rin
cris4
5criti
cro4pl
crop5o
cros4e
cru4d
4c3s2
2c1t
cta4b
ct5ang
c5tant
c2te
c3ter
c4ticu
ctim3i
ctu4r
c4tw
cud5
c4uf
c4ui
cu5ity
5culi
cul4tis
3cultu
cu2ma
c3ume
cu4mi
3cun
cu3pi
cu5py
cur5a4b
cu5ria
1cus
cuss4i
3c4ut
cu4tie
4c5utiv
4cutr
1cy
cze4
1d2a
5da.
2d3a4b
dach4
4daf
2dag
da2m2
dan3g
dard5
dark5
4dary
3dat
4dativ
4dato
5dav4
dav5e
5day
d1b
d5c
d1d4
2de.
deaf5
deb5it
de4bon
decan4
de4cil
de5com
2d1ed
4dee.
de5if
deli4e
del5i5q
de5lo
d4em
5dem.
3demic
dem5ic.
de5mil
de4mons
demor5
1den
de4nar
de3no
denti5f
de3nu
de1p
de3pa
depi4
de2pu
d3eq
d4erh
5derm
dern5iz
der5s
des2
d2es.
de1sc
de2s5o
des3ti
de3str
de4su
de1t
de2to
de1v
dev3il
4dey
4d1f
d4ga
d3ge4t
dg1i
d2gy
d1h2
5di.
1d4i3a
dia5b
di4cam
d4ice
3dict
3did
5di3en
d1if
di3ge
di4lato
d1in
1dina
3dine.
5dini
di5niz
1dio
dio5g
di4pl
dir2
di1re
dirt5i
dis1
5disi
d4is3t
d2iti
1di1v
d1j
d5k2
4d5la
3dle.
3dled
3dles.
4dless
2d3lo
4d5lu
2dly
d1m
4d1n4
1do
3do.
do5de
5doe
2d5of
d4og
do4la
doli4
do5lor
dom5iz
do3nat
doni4
doo3d
dop4p
d4or
3dos
4d5out
do4v
3dox
d1p
1dr
drag5on
4drai
dre4
drea5r
5dren
dri4b
dril4
dro4p
4drow
5drupli
4dry
2d1s2
ds4p
d4sw
d4sy
d2th
1du
d1u1a
du2c
d1uca
duc5er
4duct.
4ducts
du5el
du4g
d3ule
dum4be
du4n
4dup
du4pe
d1v
d1w
d2y
5dyn
dy4se
dys5p
e1a4b
e3act
ead1
ead5ie
ea4ge
ea5ger
ea4l
eal5er
eal3ou
eam3er
e5and
ear3a
ear4c
ear5es
ear4ic
ear4il
ear5k
ear2t
eart3e
ea5sp
e3ass
east3
ea2t
eat5en
eath3i
e5atif
e4a3tu
ea2v
eav3en
eav5i
eav5o
2e1b
e4bel.
e4bels
e4ben
e4bit
e3br
e4cad
ecan5c
ecca5
e1ce
ec5essa
ec2i
e4cib
ec5ificat
ec5ifie
ec5ify
ec3im
eci4t
e5cite
e4clam
e4clus
e2col
e4comm
e4compe
e4conc
e2cor
ec3ora
eco5ro
e1cr
e4crem
ec4tan
ec4te
e1cu
e4cul
ec3ula
2e2da
4ed3d
e4d1er
ede4s
4edi
e3dia
ed3ib
ed3ica
ed3im
ed1it
edi5z
4edo
e4dol
edon2
e4dri
e4dul
ed5ulo
ee2c
eed3i
ee2f
eel3i
ee4ly
ee2m
ee4na
ee4p1
ee2s4
eest4
ee4ty
e5ex
e1f
e4f3ere
1eff
e4fic
5efici
efil4
e3fine
ef5i5nite
3efit
efor5es
e4fuse.
4egal
eger4
eg5ib
eg4ic
eg5ing
e5git5
eg5n
e4go.
e4gos
eg1ul
e5gur
5egy
e1h4
eher4
ei2
e5ic
ei5d
eig2
ei5gl
e3imb
e3inf
e1ing
e5inst
eir4d
eit3e
ei3th
e5ity
e1j
e4jud
ej5udi
eki4n
ek4la
e1la
e4la.
e4lac
elan4d
el5ativ
e4law
elaxa4
e3lea
el5ebra
5elec
e4led
el3ega
e5len
e4l1er
e1les
el2f
el2i
e3libe
e4l5ic.
el3ica
e3lier
el5igib
e5lim
e4l3ing
e3lio
e2lis
el5ish
e3liv3
4ella
el4lab
ello4
e5loc
el5og
el3op.
el2sh
el4ta
e5lud
el5ug
e4mac
e4mag
e5man
em5ana
em5b
e1me
e2mel
e4met
em3ica
emi4e
em5igra
em1in2
em5ine
em3i3ni
e4mis
em5ish
e5miss
em3iz
5emniz
emo4g
emoni5o
em3pi
e4mul
em5ula
emu3n
e3my
en5amo
e4nant
ench4er
en3dic
e5nea
e5nee
en3em
en5ero
en5esi
en5est
en3etr
e3new
en5ics
e5nie
e5nil
e3nio
en3ish
en3it
e5niu
5eniz
4enn
4eno
eno4g
e4nos
en3ov
en4sw
ent5age
4enthes
en3ua
en5uf
e3ny.
4en3z
e5of
eo2g
e4oi4
e3ol
eop3ar
e1or
eo3re
eo5rol
eos4
e4ot
eo4to
e5out
e5ow
e2pa
e3pai
ep5anc
e5pel
e3pent
ep5etitio
ephe4
e4pli
e1po
e4prec
ep5reca
e4pred
ep3reh
e3pro
e4prob
ep4sh
ep5ti5b
e4put
ep5uta
e1q
equi3l
e4q3ui3s
er1a
era4b
4erand
er3ar
4erati.
2erb
er4bl
er3ch
er4che
2ere.
e3real
ere5co
ere3in
er5el.
er3emo
er5ena
er5ence
4erene
er3ent
ere4q
er5ess
er3est
eret4
er1h
er1i
e1ria4
5erick
e3rien
eri4er
er3ine
e1rio
4erit
er4iu
eri4v
e4riva
er3m4
er4nis
4ernit
5erniz
er3no
2ero
er5ob
e5roc
ero4r
er1ou
er1s
er3set
ert3er
4ertl
er3tw
4eru
eru4t
5erwau
e1s4a
e4sage.
e4sages
es2c
e2sca
es5can
e3scr
es5cu
e1s2e
e2sec
es5ecr
es5enc
e4sert.
e4serts
e4serva
4esh
e3sha
esh5en
e1si
e2sic
e2sid
es5iden
es5igna
e2s5im
es4i4n
esis4te
esi4u
e5skin
es4mi
e2sol
es3olu
e2son
es5ona
e1sp
es3per
es5pira
es4pre
2ess
es4si4b
estan4
es3tig
es5tim
4es2to
e3ston
2estr
e5stro
estruc5
e2sur
es5urr
es4w
eta4b
eten4d
e3teo
ethod3
et1ic
e5tide
etin4
eti4no
e5tir
e5titio
et5itiv
4etn
et5ona
e3tra
e3tre
et3ric
et5rif
et3rog
et5ros
et3ua
et5ym
et5z
4eu
e5un
e3up
eu3ro
eus4
eute4
euti5l
eu5tr
eva2p5
e2vas
ev5ast
e5vea
ev3ell
evel3o
e5veng
even4i
ev1er
e5verb
e1vi
ev3id
evi4l
e4vin
evi4v
e5voc
e5vu
e1wa
e4wag
e5wee
e3wh
ewil5
ew3ing
e3wit
1exp
5eyc
5eye.
eys4
1fa
fa3bl
fab3r
fa4ce
4fag
fain4
fall5e
4fa4ma
fam5is
5far
far5th
fa3ta
fa3the
4fato
fault5
4f5b
4fd
4fe.
feas4
feath3
fe4b
4feca
5fect
2fed
fe3li
fe4mo
fen2d
fend5e
fer1
5ferr
fev4
4f1f
f4fes
f4fie
f5fin.
f2f5is
f4fly
f2fy
4fh
1fi
fi3a
2f3ic.
4f3ical
f3ican
4ficate
f3icen
fi3cer
fic4i
5ficia
5ficie
4fics
fi3cu
fi5del
fight5
fil5i
fill5in
4fily
2fin
5fina
fin2d5
fi2ne
f1in3g
fin4n
fis4ti
f4l2
f5less
flin4
flo3re
f2ly5
4fm
4fn
1fo
5fon
fon4de
fon4t
fo2r
fo5rat
for5ay
fore5t
for4i
fort5a
fos5
4f5p
fra4t
f5rea
fres5c
fri2
fril4
frol5
2f3s
2ft
f4to
f2ty
3fu
fu5el
4fug
fu4min
fu5ne
fu3ri
fusi4
fus4s
4futa
1fy
1ga
gaf4
5gal.
3gali
ga3lo
2gam
ga5met
g5amo
gan5is
ga3niz
gani5za
4gano
gar5n4
gass4
gath3
4gativ
4gaz
g3b
gd4
2ge.
2ged
geez4
gel4in
ge5lis
ge5liz
4gely
1gen
ge4nat
ge5niz
4geno
4geny
1geo
ge3om
g4ery
5gesi
geth5
4geto
ge4ty
ge4v
4g1g2
g2ge
g3ger
gglu5
ggo4
gh3in
gh5out
gh4to
5gi.
1gi4a
gia5r
g1ic
5gicia
g4ico
gien5
5gies.
gil4
g3imen
3g4in.
gin5ge
5g4ins
5gio
3gir
gir4l
g3isl
gi4u
5giv
3giz
gl2
gla4
glad5i
5glas
1gle
gli4b
g3lig
3glo
glo3r
g1m
g4my
gn4a
g4na.
gnet4t
g1ni
g2nin
g4nio
g1no
g4non
1go
3go.
gob5
5goe
3g4o4g
go3is
gon2
4g3o3na
gondo5
go3ni
5goo
go5riz
gor5ou
5gos.
gov1
g3p
1gr
4grada
g4rai
gran2
5graph.
g5rapher
5graphic
4graphy
4gray
gre4n
4gress.
4grit
g4ro
gruf4
gs2
g5ste
gth3
gu4a
3guard
2gue
5gui5t
3gun
3gus
4gu4t
g3w
1gy
2g5y3n
gy5ra
h3ab4l
hach4
hae4m
hae4t
h5agu
ha3la
hala3m
ha4m
han4ci
han4cy
5hand.
han4g
hang5er
hang5o
h5a5niz
han4k
han4te
hap3l
hap5t
ha3ran
ha5ras
har2d
hard3e
har4le
harp5en
har5ter
has5s
haun4
5haz
haz3a
h1b
1head
3hear
he4can
h5ecat
h4ed
he5do5
he3l4i
hel4lis
hel4ly
h5elo
hem4p
he2n
hena4
hen5at
heo5r
hep5
h4era
hera3p
her4ba
here5a
h3ern
h5erou
h3ery
h1es
he2s5p
he4t
het4ed
heu4
h1f
h1h
hi5an
hi4co
high5
h4il2
himer4
h4ina
hion4e
hi4p
hir4l
hi3ro
hir4p
hir4r
his3el
his4s
hith5er
hi2v
4hk
4h1l4
hlan4
h2lo
hlo3ri
4h1m
hmet4
2h1n
h5odiz
h5ods
ho4g
hoge4
hol5ar
3hol4e
ho4ma
home3
hon4a
ho5ny
3hood
hoon4
hor5at
ho5ris
hort3e
ho5ru
hos4e
ho5sen
hos1p
1hous
house3
hov5el
4h5p
4hr4
hree5
hro5niz
hro3po
4h1s2
h4sh
h4tar
ht1en
ht5es
h4ty
hu4g
hu4min
hun5ke
hun4t
hus3t4
hu4t
h1w
h4wart
hy3pe
hy3ph
hy2s
2i1a
i2al
iam4
iam5ete
i2an
4ianc
ian3i
4ian4t
ia5pe
iass4
i4ativ
ia4tric
i4atu
ibe4
ib3era
ib5ert
ib5ia
ib3in
ib5it.
ib5ite
i1bl
ib3li
i5bo
i1br
i2b5ri
i5bun
4icam
5icap
4icar
i4car.
i4cara
icas5
i4cay
iccu4
4iceo
4ich
2ici
i5cid
ic5ina
i2cip
ic3ipa
i4cly
i2c5oc
4i1cr
5icra
i4cry
ic4te
ictu2
ic4t3ua
ic3ula
ic4um
ic5uo
i3cur
2id
i4dai
id5anc
id5d
ide3al
ide4s
i2di
id5ian
idi4ar
i5die
id3io
idi5ou
id1it
id5iu
i3dle
i4dom
id3ow
i4dr
i2du
id5uo
2ie4
ied4e
5ie5ga
ield3
ien5a4
ien4e
i5enn
i3enti
i1er.
i3esc
i1est
i3et
4if.
if5ero
iff5en
if4fr
4ific.
i3fie
i3fl
4ift
2ig
iga5b
ig3era
ight3i
4igi
i3gib
ig3il
ig3in
ig3it
i4g4l
i2go
ig3or
ig5ot
i5gre
igu5i
ig1ur
i3h
4i5i4
i3j
4ik
i1la
il3a4b
i4lade
i2l5am
ila5ra
i3leg
il1er
ilev4
il5f
il1i
il3ia
il2ib
il3io
il4ist
2ilit
il2iz
ill5ab
4iln
il3oq
il4ty
il5ur
il3v
i4mag
im3age
ima5ry
imenta5r
4imet
im1i
im5ida
imi5le
i5mini
4imit
im4ni
i3mon
i2mu
im3ula
2in.
i4n3au
4inav
incel4
in3cer
4ind
in5dling
2ine
i3nee
iner4ar
i5ness
4inga
4inge
in5gen
4ingi
in5gling
4ingo
4ingu
2ini
i5ni.
i4nia
in3io
in1is
i5nite.
5initio
in3ity
4ink
4inl
2inn
2i1no
i4no4c
ino4s
i4not
2ins
in3se
insur5a
2int.
2in4th
in1u
i5nus
4iny
2io
4io.
ioge4
io2gr
i1ol
io4m
ion3at
ion4ery
ion3i
io5ph
ior3i
i4os
io5th
i5oti
io4to
i4our
2ip
ipe4
iphras4
ip3i
ip4ic
ip4re4
ip3ul
i3qua
iq5uef
iq3uid
iq3ui3t
4ir
i1ra
ira4b
i4rac
ird5e
ire4de
i4ref
i4rel4
i4res
ir5gi
ir1i
iri5de
ir4is
iri3tu
5i5r2iz
ir4min
iro4g
5iron.
ir5ul
2is.
is5ag
is3ar
isas5
2is1c
is3ch
4ise
is3er
3isf
is5han
is3hon
ish5op
is3ib
isi4d
i5sis
is5itiv
4is4k
islan4
4isms
i2so
iso5mer
is1p
is2pi
is4py
4is1s
is4sal
issen4
is4ses
is4ta.
is1te
is1ti
ist4ly
4istral
i2su
is5us
4ita.
ita4bi
i4tag
4ita5m
i3tan
i3tat
2ite
it3era
i5teri
it4es
2ith
i1ti
4itia
4i2tic
it3ica
5i5tick
it3ig
it5ill
i2tim
2itio
4itis
i4tism
i2t5o5m
4iton
i4tram
it5ry
4itt
it3uat
i5tud
it3ul
4itz.
i1u
2iv
iv3ell
iv3en.
i4v3er.
i4vers.
iv5il.
iv5io
iv1it
i5vore
iv3o3ro
i4v3ot
4i5w
ix4o
4iy
4izar
izi4
5izont
5ja
jac4q
ja4p
1je
jer5s
4jestie
4jesty
jew3
jo4p
5judg
3ka.
k3ab
k5ag
kais4
kal4
k1b
k2ed
1kee
ke4g
ke5li
k3en4d
k1er
kes4
k3est.
ke4ty
k3f
kh4
k1i
5ki.
5k2ic
k4ill
kilo5
k4im
k4in.
kin4de
k5iness
kin4g
ki4p
kis4
k5ish
kk4
k1l
4kley
4kly
k1m
k5nes
1k2no
ko5r
kosh4
k3ou
kro5n
4k1s2
k4sc
ks4l
k4sy
k5t
k1w
lab3ic
l4abo
laci4
l4ade
la3dy
lag4n
lam3o
3land
lan4dl
lan5et
lan4te
lar4g
lar3i
las4e
la5tan
4lateli
4lativ
4lav
la4v4a
2l1b
lbin4
4l1c2
lce4
l3ci
2ld
l2de
ld4ere
ld4eri
ldi4
ld5is
l3dr
l4dri
le2a
le4bi
left5
5leg.
5legg
le4mat
lem5atic
4len.
3lenc
5lene.
1lent
le3ph
le4pr
lera5b
ler4e
3lerg
3l4eri
l4ero
les2
le5sco
5lesq
3less
5less.
l3eva
lev4er.
lev4era
lev4ers
3ley
4leye
2lf
l5fr
4l1g4
l5ga
lgar3
l4ges
lgo3
2l3h
li4ag
li2am
liar5iz
li4as
li4ato
li5bi
5licio
li4cor
4lics
4lict.
l4icu
l3icy
l3ida
lid5er
3lidi
lif3er
l4iff
li4fl
5ligate
3ligh
li4gra
3lik
4l4i4l
lim4bl
lim3i
li4mo
l4im4p
l4ina
1l4ine
lin3ea
lin3i
link5er
li5og
4l4iq
lis4p
l1it
l2it.
5litica
l5i5tics
liv3er
l1iz
4lj
lka3
l3kal
lka4t
l1l
l4law
l2le
l5lea
l3lec
l3leg
l3lel
l3le4n
l3le4t
ll2i
l2lin4
l5lina
ll4o
lloqui5
ll5out
l5low
2lm
l5met
lm3ing
l4mod
lmon4
2l1n2
3lo.
lob5al
lo4ci
4lof
3logic
l5ogo
3logu
lom3er
5long
lon4i
l3o3niz
lood5
5lope.
lop3i
l3opm
lora4
lo4rato
lo5rie
lor5ou
5los.
los5et
5losophiz
5losophy
los4t
lo4ta
loun5d
2lout
4lov
2lp
lpa5b
l3pha
l5phi
lp5ing
l3pit
l4pl
l5pr
4l1r
2l1s2
l4sc
l2se
l4sie
4lt
lt5ag
ltane5
l1te
lten4
ltera4
lth3i
l5ties.
ltis4
l1tr
ltu2
ltur3a
lu5a
lu3br
luch4
lu3ci
lu3en
luf4
lu5id
lu4ma
5lumi
l5umn.
5lumnia
lu3o
luo3r
4lup
luss4
lus3te
1lut
l5ven
l5vet4
2l1w
1ly
4lya
4lyb
ly5me
ly3no
2lys4
l5yse
1ma
2mab
ma2ca
ma5chine
ma4cl
mag5in
5magn
2mah
maid5
4mald
ma3lig
ma5lin
mal4li
mal4ty
5mania
man5is
man3iz
4map
ma5rine.
ma5riz
mar4ly
mar3v
ma5sce
mas4e
mas1t
5mate
math3
ma3tis
4matiza
4m1b
mba4t5
m5bil
m4b3ing
mbi4v
4m5c
4me.
2med
4med.
5media
me3die
m5e5dy
me2g
mel5on
mel4t
me2m
mem1o3
1men
men4a
men5ac
men4de
4mene
men4i
mens4
mensu5
3ment
men4te
me5on
m5ersa
2mes
3mesti
me4ta
met3al
me1te
me5thi
m4etr
5metric
me5trie
me3try
me4v
4m1f
2mh
5mi.
mi3a
mid4a
mid4g
mig4
3milia
m5i5lie
m4ill
min4a
3mind
m5inee
m4ingl
min5gli
m5ingly
min4t
m4inu
miot4
m2is
mis4er.
mis5l
mis4ti
m5istry
4mith
m2iz
4mk
4m1l
m1m
mma5ry
4m1n
mn4a
m4nin
mn4o
1mo
4mocr
5mocratiz
mo2d1
mo4go
mois2
moi5se
4mok
mo5lest
mo3me
mon5et
mon5ge
moni3a
mon4ism
mon4ist
mo3niz
monol4
mo3ny.
mo2r
4mora.
mos2
mo5sey
mo3sp
moth3
m5ouf
3mous
mo2v
4m1p
mpara5
mpa5rab
mpar5i
m3pet
mphas4
m2pi
mpi4a
mp5ies
m4p1in
m5pir
mp5is
mpo3ri
mpos5ite
m4pous
mpov5
mp4tr
m2py
4m3r
4m1s2
m4sh
m5si
4mt
1mu
mula5r4
5mult
multi3
3mum
mun2
4mup
mu4u
4mw
1na
2n1a2b
n4abu
4nac.
na4ca
n5act
nag5er.
nak4
na4li
na5lia
4nalt
na5mit
n2an
nanci4
nan4it
nank4
nar3c
4nare
nar3i
nar4l
n5arm
n4as
nas4c
nas5ti
n2at
na3tal
nato5miz
n2au
nau3se
3naut
nav4e
4n1b4
ncar5
n4ces.
n3cha
n5cheo
n5chil
n3chis
nc1in
nc4it
ncour5a
n1cr
n1cu
n4dai
n5dan
n1de
nd5est.
ndi4b
n5d2if
n1dit
n3diz
n5duc
ndu4r
nd2we
2ne.
n3ear
ne2b
neb3u
ne2c
5neck
2ned
ne4gat
neg5ativ
5nege
ne4la
nel5iz
ne5mi
ne4mo
1nen
4nene
3neo
ne4po
ne2q
n1er
nera5b
n4erar
n2ere
n4er5i
ner4r
1nes
2nes.
4nesp
2nest
4nesw
3netic
ne4v
n5eve
ne4w
n3f
n4gab
n3gel
nge4n4e
n5gere
n3geri
ng5ha
n3gib
ng1in
n5git
n4gla
ngov4
ng5sh
n1gu
n4gum
n2gy
4n1h4
nha4
nhab3
nhe4
3n4ia
ni3an
ni4ap
ni3ba
ni4bl
ni4d
ni5di
ni4er
ni2fi
ni5ficat
n5igr
nik4
n1im
ni3miz
n1in
5nine.
nin4g
ni4o
5nis.
nis4ta
n2it
n4ith
3nitio
n3itor
ni3tr
n1j
4nk2
n5kero
n3ket
nk3in
n1kl
4n1l
n5m
nme4
nmet4
4n1n2
nne4
nni3al
nni4v
nob4l
no3ble
n5ocl
4n3o2d
3noe
4nog
noge4
nois5i
no5l4i
5nologis
3nomic
n5o5miz
no4mo
no3my
no4n
non4ag
non5i
n5oniz
4nop
5nop5o5li
nor5ab
no4rary
4nosc
nos4e
nos5t
no5ta
1nou
3noun
nov3el3
nowl3
n1p4
npi4
npre4c
n1q
n1r
nru4
2n1s2
ns5ab
nsati4
ns4c
n2se
n4s3es
nsid1
nsig4
n2sl
ns3m
n4soc
ns4pe
n5spi
nsta5bl
n1t
nta4b
nter3s
nt2i
n5tib
nti4er
nti2f
n3tine
n4t3ing
nti4p
ntrol5li
nt4s
ntu3me
nu1a
nu4d
nu5en
nuf4fe
n3uin
3nu3it
n4um
nu1me
n5umi
3nu4n
n3uo
nu3tr
n1v2
n1w4
nym4
nyp4
4nz
n3za
4oa
oad3
o5a5les
oard3
oas4e
oast5e
oat5i
ob3a3b
o5bar
obe4l
o1bi
o2bin
ob5ing
o3br
ob3ul
o1ce
och4
o3chet
ocif3
o4cil
o4clam
o4cod
oc3rac
oc5ratiz
ocre3
5ocrit
octor5a
oc3ula
o5cure
od5ded
od3ic
odi3o
o2do4
odor3
od5uct.
od5ucts
o4el
o5eng
o3er
oe4ta
o3ev
o2fi
of5ite
ofit4t
o2g5a5r
og5ativ
o4gato
o1ge
o5gene
o5geo
o4ger
o3gie
1o1gis
og3it
o4gl
o5g2ly
3ogniz
o4gro
ogu5i
1ogy
2ogyn
o1h2
ohab5
oi2
oic3es
oi3der
oiff4
oig4
oi5let
o3ing
oint5er
o5ism
oi5son
oist5en
oi3ter
o5j
2ok
o3ken
ok5ie
o1la
o4lan
olass4
ol2d
old1e
ol3er
o3lesc
o3let
ol4fi
ol2i
o3lia
o3lice
ol5id.
o3li4f
o5lil
ol3ing
o5lio
o5lis.
ol3ish
o5lite
o5litio
o5liv
olli4e
ol5ogiz
olo4r
ol5pl
ol2t
ol3ub
ol3ume
ol3un
o5lus
ol2v
o2ly
om5ah
oma5l
om5atiz
om2be
om4bl
o2me
om3ena
om5erse
o4met
om5etry
o3mia
om3ic.
om3ica
o5mid
om1in
o5mini
5ommend
omo4ge
o4mon
om3pi
ompro5
o2n
on1a
on4ac
o3nan
on1c
3oncil
2ond
on5do
o3nen
on5est
on4gu
on1ic
o3nio
on1is
o5niu
on3key
on4odi
on3omy
on3s
onspi4
onspir5a
onsu4
onten4
on3t4i
ontif5
on5um
onva5
oo2
ood5e
ood5i
oo4k
oop3i
o3ord
oost5
o2pa
ope5d
op1er
3opera
4operag
2oph
o5phan
o5pher
op3ing
o3pit
o5pon
o4posi
o1pr
op1u
opy5
o1q
o1ra
o5ra.
o4r3ag
or5aliz
or5ange
ore5a
o5real
or3ei
ore5sh
or5est.
orew4
or4gu
4o5ria
or3ica
o5ril
or1in
o1rio
or3ity
o3riu
or2mi
orn2e
o5rof
or3oug
or5pe
3orrh
or4se
ors5en
orst4
or3thi
or3thy
or4ty
o5rum
o1ry
os3al
os2c
os4ce
o3scop
4oscopi
o5scr
os4i4e
os5itiv
os3ito
os3ity
osi4u
os4l
o2so
os4pa
os4po
os2ta
o5stati
os5til
os5tit
o4tan
otele4g
ot3er.
ot5ers
o4tes
4oth
oth5esi
oth3i4
ot3ic.
ot5ica
o3tice
o3tif
o3tis
oto5s
ou2
ou3bl
ouch5i
ou5et
ou4l
ounc5er
oun2d
ou5v
ov4en
over4ne
over3s
ov4ert
o3vis
oviti4
o5v4ol
ow3der
ow3el
ow5est
ow1i
own5i
o4wo
oy1a
1pa
pa4ca
pa4ce
pac4t
p4ad
5pagan
p3agat
p4ai
pain4
p4al
pan4a
pan3el
pan4ty
pa3ny
pa1p
pa4pu
para5bl
par5age
par5di
3pare
par5el
p4a4ri
par4is
pa2te
pa5ter
5pathic
pa5thy
pa4tric
pav4
3pay
4p1b
pd4
4pe.
3pe4a
pear4l
pe2c
2p2ed
3pede
3pedi
pedia4
ped4ic
p4ee
pee4d
pek4
pe4la
peli4e
pe4nan
p4enc
pen4th
pe5on
p4era.
pera5bl
p4erag
p4eri
peri5st
per4mal
perme5
p4ern
per3o
per3ti
pe5ru
per1v
pe2t
pe5ten
pe5tiz
4pf
4pg
4ph.
phar5i
phe3no
ph4er
ph4es.
ph1ic
5phie
ph5ing
5phisti
3phiz
ph2l
3phob
3phone
5phoni
pho4r
4phs
ph3t
5phu
1phy
pi3a
pian4
pi4cie
pi4cy
p4id
p5ida
pi3de
5pidi
3piec
pi3en
pi4grap
pi3lo
pi2n
p4in.
pind4
p4ino
3pi1o
pion4
p3ith
pi5tha
pi2tu
2p3k2
1p2l2
3plan
plas5t
pli3a
pli5er
4plig
pli4n
ploi4
plu4m
plum4b
4p1m
2p3n
po4c
5pod.
po5em
po3et5
5po4g
poin2
5point
poly5t
po4ni
po4p
1p4or
po4ry
1pos
pos1s
p4ot
po4ta
5poun
4p1p
ppa5ra
p2pe
p4ped
p5pel
p3pen
p3per
p3pet
ppo5site
pr2
pray4e
5preci
pre5co
pre3em
pref5ac
pre4la
pre3r
p3rese
3press
pre5ten
pre3v
5pri4e
prin4t3
pri4s
pris3o
p3roca
prof5it
pro3l
pros3e
pro1t
2p1s2
p2se
ps4h
p4sib
2p1t
pt5a4b
p2te
p2th
pti3m
ptu4r
p4tw
pub3
pue4
puf4
pul3c
pu4m
pu2n
pur4r
5pus
pu2t
5pute
put3er
pu3tr
put4ted
put4tin
p3w
qu2
qua5v
2que.
3quer
3quet
2rab
ra3bi
rach4e
r5acl
raf5fi
raf4t
r2ai
ra4lo
ram3et
r2ami
rane5o
ran4ge
r4ani
ra5no
rap3er
3raphy
rar5c
rare4
rar5ef
4raril
r2as
ration4
rau4t
ra5vai
rav3el
ra5zie
r1b
r4bab
r4bag
rbi2
rbi4f
r2bin
r5bine
rb5ing.
rb4o
r1c
r2ce
rcen4
r3cha
rch4er
r4ci4b
rc4it
rcum3
r4dal
rd2i
rdi4a
rdi4er
rdin4
rd3ing
2re.
re1al
re3an
re5arr
5reav
re4aw
r5ebrat
rec5oll
rec5ompe
re4cre
2r2ed
re1de
re3dis
red5it
re4fac
re2fe
re5fer.
re3fi
re4fy
reg3is
re5it
re1li
re5lu
r4en4ta
ren4te
re1o
re5pin
re4posi
re1pu
r1er4
r4eri
rero4
re5ru
r4es.
re4spi
ress5ib
res2t
re5stal
re3str
re4ter
re4ti4z
re3tri
reu2
re5uti
rev2
re4val
rev3el
r5ev5er.
re5vers
re5vert
re5vil
rev5olu
re4wh
r1f
rfu4
r4fy
rg2
rg3er
r3get
r3gic
rgi4n
rg3ing
r5gis
r5git
r1gl
rgo4n
r3gu
rh4
4rh.
4rhal
ri3a
ria4b
ri4ag
r4ib
rib3a
ric5as
r4ice
4rici
5ricid
ri4cie
r4ico
rid5er
ri3enc
ri3ent
ri1er
ri5et
rig5an
5rigi
ril3iz
5riman
rim5i
3rimo
rim4pe
r2ina
5rina.
rin4d
rin4e
rin4g
ri1o
5riph
riph5e
ri2pl
rip5lic
r4iq
r2is
r4is.
ris4c
r3ish
ris4p
ri3ta3b
r5ited.
rit5er.
rit5ers
rit3ic
ri2tu
rit5ur
riv5el
riv3et
riv3i
r3j
r3ket
rk4le
rk4lin
r1l
rle4
r2led
r4lig
r4lis
rl5ish
r3lo4
r1m
rma5c
r2me
r3men
rm5ers
rm3ing
r4ming.
r4mio
r3mit
r4my
r4nar
r3nel
r4ner
r5net
r3ney
r5nic
r1nis4
r3nit
r3niv
rno4
r4nou
r3nu
rob3l
r2oc
ro3cr
ro4e
ro1fe
ro5fil
rok2
ro5ker
5role.
rom5ete
rom4i
rom4p
ron4al
ron4e
ro5n4is
ron4ta
1room
5root
ro3pel
rop3ic
ror3i
ro5ro
ros5per
ros4s
ro4the
ro4ty
ro4va
rov5el
rox5
r1p
r4pea
r5pent
rp5er.
r3pet
rp4h4
rp3ing
r3po
r1r4
rre4c
rre4f
r4reo
rre4st
rri4o
rri4v
rron4
rros4
rrys4
4rs2
r1sa
rsa5ti
rs4c
r2se
r3sec
rse4cr
rs5er.
rs3es
rse5v2
r1sh
r5sha
r1si
r4si4b
rson3
r1sp
r5sw
rtach4
r4tag
r3teb
rten4d
rte5o
r1ti
rt5ib
rti4d
r4tier
r3tig
rtil3i
rtil4l
r4tily
r4tist
r4tiv
r3tri
rtroph4
rt4sh
ru3a
ru3e4l
ru3en
ru4gl
ru3in
rum3pl
ru2n
runk5
run4ty
r5usc
ruti5n
rv4e
rvel4i
r3ven
rv5er.
r5vest
r3vey
r3vic
rvi4v
r3vo
r1w
ry4c
5rynge
ry3t
sa2
2s1ab
5sack
sac3ri
s3act
5sai
salar4
sal4m
sa5lo
sal4t
3sanc
san4de
s1ap
sa5ta
5sa3tio
sat3u
sau4
sa5vor
5saw
4s5b
scan4t5
sca4p
scav5
s4ced
4scei
s4ces
sch2
s4cho
3s4cie
5scin4d
scle5
s4cli
scof4
4scopy
scour5a
s1cu
4s5d
4se.
se4a
seas4
sea5w
se2c3o
3sect
4s4ed
se4d4e
s5edl
se2g
seg3r
5sei
se1le
5self
5selv
4seme
se4mol
sen5at
4senc
sen4d
s5ened
sen5g
s5enin
4sentd
4sentl
sep3a3
4s1er.
s4erl
ser4o
4servo
s1e4s
se5sh
ses5t
5se5um
5sev
sev3en
sew4i
5sex
4s3f
2s3g
s2h
2sh.
sh1er
5shev
sh1in
sh3io
3ship
shiv5
sho4
sh5old
shon3
shor4
short5
4shw
si1b
s5icc
3side.
5sides
5sidi
si5diz
4signa
sil4e
4sily
2s1in
s2ina
5sine.
s3ing
1sio
5sion
sion5a
si2r
sir5a
1sis
3sitio
5siu
1siv
5siz
sk2
4ske
s3ket
sk5ine
sk5ing
s1l2
s3lat
s2le
slith5
2s1m
s3ma
small3
sman3
smel4
s5men
5smith
smol5d4
s1n4
1so
so4ce
soft3
so4lab
sol3d2
so3lic
5solv
3som
3s4on.
sona4
son4g
s4op
5sophic
s5ophiz
s5ophy
sor5c
sor5d
4sov
so5vi
2spa
5spai
spa4n
spen4d
2s5peo
2sper
s2phe
3spher
spho5
spil4
sp5ing
4spio
s4ply
s4pon
spor4
4spot
squal4l
s1r
2ss
s1sa
ssas3
s2s5c
s3sel
s5seng
s4ses.
s5set
s1si
s4sie
ssi4er
ss5ily
s4sl
ss4li
s4sn
sspend4
ss2t
ssur5a
ss5w
2st.
s2tag
s2tal
stam4i
5stand
s4ta4p
5stat.
s4ted
stern5i
s5tero
ste2w
stew5a
s3the
st2i
s4ti.
s5tia
s1tic
5stick
s4tie
s3tif
st3ing
5stir
s1tle
5stock
stom3a
5stone
s4top
3store
st4r
s4trad
5stratu
s4tray
s4trid
4stry
4st3w
s2ty
1su
su1al
su4b3
su2g3
su5is
suit3
s4ul
su2m
sum3i
su2n
su2r
4sv
sw2
4swo
s4y
4syc
3syl
syn5o
sy5rin
1ta
3ta.
2tab
ta5bles
5taboliz
4taci
ta5do
4taf4
tai5lo
ta2l
ta5la
tal5en
tal3i
4talk
tal4lis
ta5log
ta5mo
tan4de
tanta3
ta5per
ta5pl
tar4a
4tarc
4tare
ta3riz
tas4e
ta5sy
4tatic
ta4tur
taun4
tav4
2taw
tax4is
2t1b
4tc
t4ch
tch5et
4t1d
4te.
tead4i
4teat
tece4
5tect
2t1ed
te5di
1tee
teg4
te5ger
te5gi
3tel.
teli4
5tels
te2ma2
tem3at
3tenan
3tenc
3tend
4tenes
1tent
ten4tag
1teo
te4p
te5pe
ter3c
5ter3d
1teri
ter5ies
ter3is
teri5za
5ternit
ter5v
4tes.
4tess
t3ess.
teth5e
3teu
3tex
4tey
2t1f
4t1g
2th.
than4
th2e
4thea
th3eas
the5at
the3is
3thet
th5ic.
th5ica
4thil
5think
4thl
th5ode
5thodic
4thoo
thor5it
tho5riz
2ths
1tia
ti4ab
ti4ato
2ti2b
4tick
t4ico
t4ic1u
5tidi
3tien
tif2
ti5fy
2tig
5tigu
till5in
1tim
4timp
tim5ul
2t1in
t2ina
3tine.
3tini
1tio
ti5oc
tion5ee
5tiq
ti3sa
3tise
tis4m
ti5so
tis4p
5tistica
ti3tl
ti4u
1tiv
tiv4a
1tiz
ti3za
ti3zen
2tl
t5la
tlan4
3tle.
3tled
3tles.
t5let.
t5lo
4t1m
tme4
2t1n2
1to
to3b
to5crat
4todo
2tof
to2gr
to5ic
to2ma
tom4b
to3my
ton4ali
to3nat
4tono
4tony
to2ra
to3rie
tor5iz
tos2
5tour
4tout
to3war
4t1p
1tra
tra3b
tra5ch
traci4
trac4it
trac4te
tras4
tra5ven
trav5es5
tre5f
tre4m
trem5i
5tria
tri5ces
5tricia
4trics
2trim
tri4v
tro5mi
tron5i
4trony
tro5phe
tro3sp
tro3v
tru5i
trus4
4t1s2
t4sc
tsh4
t4sw
4t3t2
t4tes
t5to
ttu4
1tu
tu1a
tu3ar
tu4bi
tud2
4tue
4tuf4
5tu3i
3tum
tu4nis
2t3up.
3ture
5turi
tur3is
tur5o
tu5ry
3tus
4tv
tw4
4t1wa
twis4
4two
1ty
4tya
2tyl
type3
ty5ph
4tz
tz4e
4uab
uac4
ua5na
uan4i
uar5ant
uar2d
uar3i
uar3t
u1at
uav4
ub4e
u4bel
u3ber
u4bero
u1b4i
u4b5ing
u3ble.
u3ca
uci4b
uc4it
ucle3
u3cr
u3cu
u4cy
ud5d
ud3er
ud5est
udev4
u1dic
ud3ied
ud3ies
ud5is
u5dit
u4don
ud4si
u4du
u4ene
uens4
uen4te
uer4il
3ufa
u3fl
ugh3en
ug5in
2ui2
uil5iz
ui4n
u1ing
uir4m
uita4
uiv3
uiv4er.
u5j
4uk
u1la
ula5b
u5lati
ulch4
5ulche
ul3der
ul4e
u1len
ul4gi
ul2i
u5lia
ul3ing
ul5ish
ul4lar
ul4li4b
ul4lis
4ul3m
u1l4o
4uls
uls5es
ul1ti
ultra3
4ultu
u3lu
ul5ul
ul5v
um5ab
um4bi
um4bly
u1mi
u4m3ing
umor5o
um2p
unat4
u2ne
un4er
u1ni
un4im
u2nin
un5ish
uni3v
un3s4
un4sw
unt3ab
un4ter.
un4tes
unu4
un5y
un5z
u4ors
u5os
u1ou
u1pe
uper5s
u5pia
up3ing
u3pl
up3p
upport5
upt5ib
uptu4
u1ra
4ura.
u4rag
u4ras
ur4be
urc4
ur1d
ure5at
ur4fer
ur4fr
u3rif
uri4fic
ur1in
u3rio
u1rit
ur3iz
ur2l
url5ing.
ur4no
uros4
ur4pe
ur4pi
urs5er
ur5tes
ur3the
urti4
ur4tie
u3ru
2us
u5sad
u5san
us4ap
usc2
us3ci
use5a
u5sia
u3sic
us4lin
us1p
us5sl
us5tere
us1tr
u2su
usur4
uta4b
u3tat
4ute.
4utel
4uten
uten4i
4u1t2i
uti5liz
u3tine
ut3ing
ution5a
u4tis
5u5tiz
u4t1l
ut5of
uto5g
uto5matic
u5ton
u4tou
uts4
u3u
uu4m
u1v2
uxu3
uz4e
1va
5va.
2v1a4b
vac5il
vac3u
vag4
va4ge
va5lie
val5o
val1u
va5mo
va5niz
va5pi
var5ied
3vat
4ve.
4ved
veg3
v3el.
vel3li
ve4lo
v4ely
ven3om
v5enue
v4erd
5vere.
v4erel
v3eren
ver5enc
v4eres
ver3ie
vermi4n
3verse
ver3th
v4e2s
4ves.
ves4te
ve4te
vet3er
ve4ty
vi5ali
5vian
5vide.
5vided
4v3iden
5vides
5vidi
v3if
vi5gn
vik4
2vil
5vilit
v3i3liz
v1in
4vi4na
v2inc
vin5d
4ving
vio3l
v3io4r
vi1ou
vi4p
vi5ro
vis3it
vi3so
vi3su
4viti
vit3r
4vity
3viv
5vo.
voi4
3vok
vo4la
v5ole
5volt
3volv
vom5i
vor5ab
vori4
vo4ry
vo4ta
4votee
4vv4
v4y
w5abl
2wac
wa5ger
wag5o
wait5
w5al.
wam4
war4t
was4t
wa1te
wa5ver
w1b
wea5rie
weath3
wed4n
weet3
wee5v
wel4l
w1er
west3
w3ev
whi4
wi2
wil2
will5in
win4de
win4g
wir4
3wise
with3
wiz5
w4k
wl4es
wl3in
w4no
1wo2
wom1
wo5ven
w5p
wra4
wri4
writa4
w3sh
ws4l
ws4pe
w5s4t
4wt
wy4
x1a
xac5e
x4ago
xam3
x4ap
xas5
x3c2
x1e
xe4cuto
x2ed
xer4i
xe5ro
x1h
xhi2
xhil5
xhu4
x3i
xi5a
xi5c
xi5di
x4ime
xi5miz
x3o
x4ob
x3p
xpan4d
xpecto5
xpe3d
x1t2
x3ti
x1u
xu3a
xx4
y5ac
3yar4
y5at
y1b
y1c
y2ce
yc5er
y3ch
ych4e
ycom4
ycot4
y1d
y5ee
y1er
y4erf
yes4
ye4t
y5gi
4y3h
y1i
y3la
ylla5bl
y3lo
y5lu
ymbol5
yme4
ympa3
yn3chr
yn5d
yn5g
yn5ic
5ynx
y1o4
yo5d
y4o5g
yom4
yo5net
y4ons
y4os
y4ped
yper5
yp3i
y3po
y4poc
yp2ta
y5pu
yra5m
yr5ia
y3ro
yr4r
ys4c
y3s2e
ys3ica
ys3io
3ysis
y4so
yss4
ys1t
ys3ta
ysur4
y3thin
yt3ic
y1w
za1
z5a2b
zar2
4zb
2ze
ze4n
ze4p
z1er
ze3ro
zet4
2z1i
z4il
z4is
5zl
4zm
1zo
zo4m
zo5ol
zte4
4z1z2
z4zy
}
\hyphenation{ % Do NOT make any alterations to this list! --- DEK
as-so-ciate
as-so-ciates
dec-li-na-tion
oblig-a-tory
phil-an-thropic
present
presents
project
projects
reci-procity
re-cog-ni-zance
ref-or-ma-tion
ret-ri-bu-tion
ta-ble
}
}
\let\@addtofilelist\@gobble
%    \end{macrocode}
%
%
%
% \subsection{Font loading}
%    Fonts loaded during the formatting process might already have
%    changed the |\font@submax| from |0pt| to something higher.
%    If so, we put out a bold warning.
% \changes{v0.1l}{1994/05/20}{Use new font warning commands}
%    \begin{macrocode}
% \changes{v1.1c}{2000/08/23}{Fix typo in warning}
\ifdim \font@submax >\z@
   \@font@warning{Size substitutions with differences\MessageBreak
                 up to \font@submax\space have occurred.\MessageBreak
                \MessageBreak
                Please check the transcript file
                carefully\MessageBreak
                and redo the format generation if necessary!
                \@gobbletwo}%
   \errhelp{Only stopped, to give you time to
            read the above message.}
   \errmessage{}
%    \end{macrocode}
%    We reset the macro. Otherwise every user will get a warning on
%    every job.
%    \begin{macrocode}
\def\font@submax{0pt}
\fi
%    \end{macrocode}
%
% \subsection{Input encoding}
%
% We temporarily define |\reserved@a| to apply |\reserved@c| to all the
% numbers in the range of its arguments.
%    \begin{macrocode}
\def\reserved@a#1#2{%
   \@tempcnta#1\relax
   \@tempcntb#2\relax
   \reserved@b
}
\def\reserved@b{%
   \ifnum\@tempcnta>\@tempcntb\else
      \reserved@c\@tempcnta
      \advance\@tempcnta\@ne
      \expandafter\reserved@b
   \fi
}
%    \end{macrocode}
% \changes{v0.1e}{1994/05/02}{Added setting the special catcodes.}
% \changes{v0.1f}{1994/05/02}{Set the catcode of control-J.}
% Set the special catcodes (although some of these are useless, since an
% error will have occurred if the catcodes have changed).  Note that
% |^^J| has catcode `other' for use in warning messages.
%    \begin{macrocode}
\catcode`\ =10
\catcode`\#=6
\catcode`\$=3
\catcode`\%=14
\catcode`\&=4
\catcode`\\=0
\catcode`\^=7
\catcode`\_=8
\catcode`\{=1
\catcode`\}=2
\catcode`\~=13
\catcode`\@=11
\catcode`\^^I=10
\catcode`\^^J=12
\catcode`\^^L=13
\catcode`\^^M=5
%    \end{macrocode}
% \changes{v0.1e}{1994/05/02}{Added setting the `other' catcodes.}
% Set the `other' catcodes.
%    \begin{macrocode}
\def\reserved@c#1{\catcode#1=12\relax}
\reserved@c{`\!}
\reserved@c{`\"}
\reserved@a{`\'}{`\?}
\reserved@c{`\[}
\reserved@c{`\]}
\reserved@c{`\`}
\reserved@c{`\|}
%    \end{macrocode}
% \changes{v0.1e}{1994/05/02}{Added setting the `letter' catcodes.}
% Set the `letter' catcodes.
%    \begin{macrocode}
\def\reserved@c#1{\catcode#1=11\relax}
\reserved@a{`\A}{`\Z}
\reserved@a{`\a}{`\z}
%    \end{macrocode}
% \changes{v0.1e}{1994/05/02}{Made slot 127 illegal}
% \changes{v1.0n}{1994/11/18}
%         {re-allow slots 127--255}
% All the characters in the range 0--31 and 127--255 are illegal,
% \emph{except} tab (|^^I|), nl (|^^J|), ff (|^^L|) and cr (|^^M|).
%
% Now allow 8-bit characters, although their use in this way is
% strongly discouraged. See |inputenc.dtx| for a supported mechanism
% for 8-bit input.
%    \begin{macrocode}
\def\reserved@c#1{\catcode#1=15\relax}
\reserved@a{0}{`\^^H}
\reserved@c{`\^^K}
\reserved@a{`\^^N}{31}
%\ifnum\inputlineno=\m@ne
  \catcode"7F=15
%\else
%  \reserved@a{"7F}{"FF}
%\fi
%    \end{macrocode}
%
% \subsection{Lccodes and uccodes}
%
% \changes{v1.1b}{1998/05/20}{Set up uc/lccodes after loading
%    hyphenation files: pr/2639}
%    We now again set up the default (T1) uc/lccodes.
%    The lower case characters need their |\uccode| and |\lccode| values
%    set. Some of this is a repeat of the set-up before loading
%    hyphenation files.
%    Depending on the \TeX{} version, we might not be allowed to do
%    this for non-ASCII characters.
% \changes{v1.0n}{1994/06/09}{For \TeX2, do not set codes for higher
%                   half of character table.}
%    \begin{macrocode}
\def\reserved@c#1{%
   \count@=#1\advance\count@ by -"20
   \uccode#1=\count@
   \lccode#1=#1
}
\reserved@a{`\a}{`\z}
\ifnum\inputlineno=\m@ne\else
  \reserved@a{"A0}{"BC}
  \reserved@a{"E0}{"FF}
\fi
%    \end{macrocode}
% The upper case characters need their |\uccode| and |\lccode| values
% set, and their |\sfcode| set to 999.
%    \begin{macrocode}
\def\reserved@c#1{%
   \count@=#1\advance\count@ by "20
   \uccode#1=#1
   \lccode#1=\count@
   \sfcode#1=999
}
\reserved@a{`\A}{`\Z}
\ifnum\inputlineno=\m@ne\else
  \reserved@a{"80}{"9C}
  \reserved@a{"C0}{"DF}
\fi
%    \end{macrocode}
% Well, it would be nice if that were correct, but unfortunately, the
% Cork encoding contains some odd slots whose uccode or lccode isn't
% quite what you'd expect.
%    \begin{macrocode}
\uccode`\^^Y=`\I     % dotless i
\lccode`\^^Y=`\^^Y   % dotless i
\uccode`\^^Z=`\J     % dotless j, ae in OT1
\lccode`\^^Z=`\^^Z   % dotless j, ae in OT1
\ifnum\inputlineno=\m@ne\else
  \lccode`\^^9d=`\i    % dotted I
  \uccode`\^^9d=`\^^9d % dotted I
  \lccode`\^^9e=`\^^9e % d-bar
  \uccode`\^^9e=`\^^d0 % d-bar
\fi
%    \end{macrocode}
% Finally here is one that helps hyphenation in the OT1 encoding.
% \changes{v1.0z}{1996/10/31}
%    {Added extra \cs{lcode}, hoping it does no harm in T1 (pr/1969)}
%    \begin{macrocode}
\lccode`\^^[=`\^^[   % oe in OT1
%    \end{macrocode}
%
% \begin{macro}{\MakeUppercase}
% \begin{macro}{\MakeUppercase}
% \begin{macro}{\@uclclist}
%
% \changes{v1.1a}{1997/10/20}{Removed \cs{aa} and \cs{AA} from
%    \cs{@uclclist} as these are macros.}
%
%    And whilst we're doing things with uc/lc tables, here are two
%    commands to upper- and lower-case a string.
%
%    \emph{Note} that this implementation is subject to change!  At
%    the moment we're not providing any way to extend the list of
%    uc/lc commands, since finding a good interface is difficult.
%    These commands have some nasty features, such as uppercasing
%    mathematics, environment names, labels, etc.  A much better
%    long-term solution is to use all-caps fonts, but these aren't
%    generally available.
%    \begin{macrocode}
\DeclareRobustCommand{\MakeUppercase}[1]{{%
      \def\i{I}\def\j{J}%
      \def\reserved@a##1##2{\let##1##2\reserved@a}%
      \expandafter\reserved@a\@uclclist\reserved@b{\reserved@b\@gobble}%
      \protected@edef\reserved@a{\uppercase{#1}}%
      \reserved@a
   }}
\DeclareRobustCommand{\MakeLowercase}[1]{{%
      \def\reserved@a##1##2{\let##2##1\reserved@a}%
      \expandafter\reserved@a\@uclclist\reserved@b{\reserved@b\@gobble}%
      \protected@edef\reserved@a{\lowercase{#1}}%
      \reserved@a
   }}
\def\@uclclist{\oe\OE\o\O\ae\AE
      \dh\DH\dj\DJ\l\L\ng\NG\ss\SS\th\TH}
%    \end{macrocode}
%    The above code works, but has the nasty side-effect that if you
%    say something like:
%\begin{verbatim}
%    \markboth{\MakeUppercase\contentsname}
%             {\MakeUppercase\contentsname}
%\end{verbatim}
%    then the uppercasing is only done to the first letter of the
%    contents name, since the mark expands out to:
%\begin{verbatim}
%    \mark{\protect\MakeUppercase Table of Contents}
%         {\protect\MakeUppercase Table of Contents}
%\end{verbatim}
%    In order to get round this, we redefine |\MakeUppercase| and
%    |\MakeLowercase| to grab their argument and brace it.  This is a
%    very low-level hack, and is \emph{not} recommended practice!
%    This is an instance of a general problem that makes it unsafe to
%    grab arguments unbraced, and probably needs a more general
%    solution.  For the moment though, this hack will do:
%    \begin{macrocode}
\protected@edef\MakeUppercase#1{\MakeUppercase{#1}}
\protected@edef\MakeLowercase#1{\MakeLowercase{#1}}
%    \end{macrocode}
% \end{macro}
% \end{macro}
% \end{macro}
%
% \changes{v1.0h}{1994/05/13}{Added output enc stuff}
% \changes{v1.0i}{1994/05/16}{moved output enc stuff to lfonts}
%
% \changes{v0.1a}{1994/03/07}{Add code from the old dump.dtx}
%
% \subsection{Applying Patch files}
% Between major releases, small patches will be distributed in
% files |ltpatch.ltx| which must be added at this point.
% \changes{v1.0m}{1994/06/08}{Add patch file system}
%    \begin{macrocode}
\IfFileExists{ltpatch.ltx}
  {\typeout{=================================^^J%
             Applying patch file ltpatch.ltx^^J%
            =================================}
   \def\fmtversion@topatch{unknown}
   \input{ltpatch.ltx}
   \ifx\fmtversion\fmtversion@topatch
      \ifx\patch@level\@undefined
        \typeout{^^J^^J^^J%
         !!!!!!!!!!!!!!!!!!!!!!!!!!!!!!!!!!!!!!!!!!!!!!!!!!!!!!^^J%
         !! Patch file `ltpatch.ltx' not suitable for this^^J%
         !! version of LaTeX.^^J^^J%
         !! Please check if initex found an old patch file:^^J%
         !! --- if so, rename it or delete it, and redo the^^J%
         !! initex run.^^J%
         !!!!!!!!!!!!!!!!!!!!!!!!!!!!!!!!!!!!!!!!!!!!!!!!!!!!!!^^J}%
        \batchmode \@@end
      \else
%    \end{macrocode}
% \changes{v1.0q}{1995/04/21}
%         {Allow initial patch level 0}
% \changes{v1.0t}{1995/06/13}
%         {Add patch level string more carefully}
% The code below adds the `patch level' string to the first |\typeout|
% in the startup banner.
%    \begin{macrocode}
        \def\fmtversion@topatch{0}%
        \ifx\fmtversion@topatch\patch@level\else
          \def\reserved@a\typeout##1##2\reserved@a{%
                 \typeout{##1 patch level \patch@level}##2}
          \everyjob\expandafter\expandafter\expandafter{%
             \expandafter\reserved@a\the\everyjob\reserved@a}
          \let\reserved@a\relax
          \the\everyjob
        \fi
      \fi
   \else
      \typeout{^^J^^J^^J%
     !!!!!!!!!!!!!!!!!!!!!!!!!!!!!!!!!!!!!!!!!!!!!!!!!!!!!!^^J%
     !! Patch file `ltpatch.ltx' (for version <\fmtversion@topatch>)^^J%
     !! is not suitable for version <\fmtversion> of LaTeX.^^J^^J%
     !! Please check if initex found an old patch file:^^J%
     !! --- if so, rename it or delete it, and redo the^^J%
     !!     initex run.^^J%
     !!!!!!!!!!!!!!!!!!!!!!!!!!!!!!!!!!!!!!!!!!!!!!!!!!!!!!^^J}%
       \batchmode \@@end
   \fi
   \let\fmtversion@topatch\relax
  }{}
%    \end{macrocode}
%
% \subsection{Freeing Memory}
%
% \begin{macro}{\reserved@a}
% \begin{macro}{\reserved@b}
% \changes{v1.0v}{1995/10/17}{reset here after the \cs{input} above}
% And just to make sure nobody relies on those definitions of
% |\reserved@b| and friends.
% These macros are reserved for use in the kernel. \emph{Do not use
% them as general scratch macros}.
%    \begin{macrocode}
\let\reserved@a\@filelist
\let\reserved@b=\@undefined
\let\reserved@c=\@undefined
\let\reserved@d=\@undefined
\let\reserved@e=\@undefined
\let\reserved@f=\@undefined
%    \end{macrocode}
% \end{macro}
% \end{macro}
%
% \begin{macro}{\toks}
% \changes{v1.0y}{1996/07/10}
%      {Free up memory from scratch registers /2213}
%    \begin{macrocode}
\toks0{}
\toks2{}
\toks4{}
\toks6{}
\toks8{}
%    \end{macrocode}
% \end{macro}
%
% \begin{macro}{\errhelp}
% \changes{v0.1g}{1994/05/05}{Set error help empty.}
% \changes{v1.1d}{2000/09/01}{Set error help empty at very end
%                             (pr/449 done correctly).}
% Empty the error help message, which may have some rubbish:
%    \begin{macrocode}
\errhelp{}
%    \end{macrocode}
% \end{macro}
%
% \subsection{Initialise file list}
%
% \begin{macro}{\@providesfile}
% \changes{v1.0v}{1995/10/17}{reset macro}
% Initialise for use in the document. During initex a modified version
% has been used which leaves debugging information for |latexbug.tex|.
%    \begin{macrocode}
\def\@providesfile#1[#2]{%
    \wlog{File: #1 #2}%
    \expandafter\xdef\csname ver@#1\endcsname{#2}%
  \endgroup}
%    \end{macrocode}
% \end{macro}
%
% \begin{macro}{\@filelist}
% \changes{v1.0w}{1995/10/19}{Move after \cs{reserved@a} setting:-)}
% \begin{macro}{\@addtofilelist}
% Reset |\@filelist| so files input while making the format are not
% listed. The list built up so far may take up a lot of memory and so
% it is moved to |\reserved@a| where it will be overwritten as soon
% as almost any \LaTeX\ command is issued in a class file.
% However the |latexbug.tex| program will be able to access this
% information and insert it into a bug report.
%    \begin{macrocode}
\let\@filelist\@gobble
\def\@addtofilelist#1{\xdef\@filelist{\@filelist,#1}}%
%    \end{macrocode}
% \end{macro}
% \end{macro}
%
% \subsection{Dumping the format}
%    Finally we make |@| into a letter, ensure the format will
% be in the `normal' error mode, and dump everything into the
%    format file.
% \changes{v1.0t}{1995/06/13}
%         {Call \cs{errorstopmode}}
%    \begin{macrocode}
\makeatother
\errorstopmode
\dump
%</2ekernel>
%    \end{macrocode}
%
% \Finale
%
