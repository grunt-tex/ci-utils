\ProvidesFile{fnlineno.tex}[2011/02/14 documenting fnlineno.sty (UL)]
\title{\textsf{fnlineno.sty}\\---\\Numbering Footnote Lines\thanks{This 
       document %%% manual %% 2010/12/28
       describes version 
       \textcolor{blue}{\UseVersionOf{fnlineno.sty}} 
       of \textsf{fnlineno.sty} as of \UseDateOf{fnlineno.sty}.}}
% \listfiles                                          %% 2010/12/22
{ \RequirePackage{makedoc}[2010/12/20] \ProcessLineMessage{} 
  \MakeJobDoc{19}{\SectionLevelThreeParseInput}     %% 2010/12/16
}
\documentclass{article}%% TODO paper dimensions!?
\input{makedoc.cfg} %% shared formatting settings
\newcommand*{\lt}{<} \newcommand*{\gt}{>}           %% 2010/12/22
\providecommand*{\strong}{\textbf}                  %% 2010/12/15
\ReadPackageInfos{fnlineno}
\usepackage{color}
% \hypersetup{bookmarksopen}    %% rm. 2010/12/21, cf. .cfg
%% 2010/12/21: %% 2010/12/26 not sure, splits code
% \makeatletter \@beginparpenalty\@highpenalty \makeatother
%% 2010/12/27:
\makeatletter \@beginparpenalty\@lowpenalty \makeatother
\sloppy
\begin{document}
\maketitle
\begin{abstract}\noindent
'fnlineno.sty' extends 
% Stephan~I. B\"ottcher's 
\CtanPkgRef{lineno}{lineno.sty}\urlpkgfoot{lineno}
(created by Stephan~I. B\"ottcher) 
such that even 
`\footnote'                                 %% `\' 2010/12/09
lines are numbered and can be referred to 
using `\linelabel', `\ref', etc. 
%% rm. 2011/02/09:
% Version v0.5 aims at working as a user expects 
% (just cf.~``Limitations"), otherwise please complain!

Making the package was motivated as support for 
\emph{critical editions}
% of scientific work from an age when footnotes 
% were a standard in publishing in print, 
%% <- 2011/02/09 ->
of \emph{printed works with footnotes} 
as opposed to scholarly critical editions of \emph{manuscripts.} 
For this purpose, an extension 'edfnotes' of the \ctanpkgref{ednotes}
package for critical editions, building on 'fnlineno', is provided 
by the \textit{ednotes} bundle.\urlfoot{CtanPkgRef}{ednotes}

'lineno.sty' has also been used for the revision process 
of \emph{submissions.} 
With 'fnlineno.sty', reference to footnotes 
in the submitted work may become possible. 

%% rm. 2011/02/09:
% Another standalone package 'finstrut' is described. 
As to \emph{implementation:}    %% 2011/02/14
1.~Some included tools for 
\emph{storing and restoring global settings} 
may be ``exported" as standalone packages later. 
2.~The method of typesetting footnotes on the main vertical list 
may later lead to applying the line numbering method to 
several \emph{parallel} texts (with footnotes) and to 
`inner' material such as table cells.
%% <- 2011/02/14 ->
% \dots

%% new 2011/02/09:
  \smallskip\noindent
\strong{Keywords:}\quad line numbers; footnotes, pagewise, 
critical editions, revision
\end{abstract}
\tableofcontents

%   \newpage                    %% rm. 2011/02/09
\section{Usage and Features}
\subsection{Package File Header (Legalize)}
\ProvidesFile{fnlineno.tex}[2011/02/14 documenting fnlineno.sty (UL)]
\title{\textsf{fnlineno.sty}\\---\\Numbering Footnote Lines\thanks{This 
       document %%% manual %% 2010/12/28
       describes version 
       \textcolor{blue}{\UseVersionOf{fnlineno.sty}} 
       of \textsf{fnlineno.sty} as of \UseDateOf{fnlineno.sty}.}}
% \listfiles                                          %% 2010/12/22
{ \RequirePackage{makedoc}[2010/12/20] \ProcessLineMessage{} 
  \MakeJobDoc{19}{\SectionLevelThreeParseInput}     %% 2010/12/16
}
\documentclass{article}%% TODO paper dimensions!?
\input{makedoc.cfg} %% shared formatting settings
\newcommand*{\lt}{<} \newcommand*{\gt}{>}           %% 2010/12/22
\providecommand*{\strong}{\textbf}                  %% 2010/12/15
\ReadPackageInfos{fnlineno}
\usepackage{color}
% \hypersetup{bookmarksopen}    %% rm. 2010/12/21, cf. .cfg
%% 2010/12/21: %% 2010/12/26 not sure, splits code
% \makeatletter \@beginparpenalty\@highpenalty \makeatother
%% 2010/12/27:
\makeatletter \@beginparpenalty\@lowpenalty \makeatother
\sloppy
\begin{document}
\maketitle
\begin{abstract}\noindent
'fnlineno.sty' extends 
% Stephan~I. B\"ottcher's 
\CtanPkgRef{lineno}{lineno.sty}\urlpkgfoot{lineno}
(created by Stephan~I. B\"ottcher) 
such that even 
`\footnote'                                 %% `\' 2010/12/09
lines are numbered and can be referred to 
using `\linelabel', `\ref', etc. 
%% rm. 2011/02/09:
% Version v0.5 aims at working as a user expects 
% (just cf.~``Limitations"), otherwise please complain!

Making the package was motivated as support for 
\emph{critical editions}
% of scientific work from an age when footnotes 
% were a standard in publishing in print, 
%% <- 2011/02/09 ->
of \emph{printed works with footnotes} 
as opposed to scholarly critical editions of \emph{manuscripts.} 
For this purpose, an extension 'edfnotes' of the \ctanpkgref{ednotes}
package for critical editions, building on 'fnlineno', is provided 
by the \textit{ednotes} bundle.\urlfoot{CtanPkgRef}{ednotes}

'lineno.sty' has also been used for the revision process 
of \emph{submissions.} 
With 'fnlineno.sty', reference to footnotes 
in the submitted work may become possible. 

%% rm. 2011/02/09:
% Another standalone package 'finstrut' is described. 
As to \emph{implementation:}    %% 2011/02/14
1.~Some included tools for 
\emph{storing and restoring global settings} 
may be ``exported" as standalone packages later. 
2.~The method of typesetting footnotes on the main vertical list 
may later lead to applying the line numbering method to 
several \emph{parallel} texts (with footnotes) and to 
`inner' material such as table cells.
%% <- 2011/02/14 ->
% \dots

%% new 2011/02/09:
  \smallskip\noindent
\strong{Keywords:}\quad line numbers; footnotes, pagewise, 
critical editions, revision
\end{abstract}
\tableofcontents

%   \newpage                    %% rm. 2011/02/09
\section{Usage and Features}
\subsection{Package File Header (Legalize)}
\ProvidesFile{fnlineno.tex}[2011/02/14 documenting fnlineno.sty (UL)]
\title{\textsf{fnlineno.sty}\\---\\Numbering Footnote Lines\thanks{This 
       document %%% manual %% 2010/12/28
       describes version 
       \textcolor{blue}{\UseVersionOf{fnlineno.sty}} 
       of \textsf{fnlineno.sty} as of \UseDateOf{fnlineno.sty}.}}
% \listfiles                                          %% 2010/12/22
{ \RequirePackage{makedoc}[2010/12/20] \ProcessLineMessage{} 
  \MakeJobDoc{19}{\SectionLevelThreeParseInput}     %% 2010/12/16
}
\documentclass{article}%% TODO paper dimensions!?
\input{makedoc.cfg} %% shared formatting settings
\newcommand*{\lt}{<} \newcommand*{\gt}{>}           %% 2010/12/22
\providecommand*{\strong}{\textbf}                  %% 2010/12/15
\ReadPackageInfos{fnlineno}
\usepackage{color}
% \hypersetup{bookmarksopen}    %% rm. 2010/12/21, cf. .cfg
%% 2010/12/21: %% 2010/12/26 not sure, splits code
% \makeatletter \@beginparpenalty\@highpenalty \makeatother
%% 2010/12/27:
\makeatletter \@beginparpenalty\@lowpenalty \makeatother
\sloppy
\begin{document}
\maketitle
\begin{abstract}\noindent
'fnlineno.sty' extends 
% Stephan~I. B\"ottcher's 
\CtanPkgRef{lineno}{lineno.sty}\urlpkgfoot{lineno}
(created by Stephan~I. B\"ottcher) 
such that even 
`\footnote'                                 %% `\' 2010/12/09
lines are numbered and can be referred to 
using `\linelabel', `\ref', etc. 
%% rm. 2011/02/09:
% Version v0.5 aims at working as a user expects 
% (just cf.~``Limitations"), otherwise please complain!

Making the package was motivated as support for 
\emph{critical editions}
% of scientific work from an age when footnotes 
% were a standard in publishing in print, 
%% <- 2011/02/09 ->
of \emph{printed works with footnotes} 
as opposed to scholarly critical editions of \emph{manuscripts.} 
For this purpose, an extension 'edfnotes' of the \ctanpkgref{ednotes}
package for critical editions, building on 'fnlineno', is provided 
by the \textit{ednotes} bundle.\urlfoot{CtanPkgRef}{ednotes}

'lineno.sty' has also been used for the revision process 
of \emph{submissions.} 
With 'fnlineno.sty', reference to footnotes 
in the submitted work may become possible. 

%% rm. 2011/02/09:
% Another standalone package 'finstrut' is described. 
As to \emph{implementation:}    %% 2011/02/14
1.~Some included tools for 
\emph{storing and restoring global settings} 
may be ``exported" as standalone packages later. 
2.~The method of typesetting footnotes on the main vertical list 
may later lead to applying the line numbering method to 
several \emph{parallel} texts (with footnotes) and to 
`inner' material such as table cells.
%% <- 2011/02/14 ->
% \dots

%% new 2011/02/09:
  \smallskip\noindent
\strong{Keywords:}\quad line numbers; footnotes, pagewise, 
critical editions, revision
\end{abstract}
\tableofcontents

%   \newpage                    %% rm. 2011/02/09
\section{Usage and Features}
\subsection{Package File Header (Legalize)}
\ProvidesFile{fnlineno.tex}[2011/02/14 documenting fnlineno.sty (UL)]
\title{\textsf{fnlineno.sty}\\---\\Numbering Footnote Lines\thanks{This 
       document %%% manual %% 2010/12/28
       describes version 
       \textcolor{blue}{\UseVersionOf{fnlineno.sty}} 
       of \textsf{fnlineno.sty} as of \UseDateOf{fnlineno.sty}.}}
% \listfiles                                          %% 2010/12/22
{ \RequirePackage{makedoc}[2010/12/20] \ProcessLineMessage{} 
  \MakeJobDoc{19}{\SectionLevelThreeParseInput}     %% 2010/12/16
}
\documentclass{article}%% TODO paper dimensions!?
\input{makedoc.cfg} %% shared formatting settings
\newcommand*{\lt}{<} \newcommand*{\gt}{>}           %% 2010/12/22
\providecommand*{\strong}{\textbf}                  %% 2010/12/15
\ReadPackageInfos{fnlineno}
\usepackage{color}
% \hypersetup{bookmarksopen}    %% rm. 2010/12/21, cf. .cfg
%% 2010/12/21: %% 2010/12/26 not sure, splits code
% \makeatletter \@beginparpenalty\@highpenalty \makeatother
%% 2010/12/27:
\makeatletter \@beginparpenalty\@lowpenalty \makeatother
\sloppy
\begin{document}
\maketitle
\begin{abstract}\noindent
'fnlineno.sty' extends 
% Stephan~I. B\"ottcher's 
\CtanPkgRef{lineno}{lineno.sty}\urlpkgfoot{lineno}
(created by Stephan~I. B\"ottcher) 
such that even 
`\footnote'                                 %% `\' 2010/12/09
lines are numbered and can be referred to 
using `\linelabel', `\ref', etc. 
%% rm. 2011/02/09:
% Version v0.5 aims at working as a user expects 
% (just cf.~``Limitations"), otherwise please complain!

Making the package was motivated as support for 
\emph{critical editions}
% of scientific work from an age when footnotes 
% were a standard in publishing in print, 
%% <- 2011/02/09 ->
of \emph{printed works with footnotes} 
as opposed to scholarly critical editions of \emph{manuscripts.} 
For this purpose, an extension 'edfnotes' of the \ctanpkgref{ednotes}
package for critical editions, building on 'fnlineno', is provided 
by the \textit{ednotes} bundle.\urlfoot{CtanPkgRef}{ednotes}

'lineno.sty' has also been used for the revision process 
of \emph{submissions.} 
With 'fnlineno.sty', reference to footnotes 
in the submitted work may become possible. 

%% rm. 2011/02/09:
% Another standalone package 'finstrut' is described. 
As to \emph{implementation:}    %% 2011/02/14
1.~Some included tools for 
\emph{storing and restoring global settings} 
may be ``exported" as standalone packages later. 
2.~The method of typesetting footnotes on the main vertical list 
may later lead to applying the line numbering method to 
several \emph{parallel} texts (with footnotes) and to 
`inner' material such as table cells.
%% <- 2011/02/14 ->
% \dots

%% new 2011/02/09:
  \smallskip\noindent
\strong{Keywords:}\quad line numbers; footnotes, pagewise, 
critical editions, revision
\end{abstract}
\tableofcontents

%   \newpage                    %% rm. 2011/02/09
\section{Usage and Features}
\subsection{Package File Header (Legalize)}
\input{fnlineno.doc}
\end{document}

VERSION HISTORY

2010/12/08  for v0.1    very first
2010/12/09  for v0.2    moved much to .sty
2010/12/15  for v0.4    \strong
2010/12/16              \SectionLevelThree... 
2010/12/22ff.           beginparpenalty varied
2010/12/28  for v0.5    abstract extended
2011/02/09              removing `finstrut'; mention `edfnotes'
2011/02/10              using \urlpkgfoot etc.
2011/02/14              abstract modified

\end{document}

VERSION HISTORY

2010/12/08  for v0.1    very first
2010/12/09  for v0.2    moved much to .sty
2010/12/15  for v0.4    \strong
2010/12/16              \SectionLevelThree... 
2010/12/22ff.           beginparpenalty varied
2010/12/28  for v0.5    abstract extended
2011/02/09              removing `finstrut'; mention `edfnotes'
2011/02/10              using \urlpkgfoot etc.
2011/02/14              abstract modified

\end{document}

VERSION HISTORY

2010/12/08  for v0.1    very first
2010/12/09  for v0.2    moved much to .sty
2010/12/15  for v0.4    \strong
2010/12/16              \SectionLevelThree... 
2010/12/22ff.           beginparpenalty varied
2010/12/28  for v0.5    abstract extended
2011/02/09              removing `finstrut'; mention `edfnotes'
2011/02/10              using \urlpkgfoot etc.
2011/02/14              abstract modified

\end{document}

VERSION HISTORY

2010/12/08  for v0.1    very first
2010/12/09  for v0.2    moved much to .sty
2010/12/15  for v0.4    \strong
2010/12/16              \SectionLevelThree... 
2010/12/22ff.           beginparpenalty varied
2010/12/28  for v0.5    abstract extended
2011/02/09              removing `finstrut'; mention `edfnotes'
2011/02/10              using \urlpkgfoot etc.
2011/02/14              abstract modified
