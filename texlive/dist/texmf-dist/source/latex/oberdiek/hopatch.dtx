% \iffalse meta-comment
%
% File: hopatch.dtx
% Version: 2012/05/28 v1.2
% Info: Wrapper for package hooks
%
% Copyright (C) 2011, 2012 by
%    Heiko Oberdiek <heiko.oberdiek at googlemail.com>
%
% This work may be distributed and/or modified under the
% conditions of the LaTeX Project Public License, either
% version 1.3c of this license or (at your option) any later
% version. This version of this license is in
%    http://www.latex-project.org/lppl/lppl-1-3c.txt
% and the latest version of this license is in
%    http://www.latex-project.org/lppl.txt
% and version 1.3 or later is part of all distributions of
% LaTeX version 2005/12/01 or later.
%
% This work has the LPPL maintenance status "maintained".
%
% This Current Maintainer of this work is Heiko Oberdiek.
%
% This work consists of the main source file hopatch.dtx
% and the derived files
%    hopatch.sty, hopatch.pdf, hopatch.ins, hopatch.drv,
%    hopatch-test1.tex, hopatch-test2.tex.
%
% Distribution:
%    CTAN:macros/latex/contrib/oberdiek/hopatch.dtx
%    CTAN:macros/latex/contrib/oberdiek/hopatch.pdf
%
% Unpacking:
%    (a) If hopatch.ins is present:
%           tex hopatch.ins
%    (b) Without hopatch.ins:
%           tex hopatch.dtx
%    (c) If you insist on using LaTeX
%           latex \let\install=y% \iffalse meta-comment
%
% File: hopatch.dtx
% Version: 2012/05/28 v1.2
% Info: Wrapper for package hooks
%
% Copyright (C) 2011, 2012 by
%    Heiko Oberdiek <heiko.oberdiek at googlemail.com>
%
% This work may be distributed and/or modified under the
% conditions of the LaTeX Project Public License, either
% version 1.3c of this license or (at your option) any later
% version. This version of this license is in
%    http://www.latex-project.org/lppl/lppl-1-3c.txt
% and the latest version of this license is in
%    http://www.latex-project.org/lppl.txt
% and version 1.3 or later is part of all distributions of
% LaTeX version 2005/12/01 or later.
%
% This work has the LPPL maintenance status "maintained".
%
% This Current Maintainer of this work is Heiko Oberdiek.
%
% This work consists of the main source file hopatch.dtx
% and the derived files
%    hopatch.sty, hopatch.pdf, hopatch.ins, hopatch.drv,
%    hopatch-test1.tex, hopatch-test2.tex.
%
% Distribution:
%    CTAN:macros/latex/contrib/oberdiek/hopatch.dtx
%    CTAN:macros/latex/contrib/oberdiek/hopatch.pdf
%
% Unpacking:
%    (a) If hopatch.ins is present:
%           tex hopatch.ins
%    (b) Without hopatch.ins:
%           tex hopatch.dtx
%    (c) If you insist on using LaTeX
%           latex \let\install=y% \iffalse meta-comment
%
% File: hopatch.dtx
% Version: 2012/05/28 v1.2
% Info: Wrapper for package hooks
%
% Copyright (C) 2011, 2012 by
%    Heiko Oberdiek <heiko.oberdiek at googlemail.com>
%
% This work may be distributed and/or modified under the
% conditions of the LaTeX Project Public License, either
% version 1.3c of this license or (at your option) any later
% version. This version of this license is in
%    http://www.latex-project.org/lppl/lppl-1-3c.txt
% and the latest version of this license is in
%    http://www.latex-project.org/lppl.txt
% and version 1.3 or later is part of all distributions of
% LaTeX version 2005/12/01 or later.
%
% This work has the LPPL maintenance status "maintained".
%
% This Current Maintainer of this work is Heiko Oberdiek.
%
% This work consists of the main source file hopatch.dtx
% and the derived files
%    hopatch.sty, hopatch.pdf, hopatch.ins, hopatch.drv,
%    hopatch-test1.tex, hopatch-test2.tex.
%
% Distribution:
%    CTAN:macros/latex/contrib/oberdiek/hopatch.dtx
%    CTAN:macros/latex/contrib/oberdiek/hopatch.pdf
%
% Unpacking:
%    (a) If hopatch.ins is present:
%           tex hopatch.ins
%    (b) Without hopatch.ins:
%           tex hopatch.dtx
%    (c) If you insist on using LaTeX
%           latex \let\install=y% \iffalse meta-comment
%
% File: hopatch.dtx
% Version: 2012/05/28 v1.2
% Info: Wrapper for package hooks
%
% Copyright (C) 2011, 2012 by
%    Heiko Oberdiek <heiko.oberdiek at googlemail.com>
%
% This work may be distributed and/or modified under the
% conditions of the LaTeX Project Public License, either
% version 1.3c of this license or (at your option) any later
% version. This version of this license is in
%    http://www.latex-project.org/lppl/lppl-1-3c.txt
% and the latest version of this license is in
%    http://www.latex-project.org/lppl.txt
% and version 1.3 or later is part of all distributions of
% LaTeX version 2005/12/01 or later.
%
% This work has the LPPL maintenance status "maintained".
%
% This Current Maintainer of this work is Heiko Oberdiek.
%
% This work consists of the main source file hopatch.dtx
% and the derived files
%    hopatch.sty, hopatch.pdf, hopatch.ins, hopatch.drv,
%    hopatch-test1.tex, hopatch-test2.tex.
%
% Distribution:
%    CTAN:macros/latex/contrib/oberdiek/hopatch.dtx
%    CTAN:macros/latex/contrib/oberdiek/hopatch.pdf
%
% Unpacking:
%    (a) If hopatch.ins is present:
%           tex hopatch.ins
%    (b) Without hopatch.ins:
%           tex hopatch.dtx
%    (c) If you insist on using LaTeX
%           latex \let\install=y\input{hopatch.dtx}
%        (quote the arguments according to the demands of your shell)
%
% Documentation:
%    (a) If hopatch.drv is present:
%           latex hopatch.drv
%    (b) Without hopatch.drv:
%           latex hopatch.dtx; ...
%    The class ltxdoc loads the configuration file ltxdoc.cfg
%    if available. Here you can specify further options, e.g.
%    use A4 as paper format:
%       \PassOptionsToClass{a4paper}{article}
%
%    Programm calls to get the documentation (example):
%       pdflatex hopatch.dtx
%       makeindex -s gind.ist hopatch.idx
%       pdflatex hopatch.dtx
%       makeindex -s gind.ist hopatch.idx
%       pdflatex hopatch.dtx
%
% Installation:
%    TDS:tex/latex/oberdiek/hopatch.sty
%    TDS:doc/latex/oberdiek/hopatch.pdf
%    TDS:doc/latex/oberdiek/test/hopatch-test1.tex
%    TDS:doc/latex/oberdiek/test/hopatch-test2.tex
%    TDS:source/latex/oberdiek/hopatch.dtx
%
%<*ignore>
\begingroup
  \catcode123=1 %
  \catcode125=2 %
  \def\x{LaTeX2e}%
\expandafter\endgroup
\ifcase 0\ifx\install y1\fi\expandafter
         \ifx\csname processbatchFile\endcsname\relax\else1\fi
         \ifx\fmtname\x\else 1\fi\relax
\else\csname fi\endcsname
%</ignore>
%<*install>
\input docstrip.tex
\Msg{************************************************************************}
\Msg{* Installation}
\Msg{* Package: hopatch 2012/05/28 v1.2 Wrapper for package hooks (HO)}
\Msg{************************************************************************}

\keepsilent
\askforoverwritefalse

\let\MetaPrefix\relax
\preamble

This is a generated file.

Project: hopatch
Version: 2012/05/28 v1.2

Copyright (C) 2011, 2012 by
   Heiko Oberdiek <heiko.oberdiek at googlemail.com>

This work may be distributed and/or modified under the
conditions of the LaTeX Project Public License, either
version 1.3c of this license or (at your option) any later
version. This version of this license is in
   http://www.latex-project.org/lppl/lppl-1-3c.txt
and the latest version of this license is in
   http://www.latex-project.org/lppl.txt
and version 1.3 or later is part of all distributions of
LaTeX version 2005/12/01 or later.

This work has the LPPL maintenance status "maintained".

This Current Maintainer of this work is Heiko Oberdiek.

This work consists of the main source file hopatch.dtx
and the derived files
   hopatch.sty, hopatch.pdf, hopatch.ins, hopatch.drv,
   hopatch-test1.tex, hopatch-test2.tex.

\endpreamble
\let\MetaPrefix\DoubleperCent

\generate{%
  \file{hopatch.ins}{\from{hopatch.dtx}{install}}%
  \file{hopatch.drv}{\from{hopatch.dtx}{driver}}%
  \usedir{tex/latex/oberdiek}%
  \file{hopatch.sty}{\from{hopatch.dtx}{package}}%
  \usedir{doc/latex/oberdiek/test}%
  \file{hopatch-test1.tex}{\from{hopatch.dtx}{test1}}%
  \file{hopatch-test2.tex}{\from{hopatch.dtx}{test2}}%
  \nopreamble
  \nopostamble
  \usedir{source/latex/oberdiek/catalogue}%
  \file{hopatch.xml}{\from{hopatch.dtx}{catalogue}}%
}

\catcode32=13\relax% active space
\let =\space%
\Msg{************************************************************************}
\Msg{*}
\Msg{* To finish the installation you have to move the following}
\Msg{* file into a directory searched by TeX:}
\Msg{*}
\Msg{*     hopatch.sty}
\Msg{*}
\Msg{* To produce the documentation run the file `hopatch.drv'}
\Msg{* through LaTeX.}
\Msg{*}
\Msg{* Happy TeXing!}
\Msg{*}
\Msg{************************************************************************}

\endbatchfile
%</install>
%<*ignore>
\fi
%</ignore>
%<*driver>
\NeedsTeXFormat{LaTeX2e}
\ProvidesFile{hopatch.drv}%
  [2012/05/28 v1.2 Wrapper for package hooks (HO)]%
\documentclass{ltxdoc}
\usepackage{holtxdoc}[2011/11/22]
\usepackage{caption}
\usepackage{array}
\setlength{\extrarowheight}{1pt}
\renewcommand*{\arraystretch}{1.1}
\begin{document}
  \DocInput{hopatch.dtx}%
\end{document}
%</driver>
% \fi
%
% \CheckSum{466}
%
% \CharacterTable
%  {Upper-case    \A\B\C\D\E\F\G\H\I\J\K\L\M\N\O\P\Q\R\S\T\U\V\W\X\Y\Z
%   Lower-case    \a\b\c\d\e\f\g\h\i\j\k\l\m\n\o\p\q\r\s\t\u\v\w\x\y\z
%   Digits        \0\1\2\3\4\5\6\7\8\9
%   Exclamation   \!     Double quote  \"     Hash (number) \#
%   Dollar        \$     Percent       \%     Ampersand     \&
%   Acute accent  \'     Left paren    \(     Right paren   \)
%   Asterisk      \*     Plus          \+     Comma         \,
%   Minus         \-     Point         \.     Solidus       \/
%   Colon         \:     Semicolon     \;     Less than     \<
%   Equals        \=     Greater than  \>     Question mark \?
%   Commercial at \@     Left bracket  \[     Backslash     \\
%   Right bracket \]     Circumflex    \^     Underscore    \_
%   Grave accent  \`     Left brace    \{     Vertical bar  \|
%   Right brace   \}     Tilde         \~}
%
% \GetFileInfo{hopatch.drv}
%
% \title{The \xpackage{hopatch} package}
% \date{2012/05/28 v1.2}
% \author{Heiko Oberdiek\\\xemail{heiko.oberdiek at googlemail.com}}
%
% \maketitle
%
% \begin{abstract}
% This packages provides a wrapper to various package hooks
% provided by other packages or classes, but does not define
% own hooks.
% \end{abstract}
%
% \tableofcontents
%
% \section{Documentation}
%
% Sometimes I want to add code right after a package
% has been loaded. Examples are bug fixes, adaptations,
% or added features as needed by package \xpackage{hyperref},
% for instance.
%
% Unhappily \hologo{LaTeX} does not provide this kind of
% hook. \cs{AtEndOfPackage} can be used inside the package
% only, because \hologo{LaTeX} clears the hook right before
% it loads the package.
%
% However, there are already many packages and classes
% that provide hooks that are executed after the package
% is loaded, see table~\ref{tab:hooks}.
%
% \begin{table}
% \caption{After package hooking}
% \label{tab:hooks}
% \begin{tabular}{ll}
% \bfseries Macro & \bfseries Provider\\
% \hline
% \cs{AfterPackage} & package \xpackage{scrlfile} \cite{scrlfile}\\
% \cs{AtEndOfPackageFile} & package \xpackage{filehook} \cite{filehook}\\
% \cs{AtEndPackage} & class \xclass{memoir} \cite{memoir}
% \end{tabular}
% \end{table}
%
% Package \xpackage{hopatch} can be used without the packages
% of table~\ref{tab:hooks}. But for an early executing right
% after a package is loaded, one of the following class or
% packages should be loaded before using \cs{hopatch@AfterPackage}:
% \begin{itemize}
%  \item package \xpackage{filehook}
%  \item package \xpackage{scrlfile}
%  \item class \xclass{memoir}
% \end{itemize}
%
% \begin{table}
% \caption{After begin document hooking}
% \begin{tabular}{ll}
% \bfseries Macro & \bfseries Provider\\
% \hline
% \cs{AtBeginDocument} & \hologo{LaTeX}'s kernel\\
% \cs{AtEndPreamble} & package \xpackage{etoolbox} \cite{etoolbox}\\
% \cs{AfterEndPreamble} & package \xpackage{etoolbox}
% \end{tabular}
% \end{table}
%
% Therefore I skip writing a new package for hooking into
% \hologo{LaTeX}'s package management and use this package
% to provide a wrapper to patch a package after it is loaded.
%
% \begin{declcs}{hopatch@AfterPackage} \M{package} \M{patch code}
% \end{declcs}
% If the package is already loaded, the \meta{patch code} is executed
% immediately.
% Otherwise the \meta{patch code} is stored in a command and tried
% at later locations until the package is available.
%
% The patch is tried in the following order:
% \begin{enumerate}
% \item If the package is already loaded, the patch is
%   applied immediately. Further locations are not tried.
% \item \cs{AtEndPackage}, provided by class \xclass{memoir} \cite{memoir},
%   and \cs{AfterPackage}, provided by package \xpackage{scrlfile} \cite{scrlfile},
%   are called right after the package file is input before
%   the hook of \hologo{LaTeX}'s \cs{AtEndOfPackage}.
% \item \cs{AtEndOfPackageFile}, provided by package \xpackage{filehook} \cite{filehook},
%   is called after the package is loaded and after the hook of
%   \hologo{LaTeX}'s \cs{AtEndOfPackage}.
% \item \cs{AtEndPreamble}, provided by package \xpackage{etoolbox} \cite{etoolbox},
%   is called at the beginning of |\begin{document}| before the hook
%   of \hologo{LaTeX}'s \cs{AtBeginDocument}.
% \item \cs{AtBeginDocument}, provided by \hologo{LaTeX}.
% \item \cs{AfterEndDocument}, provided by package \xpackage{etoolbox} \cite{etoolbox},
%   is called at the very end of |\begin{document}|. Preamble commands are
%   already forbidden there.
% \end{enumerate}
% Because of the various locations the patch code is restricted to
% limitations:
% \begin{itemize}
% \item
% Preamble commands, see \hologo{LaTeX}'s \cs{@onlypreamble}
% throw an error if used after |\begin{document}|. This is already
% the case for \cs{AfterEndDocument}. Therefore preamble commands
% are forbidden in the patching code.
% There are four exceptions \cs{@ifpackageloaded}, \cs{@ifclassloaded},
% \cs{@ifpackagelater} and \cs{@ifclasslater}. They are redefined
% during \cs{AfterEndDocument} using the counterparts of
% package \xpackage{ltxcmds} \cite{ltxcmds}.
% \item
% \cs{AfterPackage} of package \xpackage{scrlfile} and \cs{AtEndPackage}
% of class \xclass{memoir} call the hook before \hologo{LaTeX}'s
% \cs{AtEndOfPackage}.
% \end{itemize}
%
% \StopEventually{
% }
%
% \section{Implementation}
%
%    \begin{macrocode}
%<*package>
%    \end{macrocode}
%
% \subsection{Catcodes and package identification}
%
%    \begin{macrocode}
\begingroup\catcode61\catcode48\catcode32=10\relax%
  \catcode13=5 % ^^M
  \endlinechar=13 %
  \catcode123=1 % {
  \catcode125=2 % }
  \catcode64=11 % @
  \def\x{\endgroup
    \expandafter\edef\csname HOpatch@AtEnd\endcsname{%
      \endlinechar=\the\endlinechar\relax
      \catcode13=\the\catcode13\relax
      \catcode32=\the\catcode32\relax
      \catcode35=\the\catcode35\relax
      \catcode61=\the\catcode61\relax
      \catcode64=\the\catcode64\relax
      \catcode123=\the\catcode123\relax
      \catcode125=\the\catcode125\relax
    }%
  }%
\x\catcode61\catcode48\catcode32=10\relax%
\catcode13=5 % ^^M
\endlinechar=13 %
\catcode35=6 % #
\catcode64=11 % @
\catcode123=1 % {
\catcode125=2 % }
\def\TMP@EnsureCode#1#2{%
  \edef\HOpatch@AtEnd{%
    \HOpatch@AtEnd
    \catcode#1=\the\catcode#1\relax
  }%
  \catcode#1=#2\relax
}
\TMP@EnsureCode{40}{12}% (
\TMP@EnsureCode{41}{12}% )
\TMP@EnsureCode{43}{12}% +
\TMP@EnsureCode{46}{12}% .
\TMP@EnsureCode{47}{12}% /
\TMP@EnsureCode{91}{12}% [
\TMP@EnsureCode{93}{12}% ]
\edef\HOpatch@AtEnd{\HOpatch@AtEnd\noexpand\endinput}
%    \end{macrocode}
%
%    Package identification.
%    \begin{macrocode}
\NeedsTeXFormat{LaTeX2e}
\ProvidesPackage{hopatch}%
  [2012/05/28 v1.2 Wrapper for package hooks (HO)]
%    \end{macrocode}
%
% \subsection{Resources}
%
%    \begin{macrocode}
\begingroup\expandafter\expandafter\expandafter\endgroup
\expandafter\ifx\csname RequirePackage\endcsname\relax
  \def\TMP@RequirePackage#1[#2]{%
    \begingroup\expandafter\expandafter\expandafter\endgroup
    \expandafter\ifx\csname ver@#1.sty\endcsname\relax
      \input #1.sty\relax
    \fi
  }%
  \TMP@RequirePackage{ltxcmds}[2010/12/12]%
\else
  \RequirePackage{ltxcmds}[2010/12/12]%
\fi
%    \end{macrocode}
%
%    \begin{macro}{\HOpatch@counter}
%    \begin{macrocode}
\def\HOpatch@counter{0}%
%    \end{macrocode}
%    \end{macro}
%    \begin{macro}{\HOpatch@StepCounter}
%    \begin{macrocode}
\ltx@ifundefined{numexpr}{%
  \def\HOpatch@StepCounter{%
    \begingroup
      \count@\HOpatch@counter\relax
      \advance\count@\ltx@one\relax
    \edef\x{\endgroup
      \noexpand\def\noexpand\HOpatch@counter{\the\count@}%
    }%
    \x
  }%
}{%
  \def\HOpatch@StepCounter{%
    \edef\HOpatch@counter{%
      \the\numexpr\HOpatch@counter+\ltx@one\relax
    }%
  }%
}
%    \end{macrocode}
%    \end{macro}
%
%    \begin{macro}{\HOpatch@list}
%    \begin{macrocode}
\def\HOpatch@list{}
%    \end{macrocode}
%    \end{macro}
%
%    \begin{macro}{\HOpatch@Add}
%    \begin{macrocode}
\def\HOpatch@Add{%
  \ltx@LocalAppendToMacro\HOpatch@list
}
%    \end{macrocode}
%    \end{macro}
%
% \subsection{Package patching}
%
%    \begin{macro}{\hopatch@AfterPackage}
%    \begin{macrocode}
\def\hopatch@AfterPackage#1{%
  \ltx@ifpackageloaded{#1}{%
    \ltx@firstofone
  }{%
    \HOpatch@AfterPackage{#1}%
  }%
}
%    \end{macrocode}
%    \end{macro}
%    \begin{macro}{\HOpatch@AfterPackage}
%    \begin{macrocode}
\def\HOpatch@AfterPackage#1{%
  \edef\HOpatch@temp{#1}%
  \HOpatch@StepCounter
  \expandafter\HOpatch@@AfterPackage
  \csname HOpatch@\HOpatch@counter\expandafter\endcsname{%
    \HOpatch@temp
  }%
}
%    \end{macrocode}
%    \end{macro}
%
%    \begin{macro}{\HOpatch@@AfterPackage}
%    \begin{macrocode}
\def\HOpatch@@AfterPackage#1#2#3{%
  \begingroup
    \toks@{#3}%
    \xdef\HOpatch@gtemp{%
      \noexpand\ltx@ifpackageloaded{#2}{%
        \noexpand\let\noexpand#1\noexpand\relax
        \the\toks@
      }{}%
    }%
  \endgroup
  \let#1\HOpatch@gtemp
  \HOpatch@Add#1%
  \HOpatch@Try{AfterPackage}{#2}#1%
  \HOpatch@Try{AtEndPackage}{#2}#1%
  \HOpatch@Try{AtEndOfPackageFile}{#2}#1%
}
%    \end{macrocode}
%    \end{macro}
%
%    \begin{macro}{\HOpatch@Try}
%    \begin{macrocode}
\def\HOpatch@Try#1#2#3{%
  \ltx@ifundefined{#1}{}{%
    \csname #1\endcsname{#2}{#3}%
  }%
}
%    \end{macrocode}
%    \end{macro}
%
%    \begin{macrocode}
\AtBeginDocument{\HOpatch@list}
\ltx@ifundefined{AtEndPreamble}{}{%
  \ltx@ifundefined{@endpreamblehook}{}{%
    \AtEndPreamble{\HOpatch@list}%
  }%
}
\ltx@ifundefined{AfterEndPreamble}{}{%
  \ltx@ifundefined{@afterendpreamblehook}{}{%
    \AfterEndPreamble{%
      \let\HOpatch@OrgIfPackageLoaded\@ifpackageloaded
      \let\HOpatch@OrgIfPackageLater\@ifpackagelater
      \let\HOpatch@OrgIfClassLoaded\@ifclassloaded
      \let\HOpatch@OrgIfClassLater\@ifclasslater
      \let\@ifpackageloaded\ltx@ifpackageloaded
      \let\@ifpackagelater\ltx@ifpackagelater
      \let\@ifclassloaded\ltx@ifclassloaded
      \let\@ifclasslater\ltx@ifclasslater
      \HOpatch@list
      \let\@ifpackageloaded\HOpatch@OrgIfPackageLoaded
      \let\@ifpackagelater\HOpatch@OrgIfPackageLater
      \let\@ifclassloaded\HOpatch@OrgIfClassLoaded
      \let\@ifclasslater\HOpatch@OrgIfClassLater
    }%
  }%
}
%    \end{macrocode}
%
%    \begin{macrocode}
\HOpatch@AtEnd%
%    \end{macrocode}
%    \begin{macrocode}
%</package>
%    \end{macrocode}
%
% \section{Test}
%
%    \begin{macrocode}
%<*test1>
\def\LoadCommand{\RequirePackage{hopatch}[2012/05/28]}
%</test1>
%    \end{macrocode}
% \subsection{Catcode checks for loading}
%
%    \begin{macrocode}
%<*test1>
%    \end{macrocode}
%    \begin{macrocode}
\catcode`\{=1 %
\catcode`\}=2 %
\catcode`\#=6 %
\catcode`\@=11 %
\expandafter\ifx\csname count@\endcsname\relax
  \countdef\count@=255 %
\fi
\expandafter\ifx\csname @gobble\endcsname\relax
  \long\def\@gobble#1{}%
\fi
\expandafter\ifx\csname @firstofone\endcsname\relax
  \long\def\@firstofone#1{#1}%
\fi
\expandafter\ifx\csname loop\endcsname\relax
  \expandafter\@firstofone
\else
  \expandafter\@gobble
\fi
{%
  \def\loop#1\repeat{%
    \def\body{#1}%
    \iterate
  }%
  \def\iterate{%
    \body
      \let\next\iterate
    \else
      \let\next\relax
    \fi
    \next
  }%
  \let\repeat=\fi
}%
\def\RestoreCatcodes{}
\count@=0 %
\loop
  \edef\RestoreCatcodes{%
    \RestoreCatcodes
    \catcode\the\count@=\the\catcode\count@\relax
  }%
\ifnum\count@<255 %
  \advance\count@ 1 %
\repeat

\def\RangeCatcodeInvalid#1#2{%
  \count@=#1\relax
  \loop
    \catcode\count@=15 %
  \ifnum\count@<#2\relax
    \advance\count@ 1 %
  \repeat
}
\def\RangeCatcodeCheck#1#2#3{%
  \count@=#1\relax
  \loop
    \ifnum#3=\catcode\count@
    \else
      \errmessage{%
        Character \the\count@\space
        with wrong catcode \the\catcode\count@\space
        instead of \number#3%
      }%
    \fi
  \ifnum\count@<#2\relax
    \advance\count@ 1 %
  \repeat
}
\def\space{ }
\expandafter\ifx\csname LoadCommand\endcsname\relax
  \def\LoadCommand{\input hopatch.sty\relax}%
\fi
\def\Test{%
  \RangeCatcodeInvalid{0}{47}%
  \RangeCatcodeInvalid{58}{64}%
  \RangeCatcodeInvalid{91}{96}%
  \RangeCatcodeInvalid{123}{255}%
  \catcode`\@=12 %
  \catcode`\\=0 %
  \catcode`\%=14 %
  \LoadCommand
  \RangeCatcodeCheck{0}{36}{15}%
  \RangeCatcodeCheck{37}{37}{14}%
  \RangeCatcodeCheck{38}{47}{15}%
  \RangeCatcodeCheck{48}{57}{12}%
  \RangeCatcodeCheck{58}{63}{15}%
  \RangeCatcodeCheck{64}{64}{12}%
  \RangeCatcodeCheck{65}{90}{11}%
  \RangeCatcodeCheck{91}{91}{15}%
  \RangeCatcodeCheck{92}{92}{0}%
  \RangeCatcodeCheck{93}{96}{15}%
  \RangeCatcodeCheck{97}{122}{11}%
  \RangeCatcodeCheck{123}{255}{15}%
  \RestoreCatcodes
}
\Test
\csname @@end\endcsname
\end
%    \end{macrocode}
%    \begin{macrocode}
%</test1>
%    \end{macrocode}
%
%    \begin{macrocode}
%<*test2>
\NeedsTeXFormat{LaTeX2e}
\providecommand\variant{0}
\RequirePackage{filecontents}
\begin{filecontents}{foo.sty}
\ProvidesPackage{foo}
\def\msg#{\immediate\write16}
\def\foo#1{%
  \msg{\fooformat{#1}}%
}
\def\fooformat#1{[#1]}% hash-ok
\foo{* Executing foo at package loading}
\end{filecontents}

\ifnum\variant=1 %
  \documentclass{memoir}%
\else
  \documentclass{article}%
\fi

\ifcase\variant\relax
\or % 1
\or % 2
  \usepackage{etoolbox}%
\or % 3
  \usepackage{scrlfile}%
\or % 4
  \usepackage{filehook}%
\fi

\AtBeginDocument{\foo{* AtBeginDocument before hopatch}}
\usepackage{hopatch}
\AtBeginDocument{\foo{* AtBeginDocument after hopatch}}

\makeatletter
\hopatch@AfterPackage{foo}{%
  \def\fooformat#1{<<#1>>}%
}
\makeatother

\AtBeginDocument{\foo{* AtBeginDocument before foo}}
\usepackage{foo}
\AtBeginDocument{\foo{* AtBeginDocument after foo}}

\foo{* Executing in preamble}

\begin{document}
\foo{* Executing in document}
\end{document}
%</test2>
%    \end{macrocode}
%
% \section{Installation}
%
% \subsection{Download}
%
% \paragraph{Package.} This package is available on
% CTAN\footnote{\url{ftp://ftp.ctan.org/tex-archive/}}:
% \begin{description}
% \item[\CTAN{macros/latex/contrib/oberdiek/hopatch.dtx}] The source file.
% \item[\CTAN{macros/latex/contrib/oberdiek/hopatch.pdf}] Documentation.
% \end{description}
%
%
% \paragraph{Bundle.} All the packages of the bundle `oberdiek'
% are also available in a TDS compliant ZIP archive. There
% the packages are already unpacked and the documentation files
% are generated. The files and directories obey the TDS standard.
% \begin{description}
% \item[\CTAN{install/macros/latex/contrib/oberdiek.tds.zip}]
% \end{description}
% \emph{TDS} refers to the standard ``A Directory Structure
% for \TeX\ Files'' (\CTAN{tds/tds.pdf}). Directories
% with \xfile{texmf} in their name are usually organized this way.
%
% \subsection{Bundle installation}
%
% \paragraph{Unpacking.} Unpack the \xfile{oberdiek.tds.zip} in the
% TDS tree (also known as \xfile{texmf} tree) of your choice.
% Example (linux):
% \begin{quote}
%   |unzip oberdiek.tds.zip -d ~/texmf|
% \end{quote}
%
% \paragraph{Script installation.}
% Check the directory \xfile{TDS:scripts/oberdiek/} for
% scripts that need further installation steps.
% Package \xpackage{attachfile2} comes with the Perl script
% \xfile{pdfatfi.pl} that should be installed in such a way
% that it can be called as \texttt{pdfatfi}.
% Example (linux):
% \begin{quote}
%   |chmod +x scripts/oberdiek/pdfatfi.pl|\\
%   |cp scripts/oberdiek/pdfatfi.pl /usr/local/bin/|
% \end{quote}
%
% \subsection{Package installation}
%
% \paragraph{Unpacking.} The \xfile{.dtx} file is a self-extracting
% \docstrip\ archive. The files are extracted by running the
% \xfile{.dtx} through \plainTeX:
% \begin{quote}
%   \verb|tex hopatch.dtx|
% \end{quote}
%
% \paragraph{TDS.} Now the different files must be moved into
% the different directories in your installation TDS tree
% (also known as \xfile{texmf} tree):
% \begin{quote}
% \def\t{^^A
% \begin{tabular}{@{}>{\ttfamily}l@{ $\rightarrow$ }>{\ttfamily}l@{}}
%   hopatch.sty & tex/latex/oberdiek/hopatch.sty\\
%   hopatch.pdf & doc/latex/oberdiek/hopatch.pdf\\
%   test/hopatch-test1.tex & doc/latex/oberdiek/test/hopatch-test1.tex\\
%   test/hopatch-test2.tex & doc/latex/oberdiek/test/hopatch-test2.tex\\
%   hopatch.dtx & source/latex/oberdiek/hopatch.dtx\\
% \end{tabular}^^A
% }^^A
% \sbox0{\t}^^A
% \ifdim\wd0>\linewidth
%   \begingroup
%     \advance\linewidth by\leftmargin
%     \advance\linewidth by\rightmargin
%   \edef\x{\endgroup
%     \def\noexpand\lw{\the\linewidth}^^A
%   }\x
%   \def\lwbox{^^A
%     \leavevmode
%     \hbox to \linewidth{^^A
%       \kern-\leftmargin\relax
%       \hss
%       \usebox0
%       \hss
%       \kern-\rightmargin\relax
%     }^^A
%   }^^A
%   \ifdim\wd0>\lw
%     \sbox0{\small\t}^^A
%     \ifdim\wd0>\linewidth
%       \ifdim\wd0>\lw
%         \sbox0{\footnotesize\t}^^A
%         \ifdim\wd0>\linewidth
%           \ifdim\wd0>\lw
%             \sbox0{\scriptsize\t}^^A
%             \ifdim\wd0>\linewidth
%               \ifdim\wd0>\lw
%                 \sbox0{\tiny\t}^^A
%                 \ifdim\wd0>\linewidth
%                   \lwbox
%                 \else
%                   \usebox0
%                 \fi
%               \else
%                 \lwbox
%               \fi
%             \else
%               \usebox0
%             \fi
%           \else
%             \lwbox
%           \fi
%         \else
%           \usebox0
%         \fi
%       \else
%         \lwbox
%       \fi
%     \else
%       \usebox0
%     \fi
%   \else
%     \lwbox
%   \fi
% \else
%   \usebox0
% \fi
% \end{quote}
% If you have a \xfile{docstrip.cfg} that configures and enables \docstrip's
% TDS installing feature, then some files can already be in the right
% place, see the documentation of \docstrip.
%
% \subsection{Refresh file name databases}
%
% If your \TeX~distribution
% (\teTeX, \mikTeX, \dots) relies on file name databases, you must refresh
% these. For example, \teTeX\ users run \verb|texhash| or
% \verb|mktexlsr|.
%
% \subsection{Some details for the interested}
%
% \paragraph{Attached source.}
%
% The PDF documentation on CTAN also includes the
% \xfile{.dtx} source file. It can be extracted by
% AcrobatReader 6 or higher. Another option is \textsf{pdftk},
% e.g. unpack the file into the current directory:
% \begin{quote}
%   \verb|pdftk hopatch.pdf unpack_files output .|
% \end{quote}
%
% \paragraph{Unpacking with \LaTeX.}
% The \xfile{.dtx} chooses its action depending on the format:
% \begin{description}
% \item[\plainTeX:] Run \docstrip\ and extract the files.
% \item[\LaTeX:] Generate the documentation.
% \end{description}
% If you insist on using \LaTeX\ for \docstrip\ (really,
% \docstrip\ does not need \LaTeX), then inform the autodetect routine
% about your intention:
% \begin{quote}
%   \verb|latex \let\install=y\input{hopatch.dtx}|
% \end{quote}
% Do not forget to quote the argument according to the demands
% of your shell.
%
% \paragraph{Generating the documentation.}
% You can use both the \xfile{.dtx} or the \xfile{.drv} to generate
% the documentation. The process can be configured by the
% configuration file \xfile{ltxdoc.cfg}. For instance, put this
% line into this file, if you want to have A4 as paper format:
% \begin{quote}
%   \verb|\PassOptionsToClass{a4paper}{article}|
% \end{quote}
% An example follows how to generate the
% documentation with pdf\LaTeX:
% \begin{quote}
%\begin{verbatim}
%pdflatex hopatch.dtx
%makeindex -s gind.ist hopatch.idx
%pdflatex hopatch.dtx
%makeindex -s gind.ist hopatch.idx
%pdflatex hopatch.dtx
%\end{verbatim}
% \end{quote}
%
% \section{Catalogue}
%
% The following XML file can be used as source for the
% \href{http://mirror.ctan.org/help/Catalogue/catalogue.html}{\TeX\ Catalogue}.
% The elements \texttt{caption} and \texttt{description} are imported
% from the original XML file from the Catalogue.
% The name of the XML file in the Catalogue is \xfile{hopatch.xml}.
%    \begin{macrocode}
%<*catalogue>
<?xml version='1.0' encoding='us-ascii'?>
<!DOCTYPE entry SYSTEM 'catalogue.dtd'>
<entry datestamp='$Date$' modifier='$Author$' id='hopatch'>
  <name>hopatch</name>
  <caption>Load patches for packages.</caption>
  <authorref id='auth:oberdiek'/>
  <copyright owner='Heiko Oberdiek' year='2011,2012'/>
  <license type='lppl1.3'/>
  <version number='1.2'/>
  <description>
    The hopatch package provides a command with which the user may
    register a piece of patch code for a particular package.  Hopatch
    will apply the patch immediately, if the relevant package has
    already been loaded; otherwise it will store the patch until the
    package appears.
    <p/>
    The package is part of the <xref refid='oberdiek'>oberdiek</xref> bundle.
  </description>
  <documentation details='Package documentation'
      href='ctan:/macros/latex/contrib/oberdiek/hopatch.pdf'/>
  <ctan file='true' path='/macros/latex/contrib/oberdiek/hopatch.dtx'/>
  <miktex location='oberdiek'/>
  <texlive location='oberdiek'/>
  <install path='/macros/latex/contrib/oberdiek/oberdiek.tds.zip'/>
</entry>
%</catalogue>
%    \end{macrocode}
%
% \begin{thebibliography}{9}
% \raggedright
%
% \bibitem{etoolbox}
% Philipp Lehman:
% \textit{The etoolbox Package}
% 2011-01-03.
% \CTAN{macros/latex/contrib/etoolbox/etoolbox.pdf}
%
% \bibitem{filehook}
% Martin Scharrer:
% \textit{The filehook Package};
% 2011-01-09.
% \CTAN{macros/latex/contrib/filehook/filehook.pdf}
%
% \bibitem{ltxcmds}
% Heiko Oberdiek:
% \textit{The ltxcmds Package};
% 2010-12-12.
% \CTAN{macros/latex/contrib/oberdiek/ltxcmds.pdf}
%
% \bibitem{memoir}
% Peter Wilson, Lars Madsen:
% \textit{The Memoir Class for Configurable Typesetting, User Guide};
% 2010.
% \CTAN{macros/latex/contrib/memoir/memman.pdf}
%
% \bibitem{scrlfile}
% Markus Kohm, Jens-Uwe Morawski:
% \textit{The Guide KOMA-Script};
% 2011-01-20.
% \CTAN{macros/latex/contrib/koma-script/scrguien.pdf}
%
% \end{thebibliography}
%
% \begin{History}
%   \begin{Version}{2011/01/30 v1.0}
%   \item
%     First public version.
%   \end{Version}
%   \begin{Version}{2011/06/24 v1.1}
%   \item
%     Fix the use of \cs{AtEndPreamble} and \cs{AfterEndPreamble}.
%     They are redefined by package \xpackage{etoolbox} after their
%     hooks are used and generate an error message then.
%   \end{Version}
%   \begin{Version}{2012/05/28 v1.2}
%   \item
%     Fix for use without \hologo{eTeX}
%     (thanks Gordon Lee).
%   \end{Version}
% \end{History}
%
% \PrintIndex
%
% \Finale
\endinput

%        (quote the arguments according to the demands of your shell)
%
% Documentation:
%    (a) If hopatch.drv is present:
%           latex hopatch.drv
%    (b) Without hopatch.drv:
%           latex hopatch.dtx; ...
%    The class ltxdoc loads the configuration file ltxdoc.cfg
%    if available. Here you can specify further options, e.g.
%    use A4 as paper format:
%       \PassOptionsToClass{a4paper}{article}
%
%    Programm calls to get the documentation (example):
%       pdflatex hopatch.dtx
%       makeindex -s gind.ist hopatch.idx
%       pdflatex hopatch.dtx
%       makeindex -s gind.ist hopatch.idx
%       pdflatex hopatch.dtx
%
% Installation:
%    TDS:tex/latex/oberdiek/hopatch.sty
%    TDS:doc/latex/oberdiek/hopatch.pdf
%    TDS:doc/latex/oberdiek/test/hopatch-test1.tex
%    TDS:doc/latex/oberdiek/test/hopatch-test2.tex
%    TDS:source/latex/oberdiek/hopatch.dtx
%
%<*ignore>
\begingroup
  \catcode123=1 %
  \catcode125=2 %
  \def\x{LaTeX2e}%
\expandafter\endgroup
\ifcase 0\ifx\install y1\fi\expandafter
         \ifx\csname processbatchFile\endcsname\relax\else1\fi
         \ifx\fmtname\x\else 1\fi\relax
\else\csname fi\endcsname
%</ignore>
%<*install>
\input docstrip.tex
\Msg{************************************************************************}
\Msg{* Installation}
\Msg{* Package: hopatch 2012/05/28 v1.2 Wrapper for package hooks (HO)}
\Msg{************************************************************************}

\keepsilent
\askforoverwritefalse

\let\MetaPrefix\relax
\preamble

This is a generated file.

Project: hopatch
Version: 2012/05/28 v1.2

Copyright (C) 2011, 2012 by
   Heiko Oberdiek <heiko.oberdiek at googlemail.com>

This work may be distributed and/or modified under the
conditions of the LaTeX Project Public License, either
version 1.3c of this license or (at your option) any later
version. This version of this license is in
   http://www.latex-project.org/lppl/lppl-1-3c.txt
and the latest version of this license is in
   http://www.latex-project.org/lppl.txt
and version 1.3 or later is part of all distributions of
LaTeX version 2005/12/01 or later.

This work has the LPPL maintenance status "maintained".

This Current Maintainer of this work is Heiko Oberdiek.

This work consists of the main source file hopatch.dtx
and the derived files
   hopatch.sty, hopatch.pdf, hopatch.ins, hopatch.drv,
   hopatch-test1.tex, hopatch-test2.tex.

\endpreamble
\let\MetaPrefix\DoubleperCent

\generate{%
  \file{hopatch.ins}{\from{hopatch.dtx}{install}}%
  \file{hopatch.drv}{\from{hopatch.dtx}{driver}}%
  \usedir{tex/latex/oberdiek}%
  \file{hopatch.sty}{\from{hopatch.dtx}{package}}%
  \usedir{doc/latex/oberdiek/test}%
  \file{hopatch-test1.tex}{\from{hopatch.dtx}{test1}}%
  \file{hopatch-test2.tex}{\from{hopatch.dtx}{test2}}%
  \nopreamble
  \nopostamble
  \usedir{source/latex/oberdiek/catalogue}%
  \file{hopatch.xml}{\from{hopatch.dtx}{catalogue}}%
}

\catcode32=13\relax% active space
\let =\space%
\Msg{************************************************************************}
\Msg{*}
\Msg{* To finish the installation you have to move the following}
\Msg{* file into a directory searched by TeX:}
\Msg{*}
\Msg{*     hopatch.sty}
\Msg{*}
\Msg{* To produce the documentation run the file `hopatch.drv'}
\Msg{* through LaTeX.}
\Msg{*}
\Msg{* Happy TeXing!}
\Msg{*}
\Msg{************************************************************************}

\endbatchfile
%</install>
%<*ignore>
\fi
%</ignore>
%<*driver>
\NeedsTeXFormat{LaTeX2e}
\ProvidesFile{hopatch.drv}%
  [2012/05/28 v1.2 Wrapper for package hooks (HO)]%
\documentclass{ltxdoc}
\usepackage{holtxdoc}[2011/11/22]
\usepackage{caption}
\usepackage{array}
\setlength{\extrarowheight}{1pt}
\renewcommand*{\arraystretch}{1.1}
\begin{document}
  \DocInput{hopatch.dtx}%
\end{document}
%</driver>
% \fi
%
% \CheckSum{466}
%
% \CharacterTable
%  {Upper-case    \A\B\C\D\E\F\G\H\I\J\K\L\M\N\O\P\Q\R\S\T\U\V\W\X\Y\Z
%   Lower-case    \a\b\c\d\e\f\g\h\i\j\k\l\m\n\o\p\q\r\s\t\u\v\w\x\y\z
%   Digits        \0\1\2\3\4\5\6\7\8\9
%   Exclamation   \!     Double quote  \"     Hash (number) \#
%   Dollar        \$     Percent       \%     Ampersand     \&
%   Acute accent  \'     Left paren    \(     Right paren   \)
%   Asterisk      \*     Plus          \+     Comma         \,
%   Minus         \-     Point         \.     Solidus       \/
%   Colon         \:     Semicolon     \;     Less than     \<
%   Equals        \=     Greater than  \>     Question mark \?
%   Commercial at \@     Left bracket  \[     Backslash     \\
%   Right bracket \]     Circumflex    \^     Underscore    \_
%   Grave accent  \`     Left brace    \{     Vertical bar  \|
%   Right brace   \}     Tilde         \~}
%
% \GetFileInfo{hopatch.drv}
%
% \title{The \xpackage{hopatch} package}
% \date{2012/05/28 v1.2}
% \author{Heiko Oberdiek\\\xemail{heiko.oberdiek at googlemail.com}}
%
% \maketitle
%
% \begin{abstract}
% This packages provides a wrapper to various package hooks
% provided by other packages or classes, but does not define
% own hooks.
% \end{abstract}
%
% \tableofcontents
%
% \section{Documentation}
%
% Sometimes I want to add code right after a package
% has been loaded. Examples are bug fixes, adaptations,
% or added features as needed by package \xpackage{hyperref},
% for instance.
%
% Unhappily \hologo{LaTeX} does not provide this kind of
% hook. \cs{AtEndOfPackage} can be used inside the package
% only, because \hologo{LaTeX} clears the hook right before
% it loads the package.
%
% However, there are already many packages and classes
% that provide hooks that are executed after the package
% is loaded, see table~\ref{tab:hooks}.
%
% \begin{table}
% \caption{After package hooking}
% \label{tab:hooks}
% \begin{tabular}{ll}
% \bfseries Macro & \bfseries Provider\\
% \hline
% \cs{AfterPackage} & package \xpackage{scrlfile} \cite{scrlfile}\\
% \cs{AtEndOfPackageFile} & package \xpackage{filehook} \cite{filehook}\\
% \cs{AtEndPackage} & class \xclass{memoir} \cite{memoir}
% \end{tabular}
% \end{table}
%
% Package \xpackage{hopatch} can be used without the packages
% of table~\ref{tab:hooks}. But for an early executing right
% after a package is loaded, one of the following class or
% packages should be loaded before using \cs{hopatch@AfterPackage}:
% \begin{itemize}
%  \item package \xpackage{filehook}
%  \item package \xpackage{scrlfile}
%  \item class \xclass{memoir}
% \end{itemize}
%
% \begin{table}
% \caption{After begin document hooking}
% \begin{tabular}{ll}
% \bfseries Macro & \bfseries Provider\\
% \hline
% \cs{AtBeginDocument} & \hologo{LaTeX}'s kernel\\
% \cs{AtEndPreamble} & package \xpackage{etoolbox} \cite{etoolbox}\\
% \cs{AfterEndPreamble} & package \xpackage{etoolbox}
% \end{tabular}
% \end{table}
%
% Therefore I skip writing a new package for hooking into
% \hologo{LaTeX}'s package management and use this package
% to provide a wrapper to patch a package after it is loaded.
%
% \begin{declcs}{hopatch@AfterPackage} \M{package} \M{patch code}
% \end{declcs}
% If the package is already loaded, the \meta{patch code} is executed
% immediately.
% Otherwise the \meta{patch code} is stored in a command and tried
% at later locations until the package is available.
%
% The patch is tried in the following order:
% \begin{enumerate}
% \item If the package is already loaded, the patch is
%   applied immediately. Further locations are not tried.
% \item \cs{AtEndPackage}, provided by class \xclass{memoir} \cite{memoir},
%   and \cs{AfterPackage}, provided by package \xpackage{scrlfile} \cite{scrlfile},
%   are called right after the package file is input before
%   the hook of \hologo{LaTeX}'s \cs{AtEndOfPackage}.
% \item \cs{AtEndOfPackageFile}, provided by package \xpackage{filehook} \cite{filehook},
%   is called after the package is loaded and after the hook of
%   \hologo{LaTeX}'s \cs{AtEndOfPackage}.
% \item \cs{AtEndPreamble}, provided by package \xpackage{etoolbox} \cite{etoolbox},
%   is called at the beginning of |\begin{document}| before the hook
%   of \hologo{LaTeX}'s \cs{AtBeginDocument}.
% \item \cs{AtBeginDocument}, provided by \hologo{LaTeX}.
% \item \cs{AfterEndDocument}, provided by package \xpackage{etoolbox} \cite{etoolbox},
%   is called at the very end of |\begin{document}|. Preamble commands are
%   already forbidden there.
% \end{enumerate}
% Because of the various locations the patch code is restricted to
% limitations:
% \begin{itemize}
% \item
% Preamble commands, see \hologo{LaTeX}'s \cs{@onlypreamble}
% throw an error if used after |\begin{document}|. This is already
% the case for \cs{AfterEndDocument}. Therefore preamble commands
% are forbidden in the patching code.
% There are four exceptions \cs{@ifpackageloaded}, \cs{@ifclassloaded},
% \cs{@ifpackagelater} and \cs{@ifclasslater}. They are redefined
% during \cs{AfterEndDocument} using the counterparts of
% package \xpackage{ltxcmds} \cite{ltxcmds}.
% \item
% \cs{AfterPackage} of package \xpackage{scrlfile} and \cs{AtEndPackage}
% of class \xclass{memoir} call the hook before \hologo{LaTeX}'s
% \cs{AtEndOfPackage}.
% \end{itemize}
%
% \StopEventually{
% }
%
% \section{Implementation}
%
%    \begin{macrocode}
%<*package>
%    \end{macrocode}
%
% \subsection{Catcodes and package identification}
%
%    \begin{macrocode}
\begingroup\catcode61\catcode48\catcode32=10\relax%
  \catcode13=5 % ^^M
  \endlinechar=13 %
  \catcode123=1 % {
  \catcode125=2 % }
  \catcode64=11 % @
  \def\x{\endgroup
    \expandafter\edef\csname HOpatch@AtEnd\endcsname{%
      \endlinechar=\the\endlinechar\relax
      \catcode13=\the\catcode13\relax
      \catcode32=\the\catcode32\relax
      \catcode35=\the\catcode35\relax
      \catcode61=\the\catcode61\relax
      \catcode64=\the\catcode64\relax
      \catcode123=\the\catcode123\relax
      \catcode125=\the\catcode125\relax
    }%
  }%
\x\catcode61\catcode48\catcode32=10\relax%
\catcode13=5 % ^^M
\endlinechar=13 %
\catcode35=6 % #
\catcode64=11 % @
\catcode123=1 % {
\catcode125=2 % }
\def\TMP@EnsureCode#1#2{%
  \edef\HOpatch@AtEnd{%
    \HOpatch@AtEnd
    \catcode#1=\the\catcode#1\relax
  }%
  \catcode#1=#2\relax
}
\TMP@EnsureCode{40}{12}% (
\TMP@EnsureCode{41}{12}% )
\TMP@EnsureCode{43}{12}% +
\TMP@EnsureCode{46}{12}% .
\TMP@EnsureCode{47}{12}% /
\TMP@EnsureCode{91}{12}% [
\TMP@EnsureCode{93}{12}% ]
\edef\HOpatch@AtEnd{\HOpatch@AtEnd\noexpand\endinput}
%    \end{macrocode}
%
%    Package identification.
%    \begin{macrocode}
\NeedsTeXFormat{LaTeX2e}
\ProvidesPackage{hopatch}%
  [2012/05/28 v1.2 Wrapper for package hooks (HO)]
%    \end{macrocode}
%
% \subsection{Resources}
%
%    \begin{macrocode}
\begingroup\expandafter\expandafter\expandafter\endgroup
\expandafter\ifx\csname RequirePackage\endcsname\relax
  \def\TMP@RequirePackage#1[#2]{%
    \begingroup\expandafter\expandafter\expandafter\endgroup
    \expandafter\ifx\csname ver@#1.sty\endcsname\relax
      \input #1.sty\relax
    \fi
  }%
  \TMP@RequirePackage{ltxcmds}[2010/12/12]%
\else
  \RequirePackage{ltxcmds}[2010/12/12]%
\fi
%    \end{macrocode}
%
%    \begin{macro}{\HOpatch@counter}
%    \begin{macrocode}
\def\HOpatch@counter{0}%
%    \end{macrocode}
%    \end{macro}
%    \begin{macro}{\HOpatch@StepCounter}
%    \begin{macrocode}
\ltx@ifundefined{numexpr}{%
  \def\HOpatch@StepCounter{%
    \begingroup
      \count@\HOpatch@counter\relax
      \advance\count@\ltx@one\relax
    \edef\x{\endgroup
      \noexpand\def\noexpand\HOpatch@counter{\the\count@}%
    }%
    \x
  }%
}{%
  \def\HOpatch@StepCounter{%
    \edef\HOpatch@counter{%
      \the\numexpr\HOpatch@counter+\ltx@one\relax
    }%
  }%
}
%    \end{macrocode}
%    \end{macro}
%
%    \begin{macro}{\HOpatch@list}
%    \begin{macrocode}
\def\HOpatch@list{}
%    \end{macrocode}
%    \end{macro}
%
%    \begin{macro}{\HOpatch@Add}
%    \begin{macrocode}
\def\HOpatch@Add{%
  \ltx@LocalAppendToMacro\HOpatch@list
}
%    \end{macrocode}
%    \end{macro}
%
% \subsection{Package patching}
%
%    \begin{macro}{\hopatch@AfterPackage}
%    \begin{macrocode}
\def\hopatch@AfterPackage#1{%
  \ltx@ifpackageloaded{#1}{%
    \ltx@firstofone
  }{%
    \HOpatch@AfterPackage{#1}%
  }%
}
%    \end{macrocode}
%    \end{macro}
%    \begin{macro}{\HOpatch@AfterPackage}
%    \begin{macrocode}
\def\HOpatch@AfterPackage#1{%
  \edef\HOpatch@temp{#1}%
  \HOpatch@StepCounter
  \expandafter\HOpatch@@AfterPackage
  \csname HOpatch@\HOpatch@counter\expandafter\endcsname{%
    \HOpatch@temp
  }%
}
%    \end{macrocode}
%    \end{macro}
%
%    \begin{macro}{\HOpatch@@AfterPackage}
%    \begin{macrocode}
\def\HOpatch@@AfterPackage#1#2#3{%
  \begingroup
    \toks@{#3}%
    \xdef\HOpatch@gtemp{%
      \noexpand\ltx@ifpackageloaded{#2}{%
        \noexpand\let\noexpand#1\noexpand\relax
        \the\toks@
      }{}%
    }%
  \endgroup
  \let#1\HOpatch@gtemp
  \HOpatch@Add#1%
  \HOpatch@Try{AfterPackage}{#2}#1%
  \HOpatch@Try{AtEndPackage}{#2}#1%
  \HOpatch@Try{AtEndOfPackageFile}{#2}#1%
}
%    \end{macrocode}
%    \end{macro}
%
%    \begin{macro}{\HOpatch@Try}
%    \begin{macrocode}
\def\HOpatch@Try#1#2#3{%
  \ltx@ifundefined{#1}{}{%
    \csname #1\endcsname{#2}{#3}%
  }%
}
%    \end{macrocode}
%    \end{macro}
%
%    \begin{macrocode}
\AtBeginDocument{\HOpatch@list}
\ltx@ifundefined{AtEndPreamble}{}{%
  \ltx@ifundefined{@endpreamblehook}{}{%
    \AtEndPreamble{\HOpatch@list}%
  }%
}
\ltx@ifundefined{AfterEndPreamble}{}{%
  \ltx@ifundefined{@afterendpreamblehook}{}{%
    \AfterEndPreamble{%
      \let\HOpatch@OrgIfPackageLoaded\@ifpackageloaded
      \let\HOpatch@OrgIfPackageLater\@ifpackagelater
      \let\HOpatch@OrgIfClassLoaded\@ifclassloaded
      \let\HOpatch@OrgIfClassLater\@ifclasslater
      \let\@ifpackageloaded\ltx@ifpackageloaded
      \let\@ifpackagelater\ltx@ifpackagelater
      \let\@ifclassloaded\ltx@ifclassloaded
      \let\@ifclasslater\ltx@ifclasslater
      \HOpatch@list
      \let\@ifpackageloaded\HOpatch@OrgIfPackageLoaded
      \let\@ifpackagelater\HOpatch@OrgIfPackageLater
      \let\@ifclassloaded\HOpatch@OrgIfClassLoaded
      \let\@ifclasslater\HOpatch@OrgIfClassLater
    }%
  }%
}
%    \end{macrocode}
%
%    \begin{macrocode}
\HOpatch@AtEnd%
%    \end{macrocode}
%    \begin{macrocode}
%</package>
%    \end{macrocode}
%
% \section{Test}
%
%    \begin{macrocode}
%<*test1>
\def\LoadCommand{\RequirePackage{hopatch}[2012/05/28]}
%</test1>
%    \end{macrocode}
% \subsection{Catcode checks for loading}
%
%    \begin{macrocode}
%<*test1>
%    \end{macrocode}
%    \begin{macrocode}
\catcode`\{=1 %
\catcode`\}=2 %
\catcode`\#=6 %
\catcode`\@=11 %
\expandafter\ifx\csname count@\endcsname\relax
  \countdef\count@=255 %
\fi
\expandafter\ifx\csname @gobble\endcsname\relax
  \long\def\@gobble#1{}%
\fi
\expandafter\ifx\csname @firstofone\endcsname\relax
  \long\def\@firstofone#1{#1}%
\fi
\expandafter\ifx\csname loop\endcsname\relax
  \expandafter\@firstofone
\else
  \expandafter\@gobble
\fi
{%
  \def\loop#1\repeat{%
    \def\body{#1}%
    \iterate
  }%
  \def\iterate{%
    \body
      \let\next\iterate
    \else
      \let\next\relax
    \fi
    \next
  }%
  \let\repeat=\fi
}%
\def\RestoreCatcodes{}
\count@=0 %
\loop
  \edef\RestoreCatcodes{%
    \RestoreCatcodes
    \catcode\the\count@=\the\catcode\count@\relax
  }%
\ifnum\count@<255 %
  \advance\count@ 1 %
\repeat

\def\RangeCatcodeInvalid#1#2{%
  \count@=#1\relax
  \loop
    \catcode\count@=15 %
  \ifnum\count@<#2\relax
    \advance\count@ 1 %
  \repeat
}
\def\RangeCatcodeCheck#1#2#3{%
  \count@=#1\relax
  \loop
    \ifnum#3=\catcode\count@
    \else
      \errmessage{%
        Character \the\count@\space
        with wrong catcode \the\catcode\count@\space
        instead of \number#3%
      }%
    \fi
  \ifnum\count@<#2\relax
    \advance\count@ 1 %
  \repeat
}
\def\space{ }
\expandafter\ifx\csname LoadCommand\endcsname\relax
  \def\LoadCommand{\input hopatch.sty\relax}%
\fi
\def\Test{%
  \RangeCatcodeInvalid{0}{47}%
  \RangeCatcodeInvalid{58}{64}%
  \RangeCatcodeInvalid{91}{96}%
  \RangeCatcodeInvalid{123}{255}%
  \catcode`\@=12 %
  \catcode`\\=0 %
  \catcode`\%=14 %
  \LoadCommand
  \RangeCatcodeCheck{0}{36}{15}%
  \RangeCatcodeCheck{37}{37}{14}%
  \RangeCatcodeCheck{38}{47}{15}%
  \RangeCatcodeCheck{48}{57}{12}%
  \RangeCatcodeCheck{58}{63}{15}%
  \RangeCatcodeCheck{64}{64}{12}%
  \RangeCatcodeCheck{65}{90}{11}%
  \RangeCatcodeCheck{91}{91}{15}%
  \RangeCatcodeCheck{92}{92}{0}%
  \RangeCatcodeCheck{93}{96}{15}%
  \RangeCatcodeCheck{97}{122}{11}%
  \RangeCatcodeCheck{123}{255}{15}%
  \RestoreCatcodes
}
\Test
\csname @@end\endcsname
\end
%    \end{macrocode}
%    \begin{macrocode}
%</test1>
%    \end{macrocode}
%
%    \begin{macrocode}
%<*test2>
\NeedsTeXFormat{LaTeX2e}
\providecommand\variant{0}
\RequirePackage{filecontents}
\begin{filecontents}{foo.sty}
\ProvidesPackage{foo}
\def\msg#{\immediate\write16}
\def\foo#1{%
  \msg{\fooformat{#1}}%
}
\def\fooformat#1{[#1]}% hash-ok
\foo{* Executing foo at package loading}
\end{filecontents}

\ifnum\variant=1 %
  \documentclass{memoir}%
\else
  \documentclass{article}%
\fi

\ifcase\variant\relax
\or % 1
\or % 2
  \usepackage{etoolbox}%
\or % 3
  \usepackage{scrlfile}%
\or % 4
  \usepackage{filehook}%
\fi

\AtBeginDocument{\foo{* AtBeginDocument before hopatch}}
\usepackage{hopatch}
\AtBeginDocument{\foo{* AtBeginDocument after hopatch}}

\makeatletter
\hopatch@AfterPackage{foo}{%
  \def\fooformat#1{<<#1>>}%
}
\makeatother

\AtBeginDocument{\foo{* AtBeginDocument before foo}}
\usepackage{foo}
\AtBeginDocument{\foo{* AtBeginDocument after foo}}

\foo{* Executing in preamble}

\begin{document}
\foo{* Executing in document}
\end{document}
%</test2>
%    \end{macrocode}
%
% \section{Installation}
%
% \subsection{Download}
%
% \paragraph{Package.} This package is available on
% CTAN\footnote{\url{ftp://ftp.ctan.org/tex-archive/}}:
% \begin{description}
% \item[\CTAN{macros/latex/contrib/oberdiek/hopatch.dtx}] The source file.
% \item[\CTAN{macros/latex/contrib/oberdiek/hopatch.pdf}] Documentation.
% \end{description}
%
%
% \paragraph{Bundle.} All the packages of the bundle `oberdiek'
% are also available in a TDS compliant ZIP archive. There
% the packages are already unpacked and the documentation files
% are generated. The files and directories obey the TDS standard.
% \begin{description}
% \item[\CTAN{install/macros/latex/contrib/oberdiek.tds.zip}]
% \end{description}
% \emph{TDS} refers to the standard ``A Directory Structure
% for \TeX\ Files'' (\CTAN{tds/tds.pdf}). Directories
% with \xfile{texmf} in their name are usually organized this way.
%
% \subsection{Bundle installation}
%
% \paragraph{Unpacking.} Unpack the \xfile{oberdiek.tds.zip} in the
% TDS tree (also known as \xfile{texmf} tree) of your choice.
% Example (linux):
% \begin{quote}
%   |unzip oberdiek.tds.zip -d ~/texmf|
% \end{quote}
%
% \paragraph{Script installation.}
% Check the directory \xfile{TDS:scripts/oberdiek/} for
% scripts that need further installation steps.
% Package \xpackage{attachfile2} comes with the Perl script
% \xfile{pdfatfi.pl} that should be installed in such a way
% that it can be called as \texttt{pdfatfi}.
% Example (linux):
% \begin{quote}
%   |chmod +x scripts/oberdiek/pdfatfi.pl|\\
%   |cp scripts/oberdiek/pdfatfi.pl /usr/local/bin/|
% \end{quote}
%
% \subsection{Package installation}
%
% \paragraph{Unpacking.} The \xfile{.dtx} file is a self-extracting
% \docstrip\ archive. The files are extracted by running the
% \xfile{.dtx} through \plainTeX:
% \begin{quote}
%   \verb|tex hopatch.dtx|
% \end{quote}
%
% \paragraph{TDS.} Now the different files must be moved into
% the different directories in your installation TDS tree
% (also known as \xfile{texmf} tree):
% \begin{quote}
% \def\t{^^A
% \begin{tabular}{@{}>{\ttfamily}l@{ $\rightarrow$ }>{\ttfamily}l@{}}
%   hopatch.sty & tex/latex/oberdiek/hopatch.sty\\
%   hopatch.pdf & doc/latex/oberdiek/hopatch.pdf\\
%   test/hopatch-test1.tex & doc/latex/oberdiek/test/hopatch-test1.tex\\
%   test/hopatch-test2.tex & doc/latex/oberdiek/test/hopatch-test2.tex\\
%   hopatch.dtx & source/latex/oberdiek/hopatch.dtx\\
% \end{tabular}^^A
% }^^A
% \sbox0{\t}^^A
% \ifdim\wd0>\linewidth
%   \begingroup
%     \advance\linewidth by\leftmargin
%     \advance\linewidth by\rightmargin
%   \edef\x{\endgroup
%     \def\noexpand\lw{\the\linewidth}^^A
%   }\x
%   \def\lwbox{^^A
%     \leavevmode
%     \hbox to \linewidth{^^A
%       \kern-\leftmargin\relax
%       \hss
%       \usebox0
%       \hss
%       \kern-\rightmargin\relax
%     }^^A
%   }^^A
%   \ifdim\wd0>\lw
%     \sbox0{\small\t}^^A
%     \ifdim\wd0>\linewidth
%       \ifdim\wd0>\lw
%         \sbox0{\footnotesize\t}^^A
%         \ifdim\wd0>\linewidth
%           \ifdim\wd0>\lw
%             \sbox0{\scriptsize\t}^^A
%             \ifdim\wd0>\linewidth
%               \ifdim\wd0>\lw
%                 \sbox0{\tiny\t}^^A
%                 \ifdim\wd0>\linewidth
%                   \lwbox
%                 \else
%                   \usebox0
%                 \fi
%               \else
%                 \lwbox
%               \fi
%             \else
%               \usebox0
%             \fi
%           \else
%             \lwbox
%           \fi
%         \else
%           \usebox0
%         \fi
%       \else
%         \lwbox
%       \fi
%     \else
%       \usebox0
%     \fi
%   \else
%     \lwbox
%   \fi
% \else
%   \usebox0
% \fi
% \end{quote}
% If you have a \xfile{docstrip.cfg} that configures and enables \docstrip's
% TDS installing feature, then some files can already be in the right
% place, see the documentation of \docstrip.
%
% \subsection{Refresh file name databases}
%
% If your \TeX~distribution
% (\teTeX, \mikTeX, \dots) relies on file name databases, you must refresh
% these. For example, \teTeX\ users run \verb|texhash| or
% \verb|mktexlsr|.
%
% \subsection{Some details for the interested}
%
% \paragraph{Attached source.}
%
% The PDF documentation on CTAN also includes the
% \xfile{.dtx} source file. It can be extracted by
% AcrobatReader 6 or higher. Another option is \textsf{pdftk},
% e.g. unpack the file into the current directory:
% \begin{quote}
%   \verb|pdftk hopatch.pdf unpack_files output .|
% \end{quote}
%
% \paragraph{Unpacking with \LaTeX.}
% The \xfile{.dtx} chooses its action depending on the format:
% \begin{description}
% \item[\plainTeX:] Run \docstrip\ and extract the files.
% \item[\LaTeX:] Generate the documentation.
% \end{description}
% If you insist on using \LaTeX\ for \docstrip\ (really,
% \docstrip\ does not need \LaTeX), then inform the autodetect routine
% about your intention:
% \begin{quote}
%   \verb|latex \let\install=y% \iffalse meta-comment
%
% File: hopatch.dtx
% Version: 2012/05/28 v1.2
% Info: Wrapper for package hooks
%
% Copyright (C) 2011, 2012 by
%    Heiko Oberdiek <heiko.oberdiek at googlemail.com>
%
% This work may be distributed and/or modified under the
% conditions of the LaTeX Project Public License, either
% version 1.3c of this license or (at your option) any later
% version. This version of this license is in
%    http://www.latex-project.org/lppl/lppl-1-3c.txt
% and the latest version of this license is in
%    http://www.latex-project.org/lppl.txt
% and version 1.3 or later is part of all distributions of
% LaTeX version 2005/12/01 or later.
%
% This work has the LPPL maintenance status "maintained".
%
% This Current Maintainer of this work is Heiko Oberdiek.
%
% This work consists of the main source file hopatch.dtx
% and the derived files
%    hopatch.sty, hopatch.pdf, hopatch.ins, hopatch.drv,
%    hopatch-test1.tex, hopatch-test2.tex.
%
% Distribution:
%    CTAN:macros/latex/contrib/oberdiek/hopatch.dtx
%    CTAN:macros/latex/contrib/oberdiek/hopatch.pdf
%
% Unpacking:
%    (a) If hopatch.ins is present:
%           tex hopatch.ins
%    (b) Without hopatch.ins:
%           tex hopatch.dtx
%    (c) If you insist on using LaTeX
%           latex \let\install=y\input{hopatch.dtx}
%        (quote the arguments according to the demands of your shell)
%
% Documentation:
%    (a) If hopatch.drv is present:
%           latex hopatch.drv
%    (b) Without hopatch.drv:
%           latex hopatch.dtx; ...
%    The class ltxdoc loads the configuration file ltxdoc.cfg
%    if available. Here you can specify further options, e.g.
%    use A4 as paper format:
%       \PassOptionsToClass{a4paper}{article}
%
%    Programm calls to get the documentation (example):
%       pdflatex hopatch.dtx
%       makeindex -s gind.ist hopatch.idx
%       pdflatex hopatch.dtx
%       makeindex -s gind.ist hopatch.idx
%       pdflatex hopatch.dtx
%
% Installation:
%    TDS:tex/latex/oberdiek/hopatch.sty
%    TDS:doc/latex/oberdiek/hopatch.pdf
%    TDS:doc/latex/oberdiek/test/hopatch-test1.tex
%    TDS:doc/latex/oberdiek/test/hopatch-test2.tex
%    TDS:source/latex/oberdiek/hopatch.dtx
%
%<*ignore>
\begingroup
  \catcode123=1 %
  \catcode125=2 %
  \def\x{LaTeX2e}%
\expandafter\endgroup
\ifcase 0\ifx\install y1\fi\expandafter
         \ifx\csname processbatchFile\endcsname\relax\else1\fi
         \ifx\fmtname\x\else 1\fi\relax
\else\csname fi\endcsname
%</ignore>
%<*install>
\input docstrip.tex
\Msg{************************************************************************}
\Msg{* Installation}
\Msg{* Package: hopatch 2012/05/28 v1.2 Wrapper for package hooks (HO)}
\Msg{************************************************************************}

\keepsilent
\askforoverwritefalse

\let\MetaPrefix\relax
\preamble

This is a generated file.

Project: hopatch
Version: 2012/05/28 v1.2

Copyright (C) 2011, 2012 by
   Heiko Oberdiek <heiko.oberdiek at googlemail.com>

This work may be distributed and/or modified under the
conditions of the LaTeX Project Public License, either
version 1.3c of this license or (at your option) any later
version. This version of this license is in
   http://www.latex-project.org/lppl/lppl-1-3c.txt
and the latest version of this license is in
   http://www.latex-project.org/lppl.txt
and version 1.3 or later is part of all distributions of
LaTeX version 2005/12/01 or later.

This work has the LPPL maintenance status "maintained".

This Current Maintainer of this work is Heiko Oberdiek.

This work consists of the main source file hopatch.dtx
and the derived files
   hopatch.sty, hopatch.pdf, hopatch.ins, hopatch.drv,
   hopatch-test1.tex, hopatch-test2.tex.

\endpreamble
\let\MetaPrefix\DoubleperCent

\generate{%
  \file{hopatch.ins}{\from{hopatch.dtx}{install}}%
  \file{hopatch.drv}{\from{hopatch.dtx}{driver}}%
  \usedir{tex/latex/oberdiek}%
  \file{hopatch.sty}{\from{hopatch.dtx}{package}}%
  \usedir{doc/latex/oberdiek/test}%
  \file{hopatch-test1.tex}{\from{hopatch.dtx}{test1}}%
  \file{hopatch-test2.tex}{\from{hopatch.dtx}{test2}}%
  \nopreamble
  \nopostamble
  \usedir{source/latex/oberdiek/catalogue}%
  \file{hopatch.xml}{\from{hopatch.dtx}{catalogue}}%
}

\catcode32=13\relax% active space
\let =\space%
\Msg{************************************************************************}
\Msg{*}
\Msg{* To finish the installation you have to move the following}
\Msg{* file into a directory searched by TeX:}
\Msg{*}
\Msg{*     hopatch.sty}
\Msg{*}
\Msg{* To produce the documentation run the file `hopatch.drv'}
\Msg{* through LaTeX.}
\Msg{*}
\Msg{* Happy TeXing!}
\Msg{*}
\Msg{************************************************************************}

\endbatchfile
%</install>
%<*ignore>
\fi
%</ignore>
%<*driver>
\NeedsTeXFormat{LaTeX2e}
\ProvidesFile{hopatch.drv}%
  [2012/05/28 v1.2 Wrapper for package hooks (HO)]%
\documentclass{ltxdoc}
\usepackage{holtxdoc}[2011/11/22]
\usepackage{caption}
\usepackage{array}
\setlength{\extrarowheight}{1pt}
\renewcommand*{\arraystretch}{1.1}
\begin{document}
  \DocInput{hopatch.dtx}%
\end{document}
%</driver>
% \fi
%
% \CheckSum{466}
%
% \CharacterTable
%  {Upper-case    \A\B\C\D\E\F\G\H\I\J\K\L\M\N\O\P\Q\R\S\T\U\V\W\X\Y\Z
%   Lower-case    \a\b\c\d\e\f\g\h\i\j\k\l\m\n\o\p\q\r\s\t\u\v\w\x\y\z
%   Digits        \0\1\2\3\4\5\6\7\8\9
%   Exclamation   \!     Double quote  \"     Hash (number) \#
%   Dollar        \$     Percent       \%     Ampersand     \&
%   Acute accent  \'     Left paren    \(     Right paren   \)
%   Asterisk      \*     Plus          \+     Comma         \,
%   Minus         \-     Point         \.     Solidus       \/
%   Colon         \:     Semicolon     \;     Less than     \<
%   Equals        \=     Greater than  \>     Question mark \?
%   Commercial at \@     Left bracket  \[     Backslash     \\
%   Right bracket \]     Circumflex    \^     Underscore    \_
%   Grave accent  \`     Left brace    \{     Vertical bar  \|
%   Right brace   \}     Tilde         \~}
%
% \GetFileInfo{hopatch.drv}
%
% \title{The \xpackage{hopatch} package}
% \date{2012/05/28 v1.2}
% \author{Heiko Oberdiek\\\xemail{heiko.oberdiek at googlemail.com}}
%
% \maketitle
%
% \begin{abstract}
% This packages provides a wrapper to various package hooks
% provided by other packages or classes, but does not define
% own hooks.
% \end{abstract}
%
% \tableofcontents
%
% \section{Documentation}
%
% Sometimes I want to add code right after a package
% has been loaded. Examples are bug fixes, adaptations,
% or added features as needed by package \xpackage{hyperref},
% for instance.
%
% Unhappily \hologo{LaTeX} does not provide this kind of
% hook. \cs{AtEndOfPackage} can be used inside the package
% only, because \hologo{LaTeX} clears the hook right before
% it loads the package.
%
% However, there are already many packages and classes
% that provide hooks that are executed after the package
% is loaded, see table~\ref{tab:hooks}.
%
% \begin{table}
% \caption{After package hooking}
% \label{tab:hooks}
% \begin{tabular}{ll}
% \bfseries Macro & \bfseries Provider\\
% \hline
% \cs{AfterPackage} & package \xpackage{scrlfile} \cite{scrlfile}\\
% \cs{AtEndOfPackageFile} & package \xpackage{filehook} \cite{filehook}\\
% \cs{AtEndPackage} & class \xclass{memoir} \cite{memoir}
% \end{tabular}
% \end{table}
%
% Package \xpackage{hopatch} can be used without the packages
% of table~\ref{tab:hooks}. But for an early executing right
% after a package is loaded, one of the following class or
% packages should be loaded before using \cs{hopatch@AfterPackage}:
% \begin{itemize}
%  \item package \xpackage{filehook}
%  \item package \xpackage{scrlfile}
%  \item class \xclass{memoir}
% \end{itemize}
%
% \begin{table}
% \caption{After begin document hooking}
% \begin{tabular}{ll}
% \bfseries Macro & \bfseries Provider\\
% \hline
% \cs{AtBeginDocument} & \hologo{LaTeX}'s kernel\\
% \cs{AtEndPreamble} & package \xpackage{etoolbox} \cite{etoolbox}\\
% \cs{AfterEndPreamble} & package \xpackage{etoolbox}
% \end{tabular}
% \end{table}
%
% Therefore I skip writing a new package for hooking into
% \hologo{LaTeX}'s package management and use this package
% to provide a wrapper to patch a package after it is loaded.
%
% \begin{declcs}{hopatch@AfterPackage} \M{package} \M{patch code}
% \end{declcs}
% If the package is already loaded, the \meta{patch code} is executed
% immediately.
% Otherwise the \meta{patch code} is stored in a command and tried
% at later locations until the package is available.
%
% The patch is tried in the following order:
% \begin{enumerate}
% \item If the package is already loaded, the patch is
%   applied immediately. Further locations are not tried.
% \item \cs{AtEndPackage}, provided by class \xclass{memoir} \cite{memoir},
%   and \cs{AfterPackage}, provided by package \xpackage{scrlfile} \cite{scrlfile},
%   are called right after the package file is input before
%   the hook of \hologo{LaTeX}'s \cs{AtEndOfPackage}.
% \item \cs{AtEndOfPackageFile}, provided by package \xpackage{filehook} \cite{filehook},
%   is called after the package is loaded and after the hook of
%   \hologo{LaTeX}'s \cs{AtEndOfPackage}.
% \item \cs{AtEndPreamble}, provided by package \xpackage{etoolbox} \cite{etoolbox},
%   is called at the beginning of |\begin{document}| before the hook
%   of \hologo{LaTeX}'s \cs{AtBeginDocument}.
% \item \cs{AtBeginDocument}, provided by \hologo{LaTeX}.
% \item \cs{AfterEndDocument}, provided by package \xpackage{etoolbox} \cite{etoolbox},
%   is called at the very end of |\begin{document}|. Preamble commands are
%   already forbidden there.
% \end{enumerate}
% Because of the various locations the patch code is restricted to
% limitations:
% \begin{itemize}
% \item
% Preamble commands, see \hologo{LaTeX}'s \cs{@onlypreamble}
% throw an error if used after |\begin{document}|. This is already
% the case for \cs{AfterEndDocument}. Therefore preamble commands
% are forbidden in the patching code.
% There are four exceptions \cs{@ifpackageloaded}, \cs{@ifclassloaded},
% \cs{@ifpackagelater} and \cs{@ifclasslater}. They are redefined
% during \cs{AfterEndDocument} using the counterparts of
% package \xpackage{ltxcmds} \cite{ltxcmds}.
% \item
% \cs{AfterPackage} of package \xpackage{scrlfile} and \cs{AtEndPackage}
% of class \xclass{memoir} call the hook before \hologo{LaTeX}'s
% \cs{AtEndOfPackage}.
% \end{itemize}
%
% \StopEventually{
% }
%
% \section{Implementation}
%
%    \begin{macrocode}
%<*package>
%    \end{macrocode}
%
% \subsection{Catcodes and package identification}
%
%    \begin{macrocode}
\begingroup\catcode61\catcode48\catcode32=10\relax%
  \catcode13=5 % ^^M
  \endlinechar=13 %
  \catcode123=1 % {
  \catcode125=2 % }
  \catcode64=11 % @
  \def\x{\endgroup
    \expandafter\edef\csname HOpatch@AtEnd\endcsname{%
      \endlinechar=\the\endlinechar\relax
      \catcode13=\the\catcode13\relax
      \catcode32=\the\catcode32\relax
      \catcode35=\the\catcode35\relax
      \catcode61=\the\catcode61\relax
      \catcode64=\the\catcode64\relax
      \catcode123=\the\catcode123\relax
      \catcode125=\the\catcode125\relax
    }%
  }%
\x\catcode61\catcode48\catcode32=10\relax%
\catcode13=5 % ^^M
\endlinechar=13 %
\catcode35=6 % #
\catcode64=11 % @
\catcode123=1 % {
\catcode125=2 % }
\def\TMP@EnsureCode#1#2{%
  \edef\HOpatch@AtEnd{%
    \HOpatch@AtEnd
    \catcode#1=\the\catcode#1\relax
  }%
  \catcode#1=#2\relax
}
\TMP@EnsureCode{40}{12}% (
\TMP@EnsureCode{41}{12}% )
\TMP@EnsureCode{43}{12}% +
\TMP@EnsureCode{46}{12}% .
\TMP@EnsureCode{47}{12}% /
\TMP@EnsureCode{91}{12}% [
\TMP@EnsureCode{93}{12}% ]
\edef\HOpatch@AtEnd{\HOpatch@AtEnd\noexpand\endinput}
%    \end{macrocode}
%
%    Package identification.
%    \begin{macrocode}
\NeedsTeXFormat{LaTeX2e}
\ProvidesPackage{hopatch}%
  [2012/05/28 v1.2 Wrapper for package hooks (HO)]
%    \end{macrocode}
%
% \subsection{Resources}
%
%    \begin{macrocode}
\begingroup\expandafter\expandafter\expandafter\endgroup
\expandafter\ifx\csname RequirePackage\endcsname\relax
  \def\TMP@RequirePackage#1[#2]{%
    \begingroup\expandafter\expandafter\expandafter\endgroup
    \expandafter\ifx\csname ver@#1.sty\endcsname\relax
      \input #1.sty\relax
    \fi
  }%
  \TMP@RequirePackage{ltxcmds}[2010/12/12]%
\else
  \RequirePackage{ltxcmds}[2010/12/12]%
\fi
%    \end{macrocode}
%
%    \begin{macro}{\HOpatch@counter}
%    \begin{macrocode}
\def\HOpatch@counter{0}%
%    \end{macrocode}
%    \end{macro}
%    \begin{macro}{\HOpatch@StepCounter}
%    \begin{macrocode}
\ltx@ifundefined{numexpr}{%
  \def\HOpatch@StepCounter{%
    \begingroup
      \count@\HOpatch@counter\relax
      \advance\count@\ltx@one\relax
    \edef\x{\endgroup
      \noexpand\def\noexpand\HOpatch@counter{\the\count@}%
    }%
    \x
  }%
}{%
  \def\HOpatch@StepCounter{%
    \edef\HOpatch@counter{%
      \the\numexpr\HOpatch@counter+\ltx@one\relax
    }%
  }%
}
%    \end{macrocode}
%    \end{macro}
%
%    \begin{macro}{\HOpatch@list}
%    \begin{macrocode}
\def\HOpatch@list{}
%    \end{macrocode}
%    \end{macro}
%
%    \begin{macro}{\HOpatch@Add}
%    \begin{macrocode}
\def\HOpatch@Add{%
  \ltx@LocalAppendToMacro\HOpatch@list
}
%    \end{macrocode}
%    \end{macro}
%
% \subsection{Package patching}
%
%    \begin{macro}{\hopatch@AfterPackage}
%    \begin{macrocode}
\def\hopatch@AfterPackage#1{%
  \ltx@ifpackageloaded{#1}{%
    \ltx@firstofone
  }{%
    \HOpatch@AfterPackage{#1}%
  }%
}
%    \end{macrocode}
%    \end{macro}
%    \begin{macro}{\HOpatch@AfterPackage}
%    \begin{macrocode}
\def\HOpatch@AfterPackage#1{%
  \edef\HOpatch@temp{#1}%
  \HOpatch@StepCounter
  \expandafter\HOpatch@@AfterPackage
  \csname HOpatch@\HOpatch@counter\expandafter\endcsname{%
    \HOpatch@temp
  }%
}
%    \end{macrocode}
%    \end{macro}
%
%    \begin{macro}{\HOpatch@@AfterPackage}
%    \begin{macrocode}
\def\HOpatch@@AfterPackage#1#2#3{%
  \begingroup
    \toks@{#3}%
    \xdef\HOpatch@gtemp{%
      \noexpand\ltx@ifpackageloaded{#2}{%
        \noexpand\let\noexpand#1\noexpand\relax
        \the\toks@
      }{}%
    }%
  \endgroup
  \let#1\HOpatch@gtemp
  \HOpatch@Add#1%
  \HOpatch@Try{AfterPackage}{#2}#1%
  \HOpatch@Try{AtEndPackage}{#2}#1%
  \HOpatch@Try{AtEndOfPackageFile}{#2}#1%
}
%    \end{macrocode}
%    \end{macro}
%
%    \begin{macro}{\HOpatch@Try}
%    \begin{macrocode}
\def\HOpatch@Try#1#2#3{%
  \ltx@ifundefined{#1}{}{%
    \csname #1\endcsname{#2}{#3}%
  }%
}
%    \end{macrocode}
%    \end{macro}
%
%    \begin{macrocode}
\AtBeginDocument{\HOpatch@list}
\ltx@ifundefined{AtEndPreamble}{}{%
  \ltx@ifundefined{@endpreamblehook}{}{%
    \AtEndPreamble{\HOpatch@list}%
  }%
}
\ltx@ifundefined{AfterEndPreamble}{}{%
  \ltx@ifundefined{@afterendpreamblehook}{}{%
    \AfterEndPreamble{%
      \let\HOpatch@OrgIfPackageLoaded\@ifpackageloaded
      \let\HOpatch@OrgIfPackageLater\@ifpackagelater
      \let\HOpatch@OrgIfClassLoaded\@ifclassloaded
      \let\HOpatch@OrgIfClassLater\@ifclasslater
      \let\@ifpackageloaded\ltx@ifpackageloaded
      \let\@ifpackagelater\ltx@ifpackagelater
      \let\@ifclassloaded\ltx@ifclassloaded
      \let\@ifclasslater\ltx@ifclasslater
      \HOpatch@list
      \let\@ifpackageloaded\HOpatch@OrgIfPackageLoaded
      \let\@ifpackagelater\HOpatch@OrgIfPackageLater
      \let\@ifclassloaded\HOpatch@OrgIfClassLoaded
      \let\@ifclasslater\HOpatch@OrgIfClassLater
    }%
  }%
}
%    \end{macrocode}
%
%    \begin{macrocode}
\HOpatch@AtEnd%
%    \end{macrocode}
%    \begin{macrocode}
%</package>
%    \end{macrocode}
%
% \section{Test}
%
%    \begin{macrocode}
%<*test1>
\def\LoadCommand{\RequirePackage{hopatch}[2012/05/28]}
%</test1>
%    \end{macrocode}
% \subsection{Catcode checks for loading}
%
%    \begin{macrocode}
%<*test1>
%    \end{macrocode}
%    \begin{macrocode}
\catcode`\{=1 %
\catcode`\}=2 %
\catcode`\#=6 %
\catcode`\@=11 %
\expandafter\ifx\csname count@\endcsname\relax
  \countdef\count@=255 %
\fi
\expandafter\ifx\csname @gobble\endcsname\relax
  \long\def\@gobble#1{}%
\fi
\expandafter\ifx\csname @firstofone\endcsname\relax
  \long\def\@firstofone#1{#1}%
\fi
\expandafter\ifx\csname loop\endcsname\relax
  \expandafter\@firstofone
\else
  \expandafter\@gobble
\fi
{%
  \def\loop#1\repeat{%
    \def\body{#1}%
    \iterate
  }%
  \def\iterate{%
    \body
      \let\next\iterate
    \else
      \let\next\relax
    \fi
    \next
  }%
  \let\repeat=\fi
}%
\def\RestoreCatcodes{}
\count@=0 %
\loop
  \edef\RestoreCatcodes{%
    \RestoreCatcodes
    \catcode\the\count@=\the\catcode\count@\relax
  }%
\ifnum\count@<255 %
  \advance\count@ 1 %
\repeat

\def\RangeCatcodeInvalid#1#2{%
  \count@=#1\relax
  \loop
    \catcode\count@=15 %
  \ifnum\count@<#2\relax
    \advance\count@ 1 %
  \repeat
}
\def\RangeCatcodeCheck#1#2#3{%
  \count@=#1\relax
  \loop
    \ifnum#3=\catcode\count@
    \else
      \errmessage{%
        Character \the\count@\space
        with wrong catcode \the\catcode\count@\space
        instead of \number#3%
      }%
    \fi
  \ifnum\count@<#2\relax
    \advance\count@ 1 %
  \repeat
}
\def\space{ }
\expandafter\ifx\csname LoadCommand\endcsname\relax
  \def\LoadCommand{\input hopatch.sty\relax}%
\fi
\def\Test{%
  \RangeCatcodeInvalid{0}{47}%
  \RangeCatcodeInvalid{58}{64}%
  \RangeCatcodeInvalid{91}{96}%
  \RangeCatcodeInvalid{123}{255}%
  \catcode`\@=12 %
  \catcode`\\=0 %
  \catcode`\%=14 %
  \LoadCommand
  \RangeCatcodeCheck{0}{36}{15}%
  \RangeCatcodeCheck{37}{37}{14}%
  \RangeCatcodeCheck{38}{47}{15}%
  \RangeCatcodeCheck{48}{57}{12}%
  \RangeCatcodeCheck{58}{63}{15}%
  \RangeCatcodeCheck{64}{64}{12}%
  \RangeCatcodeCheck{65}{90}{11}%
  \RangeCatcodeCheck{91}{91}{15}%
  \RangeCatcodeCheck{92}{92}{0}%
  \RangeCatcodeCheck{93}{96}{15}%
  \RangeCatcodeCheck{97}{122}{11}%
  \RangeCatcodeCheck{123}{255}{15}%
  \RestoreCatcodes
}
\Test
\csname @@end\endcsname
\end
%    \end{macrocode}
%    \begin{macrocode}
%</test1>
%    \end{macrocode}
%
%    \begin{macrocode}
%<*test2>
\NeedsTeXFormat{LaTeX2e}
\providecommand\variant{0}
\RequirePackage{filecontents}
\begin{filecontents}{foo.sty}
\ProvidesPackage{foo}
\def\msg#{\immediate\write16}
\def\foo#1{%
  \msg{\fooformat{#1}}%
}
\def\fooformat#1{[#1]}% hash-ok
\foo{* Executing foo at package loading}
\end{filecontents}

\ifnum\variant=1 %
  \documentclass{memoir}%
\else
  \documentclass{article}%
\fi

\ifcase\variant\relax
\or % 1
\or % 2
  \usepackage{etoolbox}%
\or % 3
  \usepackage{scrlfile}%
\or % 4
  \usepackage{filehook}%
\fi

\AtBeginDocument{\foo{* AtBeginDocument before hopatch}}
\usepackage{hopatch}
\AtBeginDocument{\foo{* AtBeginDocument after hopatch}}

\makeatletter
\hopatch@AfterPackage{foo}{%
  \def\fooformat#1{<<#1>>}%
}
\makeatother

\AtBeginDocument{\foo{* AtBeginDocument before foo}}
\usepackage{foo}
\AtBeginDocument{\foo{* AtBeginDocument after foo}}

\foo{* Executing in preamble}

\begin{document}
\foo{* Executing in document}
\end{document}
%</test2>
%    \end{macrocode}
%
% \section{Installation}
%
% \subsection{Download}
%
% \paragraph{Package.} This package is available on
% CTAN\footnote{\url{ftp://ftp.ctan.org/tex-archive/}}:
% \begin{description}
% \item[\CTAN{macros/latex/contrib/oberdiek/hopatch.dtx}] The source file.
% \item[\CTAN{macros/latex/contrib/oberdiek/hopatch.pdf}] Documentation.
% \end{description}
%
%
% \paragraph{Bundle.} All the packages of the bundle `oberdiek'
% are also available in a TDS compliant ZIP archive. There
% the packages are already unpacked and the documentation files
% are generated. The files and directories obey the TDS standard.
% \begin{description}
% \item[\CTAN{install/macros/latex/contrib/oberdiek.tds.zip}]
% \end{description}
% \emph{TDS} refers to the standard ``A Directory Structure
% for \TeX\ Files'' (\CTAN{tds/tds.pdf}). Directories
% with \xfile{texmf} in their name are usually organized this way.
%
% \subsection{Bundle installation}
%
% \paragraph{Unpacking.} Unpack the \xfile{oberdiek.tds.zip} in the
% TDS tree (also known as \xfile{texmf} tree) of your choice.
% Example (linux):
% \begin{quote}
%   |unzip oberdiek.tds.zip -d ~/texmf|
% \end{quote}
%
% \paragraph{Script installation.}
% Check the directory \xfile{TDS:scripts/oberdiek/} for
% scripts that need further installation steps.
% Package \xpackage{attachfile2} comes with the Perl script
% \xfile{pdfatfi.pl} that should be installed in such a way
% that it can be called as \texttt{pdfatfi}.
% Example (linux):
% \begin{quote}
%   |chmod +x scripts/oberdiek/pdfatfi.pl|\\
%   |cp scripts/oberdiek/pdfatfi.pl /usr/local/bin/|
% \end{quote}
%
% \subsection{Package installation}
%
% \paragraph{Unpacking.} The \xfile{.dtx} file is a self-extracting
% \docstrip\ archive. The files are extracted by running the
% \xfile{.dtx} through \plainTeX:
% \begin{quote}
%   \verb|tex hopatch.dtx|
% \end{quote}
%
% \paragraph{TDS.} Now the different files must be moved into
% the different directories in your installation TDS tree
% (also known as \xfile{texmf} tree):
% \begin{quote}
% \def\t{^^A
% \begin{tabular}{@{}>{\ttfamily}l@{ $\rightarrow$ }>{\ttfamily}l@{}}
%   hopatch.sty & tex/latex/oberdiek/hopatch.sty\\
%   hopatch.pdf & doc/latex/oberdiek/hopatch.pdf\\
%   test/hopatch-test1.tex & doc/latex/oberdiek/test/hopatch-test1.tex\\
%   test/hopatch-test2.tex & doc/latex/oberdiek/test/hopatch-test2.tex\\
%   hopatch.dtx & source/latex/oberdiek/hopatch.dtx\\
% \end{tabular}^^A
% }^^A
% \sbox0{\t}^^A
% \ifdim\wd0>\linewidth
%   \begingroup
%     \advance\linewidth by\leftmargin
%     \advance\linewidth by\rightmargin
%   \edef\x{\endgroup
%     \def\noexpand\lw{\the\linewidth}^^A
%   }\x
%   \def\lwbox{^^A
%     \leavevmode
%     \hbox to \linewidth{^^A
%       \kern-\leftmargin\relax
%       \hss
%       \usebox0
%       \hss
%       \kern-\rightmargin\relax
%     }^^A
%   }^^A
%   \ifdim\wd0>\lw
%     \sbox0{\small\t}^^A
%     \ifdim\wd0>\linewidth
%       \ifdim\wd0>\lw
%         \sbox0{\footnotesize\t}^^A
%         \ifdim\wd0>\linewidth
%           \ifdim\wd0>\lw
%             \sbox0{\scriptsize\t}^^A
%             \ifdim\wd0>\linewidth
%               \ifdim\wd0>\lw
%                 \sbox0{\tiny\t}^^A
%                 \ifdim\wd0>\linewidth
%                   \lwbox
%                 \else
%                   \usebox0
%                 \fi
%               \else
%                 \lwbox
%               \fi
%             \else
%               \usebox0
%             \fi
%           \else
%             \lwbox
%           \fi
%         \else
%           \usebox0
%         \fi
%       \else
%         \lwbox
%       \fi
%     \else
%       \usebox0
%     \fi
%   \else
%     \lwbox
%   \fi
% \else
%   \usebox0
% \fi
% \end{quote}
% If you have a \xfile{docstrip.cfg} that configures and enables \docstrip's
% TDS installing feature, then some files can already be in the right
% place, see the documentation of \docstrip.
%
% \subsection{Refresh file name databases}
%
% If your \TeX~distribution
% (\teTeX, \mikTeX, \dots) relies on file name databases, you must refresh
% these. For example, \teTeX\ users run \verb|texhash| or
% \verb|mktexlsr|.
%
% \subsection{Some details for the interested}
%
% \paragraph{Attached source.}
%
% The PDF documentation on CTAN also includes the
% \xfile{.dtx} source file. It can be extracted by
% AcrobatReader 6 or higher. Another option is \textsf{pdftk},
% e.g. unpack the file into the current directory:
% \begin{quote}
%   \verb|pdftk hopatch.pdf unpack_files output .|
% \end{quote}
%
% \paragraph{Unpacking with \LaTeX.}
% The \xfile{.dtx} chooses its action depending on the format:
% \begin{description}
% \item[\plainTeX:] Run \docstrip\ and extract the files.
% \item[\LaTeX:] Generate the documentation.
% \end{description}
% If you insist on using \LaTeX\ for \docstrip\ (really,
% \docstrip\ does not need \LaTeX), then inform the autodetect routine
% about your intention:
% \begin{quote}
%   \verb|latex \let\install=y\input{hopatch.dtx}|
% \end{quote}
% Do not forget to quote the argument according to the demands
% of your shell.
%
% \paragraph{Generating the documentation.}
% You can use both the \xfile{.dtx} or the \xfile{.drv} to generate
% the documentation. The process can be configured by the
% configuration file \xfile{ltxdoc.cfg}. For instance, put this
% line into this file, if you want to have A4 as paper format:
% \begin{quote}
%   \verb|\PassOptionsToClass{a4paper}{article}|
% \end{quote}
% An example follows how to generate the
% documentation with pdf\LaTeX:
% \begin{quote}
%\begin{verbatim}
%pdflatex hopatch.dtx
%makeindex -s gind.ist hopatch.idx
%pdflatex hopatch.dtx
%makeindex -s gind.ist hopatch.idx
%pdflatex hopatch.dtx
%\end{verbatim}
% \end{quote}
%
% \section{Catalogue}
%
% The following XML file can be used as source for the
% \href{http://mirror.ctan.org/help/Catalogue/catalogue.html}{\TeX\ Catalogue}.
% The elements \texttt{caption} and \texttt{description} are imported
% from the original XML file from the Catalogue.
% The name of the XML file in the Catalogue is \xfile{hopatch.xml}.
%    \begin{macrocode}
%<*catalogue>
<?xml version='1.0' encoding='us-ascii'?>
<!DOCTYPE entry SYSTEM 'catalogue.dtd'>
<entry datestamp='$Date$' modifier='$Author$' id='hopatch'>
  <name>hopatch</name>
  <caption>Load patches for packages.</caption>
  <authorref id='auth:oberdiek'/>
  <copyright owner='Heiko Oberdiek' year='2011,2012'/>
  <license type='lppl1.3'/>
  <version number='1.2'/>
  <description>
    The hopatch package provides a command with which the user may
    register a piece of patch code for a particular package.  Hopatch
    will apply the patch immediately, if the relevant package has
    already been loaded; otherwise it will store the patch until the
    package appears.
    <p/>
    The package is part of the <xref refid='oberdiek'>oberdiek</xref> bundle.
  </description>
  <documentation details='Package documentation'
      href='ctan:/macros/latex/contrib/oberdiek/hopatch.pdf'/>
  <ctan file='true' path='/macros/latex/contrib/oberdiek/hopatch.dtx'/>
  <miktex location='oberdiek'/>
  <texlive location='oberdiek'/>
  <install path='/macros/latex/contrib/oberdiek/oberdiek.tds.zip'/>
</entry>
%</catalogue>
%    \end{macrocode}
%
% \begin{thebibliography}{9}
% \raggedright
%
% \bibitem{etoolbox}
% Philipp Lehman:
% \textit{The etoolbox Package}
% 2011-01-03.
% \CTAN{macros/latex/contrib/etoolbox/etoolbox.pdf}
%
% \bibitem{filehook}
% Martin Scharrer:
% \textit{The filehook Package};
% 2011-01-09.
% \CTAN{macros/latex/contrib/filehook/filehook.pdf}
%
% \bibitem{ltxcmds}
% Heiko Oberdiek:
% \textit{The ltxcmds Package};
% 2010-12-12.
% \CTAN{macros/latex/contrib/oberdiek/ltxcmds.pdf}
%
% \bibitem{memoir}
% Peter Wilson, Lars Madsen:
% \textit{The Memoir Class for Configurable Typesetting, User Guide};
% 2010.
% \CTAN{macros/latex/contrib/memoir/memman.pdf}
%
% \bibitem{scrlfile}
% Markus Kohm, Jens-Uwe Morawski:
% \textit{The Guide KOMA-Script};
% 2011-01-20.
% \CTAN{macros/latex/contrib/koma-script/scrguien.pdf}
%
% \end{thebibliography}
%
% \begin{History}
%   \begin{Version}{2011/01/30 v1.0}
%   \item
%     First public version.
%   \end{Version}
%   \begin{Version}{2011/06/24 v1.1}
%   \item
%     Fix the use of \cs{AtEndPreamble} and \cs{AfterEndPreamble}.
%     They are redefined by package \xpackage{etoolbox} after their
%     hooks are used and generate an error message then.
%   \end{Version}
%   \begin{Version}{2012/05/28 v1.2}
%   \item
%     Fix for use without \hologo{eTeX}
%     (thanks Gordon Lee).
%   \end{Version}
% \end{History}
%
% \PrintIndex
%
% \Finale
\endinput
|
% \end{quote}
% Do not forget to quote the argument according to the demands
% of your shell.
%
% \paragraph{Generating the documentation.}
% You can use both the \xfile{.dtx} or the \xfile{.drv} to generate
% the documentation. The process can be configured by the
% configuration file \xfile{ltxdoc.cfg}. For instance, put this
% line into this file, if you want to have A4 as paper format:
% \begin{quote}
%   \verb|\PassOptionsToClass{a4paper}{article}|
% \end{quote}
% An example follows how to generate the
% documentation with pdf\LaTeX:
% \begin{quote}
%\begin{verbatim}
%pdflatex hopatch.dtx
%makeindex -s gind.ist hopatch.idx
%pdflatex hopatch.dtx
%makeindex -s gind.ist hopatch.idx
%pdflatex hopatch.dtx
%\end{verbatim}
% \end{quote}
%
% \section{Catalogue}
%
% The following XML file can be used as source for the
% \href{http://mirror.ctan.org/help/Catalogue/catalogue.html}{\TeX\ Catalogue}.
% The elements \texttt{caption} and \texttt{description} are imported
% from the original XML file from the Catalogue.
% The name of the XML file in the Catalogue is \xfile{hopatch.xml}.
%    \begin{macrocode}
%<*catalogue>
<?xml version='1.0' encoding='us-ascii'?>
<!DOCTYPE entry SYSTEM 'catalogue.dtd'>
<entry datestamp='$Date$' modifier='$Author$' id='hopatch'>
  <name>hopatch</name>
  <caption>Load patches for packages.</caption>
  <authorref id='auth:oberdiek'/>
  <copyright owner='Heiko Oberdiek' year='2011,2012'/>
  <license type='lppl1.3'/>
  <version number='1.2'/>
  <description>
    The hopatch package provides a command with which the user may
    register a piece of patch code for a particular package.  Hopatch
    will apply the patch immediately, if the relevant package has
    already been loaded; otherwise it will store the patch until the
    package appears.
    <p/>
    The package is part of the <xref refid='oberdiek'>oberdiek</xref> bundle.
  </description>
  <documentation details='Package documentation'
      href='ctan:/macros/latex/contrib/oberdiek/hopatch.pdf'/>
  <ctan file='true' path='/macros/latex/contrib/oberdiek/hopatch.dtx'/>
  <miktex location='oberdiek'/>
  <texlive location='oberdiek'/>
  <install path='/macros/latex/contrib/oberdiek/oberdiek.tds.zip'/>
</entry>
%</catalogue>
%    \end{macrocode}
%
% \begin{thebibliography}{9}
% \raggedright
%
% \bibitem{etoolbox}
% Philipp Lehman:
% \textit{The etoolbox Package}
% 2011-01-03.
% \CTAN{macros/latex/contrib/etoolbox/etoolbox.pdf}
%
% \bibitem{filehook}
% Martin Scharrer:
% \textit{The filehook Package};
% 2011-01-09.
% \CTAN{macros/latex/contrib/filehook/filehook.pdf}
%
% \bibitem{ltxcmds}
% Heiko Oberdiek:
% \textit{The ltxcmds Package};
% 2010-12-12.
% \CTAN{macros/latex/contrib/oberdiek/ltxcmds.pdf}
%
% \bibitem{memoir}
% Peter Wilson, Lars Madsen:
% \textit{The Memoir Class for Configurable Typesetting, User Guide};
% 2010.
% \CTAN{macros/latex/contrib/memoir/memman.pdf}
%
% \bibitem{scrlfile}
% Markus Kohm, Jens-Uwe Morawski:
% \textit{The Guide KOMA-Script};
% 2011-01-20.
% \CTAN{macros/latex/contrib/koma-script/scrguien.pdf}
%
% \end{thebibliography}
%
% \begin{History}
%   \begin{Version}{2011/01/30 v1.0}
%   \item
%     First public version.
%   \end{Version}
%   \begin{Version}{2011/06/24 v1.1}
%   \item
%     Fix the use of \cs{AtEndPreamble} and \cs{AfterEndPreamble}.
%     They are redefined by package \xpackage{etoolbox} after their
%     hooks are used and generate an error message then.
%   \end{Version}
%   \begin{Version}{2012/05/28 v1.2}
%   \item
%     Fix for use without \hologo{eTeX}
%     (thanks Gordon Lee).
%   \end{Version}
% \end{History}
%
% \PrintIndex
%
% \Finale
\endinput

%        (quote the arguments according to the demands of your shell)
%
% Documentation:
%    (a) If hopatch.drv is present:
%           latex hopatch.drv
%    (b) Without hopatch.drv:
%           latex hopatch.dtx; ...
%    The class ltxdoc loads the configuration file ltxdoc.cfg
%    if available. Here you can specify further options, e.g.
%    use A4 as paper format:
%       \PassOptionsToClass{a4paper}{article}
%
%    Programm calls to get the documentation (example):
%       pdflatex hopatch.dtx
%       makeindex -s gind.ist hopatch.idx
%       pdflatex hopatch.dtx
%       makeindex -s gind.ist hopatch.idx
%       pdflatex hopatch.dtx
%
% Installation:
%    TDS:tex/latex/oberdiek/hopatch.sty
%    TDS:doc/latex/oberdiek/hopatch.pdf
%    TDS:doc/latex/oberdiek/test/hopatch-test1.tex
%    TDS:doc/latex/oberdiek/test/hopatch-test2.tex
%    TDS:source/latex/oberdiek/hopatch.dtx
%
%<*ignore>
\begingroup
  \catcode123=1 %
  \catcode125=2 %
  \def\x{LaTeX2e}%
\expandafter\endgroup
\ifcase 0\ifx\install y1\fi\expandafter
         \ifx\csname processbatchFile\endcsname\relax\else1\fi
         \ifx\fmtname\x\else 1\fi\relax
\else\csname fi\endcsname
%</ignore>
%<*install>
\input docstrip.tex
\Msg{************************************************************************}
\Msg{* Installation}
\Msg{* Package: hopatch 2012/05/28 v1.2 Wrapper for package hooks (HO)}
\Msg{************************************************************************}

\keepsilent
\askforoverwritefalse

\let\MetaPrefix\relax
\preamble

This is a generated file.

Project: hopatch
Version: 2012/05/28 v1.2

Copyright (C) 2011, 2012 by
   Heiko Oberdiek <heiko.oberdiek at googlemail.com>

This work may be distributed and/or modified under the
conditions of the LaTeX Project Public License, either
version 1.3c of this license or (at your option) any later
version. This version of this license is in
   http://www.latex-project.org/lppl/lppl-1-3c.txt
and the latest version of this license is in
   http://www.latex-project.org/lppl.txt
and version 1.3 or later is part of all distributions of
LaTeX version 2005/12/01 or later.

This work has the LPPL maintenance status "maintained".

This Current Maintainer of this work is Heiko Oberdiek.

This work consists of the main source file hopatch.dtx
and the derived files
   hopatch.sty, hopatch.pdf, hopatch.ins, hopatch.drv,
   hopatch-test1.tex, hopatch-test2.tex.

\endpreamble
\let\MetaPrefix\DoubleperCent

\generate{%
  \file{hopatch.ins}{\from{hopatch.dtx}{install}}%
  \file{hopatch.drv}{\from{hopatch.dtx}{driver}}%
  \usedir{tex/latex/oberdiek}%
  \file{hopatch.sty}{\from{hopatch.dtx}{package}}%
  \usedir{doc/latex/oberdiek/test}%
  \file{hopatch-test1.tex}{\from{hopatch.dtx}{test1}}%
  \file{hopatch-test2.tex}{\from{hopatch.dtx}{test2}}%
  \nopreamble
  \nopostamble
  \usedir{source/latex/oberdiek/catalogue}%
  \file{hopatch.xml}{\from{hopatch.dtx}{catalogue}}%
}

\catcode32=13\relax% active space
\let =\space%
\Msg{************************************************************************}
\Msg{*}
\Msg{* To finish the installation you have to move the following}
\Msg{* file into a directory searched by TeX:}
\Msg{*}
\Msg{*     hopatch.sty}
\Msg{*}
\Msg{* To produce the documentation run the file `hopatch.drv'}
\Msg{* through LaTeX.}
\Msg{*}
\Msg{* Happy TeXing!}
\Msg{*}
\Msg{************************************************************************}

\endbatchfile
%</install>
%<*ignore>
\fi
%</ignore>
%<*driver>
\NeedsTeXFormat{LaTeX2e}
\ProvidesFile{hopatch.drv}%
  [2012/05/28 v1.2 Wrapper for package hooks (HO)]%
\documentclass{ltxdoc}
\usepackage{holtxdoc}[2011/11/22]
\usepackage{caption}
\usepackage{array}
\setlength{\extrarowheight}{1pt}
\renewcommand*{\arraystretch}{1.1}
\begin{document}
  \DocInput{hopatch.dtx}%
\end{document}
%</driver>
% \fi
%
% \CheckSum{466}
%
% \CharacterTable
%  {Upper-case    \A\B\C\D\E\F\G\H\I\J\K\L\M\N\O\P\Q\R\S\T\U\V\W\X\Y\Z
%   Lower-case    \a\b\c\d\e\f\g\h\i\j\k\l\m\n\o\p\q\r\s\t\u\v\w\x\y\z
%   Digits        \0\1\2\3\4\5\6\7\8\9
%   Exclamation   \!     Double quote  \"     Hash (number) \#
%   Dollar        \$     Percent       \%     Ampersand     \&
%   Acute accent  \'     Left paren    \(     Right paren   \)
%   Asterisk      \*     Plus          \+     Comma         \,
%   Minus         \-     Point         \.     Solidus       \/
%   Colon         \:     Semicolon     \;     Less than     \<
%   Equals        \=     Greater than  \>     Question mark \?
%   Commercial at \@     Left bracket  \[     Backslash     \\
%   Right bracket \]     Circumflex    \^     Underscore    \_
%   Grave accent  \`     Left brace    \{     Vertical bar  \|
%   Right brace   \}     Tilde         \~}
%
% \GetFileInfo{hopatch.drv}
%
% \title{The \xpackage{hopatch} package}
% \date{2012/05/28 v1.2}
% \author{Heiko Oberdiek\\\xemail{heiko.oberdiek at googlemail.com}}
%
% \maketitle
%
% \begin{abstract}
% This packages provides a wrapper to various package hooks
% provided by other packages or classes, but does not define
% own hooks.
% \end{abstract}
%
% \tableofcontents
%
% \section{Documentation}
%
% Sometimes I want to add code right after a package
% has been loaded. Examples are bug fixes, adaptations,
% or added features as needed by package \xpackage{hyperref},
% for instance.
%
% Unhappily \hologo{LaTeX} does not provide this kind of
% hook. \cs{AtEndOfPackage} can be used inside the package
% only, because \hologo{LaTeX} clears the hook right before
% it loads the package.
%
% However, there are already many packages and classes
% that provide hooks that are executed after the package
% is loaded, see table~\ref{tab:hooks}.
%
% \begin{table}
% \caption{After package hooking}
% \label{tab:hooks}
% \begin{tabular}{ll}
% \bfseries Macro & \bfseries Provider\\
% \hline
% \cs{AfterPackage} & package \xpackage{scrlfile} \cite{scrlfile}\\
% \cs{AtEndOfPackageFile} & package \xpackage{filehook} \cite{filehook}\\
% \cs{AtEndPackage} & class \xclass{memoir} \cite{memoir}
% \end{tabular}
% \end{table}
%
% Package \xpackage{hopatch} can be used without the packages
% of table~\ref{tab:hooks}. But for an early executing right
% after a package is loaded, one of the following class or
% packages should be loaded before using \cs{hopatch@AfterPackage}:
% \begin{itemize}
%  \item package \xpackage{filehook}
%  \item package \xpackage{scrlfile}
%  \item class \xclass{memoir}
% \end{itemize}
%
% \begin{table}
% \caption{After begin document hooking}
% \begin{tabular}{ll}
% \bfseries Macro & \bfseries Provider\\
% \hline
% \cs{AtBeginDocument} & \hologo{LaTeX}'s kernel\\
% \cs{AtEndPreamble} & package \xpackage{etoolbox} \cite{etoolbox}\\
% \cs{AfterEndPreamble} & package \xpackage{etoolbox}
% \end{tabular}
% \end{table}
%
% Therefore I skip writing a new package for hooking into
% \hologo{LaTeX}'s package management and use this package
% to provide a wrapper to patch a package after it is loaded.
%
% \begin{declcs}{hopatch@AfterPackage} \M{package} \M{patch code}
% \end{declcs}
% If the package is already loaded, the \meta{patch code} is executed
% immediately.
% Otherwise the \meta{patch code} is stored in a command and tried
% at later locations until the package is available.
%
% The patch is tried in the following order:
% \begin{enumerate}
% \item If the package is already loaded, the patch is
%   applied immediately. Further locations are not tried.
% \item \cs{AtEndPackage}, provided by class \xclass{memoir} \cite{memoir},
%   and \cs{AfterPackage}, provided by package \xpackage{scrlfile} \cite{scrlfile},
%   are called right after the package file is input before
%   the hook of \hologo{LaTeX}'s \cs{AtEndOfPackage}.
% \item \cs{AtEndOfPackageFile}, provided by package \xpackage{filehook} \cite{filehook},
%   is called after the package is loaded and after the hook of
%   \hologo{LaTeX}'s \cs{AtEndOfPackage}.
% \item \cs{AtEndPreamble}, provided by package \xpackage{etoolbox} \cite{etoolbox},
%   is called at the beginning of |\begin{document}| before the hook
%   of \hologo{LaTeX}'s \cs{AtBeginDocument}.
% \item \cs{AtBeginDocument}, provided by \hologo{LaTeX}.
% \item \cs{AfterEndDocument}, provided by package \xpackage{etoolbox} \cite{etoolbox},
%   is called at the very end of |\begin{document}|. Preamble commands are
%   already forbidden there.
% \end{enumerate}
% Because of the various locations the patch code is restricted to
% limitations:
% \begin{itemize}
% \item
% Preamble commands, see \hologo{LaTeX}'s \cs{@onlypreamble}
% throw an error if used after |\begin{document}|. This is already
% the case for \cs{AfterEndDocument}. Therefore preamble commands
% are forbidden in the patching code.
% There are four exceptions \cs{@ifpackageloaded}, \cs{@ifclassloaded},
% \cs{@ifpackagelater} and \cs{@ifclasslater}. They are redefined
% during \cs{AfterEndDocument} using the counterparts of
% package \xpackage{ltxcmds} \cite{ltxcmds}.
% \item
% \cs{AfterPackage} of package \xpackage{scrlfile} and \cs{AtEndPackage}
% of class \xclass{memoir} call the hook before \hologo{LaTeX}'s
% \cs{AtEndOfPackage}.
% \end{itemize}
%
% \StopEventually{
% }
%
% \section{Implementation}
%
%    \begin{macrocode}
%<*package>
%    \end{macrocode}
%
% \subsection{Catcodes and package identification}
%
%    \begin{macrocode}
\begingroup\catcode61\catcode48\catcode32=10\relax%
  \catcode13=5 % ^^M
  \endlinechar=13 %
  \catcode123=1 % {
  \catcode125=2 % }
  \catcode64=11 % @
  \def\x{\endgroup
    \expandafter\edef\csname HOpatch@AtEnd\endcsname{%
      \endlinechar=\the\endlinechar\relax
      \catcode13=\the\catcode13\relax
      \catcode32=\the\catcode32\relax
      \catcode35=\the\catcode35\relax
      \catcode61=\the\catcode61\relax
      \catcode64=\the\catcode64\relax
      \catcode123=\the\catcode123\relax
      \catcode125=\the\catcode125\relax
    }%
  }%
\x\catcode61\catcode48\catcode32=10\relax%
\catcode13=5 % ^^M
\endlinechar=13 %
\catcode35=6 % #
\catcode64=11 % @
\catcode123=1 % {
\catcode125=2 % }
\def\TMP@EnsureCode#1#2{%
  \edef\HOpatch@AtEnd{%
    \HOpatch@AtEnd
    \catcode#1=\the\catcode#1\relax
  }%
  \catcode#1=#2\relax
}
\TMP@EnsureCode{40}{12}% (
\TMP@EnsureCode{41}{12}% )
\TMP@EnsureCode{43}{12}% +
\TMP@EnsureCode{46}{12}% .
\TMP@EnsureCode{47}{12}% /
\TMP@EnsureCode{91}{12}% [
\TMP@EnsureCode{93}{12}% ]
\edef\HOpatch@AtEnd{\HOpatch@AtEnd\noexpand\endinput}
%    \end{macrocode}
%
%    Package identification.
%    \begin{macrocode}
\NeedsTeXFormat{LaTeX2e}
\ProvidesPackage{hopatch}%
  [2012/05/28 v1.2 Wrapper for package hooks (HO)]
%    \end{macrocode}
%
% \subsection{Resources}
%
%    \begin{macrocode}
\begingroup\expandafter\expandafter\expandafter\endgroup
\expandafter\ifx\csname RequirePackage\endcsname\relax
  \def\TMP@RequirePackage#1[#2]{%
    \begingroup\expandafter\expandafter\expandafter\endgroup
    \expandafter\ifx\csname ver@#1.sty\endcsname\relax
      \input #1.sty\relax
    \fi
  }%
  \TMP@RequirePackage{ltxcmds}[2010/12/12]%
\else
  \RequirePackage{ltxcmds}[2010/12/12]%
\fi
%    \end{macrocode}
%
%    \begin{macro}{\HOpatch@counter}
%    \begin{macrocode}
\def\HOpatch@counter{0}%
%    \end{macrocode}
%    \end{macro}
%    \begin{macro}{\HOpatch@StepCounter}
%    \begin{macrocode}
\ltx@ifundefined{numexpr}{%
  \def\HOpatch@StepCounter{%
    \begingroup
      \count@\HOpatch@counter\relax
      \advance\count@\ltx@one\relax
    \edef\x{\endgroup
      \noexpand\def\noexpand\HOpatch@counter{\the\count@}%
    }%
    \x
  }%
}{%
  \def\HOpatch@StepCounter{%
    \edef\HOpatch@counter{%
      \the\numexpr\HOpatch@counter+\ltx@one\relax
    }%
  }%
}
%    \end{macrocode}
%    \end{macro}
%
%    \begin{macro}{\HOpatch@list}
%    \begin{macrocode}
\def\HOpatch@list{}
%    \end{macrocode}
%    \end{macro}
%
%    \begin{macro}{\HOpatch@Add}
%    \begin{macrocode}
\def\HOpatch@Add{%
  \ltx@LocalAppendToMacro\HOpatch@list
}
%    \end{macrocode}
%    \end{macro}
%
% \subsection{Package patching}
%
%    \begin{macro}{\hopatch@AfterPackage}
%    \begin{macrocode}
\def\hopatch@AfterPackage#1{%
  \ltx@ifpackageloaded{#1}{%
    \ltx@firstofone
  }{%
    \HOpatch@AfterPackage{#1}%
  }%
}
%    \end{macrocode}
%    \end{macro}
%    \begin{macro}{\HOpatch@AfterPackage}
%    \begin{macrocode}
\def\HOpatch@AfterPackage#1{%
  \edef\HOpatch@temp{#1}%
  \HOpatch@StepCounter
  \expandafter\HOpatch@@AfterPackage
  \csname HOpatch@\HOpatch@counter\expandafter\endcsname{%
    \HOpatch@temp
  }%
}
%    \end{macrocode}
%    \end{macro}
%
%    \begin{macro}{\HOpatch@@AfterPackage}
%    \begin{macrocode}
\def\HOpatch@@AfterPackage#1#2#3{%
  \begingroup
    \toks@{#3}%
    \xdef\HOpatch@gtemp{%
      \noexpand\ltx@ifpackageloaded{#2}{%
        \noexpand\let\noexpand#1\noexpand\relax
        \the\toks@
      }{}%
    }%
  \endgroup
  \let#1\HOpatch@gtemp
  \HOpatch@Add#1%
  \HOpatch@Try{AfterPackage}{#2}#1%
  \HOpatch@Try{AtEndPackage}{#2}#1%
  \HOpatch@Try{AtEndOfPackageFile}{#2}#1%
}
%    \end{macrocode}
%    \end{macro}
%
%    \begin{macro}{\HOpatch@Try}
%    \begin{macrocode}
\def\HOpatch@Try#1#2#3{%
  \ltx@ifundefined{#1}{}{%
    \csname #1\endcsname{#2}{#3}%
  }%
}
%    \end{macrocode}
%    \end{macro}
%
%    \begin{macrocode}
\AtBeginDocument{\HOpatch@list}
\ltx@ifundefined{AtEndPreamble}{}{%
  \ltx@ifundefined{@endpreamblehook}{}{%
    \AtEndPreamble{\HOpatch@list}%
  }%
}
\ltx@ifundefined{AfterEndPreamble}{}{%
  \ltx@ifundefined{@afterendpreamblehook}{}{%
    \AfterEndPreamble{%
      \let\HOpatch@OrgIfPackageLoaded\@ifpackageloaded
      \let\HOpatch@OrgIfPackageLater\@ifpackagelater
      \let\HOpatch@OrgIfClassLoaded\@ifclassloaded
      \let\HOpatch@OrgIfClassLater\@ifclasslater
      \let\@ifpackageloaded\ltx@ifpackageloaded
      \let\@ifpackagelater\ltx@ifpackagelater
      \let\@ifclassloaded\ltx@ifclassloaded
      \let\@ifclasslater\ltx@ifclasslater
      \HOpatch@list
      \let\@ifpackageloaded\HOpatch@OrgIfPackageLoaded
      \let\@ifpackagelater\HOpatch@OrgIfPackageLater
      \let\@ifclassloaded\HOpatch@OrgIfClassLoaded
      \let\@ifclasslater\HOpatch@OrgIfClassLater
    }%
  }%
}
%    \end{macrocode}
%
%    \begin{macrocode}
\HOpatch@AtEnd%
%    \end{macrocode}
%    \begin{macrocode}
%</package>
%    \end{macrocode}
%
% \section{Test}
%
%    \begin{macrocode}
%<*test1>
\def\LoadCommand{\RequirePackage{hopatch}[2012/05/28]}
%</test1>
%    \end{macrocode}
% \subsection{Catcode checks for loading}
%
%    \begin{macrocode}
%<*test1>
%    \end{macrocode}
%    \begin{macrocode}
\catcode`\{=1 %
\catcode`\}=2 %
\catcode`\#=6 %
\catcode`\@=11 %
\expandafter\ifx\csname count@\endcsname\relax
  \countdef\count@=255 %
\fi
\expandafter\ifx\csname @gobble\endcsname\relax
  \long\def\@gobble#1{}%
\fi
\expandafter\ifx\csname @firstofone\endcsname\relax
  \long\def\@firstofone#1{#1}%
\fi
\expandafter\ifx\csname loop\endcsname\relax
  \expandafter\@firstofone
\else
  \expandafter\@gobble
\fi
{%
  \def\loop#1\repeat{%
    \def\body{#1}%
    \iterate
  }%
  \def\iterate{%
    \body
      \let\next\iterate
    \else
      \let\next\relax
    \fi
    \next
  }%
  \let\repeat=\fi
}%
\def\RestoreCatcodes{}
\count@=0 %
\loop
  \edef\RestoreCatcodes{%
    \RestoreCatcodes
    \catcode\the\count@=\the\catcode\count@\relax
  }%
\ifnum\count@<255 %
  \advance\count@ 1 %
\repeat

\def\RangeCatcodeInvalid#1#2{%
  \count@=#1\relax
  \loop
    \catcode\count@=15 %
  \ifnum\count@<#2\relax
    \advance\count@ 1 %
  \repeat
}
\def\RangeCatcodeCheck#1#2#3{%
  \count@=#1\relax
  \loop
    \ifnum#3=\catcode\count@
    \else
      \errmessage{%
        Character \the\count@\space
        with wrong catcode \the\catcode\count@\space
        instead of \number#3%
      }%
    \fi
  \ifnum\count@<#2\relax
    \advance\count@ 1 %
  \repeat
}
\def\space{ }
\expandafter\ifx\csname LoadCommand\endcsname\relax
  \def\LoadCommand{\input hopatch.sty\relax}%
\fi
\def\Test{%
  \RangeCatcodeInvalid{0}{47}%
  \RangeCatcodeInvalid{58}{64}%
  \RangeCatcodeInvalid{91}{96}%
  \RangeCatcodeInvalid{123}{255}%
  \catcode`\@=12 %
  \catcode`\\=0 %
  \catcode`\%=14 %
  \LoadCommand
  \RangeCatcodeCheck{0}{36}{15}%
  \RangeCatcodeCheck{37}{37}{14}%
  \RangeCatcodeCheck{38}{47}{15}%
  \RangeCatcodeCheck{48}{57}{12}%
  \RangeCatcodeCheck{58}{63}{15}%
  \RangeCatcodeCheck{64}{64}{12}%
  \RangeCatcodeCheck{65}{90}{11}%
  \RangeCatcodeCheck{91}{91}{15}%
  \RangeCatcodeCheck{92}{92}{0}%
  \RangeCatcodeCheck{93}{96}{15}%
  \RangeCatcodeCheck{97}{122}{11}%
  \RangeCatcodeCheck{123}{255}{15}%
  \RestoreCatcodes
}
\Test
\csname @@end\endcsname
\end
%    \end{macrocode}
%    \begin{macrocode}
%</test1>
%    \end{macrocode}
%
%    \begin{macrocode}
%<*test2>
\NeedsTeXFormat{LaTeX2e}
\providecommand\variant{0}
\RequirePackage{filecontents}
\begin{filecontents}{foo.sty}
\ProvidesPackage{foo}
\def\msg#{\immediate\write16}
\def\foo#1{%
  \msg{\fooformat{#1}}%
}
\def\fooformat#1{[#1]}% hash-ok
\foo{* Executing foo at package loading}
\end{filecontents}

\ifnum\variant=1 %
  \documentclass{memoir}%
\else
  \documentclass{article}%
\fi

\ifcase\variant\relax
\or % 1
\or % 2
  \usepackage{etoolbox}%
\or % 3
  \usepackage{scrlfile}%
\or % 4
  \usepackage{filehook}%
\fi

\AtBeginDocument{\foo{* AtBeginDocument before hopatch}}
\usepackage{hopatch}
\AtBeginDocument{\foo{* AtBeginDocument after hopatch}}

\makeatletter
\hopatch@AfterPackage{foo}{%
  \def\fooformat#1{<<#1>>}%
}
\makeatother

\AtBeginDocument{\foo{* AtBeginDocument before foo}}
\usepackage{foo}
\AtBeginDocument{\foo{* AtBeginDocument after foo}}

\foo{* Executing in preamble}

\begin{document}
\foo{* Executing in document}
\end{document}
%</test2>
%    \end{macrocode}
%
% \section{Installation}
%
% \subsection{Download}
%
% \paragraph{Package.} This package is available on
% CTAN\footnote{\url{ftp://ftp.ctan.org/tex-archive/}}:
% \begin{description}
% \item[\CTAN{macros/latex/contrib/oberdiek/hopatch.dtx}] The source file.
% \item[\CTAN{macros/latex/contrib/oberdiek/hopatch.pdf}] Documentation.
% \end{description}
%
%
% \paragraph{Bundle.} All the packages of the bundle `oberdiek'
% are also available in a TDS compliant ZIP archive. There
% the packages are already unpacked and the documentation files
% are generated. The files and directories obey the TDS standard.
% \begin{description}
% \item[\CTAN{install/macros/latex/contrib/oberdiek.tds.zip}]
% \end{description}
% \emph{TDS} refers to the standard ``A Directory Structure
% for \TeX\ Files'' (\CTAN{tds/tds.pdf}). Directories
% with \xfile{texmf} in their name are usually organized this way.
%
% \subsection{Bundle installation}
%
% \paragraph{Unpacking.} Unpack the \xfile{oberdiek.tds.zip} in the
% TDS tree (also known as \xfile{texmf} tree) of your choice.
% Example (linux):
% \begin{quote}
%   |unzip oberdiek.tds.zip -d ~/texmf|
% \end{quote}
%
% \paragraph{Script installation.}
% Check the directory \xfile{TDS:scripts/oberdiek/} for
% scripts that need further installation steps.
% Package \xpackage{attachfile2} comes with the Perl script
% \xfile{pdfatfi.pl} that should be installed in such a way
% that it can be called as \texttt{pdfatfi}.
% Example (linux):
% \begin{quote}
%   |chmod +x scripts/oberdiek/pdfatfi.pl|\\
%   |cp scripts/oberdiek/pdfatfi.pl /usr/local/bin/|
% \end{quote}
%
% \subsection{Package installation}
%
% \paragraph{Unpacking.} The \xfile{.dtx} file is a self-extracting
% \docstrip\ archive. The files are extracted by running the
% \xfile{.dtx} through \plainTeX:
% \begin{quote}
%   \verb|tex hopatch.dtx|
% \end{quote}
%
% \paragraph{TDS.} Now the different files must be moved into
% the different directories in your installation TDS tree
% (also known as \xfile{texmf} tree):
% \begin{quote}
% \def\t{^^A
% \begin{tabular}{@{}>{\ttfamily}l@{ $\rightarrow$ }>{\ttfamily}l@{}}
%   hopatch.sty & tex/latex/oberdiek/hopatch.sty\\
%   hopatch.pdf & doc/latex/oberdiek/hopatch.pdf\\
%   test/hopatch-test1.tex & doc/latex/oberdiek/test/hopatch-test1.tex\\
%   test/hopatch-test2.tex & doc/latex/oberdiek/test/hopatch-test2.tex\\
%   hopatch.dtx & source/latex/oberdiek/hopatch.dtx\\
% \end{tabular}^^A
% }^^A
% \sbox0{\t}^^A
% \ifdim\wd0>\linewidth
%   \begingroup
%     \advance\linewidth by\leftmargin
%     \advance\linewidth by\rightmargin
%   \edef\x{\endgroup
%     \def\noexpand\lw{\the\linewidth}^^A
%   }\x
%   \def\lwbox{^^A
%     \leavevmode
%     \hbox to \linewidth{^^A
%       \kern-\leftmargin\relax
%       \hss
%       \usebox0
%       \hss
%       \kern-\rightmargin\relax
%     }^^A
%   }^^A
%   \ifdim\wd0>\lw
%     \sbox0{\small\t}^^A
%     \ifdim\wd0>\linewidth
%       \ifdim\wd0>\lw
%         \sbox0{\footnotesize\t}^^A
%         \ifdim\wd0>\linewidth
%           \ifdim\wd0>\lw
%             \sbox0{\scriptsize\t}^^A
%             \ifdim\wd0>\linewidth
%               \ifdim\wd0>\lw
%                 \sbox0{\tiny\t}^^A
%                 \ifdim\wd0>\linewidth
%                   \lwbox
%                 \else
%                   \usebox0
%                 \fi
%               \else
%                 \lwbox
%               \fi
%             \else
%               \usebox0
%             \fi
%           \else
%             \lwbox
%           \fi
%         \else
%           \usebox0
%         \fi
%       \else
%         \lwbox
%       \fi
%     \else
%       \usebox0
%     \fi
%   \else
%     \lwbox
%   \fi
% \else
%   \usebox0
% \fi
% \end{quote}
% If you have a \xfile{docstrip.cfg} that configures and enables \docstrip's
% TDS installing feature, then some files can already be in the right
% place, see the documentation of \docstrip.
%
% \subsection{Refresh file name databases}
%
% If your \TeX~distribution
% (\teTeX, \mikTeX, \dots) relies on file name databases, you must refresh
% these. For example, \teTeX\ users run \verb|texhash| or
% \verb|mktexlsr|.
%
% \subsection{Some details for the interested}
%
% \paragraph{Attached source.}
%
% The PDF documentation on CTAN also includes the
% \xfile{.dtx} source file. It can be extracted by
% AcrobatReader 6 or higher. Another option is \textsf{pdftk},
% e.g. unpack the file into the current directory:
% \begin{quote}
%   \verb|pdftk hopatch.pdf unpack_files output .|
% \end{quote}
%
% \paragraph{Unpacking with \LaTeX.}
% The \xfile{.dtx} chooses its action depending on the format:
% \begin{description}
% \item[\plainTeX:] Run \docstrip\ and extract the files.
% \item[\LaTeX:] Generate the documentation.
% \end{description}
% If you insist on using \LaTeX\ for \docstrip\ (really,
% \docstrip\ does not need \LaTeX), then inform the autodetect routine
% about your intention:
% \begin{quote}
%   \verb|latex \let\install=y% \iffalse meta-comment
%
% File: hopatch.dtx
% Version: 2012/05/28 v1.2
% Info: Wrapper for package hooks
%
% Copyright (C) 2011, 2012 by
%    Heiko Oberdiek <heiko.oberdiek at googlemail.com>
%
% This work may be distributed and/or modified under the
% conditions of the LaTeX Project Public License, either
% version 1.3c of this license or (at your option) any later
% version. This version of this license is in
%    http://www.latex-project.org/lppl/lppl-1-3c.txt
% and the latest version of this license is in
%    http://www.latex-project.org/lppl.txt
% and version 1.3 or later is part of all distributions of
% LaTeX version 2005/12/01 or later.
%
% This work has the LPPL maintenance status "maintained".
%
% This Current Maintainer of this work is Heiko Oberdiek.
%
% This work consists of the main source file hopatch.dtx
% and the derived files
%    hopatch.sty, hopatch.pdf, hopatch.ins, hopatch.drv,
%    hopatch-test1.tex, hopatch-test2.tex.
%
% Distribution:
%    CTAN:macros/latex/contrib/oberdiek/hopatch.dtx
%    CTAN:macros/latex/contrib/oberdiek/hopatch.pdf
%
% Unpacking:
%    (a) If hopatch.ins is present:
%           tex hopatch.ins
%    (b) Without hopatch.ins:
%           tex hopatch.dtx
%    (c) If you insist on using LaTeX
%           latex \let\install=y% \iffalse meta-comment
%
% File: hopatch.dtx
% Version: 2012/05/28 v1.2
% Info: Wrapper for package hooks
%
% Copyright (C) 2011, 2012 by
%    Heiko Oberdiek <heiko.oberdiek at googlemail.com>
%
% This work may be distributed and/or modified under the
% conditions of the LaTeX Project Public License, either
% version 1.3c of this license or (at your option) any later
% version. This version of this license is in
%    http://www.latex-project.org/lppl/lppl-1-3c.txt
% and the latest version of this license is in
%    http://www.latex-project.org/lppl.txt
% and version 1.3 or later is part of all distributions of
% LaTeX version 2005/12/01 or later.
%
% This work has the LPPL maintenance status "maintained".
%
% This Current Maintainer of this work is Heiko Oberdiek.
%
% This work consists of the main source file hopatch.dtx
% and the derived files
%    hopatch.sty, hopatch.pdf, hopatch.ins, hopatch.drv,
%    hopatch-test1.tex, hopatch-test2.tex.
%
% Distribution:
%    CTAN:macros/latex/contrib/oberdiek/hopatch.dtx
%    CTAN:macros/latex/contrib/oberdiek/hopatch.pdf
%
% Unpacking:
%    (a) If hopatch.ins is present:
%           tex hopatch.ins
%    (b) Without hopatch.ins:
%           tex hopatch.dtx
%    (c) If you insist on using LaTeX
%           latex \let\install=y\input{hopatch.dtx}
%        (quote the arguments according to the demands of your shell)
%
% Documentation:
%    (a) If hopatch.drv is present:
%           latex hopatch.drv
%    (b) Without hopatch.drv:
%           latex hopatch.dtx; ...
%    The class ltxdoc loads the configuration file ltxdoc.cfg
%    if available. Here you can specify further options, e.g.
%    use A4 as paper format:
%       \PassOptionsToClass{a4paper}{article}
%
%    Programm calls to get the documentation (example):
%       pdflatex hopatch.dtx
%       makeindex -s gind.ist hopatch.idx
%       pdflatex hopatch.dtx
%       makeindex -s gind.ist hopatch.idx
%       pdflatex hopatch.dtx
%
% Installation:
%    TDS:tex/latex/oberdiek/hopatch.sty
%    TDS:doc/latex/oberdiek/hopatch.pdf
%    TDS:doc/latex/oberdiek/test/hopatch-test1.tex
%    TDS:doc/latex/oberdiek/test/hopatch-test2.tex
%    TDS:source/latex/oberdiek/hopatch.dtx
%
%<*ignore>
\begingroup
  \catcode123=1 %
  \catcode125=2 %
  \def\x{LaTeX2e}%
\expandafter\endgroup
\ifcase 0\ifx\install y1\fi\expandafter
         \ifx\csname processbatchFile\endcsname\relax\else1\fi
         \ifx\fmtname\x\else 1\fi\relax
\else\csname fi\endcsname
%</ignore>
%<*install>
\input docstrip.tex
\Msg{************************************************************************}
\Msg{* Installation}
\Msg{* Package: hopatch 2012/05/28 v1.2 Wrapper for package hooks (HO)}
\Msg{************************************************************************}

\keepsilent
\askforoverwritefalse

\let\MetaPrefix\relax
\preamble

This is a generated file.

Project: hopatch
Version: 2012/05/28 v1.2

Copyright (C) 2011, 2012 by
   Heiko Oberdiek <heiko.oberdiek at googlemail.com>

This work may be distributed and/or modified under the
conditions of the LaTeX Project Public License, either
version 1.3c of this license or (at your option) any later
version. This version of this license is in
   http://www.latex-project.org/lppl/lppl-1-3c.txt
and the latest version of this license is in
   http://www.latex-project.org/lppl.txt
and version 1.3 or later is part of all distributions of
LaTeX version 2005/12/01 or later.

This work has the LPPL maintenance status "maintained".

This Current Maintainer of this work is Heiko Oberdiek.

This work consists of the main source file hopatch.dtx
and the derived files
   hopatch.sty, hopatch.pdf, hopatch.ins, hopatch.drv,
   hopatch-test1.tex, hopatch-test2.tex.

\endpreamble
\let\MetaPrefix\DoubleperCent

\generate{%
  \file{hopatch.ins}{\from{hopatch.dtx}{install}}%
  \file{hopatch.drv}{\from{hopatch.dtx}{driver}}%
  \usedir{tex/latex/oberdiek}%
  \file{hopatch.sty}{\from{hopatch.dtx}{package}}%
  \usedir{doc/latex/oberdiek/test}%
  \file{hopatch-test1.tex}{\from{hopatch.dtx}{test1}}%
  \file{hopatch-test2.tex}{\from{hopatch.dtx}{test2}}%
  \nopreamble
  \nopostamble
  \usedir{source/latex/oberdiek/catalogue}%
  \file{hopatch.xml}{\from{hopatch.dtx}{catalogue}}%
}

\catcode32=13\relax% active space
\let =\space%
\Msg{************************************************************************}
\Msg{*}
\Msg{* To finish the installation you have to move the following}
\Msg{* file into a directory searched by TeX:}
\Msg{*}
\Msg{*     hopatch.sty}
\Msg{*}
\Msg{* To produce the documentation run the file `hopatch.drv'}
\Msg{* through LaTeX.}
\Msg{*}
\Msg{* Happy TeXing!}
\Msg{*}
\Msg{************************************************************************}

\endbatchfile
%</install>
%<*ignore>
\fi
%</ignore>
%<*driver>
\NeedsTeXFormat{LaTeX2e}
\ProvidesFile{hopatch.drv}%
  [2012/05/28 v1.2 Wrapper for package hooks (HO)]%
\documentclass{ltxdoc}
\usepackage{holtxdoc}[2011/11/22]
\usepackage{caption}
\usepackage{array}
\setlength{\extrarowheight}{1pt}
\renewcommand*{\arraystretch}{1.1}
\begin{document}
  \DocInput{hopatch.dtx}%
\end{document}
%</driver>
% \fi
%
% \CheckSum{466}
%
% \CharacterTable
%  {Upper-case    \A\B\C\D\E\F\G\H\I\J\K\L\M\N\O\P\Q\R\S\T\U\V\W\X\Y\Z
%   Lower-case    \a\b\c\d\e\f\g\h\i\j\k\l\m\n\o\p\q\r\s\t\u\v\w\x\y\z
%   Digits        \0\1\2\3\4\5\6\7\8\9
%   Exclamation   \!     Double quote  \"     Hash (number) \#
%   Dollar        \$     Percent       \%     Ampersand     \&
%   Acute accent  \'     Left paren    \(     Right paren   \)
%   Asterisk      \*     Plus          \+     Comma         \,
%   Minus         \-     Point         \.     Solidus       \/
%   Colon         \:     Semicolon     \;     Less than     \<
%   Equals        \=     Greater than  \>     Question mark \?
%   Commercial at \@     Left bracket  \[     Backslash     \\
%   Right bracket \]     Circumflex    \^     Underscore    \_
%   Grave accent  \`     Left brace    \{     Vertical bar  \|
%   Right brace   \}     Tilde         \~}
%
% \GetFileInfo{hopatch.drv}
%
% \title{The \xpackage{hopatch} package}
% \date{2012/05/28 v1.2}
% \author{Heiko Oberdiek\\\xemail{heiko.oberdiek at googlemail.com}}
%
% \maketitle
%
% \begin{abstract}
% This packages provides a wrapper to various package hooks
% provided by other packages or classes, but does not define
% own hooks.
% \end{abstract}
%
% \tableofcontents
%
% \section{Documentation}
%
% Sometimes I want to add code right after a package
% has been loaded. Examples are bug fixes, adaptations,
% or added features as needed by package \xpackage{hyperref},
% for instance.
%
% Unhappily \hologo{LaTeX} does not provide this kind of
% hook. \cs{AtEndOfPackage} can be used inside the package
% only, because \hologo{LaTeX} clears the hook right before
% it loads the package.
%
% However, there are already many packages and classes
% that provide hooks that are executed after the package
% is loaded, see table~\ref{tab:hooks}.
%
% \begin{table}
% \caption{After package hooking}
% \label{tab:hooks}
% \begin{tabular}{ll}
% \bfseries Macro & \bfseries Provider\\
% \hline
% \cs{AfterPackage} & package \xpackage{scrlfile} \cite{scrlfile}\\
% \cs{AtEndOfPackageFile} & package \xpackage{filehook} \cite{filehook}\\
% \cs{AtEndPackage} & class \xclass{memoir} \cite{memoir}
% \end{tabular}
% \end{table}
%
% Package \xpackage{hopatch} can be used without the packages
% of table~\ref{tab:hooks}. But for an early executing right
% after a package is loaded, one of the following class or
% packages should be loaded before using \cs{hopatch@AfterPackage}:
% \begin{itemize}
%  \item package \xpackage{filehook}
%  \item package \xpackage{scrlfile}
%  \item class \xclass{memoir}
% \end{itemize}
%
% \begin{table}
% \caption{After begin document hooking}
% \begin{tabular}{ll}
% \bfseries Macro & \bfseries Provider\\
% \hline
% \cs{AtBeginDocument} & \hologo{LaTeX}'s kernel\\
% \cs{AtEndPreamble} & package \xpackage{etoolbox} \cite{etoolbox}\\
% \cs{AfterEndPreamble} & package \xpackage{etoolbox}
% \end{tabular}
% \end{table}
%
% Therefore I skip writing a new package for hooking into
% \hologo{LaTeX}'s package management and use this package
% to provide a wrapper to patch a package after it is loaded.
%
% \begin{declcs}{hopatch@AfterPackage} \M{package} \M{patch code}
% \end{declcs}
% If the package is already loaded, the \meta{patch code} is executed
% immediately.
% Otherwise the \meta{patch code} is stored in a command and tried
% at later locations until the package is available.
%
% The patch is tried in the following order:
% \begin{enumerate}
% \item If the package is already loaded, the patch is
%   applied immediately. Further locations are not tried.
% \item \cs{AtEndPackage}, provided by class \xclass{memoir} \cite{memoir},
%   and \cs{AfterPackage}, provided by package \xpackage{scrlfile} \cite{scrlfile},
%   are called right after the package file is input before
%   the hook of \hologo{LaTeX}'s \cs{AtEndOfPackage}.
% \item \cs{AtEndOfPackageFile}, provided by package \xpackage{filehook} \cite{filehook},
%   is called after the package is loaded and after the hook of
%   \hologo{LaTeX}'s \cs{AtEndOfPackage}.
% \item \cs{AtEndPreamble}, provided by package \xpackage{etoolbox} \cite{etoolbox},
%   is called at the beginning of |\begin{document}| before the hook
%   of \hologo{LaTeX}'s \cs{AtBeginDocument}.
% \item \cs{AtBeginDocument}, provided by \hologo{LaTeX}.
% \item \cs{AfterEndDocument}, provided by package \xpackage{etoolbox} \cite{etoolbox},
%   is called at the very end of |\begin{document}|. Preamble commands are
%   already forbidden there.
% \end{enumerate}
% Because of the various locations the patch code is restricted to
% limitations:
% \begin{itemize}
% \item
% Preamble commands, see \hologo{LaTeX}'s \cs{@onlypreamble}
% throw an error if used after |\begin{document}|. This is already
% the case for \cs{AfterEndDocument}. Therefore preamble commands
% are forbidden in the patching code.
% There are four exceptions \cs{@ifpackageloaded}, \cs{@ifclassloaded},
% \cs{@ifpackagelater} and \cs{@ifclasslater}. They are redefined
% during \cs{AfterEndDocument} using the counterparts of
% package \xpackage{ltxcmds} \cite{ltxcmds}.
% \item
% \cs{AfterPackage} of package \xpackage{scrlfile} and \cs{AtEndPackage}
% of class \xclass{memoir} call the hook before \hologo{LaTeX}'s
% \cs{AtEndOfPackage}.
% \end{itemize}
%
% \StopEventually{
% }
%
% \section{Implementation}
%
%    \begin{macrocode}
%<*package>
%    \end{macrocode}
%
% \subsection{Catcodes and package identification}
%
%    \begin{macrocode}
\begingroup\catcode61\catcode48\catcode32=10\relax%
  \catcode13=5 % ^^M
  \endlinechar=13 %
  \catcode123=1 % {
  \catcode125=2 % }
  \catcode64=11 % @
  \def\x{\endgroup
    \expandafter\edef\csname HOpatch@AtEnd\endcsname{%
      \endlinechar=\the\endlinechar\relax
      \catcode13=\the\catcode13\relax
      \catcode32=\the\catcode32\relax
      \catcode35=\the\catcode35\relax
      \catcode61=\the\catcode61\relax
      \catcode64=\the\catcode64\relax
      \catcode123=\the\catcode123\relax
      \catcode125=\the\catcode125\relax
    }%
  }%
\x\catcode61\catcode48\catcode32=10\relax%
\catcode13=5 % ^^M
\endlinechar=13 %
\catcode35=6 % #
\catcode64=11 % @
\catcode123=1 % {
\catcode125=2 % }
\def\TMP@EnsureCode#1#2{%
  \edef\HOpatch@AtEnd{%
    \HOpatch@AtEnd
    \catcode#1=\the\catcode#1\relax
  }%
  \catcode#1=#2\relax
}
\TMP@EnsureCode{40}{12}% (
\TMP@EnsureCode{41}{12}% )
\TMP@EnsureCode{43}{12}% +
\TMP@EnsureCode{46}{12}% .
\TMP@EnsureCode{47}{12}% /
\TMP@EnsureCode{91}{12}% [
\TMP@EnsureCode{93}{12}% ]
\edef\HOpatch@AtEnd{\HOpatch@AtEnd\noexpand\endinput}
%    \end{macrocode}
%
%    Package identification.
%    \begin{macrocode}
\NeedsTeXFormat{LaTeX2e}
\ProvidesPackage{hopatch}%
  [2012/05/28 v1.2 Wrapper for package hooks (HO)]
%    \end{macrocode}
%
% \subsection{Resources}
%
%    \begin{macrocode}
\begingroup\expandafter\expandafter\expandafter\endgroup
\expandafter\ifx\csname RequirePackage\endcsname\relax
  \def\TMP@RequirePackage#1[#2]{%
    \begingroup\expandafter\expandafter\expandafter\endgroup
    \expandafter\ifx\csname ver@#1.sty\endcsname\relax
      \input #1.sty\relax
    \fi
  }%
  \TMP@RequirePackage{ltxcmds}[2010/12/12]%
\else
  \RequirePackage{ltxcmds}[2010/12/12]%
\fi
%    \end{macrocode}
%
%    \begin{macro}{\HOpatch@counter}
%    \begin{macrocode}
\def\HOpatch@counter{0}%
%    \end{macrocode}
%    \end{macro}
%    \begin{macro}{\HOpatch@StepCounter}
%    \begin{macrocode}
\ltx@ifundefined{numexpr}{%
  \def\HOpatch@StepCounter{%
    \begingroup
      \count@\HOpatch@counter\relax
      \advance\count@\ltx@one\relax
    \edef\x{\endgroup
      \noexpand\def\noexpand\HOpatch@counter{\the\count@}%
    }%
    \x
  }%
}{%
  \def\HOpatch@StepCounter{%
    \edef\HOpatch@counter{%
      \the\numexpr\HOpatch@counter+\ltx@one\relax
    }%
  }%
}
%    \end{macrocode}
%    \end{macro}
%
%    \begin{macro}{\HOpatch@list}
%    \begin{macrocode}
\def\HOpatch@list{}
%    \end{macrocode}
%    \end{macro}
%
%    \begin{macro}{\HOpatch@Add}
%    \begin{macrocode}
\def\HOpatch@Add{%
  \ltx@LocalAppendToMacro\HOpatch@list
}
%    \end{macrocode}
%    \end{macro}
%
% \subsection{Package patching}
%
%    \begin{macro}{\hopatch@AfterPackage}
%    \begin{macrocode}
\def\hopatch@AfterPackage#1{%
  \ltx@ifpackageloaded{#1}{%
    \ltx@firstofone
  }{%
    \HOpatch@AfterPackage{#1}%
  }%
}
%    \end{macrocode}
%    \end{macro}
%    \begin{macro}{\HOpatch@AfterPackage}
%    \begin{macrocode}
\def\HOpatch@AfterPackage#1{%
  \edef\HOpatch@temp{#1}%
  \HOpatch@StepCounter
  \expandafter\HOpatch@@AfterPackage
  \csname HOpatch@\HOpatch@counter\expandafter\endcsname{%
    \HOpatch@temp
  }%
}
%    \end{macrocode}
%    \end{macro}
%
%    \begin{macro}{\HOpatch@@AfterPackage}
%    \begin{macrocode}
\def\HOpatch@@AfterPackage#1#2#3{%
  \begingroup
    \toks@{#3}%
    \xdef\HOpatch@gtemp{%
      \noexpand\ltx@ifpackageloaded{#2}{%
        \noexpand\let\noexpand#1\noexpand\relax
        \the\toks@
      }{}%
    }%
  \endgroup
  \let#1\HOpatch@gtemp
  \HOpatch@Add#1%
  \HOpatch@Try{AfterPackage}{#2}#1%
  \HOpatch@Try{AtEndPackage}{#2}#1%
  \HOpatch@Try{AtEndOfPackageFile}{#2}#1%
}
%    \end{macrocode}
%    \end{macro}
%
%    \begin{macro}{\HOpatch@Try}
%    \begin{macrocode}
\def\HOpatch@Try#1#2#3{%
  \ltx@ifundefined{#1}{}{%
    \csname #1\endcsname{#2}{#3}%
  }%
}
%    \end{macrocode}
%    \end{macro}
%
%    \begin{macrocode}
\AtBeginDocument{\HOpatch@list}
\ltx@ifundefined{AtEndPreamble}{}{%
  \ltx@ifundefined{@endpreamblehook}{}{%
    \AtEndPreamble{\HOpatch@list}%
  }%
}
\ltx@ifundefined{AfterEndPreamble}{}{%
  \ltx@ifundefined{@afterendpreamblehook}{}{%
    \AfterEndPreamble{%
      \let\HOpatch@OrgIfPackageLoaded\@ifpackageloaded
      \let\HOpatch@OrgIfPackageLater\@ifpackagelater
      \let\HOpatch@OrgIfClassLoaded\@ifclassloaded
      \let\HOpatch@OrgIfClassLater\@ifclasslater
      \let\@ifpackageloaded\ltx@ifpackageloaded
      \let\@ifpackagelater\ltx@ifpackagelater
      \let\@ifclassloaded\ltx@ifclassloaded
      \let\@ifclasslater\ltx@ifclasslater
      \HOpatch@list
      \let\@ifpackageloaded\HOpatch@OrgIfPackageLoaded
      \let\@ifpackagelater\HOpatch@OrgIfPackageLater
      \let\@ifclassloaded\HOpatch@OrgIfClassLoaded
      \let\@ifclasslater\HOpatch@OrgIfClassLater
    }%
  }%
}
%    \end{macrocode}
%
%    \begin{macrocode}
\HOpatch@AtEnd%
%    \end{macrocode}
%    \begin{macrocode}
%</package>
%    \end{macrocode}
%
% \section{Test}
%
%    \begin{macrocode}
%<*test1>
\def\LoadCommand{\RequirePackage{hopatch}[2012/05/28]}
%</test1>
%    \end{macrocode}
% \subsection{Catcode checks for loading}
%
%    \begin{macrocode}
%<*test1>
%    \end{macrocode}
%    \begin{macrocode}
\catcode`\{=1 %
\catcode`\}=2 %
\catcode`\#=6 %
\catcode`\@=11 %
\expandafter\ifx\csname count@\endcsname\relax
  \countdef\count@=255 %
\fi
\expandafter\ifx\csname @gobble\endcsname\relax
  \long\def\@gobble#1{}%
\fi
\expandafter\ifx\csname @firstofone\endcsname\relax
  \long\def\@firstofone#1{#1}%
\fi
\expandafter\ifx\csname loop\endcsname\relax
  \expandafter\@firstofone
\else
  \expandafter\@gobble
\fi
{%
  \def\loop#1\repeat{%
    \def\body{#1}%
    \iterate
  }%
  \def\iterate{%
    \body
      \let\next\iterate
    \else
      \let\next\relax
    \fi
    \next
  }%
  \let\repeat=\fi
}%
\def\RestoreCatcodes{}
\count@=0 %
\loop
  \edef\RestoreCatcodes{%
    \RestoreCatcodes
    \catcode\the\count@=\the\catcode\count@\relax
  }%
\ifnum\count@<255 %
  \advance\count@ 1 %
\repeat

\def\RangeCatcodeInvalid#1#2{%
  \count@=#1\relax
  \loop
    \catcode\count@=15 %
  \ifnum\count@<#2\relax
    \advance\count@ 1 %
  \repeat
}
\def\RangeCatcodeCheck#1#2#3{%
  \count@=#1\relax
  \loop
    \ifnum#3=\catcode\count@
    \else
      \errmessage{%
        Character \the\count@\space
        with wrong catcode \the\catcode\count@\space
        instead of \number#3%
      }%
    \fi
  \ifnum\count@<#2\relax
    \advance\count@ 1 %
  \repeat
}
\def\space{ }
\expandafter\ifx\csname LoadCommand\endcsname\relax
  \def\LoadCommand{\input hopatch.sty\relax}%
\fi
\def\Test{%
  \RangeCatcodeInvalid{0}{47}%
  \RangeCatcodeInvalid{58}{64}%
  \RangeCatcodeInvalid{91}{96}%
  \RangeCatcodeInvalid{123}{255}%
  \catcode`\@=12 %
  \catcode`\\=0 %
  \catcode`\%=14 %
  \LoadCommand
  \RangeCatcodeCheck{0}{36}{15}%
  \RangeCatcodeCheck{37}{37}{14}%
  \RangeCatcodeCheck{38}{47}{15}%
  \RangeCatcodeCheck{48}{57}{12}%
  \RangeCatcodeCheck{58}{63}{15}%
  \RangeCatcodeCheck{64}{64}{12}%
  \RangeCatcodeCheck{65}{90}{11}%
  \RangeCatcodeCheck{91}{91}{15}%
  \RangeCatcodeCheck{92}{92}{0}%
  \RangeCatcodeCheck{93}{96}{15}%
  \RangeCatcodeCheck{97}{122}{11}%
  \RangeCatcodeCheck{123}{255}{15}%
  \RestoreCatcodes
}
\Test
\csname @@end\endcsname
\end
%    \end{macrocode}
%    \begin{macrocode}
%</test1>
%    \end{macrocode}
%
%    \begin{macrocode}
%<*test2>
\NeedsTeXFormat{LaTeX2e}
\providecommand\variant{0}
\RequirePackage{filecontents}
\begin{filecontents}{foo.sty}
\ProvidesPackage{foo}
\def\msg#{\immediate\write16}
\def\foo#1{%
  \msg{\fooformat{#1}}%
}
\def\fooformat#1{[#1]}% hash-ok
\foo{* Executing foo at package loading}
\end{filecontents}

\ifnum\variant=1 %
  \documentclass{memoir}%
\else
  \documentclass{article}%
\fi

\ifcase\variant\relax
\or % 1
\or % 2
  \usepackage{etoolbox}%
\or % 3
  \usepackage{scrlfile}%
\or % 4
  \usepackage{filehook}%
\fi

\AtBeginDocument{\foo{* AtBeginDocument before hopatch}}
\usepackage{hopatch}
\AtBeginDocument{\foo{* AtBeginDocument after hopatch}}

\makeatletter
\hopatch@AfterPackage{foo}{%
  \def\fooformat#1{<<#1>>}%
}
\makeatother

\AtBeginDocument{\foo{* AtBeginDocument before foo}}
\usepackage{foo}
\AtBeginDocument{\foo{* AtBeginDocument after foo}}

\foo{* Executing in preamble}

\begin{document}
\foo{* Executing in document}
\end{document}
%</test2>
%    \end{macrocode}
%
% \section{Installation}
%
% \subsection{Download}
%
% \paragraph{Package.} This package is available on
% CTAN\footnote{\url{ftp://ftp.ctan.org/tex-archive/}}:
% \begin{description}
% \item[\CTAN{macros/latex/contrib/oberdiek/hopatch.dtx}] The source file.
% \item[\CTAN{macros/latex/contrib/oberdiek/hopatch.pdf}] Documentation.
% \end{description}
%
%
% \paragraph{Bundle.} All the packages of the bundle `oberdiek'
% are also available in a TDS compliant ZIP archive. There
% the packages are already unpacked and the documentation files
% are generated. The files and directories obey the TDS standard.
% \begin{description}
% \item[\CTAN{install/macros/latex/contrib/oberdiek.tds.zip}]
% \end{description}
% \emph{TDS} refers to the standard ``A Directory Structure
% for \TeX\ Files'' (\CTAN{tds/tds.pdf}). Directories
% with \xfile{texmf} in their name are usually organized this way.
%
% \subsection{Bundle installation}
%
% \paragraph{Unpacking.} Unpack the \xfile{oberdiek.tds.zip} in the
% TDS tree (also known as \xfile{texmf} tree) of your choice.
% Example (linux):
% \begin{quote}
%   |unzip oberdiek.tds.zip -d ~/texmf|
% \end{quote}
%
% \paragraph{Script installation.}
% Check the directory \xfile{TDS:scripts/oberdiek/} for
% scripts that need further installation steps.
% Package \xpackage{attachfile2} comes with the Perl script
% \xfile{pdfatfi.pl} that should be installed in such a way
% that it can be called as \texttt{pdfatfi}.
% Example (linux):
% \begin{quote}
%   |chmod +x scripts/oberdiek/pdfatfi.pl|\\
%   |cp scripts/oberdiek/pdfatfi.pl /usr/local/bin/|
% \end{quote}
%
% \subsection{Package installation}
%
% \paragraph{Unpacking.} The \xfile{.dtx} file is a self-extracting
% \docstrip\ archive. The files are extracted by running the
% \xfile{.dtx} through \plainTeX:
% \begin{quote}
%   \verb|tex hopatch.dtx|
% \end{quote}
%
% \paragraph{TDS.} Now the different files must be moved into
% the different directories in your installation TDS tree
% (also known as \xfile{texmf} tree):
% \begin{quote}
% \def\t{^^A
% \begin{tabular}{@{}>{\ttfamily}l@{ $\rightarrow$ }>{\ttfamily}l@{}}
%   hopatch.sty & tex/latex/oberdiek/hopatch.sty\\
%   hopatch.pdf & doc/latex/oberdiek/hopatch.pdf\\
%   test/hopatch-test1.tex & doc/latex/oberdiek/test/hopatch-test1.tex\\
%   test/hopatch-test2.tex & doc/latex/oberdiek/test/hopatch-test2.tex\\
%   hopatch.dtx & source/latex/oberdiek/hopatch.dtx\\
% \end{tabular}^^A
% }^^A
% \sbox0{\t}^^A
% \ifdim\wd0>\linewidth
%   \begingroup
%     \advance\linewidth by\leftmargin
%     \advance\linewidth by\rightmargin
%   \edef\x{\endgroup
%     \def\noexpand\lw{\the\linewidth}^^A
%   }\x
%   \def\lwbox{^^A
%     \leavevmode
%     \hbox to \linewidth{^^A
%       \kern-\leftmargin\relax
%       \hss
%       \usebox0
%       \hss
%       \kern-\rightmargin\relax
%     }^^A
%   }^^A
%   \ifdim\wd0>\lw
%     \sbox0{\small\t}^^A
%     \ifdim\wd0>\linewidth
%       \ifdim\wd0>\lw
%         \sbox0{\footnotesize\t}^^A
%         \ifdim\wd0>\linewidth
%           \ifdim\wd0>\lw
%             \sbox0{\scriptsize\t}^^A
%             \ifdim\wd0>\linewidth
%               \ifdim\wd0>\lw
%                 \sbox0{\tiny\t}^^A
%                 \ifdim\wd0>\linewidth
%                   \lwbox
%                 \else
%                   \usebox0
%                 \fi
%               \else
%                 \lwbox
%               \fi
%             \else
%               \usebox0
%             \fi
%           \else
%             \lwbox
%           \fi
%         \else
%           \usebox0
%         \fi
%       \else
%         \lwbox
%       \fi
%     \else
%       \usebox0
%     \fi
%   \else
%     \lwbox
%   \fi
% \else
%   \usebox0
% \fi
% \end{quote}
% If you have a \xfile{docstrip.cfg} that configures and enables \docstrip's
% TDS installing feature, then some files can already be in the right
% place, see the documentation of \docstrip.
%
% \subsection{Refresh file name databases}
%
% If your \TeX~distribution
% (\teTeX, \mikTeX, \dots) relies on file name databases, you must refresh
% these. For example, \teTeX\ users run \verb|texhash| or
% \verb|mktexlsr|.
%
% \subsection{Some details for the interested}
%
% \paragraph{Attached source.}
%
% The PDF documentation on CTAN also includes the
% \xfile{.dtx} source file. It can be extracted by
% AcrobatReader 6 or higher. Another option is \textsf{pdftk},
% e.g. unpack the file into the current directory:
% \begin{quote}
%   \verb|pdftk hopatch.pdf unpack_files output .|
% \end{quote}
%
% \paragraph{Unpacking with \LaTeX.}
% The \xfile{.dtx} chooses its action depending on the format:
% \begin{description}
% \item[\plainTeX:] Run \docstrip\ and extract the files.
% \item[\LaTeX:] Generate the documentation.
% \end{description}
% If you insist on using \LaTeX\ for \docstrip\ (really,
% \docstrip\ does not need \LaTeX), then inform the autodetect routine
% about your intention:
% \begin{quote}
%   \verb|latex \let\install=y\input{hopatch.dtx}|
% \end{quote}
% Do not forget to quote the argument according to the demands
% of your shell.
%
% \paragraph{Generating the documentation.}
% You can use both the \xfile{.dtx} or the \xfile{.drv} to generate
% the documentation. The process can be configured by the
% configuration file \xfile{ltxdoc.cfg}. For instance, put this
% line into this file, if you want to have A4 as paper format:
% \begin{quote}
%   \verb|\PassOptionsToClass{a4paper}{article}|
% \end{quote}
% An example follows how to generate the
% documentation with pdf\LaTeX:
% \begin{quote}
%\begin{verbatim}
%pdflatex hopatch.dtx
%makeindex -s gind.ist hopatch.idx
%pdflatex hopatch.dtx
%makeindex -s gind.ist hopatch.idx
%pdflatex hopatch.dtx
%\end{verbatim}
% \end{quote}
%
% \section{Catalogue}
%
% The following XML file can be used as source for the
% \href{http://mirror.ctan.org/help/Catalogue/catalogue.html}{\TeX\ Catalogue}.
% The elements \texttt{caption} and \texttt{description} are imported
% from the original XML file from the Catalogue.
% The name of the XML file in the Catalogue is \xfile{hopatch.xml}.
%    \begin{macrocode}
%<*catalogue>
<?xml version='1.0' encoding='us-ascii'?>
<!DOCTYPE entry SYSTEM 'catalogue.dtd'>
<entry datestamp='$Date$' modifier='$Author$' id='hopatch'>
  <name>hopatch</name>
  <caption>Load patches for packages.</caption>
  <authorref id='auth:oberdiek'/>
  <copyright owner='Heiko Oberdiek' year='2011,2012'/>
  <license type='lppl1.3'/>
  <version number='1.2'/>
  <description>
    The hopatch package provides a command with which the user may
    register a piece of patch code for a particular package.  Hopatch
    will apply the patch immediately, if the relevant package has
    already been loaded; otherwise it will store the patch until the
    package appears.
    <p/>
    The package is part of the <xref refid='oberdiek'>oberdiek</xref> bundle.
  </description>
  <documentation details='Package documentation'
      href='ctan:/macros/latex/contrib/oberdiek/hopatch.pdf'/>
  <ctan file='true' path='/macros/latex/contrib/oberdiek/hopatch.dtx'/>
  <miktex location='oberdiek'/>
  <texlive location='oberdiek'/>
  <install path='/macros/latex/contrib/oberdiek/oberdiek.tds.zip'/>
</entry>
%</catalogue>
%    \end{macrocode}
%
% \begin{thebibliography}{9}
% \raggedright
%
% \bibitem{etoolbox}
% Philipp Lehman:
% \textit{The etoolbox Package}
% 2011-01-03.
% \CTAN{macros/latex/contrib/etoolbox/etoolbox.pdf}
%
% \bibitem{filehook}
% Martin Scharrer:
% \textit{The filehook Package};
% 2011-01-09.
% \CTAN{macros/latex/contrib/filehook/filehook.pdf}
%
% \bibitem{ltxcmds}
% Heiko Oberdiek:
% \textit{The ltxcmds Package};
% 2010-12-12.
% \CTAN{macros/latex/contrib/oberdiek/ltxcmds.pdf}
%
% \bibitem{memoir}
% Peter Wilson, Lars Madsen:
% \textit{The Memoir Class for Configurable Typesetting, User Guide};
% 2010.
% \CTAN{macros/latex/contrib/memoir/memman.pdf}
%
% \bibitem{scrlfile}
% Markus Kohm, Jens-Uwe Morawski:
% \textit{The Guide KOMA-Script};
% 2011-01-20.
% \CTAN{macros/latex/contrib/koma-script/scrguien.pdf}
%
% \end{thebibliography}
%
% \begin{History}
%   \begin{Version}{2011/01/30 v1.0}
%   \item
%     First public version.
%   \end{Version}
%   \begin{Version}{2011/06/24 v1.1}
%   \item
%     Fix the use of \cs{AtEndPreamble} and \cs{AfterEndPreamble}.
%     They are redefined by package \xpackage{etoolbox} after their
%     hooks are used and generate an error message then.
%   \end{Version}
%   \begin{Version}{2012/05/28 v1.2}
%   \item
%     Fix for use without \hologo{eTeX}
%     (thanks Gordon Lee).
%   \end{Version}
% \end{History}
%
% \PrintIndex
%
% \Finale
\endinput

%        (quote the arguments according to the demands of your shell)
%
% Documentation:
%    (a) If hopatch.drv is present:
%           latex hopatch.drv
%    (b) Without hopatch.drv:
%           latex hopatch.dtx; ...
%    The class ltxdoc loads the configuration file ltxdoc.cfg
%    if available. Here you can specify further options, e.g.
%    use A4 as paper format:
%       \PassOptionsToClass{a4paper}{article}
%
%    Programm calls to get the documentation (example):
%       pdflatex hopatch.dtx
%       makeindex -s gind.ist hopatch.idx
%       pdflatex hopatch.dtx
%       makeindex -s gind.ist hopatch.idx
%       pdflatex hopatch.dtx
%
% Installation:
%    TDS:tex/latex/oberdiek/hopatch.sty
%    TDS:doc/latex/oberdiek/hopatch.pdf
%    TDS:doc/latex/oberdiek/test/hopatch-test1.tex
%    TDS:doc/latex/oberdiek/test/hopatch-test2.tex
%    TDS:source/latex/oberdiek/hopatch.dtx
%
%<*ignore>
\begingroup
  \catcode123=1 %
  \catcode125=2 %
  \def\x{LaTeX2e}%
\expandafter\endgroup
\ifcase 0\ifx\install y1\fi\expandafter
         \ifx\csname processbatchFile\endcsname\relax\else1\fi
         \ifx\fmtname\x\else 1\fi\relax
\else\csname fi\endcsname
%</ignore>
%<*install>
\input docstrip.tex
\Msg{************************************************************************}
\Msg{* Installation}
\Msg{* Package: hopatch 2012/05/28 v1.2 Wrapper for package hooks (HO)}
\Msg{************************************************************************}

\keepsilent
\askforoverwritefalse

\let\MetaPrefix\relax
\preamble

This is a generated file.

Project: hopatch
Version: 2012/05/28 v1.2

Copyright (C) 2011, 2012 by
   Heiko Oberdiek <heiko.oberdiek at googlemail.com>

This work may be distributed and/or modified under the
conditions of the LaTeX Project Public License, either
version 1.3c of this license or (at your option) any later
version. This version of this license is in
   http://www.latex-project.org/lppl/lppl-1-3c.txt
and the latest version of this license is in
   http://www.latex-project.org/lppl.txt
and version 1.3 or later is part of all distributions of
LaTeX version 2005/12/01 or later.

This work has the LPPL maintenance status "maintained".

This Current Maintainer of this work is Heiko Oberdiek.

This work consists of the main source file hopatch.dtx
and the derived files
   hopatch.sty, hopatch.pdf, hopatch.ins, hopatch.drv,
   hopatch-test1.tex, hopatch-test2.tex.

\endpreamble
\let\MetaPrefix\DoubleperCent

\generate{%
  \file{hopatch.ins}{\from{hopatch.dtx}{install}}%
  \file{hopatch.drv}{\from{hopatch.dtx}{driver}}%
  \usedir{tex/latex/oberdiek}%
  \file{hopatch.sty}{\from{hopatch.dtx}{package}}%
  \usedir{doc/latex/oberdiek/test}%
  \file{hopatch-test1.tex}{\from{hopatch.dtx}{test1}}%
  \file{hopatch-test2.tex}{\from{hopatch.dtx}{test2}}%
  \nopreamble
  \nopostamble
  \usedir{source/latex/oberdiek/catalogue}%
  \file{hopatch.xml}{\from{hopatch.dtx}{catalogue}}%
}

\catcode32=13\relax% active space
\let =\space%
\Msg{************************************************************************}
\Msg{*}
\Msg{* To finish the installation you have to move the following}
\Msg{* file into a directory searched by TeX:}
\Msg{*}
\Msg{*     hopatch.sty}
\Msg{*}
\Msg{* To produce the documentation run the file `hopatch.drv'}
\Msg{* through LaTeX.}
\Msg{*}
\Msg{* Happy TeXing!}
\Msg{*}
\Msg{************************************************************************}

\endbatchfile
%</install>
%<*ignore>
\fi
%</ignore>
%<*driver>
\NeedsTeXFormat{LaTeX2e}
\ProvidesFile{hopatch.drv}%
  [2012/05/28 v1.2 Wrapper for package hooks (HO)]%
\documentclass{ltxdoc}
\usepackage{holtxdoc}[2011/11/22]
\usepackage{caption}
\usepackage{array}
\setlength{\extrarowheight}{1pt}
\renewcommand*{\arraystretch}{1.1}
\begin{document}
  \DocInput{hopatch.dtx}%
\end{document}
%</driver>
% \fi
%
% \CheckSum{466}
%
% \CharacterTable
%  {Upper-case    \A\B\C\D\E\F\G\H\I\J\K\L\M\N\O\P\Q\R\S\T\U\V\W\X\Y\Z
%   Lower-case    \a\b\c\d\e\f\g\h\i\j\k\l\m\n\o\p\q\r\s\t\u\v\w\x\y\z
%   Digits        \0\1\2\3\4\5\6\7\8\9
%   Exclamation   \!     Double quote  \"     Hash (number) \#
%   Dollar        \$     Percent       \%     Ampersand     \&
%   Acute accent  \'     Left paren    \(     Right paren   \)
%   Asterisk      \*     Plus          \+     Comma         \,
%   Minus         \-     Point         \.     Solidus       \/
%   Colon         \:     Semicolon     \;     Less than     \<
%   Equals        \=     Greater than  \>     Question mark \?
%   Commercial at \@     Left bracket  \[     Backslash     \\
%   Right bracket \]     Circumflex    \^     Underscore    \_
%   Grave accent  \`     Left brace    \{     Vertical bar  \|
%   Right brace   \}     Tilde         \~}
%
% \GetFileInfo{hopatch.drv}
%
% \title{The \xpackage{hopatch} package}
% \date{2012/05/28 v1.2}
% \author{Heiko Oberdiek\\\xemail{heiko.oberdiek at googlemail.com}}
%
% \maketitle
%
% \begin{abstract}
% This packages provides a wrapper to various package hooks
% provided by other packages or classes, but does not define
% own hooks.
% \end{abstract}
%
% \tableofcontents
%
% \section{Documentation}
%
% Sometimes I want to add code right after a package
% has been loaded. Examples are bug fixes, adaptations,
% or added features as needed by package \xpackage{hyperref},
% for instance.
%
% Unhappily \hologo{LaTeX} does not provide this kind of
% hook. \cs{AtEndOfPackage} can be used inside the package
% only, because \hologo{LaTeX} clears the hook right before
% it loads the package.
%
% However, there are already many packages and classes
% that provide hooks that are executed after the package
% is loaded, see table~\ref{tab:hooks}.
%
% \begin{table}
% \caption{After package hooking}
% \label{tab:hooks}
% \begin{tabular}{ll}
% \bfseries Macro & \bfseries Provider\\
% \hline
% \cs{AfterPackage} & package \xpackage{scrlfile} \cite{scrlfile}\\
% \cs{AtEndOfPackageFile} & package \xpackage{filehook} \cite{filehook}\\
% \cs{AtEndPackage} & class \xclass{memoir} \cite{memoir}
% \end{tabular}
% \end{table}
%
% Package \xpackage{hopatch} can be used without the packages
% of table~\ref{tab:hooks}. But for an early executing right
% after a package is loaded, one of the following class or
% packages should be loaded before using \cs{hopatch@AfterPackage}:
% \begin{itemize}
%  \item package \xpackage{filehook}
%  \item package \xpackage{scrlfile}
%  \item class \xclass{memoir}
% \end{itemize}
%
% \begin{table}
% \caption{After begin document hooking}
% \begin{tabular}{ll}
% \bfseries Macro & \bfseries Provider\\
% \hline
% \cs{AtBeginDocument} & \hologo{LaTeX}'s kernel\\
% \cs{AtEndPreamble} & package \xpackage{etoolbox} \cite{etoolbox}\\
% \cs{AfterEndPreamble} & package \xpackage{etoolbox}
% \end{tabular}
% \end{table}
%
% Therefore I skip writing a new package for hooking into
% \hologo{LaTeX}'s package management and use this package
% to provide a wrapper to patch a package after it is loaded.
%
% \begin{declcs}{hopatch@AfterPackage} \M{package} \M{patch code}
% \end{declcs}
% If the package is already loaded, the \meta{patch code} is executed
% immediately.
% Otherwise the \meta{patch code} is stored in a command and tried
% at later locations until the package is available.
%
% The patch is tried in the following order:
% \begin{enumerate}
% \item If the package is already loaded, the patch is
%   applied immediately. Further locations are not tried.
% \item \cs{AtEndPackage}, provided by class \xclass{memoir} \cite{memoir},
%   and \cs{AfterPackage}, provided by package \xpackage{scrlfile} \cite{scrlfile},
%   are called right after the package file is input before
%   the hook of \hologo{LaTeX}'s \cs{AtEndOfPackage}.
% \item \cs{AtEndOfPackageFile}, provided by package \xpackage{filehook} \cite{filehook},
%   is called after the package is loaded and after the hook of
%   \hologo{LaTeX}'s \cs{AtEndOfPackage}.
% \item \cs{AtEndPreamble}, provided by package \xpackage{etoolbox} \cite{etoolbox},
%   is called at the beginning of |\begin{document}| before the hook
%   of \hologo{LaTeX}'s \cs{AtBeginDocument}.
% \item \cs{AtBeginDocument}, provided by \hologo{LaTeX}.
% \item \cs{AfterEndDocument}, provided by package \xpackage{etoolbox} \cite{etoolbox},
%   is called at the very end of |\begin{document}|. Preamble commands are
%   already forbidden there.
% \end{enumerate}
% Because of the various locations the patch code is restricted to
% limitations:
% \begin{itemize}
% \item
% Preamble commands, see \hologo{LaTeX}'s \cs{@onlypreamble}
% throw an error if used after |\begin{document}|. This is already
% the case for \cs{AfterEndDocument}. Therefore preamble commands
% are forbidden in the patching code.
% There are four exceptions \cs{@ifpackageloaded}, \cs{@ifclassloaded},
% \cs{@ifpackagelater} and \cs{@ifclasslater}. They are redefined
% during \cs{AfterEndDocument} using the counterparts of
% package \xpackage{ltxcmds} \cite{ltxcmds}.
% \item
% \cs{AfterPackage} of package \xpackage{scrlfile} and \cs{AtEndPackage}
% of class \xclass{memoir} call the hook before \hologo{LaTeX}'s
% \cs{AtEndOfPackage}.
% \end{itemize}
%
% \StopEventually{
% }
%
% \section{Implementation}
%
%    \begin{macrocode}
%<*package>
%    \end{macrocode}
%
% \subsection{Catcodes and package identification}
%
%    \begin{macrocode}
\begingroup\catcode61\catcode48\catcode32=10\relax%
  \catcode13=5 % ^^M
  \endlinechar=13 %
  \catcode123=1 % {
  \catcode125=2 % }
  \catcode64=11 % @
  \def\x{\endgroup
    \expandafter\edef\csname HOpatch@AtEnd\endcsname{%
      \endlinechar=\the\endlinechar\relax
      \catcode13=\the\catcode13\relax
      \catcode32=\the\catcode32\relax
      \catcode35=\the\catcode35\relax
      \catcode61=\the\catcode61\relax
      \catcode64=\the\catcode64\relax
      \catcode123=\the\catcode123\relax
      \catcode125=\the\catcode125\relax
    }%
  }%
\x\catcode61\catcode48\catcode32=10\relax%
\catcode13=5 % ^^M
\endlinechar=13 %
\catcode35=6 % #
\catcode64=11 % @
\catcode123=1 % {
\catcode125=2 % }
\def\TMP@EnsureCode#1#2{%
  \edef\HOpatch@AtEnd{%
    \HOpatch@AtEnd
    \catcode#1=\the\catcode#1\relax
  }%
  \catcode#1=#2\relax
}
\TMP@EnsureCode{40}{12}% (
\TMP@EnsureCode{41}{12}% )
\TMP@EnsureCode{43}{12}% +
\TMP@EnsureCode{46}{12}% .
\TMP@EnsureCode{47}{12}% /
\TMP@EnsureCode{91}{12}% [
\TMP@EnsureCode{93}{12}% ]
\edef\HOpatch@AtEnd{\HOpatch@AtEnd\noexpand\endinput}
%    \end{macrocode}
%
%    Package identification.
%    \begin{macrocode}
\NeedsTeXFormat{LaTeX2e}
\ProvidesPackage{hopatch}%
  [2012/05/28 v1.2 Wrapper for package hooks (HO)]
%    \end{macrocode}
%
% \subsection{Resources}
%
%    \begin{macrocode}
\begingroup\expandafter\expandafter\expandafter\endgroup
\expandafter\ifx\csname RequirePackage\endcsname\relax
  \def\TMP@RequirePackage#1[#2]{%
    \begingroup\expandafter\expandafter\expandafter\endgroup
    \expandafter\ifx\csname ver@#1.sty\endcsname\relax
      \input #1.sty\relax
    \fi
  }%
  \TMP@RequirePackage{ltxcmds}[2010/12/12]%
\else
  \RequirePackage{ltxcmds}[2010/12/12]%
\fi
%    \end{macrocode}
%
%    \begin{macro}{\HOpatch@counter}
%    \begin{macrocode}
\def\HOpatch@counter{0}%
%    \end{macrocode}
%    \end{macro}
%    \begin{macro}{\HOpatch@StepCounter}
%    \begin{macrocode}
\ltx@ifundefined{numexpr}{%
  \def\HOpatch@StepCounter{%
    \begingroup
      \count@\HOpatch@counter\relax
      \advance\count@\ltx@one\relax
    \edef\x{\endgroup
      \noexpand\def\noexpand\HOpatch@counter{\the\count@}%
    }%
    \x
  }%
}{%
  \def\HOpatch@StepCounter{%
    \edef\HOpatch@counter{%
      \the\numexpr\HOpatch@counter+\ltx@one\relax
    }%
  }%
}
%    \end{macrocode}
%    \end{macro}
%
%    \begin{macro}{\HOpatch@list}
%    \begin{macrocode}
\def\HOpatch@list{}
%    \end{macrocode}
%    \end{macro}
%
%    \begin{macro}{\HOpatch@Add}
%    \begin{macrocode}
\def\HOpatch@Add{%
  \ltx@LocalAppendToMacro\HOpatch@list
}
%    \end{macrocode}
%    \end{macro}
%
% \subsection{Package patching}
%
%    \begin{macro}{\hopatch@AfterPackage}
%    \begin{macrocode}
\def\hopatch@AfterPackage#1{%
  \ltx@ifpackageloaded{#1}{%
    \ltx@firstofone
  }{%
    \HOpatch@AfterPackage{#1}%
  }%
}
%    \end{macrocode}
%    \end{macro}
%    \begin{macro}{\HOpatch@AfterPackage}
%    \begin{macrocode}
\def\HOpatch@AfterPackage#1{%
  \edef\HOpatch@temp{#1}%
  \HOpatch@StepCounter
  \expandafter\HOpatch@@AfterPackage
  \csname HOpatch@\HOpatch@counter\expandafter\endcsname{%
    \HOpatch@temp
  }%
}
%    \end{macrocode}
%    \end{macro}
%
%    \begin{macro}{\HOpatch@@AfterPackage}
%    \begin{macrocode}
\def\HOpatch@@AfterPackage#1#2#3{%
  \begingroup
    \toks@{#3}%
    \xdef\HOpatch@gtemp{%
      \noexpand\ltx@ifpackageloaded{#2}{%
        \noexpand\let\noexpand#1\noexpand\relax
        \the\toks@
      }{}%
    }%
  \endgroup
  \let#1\HOpatch@gtemp
  \HOpatch@Add#1%
  \HOpatch@Try{AfterPackage}{#2}#1%
  \HOpatch@Try{AtEndPackage}{#2}#1%
  \HOpatch@Try{AtEndOfPackageFile}{#2}#1%
}
%    \end{macrocode}
%    \end{macro}
%
%    \begin{macro}{\HOpatch@Try}
%    \begin{macrocode}
\def\HOpatch@Try#1#2#3{%
  \ltx@ifundefined{#1}{}{%
    \csname #1\endcsname{#2}{#3}%
  }%
}
%    \end{macrocode}
%    \end{macro}
%
%    \begin{macrocode}
\AtBeginDocument{\HOpatch@list}
\ltx@ifundefined{AtEndPreamble}{}{%
  \ltx@ifundefined{@endpreamblehook}{}{%
    \AtEndPreamble{\HOpatch@list}%
  }%
}
\ltx@ifundefined{AfterEndPreamble}{}{%
  \ltx@ifundefined{@afterendpreamblehook}{}{%
    \AfterEndPreamble{%
      \let\HOpatch@OrgIfPackageLoaded\@ifpackageloaded
      \let\HOpatch@OrgIfPackageLater\@ifpackagelater
      \let\HOpatch@OrgIfClassLoaded\@ifclassloaded
      \let\HOpatch@OrgIfClassLater\@ifclasslater
      \let\@ifpackageloaded\ltx@ifpackageloaded
      \let\@ifpackagelater\ltx@ifpackagelater
      \let\@ifclassloaded\ltx@ifclassloaded
      \let\@ifclasslater\ltx@ifclasslater
      \HOpatch@list
      \let\@ifpackageloaded\HOpatch@OrgIfPackageLoaded
      \let\@ifpackagelater\HOpatch@OrgIfPackageLater
      \let\@ifclassloaded\HOpatch@OrgIfClassLoaded
      \let\@ifclasslater\HOpatch@OrgIfClassLater
    }%
  }%
}
%    \end{macrocode}
%
%    \begin{macrocode}
\HOpatch@AtEnd%
%    \end{macrocode}
%    \begin{macrocode}
%</package>
%    \end{macrocode}
%
% \section{Test}
%
%    \begin{macrocode}
%<*test1>
\def\LoadCommand{\RequirePackage{hopatch}[2012/05/28]}
%</test1>
%    \end{macrocode}
% \subsection{Catcode checks for loading}
%
%    \begin{macrocode}
%<*test1>
%    \end{macrocode}
%    \begin{macrocode}
\catcode`\{=1 %
\catcode`\}=2 %
\catcode`\#=6 %
\catcode`\@=11 %
\expandafter\ifx\csname count@\endcsname\relax
  \countdef\count@=255 %
\fi
\expandafter\ifx\csname @gobble\endcsname\relax
  \long\def\@gobble#1{}%
\fi
\expandafter\ifx\csname @firstofone\endcsname\relax
  \long\def\@firstofone#1{#1}%
\fi
\expandafter\ifx\csname loop\endcsname\relax
  \expandafter\@firstofone
\else
  \expandafter\@gobble
\fi
{%
  \def\loop#1\repeat{%
    \def\body{#1}%
    \iterate
  }%
  \def\iterate{%
    \body
      \let\next\iterate
    \else
      \let\next\relax
    \fi
    \next
  }%
  \let\repeat=\fi
}%
\def\RestoreCatcodes{}
\count@=0 %
\loop
  \edef\RestoreCatcodes{%
    \RestoreCatcodes
    \catcode\the\count@=\the\catcode\count@\relax
  }%
\ifnum\count@<255 %
  \advance\count@ 1 %
\repeat

\def\RangeCatcodeInvalid#1#2{%
  \count@=#1\relax
  \loop
    \catcode\count@=15 %
  \ifnum\count@<#2\relax
    \advance\count@ 1 %
  \repeat
}
\def\RangeCatcodeCheck#1#2#3{%
  \count@=#1\relax
  \loop
    \ifnum#3=\catcode\count@
    \else
      \errmessage{%
        Character \the\count@\space
        with wrong catcode \the\catcode\count@\space
        instead of \number#3%
      }%
    \fi
  \ifnum\count@<#2\relax
    \advance\count@ 1 %
  \repeat
}
\def\space{ }
\expandafter\ifx\csname LoadCommand\endcsname\relax
  \def\LoadCommand{\input hopatch.sty\relax}%
\fi
\def\Test{%
  \RangeCatcodeInvalid{0}{47}%
  \RangeCatcodeInvalid{58}{64}%
  \RangeCatcodeInvalid{91}{96}%
  \RangeCatcodeInvalid{123}{255}%
  \catcode`\@=12 %
  \catcode`\\=0 %
  \catcode`\%=14 %
  \LoadCommand
  \RangeCatcodeCheck{0}{36}{15}%
  \RangeCatcodeCheck{37}{37}{14}%
  \RangeCatcodeCheck{38}{47}{15}%
  \RangeCatcodeCheck{48}{57}{12}%
  \RangeCatcodeCheck{58}{63}{15}%
  \RangeCatcodeCheck{64}{64}{12}%
  \RangeCatcodeCheck{65}{90}{11}%
  \RangeCatcodeCheck{91}{91}{15}%
  \RangeCatcodeCheck{92}{92}{0}%
  \RangeCatcodeCheck{93}{96}{15}%
  \RangeCatcodeCheck{97}{122}{11}%
  \RangeCatcodeCheck{123}{255}{15}%
  \RestoreCatcodes
}
\Test
\csname @@end\endcsname
\end
%    \end{macrocode}
%    \begin{macrocode}
%</test1>
%    \end{macrocode}
%
%    \begin{macrocode}
%<*test2>
\NeedsTeXFormat{LaTeX2e}
\providecommand\variant{0}
\RequirePackage{filecontents}
\begin{filecontents}{foo.sty}
\ProvidesPackage{foo}
\def\msg#{\immediate\write16}
\def\foo#1{%
  \msg{\fooformat{#1}}%
}
\def\fooformat#1{[#1]}% hash-ok
\foo{* Executing foo at package loading}
\end{filecontents}

\ifnum\variant=1 %
  \documentclass{memoir}%
\else
  \documentclass{article}%
\fi

\ifcase\variant\relax
\or % 1
\or % 2
  \usepackage{etoolbox}%
\or % 3
  \usepackage{scrlfile}%
\or % 4
  \usepackage{filehook}%
\fi

\AtBeginDocument{\foo{* AtBeginDocument before hopatch}}
\usepackage{hopatch}
\AtBeginDocument{\foo{* AtBeginDocument after hopatch}}

\makeatletter
\hopatch@AfterPackage{foo}{%
  \def\fooformat#1{<<#1>>}%
}
\makeatother

\AtBeginDocument{\foo{* AtBeginDocument before foo}}
\usepackage{foo}
\AtBeginDocument{\foo{* AtBeginDocument after foo}}

\foo{* Executing in preamble}

\begin{document}
\foo{* Executing in document}
\end{document}
%</test2>
%    \end{macrocode}
%
% \section{Installation}
%
% \subsection{Download}
%
% \paragraph{Package.} This package is available on
% CTAN\footnote{\url{ftp://ftp.ctan.org/tex-archive/}}:
% \begin{description}
% \item[\CTAN{macros/latex/contrib/oberdiek/hopatch.dtx}] The source file.
% \item[\CTAN{macros/latex/contrib/oberdiek/hopatch.pdf}] Documentation.
% \end{description}
%
%
% \paragraph{Bundle.} All the packages of the bundle `oberdiek'
% are also available in a TDS compliant ZIP archive. There
% the packages are already unpacked and the documentation files
% are generated. The files and directories obey the TDS standard.
% \begin{description}
% \item[\CTAN{install/macros/latex/contrib/oberdiek.tds.zip}]
% \end{description}
% \emph{TDS} refers to the standard ``A Directory Structure
% for \TeX\ Files'' (\CTAN{tds/tds.pdf}). Directories
% with \xfile{texmf} in their name are usually organized this way.
%
% \subsection{Bundle installation}
%
% \paragraph{Unpacking.} Unpack the \xfile{oberdiek.tds.zip} in the
% TDS tree (also known as \xfile{texmf} tree) of your choice.
% Example (linux):
% \begin{quote}
%   |unzip oberdiek.tds.zip -d ~/texmf|
% \end{quote}
%
% \paragraph{Script installation.}
% Check the directory \xfile{TDS:scripts/oberdiek/} for
% scripts that need further installation steps.
% Package \xpackage{attachfile2} comes with the Perl script
% \xfile{pdfatfi.pl} that should be installed in such a way
% that it can be called as \texttt{pdfatfi}.
% Example (linux):
% \begin{quote}
%   |chmod +x scripts/oberdiek/pdfatfi.pl|\\
%   |cp scripts/oberdiek/pdfatfi.pl /usr/local/bin/|
% \end{quote}
%
% \subsection{Package installation}
%
% \paragraph{Unpacking.} The \xfile{.dtx} file is a self-extracting
% \docstrip\ archive. The files are extracted by running the
% \xfile{.dtx} through \plainTeX:
% \begin{quote}
%   \verb|tex hopatch.dtx|
% \end{quote}
%
% \paragraph{TDS.} Now the different files must be moved into
% the different directories in your installation TDS tree
% (also known as \xfile{texmf} tree):
% \begin{quote}
% \def\t{^^A
% \begin{tabular}{@{}>{\ttfamily}l@{ $\rightarrow$ }>{\ttfamily}l@{}}
%   hopatch.sty & tex/latex/oberdiek/hopatch.sty\\
%   hopatch.pdf & doc/latex/oberdiek/hopatch.pdf\\
%   test/hopatch-test1.tex & doc/latex/oberdiek/test/hopatch-test1.tex\\
%   test/hopatch-test2.tex & doc/latex/oberdiek/test/hopatch-test2.tex\\
%   hopatch.dtx & source/latex/oberdiek/hopatch.dtx\\
% \end{tabular}^^A
% }^^A
% \sbox0{\t}^^A
% \ifdim\wd0>\linewidth
%   \begingroup
%     \advance\linewidth by\leftmargin
%     \advance\linewidth by\rightmargin
%   \edef\x{\endgroup
%     \def\noexpand\lw{\the\linewidth}^^A
%   }\x
%   \def\lwbox{^^A
%     \leavevmode
%     \hbox to \linewidth{^^A
%       \kern-\leftmargin\relax
%       \hss
%       \usebox0
%       \hss
%       \kern-\rightmargin\relax
%     }^^A
%   }^^A
%   \ifdim\wd0>\lw
%     \sbox0{\small\t}^^A
%     \ifdim\wd0>\linewidth
%       \ifdim\wd0>\lw
%         \sbox0{\footnotesize\t}^^A
%         \ifdim\wd0>\linewidth
%           \ifdim\wd0>\lw
%             \sbox0{\scriptsize\t}^^A
%             \ifdim\wd0>\linewidth
%               \ifdim\wd0>\lw
%                 \sbox0{\tiny\t}^^A
%                 \ifdim\wd0>\linewidth
%                   \lwbox
%                 \else
%                   \usebox0
%                 \fi
%               \else
%                 \lwbox
%               \fi
%             \else
%               \usebox0
%             \fi
%           \else
%             \lwbox
%           \fi
%         \else
%           \usebox0
%         \fi
%       \else
%         \lwbox
%       \fi
%     \else
%       \usebox0
%     \fi
%   \else
%     \lwbox
%   \fi
% \else
%   \usebox0
% \fi
% \end{quote}
% If you have a \xfile{docstrip.cfg} that configures and enables \docstrip's
% TDS installing feature, then some files can already be in the right
% place, see the documentation of \docstrip.
%
% \subsection{Refresh file name databases}
%
% If your \TeX~distribution
% (\teTeX, \mikTeX, \dots) relies on file name databases, you must refresh
% these. For example, \teTeX\ users run \verb|texhash| or
% \verb|mktexlsr|.
%
% \subsection{Some details for the interested}
%
% \paragraph{Attached source.}
%
% The PDF documentation on CTAN also includes the
% \xfile{.dtx} source file. It can be extracted by
% AcrobatReader 6 or higher. Another option is \textsf{pdftk},
% e.g. unpack the file into the current directory:
% \begin{quote}
%   \verb|pdftk hopatch.pdf unpack_files output .|
% \end{quote}
%
% \paragraph{Unpacking with \LaTeX.}
% The \xfile{.dtx} chooses its action depending on the format:
% \begin{description}
% \item[\plainTeX:] Run \docstrip\ and extract the files.
% \item[\LaTeX:] Generate the documentation.
% \end{description}
% If you insist on using \LaTeX\ for \docstrip\ (really,
% \docstrip\ does not need \LaTeX), then inform the autodetect routine
% about your intention:
% \begin{quote}
%   \verb|latex \let\install=y% \iffalse meta-comment
%
% File: hopatch.dtx
% Version: 2012/05/28 v1.2
% Info: Wrapper for package hooks
%
% Copyright (C) 2011, 2012 by
%    Heiko Oberdiek <heiko.oberdiek at googlemail.com>
%
% This work may be distributed and/or modified under the
% conditions of the LaTeX Project Public License, either
% version 1.3c of this license or (at your option) any later
% version. This version of this license is in
%    http://www.latex-project.org/lppl/lppl-1-3c.txt
% and the latest version of this license is in
%    http://www.latex-project.org/lppl.txt
% and version 1.3 or later is part of all distributions of
% LaTeX version 2005/12/01 or later.
%
% This work has the LPPL maintenance status "maintained".
%
% This Current Maintainer of this work is Heiko Oberdiek.
%
% This work consists of the main source file hopatch.dtx
% and the derived files
%    hopatch.sty, hopatch.pdf, hopatch.ins, hopatch.drv,
%    hopatch-test1.tex, hopatch-test2.tex.
%
% Distribution:
%    CTAN:macros/latex/contrib/oberdiek/hopatch.dtx
%    CTAN:macros/latex/contrib/oberdiek/hopatch.pdf
%
% Unpacking:
%    (a) If hopatch.ins is present:
%           tex hopatch.ins
%    (b) Without hopatch.ins:
%           tex hopatch.dtx
%    (c) If you insist on using LaTeX
%           latex \let\install=y\input{hopatch.dtx}
%        (quote the arguments according to the demands of your shell)
%
% Documentation:
%    (a) If hopatch.drv is present:
%           latex hopatch.drv
%    (b) Without hopatch.drv:
%           latex hopatch.dtx; ...
%    The class ltxdoc loads the configuration file ltxdoc.cfg
%    if available. Here you can specify further options, e.g.
%    use A4 as paper format:
%       \PassOptionsToClass{a4paper}{article}
%
%    Programm calls to get the documentation (example):
%       pdflatex hopatch.dtx
%       makeindex -s gind.ist hopatch.idx
%       pdflatex hopatch.dtx
%       makeindex -s gind.ist hopatch.idx
%       pdflatex hopatch.dtx
%
% Installation:
%    TDS:tex/latex/oberdiek/hopatch.sty
%    TDS:doc/latex/oberdiek/hopatch.pdf
%    TDS:doc/latex/oberdiek/test/hopatch-test1.tex
%    TDS:doc/latex/oberdiek/test/hopatch-test2.tex
%    TDS:source/latex/oberdiek/hopatch.dtx
%
%<*ignore>
\begingroup
  \catcode123=1 %
  \catcode125=2 %
  \def\x{LaTeX2e}%
\expandafter\endgroup
\ifcase 0\ifx\install y1\fi\expandafter
         \ifx\csname processbatchFile\endcsname\relax\else1\fi
         \ifx\fmtname\x\else 1\fi\relax
\else\csname fi\endcsname
%</ignore>
%<*install>
\input docstrip.tex
\Msg{************************************************************************}
\Msg{* Installation}
\Msg{* Package: hopatch 2012/05/28 v1.2 Wrapper for package hooks (HO)}
\Msg{************************************************************************}

\keepsilent
\askforoverwritefalse

\let\MetaPrefix\relax
\preamble

This is a generated file.

Project: hopatch
Version: 2012/05/28 v1.2

Copyright (C) 2011, 2012 by
   Heiko Oberdiek <heiko.oberdiek at googlemail.com>

This work may be distributed and/or modified under the
conditions of the LaTeX Project Public License, either
version 1.3c of this license or (at your option) any later
version. This version of this license is in
   http://www.latex-project.org/lppl/lppl-1-3c.txt
and the latest version of this license is in
   http://www.latex-project.org/lppl.txt
and version 1.3 or later is part of all distributions of
LaTeX version 2005/12/01 or later.

This work has the LPPL maintenance status "maintained".

This Current Maintainer of this work is Heiko Oberdiek.

This work consists of the main source file hopatch.dtx
and the derived files
   hopatch.sty, hopatch.pdf, hopatch.ins, hopatch.drv,
   hopatch-test1.tex, hopatch-test2.tex.

\endpreamble
\let\MetaPrefix\DoubleperCent

\generate{%
  \file{hopatch.ins}{\from{hopatch.dtx}{install}}%
  \file{hopatch.drv}{\from{hopatch.dtx}{driver}}%
  \usedir{tex/latex/oberdiek}%
  \file{hopatch.sty}{\from{hopatch.dtx}{package}}%
  \usedir{doc/latex/oberdiek/test}%
  \file{hopatch-test1.tex}{\from{hopatch.dtx}{test1}}%
  \file{hopatch-test2.tex}{\from{hopatch.dtx}{test2}}%
  \nopreamble
  \nopostamble
  \usedir{source/latex/oberdiek/catalogue}%
  \file{hopatch.xml}{\from{hopatch.dtx}{catalogue}}%
}

\catcode32=13\relax% active space
\let =\space%
\Msg{************************************************************************}
\Msg{*}
\Msg{* To finish the installation you have to move the following}
\Msg{* file into a directory searched by TeX:}
\Msg{*}
\Msg{*     hopatch.sty}
\Msg{*}
\Msg{* To produce the documentation run the file `hopatch.drv'}
\Msg{* through LaTeX.}
\Msg{*}
\Msg{* Happy TeXing!}
\Msg{*}
\Msg{************************************************************************}

\endbatchfile
%</install>
%<*ignore>
\fi
%</ignore>
%<*driver>
\NeedsTeXFormat{LaTeX2e}
\ProvidesFile{hopatch.drv}%
  [2012/05/28 v1.2 Wrapper for package hooks (HO)]%
\documentclass{ltxdoc}
\usepackage{holtxdoc}[2011/11/22]
\usepackage{caption}
\usepackage{array}
\setlength{\extrarowheight}{1pt}
\renewcommand*{\arraystretch}{1.1}
\begin{document}
  \DocInput{hopatch.dtx}%
\end{document}
%</driver>
% \fi
%
% \CheckSum{466}
%
% \CharacterTable
%  {Upper-case    \A\B\C\D\E\F\G\H\I\J\K\L\M\N\O\P\Q\R\S\T\U\V\W\X\Y\Z
%   Lower-case    \a\b\c\d\e\f\g\h\i\j\k\l\m\n\o\p\q\r\s\t\u\v\w\x\y\z
%   Digits        \0\1\2\3\4\5\6\7\8\9
%   Exclamation   \!     Double quote  \"     Hash (number) \#
%   Dollar        \$     Percent       \%     Ampersand     \&
%   Acute accent  \'     Left paren    \(     Right paren   \)
%   Asterisk      \*     Plus          \+     Comma         \,
%   Minus         \-     Point         \.     Solidus       \/
%   Colon         \:     Semicolon     \;     Less than     \<
%   Equals        \=     Greater than  \>     Question mark \?
%   Commercial at \@     Left bracket  \[     Backslash     \\
%   Right bracket \]     Circumflex    \^     Underscore    \_
%   Grave accent  \`     Left brace    \{     Vertical bar  \|
%   Right brace   \}     Tilde         \~}
%
% \GetFileInfo{hopatch.drv}
%
% \title{The \xpackage{hopatch} package}
% \date{2012/05/28 v1.2}
% \author{Heiko Oberdiek\\\xemail{heiko.oberdiek at googlemail.com}}
%
% \maketitle
%
% \begin{abstract}
% This packages provides a wrapper to various package hooks
% provided by other packages or classes, but does not define
% own hooks.
% \end{abstract}
%
% \tableofcontents
%
% \section{Documentation}
%
% Sometimes I want to add code right after a package
% has been loaded. Examples are bug fixes, adaptations,
% or added features as needed by package \xpackage{hyperref},
% for instance.
%
% Unhappily \hologo{LaTeX} does not provide this kind of
% hook. \cs{AtEndOfPackage} can be used inside the package
% only, because \hologo{LaTeX} clears the hook right before
% it loads the package.
%
% However, there are already many packages and classes
% that provide hooks that are executed after the package
% is loaded, see table~\ref{tab:hooks}.
%
% \begin{table}
% \caption{After package hooking}
% \label{tab:hooks}
% \begin{tabular}{ll}
% \bfseries Macro & \bfseries Provider\\
% \hline
% \cs{AfterPackage} & package \xpackage{scrlfile} \cite{scrlfile}\\
% \cs{AtEndOfPackageFile} & package \xpackage{filehook} \cite{filehook}\\
% \cs{AtEndPackage} & class \xclass{memoir} \cite{memoir}
% \end{tabular}
% \end{table}
%
% Package \xpackage{hopatch} can be used without the packages
% of table~\ref{tab:hooks}. But for an early executing right
% after a package is loaded, one of the following class or
% packages should be loaded before using \cs{hopatch@AfterPackage}:
% \begin{itemize}
%  \item package \xpackage{filehook}
%  \item package \xpackage{scrlfile}
%  \item class \xclass{memoir}
% \end{itemize}
%
% \begin{table}
% \caption{After begin document hooking}
% \begin{tabular}{ll}
% \bfseries Macro & \bfseries Provider\\
% \hline
% \cs{AtBeginDocument} & \hologo{LaTeX}'s kernel\\
% \cs{AtEndPreamble} & package \xpackage{etoolbox} \cite{etoolbox}\\
% \cs{AfterEndPreamble} & package \xpackage{etoolbox}
% \end{tabular}
% \end{table}
%
% Therefore I skip writing a new package for hooking into
% \hologo{LaTeX}'s package management and use this package
% to provide a wrapper to patch a package after it is loaded.
%
% \begin{declcs}{hopatch@AfterPackage} \M{package} \M{patch code}
% \end{declcs}
% If the package is already loaded, the \meta{patch code} is executed
% immediately.
% Otherwise the \meta{patch code} is stored in a command and tried
% at later locations until the package is available.
%
% The patch is tried in the following order:
% \begin{enumerate}
% \item If the package is already loaded, the patch is
%   applied immediately. Further locations are not tried.
% \item \cs{AtEndPackage}, provided by class \xclass{memoir} \cite{memoir},
%   and \cs{AfterPackage}, provided by package \xpackage{scrlfile} \cite{scrlfile},
%   are called right after the package file is input before
%   the hook of \hologo{LaTeX}'s \cs{AtEndOfPackage}.
% \item \cs{AtEndOfPackageFile}, provided by package \xpackage{filehook} \cite{filehook},
%   is called after the package is loaded and after the hook of
%   \hologo{LaTeX}'s \cs{AtEndOfPackage}.
% \item \cs{AtEndPreamble}, provided by package \xpackage{etoolbox} \cite{etoolbox},
%   is called at the beginning of |\begin{document}| before the hook
%   of \hologo{LaTeX}'s \cs{AtBeginDocument}.
% \item \cs{AtBeginDocument}, provided by \hologo{LaTeX}.
% \item \cs{AfterEndDocument}, provided by package \xpackage{etoolbox} \cite{etoolbox},
%   is called at the very end of |\begin{document}|. Preamble commands are
%   already forbidden there.
% \end{enumerate}
% Because of the various locations the patch code is restricted to
% limitations:
% \begin{itemize}
% \item
% Preamble commands, see \hologo{LaTeX}'s \cs{@onlypreamble}
% throw an error if used after |\begin{document}|. This is already
% the case for \cs{AfterEndDocument}. Therefore preamble commands
% are forbidden in the patching code.
% There are four exceptions \cs{@ifpackageloaded}, \cs{@ifclassloaded},
% \cs{@ifpackagelater} and \cs{@ifclasslater}. They are redefined
% during \cs{AfterEndDocument} using the counterparts of
% package \xpackage{ltxcmds} \cite{ltxcmds}.
% \item
% \cs{AfterPackage} of package \xpackage{scrlfile} and \cs{AtEndPackage}
% of class \xclass{memoir} call the hook before \hologo{LaTeX}'s
% \cs{AtEndOfPackage}.
% \end{itemize}
%
% \StopEventually{
% }
%
% \section{Implementation}
%
%    \begin{macrocode}
%<*package>
%    \end{macrocode}
%
% \subsection{Catcodes and package identification}
%
%    \begin{macrocode}
\begingroup\catcode61\catcode48\catcode32=10\relax%
  \catcode13=5 % ^^M
  \endlinechar=13 %
  \catcode123=1 % {
  \catcode125=2 % }
  \catcode64=11 % @
  \def\x{\endgroup
    \expandafter\edef\csname HOpatch@AtEnd\endcsname{%
      \endlinechar=\the\endlinechar\relax
      \catcode13=\the\catcode13\relax
      \catcode32=\the\catcode32\relax
      \catcode35=\the\catcode35\relax
      \catcode61=\the\catcode61\relax
      \catcode64=\the\catcode64\relax
      \catcode123=\the\catcode123\relax
      \catcode125=\the\catcode125\relax
    }%
  }%
\x\catcode61\catcode48\catcode32=10\relax%
\catcode13=5 % ^^M
\endlinechar=13 %
\catcode35=6 % #
\catcode64=11 % @
\catcode123=1 % {
\catcode125=2 % }
\def\TMP@EnsureCode#1#2{%
  \edef\HOpatch@AtEnd{%
    \HOpatch@AtEnd
    \catcode#1=\the\catcode#1\relax
  }%
  \catcode#1=#2\relax
}
\TMP@EnsureCode{40}{12}% (
\TMP@EnsureCode{41}{12}% )
\TMP@EnsureCode{43}{12}% +
\TMP@EnsureCode{46}{12}% .
\TMP@EnsureCode{47}{12}% /
\TMP@EnsureCode{91}{12}% [
\TMP@EnsureCode{93}{12}% ]
\edef\HOpatch@AtEnd{\HOpatch@AtEnd\noexpand\endinput}
%    \end{macrocode}
%
%    Package identification.
%    \begin{macrocode}
\NeedsTeXFormat{LaTeX2e}
\ProvidesPackage{hopatch}%
  [2012/05/28 v1.2 Wrapper for package hooks (HO)]
%    \end{macrocode}
%
% \subsection{Resources}
%
%    \begin{macrocode}
\begingroup\expandafter\expandafter\expandafter\endgroup
\expandafter\ifx\csname RequirePackage\endcsname\relax
  \def\TMP@RequirePackage#1[#2]{%
    \begingroup\expandafter\expandafter\expandafter\endgroup
    \expandafter\ifx\csname ver@#1.sty\endcsname\relax
      \input #1.sty\relax
    \fi
  }%
  \TMP@RequirePackage{ltxcmds}[2010/12/12]%
\else
  \RequirePackage{ltxcmds}[2010/12/12]%
\fi
%    \end{macrocode}
%
%    \begin{macro}{\HOpatch@counter}
%    \begin{macrocode}
\def\HOpatch@counter{0}%
%    \end{macrocode}
%    \end{macro}
%    \begin{macro}{\HOpatch@StepCounter}
%    \begin{macrocode}
\ltx@ifundefined{numexpr}{%
  \def\HOpatch@StepCounter{%
    \begingroup
      \count@\HOpatch@counter\relax
      \advance\count@\ltx@one\relax
    \edef\x{\endgroup
      \noexpand\def\noexpand\HOpatch@counter{\the\count@}%
    }%
    \x
  }%
}{%
  \def\HOpatch@StepCounter{%
    \edef\HOpatch@counter{%
      \the\numexpr\HOpatch@counter+\ltx@one\relax
    }%
  }%
}
%    \end{macrocode}
%    \end{macro}
%
%    \begin{macro}{\HOpatch@list}
%    \begin{macrocode}
\def\HOpatch@list{}
%    \end{macrocode}
%    \end{macro}
%
%    \begin{macro}{\HOpatch@Add}
%    \begin{macrocode}
\def\HOpatch@Add{%
  \ltx@LocalAppendToMacro\HOpatch@list
}
%    \end{macrocode}
%    \end{macro}
%
% \subsection{Package patching}
%
%    \begin{macro}{\hopatch@AfterPackage}
%    \begin{macrocode}
\def\hopatch@AfterPackage#1{%
  \ltx@ifpackageloaded{#1}{%
    \ltx@firstofone
  }{%
    \HOpatch@AfterPackage{#1}%
  }%
}
%    \end{macrocode}
%    \end{macro}
%    \begin{macro}{\HOpatch@AfterPackage}
%    \begin{macrocode}
\def\HOpatch@AfterPackage#1{%
  \edef\HOpatch@temp{#1}%
  \HOpatch@StepCounter
  \expandafter\HOpatch@@AfterPackage
  \csname HOpatch@\HOpatch@counter\expandafter\endcsname{%
    \HOpatch@temp
  }%
}
%    \end{macrocode}
%    \end{macro}
%
%    \begin{macro}{\HOpatch@@AfterPackage}
%    \begin{macrocode}
\def\HOpatch@@AfterPackage#1#2#3{%
  \begingroup
    \toks@{#3}%
    \xdef\HOpatch@gtemp{%
      \noexpand\ltx@ifpackageloaded{#2}{%
        \noexpand\let\noexpand#1\noexpand\relax
        \the\toks@
      }{}%
    }%
  \endgroup
  \let#1\HOpatch@gtemp
  \HOpatch@Add#1%
  \HOpatch@Try{AfterPackage}{#2}#1%
  \HOpatch@Try{AtEndPackage}{#2}#1%
  \HOpatch@Try{AtEndOfPackageFile}{#2}#1%
}
%    \end{macrocode}
%    \end{macro}
%
%    \begin{macro}{\HOpatch@Try}
%    \begin{macrocode}
\def\HOpatch@Try#1#2#3{%
  \ltx@ifundefined{#1}{}{%
    \csname #1\endcsname{#2}{#3}%
  }%
}
%    \end{macrocode}
%    \end{macro}
%
%    \begin{macrocode}
\AtBeginDocument{\HOpatch@list}
\ltx@ifundefined{AtEndPreamble}{}{%
  \ltx@ifundefined{@endpreamblehook}{}{%
    \AtEndPreamble{\HOpatch@list}%
  }%
}
\ltx@ifundefined{AfterEndPreamble}{}{%
  \ltx@ifundefined{@afterendpreamblehook}{}{%
    \AfterEndPreamble{%
      \let\HOpatch@OrgIfPackageLoaded\@ifpackageloaded
      \let\HOpatch@OrgIfPackageLater\@ifpackagelater
      \let\HOpatch@OrgIfClassLoaded\@ifclassloaded
      \let\HOpatch@OrgIfClassLater\@ifclasslater
      \let\@ifpackageloaded\ltx@ifpackageloaded
      \let\@ifpackagelater\ltx@ifpackagelater
      \let\@ifclassloaded\ltx@ifclassloaded
      \let\@ifclasslater\ltx@ifclasslater
      \HOpatch@list
      \let\@ifpackageloaded\HOpatch@OrgIfPackageLoaded
      \let\@ifpackagelater\HOpatch@OrgIfPackageLater
      \let\@ifclassloaded\HOpatch@OrgIfClassLoaded
      \let\@ifclasslater\HOpatch@OrgIfClassLater
    }%
  }%
}
%    \end{macrocode}
%
%    \begin{macrocode}
\HOpatch@AtEnd%
%    \end{macrocode}
%    \begin{macrocode}
%</package>
%    \end{macrocode}
%
% \section{Test}
%
%    \begin{macrocode}
%<*test1>
\def\LoadCommand{\RequirePackage{hopatch}[2012/05/28]}
%</test1>
%    \end{macrocode}
% \subsection{Catcode checks for loading}
%
%    \begin{macrocode}
%<*test1>
%    \end{macrocode}
%    \begin{macrocode}
\catcode`\{=1 %
\catcode`\}=2 %
\catcode`\#=6 %
\catcode`\@=11 %
\expandafter\ifx\csname count@\endcsname\relax
  \countdef\count@=255 %
\fi
\expandafter\ifx\csname @gobble\endcsname\relax
  \long\def\@gobble#1{}%
\fi
\expandafter\ifx\csname @firstofone\endcsname\relax
  \long\def\@firstofone#1{#1}%
\fi
\expandafter\ifx\csname loop\endcsname\relax
  \expandafter\@firstofone
\else
  \expandafter\@gobble
\fi
{%
  \def\loop#1\repeat{%
    \def\body{#1}%
    \iterate
  }%
  \def\iterate{%
    \body
      \let\next\iterate
    \else
      \let\next\relax
    \fi
    \next
  }%
  \let\repeat=\fi
}%
\def\RestoreCatcodes{}
\count@=0 %
\loop
  \edef\RestoreCatcodes{%
    \RestoreCatcodes
    \catcode\the\count@=\the\catcode\count@\relax
  }%
\ifnum\count@<255 %
  \advance\count@ 1 %
\repeat

\def\RangeCatcodeInvalid#1#2{%
  \count@=#1\relax
  \loop
    \catcode\count@=15 %
  \ifnum\count@<#2\relax
    \advance\count@ 1 %
  \repeat
}
\def\RangeCatcodeCheck#1#2#3{%
  \count@=#1\relax
  \loop
    \ifnum#3=\catcode\count@
    \else
      \errmessage{%
        Character \the\count@\space
        with wrong catcode \the\catcode\count@\space
        instead of \number#3%
      }%
    \fi
  \ifnum\count@<#2\relax
    \advance\count@ 1 %
  \repeat
}
\def\space{ }
\expandafter\ifx\csname LoadCommand\endcsname\relax
  \def\LoadCommand{\input hopatch.sty\relax}%
\fi
\def\Test{%
  \RangeCatcodeInvalid{0}{47}%
  \RangeCatcodeInvalid{58}{64}%
  \RangeCatcodeInvalid{91}{96}%
  \RangeCatcodeInvalid{123}{255}%
  \catcode`\@=12 %
  \catcode`\\=0 %
  \catcode`\%=14 %
  \LoadCommand
  \RangeCatcodeCheck{0}{36}{15}%
  \RangeCatcodeCheck{37}{37}{14}%
  \RangeCatcodeCheck{38}{47}{15}%
  \RangeCatcodeCheck{48}{57}{12}%
  \RangeCatcodeCheck{58}{63}{15}%
  \RangeCatcodeCheck{64}{64}{12}%
  \RangeCatcodeCheck{65}{90}{11}%
  \RangeCatcodeCheck{91}{91}{15}%
  \RangeCatcodeCheck{92}{92}{0}%
  \RangeCatcodeCheck{93}{96}{15}%
  \RangeCatcodeCheck{97}{122}{11}%
  \RangeCatcodeCheck{123}{255}{15}%
  \RestoreCatcodes
}
\Test
\csname @@end\endcsname
\end
%    \end{macrocode}
%    \begin{macrocode}
%</test1>
%    \end{macrocode}
%
%    \begin{macrocode}
%<*test2>
\NeedsTeXFormat{LaTeX2e}
\providecommand\variant{0}
\RequirePackage{filecontents}
\begin{filecontents}{foo.sty}
\ProvidesPackage{foo}
\def\msg#{\immediate\write16}
\def\foo#1{%
  \msg{\fooformat{#1}}%
}
\def\fooformat#1{[#1]}% hash-ok
\foo{* Executing foo at package loading}
\end{filecontents}

\ifnum\variant=1 %
  \documentclass{memoir}%
\else
  \documentclass{article}%
\fi

\ifcase\variant\relax
\or % 1
\or % 2
  \usepackage{etoolbox}%
\or % 3
  \usepackage{scrlfile}%
\or % 4
  \usepackage{filehook}%
\fi

\AtBeginDocument{\foo{* AtBeginDocument before hopatch}}
\usepackage{hopatch}
\AtBeginDocument{\foo{* AtBeginDocument after hopatch}}

\makeatletter
\hopatch@AfterPackage{foo}{%
  \def\fooformat#1{<<#1>>}%
}
\makeatother

\AtBeginDocument{\foo{* AtBeginDocument before foo}}
\usepackage{foo}
\AtBeginDocument{\foo{* AtBeginDocument after foo}}

\foo{* Executing in preamble}

\begin{document}
\foo{* Executing in document}
\end{document}
%</test2>
%    \end{macrocode}
%
% \section{Installation}
%
% \subsection{Download}
%
% \paragraph{Package.} This package is available on
% CTAN\footnote{\url{ftp://ftp.ctan.org/tex-archive/}}:
% \begin{description}
% \item[\CTAN{macros/latex/contrib/oberdiek/hopatch.dtx}] The source file.
% \item[\CTAN{macros/latex/contrib/oberdiek/hopatch.pdf}] Documentation.
% \end{description}
%
%
% \paragraph{Bundle.} All the packages of the bundle `oberdiek'
% are also available in a TDS compliant ZIP archive. There
% the packages are already unpacked and the documentation files
% are generated. The files and directories obey the TDS standard.
% \begin{description}
% \item[\CTAN{install/macros/latex/contrib/oberdiek.tds.zip}]
% \end{description}
% \emph{TDS} refers to the standard ``A Directory Structure
% for \TeX\ Files'' (\CTAN{tds/tds.pdf}). Directories
% with \xfile{texmf} in their name are usually organized this way.
%
% \subsection{Bundle installation}
%
% \paragraph{Unpacking.} Unpack the \xfile{oberdiek.tds.zip} in the
% TDS tree (also known as \xfile{texmf} tree) of your choice.
% Example (linux):
% \begin{quote}
%   |unzip oberdiek.tds.zip -d ~/texmf|
% \end{quote}
%
% \paragraph{Script installation.}
% Check the directory \xfile{TDS:scripts/oberdiek/} for
% scripts that need further installation steps.
% Package \xpackage{attachfile2} comes with the Perl script
% \xfile{pdfatfi.pl} that should be installed in such a way
% that it can be called as \texttt{pdfatfi}.
% Example (linux):
% \begin{quote}
%   |chmod +x scripts/oberdiek/pdfatfi.pl|\\
%   |cp scripts/oberdiek/pdfatfi.pl /usr/local/bin/|
% \end{quote}
%
% \subsection{Package installation}
%
% \paragraph{Unpacking.} The \xfile{.dtx} file is a self-extracting
% \docstrip\ archive. The files are extracted by running the
% \xfile{.dtx} through \plainTeX:
% \begin{quote}
%   \verb|tex hopatch.dtx|
% \end{quote}
%
% \paragraph{TDS.} Now the different files must be moved into
% the different directories in your installation TDS tree
% (also known as \xfile{texmf} tree):
% \begin{quote}
% \def\t{^^A
% \begin{tabular}{@{}>{\ttfamily}l@{ $\rightarrow$ }>{\ttfamily}l@{}}
%   hopatch.sty & tex/latex/oberdiek/hopatch.sty\\
%   hopatch.pdf & doc/latex/oberdiek/hopatch.pdf\\
%   test/hopatch-test1.tex & doc/latex/oberdiek/test/hopatch-test1.tex\\
%   test/hopatch-test2.tex & doc/latex/oberdiek/test/hopatch-test2.tex\\
%   hopatch.dtx & source/latex/oberdiek/hopatch.dtx\\
% \end{tabular}^^A
% }^^A
% \sbox0{\t}^^A
% \ifdim\wd0>\linewidth
%   \begingroup
%     \advance\linewidth by\leftmargin
%     \advance\linewidth by\rightmargin
%   \edef\x{\endgroup
%     \def\noexpand\lw{\the\linewidth}^^A
%   }\x
%   \def\lwbox{^^A
%     \leavevmode
%     \hbox to \linewidth{^^A
%       \kern-\leftmargin\relax
%       \hss
%       \usebox0
%       \hss
%       \kern-\rightmargin\relax
%     }^^A
%   }^^A
%   \ifdim\wd0>\lw
%     \sbox0{\small\t}^^A
%     \ifdim\wd0>\linewidth
%       \ifdim\wd0>\lw
%         \sbox0{\footnotesize\t}^^A
%         \ifdim\wd0>\linewidth
%           \ifdim\wd0>\lw
%             \sbox0{\scriptsize\t}^^A
%             \ifdim\wd0>\linewidth
%               \ifdim\wd0>\lw
%                 \sbox0{\tiny\t}^^A
%                 \ifdim\wd0>\linewidth
%                   \lwbox
%                 \else
%                   \usebox0
%                 \fi
%               \else
%                 \lwbox
%               \fi
%             \else
%               \usebox0
%             \fi
%           \else
%             \lwbox
%           \fi
%         \else
%           \usebox0
%         \fi
%       \else
%         \lwbox
%       \fi
%     \else
%       \usebox0
%     \fi
%   \else
%     \lwbox
%   \fi
% \else
%   \usebox0
% \fi
% \end{quote}
% If you have a \xfile{docstrip.cfg} that configures and enables \docstrip's
% TDS installing feature, then some files can already be in the right
% place, see the documentation of \docstrip.
%
% \subsection{Refresh file name databases}
%
% If your \TeX~distribution
% (\teTeX, \mikTeX, \dots) relies on file name databases, you must refresh
% these. For example, \teTeX\ users run \verb|texhash| or
% \verb|mktexlsr|.
%
% \subsection{Some details for the interested}
%
% \paragraph{Attached source.}
%
% The PDF documentation on CTAN also includes the
% \xfile{.dtx} source file. It can be extracted by
% AcrobatReader 6 or higher. Another option is \textsf{pdftk},
% e.g. unpack the file into the current directory:
% \begin{quote}
%   \verb|pdftk hopatch.pdf unpack_files output .|
% \end{quote}
%
% \paragraph{Unpacking with \LaTeX.}
% The \xfile{.dtx} chooses its action depending on the format:
% \begin{description}
% \item[\plainTeX:] Run \docstrip\ and extract the files.
% \item[\LaTeX:] Generate the documentation.
% \end{description}
% If you insist on using \LaTeX\ for \docstrip\ (really,
% \docstrip\ does not need \LaTeX), then inform the autodetect routine
% about your intention:
% \begin{quote}
%   \verb|latex \let\install=y\input{hopatch.dtx}|
% \end{quote}
% Do not forget to quote the argument according to the demands
% of your shell.
%
% \paragraph{Generating the documentation.}
% You can use both the \xfile{.dtx} or the \xfile{.drv} to generate
% the documentation. The process can be configured by the
% configuration file \xfile{ltxdoc.cfg}. For instance, put this
% line into this file, if you want to have A4 as paper format:
% \begin{quote}
%   \verb|\PassOptionsToClass{a4paper}{article}|
% \end{quote}
% An example follows how to generate the
% documentation with pdf\LaTeX:
% \begin{quote}
%\begin{verbatim}
%pdflatex hopatch.dtx
%makeindex -s gind.ist hopatch.idx
%pdflatex hopatch.dtx
%makeindex -s gind.ist hopatch.idx
%pdflatex hopatch.dtx
%\end{verbatim}
% \end{quote}
%
% \section{Catalogue}
%
% The following XML file can be used as source for the
% \href{http://mirror.ctan.org/help/Catalogue/catalogue.html}{\TeX\ Catalogue}.
% The elements \texttt{caption} and \texttt{description} are imported
% from the original XML file from the Catalogue.
% The name of the XML file in the Catalogue is \xfile{hopatch.xml}.
%    \begin{macrocode}
%<*catalogue>
<?xml version='1.0' encoding='us-ascii'?>
<!DOCTYPE entry SYSTEM 'catalogue.dtd'>
<entry datestamp='$Date$' modifier='$Author$' id='hopatch'>
  <name>hopatch</name>
  <caption>Load patches for packages.</caption>
  <authorref id='auth:oberdiek'/>
  <copyright owner='Heiko Oberdiek' year='2011,2012'/>
  <license type='lppl1.3'/>
  <version number='1.2'/>
  <description>
    The hopatch package provides a command with which the user may
    register a piece of patch code for a particular package.  Hopatch
    will apply the patch immediately, if the relevant package has
    already been loaded; otherwise it will store the patch until the
    package appears.
    <p/>
    The package is part of the <xref refid='oberdiek'>oberdiek</xref> bundle.
  </description>
  <documentation details='Package documentation'
      href='ctan:/macros/latex/contrib/oberdiek/hopatch.pdf'/>
  <ctan file='true' path='/macros/latex/contrib/oberdiek/hopatch.dtx'/>
  <miktex location='oberdiek'/>
  <texlive location='oberdiek'/>
  <install path='/macros/latex/contrib/oberdiek/oberdiek.tds.zip'/>
</entry>
%</catalogue>
%    \end{macrocode}
%
% \begin{thebibliography}{9}
% \raggedright
%
% \bibitem{etoolbox}
% Philipp Lehman:
% \textit{The etoolbox Package}
% 2011-01-03.
% \CTAN{macros/latex/contrib/etoolbox/etoolbox.pdf}
%
% \bibitem{filehook}
% Martin Scharrer:
% \textit{The filehook Package};
% 2011-01-09.
% \CTAN{macros/latex/contrib/filehook/filehook.pdf}
%
% \bibitem{ltxcmds}
% Heiko Oberdiek:
% \textit{The ltxcmds Package};
% 2010-12-12.
% \CTAN{macros/latex/contrib/oberdiek/ltxcmds.pdf}
%
% \bibitem{memoir}
% Peter Wilson, Lars Madsen:
% \textit{The Memoir Class for Configurable Typesetting, User Guide};
% 2010.
% \CTAN{macros/latex/contrib/memoir/memman.pdf}
%
% \bibitem{scrlfile}
% Markus Kohm, Jens-Uwe Morawski:
% \textit{The Guide KOMA-Script};
% 2011-01-20.
% \CTAN{macros/latex/contrib/koma-script/scrguien.pdf}
%
% \end{thebibliography}
%
% \begin{History}
%   \begin{Version}{2011/01/30 v1.0}
%   \item
%     First public version.
%   \end{Version}
%   \begin{Version}{2011/06/24 v1.1}
%   \item
%     Fix the use of \cs{AtEndPreamble} and \cs{AfterEndPreamble}.
%     They are redefined by package \xpackage{etoolbox} after their
%     hooks are used and generate an error message then.
%   \end{Version}
%   \begin{Version}{2012/05/28 v1.2}
%   \item
%     Fix for use without \hologo{eTeX}
%     (thanks Gordon Lee).
%   \end{Version}
% \end{History}
%
% \PrintIndex
%
% \Finale
\endinput
|
% \end{quote}
% Do not forget to quote the argument according to the demands
% of your shell.
%
% \paragraph{Generating the documentation.}
% You can use both the \xfile{.dtx} or the \xfile{.drv} to generate
% the documentation. The process can be configured by the
% configuration file \xfile{ltxdoc.cfg}. For instance, put this
% line into this file, if you want to have A4 as paper format:
% \begin{quote}
%   \verb|\PassOptionsToClass{a4paper}{article}|
% \end{quote}
% An example follows how to generate the
% documentation with pdf\LaTeX:
% \begin{quote}
%\begin{verbatim}
%pdflatex hopatch.dtx
%makeindex -s gind.ist hopatch.idx
%pdflatex hopatch.dtx
%makeindex -s gind.ist hopatch.idx
%pdflatex hopatch.dtx
%\end{verbatim}
% \end{quote}
%
% \section{Catalogue}
%
% The following XML file can be used as source for the
% \href{http://mirror.ctan.org/help/Catalogue/catalogue.html}{\TeX\ Catalogue}.
% The elements \texttt{caption} and \texttt{description} are imported
% from the original XML file from the Catalogue.
% The name of the XML file in the Catalogue is \xfile{hopatch.xml}.
%    \begin{macrocode}
%<*catalogue>
<?xml version='1.0' encoding='us-ascii'?>
<!DOCTYPE entry SYSTEM 'catalogue.dtd'>
<entry datestamp='$Date$' modifier='$Author$' id='hopatch'>
  <name>hopatch</name>
  <caption>Load patches for packages.</caption>
  <authorref id='auth:oberdiek'/>
  <copyright owner='Heiko Oberdiek' year='2011,2012'/>
  <license type='lppl1.3'/>
  <version number='1.2'/>
  <description>
    The hopatch package provides a command with which the user may
    register a piece of patch code for a particular package.  Hopatch
    will apply the patch immediately, if the relevant package has
    already been loaded; otherwise it will store the patch until the
    package appears.
    <p/>
    The package is part of the <xref refid='oberdiek'>oberdiek</xref> bundle.
  </description>
  <documentation details='Package documentation'
      href='ctan:/macros/latex/contrib/oberdiek/hopatch.pdf'/>
  <ctan file='true' path='/macros/latex/contrib/oberdiek/hopatch.dtx'/>
  <miktex location='oberdiek'/>
  <texlive location='oberdiek'/>
  <install path='/macros/latex/contrib/oberdiek/oberdiek.tds.zip'/>
</entry>
%</catalogue>
%    \end{macrocode}
%
% \begin{thebibliography}{9}
% \raggedright
%
% \bibitem{etoolbox}
% Philipp Lehman:
% \textit{The etoolbox Package}
% 2011-01-03.
% \CTAN{macros/latex/contrib/etoolbox/etoolbox.pdf}
%
% \bibitem{filehook}
% Martin Scharrer:
% \textit{The filehook Package};
% 2011-01-09.
% \CTAN{macros/latex/contrib/filehook/filehook.pdf}
%
% \bibitem{ltxcmds}
% Heiko Oberdiek:
% \textit{The ltxcmds Package};
% 2010-12-12.
% \CTAN{macros/latex/contrib/oberdiek/ltxcmds.pdf}
%
% \bibitem{memoir}
% Peter Wilson, Lars Madsen:
% \textit{The Memoir Class for Configurable Typesetting, User Guide};
% 2010.
% \CTAN{macros/latex/contrib/memoir/memman.pdf}
%
% \bibitem{scrlfile}
% Markus Kohm, Jens-Uwe Morawski:
% \textit{The Guide KOMA-Script};
% 2011-01-20.
% \CTAN{macros/latex/contrib/koma-script/scrguien.pdf}
%
% \end{thebibliography}
%
% \begin{History}
%   \begin{Version}{2011/01/30 v1.0}
%   \item
%     First public version.
%   \end{Version}
%   \begin{Version}{2011/06/24 v1.1}
%   \item
%     Fix the use of \cs{AtEndPreamble} and \cs{AfterEndPreamble}.
%     They are redefined by package \xpackage{etoolbox} after their
%     hooks are used and generate an error message then.
%   \end{Version}
%   \begin{Version}{2012/05/28 v1.2}
%   \item
%     Fix for use without \hologo{eTeX}
%     (thanks Gordon Lee).
%   \end{Version}
% \end{History}
%
% \PrintIndex
%
% \Finale
\endinput
|
% \end{quote}
% Do not forget to quote the argument according to the demands
% of your shell.
%
% \paragraph{Generating the documentation.}
% You can use both the \xfile{.dtx} or the \xfile{.drv} to generate
% the documentation. The process can be configured by the
% configuration file \xfile{ltxdoc.cfg}. For instance, put this
% line into this file, if you want to have A4 as paper format:
% \begin{quote}
%   \verb|\PassOptionsToClass{a4paper}{article}|
% \end{quote}
% An example follows how to generate the
% documentation with pdf\LaTeX:
% \begin{quote}
%\begin{verbatim}
%pdflatex hopatch.dtx
%makeindex -s gind.ist hopatch.idx
%pdflatex hopatch.dtx
%makeindex -s gind.ist hopatch.idx
%pdflatex hopatch.dtx
%\end{verbatim}
% \end{quote}
%
% \section{Catalogue}
%
% The following XML file can be used as source for the
% \href{http://mirror.ctan.org/help/Catalogue/catalogue.html}{\TeX\ Catalogue}.
% The elements \texttt{caption} and \texttt{description} are imported
% from the original XML file from the Catalogue.
% The name of the XML file in the Catalogue is \xfile{hopatch.xml}.
%    \begin{macrocode}
%<*catalogue>
<?xml version='1.0' encoding='us-ascii'?>
<!DOCTYPE entry SYSTEM 'catalogue.dtd'>
<entry datestamp='$Date$' modifier='$Author$' id='hopatch'>
  <name>hopatch</name>
  <caption>Load patches for packages.</caption>
  <authorref id='auth:oberdiek'/>
  <copyright owner='Heiko Oberdiek' year='2011,2012'/>
  <license type='lppl1.3'/>
  <version number='1.2'/>
  <description>
    The hopatch package provides a command with which the user may
    register a piece of patch code for a particular package.  Hopatch
    will apply the patch immediately, if the relevant package has
    already been loaded; otherwise it will store the patch until the
    package appears.
    <p/>
    The package is part of the <xref refid='oberdiek'>oberdiek</xref> bundle.
  </description>
  <documentation details='Package documentation'
      href='ctan:/macros/latex/contrib/oberdiek/hopatch.pdf'/>
  <ctan file='true' path='/macros/latex/contrib/oberdiek/hopatch.dtx'/>
  <miktex location='oberdiek'/>
  <texlive location='oberdiek'/>
  <install path='/macros/latex/contrib/oberdiek/oberdiek.tds.zip'/>
</entry>
%</catalogue>
%    \end{macrocode}
%
% \begin{thebibliography}{9}
% \raggedright
%
% \bibitem{etoolbox}
% Philipp Lehman:
% \textit{The etoolbox Package}
% 2011-01-03.
% \CTAN{macros/latex/contrib/etoolbox/etoolbox.pdf}
%
% \bibitem{filehook}
% Martin Scharrer:
% \textit{The filehook Package};
% 2011-01-09.
% \CTAN{macros/latex/contrib/filehook/filehook.pdf}
%
% \bibitem{ltxcmds}
% Heiko Oberdiek:
% \textit{The ltxcmds Package};
% 2010-12-12.
% \CTAN{macros/latex/contrib/oberdiek/ltxcmds.pdf}
%
% \bibitem{memoir}
% Peter Wilson, Lars Madsen:
% \textit{The Memoir Class for Configurable Typesetting, User Guide};
% 2010.
% \CTAN{macros/latex/contrib/memoir/memman.pdf}
%
% \bibitem{scrlfile}
% Markus Kohm, Jens-Uwe Morawski:
% \textit{The Guide KOMA-Script};
% 2011-01-20.
% \CTAN{macros/latex/contrib/koma-script/scrguien.pdf}
%
% \end{thebibliography}
%
% \begin{History}
%   \begin{Version}{2011/01/30 v1.0}
%   \item
%     First public version.
%   \end{Version}
%   \begin{Version}{2011/06/24 v1.1}
%   \item
%     Fix the use of \cs{AtEndPreamble} and \cs{AfterEndPreamble}.
%     They are redefined by package \xpackage{etoolbox} after their
%     hooks are used and generate an error message then.
%   \end{Version}
%   \begin{Version}{2012/05/28 v1.2}
%   \item
%     Fix for use without \hologo{eTeX}
%     (thanks Gordon Lee).
%   \end{Version}
% \end{History}
%
% \PrintIndex
%
% \Finale
\endinput

%        (quote the arguments according to the demands of your shell)
%
% Documentation:
%    (a) If hopatch.drv is present:
%           latex hopatch.drv
%    (b) Without hopatch.drv:
%           latex hopatch.dtx; ...
%    The class ltxdoc loads the configuration file ltxdoc.cfg
%    if available. Here you can specify further options, e.g.
%    use A4 as paper format:
%       \PassOptionsToClass{a4paper}{article}
%
%    Programm calls to get the documentation (example):
%       pdflatex hopatch.dtx
%       makeindex -s gind.ist hopatch.idx
%       pdflatex hopatch.dtx
%       makeindex -s gind.ist hopatch.idx
%       pdflatex hopatch.dtx
%
% Installation:
%    TDS:tex/latex/oberdiek/hopatch.sty
%    TDS:doc/latex/oberdiek/hopatch.pdf
%    TDS:doc/latex/oberdiek/test/hopatch-test1.tex
%    TDS:doc/latex/oberdiek/test/hopatch-test2.tex
%    TDS:source/latex/oberdiek/hopatch.dtx
%
%<*ignore>
\begingroup
  \catcode123=1 %
  \catcode125=2 %
  \def\x{LaTeX2e}%
\expandafter\endgroup
\ifcase 0\ifx\install y1\fi\expandafter
         \ifx\csname processbatchFile\endcsname\relax\else1\fi
         \ifx\fmtname\x\else 1\fi\relax
\else\csname fi\endcsname
%</ignore>
%<*install>
\input docstrip.tex
\Msg{************************************************************************}
\Msg{* Installation}
\Msg{* Package: hopatch 2012/05/28 v1.2 Wrapper for package hooks (HO)}
\Msg{************************************************************************}

\keepsilent
\askforoverwritefalse

\let\MetaPrefix\relax
\preamble

This is a generated file.

Project: hopatch
Version: 2012/05/28 v1.2

Copyright (C) 2011, 2012 by
   Heiko Oberdiek <heiko.oberdiek at googlemail.com>

This work may be distributed and/or modified under the
conditions of the LaTeX Project Public License, either
version 1.3c of this license or (at your option) any later
version. This version of this license is in
   http://www.latex-project.org/lppl/lppl-1-3c.txt
and the latest version of this license is in
   http://www.latex-project.org/lppl.txt
and version 1.3 or later is part of all distributions of
LaTeX version 2005/12/01 or later.

This work has the LPPL maintenance status "maintained".

This Current Maintainer of this work is Heiko Oberdiek.

This work consists of the main source file hopatch.dtx
and the derived files
   hopatch.sty, hopatch.pdf, hopatch.ins, hopatch.drv,
   hopatch-test1.tex, hopatch-test2.tex.

\endpreamble
\let\MetaPrefix\DoubleperCent

\generate{%
  \file{hopatch.ins}{\from{hopatch.dtx}{install}}%
  \file{hopatch.drv}{\from{hopatch.dtx}{driver}}%
  \usedir{tex/latex/oberdiek}%
  \file{hopatch.sty}{\from{hopatch.dtx}{package}}%
  \usedir{doc/latex/oberdiek/test}%
  \file{hopatch-test1.tex}{\from{hopatch.dtx}{test1}}%
  \file{hopatch-test2.tex}{\from{hopatch.dtx}{test2}}%
  \nopreamble
  \nopostamble
  \usedir{source/latex/oberdiek/catalogue}%
  \file{hopatch.xml}{\from{hopatch.dtx}{catalogue}}%
}

\catcode32=13\relax% active space
\let =\space%
\Msg{************************************************************************}
\Msg{*}
\Msg{* To finish the installation you have to move the following}
\Msg{* file into a directory searched by TeX:}
\Msg{*}
\Msg{*     hopatch.sty}
\Msg{*}
\Msg{* To produce the documentation run the file `hopatch.drv'}
\Msg{* through LaTeX.}
\Msg{*}
\Msg{* Happy TeXing!}
\Msg{*}
\Msg{************************************************************************}

\endbatchfile
%</install>
%<*ignore>
\fi
%</ignore>
%<*driver>
\NeedsTeXFormat{LaTeX2e}
\ProvidesFile{hopatch.drv}%
  [2012/05/28 v1.2 Wrapper for package hooks (HO)]%
\documentclass{ltxdoc}
\usepackage{holtxdoc}[2011/11/22]
\usepackage{caption}
\usepackage{array}
\setlength{\extrarowheight}{1pt}
\renewcommand*{\arraystretch}{1.1}
\begin{document}
  \DocInput{hopatch.dtx}%
\end{document}
%</driver>
% \fi
%
% \CheckSum{466}
%
% \CharacterTable
%  {Upper-case    \A\B\C\D\E\F\G\H\I\J\K\L\M\N\O\P\Q\R\S\T\U\V\W\X\Y\Z
%   Lower-case    \a\b\c\d\e\f\g\h\i\j\k\l\m\n\o\p\q\r\s\t\u\v\w\x\y\z
%   Digits        \0\1\2\3\4\5\6\7\8\9
%   Exclamation   \!     Double quote  \"     Hash (number) \#
%   Dollar        \$     Percent       \%     Ampersand     \&
%   Acute accent  \'     Left paren    \(     Right paren   \)
%   Asterisk      \*     Plus          \+     Comma         \,
%   Minus         \-     Point         \.     Solidus       \/
%   Colon         \:     Semicolon     \;     Less than     \<
%   Equals        \=     Greater than  \>     Question mark \?
%   Commercial at \@     Left bracket  \[     Backslash     \\
%   Right bracket \]     Circumflex    \^     Underscore    \_
%   Grave accent  \`     Left brace    \{     Vertical bar  \|
%   Right brace   \}     Tilde         \~}
%
% \GetFileInfo{hopatch.drv}
%
% \title{The \xpackage{hopatch} package}
% \date{2012/05/28 v1.2}
% \author{Heiko Oberdiek\\\xemail{heiko.oberdiek at googlemail.com}}
%
% \maketitle
%
% \begin{abstract}
% This packages provides a wrapper to various package hooks
% provided by other packages or classes, but does not define
% own hooks.
% \end{abstract}
%
% \tableofcontents
%
% \section{Documentation}
%
% Sometimes I want to add code right after a package
% has been loaded. Examples are bug fixes, adaptations,
% or added features as needed by package \xpackage{hyperref},
% for instance.
%
% Unhappily \hologo{LaTeX} does not provide this kind of
% hook. \cs{AtEndOfPackage} can be used inside the package
% only, because \hologo{LaTeX} clears the hook right before
% it loads the package.
%
% However, there are already many packages and classes
% that provide hooks that are executed after the package
% is loaded, see table~\ref{tab:hooks}.
%
% \begin{table}
% \caption{After package hooking}
% \label{tab:hooks}
% \begin{tabular}{ll}
% \bfseries Macro & \bfseries Provider\\
% \hline
% \cs{AfterPackage} & package \xpackage{scrlfile} \cite{scrlfile}\\
% \cs{AtEndOfPackageFile} & package \xpackage{filehook} \cite{filehook}\\
% \cs{AtEndPackage} & class \xclass{memoir} \cite{memoir}
% \end{tabular}
% \end{table}
%
% Package \xpackage{hopatch} can be used without the packages
% of table~\ref{tab:hooks}. But for an early executing right
% after a package is loaded, one of the following class or
% packages should be loaded before using \cs{hopatch@AfterPackage}:
% \begin{itemize}
%  \item package \xpackage{filehook}
%  \item package \xpackage{scrlfile}
%  \item class \xclass{memoir}
% \end{itemize}
%
% \begin{table}
% \caption{After begin document hooking}
% \begin{tabular}{ll}
% \bfseries Macro & \bfseries Provider\\
% \hline
% \cs{AtBeginDocument} & \hologo{LaTeX}'s kernel\\
% \cs{AtEndPreamble} & package \xpackage{etoolbox} \cite{etoolbox}\\
% \cs{AfterEndPreamble} & package \xpackage{etoolbox}
% \end{tabular}
% \end{table}
%
% Therefore I skip writing a new package for hooking into
% \hologo{LaTeX}'s package management and use this package
% to provide a wrapper to patch a package after it is loaded.
%
% \begin{declcs}{hopatch@AfterPackage} \M{package} \M{patch code}
% \end{declcs}
% If the package is already loaded, the \meta{patch code} is executed
% immediately.
% Otherwise the \meta{patch code} is stored in a command and tried
% at later locations until the package is available.
%
% The patch is tried in the following order:
% \begin{enumerate}
% \item If the package is already loaded, the patch is
%   applied immediately. Further locations are not tried.
% \item \cs{AtEndPackage}, provided by class \xclass{memoir} \cite{memoir},
%   and \cs{AfterPackage}, provided by package \xpackage{scrlfile} \cite{scrlfile},
%   are called right after the package file is input before
%   the hook of \hologo{LaTeX}'s \cs{AtEndOfPackage}.
% \item \cs{AtEndOfPackageFile}, provided by package \xpackage{filehook} \cite{filehook},
%   is called after the package is loaded and after the hook of
%   \hologo{LaTeX}'s \cs{AtEndOfPackage}.
% \item \cs{AtEndPreamble}, provided by package \xpackage{etoolbox} \cite{etoolbox},
%   is called at the beginning of |\begin{document}| before the hook
%   of \hologo{LaTeX}'s \cs{AtBeginDocument}.
% \item \cs{AtBeginDocument}, provided by \hologo{LaTeX}.
% \item \cs{AfterEndDocument}, provided by package \xpackage{etoolbox} \cite{etoolbox},
%   is called at the very end of |\begin{document}|. Preamble commands are
%   already forbidden there.
% \end{enumerate}
% Because of the various locations the patch code is restricted to
% limitations:
% \begin{itemize}
% \item
% Preamble commands, see \hologo{LaTeX}'s \cs{@onlypreamble}
% throw an error if used after |\begin{document}|. This is already
% the case for \cs{AfterEndDocument}. Therefore preamble commands
% are forbidden in the patching code.
% There are four exceptions \cs{@ifpackageloaded}, \cs{@ifclassloaded},
% \cs{@ifpackagelater} and \cs{@ifclasslater}. They are redefined
% during \cs{AfterEndDocument} using the counterparts of
% package \xpackage{ltxcmds} \cite{ltxcmds}.
% \item
% \cs{AfterPackage} of package \xpackage{scrlfile} and \cs{AtEndPackage}
% of class \xclass{memoir} call the hook before \hologo{LaTeX}'s
% \cs{AtEndOfPackage}.
% \end{itemize}
%
% \StopEventually{
% }
%
% \section{Implementation}
%
%    \begin{macrocode}
%<*package>
%    \end{macrocode}
%
% \subsection{Catcodes and package identification}
%
%    \begin{macrocode}
\begingroup\catcode61\catcode48\catcode32=10\relax%
  \catcode13=5 % ^^M
  \endlinechar=13 %
  \catcode123=1 % {
  \catcode125=2 % }
  \catcode64=11 % @
  \def\x{\endgroup
    \expandafter\edef\csname HOpatch@AtEnd\endcsname{%
      \endlinechar=\the\endlinechar\relax
      \catcode13=\the\catcode13\relax
      \catcode32=\the\catcode32\relax
      \catcode35=\the\catcode35\relax
      \catcode61=\the\catcode61\relax
      \catcode64=\the\catcode64\relax
      \catcode123=\the\catcode123\relax
      \catcode125=\the\catcode125\relax
    }%
  }%
\x\catcode61\catcode48\catcode32=10\relax%
\catcode13=5 % ^^M
\endlinechar=13 %
\catcode35=6 % #
\catcode64=11 % @
\catcode123=1 % {
\catcode125=2 % }
\def\TMP@EnsureCode#1#2{%
  \edef\HOpatch@AtEnd{%
    \HOpatch@AtEnd
    \catcode#1=\the\catcode#1\relax
  }%
  \catcode#1=#2\relax
}
\TMP@EnsureCode{40}{12}% (
\TMP@EnsureCode{41}{12}% )
\TMP@EnsureCode{43}{12}% +
\TMP@EnsureCode{46}{12}% .
\TMP@EnsureCode{47}{12}% /
\TMP@EnsureCode{91}{12}% [
\TMP@EnsureCode{93}{12}% ]
\edef\HOpatch@AtEnd{\HOpatch@AtEnd\noexpand\endinput}
%    \end{macrocode}
%
%    Package identification.
%    \begin{macrocode}
\NeedsTeXFormat{LaTeX2e}
\ProvidesPackage{hopatch}%
  [2012/05/28 v1.2 Wrapper for package hooks (HO)]
%    \end{macrocode}
%
% \subsection{Resources}
%
%    \begin{macrocode}
\begingroup\expandafter\expandafter\expandafter\endgroup
\expandafter\ifx\csname RequirePackage\endcsname\relax
  \def\TMP@RequirePackage#1[#2]{%
    \begingroup\expandafter\expandafter\expandafter\endgroup
    \expandafter\ifx\csname ver@#1.sty\endcsname\relax
      \input #1.sty\relax
    \fi
  }%
  \TMP@RequirePackage{ltxcmds}[2010/12/12]%
\else
  \RequirePackage{ltxcmds}[2010/12/12]%
\fi
%    \end{macrocode}
%
%    \begin{macro}{\HOpatch@counter}
%    \begin{macrocode}
\def\HOpatch@counter{0}%
%    \end{macrocode}
%    \end{macro}
%    \begin{macro}{\HOpatch@StepCounter}
%    \begin{macrocode}
\ltx@ifundefined{numexpr}{%
  \def\HOpatch@StepCounter{%
    \begingroup
      \count@\HOpatch@counter\relax
      \advance\count@\ltx@one\relax
    \edef\x{\endgroup
      \noexpand\def\noexpand\HOpatch@counter{\the\count@}%
    }%
    \x
  }%
}{%
  \def\HOpatch@StepCounter{%
    \edef\HOpatch@counter{%
      \the\numexpr\HOpatch@counter+\ltx@one\relax
    }%
  }%
}
%    \end{macrocode}
%    \end{macro}
%
%    \begin{macro}{\HOpatch@list}
%    \begin{macrocode}
\def\HOpatch@list{}
%    \end{macrocode}
%    \end{macro}
%
%    \begin{macro}{\HOpatch@Add}
%    \begin{macrocode}
\def\HOpatch@Add{%
  \ltx@LocalAppendToMacro\HOpatch@list
}
%    \end{macrocode}
%    \end{macro}
%
% \subsection{Package patching}
%
%    \begin{macro}{\hopatch@AfterPackage}
%    \begin{macrocode}
\def\hopatch@AfterPackage#1{%
  \ltx@ifpackageloaded{#1}{%
    \ltx@firstofone
  }{%
    \HOpatch@AfterPackage{#1}%
  }%
}
%    \end{macrocode}
%    \end{macro}
%    \begin{macro}{\HOpatch@AfterPackage}
%    \begin{macrocode}
\def\HOpatch@AfterPackage#1{%
  \edef\HOpatch@temp{#1}%
  \HOpatch@StepCounter
  \expandafter\HOpatch@@AfterPackage
  \csname HOpatch@\HOpatch@counter\expandafter\endcsname{%
    \HOpatch@temp
  }%
}
%    \end{macrocode}
%    \end{macro}
%
%    \begin{macro}{\HOpatch@@AfterPackage}
%    \begin{macrocode}
\def\HOpatch@@AfterPackage#1#2#3{%
  \begingroup
    \toks@{#3}%
    \xdef\HOpatch@gtemp{%
      \noexpand\ltx@ifpackageloaded{#2}{%
        \noexpand\let\noexpand#1\noexpand\relax
        \the\toks@
      }{}%
    }%
  \endgroup
  \let#1\HOpatch@gtemp
  \HOpatch@Add#1%
  \HOpatch@Try{AfterPackage}{#2}#1%
  \HOpatch@Try{AtEndPackage}{#2}#1%
  \HOpatch@Try{AtEndOfPackageFile}{#2}#1%
}
%    \end{macrocode}
%    \end{macro}
%
%    \begin{macro}{\HOpatch@Try}
%    \begin{macrocode}
\def\HOpatch@Try#1#2#3{%
  \ltx@ifundefined{#1}{}{%
    \csname #1\endcsname{#2}{#3}%
  }%
}
%    \end{macrocode}
%    \end{macro}
%
%    \begin{macrocode}
\AtBeginDocument{\HOpatch@list}
\ltx@ifundefined{AtEndPreamble}{}{%
  \ltx@ifundefined{@endpreamblehook}{}{%
    \AtEndPreamble{\HOpatch@list}%
  }%
}
\ltx@ifundefined{AfterEndPreamble}{}{%
  \ltx@ifundefined{@afterendpreamblehook}{}{%
    \AfterEndPreamble{%
      \let\HOpatch@OrgIfPackageLoaded\@ifpackageloaded
      \let\HOpatch@OrgIfPackageLater\@ifpackagelater
      \let\HOpatch@OrgIfClassLoaded\@ifclassloaded
      \let\HOpatch@OrgIfClassLater\@ifclasslater
      \let\@ifpackageloaded\ltx@ifpackageloaded
      \let\@ifpackagelater\ltx@ifpackagelater
      \let\@ifclassloaded\ltx@ifclassloaded
      \let\@ifclasslater\ltx@ifclasslater
      \HOpatch@list
      \let\@ifpackageloaded\HOpatch@OrgIfPackageLoaded
      \let\@ifpackagelater\HOpatch@OrgIfPackageLater
      \let\@ifclassloaded\HOpatch@OrgIfClassLoaded
      \let\@ifclasslater\HOpatch@OrgIfClassLater
    }%
  }%
}
%    \end{macrocode}
%
%    \begin{macrocode}
\HOpatch@AtEnd%
%    \end{macrocode}
%    \begin{macrocode}
%</package>
%    \end{macrocode}
%
% \section{Test}
%
%    \begin{macrocode}
%<*test1>
\def\LoadCommand{\RequirePackage{hopatch}[2012/05/28]}
%</test1>
%    \end{macrocode}
% \subsection{Catcode checks for loading}
%
%    \begin{macrocode}
%<*test1>
%    \end{macrocode}
%    \begin{macrocode}
\catcode`\{=1 %
\catcode`\}=2 %
\catcode`\#=6 %
\catcode`\@=11 %
\expandafter\ifx\csname count@\endcsname\relax
  \countdef\count@=255 %
\fi
\expandafter\ifx\csname @gobble\endcsname\relax
  \long\def\@gobble#1{}%
\fi
\expandafter\ifx\csname @firstofone\endcsname\relax
  \long\def\@firstofone#1{#1}%
\fi
\expandafter\ifx\csname loop\endcsname\relax
  \expandafter\@firstofone
\else
  \expandafter\@gobble
\fi
{%
  \def\loop#1\repeat{%
    \def\body{#1}%
    \iterate
  }%
  \def\iterate{%
    \body
      \let\next\iterate
    \else
      \let\next\relax
    \fi
    \next
  }%
  \let\repeat=\fi
}%
\def\RestoreCatcodes{}
\count@=0 %
\loop
  \edef\RestoreCatcodes{%
    \RestoreCatcodes
    \catcode\the\count@=\the\catcode\count@\relax
  }%
\ifnum\count@<255 %
  \advance\count@ 1 %
\repeat

\def\RangeCatcodeInvalid#1#2{%
  \count@=#1\relax
  \loop
    \catcode\count@=15 %
  \ifnum\count@<#2\relax
    \advance\count@ 1 %
  \repeat
}
\def\RangeCatcodeCheck#1#2#3{%
  \count@=#1\relax
  \loop
    \ifnum#3=\catcode\count@
    \else
      \errmessage{%
        Character \the\count@\space
        with wrong catcode \the\catcode\count@\space
        instead of \number#3%
      }%
    \fi
  \ifnum\count@<#2\relax
    \advance\count@ 1 %
  \repeat
}
\def\space{ }
\expandafter\ifx\csname LoadCommand\endcsname\relax
  \def\LoadCommand{\input hopatch.sty\relax}%
\fi
\def\Test{%
  \RangeCatcodeInvalid{0}{47}%
  \RangeCatcodeInvalid{58}{64}%
  \RangeCatcodeInvalid{91}{96}%
  \RangeCatcodeInvalid{123}{255}%
  \catcode`\@=12 %
  \catcode`\\=0 %
  \catcode`\%=14 %
  \LoadCommand
  \RangeCatcodeCheck{0}{36}{15}%
  \RangeCatcodeCheck{37}{37}{14}%
  \RangeCatcodeCheck{38}{47}{15}%
  \RangeCatcodeCheck{48}{57}{12}%
  \RangeCatcodeCheck{58}{63}{15}%
  \RangeCatcodeCheck{64}{64}{12}%
  \RangeCatcodeCheck{65}{90}{11}%
  \RangeCatcodeCheck{91}{91}{15}%
  \RangeCatcodeCheck{92}{92}{0}%
  \RangeCatcodeCheck{93}{96}{15}%
  \RangeCatcodeCheck{97}{122}{11}%
  \RangeCatcodeCheck{123}{255}{15}%
  \RestoreCatcodes
}
\Test
\csname @@end\endcsname
\end
%    \end{macrocode}
%    \begin{macrocode}
%</test1>
%    \end{macrocode}
%
%    \begin{macrocode}
%<*test2>
\NeedsTeXFormat{LaTeX2e}
\providecommand\variant{0}
\RequirePackage{filecontents}
\begin{filecontents}{foo.sty}
\ProvidesPackage{foo}
\def\msg#{\immediate\write16}
\def\foo#1{%
  \msg{\fooformat{#1}}%
}
\def\fooformat#1{[#1]}% hash-ok
\foo{* Executing foo at package loading}
\end{filecontents}

\ifnum\variant=1 %
  \documentclass{memoir}%
\else
  \documentclass{article}%
\fi

\ifcase\variant\relax
\or % 1
\or % 2
  \usepackage{etoolbox}%
\or % 3
  \usepackage{scrlfile}%
\or % 4
  \usepackage{filehook}%
\fi

\AtBeginDocument{\foo{* AtBeginDocument before hopatch}}
\usepackage{hopatch}
\AtBeginDocument{\foo{* AtBeginDocument after hopatch}}

\makeatletter
\hopatch@AfterPackage{foo}{%
  \def\fooformat#1{<<#1>>}%
}
\makeatother

\AtBeginDocument{\foo{* AtBeginDocument before foo}}
\usepackage{foo}
\AtBeginDocument{\foo{* AtBeginDocument after foo}}

\foo{* Executing in preamble}

\begin{document}
\foo{* Executing in document}
\end{document}
%</test2>
%    \end{macrocode}
%
% \section{Installation}
%
% \subsection{Download}
%
% \paragraph{Package.} This package is available on
% CTAN\footnote{\url{ftp://ftp.ctan.org/tex-archive/}}:
% \begin{description}
% \item[\CTAN{macros/latex/contrib/oberdiek/hopatch.dtx}] The source file.
% \item[\CTAN{macros/latex/contrib/oberdiek/hopatch.pdf}] Documentation.
% \end{description}
%
%
% \paragraph{Bundle.} All the packages of the bundle `oberdiek'
% are also available in a TDS compliant ZIP archive. There
% the packages are already unpacked and the documentation files
% are generated. The files and directories obey the TDS standard.
% \begin{description}
% \item[\CTAN{install/macros/latex/contrib/oberdiek.tds.zip}]
% \end{description}
% \emph{TDS} refers to the standard ``A Directory Structure
% for \TeX\ Files'' (\CTAN{tds/tds.pdf}). Directories
% with \xfile{texmf} in their name are usually organized this way.
%
% \subsection{Bundle installation}
%
% \paragraph{Unpacking.} Unpack the \xfile{oberdiek.tds.zip} in the
% TDS tree (also known as \xfile{texmf} tree) of your choice.
% Example (linux):
% \begin{quote}
%   |unzip oberdiek.tds.zip -d ~/texmf|
% \end{quote}
%
% \paragraph{Script installation.}
% Check the directory \xfile{TDS:scripts/oberdiek/} for
% scripts that need further installation steps.
% Package \xpackage{attachfile2} comes with the Perl script
% \xfile{pdfatfi.pl} that should be installed in such a way
% that it can be called as \texttt{pdfatfi}.
% Example (linux):
% \begin{quote}
%   |chmod +x scripts/oberdiek/pdfatfi.pl|\\
%   |cp scripts/oberdiek/pdfatfi.pl /usr/local/bin/|
% \end{quote}
%
% \subsection{Package installation}
%
% \paragraph{Unpacking.} The \xfile{.dtx} file is a self-extracting
% \docstrip\ archive. The files are extracted by running the
% \xfile{.dtx} through \plainTeX:
% \begin{quote}
%   \verb|tex hopatch.dtx|
% \end{quote}
%
% \paragraph{TDS.} Now the different files must be moved into
% the different directories in your installation TDS tree
% (also known as \xfile{texmf} tree):
% \begin{quote}
% \def\t{^^A
% \begin{tabular}{@{}>{\ttfamily}l@{ $\rightarrow$ }>{\ttfamily}l@{}}
%   hopatch.sty & tex/latex/oberdiek/hopatch.sty\\
%   hopatch.pdf & doc/latex/oberdiek/hopatch.pdf\\
%   test/hopatch-test1.tex & doc/latex/oberdiek/test/hopatch-test1.tex\\
%   test/hopatch-test2.tex & doc/latex/oberdiek/test/hopatch-test2.tex\\
%   hopatch.dtx & source/latex/oberdiek/hopatch.dtx\\
% \end{tabular}^^A
% }^^A
% \sbox0{\t}^^A
% \ifdim\wd0>\linewidth
%   \begingroup
%     \advance\linewidth by\leftmargin
%     \advance\linewidth by\rightmargin
%   \edef\x{\endgroup
%     \def\noexpand\lw{\the\linewidth}^^A
%   }\x
%   \def\lwbox{^^A
%     \leavevmode
%     \hbox to \linewidth{^^A
%       \kern-\leftmargin\relax
%       \hss
%       \usebox0
%       \hss
%       \kern-\rightmargin\relax
%     }^^A
%   }^^A
%   \ifdim\wd0>\lw
%     \sbox0{\small\t}^^A
%     \ifdim\wd0>\linewidth
%       \ifdim\wd0>\lw
%         \sbox0{\footnotesize\t}^^A
%         \ifdim\wd0>\linewidth
%           \ifdim\wd0>\lw
%             \sbox0{\scriptsize\t}^^A
%             \ifdim\wd0>\linewidth
%               \ifdim\wd0>\lw
%                 \sbox0{\tiny\t}^^A
%                 \ifdim\wd0>\linewidth
%                   \lwbox
%                 \else
%                   \usebox0
%                 \fi
%               \else
%                 \lwbox
%               \fi
%             \else
%               \usebox0
%             \fi
%           \else
%             \lwbox
%           \fi
%         \else
%           \usebox0
%         \fi
%       \else
%         \lwbox
%       \fi
%     \else
%       \usebox0
%     \fi
%   \else
%     \lwbox
%   \fi
% \else
%   \usebox0
% \fi
% \end{quote}
% If you have a \xfile{docstrip.cfg} that configures and enables \docstrip's
% TDS installing feature, then some files can already be in the right
% place, see the documentation of \docstrip.
%
% \subsection{Refresh file name databases}
%
% If your \TeX~distribution
% (\teTeX, \mikTeX, \dots) relies on file name databases, you must refresh
% these. For example, \teTeX\ users run \verb|texhash| or
% \verb|mktexlsr|.
%
% \subsection{Some details for the interested}
%
% \paragraph{Attached source.}
%
% The PDF documentation on CTAN also includes the
% \xfile{.dtx} source file. It can be extracted by
% AcrobatReader 6 or higher. Another option is \textsf{pdftk},
% e.g. unpack the file into the current directory:
% \begin{quote}
%   \verb|pdftk hopatch.pdf unpack_files output .|
% \end{quote}
%
% \paragraph{Unpacking with \LaTeX.}
% The \xfile{.dtx} chooses its action depending on the format:
% \begin{description}
% \item[\plainTeX:] Run \docstrip\ and extract the files.
% \item[\LaTeX:] Generate the documentation.
% \end{description}
% If you insist on using \LaTeX\ for \docstrip\ (really,
% \docstrip\ does not need \LaTeX), then inform the autodetect routine
% about your intention:
% \begin{quote}
%   \verb|latex \let\install=y% \iffalse meta-comment
%
% File: hopatch.dtx
% Version: 2012/05/28 v1.2
% Info: Wrapper for package hooks
%
% Copyright (C) 2011, 2012 by
%    Heiko Oberdiek <heiko.oberdiek at googlemail.com>
%
% This work may be distributed and/or modified under the
% conditions of the LaTeX Project Public License, either
% version 1.3c of this license or (at your option) any later
% version. This version of this license is in
%    http://www.latex-project.org/lppl/lppl-1-3c.txt
% and the latest version of this license is in
%    http://www.latex-project.org/lppl.txt
% and version 1.3 or later is part of all distributions of
% LaTeX version 2005/12/01 or later.
%
% This work has the LPPL maintenance status "maintained".
%
% This Current Maintainer of this work is Heiko Oberdiek.
%
% This work consists of the main source file hopatch.dtx
% and the derived files
%    hopatch.sty, hopatch.pdf, hopatch.ins, hopatch.drv,
%    hopatch-test1.tex, hopatch-test2.tex.
%
% Distribution:
%    CTAN:macros/latex/contrib/oberdiek/hopatch.dtx
%    CTAN:macros/latex/contrib/oberdiek/hopatch.pdf
%
% Unpacking:
%    (a) If hopatch.ins is present:
%           tex hopatch.ins
%    (b) Without hopatch.ins:
%           tex hopatch.dtx
%    (c) If you insist on using LaTeX
%           latex \let\install=y% \iffalse meta-comment
%
% File: hopatch.dtx
% Version: 2012/05/28 v1.2
% Info: Wrapper for package hooks
%
% Copyright (C) 2011, 2012 by
%    Heiko Oberdiek <heiko.oberdiek at googlemail.com>
%
% This work may be distributed and/or modified under the
% conditions of the LaTeX Project Public License, either
% version 1.3c of this license or (at your option) any later
% version. This version of this license is in
%    http://www.latex-project.org/lppl/lppl-1-3c.txt
% and the latest version of this license is in
%    http://www.latex-project.org/lppl.txt
% and version 1.3 or later is part of all distributions of
% LaTeX version 2005/12/01 or later.
%
% This work has the LPPL maintenance status "maintained".
%
% This Current Maintainer of this work is Heiko Oberdiek.
%
% This work consists of the main source file hopatch.dtx
% and the derived files
%    hopatch.sty, hopatch.pdf, hopatch.ins, hopatch.drv,
%    hopatch-test1.tex, hopatch-test2.tex.
%
% Distribution:
%    CTAN:macros/latex/contrib/oberdiek/hopatch.dtx
%    CTAN:macros/latex/contrib/oberdiek/hopatch.pdf
%
% Unpacking:
%    (a) If hopatch.ins is present:
%           tex hopatch.ins
%    (b) Without hopatch.ins:
%           tex hopatch.dtx
%    (c) If you insist on using LaTeX
%           latex \let\install=y% \iffalse meta-comment
%
% File: hopatch.dtx
% Version: 2012/05/28 v1.2
% Info: Wrapper for package hooks
%
% Copyright (C) 2011, 2012 by
%    Heiko Oberdiek <heiko.oberdiek at googlemail.com>
%
% This work may be distributed and/or modified under the
% conditions of the LaTeX Project Public License, either
% version 1.3c of this license or (at your option) any later
% version. This version of this license is in
%    http://www.latex-project.org/lppl/lppl-1-3c.txt
% and the latest version of this license is in
%    http://www.latex-project.org/lppl.txt
% and version 1.3 or later is part of all distributions of
% LaTeX version 2005/12/01 or later.
%
% This work has the LPPL maintenance status "maintained".
%
% This Current Maintainer of this work is Heiko Oberdiek.
%
% This work consists of the main source file hopatch.dtx
% and the derived files
%    hopatch.sty, hopatch.pdf, hopatch.ins, hopatch.drv,
%    hopatch-test1.tex, hopatch-test2.tex.
%
% Distribution:
%    CTAN:macros/latex/contrib/oberdiek/hopatch.dtx
%    CTAN:macros/latex/contrib/oberdiek/hopatch.pdf
%
% Unpacking:
%    (a) If hopatch.ins is present:
%           tex hopatch.ins
%    (b) Without hopatch.ins:
%           tex hopatch.dtx
%    (c) If you insist on using LaTeX
%           latex \let\install=y\input{hopatch.dtx}
%        (quote the arguments according to the demands of your shell)
%
% Documentation:
%    (a) If hopatch.drv is present:
%           latex hopatch.drv
%    (b) Without hopatch.drv:
%           latex hopatch.dtx; ...
%    The class ltxdoc loads the configuration file ltxdoc.cfg
%    if available. Here you can specify further options, e.g.
%    use A4 as paper format:
%       \PassOptionsToClass{a4paper}{article}
%
%    Programm calls to get the documentation (example):
%       pdflatex hopatch.dtx
%       makeindex -s gind.ist hopatch.idx
%       pdflatex hopatch.dtx
%       makeindex -s gind.ist hopatch.idx
%       pdflatex hopatch.dtx
%
% Installation:
%    TDS:tex/latex/oberdiek/hopatch.sty
%    TDS:doc/latex/oberdiek/hopatch.pdf
%    TDS:doc/latex/oberdiek/test/hopatch-test1.tex
%    TDS:doc/latex/oberdiek/test/hopatch-test2.tex
%    TDS:source/latex/oberdiek/hopatch.dtx
%
%<*ignore>
\begingroup
  \catcode123=1 %
  \catcode125=2 %
  \def\x{LaTeX2e}%
\expandafter\endgroup
\ifcase 0\ifx\install y1\fi\expandafter
         \ifx\csname processbatchFile\endcsname\relax\else1\fi
         \ifx\fmtname\x\else 1\fi\relax
\else\csname fi\endcsname
%</ignore>
%<*install>
\input docstrip.tex
\Msg{************************************************************************}
\Msg{* Installation}
\Msg{* Package: hopatch 2012/05/28 v1.2 Wrapper for package hooks (HO)}
\Msg{************************************************************************}

\keepsilent
\askforoverwritefalse

\let\MetaPrefix\relax
\preamble

This is a generated file.

Project: hopatch
Version: 2012/05/28 v1.2

Copyright (C) 2011, 2012 by
   Heiko Oberdiek <heiko.oberdiek at googlemail.com>

This work may be distributed and/or modified under the
conditions of the LaTeX Project Public License, either
version 1.3c of this license or (at your option) any later
version. This version of this license is in
   http://www.latex-project.org/lppl/lppl-1-3c.txt
and the latest version of this license is in
   http://www.latex-project.org/lppl.txt
and version 1.3 or later is part of all distributions of
LaTeX version 2005/12/01 or later.

This work has the LPPL maintenance status "maintained".

This Current Maintainer of this work is Heiko Oberdiek.

This work consists of the main source file hopatch.dtx
and the derived files
   hopatch.sty, hopatch.pdf, hopatch.ins, hopatch.drv,
   hopatch-test1.tex, hopatch-test2.tex.

\endpreamble
\let\MetaPrefix\DoubleperCent

\generate{%
  \file{hopatch.ins}{\from{hopatch.dtx}{install}}%
  \file{hopatch.drv}{\from{hopatch.dtx}{driver}}%
  \usedir{tex/latex/oberdiek}%
  \file{hopatch.sty}{\from{hopatch.dtx}{package}}%
  \usedir{doc/latex/oberdiek/test}%
  \file{hopatch-test1.tex}{\from{hopatch.dtx}{test1}}%
  \file{hopatch-test2.tex}{\from{hopatch.dtx}{test2}}%
  \nopreamble
  \nopostamble
  \usedir{source/latex/oberdiek/catalogue}%
  \file{hopatch.xml}{\from{hopatch.dtx}{catalogue}}%
}

\catcode32=13\relax% active space
\let =\space%
\Msg{************************************************************************}
\Msg{*}
\Msg{* To finish the installation you have to move the following}
\Msg{* file into a directory searched by TeX:}
\Msg{*}
\Msg{*     hopatch.sty}
\Msg{*}
\Msg{* To produce the documentation run the file `hopatch.drv'}
\Msg{* through LaTeX.}
\Msg{*}
\Msg{* Happy TeXing!}
\Msg{*}
\Msg{************************************************************************}

\endbatchfile
%</install>
%<*ignore>
\fi
%</ignore>
%<*driver>
\NeedsTeXFormat{LaTeX2e}
\ProvidesFile{hopatch.drv}%
  [2012/05/28 v1.2 Wrapper for package hooks (HO)]%
\documentclass{ltxdoc}
\usepackage{holtxdoc}[2011/11/22]
\usepackage{caption}
\usepackage{array}
\setlength{\extrarowheight}{1pt}
\renewcommand*{\arraystretch}{1.1}
\begin{document}
  \DocInput{hopatch.dtx}%
\end{document}
%</driver>
% \fi
%
% \CheckSum{466}
%
% \CharacterTable
%  {Upper-case    \A\B\C\D\E\F\G\H\I\J\K\L\M\N\O\P\Q\R\S\T\U\V\W\X\Y\Z
%   Lower-case    \a\b\c\d\e\f\g\h\i\j\k\l\m\n\o\p\q\r\s\t\u\v\w\x\y\z
%   Digits        \0\1\2\3\4\5\6\7\8\9
%   Exclamation   \!     Double quote  \"     Hash (number) \#
%   Dollar        \$     Percent       \%     Ampersand     \&
%   Acute accent  \'     Left paren    \(     Right paren   \)
%   Asterisk      \*     Plus          \+     Comma         \,
%   Minus         \-     Point         \.     Solidus       \/
%   Colon         \:     Semicolon     \;     Less than     \<
%   Equals        \=     Greater than  \>     Question mark \?
%   Commercial at \@     Left bracket  \[     Backslash     \\
%   Right bracket \]     Circumflex    \^     Underscore    \_
%   Grave accent  \`     Left brace    \{     Vertical bar  \|
%   Right brace   \}     Tilde         \~}
%
% \GetFileInfo{hopatch.drv}
%
% \title{The \xpackage{hopatch} package}
% \date{2012/05/28 v1.2}
% \author{Heiko Oberdiek\\\xemail{heiko.oberdiek at googlemail.com}}
%
% \maketitle
%
% \begin{abstract}
% This packages provides a wrapper to various package hooks
% provided by other packages or classes, but does not define
% own hooks.
% \end{abstract}
%
% \tableofcontents
%
% \section{Documentation}
%
% Sometimes I want to add code right after a package
% has been loaded. Examples are bug fixes, adaptations,
% or added features as needed by package \xpackage{hyperref},
% for instance.
%
% Unhappily \hologo{LaTeX} does not provide this kind of
% hook. \cs{AtEndOfPackage} can be used inside the package
% only, because \hologo{LaTeX} clears the hook right before
% it loads the package.
%
% However, there are already many packages and classes
% that provide hooks that are executed after the package
% is loaded, see table~\ref{tab:hooks}.
%
% \begin{table}
% \caption{After package hooking}
% \label{tab:hooks}
% \begin{tabular}{ll}
% \bfseries Macro & \bfseries Provider\\
% \hline
% \cs{AfterPackage} & package \xpackage{scrlfile} \cite{scrlfile}\\
% \cs{AtEndOfPackageFile} & package \xpackage{filehook} \cite{filehook}\\
% \cs{AtEndPackage} & class \xclass{memoir} \cite{memoir}
% \end{tabular}
% \end{table}
%
% Package \xpackage{hopatch} can be used without the packages
% of table~\ref{tab:hooks}. But for an early executing right
% after a package is loaded, one of the following class or
% packages should be loaded before using \cs{hopatch@AfterPackage}:
% \begin{itemize}
%  \item package \xpackage{filehook}
%  \item package \xpackage{scrlfile}
%  \item class \xclass{memoir}
% \end{itemize}
%
% \begin{table}
% \caption{After begin document hooking}
% \begin{tabular}{ll}
% \bfseries Macro & \bfseries Provider\\
% \hline
% \cs{AtBeginDocument} & \hologo{LaTeX}'s kernel\\
% \cs{AtEndPreamble} & package \xpackage{etoolbox} \cite{etoolbox}\\
% \cs{AfterEndPreamble} & package \xpackage{etoolbox}
% \end{tabular}
% \end{table}
%
% Therefore I skip writing a new package for hooking into
% \hologo{LaTeX}'s package management and use this package
% to provide a wrapper to patch a package after it is loaded.
%
% \begin{declcs}{hopatch@AfterPackage} \M{package} \M{patch code}
% \end{declcs}
% If the package is already loaded, the \meta{patch code} is executed
% immediately.
% Otherwise the \meta{patch code} is stored in a command and tried
% at later locations until the package is available.
%
% The patch is tried in the following order:
% \begin{enumerate}
% \item If the package is already loaded, the patch is
%   applied immediately. Further locations are not tried.
% \item \cs{AtEndPackage}, provided by class \xclass{memoir} \cite{memoir},
%   and \cs{AfterPackage}, provided by package \xpackage{scrlfile} \cite{scrlfile},
%   are called right after the package file is input before
%   the hook of \hologo{LaTeX}'s \cs{AtEndOfPackage}.
% \item \cs{AtEndOfPackageFile}, provided by package \xpackage{filehook} \cite{filehook},
%   is called after the package is loaded and after the hook of
%   \hologo{LaTeX}'s \cs{AtEndOfPackage}.
% \item \cs{AtEndPreamble}, provided by package \xpackage{etoolbox} \cite{etoolbox},
%   is called at the beginning of |\begin{document}| before the hook
%   of \hologo{LaTeX}'s \cs{AtBeginDocument}.
% \item \cs{AtBeginDocument}, provided by \hologo{LaTeX}.
% \item \cs{AfterEndDocument}, provided by package \xpackage{etoolbox} \cite{etoolbox},
%   is called at the very end of |\begin{document}|. Preamble commands are
%   already forbidden there.
% \end{enumerate}
% Because of the various locations the patch code is restricted to
% limitations:
% \begin{itemize}
% \item
% Preamble commands, see \hologo{LaTeX}'s \cs{@onlypreamble}
% throw an error if used after |\begin{document}|. This is already
% the case for \cs{AfterEndDocument}. Therefore preamble commands
% are forbidden in the patching code.
% There are four exceptions \cs{@ifpackageloaded}, \cs{@ifclassloaded},
% \cs{@ifpackagelater} and \cs{@ifclasslater}. They are redefined
% during \cs{AfterEndDocument} using the counterparts of
% package \xpackage{ltxcmds} \cite{ltxcmds}.
% \item
% \cs{AfterPackage} of package \xpackage{scrlfile} and \cs{AtEndPackage}
% of class \xclass{memoir} call the hook before \hologo{LaTeX}'s
% \cs{AtEndOfPackage}.
% \end{itemize}
%
% \StopEventually{
% }
%
% \section{Implementation}
%
%    \begin{macrocode}
%<*package>
%    \end{macrocode}
%
% \subsection{Catcodes and package identification}
%
%    \begin{macrocode}
\begingroup\catcode61\catcode48\catcode32=10\relax%
  \catcode13=5 % ^^M
  \endlinechar=13 %
  \catcode123=1 % {
  \catcode125=2 % }
  \catcode64=11 % @
  \def\x{\endgroup
    \expandafter\edef\csname HOpatch@AtEnd\endcsname{%
      \endlinechar=\the\endlinechar\relax
      \catcode13=\the\catcode13\relax
      \catcode32=\the\catcode32\relax
      \catcode35=\the\catcode35\relax
      \catcode61=\the\catcode61\relax
      \catcode64=\the\catcode64\relax
      \catcode123=\the\catcode123\relax
      \catcode125=\the\catcode125\relax
    }%
  }%
\x\catcode61\catcode48\catcode32=10\relax%
\catcode13=5 % ^^M
\endlinechar=13 %
\catcode35=6 % #
\catcode64=11 % @
\catcode123=1 % {
\catcode125=2 % }
\def\TMP@EnsureCode#1#2{%
  \edef\HOpatch@AtEnd{%
    \HOpatch@AtEnd
    \catcode#1=\the\catcode#1\relax
  }%
  \catcode#1=#2\relax
}
\TMP@EnsureCode{40}{12}% (
\TMP@EnsureCode{41}{12}% )
\TMP@EnsureCode{43}{12}% +
\TMP@EnsureCode{46}{12}% .
\TMP@EnsureCode{47}{12}% /
\TMP@EnsureCode{91}{12}% [
\TMP@EnsureCode{93}{12}% ]
\edef\HOpatch@AtEnd{\HOpatch@AtEnd\noexpand\endinput}
%    \end{macrocode}
%
%    Package identification.
%    \begin{macrocode}
\NeedsTeXFormat{LaTeX2e}
\ProvidesPackage{hopatch}%
  [2012/05/28 v1.2 Wrapper for package hooks (HO)]
%    \end{macrocode}
%
% \subsection{Resources}
%
%    \begin{macrocode}
\begingroup\expandafter\expandafter\expandafter\endgroup
\expandafter\ifx\csname RequirePackage\endcsname\relax
  \def\TMP@RequirePackage#1[#2]{%
    \begingroup\expandafter\expandafter\expandafter\endgroup
    \expandafter\ifx\csname ver@#1.sty\endcsname\relax
      \input #1.sty\relax
    \fi
  }%
  \TMP@RequirePackage{ltxcmds}[2010/12/12]%
\else
  \RequirePackage{ltxcmds}[2010/12/12]%
\fi
%    \end{macrocode}
%
%    \begin{macro}{\HOpatch@counter}
%    \begin{macrocode}
\def\HOpatch@counter{0}%
%    \end{macrocode}
%    \end{macro}
%    \begin{macro}{\HOpatch@StepCounter}
%    \begin{macrocode}
\ltx@ifundefined{numexpr}{%
  \def\HOpatch@StepCounter{%
    \begingroup
      \count@\HOpatch@counter\relax
      \advance\count@\ltx@one\relax
    \edef\x{\endgroup
      \noexpand\def\noexpand\HOpatch@counter{\the\count@}%
    }%
    \x
  }%
}{%
  \def\HOpatch@StepCounter{%
    \edef\HOpatch@counter{%
      \the\numexpr\HOpatch@counter+\ltx@one\relax
    }%
  }%
}
%    \end{macrocode}
%    \end{macro}
%
%    \begin{macro}{\HOpatch@list}
%    \begin{macrocode}
\def\HOpatch@list{}
%    \end{macrocode}
%    \end{macro}
%
%    \begin{macro}{\HOpatch@Add}
%    \begin{macrocode}
\def\HOpatch@Add{%
  \ltx@LocalAppendToMacro\HOpatch@list
}
%    \end{macrocode}
%    \end{macro}
%
% \subsection{Package patching}
%
%    \begin{macro}{\hopatch@AfterPackage}
%    \begin{macrocode}
\def\hopatch@AfterPackage#1{%
  \ltx@ifpackageloaded{#1}{%
    \ltx@firstofone
  }{%
    \HOpatch@AfterPackage{#1}%
  }%
}
%    \end{macrocode}
%    \end{macro}
%    \begin{macro}{\HOpatch@AfterPackage}
%    \begin{macrocode}
\def\HOpatch@AfterPackage#1{%
  \edef\HOpatch@temp{#1}%
  \HOpatch@StepCounter
  \expandafter\HOpatch@@AfterPackage
  \csname HOpatch@\HOpatch@counter\expandafter\endcsname{%
    \HOpatch@temp
  }%
}
%    \end{macrocode}
%    \end{macro}
%
%    \begin{macro}{\HOpatch@@AfterPackage}
%    \begin{macrocode}
\def\HOpatch@@AfterPackage#1#2#3{%
  \begingroup
    \toks@{#3}%
    \xdef\HOpatch@gtemp{%
      \noexpand\ltx@ifpackageloaded{#2}{%
        \noexpand\let\noexpand#1\noexpand\relax
        \the\toks@
      }{}%
    }%
  \endgroup
  \let#1\HOpatch@gtemp
  \HOpatch@Add#1%
  \HOpatch@Try{AfterPackage}{#2}#1%
  \HOpatch@Try{AtEndPackage}{#2}#1%
  \HOpatch@Try{AtEndOfPackageFile}{#2}#1%
}
%    \end{macrocode}
%    \end{macro}
%
%    \begin{macro}{\HOpatch@Try}
%    \begin{macrocode}
\def\HOpatch@Try#1#2#3{%
  \ltx@ifundefined{#1}{}{%
    \csname #1\endcsname{#2}{#3}%
  }%
}
%    \end{macrocode}
%    \end{macro}
%
%    \begin{macrocode}
\AtBeginDocument{\HOpatch@list}
\ltx@ifundefined{AtEndPreamble}{}{%
  \ltx@ifundefined{@endpreamblehook}{}{%
    \AtEndPreamble{\HOpatch@list}%
  }%
}
\ltx@ifundefined{AfterEndPreamble}{}{%
  \ltx@ifundefined{@afterendpreamblehook}{}{%
    \AfterEndPreamble{%
      \let\HOpatch@OrgIfPackageLoaded\@ifpackageloaded
      \let\HOpatch@OrgIfPackageLater\@ifpackagelater
      \let\HOpatch@OrgIfClassLoaded\@ifclassloaded
      \let\HOpatch@OrgIfClassLater\@ifclasslater
      \let\@ifpackageloaded\ltx@ifpackageloaded
      \let\@ifpackagelater\ltx@ifpackagelater
      \let\@ifclassloaded\ltx@ifclassloaded
      \let\@ifclasslater\ltx@ifclasslater
      \HOpatch@list
      \let\@ifpackageloaded\HOpatch@OrgIfPackageLoaded
      \let\@ifpackagelater\HOpatch@OrgIfPackageLater
      \let\@ifclassloaded\HOpatch@OrgIfClassLoaded
      \let\@ifclasslater\HOpatch@OrgIfClassLater
    }%
  }%
}
%    \end{macrocode}
%
%    \begin{macrocode}
\HOpatch@AtEnd%
%    \end{macrocode}
%    \begin{macrocode}
%</package>
%    \end{macrocode}
%
% \section{Test}
%
%    \begin{macrocode}
%<*test1>
\def\LoadCommand{\RequirePackage{hopatch}[2012/05/28]}
%</test1>
%    \end{macrocode}
% \subsection{Catcode checks for loading}
%
%    \begin{macrocode}
%<*test1>
%    \end{macrocode}
%    \begin{macrocode}
\catcode`\{=1 %
\catcode`\}=2 %
\catcode`\#=6 %
\catcode`\@=11 %
\expandafter\ifx\csname count@\endcsname\relax
  \countdef\count@=255 %
\fi
\expandafter\ifx\csname @gobble\endcsname\relax
  \long\def\@gobble#1{}%
\fi
\expandafter\ifx\csname @firstofone\endcsname\relax
  \long\def\@firstofone#1{#1}%
\fi
\expandafter\ifx\csname loop\endcsname\relax
  \expandafter\@firstofone
\else
  \expandafter\@gobble
\fi
{%
  \def\loop#1\repeat{%
    \def\body{#1}%
    \iterate
  }%
  \def\iterate{%
    \body
      \let\next\iterate
    \else
      \let\next\relax
    \fi
    \next
  }%
  \let\repeat=\fi
}%
\def\RestoreCatcodes{}
\count@=0 %
\loop
  \edef\RestoreCatcodes{%
    \RestoreCatcodes
    \catcode\the\count@=\the\catcode\count@\relax
  }%
\ifnum\count@<255 %
  \advance\count@ 1 %
\repeat

\def\RangeCatcodeInvalid#1#2{%
  \count@=#1\relax
  \loop
    \catcode\count@=15 %
  \ifnum\count@<#2\relax
    \advance\count@ 1 %
  \repeat
}
\def\RangeCatcodeCheck#1#2#3{%
  \count@=#1\relax
  \loop
    \ifnum#3=\catcode\count@
    \else
      \errmessage{%
        Character \the\count@\space
        with wrong catcode \the\catcode\count@\space
        instead of \number#3%
      }%
    \fi
  \ifnum\count@<#2\relax
    \advance\count@ 1 %
  \repeat
}
\def\space{ }
\expandafter\ifx\csname LoadCommand\endcsname\relax
  \def\LoadCommand{\input hopatch.sty\relax}%
\fi
\def\Test{%
  \RangeCatcodeInvalid{0}{47}%
  \RangeCatcodeInvalid{58}{64}%
  \RangeCatcodeInvalid{91}{96}%
  \RangeCatcodeInvalid{123}{255}%
  \catcode`\@=12 %
  \catcode`\\=0 %
  \catcode`\%=14 %
  \LoadCommand
  \RangeCatcodeCheck{0}{36}{15}%
  \RangeCatcodeCheck{37}{37}{14}%
  \RangeCatcodeCheck{38}{47}{15}%
  \RangeCatcodeCheck{48}{57}{12}%
  \RangeCatcodeCheck{58}{63}{15}%
  \RangeCatcodeCheck{64}{64}{12}%
  \RangeCatcodeCheck{65}{90}{11}%
  \RangeCatcodeCheck{91}{91}{15}%
  \RangeCatcodeCheck{92}{92}{0}%
  \RangeCatcodeCheck{93}{96}{15}%
  \RangeCatcodeCheck{97}{122}{11}%
  \RangeCatcodeCheck{123}{255}{15}%
  \RestoreCatcodes
}
\Test
\csname @@end\endcsname
\end
%    \end{macrocode}
%    \begin{macrocode}
%</test1>
%    \end{macrocode}
%
%    \begin{macrocode}
%<*test2>
\NeedsTeXFormat{LaTeX2e}
\providecommand\variant{0}
\RequirePackage{filecontents}
\begin{filecontents}{foo.sty}
\ProvidesPackage{foo}
\def\msg#{\immediate\write16}
\def\foo#1{%
  \msg{\fooformat{#1}}%
}
\def\fooformat#1{[#1]}% hash-ok
\foo{* Executing foo at package loading}
\end{filecontents}

\ifnum\variant=1 %
  \documentclass{memoir}%
\else
  \documentclass{article}%
\fi

\ifcase\variant\relax
\or % 1
\or % 2
  \usepackage{etoolbox}%
\or % 3
  \usepackage{scrlfile}%
\or % 4
  \usepackage{filehook}%
\fi

\AtBeginDocument{\foo{* AtBeginDocument before hopatch}}
\usepackage{hopatch}
\AtBeginDocument{\foo{* AtBeginDocument after hopatch}}

\makeatletter
\hopatch@AfterPackage{foo}{%
  \def\fooformat#1{<<#1>>}%
}
\makeatother

\AtBeginDocument{\foo{* AtBeginDocument before foo}}
\usepackage{foo}
\AtBeginDocument{\foo{* AtBeginDocument after foo}}

\foo{* Executing in preamble}

\begin{document}
\foo{* Executing in document}
\end{document}
%</test2>
%    \end{macrocode}
%
% \section{Installation}
%
% \subsection{Download}
%
% \paragraph{Package.} This package is available on
% CTAN\footnote{\url{ftp://ftp.ctan.org/tex-archive/}}:
% \begin{description}
% \item[\CTAN{macros/latex/contrib/oberdiek/hopatch.dtx}] The source file.
% \item[\CTAN{macros/latex/contrib/oberdiek/hopatch.pdf}] Documentation.
% \end{description}
%
%
% \paragraph{Bundle.} All the packages of the bundle `oberdiek'
% are also available in a TDS compliant ZIP archive. There
% the packages are already unpacked and the documentation files
% are generated. The files and directories obey the TDS standard.
% \begin{description}
% \item[\CTAN{install/macros/latex/contrib/oberdiek.tds.zip}]
% \end{description}
% \emph{TDS} refers to the standard ``A Directory Structure
% for \TeX\ Files'' (\CTAN{tds/tds.pdf}). Directories
% with \xfile{texmf} in their name are usually organized this way.
%
% \subsection{Bundle installation}
%
% \paragraph{Unpacking.} Unpack the \xfile{oberdiek.tds.zip} in the
% TDS tree (also known as \xfile{texmf} tree) of your choice.
% Example (linux):
% \begin{quote}
%   |unzip oberdiek.tds.zip -d ~/texmf|
% \end{quote}
%
% \paragraph{Script installation.}
% Check the directory \xfile{TDS:scripts/oberdiek/} for
% scripts that need further installation steps.
% Package \xpackage{attachfile2} comes with the Perl script
% \xfile{pdfatfi.pl} that should be installed in such a way
% that it can be called as \texttt{pdfatfi}.
% Example (linux):
% \begin{quote}
%   |chmod +x scripts/oberdiek/pdfatfi.pl|\\
%   |cp scripts/oberdiek/pdfatfi.pl /usr/local/bin/|
% \end{quote}
%
% \subsection{Package installation}
%
% \paragraph{Unpacking.} The \xfile{.dtx} file is a self-extracting
% \docstrip\ archive. The files are extracted by running the
% \xfile{.dtx} through \plainTeX:
% \begin{quote}
%   \verb|tex hopatch.dtx|
% \end{quote}
%
% \paragraph{TDS.} Now the different files must be moved into
% the different directories in your installation TDS tree
% (also known as \xfile{texmf} tree):
% \begin{quote}
% \def\t{^^A
% \begin{tabular}{@{}>{\ttfamily}l@{ $\rightarrow$ }>{\ttfamily}l@{}}
%   hopatch.sty & tex/latex/oberdiek/hopatch.sty\\
%   hopatch.pdf & doc/latex/oberdiek/hopatch.pdf\\
%   test/hopatch-test1.tex & doc/latex/oberdiek/test/hopatch-test1.tex\\
%   test/hopatch-test2.tex & doc/latex/oberdiek/test/hopatch-test2.tex\\
%   hopatch.dtx & source/latex/oberdiek/hopatch.dtx\\
% \end{tabular}^^A
% }^^A
% \sbox0{\t}^^A
% \ifdim\wd0>\linewidth
%   \begingroup
%     \advance\linewidth by\leftmargin
%     \advance\linewidth by\rightmargin
%   \edef\x{\endgroup
%     \def\noexpand\lw{\the\linewidth}^^A
%   }\x
%   \def\lwbox{^^A
%     \leavevmode
%     \hbox to \linewidth{^^A
%       \kern-\leftmargin\relax
%       \hss
%       \usebox0
%       \hss
%       \kern-\rightmargin\relax
%     }^^A
%   }^^A
%   \ifdim\wd0>\lw
%     \sbox0{\small\t}^^A
%     \ifdim\wd0>\linewidth
%       \ifdim\wd0>\lw
%         \sbox0{\footnotesize\t}^^A
%         \ifdim\wd0>\linewidth
%           \ifdim\wd0>\lw
%             \sbox0{\scriptsize\t}^^A
%             \ifdim\wd0>\linewidth
%               \ifdim\wd0>\lw
%                 \sbox0{\tiny\t}^^A
%                 \ifdim\wd0>\linewidth
%                   \lwbox
%                 \else
%                   \usebox0
%                 \fi
%               \else
%                 \lwbox
%               \fi
%             \else
%               \usebox0
%             \fi
%           \else
%             \lwbox
%           \fi
%         \else
%           \usebox0
%         \fi
%       \else
%         \lwbox
%       \fi
%     \else
%       \usebox0
%     \fi
%   \else
%     \lwbox
%   \fi
% \else
%   \usebox0
% \fi
% \end{quote}
% If you have a \xfile{docstrip.cfg} that configures and enables \docstrip's
% TDS installing feature, then some files can already be in the right
% place, see the documentation of \docstrip.
%
% \subsection{Refresh file name databases}
%
% If your \TeX~distribution
% (\teTeX, \mikTeX, \dots) relies on file name databases, you must refresh
% these. For example, \teTeX\ users run \verb|texhash| or
% \verb|mktexlsr|.
%
% \subsection{Some details for the interested}
%
% \paragraph{Attached source.}
%
% The PDF documentation on CTAN also includes the
% \xfile{.dtx} source file. It can be extracted by
% AcrobatReader 6 or higher. Another option is \textsf{pdftk},
% e.g. unpack the file into the current directory:
% \begin{quote}
%   \verb|pdftk hopatch.pdf unpack_files output .|
% \end{quote}
%
% \paragraph{Unpacking with \LaTeX.}
% The \xfile{.dtx} chooses its action depending on the format:
% \begin{description}
% \item[\plainTeX:] Run \docstrip\ and extract the files.
% \item[\LaTeX:] Generate the documentation.
% \end{description}
% If you insist on using \LaTeX\ for \docstrip\ (really,
% \docstrip\ does not need \LaTeX), then inform the autodetect routine
% about your intention:
% \begin{quote}
%   \verb|latex \let\install=y\input{hopatch.dtx}|
% \end{quote}
% Do not forget to quote the argument according to the demands
% of your shell.
%
% \paragraph{Generating the documentation.}
% You can use both the \xfile{.dtx} or the \xfile{.drv} to generate
% the documentation. The process can be configured by the
% configuration file \xfile{ltxdoc.cfg}. For instance, put this
% line into this file, if you want to have A4 as paper format:
% \begin{quote}
%   \verb|\PassOptionsToClass{a4paper}{article}|
% \end{quote}
% An example follows how to generate the
% documentation with pdf\LaTeX:
% \begin{quote}
%\begin{verbatim}
%pdflatex hopatch.dtx
%makeindex -s gind.ist hopatch.idx
%pdflatex hopatch.dtx
%makeindex -s gind.ist hopatch.idx
%pdflatex hopatch.dtx
%\end{verbatim}
% \end{quote}
%
% \section{Catalogue}
%
% The following XML file can be used as source for the
% \href{http://mirror.ctan.org/help/Catalogue/catalogue.html}{\TeX\ Catalogue}.
% The elements \texttt{caption} and \texttt{description} are imported
% from the original XML file from the Catalogue.
% The name of the XML file in the Catalogue is \xfile{hopatch.xml}.
%    \begin{macrocode}
%<*catalogue>
<?xml version='1.0' encoding='us-ascii'?>
<!DOCTYPE entry SYSTEM 'catalogue.dtd'>
<entry datestamp='$Date$' modifier='$Author$' id='hopatch'>
  <name>hopatch</name>
  <caption>Load patches for packages.</caption>
  <authorref id='auth:oberdiek'/>
  <copyright owner='Heiko Oberdiek' year='2011,2012'/>
  <license type='lppl1.3'/>
  <version number='1.2'/>
  <description>
    The hopatch package provides a command with which the user may
    register a piece of patch code for a particular package.  Hopatch
    will apply the patch immediately, if the relevant package has
    already been loaded; otherwise it will store the patch until the
    package appears.
    <p/>
    The package is part of the <xref refid='oberdiek'>oberdiek</xref> bundle.
  </description>
  <documentation details='Package documentation'
      href='ctan:/macros/latex/contrib/oberdiek/hopatch.pdf'/>
  <ctan file='true' path='/macros/latex/contrib/oberdiek/hopatch.dtx'/>
  <miktex location='oberdiek'/>
  <texlive location='oberdiek'/>
  <install path='/macros/latex/contrib/oberdiek/oberdiek.tds.zip'/>
</entry>
%</catalogue>
%    \end{macrocode}
%
% \begin{thebibliography}{9}
% \raggedright
%
% \bibitem{etoolbox}
% Philipp Lehman:
% \textit{The etoolbox Package}
% 2011-01-03.
% \CTAN{macros/latex/contrib/etoolbox/etoolbox.pdf}
%
% \bibitem{filehook}
% Martin Scharrer:
% \textit{The filehook Package};
% 2011-01-09.
% \CTAN{macros/latex/contrib/filehook/filehook.pdf}
%
% \bibitem{ltxcmds}
% Heiko Oberdiek:
% \textit{The ltxcmds Package};
% 2010-12-12.
% \CTAN{macros/latex/contrib/oberdiek/ltxcmds.pdf}
%
% \bibitem{memoir}
% Peter Wilson, Lars Madsen:
% \textit{The Memoir Class for Configurable Typesetting, User Guide};
% 2010.
% \CTAN{macros/latex/contrib/memoir/memman.pdf}
%
% \bibitem{scrlfile}
% Markus Kohm, Jens-Uwe Morawski:
% \textit{The Guide KOMA-Script};
% 2011-01-20.
% \CTAN{macros/latex/contrib/koma-script/scrguien.pdf}
%
% \end{thebibliography}
%
% \begin{History}
%   \begin{Version}{2011/01/30 v1.0}
%   \item
%     First public version.
%   \end{Version}
%   \begin{Version}{2011/06/24 v1.1}
%   \item
%     Fix the use of \cs{AtEndPreamble} and \cs{AfterEndPreamble}.
%     They are redefined by package \xpackage{etoolbox} after their
%     hooks are used and generate an error message then.
%   \end{Version}
%   \begin{Version}{2012/05/28 v1.2}
%   \item
%     Fix for use without \hologo{eTeX}
%     (thanks Gordon Lee).
%   \end{Version}
% \end{History}
%
% \PrintIndex
%
% \Finale
\endinput

%        (quote the arguments according to the demands of your shell)
%
% Documentation:
%    (a) If hopatch.drv is present:
%           latex hopatch.drv
%    (b) Without hopatch.drv:
%           latex hopatch.dtx; ...
%    The class ltxdoc loads the configuration file ltxdoc.cfg
%    if available. Here you can specify further options, e.g.
%    use A4 as paper format:
%       \PassOptionsToClass{a4paper}{article}
%
%    Programm calls to get the documentation (example):
%       pdflatex hopatch.dtx
%       makeindex -s gind.ist hopatch.idx
%       pdflatex hopatch.dtx
%       makeindex -s gind.ist hopatch.idx
%       pdflatex hopatch.dtx
%
% Installation:
%    TDS:tex/latex/oberdiek/hopatch.sty
%    TDS:doc/latex/oberdiek/hopatch.pdf
%    TDS:doc/latex/oberdiek/test/hopatch-test1.tex
%    TDS:doc/latex/oberdiek/test/hopatch-test2.tex
%    TDS:source/latex/oberdiek/hopatch.dtx
%
%<*ignore>
\begingroup
  \catcode123=1 %
  \catcode125=2 %
  \def\x{LaTeX2e}%
\expandafter\endgroup
\ifcase 0\ifx\install y1\fi\expandafter
         \ifx\csname processbatchFile\endcsname\relax\else1\fi
         \ifx\fmtname\x\else 1\fi\relax
\else\csname fi\endcsname
%</ignore>
%<*install>
\input docstrip.tex
\Msg{************************************************************************}
\Msg{* Installation}
\Msg{* Package: hopatch 2012/05/28 v1.2 Wrapper for package hooks (HO)}
\Msg{************************************************************************}

\keepsilent
\askforoverwritefalse

\let\MetaPrefix\relax
\preamble

This is a generated file.

Project: hopatch
Version: 2012/05/28 v1.2

Copyright (C) 2011, 2012 by
   Heiko Oberdiek <heiko.oberdiek at googlemail.com>

This work may be distributed and/or modified under the
conditions of the LaTeX Project Public License, either
version 1.3c of this license or (at your option) any later
version. This version of this license is in
   http://www.latex-project.org/lppl/lppl-1-3c.txt
and the latest version of this license is in
   http://www.latex-project.org/lppl.txt
and version 1.3 or later is part of all distributions of
LaTeX version 2005/12/01 or later.

This work has the LPPL maintenance status "maintained".

This Current Maintainer of this work is Heiko Oberdiek.

This work consists of the main source file hopatch.dtx
and the derived files
   hopatch.sty, hopatch.pdf, hopatch.ins, hopatch.drv,
   hopatch-test1.tex, hopatch-test2.tex.

\endpreamble
\let\MetaPrefix\DoubleperCent

\generate{%
  \file{hopatch.ins}{\from{hopatch.dtx}{install}}%
  \file{hopatch.drv}{\from{hopatch.dtx}{driver}}%
  \usedir{tex/latex/oberdiek}%
  \file{hopatch.sty}{\from{hopatch.dtx}{package}}%
  \usedir{doc/latex/oberdiek/test}%
  \file{hopatch-test1.tex}{\from{hopatch.dtx}{test1}}%
  \file{hopatch-test2.tex}{\from{hopatch.dtx}{test2}}%
  \nopreamble
  \nopostamble
  \usedir{source/latex/oberdiek/catalogue}%
  \file{hopatch.xml}{\from{hopatch.dtx}{catalogue}}%
}

\catcode32=13\relax% active space
\let =\space%
\Msg{************************************************************************}
\Msg{*}
\Msg{* To finish the installation you have to move the following}
\Msg{* file into a directory searched by TeX:}
\Msg{*}
\Msg{*     hopatch.sty}
\Msg{*}
\Msg{* To produce the documentation run the file `hopatch.drv'}
\Msg{* through LaTeX.}
\Msg{*}
\Msg{* Happy TeXing!}
\Msg{*}
\Msg{************************************************************************}

\endbatchfile
%</install>
%<*ignore>
\fi
%</ignore>
%<*driver>
\NeedsTeXFormat{LaTeX2e}
\ProvidesFile{hopatch.drv}%
  [2012/05/28 v1.2 Wrapper for package hooks (HO)]%
\documentclass{ltxdoc}
\usepackage{holtxdoc}[2011/11/22]
\usepackage{caption}
\usepackage{array}
\setlength{\extrarowheight}{1pt}
\renewcommand*{\arraystretch}{1.1}
\begin{document}
  \DocInput{hopatch.dtx}%
\end{document}
%</driver>
% \fi
%
% \CheckSum{466}
%
% \CharacterTable
%  {Upper-case    \A\B\C\D\E\F\G\H\I\J\K\L\M\N\O\P\Q\R\S\T\U\V\W\X\Y\Z
%   Lower-case    \a\b\c\d\e\f\g\h\i\j\k\l\m\n\o\p\q\r\s\t\u\v\w\x\y\z
%   Digits        \0\1\2\3\4\5\6\7\8\9
%   Exclamation   \!     Double quote  \"     Hash (number) \#
%   Dollar        \$     Percent       \%     Ampersand     \&
%   Acute accent  \'     Left paren    \(     Right paren   \)
%   Asterisk      \*     Plus          \+     Comma         \,
%   Minus         \-     Point         \.     Solidus       \/
%   Colon         \:     Semicolon     \;     Less than     \<
%   Equals        \=     Greater than  \>     Question mark \?
%   Commercial at \@     Left bracket  \[     Backslash     \\
%   Right bracket \]     Circumflex    \^     Underscore    \_
%   Grave accent  \`     Left brace    \{     Vertical bar  \|
%   Right brace   \}     Tilde         \~}
%
% \GetFileInfo{hopatch.drv}
%
% \title{The \xpackage{hopatch} package}
% \date{2012/05/28 v1.2}
% \author{Heiko Oberdiek\\\xemail{heiko.oberdiek at googlemail.com}}
%
% \maketitle
%
% \begin{abstract}
% This packages provides a wrapper to various package hooks
% provided by other packages or classes, but does not define
% own hooks.
% \end{abstract}
%
% \tableofcontents
%
% \section{Documentation}
%
% Sometimes I want to add code right after a package
% has been loaded. Examples are bug fixes, adaptations,
% or added features as needed by package \xpackage{hyperref},
% for instance.
%
% Unhappily \hologo{LaTeX} does not provide this kind of
% hook. \cs{AtEndOfPackage} can be used inside the package
% only, because \hologo{LaTeX} clears the hook right before
% it loads the package.
%
% However, there are already many packages and classes
% that provide hooks that are executed after the package
% is loaded, see table~\ref{tab:hooks}.
%
% \begin{table}
% \caption{After package hooking}
% \label{tab:hooks}
% \begin{tabular}{ll}
% \bfseries Macro & \bfseries Provider\\
% \hline
% \cs{AfterPackage} & package \xpackage{scrlfile} \cite{scrlfile}\\
% \cs{AtEndOfPackageFile} & package \xpackage{filehook} \cite{filehook}\\
% \cs{AtEndPackage} & class \xclass{memoir} \cite{memoir}
% \end{tabular}
% \end{table}
%
% Package \xpackage{hopatch} can be used without the packages
% of table~\ref{tab:hooks}. But for an early executing right
% after a package is loaded, one of the following class or
% packages should be loaded before using \cs{hopatch@AfterPackage}:
% \begin{itemize}
%  \item package \xpackage{filehook}
%  \item package \xpackage{scrlfile}
%  \item class \xclass{memoir}
% \end{itemize}
%
% \begin{table}
% \caption{After begin document hooking}
% \begin{tabular}{ll}
% \bfseries Macro & \bfseries Provider\\
% \hline
% \cs{AtBeginDocument} & \hologo{LaTeX}'s kernel\\
% \cs{AtEndPreamble} & package \xpackage{etoolbox} \cite{etoolbox}\\
% \cs{AfterEndPreamble} & package \xpackage{etoolbox}
% \end{tabular}
% \end{table}
%
% Therefore I skip writing a new package for hooking into
% \hologo{LaTeX}'s package management and use this package
% to provide a wrapper to patch a package after it is loaded.
%
% \begin{declcs}{hopatch@AfterPackage} \M{package} \M{patch code}
% \end{declcs}
% If the package is already loaded, the \meta{patch code} is executed
% immediately.
% Otherwise the \meta{patch code} is stored in a command and tried
% at later locations until the package is available.
%
% The patch is tried in the following order:
% \begin{enumerate}
% \item If the package is already loaded, the patch is
%   applied immediately. Further locations are not tried.
% \item \cs{AtEndPackage}, provided by class \xclass{memoir} \cite{memoir},
%   and \cs{AfterPackage}, provided by package \xpackage{scrlfile} \cite{scrlfile},
%   are called right after the package file is input before
%   the hook of \hologo{LaTeX}'s \cs{AtEndOfPackage}.
% \item \cs{AtEndOfPackageFile}, provided by package \xpackage{filehook} \cite{filehook},
%   is called after the package is loaded and after the hook of
%   \hologo{LaTeX}'s \cs{AtEndOfPackage}.
% \item \cs{AtEndPreamble}, provided by package \xpackage{etoolbox} \cite{etoolbox},
%   is called at the beginning of |\begin{document}| before the hook
%   of \hologo{LaTeX}'s \cs{AtBeginDocument}.
% \item \cs{AtBeginDocument}, provided by \hologo{LaTeX}.
% \item \cs{AfterEndDocument}, provided by package \xpackage{etoolbox} \cite{etoolbox},
%   is called at the very end of |\begin{document}|. Preamble commands are
%   already forbidden there.
% \end{enumerate}
% Because of the various locations the patch code is restricted to
% limitations:
% \begin{itemize}
% \item
% Preamble commands, see \hologo{LaTeX}'s \cs{@onlypreamble}
% throw an error if used after |\begin{document}|. This is already
% the case for \cs{AfterEndDocument}. Therefore preamble commands
% are forbidden in the patching code.
% There are four exceptions \cs{@ifpackageloaded}, \cs{@ifclassloaded},
% \cs{@ifpackagelater} and \cs{@ifclasslater}. They are redefined
% during \cs{AfterEndDocument} using the counterparts of
% package \xpackage{ltxcmds} \cite{ltxcmds}.
% \item
% \cs{AfterPackage} of package \xpackage{scrlfile} and \cs{AtEndPackage}
% of class \xclass{memoir} call the hook before \hologo{LaTeX}'s
% \cs{AtEndOfPackage}.
% \end{itemize}
%
% \StopEventually{
% }
%
% \section{Implementation}
%
%    \begin{macrocode}
%<*package>
%    \end{macrocode}
%
% \subsection{Catcodes and package identification}
%
%    \begin{macrocode}
\begingroup\catcode61\catcode48\catcode32=10\relax%
  \catcode13=5 % ^^M
  \endlinechar=13 %
  \catcode123=1 % {
  \catcode125=2 % }
  \catcode64=11 % @
  \def\x{\endgroup
    \expandafter\edef\csname HOpatch@AtEnd\endcsname{%
      \endlinechar=\the\endlinechar\relax
      \catcode13=\the\catcode13\relax
      \catcode32=\the\catcode32\relax
      \catcode35=\the\catcode35\relax
      \catcode61=\the\catcode61\relax
      \catcode64=\the\catcode64\relax
      \catcode123=\the\catcode123\relax
      \catcode125=\the\catcode125\relax
    }%
  }%
\x\catcode61\catcode48\catcode32=10\relax%
\catcode13=5 % ^^M
\endlinechar=13 %
\catcode35=6 % #
\catcode64=11 % @
\catcode123=1 % {
\catcode125=2 % }
\def\TMP@EnsureCode#1#2{%
  \edef\HOpatch@AtEnd{%
    \HOpatch@AtEnd
    \catcode#1=\the\catcode#1\relax
  }%
  \catcode#1=#2\relax
}
\TMP@EnsureCode{40}{12}% (
\TMP@EnsureCode{41}{12}% )
\TMP@EnsureCode{43}{12}% +
\TMP@EnsureCode{46}{12}% .
\TMP@EnsureCode{47}{12}% /
\TMP@EnsureCode{91}{12}% [
\TMP@EnsureCode{93}{12}% ]
\edef\HOpatch@AtEnd{\HOpatch@AtEnd\noexpand\endinput}
%    \end{macrocode}
%
%    Package identification.
%    \begin{macrocode}
\NeedsTeXFormat{LaTeX2e}
\ProvidesPackage{hopatch}%
  [2012/05/28 v1.2 Wrapper for package hooks (HO)]
%    \end{macrocode}
%
% \subsection{Resources}
%
%    \begin{macrocode}
\begingroup\expandafter\expandafter\expandafter\endgroup
\expandafter\ifx\csname RequirePackage\endcsname\relax
  \def\TMP@RequirePackage#1[#2]{%
    \begingroup\expandafter\expandafter\expandafter\endgroup
    \expandafter\ifx\csname ver@#1.sty\endcsname\relax
      \input #1.sty\relax
    \fi
  }%
  \TMP@RequirePackage{ltxcmds}[2010/12/12]%
\else
  \RequirePackage{ltxcmds}[2010/12/12]%
\fi
%    \end{macrocode}
%
%    \begin{macro}{\HOpatch@counter}
%    \begin{macrocode}
\def\HOpatch@counter{0}%
%    \end{macrocode}
%    \end{macro}
%    \begin{macro}{\HOpatch@StepCounter}
%    \begin{macrocode}
\ltx@ifundefined{numexpr}{%
  \def\HOpatch@StepCounter{%
    \begingroup
      \count@\HOpatch@counter\relax
      \advance\count@\ltx@one\relax
    \edef\x{\endgroup
      \noexpand\def\noexpand\HOpatch@counter{\the\count@}%
    }%
    \x
  }%
}{%
  \def\HOpatch@StepCounter{%
    \edef\HOpatch@counter{%
      \the\numexpr\HOpatch@counter+\ltx@one\relax
    }%
  }%
}
%    \end{macrocode}
%    \end{macro}
%
%    \begin{macro}{\HOpatch@list}
%    \begin{macrocode}
\def\HOpatch@list{}
%    \end{macrocode}
%    \end{macro}
%
%    \begin{macro}{\HOpatch@Add}
%    \begin{macrocode}
\def\HOpatch@Add{%
  \ltx@LocalAppendToMacro\HOpatch@list
}
%    \end{macrocode}
%    \end{macro}
%
% \subsection{Package patching}
%
%    \begin{macro}{\hopatch@AfterPackage}
%    \begin{macrocode}
\def\hopatch@AfterPackage#1{%
  \ltx@ifpackageloaded{#1}{%
    \ltx@firstofone
  }{%
    \HOpatch@AfterPackage{#1}%
  }%
}
%    \end{macrocode}
%    \end{macro}
%    \begin{macro}{\HOpatch@AfterPackage}
%    \begin{macrocode}
\def\HOpatch@AfterPackage#1{%
  \edef\HOpatch@temp{#1}%
  \HOpatch@StepCounter
  \expandafter\HOpatch@@AfterPackage
  \csname HOpatch@\HOpatch@counter\expandafter\endcsname{%
    \HOpatch@temp
  }%
}
%    \end{macrocode}
%    \end{macro}
%
%    \begin{macro}{\HOpatch@@AfterPackage}
%    \begin{macrocode}
\def\HOpatch@@AfterPackage#1#2#3{%
  \begingroup
    \toks@{#3}%
    \xdef\HOpatch@gtemp{%
      \noexpand\ltx@ifpackageloaded{#2}{%
        \noexpand\let\noexpand#1\noexpand\relax
        \the\toks@
      }{}%
    }%
  \endgroup
  \let#1\HOpatch@gtemp
  \HOpatch@Add#1%
  \HOpatch@Try{AfterPackage}{#2}#1%
  \HOpatch@Try{AtEndPackage}{#2}#1%
  \HOpatch@Try{AtEndOfPackageFile}{#2}#1%
}
%    \end{macrocode}
%    \end{macro}
%
%    \begin{macro}{\HOpatch@Try}
%    \begin{macrocode}
\def\HOpatch@Try#1#2#3{%
  \ltx@ifundefined{#1}{}{%
    \csname #1\endcsname{#2}{#3}%
  }%
}
%    \end{macrocode}
%    \end{macro}
%
%    \begin{macrocode}
\AtBeginDocument{\HOpatch@list}
\ltx@ifundefined{AtEndPreamble}{}{%
  \ltx@ifundefined{@endpreamblehook}{}{%
    \AtEndPreamble{\HOpatch@list}%
  }%
}
\ltx@ifundefined{AfterEndPreamble}{}{%
  \ltx@ifundefined{@afterendpreamblehook}{}{%
    \AfterEndPreamble{%
      \let\HOpatch@OrgIfPackageLoaded\@ifpackageloaded
      \let\HOpatch@OrgIfPackageLater\@ifpackagelater
      \let\HOpatch@OrgIfClassLoaded\@ifclassloaded
      \let\HOpatch@OrgIfClassLater\@ifclasslater
      \let\@ifpackageloaded\ltx@ifpackageloaded
      \let\@ifpackagelater\ltx@ifpackagelater
      \let\@ifclassloaded\ltx@ifclassloaded
      \let\@ifclasslater\ltx@ifclasslater
      \HOpatch@list
      \let\@ifpackageloaded\HOpatch@OrgIfPackageLoaded
      \let\@ifpackagelater\HOpatch@OrgIfPackageLater
      \let\@ifclassloaded\HOpatch@OrgIfClassLoaded
      \let\@ifclasslater\HOpatch@OrgIfClassLater
    }%
  }%
}
%    \end{macrocode}
%
%    \begin{macrocode}
\HOpatch@AtEnd%
%    \end{macrocode}
%    \begin{macrocode}
%</package>
%    \end{macrocode}
%
% \section{Test}
%
%    \begin{macrocode}
%<*test1>
\def\LoadCommand{\RequirePackage{hopatch}[2012/05/28]}
%</test1>
%    \end{macrocode}
% \subsection{Catcode checks for loading}
%
%    \begin{macrocode}
%<*test1>
%    \end{macrocode}
%    \begin{macrocode}
\catcode`\{=1 %
\catcode`\}=2 %
\catcode`\#=6 %
\catcode`\@=11 %
\expandafter\ifx\csname count@\endcsname\relax
  \countdef\count@=255 %
\fi
\expandafter\ifx\csname @gobble\endcsname\relax
  \long\def\@gobble#1{}%
\fi
\expandafter\ifx\csname @firstofone\endcsname\relax
  \long\def\@firstofone#1{#1}%
\fi
\expandafter\ifx\csname loop\endcsname\relax
  \expandafter\@firstofone
\else
  \expandafter\@gobble
\fi
{%
  \def\loop#1\repeat{%
    \def\body{#1}%
    \iterate
  }%
  \def\iterate{%
    \body
      \let\next\iterate
    \else
      \let\next\relax
    \fi
    \next
  }%
  \let\repeat=\fi
}%
\def\RestoreCatcodes{}
\count@=0 %
\loop
  \edef\RestoreCatcodes{%
    \RestoreCatcodes
    \catcode\the\count@=\the\catcode\count@\relax
  }%
\ifnum\count@<255 %
  \advance\count@ 1 %
\repeat

\def\RangeCatcodeInvalid#1#2{%
  \count@=#1\relax
  \loop
    \catcode\count@=15 %
  \ifnum\count@<#2\relax
    \advance\count@ 1 %
  \repeat
}
\def\RangeCatcodeCheck#1#2#3{%
  \count@=#1\relax
  \loop
    \ifnum#3=\catcode\count@
    \else
      \errmessage{%
        Character \the\count@\space
        with wrong catcode \the\catcode\count@\space
        instead of \number#3%
      }%
    \fi
  \ifnum\count@<#2\relax
    \advance\count@ 1 %
  \repeat
}
\def\space{ }
\expandafter\ifx\csname LoadCommand\endcsname\relax
  \def\LoadCommand{\input hopatch.sty\relax}%
\fi
\def\Test{%
  \RangeCatcodeInvalid{0}{47}%
  \RangeCatcodeInvalid{58}{64}%
  \RangeCatcodeInvalid{91}{96}%
  \RangeCatcodeInvalid{123}{255}%
  \catcode`\@=12 %
  \catcode`\\=0 %
  \catcode`\%=14 %
  \LoadCommand
  \RangeCatcodeCheck{0}{36}{15}%
  \RangeCatcodeCheck{37}{37}{14}%
  \RangeCatcodeCheck{38}{47}{15}%
  \RangeCatcodeCheck{48}{57}{12}%
  \RangeCatcodeCheck{58}{63}{15}%
  \RangeCatcodeCheck{64}{64}{12}%
  \RangeCatcodeCheck{65}{90}{11}%
  \RangeCatcodeCheck{91}{91}{15}%
  \RangeCatcodeCheck{92}{92}{0}%
  \RangeCatcodeCheck{93}{96}{15}%
  \RangeCatcodeCheck{97}{122}{11}%
  \RangeCatcodeCheck{123}{255}{15}%
  \RestoreCatcodes
}
\Test
\csname @@end\endcsname
\end
%    \end{macrocode}
%    \begin{macrocode}
%</test1>
%    \end{macrocode}
%
%    \begin{macrocode}
%<*test2>
\NeedsTeXFormat{LaTeX2e}
\providecommand\variant{0}
\RequirePackage{filecontents}
\begin{filecontents}{foo.sty}
\ProvidesPackage{foo}
\def\msg#{\immediate\write16}
\def\foo#1{%
  \msg{\fooformat{#1}}%
}
\def\fooformat#1{[#1]}% hash-ok
\foo{* Executing foo at package loading}
\end{filecontents}

\ifnum\variant=1 %
  \documentclass{memoir}%
\else
  \documentclass{article}%
\fi

\ifcase\variant\relax
\or % 1
\or % 2
  \usepackage{etoolbox}%
\or % 3
  \usepackage{scrlfile}%
\or % 4
  \usepackage{filehook}%
\fi

\AtBeginDocument{\foo{* AtBeginDocument before hopatch}}
\usepackage{hopatch}
\AtBeginDocument{\foo{* AtBeginDocument after hopatch}}

\makeatletter
\hopatch@AfterPackage{foo}{%
  \def\fooformat#1{<<#1>>}%
}
\makeatother

\AtBeginDocument{\foo{* AtBeginDocument before foo}}
\usepackage{foo}
\AtBeginDocument{\foo{* AtBeginDocument after foo}}

\foo{* Executing in preamble}

\begin{document}
\foo{* Executing in document}
\end{document}
%</test2>
%    \end{macrocode}
%
% \section{Installation}
%
% \subsection{Download}
%
% \paragraph{Package.} This package is available on
% CTAN\footnote{\url{ftp://ftp.ctan.org/tex-archive/}}:
% \begin{description}
% \item[\CTAN{macros/latex/contrib/oberdiek/hopatch.dtx}] The source file.
% \item[\CTAN{macros/latex/contrib/oberdiek/hopatch.pdf}] Documentation.
% \end{description}
%
%
% \paragraph{Bundle.} All the packages of the bundle `oberdiek'
% are also available in a TDS compliant ZIP archive. There
% the packages are already unpacked and the documentation files
% are generated. The files and directories obey the TDS standard.
% \begin{description}
% \item[\CTAN{install/macros/latex/contrib/oberdiek.tds.zip}]
% \end{description}
% \emph{TDS} refers to the standard ``A Directory Structure
% for \TeX\ Files'' (\CTAN{tds/tds.pdf}). Directories
% with \xfile{texmf} in their name are usually organized this way.
%
% \subsection{Bundle installation}
%
% \paragraph{Unpacking.} Unpack the \xfile{oberdiek.tds.zip} in the
% TDS tree (also known as \xfile{texmf} tree) of your choice.
% Example (linux):
% \begin{quote}
%   |unzip oberdiek.tds.zip -d ~/texmf|
% \end{quote}
%
% \paragraph{Script installation.}
% Check the directory \xfile{TDS:scripts/oberdiek/} for
% scripts that need further installation steps.
% Package \xpackage{attachfile2} comes with the Perl script
% \xfile{pdfatfi.pl} that should be installed in such a way
% that it can be called as \texttt{pdfatfi}.
% Example (linux):
% \begin{quote}
%   |chmod +x scripts/oberdiek/pdfatfi.pl|\\
%   |cp scripts/oberdiek/pdfatfi.pl /usr/local/bin/|
% \end{quote}
%
% \subsection{Package installation}
%
% \paragraph{Unpacking.} The \xfile{.dtx} file is a self-extracting
% \docstrip\ archive. The files are extracted by running the
% \xfile{.dtx} through \plainTeX:
% \begin{quote}
%   \verb|tex hopatch.dtx|
% \end{quote}
%
% \paragraph{TDS.} Now the different files must be moved into
% the different directories in your installation TDS tree
% (also known as \xfile{texmf} tree):
% \begin{quote}
% \def\t{^^A
% \begin{tabular}{@{}>{\ttfamily}l@{ $\rightarrow$ }>{\ttfamily}l@{}}
%   hopatch.sty & tex/latex/oberdiek/hopatch.sty\\
%   hopatch.pdf & doc/latex/oberdiek/hopatch.pdf\\
%   test/hopatch-test1.tex & doc/latex/oberdiek/test/hopatch-test1.tex\\
%   test/hopatch-test2.tex & doc/latex/oberdiek/test/hopatch-test2.tex\\
%   hopatch.dtx & source/latex/oberdiek/hopatch.dtx\\
% \end{tabular}^^A
% }^^A
% \sbox0{\t}^^A
% \ifdim\wd0>\linewidth
%   \begingroup
%     \advance\linewidth by\leftmargin
%     \advance\linewidth by\rightmargin
%   \edef\x{\endgroup
%     \def\noexpand\lw{\the\linewidth}^^A
%   }\x
%   \def\lwbox{^^A
%     \leavevmode
%     \hbox to \linewidth{^^A
%       \kern-\leftmargin\relax
%       \hss
%       \usebox0
%       \hss
%       \kern-\rightmargin\relax
%     }^^A
%   }^^A
%   \ifdim\wd0>\lw
%     \sbox0{\small\t}^^A
%     \ifdim\wd0>\linewidth
%       \ifdim\wd0>\lw
%         \sbox0{\footnotesize\t}^^A
%         \ifdim\wd0>\linewidth
%           \ifdim\wd0>\lw
%             \sbox0{\scriptsize\t}^^A
%             \ifdim\wd0>\linewidth
%               \ifdim\wd0>\lw
%                 \sbox0{\tiny\t}^^A
%                 \ifdim\wd0>\linewidth
%                   \lwbox
%                 \else
%                   \usebox0
%                 \fi
%               \else
%                 \lwbox
%               \fi
%             \else
%               \usebox0
%             \fi
%           \else
%             \lwbox
%           \fi
%         \else
%           \usebox0
%         \fi
%       \else
%         \lwbox
%       \fi
%     \else
%       \usebox0
%     \fi
%   \else
%     \lwbox
%   \fi
% \else
%   \usebox0
% \fi
% \end{quote}
% If you have a \xfile{docstrip.cfg} that configures and enables \docstrip's
% TDS installing feature, then some files can already be in the right
% place, see the documentation of \docstrip.
%
% \subsection{Refresh file name databases}
%
% If your \TeX~distribution
% (\teTeX, \mikTeX, \dots) relies on file name databases, you must refresh
% these. For example, \teTeX\ users run \verb|texhash| or
% \verb|mktexlsr|.
%
% \subsection{Some details for the interested}
%
% \paragraph{Attached source.}
%
% The PDF documentation on CTAN also includes the
% \xfile{.dtx} source file. It can be extracted by
% AcrobatReader 6 or higher. Another option is \textsf{pdftk},
% e.g. unpack the file into the current directory:
% \begin{quote}
%   \verb|pdftk hopatch.pdf unpack_files output .|
% \end{quote}
%
% \paragraph{Unpacking with \LaTeX.}
% The \xfile{.dtx} chooses its action depending on the format:
% \begin{description}
% \item[\plainTeX:] Run \docstrip\ and extract the files.
% \item[\LaTeX:] Generate the documentation.
% \end{description}
% If you insist on using \LaTeX\ for \docstrip\ (really,
% \docstrip\ does not need \LaTeX), then inform the autodetect routine
% about your intention:
% \begin{quote}
%   \verb|latex \let\install=y% \iffalse meta-comment
%
% File: hopatch.dtx
% Version: 2012/05/28 v1.2
% Info: Wrapper for package hooks
%
% Copyright (C) 2011, 2012 by
%    Heiko Oberdiek <heiko.oberdiek at googlemail.com>
%
% This work may be distributed and/or modified under the
% conditions of the LaTeX Project Public License, either
% version 1.3c of this license or (at your option) any later
% version. This version of this license is in
%    http://www.latex-project.org/lppl/lppl-1-3c.txt
% and the latest version of this license is in
%    http://www.latex-project.org/lppl.txt
% and version 1.3 or later is part of all distributions of
% LaTeX version 2005/12/01 or later.
%
% This work has the LPPL maintenance status "maintained".
%
% This Current Maintainer of this work is Heiko Oberdiek.
%
% This work consists of the main source file hopatch.dtx
% and the derived files
%    hopatch.sty, hopatch.pdf, hopatch.ins, hopatch.drv,
%    hopatch-test1.tex, hopatch-test2.tex.
%
% Distribution:
%    CTAN:macros/latex/contrib/oberdiek/hopatch.dtx
%    CTAN:macros/latex/contrib/oberdiek/hopatch.pdf
%
% Unpacking:
%    (a) If hopatch.ins is present:
%           tex hopatch.ins
%    (b) Without hopatch.ins:
%           tex hopatch.dtx
%    (c) If you insist on using LaTeX
%           latex \let\install=y\input{hopatch.dtx}
%        (quote the arguments according to the demands of your shell)
%
% Documentation:
%    (a) If hopatch.drv is present:
%           latex hopatch.drv
%    (b) Without hopatch.drv:
%           latex hopatch.dtx; ...
%    The class ltxdoc loads the configuration file ltxdoc.cfg
%    if available. Here you can specify further options, e.g.
%    use A4 as paper format:
%       \PassOptionsToClass{a4paper}{article}
%
%    Programm calls to get the documentation (example):
%       pdflatex hopatch.dtx
%       makeindex -s gind.ist hopatch.idx
%       pdflatex hopatch.dtx
%       makeindex -s gind.ist hopatch.idx
%       pdflatex hopatch.dtx
%
% Installation:
%    TDS:tex/latex/oberdiek/hopatch.sty
%    TDS:doc/latex/oberdiek/hopatch.pdf
%    TDS:doc/latex/oberdiek/test/hopatch-test1.tex
%    TDS:doc/latex/oberdiek/test/hopatch-test2.tex
%    TDS:source/latex/oberdiek/hopatch.dtx
%
%<*ignore>
\begingroup
  \catcode123=1 %
  \catcode125=2 %
  \def\x{LaTeX2e}%
\expandafter\endgroup
\ifcase 0\ifx\install y1\fi\expandafter
         \ifx\csname processbatchFile\endcsname\relax\else1\fi
         \ifx\fmtname\x\else 1\fi\relax
\else\csname fi\endcsname
%</ignore>
%<*install>
\input docstrip.tex
\Msg{************************************************************************}
\Msg{* Installation}
\Msg{* Package: hopatch 2012/05/28 v1.2 Wrapper for package hooks (HO)}
\Msg{************************************************************************}

\keepsilent
\askforoverwritefalse

\let\MetaPrefix\relax
\preamble

This is a generated file.

Project: hopatch
Version: 2012/05/28 v1.2

Copyright (C) 2011, 2012 by
   Heiko Oberdiek <heiko.oberdiek at googlemail.com>

This work may be distributed and/or modified under the
conditions of the LaTeX Project Public License, either
version 1.3c of this license or (at your option) any later
version. This version of this license is in
   http://www.latex-project.org/lppl/lppl-1-3c.txt
and the latest version of this license is in
   http://www.latex-project.org/lppl.txt
and version 1.3 or later is part of all distributions of
LaTeX version 2005/12/01 or later.

This work has the LPPL maintenance status "maintained".

This Current Maintainer of this work is Heiko Oberdiek.

This work consists of the main source file hopatch.dtx
and the derived files
   hopatch.sty, hopatch.pdf, hopatch.ins, hopatch.drv,
   hopatch-test1.tex, hopatch-test2.tex.

\endpreamble
\let\MetaPrefix\DoubleperCent

\generate{%
  \file{hopatch.ins}{\from{hopatch.dtx}{install}}%
  \file{hopatch.drv}{\from{hopatch.dtx}{driver}}%
  \usedir{tex/latex/oberdiek}%
  \file{hopatch.sty}{\from{hopatch.dtx}{package}}%
  \usedir{doc/latex/oberdiek/test}%
  \file{hopatch-test1.tex}{\from{hopatch.dtx}{test1}}%
  \file{hopatch-test2.tex}{\from{hopatch.dtx}{test2}}%
  \nopreamble
  \nopostamble
  \usedir{source/latex/oberdiek/catalogue}%
  \file{hopatch.xml}{\from{hopatch.dtx}{catalogue}}%
}

\catcode32=13\relax% active space
\let =\space%
\Msg{************************************************************************}
\Msg{*}
\Msg{* To finish the installation you have to move the following}
\Msg{* file into a directory searched by TeX:}
\Msg{*}
\Msg{*     hopatch.sty}
\Msg{*}
\Msg{* To produce the documentation run the file `hopatch.drv'}
\Msg{* through LaTeX.}
\Msg{*}
\Msg{* Happy TeXing!}
\Msg{*}
\Msg{************************************************************************}

\endbatchfile
%</install>
%<*ignore>
\fi
%</ignore>
%<*driver>
\NeedsTeXFormat{LaTeX2e}
\ProvidesFile{hopatch.drv}%
  [2012/05/28 v1.2 Wrapper for package hooks (HO)]%
\documentclass{ltxdoc}
\usepackage{holtxdoc}[2011/11/22]
\usepackage{caption}
\usepackage{array}
\setlength{\extrarowheight}{1pt}
\renewcommand*{\arraystretch}{1.1}
\begin{document}
  \DocInput{hopatch.dtx}%
\end{document}
%</driver>
% \fi
%
% \CheckSum{466}
%
% \CharacterTable
%  {Upper-case    \A\B\C\D\E\F\G\H\I\J\K\L\M\N\O\P\Q\R\S\T\U\V\W\X\Y\Z
%   Lower-case    \a\b\c\d\e\f\g\h\i\j\k\l\m\n\o\p\q\r\s\t\u\v\w\x\y\z
%   Digits        \0\1\2\3\4\5\6\7\8\9
%   Exclamation   \!     Double quote  \"     Hash (number) \#
%   Dollar        \$     Percent       \%     Ampersand     \&
%   Acute accent  \'     Left paren    \(     Right paren   \)
%   Asterisk      \*     Plus          \+     Comma         \,
%   Minus         \-     Point         \.     Solidus       \/
%   Colon         \:     Semicolon     \;     Less than     \<
%   Equals        \=     Greater than  \>     Question mark \?
%   Commercial at \@     Left bracket  \[     Backslash     \\
%   Right bracket \]     Circumflex    \^     Underscore    \_
%   Grave accent  \`     Left brace    \{     Vertical bar  \|
%   Right brace   \}     Tilde         \~}
%
% \GetFileInfo{hopatch.drv}
%
% \title{The \xpackage{hopatch} package}
% \date{2012/05/28 v1.2}
% \author{Heiko Oberdiek\\\xemail{heiko.oberdiek at googlemail.com}}
%
% \maketitle
%
% \begin{abstract}
% This packages provides a wrapper to various package hooks
% provided by other packages or classes, but does not define
% own hooks.
% \end{abstract}
%
% \tableofcontents
%
% \section{Documentation}
%
% Sometimes I want to add code right after a package
% has been loaded. Examples are bug fixes, adaptations,
% or added features as needed by package \xpackage{hyperref},
% for instance.
%
% Unhappily \hologo{LaTeX} does not provide this kind of
% hook. \cs{AtEndOfPackage} can be used inside the package
% only, because \hologo{LaTeX} clears the hook right before
% it loads the package.
%
% However, there are already many packages and classes
% that provide hooks that are executed after the package
% is loaded, see table~\ref{tab:hooks}.
%
% \begin{table}
% \caption{After package hooking}
% \label{tab:hooks}
% \begin{tabular}{ll}
% \bfseries Macro & \bfseries Provider\\
% \hline
% \cs{AfterPackage} & package \xpackage{scrlfile} \cite{scrlfile}\\
% \cs{AtEndOfPackageFile} & package \xpackage{filehook} \cite{filehook}\\
% \cs{AtEndPackage} & class \xclass{memoir} \cite{memoir}
% \end{tabular}
% \end{table}
%
% Package \xpackage{hopatch} can be used without the packages
% of table~\ref{tab:hooks}. But for an early executing right
% after a package is loaded, one of the following class or
% packages should be loaded before using \cs{hopatch@AfterPackage}:
% \begin{itemize}
%  \item package \xpackage{filehook}
%  \item package \xpackage{scrlfile}
%  \item class \xclass{memoir}
% \end{itemize}
%
% \begin{table}
% \caption{After begin document hooking}
% \begin{tabular}{ll}
% \bfseries Macro & \bfseries Provider\\
% \hline
% \cs{AtBeginDocument} & \hologo{LaTeX}'s kernel\\
% \cs{AtEndPreamble} & package \xpackage{etoolbox} \cite{etoolbox}\\
% \cs{AfterEndPreamble} & package \xpackage{etoolbox}
% \end{tabular}
% \end{table}
%
% Therefore I skip writing a new package for hooking into
% \hologo{LaTeX}'s package management and use this package
% to provide a wrapper to patch a package after it is loaded.
%
% \begin{declcs}{hopatch@AfterPackage} \M{package} \M{patch code}
% \end{declcs}
% If the package is already loaded, the \meta{patch code} is executed
% immediately.
% Otherwise the \meta{patch code} is stored in a command and tried
% at later locations until the package is available.
%
% The patch is tried in the following order:
% \begin{enumerate}
% \item If the package is already loaded, the patch is
%   applied immediately. Further locations are not tried.
% \item \cs{AtEndPackage}, provided by class \xclass{memoir} \cite{memoir},
%   and \cs{AfterPackage}, provided by package \xpackage{scrlfile} \cite{scrlfile},
%   are called right after the package file is input before
%   the hook of \hologo{LaTeX}'s \cs{AtEndOfPackage}.
% \item \cs{AtEndOfPackageFile}, provided by package \xpackage{filehook} \cite{filehook},
%   is called after the package is loaded and after the hook of
%   \hologo{LaTeX}'s \cs{AtEndOfPackage}.
% \item \cs{AtEndPreamble}, provided by package \xpackage{etoolbox} \cite{etoolbox},
%   is called at the beginning of |\begin{document}| before the hook
%   of \hologo{LaTeX}'s \cs{AtBeginDocument}.
% \item \cs{AtBeginDocument}, provided by \hologo{LaTeX}.
% \item \cs{AfterEndDocument}, provided by package \xpackage{etoolbox} \cite{etoolbox},
%   is called at the very end of |\begin{document}|. Preamble commands are
%   already forbidden there.
% \end{enumerate}
% Because of the various locations the patch code is restricted to
% limitations:
% \begin{itemize}
% \item
% Preamble commands, see \hologo{LaTeX}'s \cs{@onlypreamble}
% throw an error if used after |\begin{document}|. This is already
% the case for \cs{AfterEndDocument}. Therefore preamble commands
% are forbidden in the patching code.
% There are four exceptions \cs{@ifpackageloaded}, \cs{@ifclassloaded},
% \cs{@ifpackagelater} and \cs{@ifclasslater}. They are redefined
% during \cs{AfterEndDocument} using the counterparts of
% package \xpackage{ltxcmds} \cite{ltxcmds}.
% \item
% \cs{AfterPackage} of package \xpackage{scrlfile} and \cs{AtEndPackage}
% of class \xclass{memoir} call the hook before \hologo{LaTeX}'s
% \cs{AtEndOfPackage}.
% \end{itemize}
%
% \StopEventually{
% }
%
% \section{Implementation}
%
%    \begin{macrocode}
%<*package>
%    \end{macrocode}
%
% \subsection{Catcodes and package identification}
%
%    \begin{macrocode}
\begingroup\catcode61\catcode48\catcode32=10\relax%
  \catcode13=5 % ^^M
  \endlinechar=13 %
  \catcode123=1 % {
  \catcode125=2 % }
  \catcode64=11 % @
  \def\x{\endgroup
    \expandafter\edef\csname HOpatch@AtEnd\endcsname{%
      \endlinechar=\the\endlinechar\relax
      \catcode13=\the\catcode13\relax
      \catcode32=\the\catcode32\relax
      \catcode35=\the\catcode35\relax
      \catcode61=\the\catcode61\relax
      \catcode64=\the\catcode64\relax
      \catcode123=\the\catcode123\relax
      \catcode125=\the\catcode125\relax
    }%
  }%
\x\catcode61\catcode48\catcode32=10\relax%
\catcode13=5 % ^^M
\endlinechar=13 %
\catcode35=6 % #
\catcode64=11 % @
\catcode123=1 % {
\catcode125=2 % }
\def\TMP@EnsureCode#1#2{%
  \edef\HOpatch@AtEnd{%
    \HOpatch@AtEnd
    \catcode#1=\the\catcode#1\relax
  }%
  \catcode#1=#2\relax
}
\TMP@EnsureCode{40}{12}% (
\TMP@EnsureCode{41}{12}% )
\TMP@EnsureCode{43}{12}% +
\TMP@EnsureCode{46}{12}% .
\TMP@EnsureCode{47}{12}% /
\TMP@EnsureCode{91}{12}% [
\TMP@EnsureCode{93}{12}% ]
\edef\HOpatch@AtEnd{\HOpatch@AtEnd\noexpand\endinput}
%    \end{macrocode}
%
%    Package identification.
%    \begin{macrocode}
\NeedsTeXFormat{LaTeX2e}
\ProvidesPackage{hopatch}%
  [2012/05/28 v1.2 Wrapper for package hooks (HO)]
%    \end{macrocode}
%
% \subsection{Resources}
%
%    \begin{macrocode}
\begingroup\expandafter\expandafter\expandafter\endgroup
\expandafter\ifx\csname RequirePackage\endcsname\relax
  \def\TMP@RequirePackage#1[#2]{%
    \begingroup\expandafter\expandafter\expandafter\endgroup
    \expandafter\ifx\csname ver@#1.sty\endcsname\relax
      \input #1.sty\relax
    \fi
  }%
  \TMP@RequirePackage{ltxcmds}[2010/12/12]%
\else
  \RequirePackage{ltxcmds}[2010/12/12]%
\fi
%    \end{macrocode}
%
%    \begin{macro}{\HOpatch@counter}
%    \begin{macrocode}
\def\HOpatch@counter{0}%
%    \end{macrocode}
%    \end{macro}
%    \begin{macro}{\HOpatch@StepCounter}
%    \begin{macrocode}
\ltx@ifundefined{numexpr}{%
  \def\HOpatch@StepCounter{%
    \begingroup
      \count@\HOpatch@counter\relax
      \advance\count@\ltx@one\relax
    \edef\x{\endgroup
      \noexpand\def\noexpand\HOpatch@counter{\the\count@}%
    }%
    \x
  }%
}{%
  \def\HOpatch@StepCounter{%
    \edef\HOpatch@counter{%
      \the\numexpr\HOpatch@counter+\ltx@one\relax
    }%
  }%
}
%    \end{macrocode}
%    \end{macro}
%
%    \begin{macro}{\HOpatch@list}
%    \begin{macrocode}
\def\HOpatch@list{}
%    \end{macrocode}
%    \end{macro}
%
%    \begin{macro}{\HOpatch@Add}
%    \begin{macrocode}
\def\HOpatch@Add{%
  \ltx@LocalAppendToMacro\HOpatch@list
}
%    \end{macrocode}
%    \end{macro}
%
% \subsection{Package patching}
%
%    \begin{macro}{\hopatch@AfterPackage}
%    \begin{macrocode}
\def\hopatch@AfterPackage#1{%
  \ltx@ifpackageloaded{#1}{%
    \ltx@firstofone
  }{%
    \HOpatch@AfterPackage{#1}%
  }%
}
%    \end{macrocode}
%    \end{macro}
%    \begin{macro}{\HOpatch@AfterPackage}
%    \begin{macrocode}
\def\HOpatch@AfterPackage#1{%
  \edef\HOpatch@temp{#1}%
  \HOpatch@StepCounter
  \expandafter\HOpatch@@AfterPackage
  \csname HOpatch@\HOpatch@counter\expandafter\endcsname{%
    \HOpatch@temp
  }%
}
%    \end{macrocode}
%    \end{macro}
%
%    \begin{macro}{\HOpatch@@AfterPackage}
%    \begin{macrocode}
\def\HOpatch@@AfterPackage#1#2#3{%
  \begingroup
    \toks@{#3}%
    \xdef\HOpatch@gtemp{%
      \noexpand\ltx@ifpackageloaded{#2}{%
        \noexpand\let\noexpand#1\noexpand\relax
        \the\toks@
      }{}%
    }%
  \endgroup
  \let#1\HOpatch@gtemp
  \HOpatch@Add#1%
  \HOpatch@Try{AfterPackage}{#2}#1%
  \HOpatch@Try{AtEndPackage}{#2}#1%
  \HOpatch@Try{AtEndOfPackageFile}{#2}#1%
}
%    \end{macrocode}
%    \end{macro}
%
%    \begin{macro}{\HOpatch@Try}
%    \begin{macrocode}
\def\HOpatch@Try#1#2#3{%
  \ltx@ifundefined{#1}{}{%
    \csname #1\endcsname{#2}{#3}%
  }%
}
%    \end{macrocode}
%    \end{macro}
%
%    \begin{macrocode}
\AtBeginDocument{\HOpatch@list}
\ltx@ifundefined{AtEndPreamble}{}{%
  \ltx@ifundefined{@endpreamblehook}{}{%
    \AtEndPreamble{\HOpatch@list}%
  }%
}
\ltx@ifundefined{AfterEndPreamble}{}{%
  \ltx@ifundefined{@afterendpreamblehook}{}{%
    \AfterEndPreamble{%
      \let\HOpatch@OrgIfPackageLoaded\@ifpackageloaded
      \let\HOpatch@OrgIfPackageLater\@ifpackagelater
      \let\HOpatch@OrgIfClassLoaded\@ifclassloaded
      \let\HOpatch@OrgIfClassLater\@ifclasslater
      \let\@ifpackageloaded\ltx@ifpackageloaded
      \let\@ifpackagelater\ltx@ifpackagelater
      \let\@ifclassloaded\ltx@ifclassloaded
      \let\@ifclasslater\ltx@ifclasslater
      \HOpatch@list
      \let\@ifpackageloaded\HOpatch@OrgIfPackageLoaded
      \let\@ifpackagelater\HOpatch@OrgIfPackageLater
      \let\@ifclassloaded\HOpatch@OrgIfClassLoaded
      \let\@ifclasslater\HOpatch@OrgIfClassLater
    }%
  }%
}
%    \end{macrocode}
%
%    \begin{macrocode}
\HOpatch@AtEnd%
%    \end{macrocode}
%    \begin{macrocode}
%</package>
%    \end{macrocode}
%
% \section{Test}
%
%    \begin{macrocode}
%<*test1>
\def\LoadCommand{\RequirePackage{hopatch}[2012/05/28]}
%</test1>
%    \end{macrocode}
% \subsection{Catcode checks for loading}
%
%    \begin{macrocode}
%<*test1>
%    \end{macrocode}
%    \begin{macrocode}
\catcode`\{=1 %
\catcode`\}=2 %
\catcode`\#=6 %
\catcode`\@=11 %
\expandafter\ifx\csname count@\endcsname\relax
  \countdef\count@=255 %
\fi
\expandafter\ifx\csname @gobble\endcsname\relax
  \long\def\@gobble#1{}%
\fi
\expandafter\ifx\csname @firstofone\endcsname\relax
  \long\def\@firstofone#1{#1}%
\fi
\expandafter\ifx\csname loop\endcsname\relax
  \expandafter\@firstofone
\else
  \expandafter\@gobble
\fi
{%
  \def\loop#1\repeat{%
    \def\body{#1}%
    \iterate
  }%
  \def\iterate{%
    \body
      \let\next\iterate
    \else
      \let\next\relax
    \fi
    \next
  }%
  \let\repeat=\fi
}%
\def\RestoreCatcodes{}
\count@=0 %
\loop
  \edef\RestoreCatcodes{%
    \RestoreCatcodes
    \catcode\the\count@=\the\catcode\count@\relax
  }%
\ifnum\count@<255 %
  \advance\count@ 1 %
\repeat

\def\RangeCatcodeInvalid#1#2{%
  \count@=#1\relax
  \loop
    \catcode\count@=15 %
  \ifnum\count@<#2\relax
    \advance\count@ 1 %
  \repeat
}
\def\RangeCatcodeCheck#1#2#3{%
  \count@=#1\relax
  \loop
    \ifnum#3=\catcode\count@
    \else
      \errmessage{%
        Character \the\count@\space
        with wrong catcode \the\catcode\count@\space
        instead of \number#3%
      }%
    \fi
  \ifnum\count@<#2\relax
    \advance\count@ 1 %
  \repeat
}
\def\space{ }
\expandafter\ifx\csname LoadCommand\endcsname\relax
  \def\LoadCommand{\input hopatch.sty\relax}%
\fi
\def\Test{%
  \RangeCatcodeInvalid{0}{47}%
  \RangeCatcodeInvalid{58}{64}%
  \RangeCatcodeInvalid{91}{96}%
  \RangeCatcodeInvalid{123}{255}%
  \catcode`\@=12 %
  \catcode`\\=0 %
  \catcode`\%=14 %
  \LoadCommand
  \RangeCatcodeCheck{0}{36}{15}%
  \RangeCatcodeCheck{37}{37}{14}%
  \RangeCatcodeCheck{38}{47}{15}%
  \RangeCatcodeCheck{48}{57}{12}%
  \RangeCatcodeCheck{58}{63}{15}%
  \RangeCatcodeCheck{64}{64}{12}%
  \RangeCatcodeCheck{65}{90}{11}%
  \RangeCatcodeCheck{91}{91}{15}%
  \RangeCatcodeCheck{92}{92}{0}%
  \RangeCatcodeCheck{93}{96}{15}%
  \RangeCatcodeCheck{97}{122}{11}%
  \RangeCatcodeCheck{123}{255}{15}%
  \RestoreCatcodes
}
\Test
\csname @@end\endcsname
\end
%    \end{macrocode}
%    \begin{macrocode}
%</test1>
%    \end{macrocode}
%
%    \begin{macrocode}
%<*test2>
\NeedsTeXFormat{LaTeX2e}
\providecommand\variant{0}
\RequirePackage{filecontents}
\begin{filecontents}{foo.sty}
\ProvidesPackage{foo}
\def\msg#{\immediate\write16}
\def\foo#1{%
  \msg{\fooformat{#1}}%
}
\def\fooformat#1{[#1]}% hash-ok
\foo{* Executing foo at package loading}
\end{filecontents}

\ifnum\variant=1 %
  \documentclass{memoir}%
\else
  \documentclass{article}%
\fi

\ifcase\variant\relax
\or % 1
\or % 2
  \usepackage{etoolbox}%
\or % 3
  \usepackage{scrlfile}%
\or % 4
  \usepackage{filehook}%
\fi

\AtBeginDocument{\foo{* AtBeginDocument before hopatch}}
\usepackage{hopatch}
\AtBeginDocument{\foo{* AtBeginDocument after hopatch}}

\makeatletter
\hopatch@AfterPackage{foo}{%
  \def\fooformat#1{<<#1>>}%
}
\makeatother

\AtBeginDocument{\foo{* AtBeginDocument before foo}}
\usepackage{foo}
\AtBeginDocument{\foo{* AtBeginDocument after foo}}

\foo{* Executing in preamble}

\begin{document}
\foo{* Executing in document}
\end{document}
%</test2>
%    \end{macrocode}
%
% \section{Installation}
%
% \subsection{Download}
%
% \paragraph{Package.} This package is available on
% CTAN\footnote{\url{ftp://ftp.ctan.org/tex-archive/}}:
% \begin{description}
% \item[\CTAN{macros/latex/contrib/oberdiek/hopatch.dtx}] The source file.
% \item[\CTAN{macros/latex/contrib/oberdiek/hopatch.pdf}] Documentation.
% \end{description}
%
%
% \paragraph{Bundle.} All the packages of the bundle `oberdiek'
% are also available in a TDS compliant ZIP archive. There
% the packages are already unpacked and the documentation files
% are generated. The files and directories obey the TDS standard.
% \begin{description}
% \item[\CTAN{install/macros/latex/contrib/oberdiek.tds.zip}]
% \end{description}
% \emph{TDS} refers to the standard ``A Directory Structure
% for \TeX\ Files'' (\CTAN{tds/tds.pdf}). Directories
% with \xfile{texmf} in their name are usually organized this way.
%
% \subsection{Bundle installation}
%
% \paragraph{Unpacking.} Unpack the \xfile{oberdiek.tds.zip} in the
% TDS tree (also known as \xfile{texmf} tree) of your choice.
% Example (linux):
% \begin{quote}
%   |unzip oberdiek.tds.zip -d ~/texmf|
% \end{quote}
%
% \paragraph{Script installation.}
% Check the directory \xfile{TDS:scripts/oberdiek/} for
% scripts that need further installation steps.
% Package \xpackage{attachfile2} comes with the Perl script
% \xfile{pdfatfi.pl} that should be installed in such a way
% that it can be called as \texttt{pdfatfi}.
% Example (linux):
% \begin{quote}
%   |chmod +x scripts/oberdiek/pdfatfi.pl|\\
%   |cp scripts/oberdiek/pdfatfi.pl /usr/local/bin/|
% \end{quote}
%
% \subsection{Package installation}
%
% \paragraph{Unpacking.} The \xfile{.dtx} file is a self-extracting
% \docstrip\ archive. The files are extracted by running the
% \xfile{.dtx} through \plainTeX:
% \begin{quote}
%   \verb|tex hopatch.dtx|
% \end{quote}
%
% \paragraph{TDS.} Now the different files must be moved into
% the different directories in your installation TDS tree
% (also known as \xfile{texmf} tree):
% \begin{quote}
% \def\t{^^A
% \begin{tabular}{@{}>{\ttfamily}l@{ $\rightarrow$ }>{\ttfamily}l@{}}
%   hopatch.sty & tex/latex/oberdiek/hopatch.sty\\
%   hopatch.pdf & doc/latex/oberdiek/hopatch.pdf\\
%   test/hopatch-test1.tex & doc/latex/oberdiek/test/hopatch-test1.tex\\
%   test/hopatch-test2.tex & doc/latex/oberdiek/test/hopatch-test2.tex\\
%   hopatch.dtx & source/latex/oberdiek/hopatch.dtx\\
% \end{tabular}^^A
% }^^A
% \sbox0{\t}^^A
% \ifdim\wd0>\linewidth
%   \begingroup
%     \advance\linewidth by\leftmargin
%     \advance\linewidth by\rightmargin
%   \edef\x{\endgroup
%     \def\noexpand\lw{\the\linewidth}^^A
%   }\x
%   \def\lwbox{^^A
%     \leavevmode
%     \hbox to \linewidth{^^A
%       \kern-\leftmargin\relax
%       \hss
%       \usebox0
%       \hss
%       \kern-\rightmargin\relax
%     }^^A
%   }^^A
%   \ifdim\wd0>\lw
%     \sbox0{\small\t}^^A
%     \ifdim\wd0>\linewidth
%       \ifdim\wd0>\lw
%         \sbox0{\footnotesize\t}^^A
%         \ifdim\wd0>\linewidth
%           \ifdim\wd0>\lw
%             \sbox0{\scriptsize\t}^^A
%             \ifdim\wd0>\linewidth
%               \ifdim\wd0>\lw
%                 \sbox0{\tiny\t}^^A
%                 \ifdim\wd0>\linewidth
%                   \lwbox
%                 \else
%                   \usebox0
%                 \fi
%               \else
%                 \lwbox
%               \fi
%             \else
%               \usebox0
%             \fi
%           \else
%             \lwbox
%           \fi
%         \else
%           \usebox0
%         \fi
%       \else
%         \lwbox
%       \fi
%     \else
%       \usebox0
%     \fi
%   \else
%     \lwbox
%   \fi
% \else
%   \usebox0
% \fi
% \end{quote}
% If you have a \xfile{docstrip.cfg} that configures and enables \docstrip's
% TDS installing feature, then some files can already be in the right
% place, see the documentation of \docstrip.
%
% \subsection{Refresh file name databases}
%
% If your \TeX~distribution
% (\teTeX, \mikTeX, \dots) relies on file name databases, you must refresh
% these. For example, \teTeX\ users run \verb|texhash| or
% \verb|mktexlsr|.
%
% \subsection{Some details for the interested}
%
% \paragraph{Attached source.}
%
% The PDF documentation on CTAN also includes the
% \xfile{.dtx} source file. It can be extracted by
% AcrobatReader 6 or higher. Another option is \textsf{pdftk},
% e.g. unpack the file into the current directory:
% \begin{quote}
%   \verb|pdftk hopatch.pdf unpack_files output .|
% \end{quote}
%
% \paragraph{Unpacking with \LaTeX.}
% The \xfile{.dtx} chooses its action depending on the format:
% \begin{description}
% \item[\plainTeX:] Run \docstrip\ and extract the files.
% \item[\LaTeX:] Generate the documentation.
% \end{description}
% If you insist on using \LaTeX\ for \docstrip\ (really,
% \docstrip\ does not need \LaTeX), then inform the autodetect routine
% about your intention:
% \begin{quote}
%   \verb|latex \let\install=y\input{hopatch.dtx}|
% \end{quote}
% Do not forget to quote the argument according to the demands
% of your shell.
%
% \paragraph{Generating the documentation.}
% You can use both the \xfile{.dtx} or the \xfile{.drv} to generate
% the documentation. The process can be configured by the
% configuration file \xfile{ltxdoc.cfg}. For instance, put this
% line into this file, if you want to have A4 as paper format:
% \begin{quote}
%   \verb|\PassOptionsToClass{a4paper}{article}|
% \end{quote}
% An example follows how to generate the
% documentation with pdf\LaTeX:
% \begin{quote}
%\begin{verbatim}
%pdflatex hopatch.dtx
%makeindex -s gind.ist hopatch.idx
%pdflatex hopatch.dtx
%makeindex -s gind.ist hopatch.idx
%pdflatex hopatch.dtx
%\end{verbatim}
% \end{quote}
%
% \section{Catalogue}
%
% The following XML file can be used as source for the
% \href{http://mirror.ctan.org/help/Catalogue/catalogue.html}{\TeX\ Catalogue}.
% The elements \texttt{caption} and \texttt{description} are imported
% from the original XML file from the Catalogue.
% The name of the XML file in the Catalogue is \xfile{hopatch.xml}.
%    \begin{macrocode}
%<*catalogue>
<?xml version='1.0' encoding='us-ascii'?>
<!DOCTYPE entry SYSTEM 'catalogue.dtd'>
<entry datestamp='$Date$' modifier='$Author$' id='hopatch'>
  <name>hopatch</name>
  <caption>Load patches for packages.</caption>
  <authorref id='auth:oberdiek'/>
  <copyright owner='Heiko Oberdiek' year='2011,2012'/>
  <license type='lppl1.3'/>
  <version number='1.2'/>
  <description>
    The hopatch package provides a command with which the user may
    register a piece of patch code for a particular package.  Hopatch
    will apply the patch immediately, if the relevant package has
    already been loaded; otherwise it will store the patch until the
    package appears.
    <p/>
    The package is part of the <xref refid='oberdiek'>oberdiek</xref> bundle.
  </description>
  <documentation details='Package documentation'
      href='ctan:/macros/latex/contrib/oberdiek/hopatch.pdf'/>
  <ctan file='true' path='/macros/latex/contrib/oberdiek/hopatch.dtx'/>
  <miktex location='oberdiek'/>
  <texlive location='oberdiek'/>
  <install path='/macros/latex/contrib/oberdiek/oberdiek.tds.zip'/>
</entry>
%</catalogue>
%    \end{macrocode}
%
% \begin{thebibliography}{9}
% \raggedright
%
% \bibitem{etoolbox}
% Philipp Lehman:
% \textit{The etoolbox Package}
% 2011-01-03.
% \CTAN{macros/latex/contrib/etoolbox/etoolbox.pdf}
%
% \bibitem{filehook}
% Martin Scharrer:
% \textit{The filehook Package};
% 2011-01-09.
% \CTAN{macros/latex/contrib/filehook/filehook.pdf}
%
% \bibitem{ltxcmds}
% Heiko Oberdiek:
% \textit{The ltxcmds Package};
% 2010-12-12.
% \CTAN{macros/latex/contrib/oberdiek/ltxcmds.pdf}
%
% \bibitem{memoir}
% Peter Wilson, Lars Madsen:
% \textit{The Memoir Class for Configurable Typesetting, User Guide};
% 2010.
% \CTAN{macros/latex/contrib/memoir/memman.pdf}
%
% \bibitem{scrlfile}
% Markus Kohm, Jens-Uwe Morawski:
% \textit{The Guide KOMA-Script};
% 2011-01-20.
% \CTAN{macros/latex/contrib/koma-script/scrguien.pdf}
%
% \end{thebibliography}
%
% \begin{History}
%   \begin{Version}{2011/01/30 v1.0}
%   \item
%     First public version.
%   \end{Version}
%   \begin{Version}{2011/06/24 v1.1}
%   \item
%     Fix the use of \cs{AtEndPreamble} and \cs{AfterEndPreamble}.
%     They are redefined by package \xpackage{etoolbox} after their
%     hooks are used and generate an error message then.
%   \end{Version}
%   \begin{Version}{2012/05/28 v1.2}
%   \item
%     Fix for use without \hologo{eTeX}
%     (thanks Gordon Lee).
%   \end{Version}
% \end{History}
%
% \PrintIndex
%
% \Finale
\endinput
|
% \end{quote}
% Do not forget to quote the argument according to the demands
% of your shell.
%
% \paragraph{Generating the documentation.}
% You can use both the \xfile{.dtx} or the \xfile{.drv} to generate
% the documentation. The process can be configured by the
% configuration file \xfile{ltxdoc.cfg}. For instance, put this
% line into this file, if you want to have A4 as paper format:
% \begin{quote}
%   \verb|\PassOptionsToClass{a4paper}{article}|
% \end{quote}
% An example follows how to generate the
% documentation with pdf\LaTeX:
% \begin{quote}
%\begin{verbatim}
%pdflatex hopatch.dtx
%makeindex -s gind.ist hopatch.idx
%pdflatex hopatch.dtx
%makeindex -s gind.ist hopatch.idx
%pdflatex hopatch.dtx
%\end{verbatim}
% \end{quote}
%
% \section{Catalogue}
%
% The following XML file can be used as source for the
% \href{http://mirror.ctan.org/help/Catalogue/catalogue.html}{\TeX\ Catalogue}.
% The elements \texttt{caption} and \texttt{description} are imported
% from the original XML file from the Catalogue.
% The name of the XML file in the Catalogue is \xfile{hopatch.xml}.
%    \begin{macrocode}
%<*catalogue>
<?xml version='1.0' encoding='us-ascii'?>
<!DOCTYPE entry SYSTEM 'catalogue.dtd'>
<entry datestamp='$Date$' modifier='$Author$' id='hopatch'>
  <name>hopatch</name>
  <caption>Load patches for packages.</caption>
  <authorref id='auth:oberdiek'/>
  <copyright owner='Heiko Oberdiek' year='2011,2012'/>
  <license type='lppl1.3'/>
  <version number='1.2'/>
  <description>
    The hopatch package provides a command with which the user may
    register a piece of patch code for a particular package.  Hopatch
    will apply the patch immediately, if the relevant package has
    already been loaded; otherwise it will store the patch until the
    package appears.
    <p/>
    The package is part of the <xref refid='oberdiek'>oberdiek</xref> bundle.
  </description>
  <documentation details='Package documentation'
      href='ctan:/macros/latex/contrib/oberdiek/hopatch.pdf'/>
  <ctan file='true' path='/macros/latex/contrib/oberdiek/hopatch.dtx'/>
  <miktex location='oberdiek'/>
  <texlive location='oberdiek'/>
  <install path='/macros/latex/contrib/oberdiek/oberdiek.tds.zip'/>
</entry>
%</catalogue>
%    \end{macrocode}
%
% \begin{thebibliography}{9}
% \raggedright
%
% \bibitem{etoolbox}
% Philipp Lehman:
% \textit{The etoolbox Package}
% 2011-01-03.
% \CTAN{macros/latex/contrib/etoolbox/etoolbox.pdf}
%
% \bibitem{filehook}
% Martin Scharrer:
% \textit{The filehook Package};
% 2011-01-09.
% \CTAN{macros/latex/contrib/filehook/filehook.pdf}
%
% \bibitem{ltxcmds}
% Heiko Oberdiek:
% \textit{The ltxcmds Package};
% 2010-12-12.
% \CTAN{macros/latex/contrib/oberdiek/ltxcmds.pdf}
%
% \bibitem{memoir}
% Peter Wilson, Lars Madsen:
% \textit{The Memoir Class for Configurable Typesetting, User Guide};
% 2010.
% \CTAN{macros/latex/contrib/memoir/memman.pdf}
%
% \bibitem{scrlfile}
% Markus Kohm, Jens-Uwe Morawski:
% \textit{The Guide KOMA-Script};
% 2011-01-20.
% \CTAN{macros/latex/contrib/koma-script/scrguien.pdf}
%
% \end{thebibliography}
%
% \begin{History}
%   \begin{Version}{2011/01/30 v1.0}
%   \item
%     First public version.
%   \end{Version}
%   \begin{Version}{2011/06/24 v1.1}
%   \item
%     Fix the use of \cs{AtEndPreamble} and \cs{AfterEndPreamble}.
%     They are redefined by package \xpackage{etoolbox} after their
%     hooks are used and generate an error message then.
%   \end{Version}
%   \begin{Version}{2012/05/28 v1.2}
%   \item
%     Fix for use without \hologo{eTeX}
%     (thanks Gordon Lee).
%   \end{Version}
% \end{History}
%
% \PrintIndex
%
% \Finale
\endinput

%        (quote the arguments according to the demands of your shell)
%
% Documentation:
%    (a) If hopatch.drv is present:
%           latex hopatch.drv
%    (b) Without hopatch.drv:
%           latex hopatch.dtx; ...
%    The class ltxdoc loads the configuration file ltxdoc.cfg
%    if available. Here you can specify further options, e.g.
%    use A4 as paper format:
%       \PassOptionsToClass{a4paper}{article}
%
%    Programm calls to get the documentation (example):
%       pdflatex hopatch.dtx
%       makeindex -s gind.ist hopatch.idx
%       pdflatex hopatch.dtx
%       makeindex -s gind.ist hopatch.idx
%       pdflatex hopatch.dtx
%
% Installation:
%    TDS:tex/latex/oberdiek/hopatch.sty
%    TDS:doc/latex/oberdiek/hopatch.pdf
%    TDS:doc/latex/oberdiek/test/hopatch-test1.tex
%    TDS:doc/latex/oberdiek/test/hopatch-test2.tex
%    TDS:source/latex/oberdiek/hopatch.dtx
%
%<*ignore>
\begingroup
  \catcode123=1 %
  \catcode125=2 %
  \def\x{LaTeX2e}%
\expandafter\endgroup
\ifcase 0\ifx\install y1\fi\expandafter
         \ifx\csname processbatchFile\endcsname\relax\else1\fi
         \ifx\fmtname\x\else 1\fi\relax
\else\csname fi\endcsname
%</ignore>
%<*install>
\input docstrip.tex
\Msg{************************************************************************}
\Msg{* Installation}
\Msg{* Package: hopatch 2012/05/28 v1.2 Wrapper for package hooks (HO)}
\Msg{************************************************************************}

\keepsilent
\askforoverwritefalse

\let\MetaPrefix\relax
\preamble

This is a generated file.

Project: hopatch
Version: 2012/05/28 v1.2

Copyright (C) 2011, 2012 by
   Heiko Oberdiek <heiko.oberdiek at googlemail.com>

This work may be distributed and/or modified under the
conditions of the LaTeX Project Public License, either
version 1.3c of this license or (at your option) any later
version. This version of this license is in
   http://www.latex-project.org/lppl/lppl-1-3c.txt
and the latest version of this license is in
   http://www.latex-project.org/lppl.txt
and version 1.3 or later is part of all distributions of
LaTeX version 2005/12/01 or later.

This work has the LPPL maintenance status "maintained".

This Current Maintainer of this work is Heiko Oberdiek.

This work consists of the main source file hopatch.dtx
and the derived files
   hopatch.sty, hopatch.pdf, hopatch.ins, hopatch.drv,
   hopatch-test1.tex, hopatch-test2.tex.

\endpreamble
\let\MetaPrefix\DoubleperCent

\generate{%
  \file{hopatch.ins}{\from{hopatch.dtx}{install}}%
  \file{hopatch.drv}{\from{hopatch.dtx}{driver}}%
  \usedir{tex/latex/oberdiek}%
  \file{hopatch.sty}{\from{hopatch.dtx}{package}}%
  \usedir{doc/latex/oberdiek/test}%
  \file{hopatch-test1.tex}{\from{hopatch.dtx}{test1}}%
  \file{hopatch-test2.tex}{\from{hopatch.dtx}{test2}}%
  \nopreamble
  \nopostamble
  \usedir{source/latex/oberdiek/catalogue}%
  \file{hopatch.xml}{\from{hopatch.dtx}{catalogue}}%
}

\catcode32=13\relax% active space
\let =\space%
\Msg{************************************************************************}
\Msg{*}
\Msg{* To finish the installation you have to move the following}
\Msg{* file into a directory searched by TeX:}
\Msg{*}
\Msg{*     hopatch.sty}
\Msg{*}
\Msg{* To produce the documentation run the file `hopatch.drv'}
\Msg{* through LaTeX.}
\Msg{*}
\Msg{* Happy TeXing!}
\Msg{*}
\Msg{************************************************************************}

\endbatchfile
%</install>
%<*ignore>
\fi
%</ignore>
%<*driver>
\NeedsTeXFormat{LaTeX2e}
\ProvidesFile{hopatch.drv}%
  [2012/05/28 v1.2 Wrapper for package hooks (HO)]%
\documentclass{ltxdoc}
\usepackage{holtxdoc}[2011/11/22]
\usepackage{caption}
\usepackage{array}
\setlength{\extrarowheight}{1pt}
\renewcommand*{\arraystretch}{1.1}
\begin{document}
  \DocInput{hopatch.dtx}%
\end{document}
%</driver>
% \fi
%
% \CheckSum{466}
%
% \CharacterTable
%  {Upper-case    \A\B\C\D\E\F\G\H\I\J\K\L\M\N\O\P\Q\R\S\T\U\V\W\X\Y\Z
%   Lower-case    \a\b\c\d\e\f\g\h\i\j\k\l\m\n\o\p\q\r\s\t\u\v\w\x\y\z
%   Digits        \0\1\2\3\4\5\6\7\8\9
%   Exclamation   \!     Double quote  \"     Hash (number) \#
%   Dollar        \$     Percent       \%     Ampersand     \&
%   Acute accent  \'     Left paren    \(     Right paren   \)
%   Asterisk      \*     Plus          \+     Comma         \,
%   Minus         \-     Point         \.     Solidus       \/
%   Colon         \:     Semicolon     \;     Less than     \<
%   Equals        \=     Greater than  \>     Question mark \?
%   Commercial at \@     Left bracket  \[     Backslash     \\
%   Right bracket \]     Circumflex    \^     Underscore    \_
%   Grave accent  \`     Left brace    \{     Vertical bar  \|
%   Right brace   \}     Tilde         \~}
%
% \GetFileInfo{hopatch.drv}
%
% \title{The \xpackage{hopatch} package}
% \date{2012/05/28 v1.2}
% \author{Heiko Oberdiek\\\xemail{heiko.oberdiek at googlemail.com}}
%
% \maketitle
%
% \begin{abstract}
% This packages provides a wrapper to various package hooks
% provided by other packages or classes, but does not define
% own hooks.
% \end{abstract}
%
% \tableofcontents
%
% \section{Documentation}
%
% Sometimes I want to add code right after a package
% has been loaded. Examples are bug fixes, adaptations,
% or added features as needed by package \xpackage{hyperref},
% for instance.
%
% Unhappily \hologo{LaTeX} does not provide this kind of
% hook. \cs{AtEndOfPackage} can be used inside the package
% only, because \hologo{LaTeX} clears the hook right before
% it loads the package.
%
% However, there are already many packages and classes
% that provide hooks that are executed after the package
% is loaded, see table~\ref{tab:hooks}.
%
% \begin{table}
% \caption{After package hooking}
% \label{tab:hooks}
% \begin{tabular}{ll}
% \bfseries Macro & \bfseries Provider\\
% \hline
% \cs{AfterPackage} & package \xpackage{scrlfile} \cite{scrlfile}\\
% \cs{AtEndOfPackageFile} & package \xpackage{filehook} \cite{filehook}\\
% \cs{AtEndPackage} & class \xclass{memoir} \cite{memoir}
% \end{tabular}
% \end{table}
%
% Package \xpackage{hopatch} can be used without the packages
% of table~\ref{tab:hooks}. But for an early executing right
% after a package is loaded, one of the following class or
% packages should be loaded before using \cs{hopatch@AfterPackage}:
% \begin{itemize}
%  \item package \xpackage{filehook}
%  \item package \xpackage{scrlfile}
%  \item class \xclass{memoir}
% \end{itemize}
%
% \begin{table}
% \caption{After begin document hooking}
% \begin{tabular}{ll}
% \bfseries Macro & \bfseries Provider\\
% \hline
% \cs{AtBeginDocument} & \hologo{LaTeX}'s kernel\\
% \cs{AtEndPreamble} & package \xpackage{etoolbox} \cite{etoolbox}\\
% \cs{AfterEndPreamble} & package \xpackage{etoolbox}
% \end{tabular}
% \end{table}
%
% Therefore I skip writing a new package for hooking into
% \hologo{LaTeX}'s package management and use this package
% to provide a wrapper to patch a package after it is loaded.
%
% \begin{declcs}{hopatch@AfterPackage} \M{package} \M{patch code}
% \end{declcs}
% If the package is already loaded, the \meta{patch code} is executed
% immediately.
% Otherwise the \meta{patch code} is stored in a command and tried
% at later locations until the package is available.
%
% The patch is tried in the following order:
% \begin{enumerate}
% \item If the package is already loaded, the patch is
%   applied immediately. Further locations are not tried.
% \item \cs{AtEndPackage}, provided by class \xclass{memoir} \cite{memoir},
%   and \cs{AfterPackage}, provided by package \xpackage{scrlfile} \cite{scrlfile},
%   are called right after the package file is input before
%   the hook of \hologo{LaTeX}'s \cs{AtEndOfPackage}.
% \item \cs{AtEndOfPackageFile}, provided by package \xpackage{filehook} \cite{filehook},
%   is called after the package is loaded and after the hook of
%   \hologo{LaTeX}'s \cs{AtEndOfPackage}.
% \item \cs{AtEndPreamble}, provided by package \xpackage{etoolbox} \cite{etoolbox},
%   is called at the beginning of |\begin{document}| before the hook
%   of \hologo{LaTeX}'s \cs{AtBeginDocument}.
% \item \cs{AtBeginDocument}, provided by \hologo{LaTeX}.
% \item \cs{AfterEndDocument}, provided by package \xpackage{etoolbox} \cite{etoolbox},
%   is called at the very end of |\begin{document}|. Preamble commands are
%   already forbidden there.
% \end{enumerate}
% Because of the various locations the patch code is restricted to
% limitations:
% \begin{itemize}
% \item
% Preamble commands, see \hologo{LaTeX}'s \cs{@onlypreamble}
% throw an error if used after |\begin{document}|. This is already
% the case for \cs{AfterEndDocument}. Therefore preamble commands
% are forbidden in the patching code.
% There are four exceptions \cs{@ifpackageloaded}, \cs{@ifclassloaded},
% \cs{@ifpackagelater} and \cs{@ifclasslater}. They are redefined
% during \cs{AfterEndDocument} using the counterparts of
% package \xpackage{ltxcmds} \cite{ltxcmds}.
% \item
% \cs{AfterPackage} of package \xpackage{scrlfile} and \cs{AtEndPackage}
% of class \xclass{memoir} call the hook before \hologo{LaTeX}'s
% \cs{AtEndOfPackage}.
% \end{itemize}
%
% \StopEventually{
% }
%
% \section{Implementation}
%
%    \begin{macrocode}
%<*package>
%    \end{macrocode}
%
% \subsection{Catcodes and package identification}
%
%    \begin{macrocode}
\begingroup\catcode61\catcode48\catcode32=10\relax%
  \catcode13=5 % ^^M
  \endlinechar=13 %
  \catcode123=1 % {
  \catcode125=2 % }
  \catcode64=11 % @
  \def\x{\endgroup
    \expandafter\edef\csname HOpatch@AtEnd\endcsname{%
      \endlinechar=\the\endlinechar\relax
      \catcode13=\the\catcode13\relax
      \catcode32=\the\catcode32\relax
      \catcode35=\the\catcode35\relax
      \catcode61=\the\catcode61\relax
      \catcode64=\the\catcode64\relax
      \catcode123=\the\catcode123\relax
      \catcode125=\the\catcode125\relax
    }%
  }%
\x\catcode61\catcode48\catcode32=10\relax%
\catcode13=5 % ^^M
\endlinechar=13 %
\catcode35=6 % #
\catcode64=11 % @
\catcode123=1 % {
\catcode125=2 % }
\def\TMP@EnsureCode#1#2{%
  \edef\HOpatch@AtEnd{%
    \HOpatch@AtEnd
    \catcode#1=\the\catcode#1\relax
  }%
  \catcode#1=#2\relax
}
\TMP@EnsureCode{40}{12}% (
\TMP@EnsureCode{41}{12}% )
\TMP@EnsureCode{43}{12}% +
\TMP@EnsureCode{46}{12}% .
\TMP@EnsureCode{47}{12}% /
\TMP@EnsureCode{91}{12}% [
\TMP@EnsureCode{93}{12}% ]
\edef\HOpatch@AtEnd{\HOpatch@AtEnd\noexpand\endinput}
%    \end{macrocode}
%
%    Package identification.
%    \begin{macrocode}
\NeedsTeXFormat{LaTeX2e}
\ProvidesPackage{hopatch}%
  [2012/05/28 v1.2 Wrapper for package hooks (HO)]
%    \end{macrocode}
%
% \subsection{Resources}
%
%    \begin{macrocode}
\begingroup\expandafter\expandafter\expandafter\endgroup
\expandafter\ifx\csname RequirePackage\endcsname\relax
  \def\TMP@RequirePackage#1[#2]{%
    \begingroup\expandafter\expandafter\expandafter\endgroup
    \expandafter\ifx\csname ver@#1.sty\endcsname\relax
      \input #1.sty\relax
    \fi
  }%
  \TMP@RequirePackage{ltxcmds}[2010/12/12]%
\else
  \RequirePackage{ltxcmds}[2010/12/12]%
\fi
%    \end{macrocode}
%
%    \begin{macro}{\HOpatch@counter}
%    \begin{macrocode}
\def\HOpatch@counter{0}%
%    \end{macrocode}
%    \end{macro}
%    \begin{macro}{\HOpatch@StepCounter}
%    \begin{macrocode}
\ltx@ifundefined{numexpr}{%
  \def\HOpatch@StepCounter{%
    \begingroup
      \count@\HOpatch@counter\relax
      \advance\count@\ltx@one\relax
    \edef\x{\endgroup
      \noexpand\def\noexpand\HOpatch@counter{\the\count@}%
    }%
    \x
  }%
}{%
  \def\HOpatch@StepCounter{%
    \edef\HOpatch@counter{%
      \the\numexpr\HOpatch@counter+\ltx@one\relax
    }%
  }%
}
%    \end{macrocode}
%    \end{macro}
%
%    \begin{macro}{\HOpatch@list}
%    \begin{macrocode}
\def\HOpatch@list{}
%    \end{macrocode}
%    \end{macro}
%
%    \begin{macro}{\HOpatch@Add}
%    \begin{macrocode}
\def\HOpatch@Add{%
  \ltx@LocalAppendToMacro\HOpatch@list
}
%    \end{macrocode}
%    \end{macro}
%
% \subsection{Package patching}
%
%    \begin{macro}{\hopatch@AfterPackage}
%    \begin{macrocode}
\def\hopatch@AfterPackage#1{%
  \ltx@ifpackageloaded{#1}{%
    \ltx@firstofone
  }{%
    \HOpatch@AfterPackage{#1}%
  }%
}
%    \end{macrocode}
%    \end{macro}
%    \begin{macro}{\HOpatch@AfterPackage}
%    \begin{macrocode}
\def\HOpatch@AfterPackage#1{%
  \edef\HOpatch@temp{#1}%
  \HOpatch@StepCounter
  \expandafter\HOpatch@@AfterPackage
  \csname HOpatch@\HOpatch@counter\expandafter\endcsname{%
    \HOpatch@temp
  }%
}
%    \end{macrocode}
%    \end{macro}
%
%    \begin{macro}{\HOpatch@@AfterPackage}
%    \begin{macrocode}
\def\HOpatch@@AfterPackage#1#2#3{%
  \begingroup
    \toks@{#3}%
    \xdef\HOpatch@gtemp{%
      \noexpand\ltx@ifpackageloaded{#2}{%
        \noexpand\let\noexpand#1\noexpand\relax
        \the\toks@
      }{}%
    }%
  \endgroup
  \let#1\HOpatch@gtemp
  \HOpatch@Add#1%
  \HOpatch@Try{AfterPackage}{#2}#1%
  \HOpatch@Try{AtEndPackage}{#2}#1%
  \HOpatch@Try{AtEndOfPackageFile}{#2}#1%
}
%    \end{macrocode}
%    \end{macro}
%
%    \begin{macro}{\HOpatch@Try}
%    \begin{macrocode}
\def\HOpatch@Try#1#2#3{%
  \ltx@ifundefined{#1}{}{%
    \csname #1\endcsname{#2}{#3}%
  }%
}
%    \end{macrocode}
%    \end{macro}
%
%    \begin{macrocode}
\AtBeginDocument{\HOpatch@list}
\ltx@ifundefined{AtEndPreamble}{}{%
  \ltx@ifundefined{@endpreamblehook}{}{%
    \AtEndPreamble{\HOpatch@list}%
  }%
}
\ltx@ifundefined{AfterEndPreamble}{}{%
  \ltx@ifundefined{@afterendpreamblehook}{}{%
    \AfterEndPreamble{%
      \let\HOpatch@OrgIfPackageLoaded\@ifpackageloaded
      \let\HOpatch@OrgIfPackageLater\@ifpackagelater
      \let\HOpatch@OrgIfClassLoaded\@ifclassloaded
      \let\HOpatch@OrgIfClassLater\@ifclasslater
      \let\@ifpackageloaded\ltx@ifpackageloaded
      \let\@ifpackagelater\ltx@ifpackagelater
      \let\@ifclassloaded\ltx@ifclassloaded
      \let\@ifclasslater\ltx@ifclasslater
      \HOpatch@list
      \let\@ifpackageloaded\HOpatch@OrgIfPackageLoaded
      \let\@ifpackagelater\HOpatch@OrgIfPackageLater
      \let\@ifclassloaded\HOpatch@OrgIfClassLoaded
      \let\@ifclasslater\HOpatch@OrgIfClassLater
    }%
  }%
}
%    \end{macrocode}
%
%    \begin{macrocode}
\HOpatch@AtEnd%
%    \end{macrocode}
%    \begin{macrocode}
%</package>
%    \end{macrocode}
%
% \section{Test}
%
%    \begin{macrocode}
%<*test1>
\def\LoadCommand{\RequirePackage{hopatch}[2012/05/28]}
%</test1>
%    \end{macrocode}
% \subsection{Catcode checks for loading}
%
%    \begin{macrocode}
%<*test1>
%    \end{macrocode}
%    \begin{macrocode}
\catcode`\{=1 %
\catcode`\}=2 %
\catcode`\#=6 %
\catcode`\@=11 %
\expandafter\ifx\csname count@\endcsname\relax
  \countdef\count@=255 %
\fi
\expandafter\ifx\csname @gobble\endcsname\relax
  \long\def\@gobble#1{}%
\fi
\expandafter\ifx\csname @firstofone\endcsname\relax
  \long\def\@firstofone#1{#1}%
\fi
\expandafter\ifx\csname loop\endcsname\relax
  \expandafter\@firstofone
\else
  \expandafter\@gobble
\fi
{%
  \def\loop#1\repeat{%
    \def\body{#1}%
    \iterate
  }%
  \def\iterate{%
    \body
      \let\next\iterate
    \else
      \let\next\relax
    \fi
    \next
  }%
  \let\repeat=\fi
}%
\def\RestoreCatcodes{}
\count@=0 %
\loop
  \edef\RestoreCatcodes{%
    \RestoreCatcodes
    \catcode\the\count@=\the\catcode\count@\relax
  }%
\ifnum\count@<255 %
  \advance\count@ 1 %
\repeat

\def\RangeCatcodeInvalid#1#2{%
  \count@=#1\relax
  \loop
    \catcode\count@=15 %
  \ifnum\count@<#2\relax
    \advance\count@ 1 %
  \repeat
}
\def\RangeCatcodeCheck#1#2#3{%
  \count@=#1\relax
  \loop
    \ifnum#3=\catcode\count@
    \else
      \errmessage{%
        Character \the\count@\space
        with wrong catcode \the\catcode\count@\space
        instead of \number#3%
      }%
    \fi
  \ifnum\count@<#2\relax
    \advance\count@ 1 %
  \repeat
}
\def\space{ }
\expandafter\ifx\csname LoadCommand\endcsname\relax
  \def\LoadCommand{\input hopatch.sty\relax}%
\fi
\def\Test{%
  \RangeCatcodeInvalid{0}{47}%
  \RangeCatcodeInvalid{58}{64}%
  \RangeCatcodeInvalid{91}{96}%
  \RangeCatcodeInvalid{123}{255}%
  \catcode`\@=12 %
  \catcode`\\=0 %
  \catcode`\%=14 %
  \LoadCommand
  \RangeCatcodeCheck{0}{36}{15}%
  \RangeCatcodeCheck{37}{37}{14}%
  \RangeCatcodeCheck{38}{47}{15}%
  \RangeCatcodeCheck{48}{57}{12}%
  \RangeCatcodeCheck{58}{63}{15}%
  \RangeCatcodeCheck{64}{64}{12}%
  \RangeCatcodeCheck{65}{90}{11}%
  \RangeCatcodeCheck{91}{91}{15}%
  \RangeCatcodeCheck{92}{92}{0}%
  \RangeCatcodeCheck{93}{96}{15}%
  \RangeCatcodeCheck{97}{122}{11}%
  \RangeCatcodeCheck{123}{255}{15}%
  \RestoreCatcodes
}
\Test
\csname @@end\endcsname
\end
%    \end{macrocode}
%    \begin{macrocode}
%</test1>
%    \end{macrocode}
%
%    \begin{macrocode}
%<*test2>
\NeedsTeXFormat{LaTeX2e}
\providecommand\variant{0}
\RequirePackage{filecontents}
\begin{filecontents}{foo.sty}
\ProvidesPackage{foo}
\def\msg#{\immediate\write16}
\def\foo#1{%
  \msg{\fooformat{#1}}%
}
\def\fooformat#1{[#1]}% hash-ok
\foo{* Executing foo at package loading}
\end{filecontents}

\ifnum\variant=1 %
  \documentclass{memoir}%
\else
  \documentclass{article}%
\fi

\ifcase\variant\relax
\or % 1
\or % 2
  \usepackage{etoolbox}%
\or % 3
  \usepackage{scrlfile}%
\or % 4
  \usepackage{filehook}%
\fi

\AtBeginDocument{\foo{* AtBeginDocument before hopatch}}
\usepackage{hopatch}
\AtBeginDocument{\foo{* AtBeginDocument after hopatch}}

\makeatletter
\hopatch@AfterPackage{foo}{%
  \def\fooformat#1{<<#1>>}%
}
\makeatother

\AtBeginDocument{\foo{* AtBeginDocument before foo}}
\usepackage{foo}
\AtBeginDocument{\foo{* AtBeginDocument after foo}}

\foo{* Executing in preamble}

\begin{document}
\foo{* Executing in document}
\end{document}
%</test2>
%    \end{macrocode}
%
% \section{Installation}
%
% \subsection{Download}
%
% \paragraph{Package.} This package is available on
% CTAN\footnote{\url{ftp://ftp.ctan.org/tex-archive/}}:
% \begin{description}
% \item[\CTAN{macros/latex/contrib/oberdiek/hopatch.dtx}] The source file.
% \item[\CTAN{macros/latex/contrib/oberdiek/hopatch.pdf}] Documentation.
% \end{description}
%
%
% \paragraph{Bundle.} All the packages of the bundle `oberdiek'
% are also available in a TDS compliant ZIP archive. There
% the packages are already unpacked and the documentation files
% are generated. The files and directories obey the TDS standard.
% \begin{description}
% \item[\CTAN{install/macros/latex/contrib/oberdiek.tds.zip}]
% \end{description}
% \emph{TDS} refers to the standard ``A Directory Structure
% for \TeX\ Files'' (\CTAN{tds/tds.pdf}). Directories
% with \xfile{texmf} in their name are usually organized this way.
%
% \subsection{Bundle installation}
%
% \paragraph{Unpacking.} Unpack the \xfile{oberdiek.tds.zip} in the
% TDS tree (also known as \xfile{texmf} tree) of your choice.
% Example (linux):
% \begin{quote}
%   |unzip oberdiek.tds.zip -d ~/texmf|
% \end{quote}
%
% \paragraph{Script installation.}
% Check the directory \xfile{TDS:scripts/oberdiek/} for
% scripts that need further installation steps.
% Package \xpackage{attachfile2} comes with the Perl script
% \xfile{pdfatfi.pl} that should be installed in such a way
% that it can be called as \texttt{pdfatfi}.
% Example (linux):
% \begin{quote}
%   |chmod +x scripts/oberdiek/pdfatfi.pl|\\
%   |cp scripts/oberdiek/pdfatfi.pl /usr/local/bin/|
% \end{quote}
%
% \subsection{Package installation}
%
% \paragraph{Unpacking.} The \xfile{.dtx} file is a self-extracting
% \docstrip\ archive. The files are extracted by running the
% \xfile{.dtx} through \plainTeX:
% \begin{quote}
%   \verb|tex hopatch.dtx|
% \end{quote}
%
% \paragraph{TDS.} Now the different files must be moved into
% the different directories in your installation TDS tree
% (also known as \xfile{texmf} tree):
% \begin{quote}
% \def\t{^^A
% \begin{tabular}{@{}>{\ttfamily}l@{ $\rightarrow$ }>{\ttfamily}l@{}}
%   hopatch.sty & tex/latex/oberdiek/hopatch.sty\\
%   hopatch.pdf & doc/latex/oberdiek/hopatch.pdf\\
%   test/hopatch-test1.tex & doc/latex/oberdiek/test/hopatch-test1.tex\\
%   test/hopatch-test2.tex & doc/latex/oberdiek/test/hopatch-test2.tex\\
%   hopatch.dtx & source/latex/oberdiek/hopatch.dtx\\
% \end{tabular}^^A
% }^^A
% \sbox0{\t}^^A
% \ifdim\wd0>\linewidth
%   \begingroup
%     \advance\linewidth by\leftmargin
%     \advance\linewidth by\rightmargin
%   \edef\x{\endgroup
%     \def\noexpand\lw{\the\linewidth}^^A
%   }\x
%   \def\lwbox{^^A
%     \leavevmode
%     \hbox to \linewidth{^^A
%       \kern-\leftmargin\relax
%       \hss
%       \usebox0
%       \hss
%       \kern-\rightmargin\relax
%     }^^A
%   }^^A
%   \ifdim\wd0>\lw
%     \sbox0{\small\t}^^A
%     \ifdim\wd0>\linewidth
%       \ifdim\wd0>\lw
%         \sbox0{\footnotesize\t}^^A
%         \ifdim\wd0>\linewidth
%           \ifdim\wd0>\lw
%             \sbox0{\scriptsize\t}^^A
%             \ifdim\wd0>\linewidth
%               \ifdim\wd0>\lw
%                 \sbox0{\tiny\t}^^A
%                 \ifdim\wd0>\linewidth
%                   \lwbox
%                 \else
%                   \usebox0
%                 \fi
%               \else
%                 \lwbox
%               \fi
%             \else
%               \usebox0
%             \fi
%           \else
%             \lwbox
%           \fi
%         \else
%           \usebox0
%         \fi
%       \else
%         \lwbox
%       \fi
%     \else
%       \usebox0
%     \fi
%   \else
%     \lwbox
%   \fi
% \else
%   \usebox0
% \fi
% \end{quote}
% If you have a \xfile{docstrip.cfg} that configures and enables \docstrip's
% TDS installing feature, then some files can already be in the right
% place, see the documentation of \docstrip.
%
% \subsection{Refresh file name databases}
%
% If your \TeX~distribution
% (\teTeX, \mikTeX, \dots) relies on file name databases, you must refresh
% these. For example, \teTeX\ users run \verb|texhash| or
% \verb|mktexlsr|.
%
% \subsection{Some details for the interested}
%
% \paragraph{Attached source.}
%
% The PDF documentation on CTAN also includes the
% \xfile{.dtx} source file. It can be extracted by
% AcrobatReader 6 or higher. Another option is \textsf{pdftk},
% e.g. unpack the file into the current directory:
% \begin{quote}
%   \verb|pdftk hopatch.pdf unpack_files output .|
% \end{quote}
%
% \paragraph{Unpacking with \LaTeX.}
% The \xfile{.dtx} chooses its action depending on the format:
% \begin{description}
% \item[\plainTeX:] Run \docstrip\ and extract the files.
% \item[\LaTeX:] Generate the documentation.
% \end{description}
% If you insist on using \LaTeX\ for \docstrip\ (really,
% \docstrip\ does not need \LaTeX), then inform the autodetect routine
% about your intention:
% \begin{quote}
%   \verb|latex \let\install=y% \iffalse meta-comment
%
% File: hopatch.dtx
% Version: 2012/05/28 v1.2
% Info: Wrapper for package hooks
%
% Copyright (C) 2011, 2012 by
%    Heiko Oberdiek <heiko.oberdiek at googlemail.com>
%
% This work may be distributed and/or modified under the
% conditions of the LaTeX Project Public License, either
% version 1.3c of this license or (at your option) any later
% version. This version of this license is in
%    http://www.latex-project.org/lppl/lppl-1-3c.txt
% and the latest version of this license is in
%    http://www.latex-project.org/lppl.txt
% and version 1.3 or later is part of all distributions of
% LaTeX version 2005/12/01 or later.
%
% This work has the LPPL maintenance status "maintained".
%
% This Current Maintainer of this work is Heiko Oberdiek.
%
% This work consists of the main source file hopatch.dtx
% and the derived files
%    hopatch.sty, hopatch.pdf, hopatch.ins, hopatch.drv,
%    hopatch-test1.tex, hopatch-test2.tex.
%
% Distribution:
%    CTAN:macros/latex/contrib/oberdiek/hopatch.dtx
%    CTAN:macros/latex/contrib/oberdiek/hopatch.pdf
%
% Unpacking:
%    (a) If hopatch.ins is present:
%           tex hopatch.ins
%    (b) Without hopatch.ins:
%           tex hopatch.dtx
%    (c) If you insist on using LaTeX
%           latex \let\install=y% \iffalse meta-comment
%
% File: hopatch.dtx
% Version: 2012/05/28 v1.2
% Info: Wrapper for package hooks
%
% Copyright (C) 2011, 2012 by
%    Heiko Oberdiek <heiko.oberdiek at googlemail.com>
%
% This work may be distributed and/or modified under the
% conditions of the LaTeX Project Public License, either
% version 1.3c of this license or (at your option) any later
% version. This version of this license is in
%    http://www.latex-project.org/lppl/lppl-1-3c.txt
% and the latest version of this license is in
%    http://www.latex-project.org/lppl.txt
% and version 1.3 or later is part of all distributions of
% LaTeX version 2005/12/01 or later.
%
% This work has the LPPL maintenance status "maintained".
%
% This Current Maintainer of this work is Heiko Oberdiek.
%
% This work consists of the main source file hopatch.dtx
% and the derived files
%    hopatch.sty, hopatch.pdf, hopatch.ins, hopatch.drv,
%    hopatch-test1.tex, hopatch-test2.tex.
%
% Distribution:
%    CTAN:macros/latex/contrib/oberdiek/hopatch.dtx
%    CTAN:macros/latex/contrib/oberdiek/hopatch.pdf
%
% Unpacking:
%    (a) If hopatch.ins is present:
%           tex hopatch.ins
%    (b) Without hopatch.ins:
%           tex hopatch.dtx
%    (c) If you insist on using LaTeX
%           latex \let\install=y\input{hopatch.dtx}
%        (quote the arguments according to the demands of your shell)
%
% Documentation:
%    (a) If hopatch.drv is present:
%           latex hopatch.drv
%    (b) Without hopatch.drv:
%           latex hopatch.dtx; ...
%    The class ltxdoc loads the configuration file ltxdoc.cfg
%    if available. Here you can specify further options, e.g.
%    use A4 as paper format:
%       \PassOptionsToClass{a4paper}{article}
%
%    Programm calls to get the documentation (example):
%       pdflatex hopatch.dtx
%       makeindex -s gind.ist hopatch.idx
%       pdflatex hopatch.dtx
%       makeindex -s gind.ist hopatch.idx
%       pdflatex hopatch.dtx
%
% Installation:
%    TDS:tex/latex/oberdiek/hopatch.sty
%    TDS:doc/latex/oberdiek/hopatch.pdf
%    TDS:doc/latex/oberdiek/test/hopatch-test1.tex
%    TDS:doc/latex/oberdiek/test/hopatch-test2.tex
%    TDS:source/latex/oberdiek/hopatch.dtx
%
%<*ignore>
\begingroup
  \catcode123=1 %
  \catcode125=2 %
  \def\x{LaTeX2e}%
\expandafter\endgroup
\ifcase 0\ifx\install y1\fi\expandafter
         \ifx\csname processbatchFile\endcsname\relax\else1\fi
         \ifx\fmtname\x\else 1\fi\relax
\else\csname fi\endcsname
%</ignore>
%<*install>
\input docstrip.tex
\Msg{************************************************************************}
\Msg{* Installation}
\Msg{* Package: hopatch 2012/05/28 v1.2 Wrapper for package hooks (HO)}
\Msg{************************************************************************}

\keepsilent
\askforoverwritefalse

\let\MetaPrefix\relax
\preamble

This is a generated file.

Project: hopatch
Version: 2012/05/28 v1.2

Copyright (C) 2011, 2012 by
   Heiko Oberdiek <heiko.oberdiek at googlemail.com>

This work may be distributed and/or modified under the
conditions of the LaTeX Project Public License, either
version 1.3c of this license or (at your option) any later
version. This version of this license is in
   http://www.latex-project.org/lppl/lppl-1-3c.txt
and the latest version of this license is in
   http://www.latex-project.org/lppl.txt
and version 1.3 or later is part of all distributions of
LaTeX version 2005/12/01 or later.

This work has the LPPL maintenance status "maintained".

This Current Maintainer of this work is Heiko Oberdiek.

This work consists of the main source file hopatch.dtx
and the derived files
   hopatch.sty, hopatch.pdf, hopatch.ins, hopatch.drv,
   hopatch-test1.tex, hopatch-test2.tex.

\endpreamble
\let\MetaPrefix\DoubleperCent

\generate{%
  \file{hopatch.ins}{\from{hopatch.dtx}{install}}%
  \file{hopatch.drv}{\from{hopatch.dtx}{driver}}%
  \usedir{tex/latex/oberdiek}%
  \file{hopatch.sty}{\from{hopatch.dtx}{package}}%
  \usedir{doc/latex/oberdiek/test}%
  \file{hopatch-test1.tex}{\from{hopatch.dtx}{test1}}%
  \file{hopatch-test2.tex}{\from{hopatch.dtx}{test2}}%
  \nopreamble
  \nopostamble
  \usedir{source/latex/oberdiek/catalogue}%
  \file{hopatch.xml}{\from{hopatch.dtx}{catalogue}}%
}

\catcode32=13\relax% active space
\let =\space%
\Msg{************************************************************************}
\Msg{*}
\Msg{* To finish the installation you have to move the following}
\Msg{* file into a directory searched by TeX:}
\Msg{*}
\Msg{*     hopatch.sty}
\Msg{*}
\Msg{* To produce the documentation run the file `hopatch.drv'}
\Msg{* through LaTeX.}
\Msg{*}
\Msg{* Happy TeXing!}
\Msg{*}
\Msg{************************************************************************}

\endbatchfile
%</install>
%<*ignore>
\fi
%</ignore>
%<*driver>
\NeedsTeXFormat{LaTeX2e}
\ProvidesFile{hopatch.drv}%
  [2012/05/28 v1.2 Wrapper for package hooks (HO)]%
\documentclass{ltxdoc}
\usepackage{holtxdoc}[2011/11/22]
\usepackage{caption}
\usepackage{array}
\setlength{\extrarowheight}{1pt}
\renewcommand*{\arraystretch}{1.1}
\begin{document}
  \DocInput{hopatch.dtx}%
\end{document}
%</driver>
% \fi
%
% \CheckSum{466}
%
% \CharacterTable
%  {Upper-case    \A\B\C\D\E\F\G\H\I\J\K\L\M\N\O\P\Q\R\S\T\U\V\W\X\Y\Z
%   Lower-case    \a\b\c\d\e\f\g\h\i\j\k\l\m\n\o\p\q\r\s\t\u\v\w\x\y\z
%   Digits        \0\1\2\3\4\5\6\7\8\9
%   Exclamation   \!     Double quote  \"     Hash (number) \#
%   Dollar        \$     Percent       \%     Ampersand     \&
%   Acute accent  \'     Left paren    \(     Right paren   \)
%   Asterisk      \*     Plus          \+     Comma         \,
%   Minus         \-     Point         \.     Solidus       \/
%   Colon         \:     Semicolon     \;     Less than     \<
%   Equals        \=     Greater than  \>     Question mark \?
%   Commercial at \@     Left bracket  \[     Backslash     \\
%   Right bracket \]     Circumflex    \^     Underscore    \_
%   Grave accent  \`     Left brace    \{     Vertical bar  \|
%   Right brace   \}     Tilde         \~}
%
% \GetFileInfo{hopatch.drv}
%
% \title{The \xpackage{hopatch} package}
% \date{2012/05/28 v1.2}
% \author{Heiko Oberdiek\\\xemail{heiko.oberdiek at googlemail.com}}
%
% \maketitle
%
% \begin{abstract}
% This packages provides a wrapper to various package hooks
% provided by other packages or classes, but does not define
% own hooks.
% \end{abstract}
%
% \tableofcontents
%
% \section{Documentation}
%
% Sometimes I want to add code right after a package
% has been loaded. Examples are bug fixes, adaptations,
% or added features as needed by package \xpackage{hyperref},
% for instance.
%
% Unhappily \hologo{LaTeX} does not provide this kind of
% hook. \cs{AtEndOfPackage} can be used inside the package
% only, because \hologo{LaTeX} clears the hook right before
% it loads the package.
%
% However, there are already many packages and classes
% that provide hooks that are executed after the package
% is loaded, see table~\ref{tab:hooks}.
%
% \begin{table}
% \caption{After package hooking}
% \label{tab:hooks}
% \begin{tabular}{ll}
% \bfseries Macro & \bfseries Provider\\
% \hline
% \cs{AfterPackage} & package \xpackage{scrlfile} \cite{scrlfile}\\
% \cs{AtEndOfPackageFile} & package \xpackage{filehook} \cite{filehook}\\
% \cs{AtEndPackage} & class \xclass{memoir} \cite{memoir}
% \end{tabular}
% \end{table}
%
% Package \xpackage{hopatch} can be used without the packages
% of table~\ref{tab:hooks}. But for an early executing right
% after a package is loaded, one of the following class or
% packages should be loaded before using \cs{hopatch@AfterPackage}:
% \begin{itemize}
%  \item package \xpackage{filehook}
%  \item package \xpackage{scrlfile}
%  \item class \xclass{memoir}
% \end{itemize}
%
% \begin{table}
% \caption{After begin document hooking}
% \begin{tabular}{ll}
% \bfseries Macro & \bfseries Provider\\
% \hline
% \cs{AtBeginDocument} & \hologo{LaTeX}'s kernel\\
% \cs{AtEndPreamble} & package \xpackage{etoolbox} \cite{etoolbox}\\
% \cs{AfterEndPreamble} & package \xpackage{etoolbox}
% \end{tabular}
% \end{table}
%
% Therefore I skip writing a new package for hooking into
% \hologo{LaTeX}'s package management and use this package
% to provide a wrapper to patch a package after it is loaded.
%
% \begin{declcs}{hopatch@AfterPackage} \M{package} \M{patch code}
% \end{declcs}
% If the package is already loaded, the \meta{patch code} is executed
% immediately.
% Otherwise the \meta{patch code} is stored in a command and tried
% at later locations until the package is available.
%
% The patch is tried in the following order:
% \begin{enumerate}
% \item If the package is already loaded, the patch is
%   applied immediately. Further locations are not tried.
% \item \cs{AtEndPackage}, provided by class \xclass{memoir} \cite{memoir},
%   and \cs{AfterPackage}, provided by package \xpackage{scrlfile} \cite{scrlfile},
%   are called right after the package file is input before
%   the hook of \hologo{LaTeX}'s \cs{AtEndOfPackage}.
% \item \cs{AtEndOfPackageFile}, provided by package \xpackage{filehook} \cite{filehook},
%   is called after the package is loaded and after the hook of
%   \hologo{LaTeX}'s \cs{AtEndOfPackage}.
% \item \cs{AtEndPreamble}, provided by package \xpackage{etoolbox} \cite{etoolbox},
%   is called at the beginning of |\begin{document}| before the hook
%   of \hologo{LaTeX}'s \cs{AtBeginDocument}.
% \item \cs{AtBeginDocument}, provided by \hologo{LaTeX}.
% \item \cs{AfterEndDocument}, provided by package \xpackage{etoolbox} \cite{etoolbox},
%   is called at the very end of |\begin{document}|. Preamble commands are
%   already forbidden there.
% \end{enumerate}
% Because of the various locations the patch code is restricted to
% limitations:
% \begin{itemize}
% \item
% Preamble commands, see \hologo{LaTeX}'s \cs{@onlypreamble}
% throw an error if used after |\begin{document}|. This is already
% the case for \cs{AfterEndDocument}. Therefore preamble commands
% are forbidden in the patching code.
% There are four exceptions \cs{@ifpackageloaded}, \cs{@ifclassloaded},
% \cs{@ifpackagelater} and \cs{@ifclasslater}. They are redefined
% during \cs{AfterEndDocument} using the counterparts of
% package \xpackage{ltxcmds} \cite{ltxcmds}.
% \item
% \cs{AfterPackage} of package \xpackage{scrlfile} and \cs{AtEndPackage}
% of class \xclass{memoir} call the hook before \hologo{LaTeX}'s
% \cs{AtEndOfPackage}.
% \end{itemize}
%
% \StopEventually{
% }
%
% \section{Implementation}
%
%    \begin{macrocode}
%<*package>
%    \end{macrocode}
%
% \subsection{Catcodes and package identification}
%
%    \begin{macrocode}
\begingroup\catcode61\catcode48\catcode32=10\relax%
  \catcode13=5 % ^^M
  \endlinechar=13 %
  \catcode123=1 % {
  \catcode125=2 % }
  \catcode64=11 % @
  \def\x{\endgroup
    \expandafter\edef\csname HOpatch@AtEnd\endcsname{%
      \endlinechar=\the\endlinechar\relax
      \catcode13=\the\catcode13\relax
      \catcode32=\the\catcode32\relax
      \catcode35=\the\catcode35\relax
      \catcode61=\the\catcode61\relax
      \catcode64=\the\catcode64\relax
      \catcode123=\the\catcode123\relax
      \catcode125=\the\catcode125\relax
    }%
  }%
\x\catcode61\catcode48\catcode32=10\relax%
\catcode13=5 % ^^M
\endlinechar=13 %
\catcode35=6 % #
\catcode64=11 % @
\catcode123=1 % {
\catcode125=2 % }
\def\TMP@EnsureCode#1#2{%
  \edef\HOpatch@AtEnd{%
    \HOpatch@AtEnd
    \catcode#1=\the\catcode#1\relax
  }%
  \catcode#1=#2\relax
}
\TMP@EnsureCode{40}{12}% (
\TMP@EnsureCode{41}{12}% )
\TMP@EnsureCode{43}{12}% +
\TMP@EnsureCode{46}{12}% .
\TMP@EnsureCode{47}{12}% /
\TMP@EnsureCode{91}{12}% [
\TMP@EnsureCode{93}{12}% ]
\edef\HOpatch@AtEnd{\HOpatch@AtEnd\noexpand\endinput}
%    \end{macrocode}
%
%    Package identification.
%    \begin{macrocode}
\NeedsTeXFormat{LaTeX2e}
\ProvidesPackage{hopatch}%
  [2012/05/28 v1.2 Wrapper for package hooks (HO)]
%    \end{macrocode}
%
% \subsection{Resources}
%
%    \begin{macrocode}
\begingroup\expandafter\expandafter\expandafter\endgroup
\expandafter\ifx\csname RequirePackage\endcsname\relax
  \def\TMP@RequirePackage#1[#2]{%
    \begingroup\expandafter\expandafter\expandafter\endgroup
    \expandafter\ifx\csname ver@#1.sty\endcsname\relax
      \input #1.sty\relax
    \fi
  }%
  \TMP@RequirePackage{ltxcmds}[2010/12/12]%
\else
  \RequirePackage{ltxcmds}[2010/12/12]%
\fi
%    \end{macrocode}
%
%    \begin{macro}{\HOpatch@counter}
%    \begin{macrocode}
\def\HOpatch@counter{0}%
%    \end{macrocode}
%    \end{macro}
%    \begin{macro}{\HOpatch@StepCounter}
%    \begin{macrocode}
\ltx@ifundefined{numexpr}{%
  \def\HOpatch@StepCounter{%
    \begingroup
      \count@\HOpatch@counter\relax
      \advance\count@\ltx@one\relax
    \edef\x{\endgroup
      \noexpand\def\noexpand\HOpatch@counter{\the\count@}%
    }%
    \x
  }%
}{%
  \def\HOpatch@StepCounter{%
    \edef\HOpatch@counter{%
      \the\numexpr\HOpatch@counter+\ltx@one\relax
    }%
  }%
}
%    \end{macrocode}
%    \end{macro}
%
%    \begin{macro}{\HOpatch@list}
%    \begin{macrocode}
\def\HOpatch@list{}
%    \end{macrocode}
%    \end{macro}
%
%    \begin{macro}{\HOpatch@Add}
%    \begin{macrocode}
\def\HOpatch@Add{%
  \ltx@LocalAppendToMacro\HOpatch@list
}
%    \end{macrocode}
%    \end{macro}
%
% \subsection{Package patching}
%
%    \begin{macro}{\hopatch@AfterPackage}
%    \begin{macrocode}
\def\hopatch@AfterPackage#1{%
  \ltx@ifpackageloaded{#1}{%
    \ltx@firstofone
  }{%
    \HOpatch@AfterPackage{#1}%
  }%
}
%    \end{macrocode}
%    \end{macro}
%    \begin{macro}{\HOpatch@AfterPackage}
%    \begin{macrocode}
\def\HOpatch@AfterPackage#1{%
  \edef\HOpatch@temp{#1}%
  \HOpatch@StepCounter
  \expandafter\HOpatch@@AfterPackage
  \csname HOpatch@\HOpatch@counter\expandafter\endcsname{%
    \HOpatch@temp
  }%
}
%    \end{macrocode}
%    \end{macro}
%
%    \begin{macro}{\HOpatch@@AfterPackage}
%    \begin{macrocode}
\def\HOpatch@@AfterPackage#1#2#3{%
  \begingroup
    \toks@{#3}%
    \xdef\HOpatch@gtemp{%
      \noexpand\ltx@ifpackageloaded{#2}{%
        \noexpand\let\noexpand#1\noexpand\relax
        \the\toks@
      }{}%
    }%
  \endgroup
  \let#1\HOpatch@gtemp
  \HOpatch@Add#1%
  \HOpatch@Try{AfterPackage}{#2}#1%
  \HOpatch@Try{AtEndPackage}{#2}#1%
  \HOpatch@Try{AtEndOfPackageFile}{#2}#1%
}
%    \end{macrocode}
%    \end{macro}
%
%    \begin{macro}{\HOpatch@Try}
%    \begin{macrocode}
\def\HOpatch@Try#1#2#3{%
  \ltx@ifundefined{#1}{}{%
    \csname #1\endcsname{#2}{#3}%
  }%
}
%    \end{macrocode}
%    \end{macro}
%
%    \begin{macrocode}
\AtBeginDocument{\HOpatch@list}
\ltx@ifundefined{AtEndPreamble}{}{%
  \ltx@ifundefined{@endpreamblehook}{}{%
    \AtEndPreamble{\HOpatch@list}%
  }%
}
\ltx@ifundefined{AfterEndPreamble}{}{%
  \ltx@ifundefined{@afterendpreamblehook}{}{%
    \AfterEndPreamble{%
      \let\HOpatch@OrgIfPackageLoaded\@ifpackageloaded
      \let\HOpatch@OrgIfPackageLater\@ifpackagelater
      \let\HOpatch@OrgIfClassLoaded\@ifclassloaded
      \let\HOpatch@OrgIfClassLater\@ifclasslater
      \let\@ifpackageloaded\ltx@ifpackageloaded
      \let\@ifpackagelater\ltx@ifpackagelater
      \let\@ifclassloaded\ltx@ifclassloaded
      \let\@ifclasslater\ltx@ifclasslater
      \HOpatch@list
      \let\@ifpackageloaded\HOpatch@OrgIfPackageLoaded
      \let\@ifpackagelater\HOpatch@OrgIfPackageLater
      \let\@ifclassloaded\HOpatch@OrgIfClassLoaded
      \let\@ifclasslater\HOpatch@OrgIfClassLater
    }%
  }%
}
%    \end{macrocode}
%
%    \begin{macrocode}
\HOpatch@AtEnd%
%    \end{macrocode}
%    \begin{macrocode}
%</package>
%    \end{macrocode}
%
% \section{Test}
%
%    \begin{macrocode}
%<*test1>
\def\LoadCommand{\RequirePackage{hopatch}[2012/05/28]}
%</test1>
%    \end{macrocode}
% \subsection{Catcode checks for loading}
%
%    \begin{macrocode}
%<*test1>
%    \end{macrocode}
%    \begin{macrocode}
\catcode`\{=1 %
\catcode`\}=2 %
\catcode`\#=6 %
\catcode`\@=11 %
\expandafter\ifx\csname count@\endcsname\relax
  \countdef\count@=255 %
\fi
\expandafter\ifx\csname @gobble\endcsname\relax
  \long\def\@gobble#1{}%
\fi
\expandafter\ifx\csname @firstofone\endcsname\relax
  \long\def\@firstofone#1{#1}%
\fi
\expandafter\ifx\csname loop\endcsname\relax
  \expandafter\@firstofone
\else
  \expandafter\@gobble
\fi
{%
  \def\loop#1\repeat{%
    \def\body{#1}%
    \iterate
  }%
  \def\iterate{%
    \body
      \let\next\iterate
    \else
      \let\next\relax
    \fi
    \next
  }%
  \let\repeat=\fi
}%
\def\RestoreCatcodes{}
\count@=0 %
\loop
  \edef\RestoreCatcodes{%
    \RestoreCatcodes
    \catcode\the\count@=\the\catcode\count@\relax
  }%
\ifnum\count@<255 %
  \advance\count@ 1 %
\repeat

\def\RangeCatcodeInvalid#1#2{%
  \count@=#1\relax
  \loop
    \catcode\count@=15 %
  \ifnum\count@<#2\relax
    \advance\count@ 1 %
  \repeat
}
\def\RangeCatcodeCheck#1#2#3{%
  \count@=#1\relax
  \loop
    \ifnum#3=\catcode\count@
    \else
      \errmessage{%
        Character \the\count@\space
        with wrong catcode \the\catcode\count@\space
        instead of \number#3%
      }%
    \fi
  \ifnum\count@<#2\relax
    \advance\count@ 1 %
  \repeat
}
\def\space{ }
\expandafter\ifx\csname LoadCommand\endcsname\relax
  \def\LoadCommand{\input hopatch.sty\relax}%
\fi
\def\Test{%
  \RangeCatcodeInvalid{0}{47}%
  \RangeCatcodeInvalid{58}{64}%
  \RangeCatcodeInvalid{91}{96}%
  \RangeCatcodeInvalid{123}{255}%
  \catcode`\@=12 %
  \catcode`\\=0 %
  \catcode`\%=14 %
  \LoadCommand
  \RangeCatcodeCheck{0}{36}{15}%
  \RangeCatcodeCheck{37}{37}{14}%
  \RangeCatcodeCheck{38}{47}{15}%
  \RangeCatcodeCheck{48}{57}{12}%
  \RangeCatcodeCheck{58}{63}{15}%
  \RangeCatcodeCheck{64}{64}{12}%
  \RangeCatcodeCheck{65}{90}{11}%
  \RangeCatcodeCheck{91}{91}{15}%
  \RangeCatcodeCheck{92}{92}{0}%
  \RangeCatcodeCheck{93}{96}{15}%
  \RangeCatcodeCheck{97}{122}{11}%
  \RangeCatcodeCheck{123}{255}{15}%
  \RestoreCatcodes
}
\Test
\csname @@end\endcsname
\end
%    \end{macrocode}
%    \begin{macrocode}
%</test1>
%    \end{macrocode}
%
%    \begin{macrocode}
%<*test2>
\NeedsTeXFormat{LaTeX2e}
\providecommand\variant{0}
\RequirePackage{filecontents}
\begin{filecontents}{foo.sty}
\ProvidesPackage{foo}
\def\msg#{\immediate\write16}
\def\foo#1{%
  \msg{\fooformat{#1}}%
}
\def\fooformat#1{[#1]}% hash-ok
\foo{* Executing foo at package loading}
\end{filecontents}

\ifnum\variant=1 %
  \documentclass{memoir}%
\else
  \documentclass{article}%
\fi

\ifcase\variant\relax
\or % 1
\or % 2
  \usepackage{etoolbox}%
\or % 3
  \usepackage{scrlfile}%
\or % 4
  \usepackage{filehook}%
\fi

\AtBeginDocument{\foo{* AtBeginDocument before hopatch}}
\usepackage{hopatch}
\AtBeginDocument{\foo{* AtBeginDocument after hopatch}}

\makeatletter
\hopatch@AfterPackage{foo}{%
  \def\fooformat#1{<<#1>>}%
}
\makeatother

\AtBeginDocument{\foo{* AtBeginDocument before foo}}
\usepackage{foo}
\AtBeginDocument{\foo{* AtBeginDocument after foo}}

\foo{* Executing in preamble}

\begin{document}
\foo{* Executing in document}
\end{document}
%</test2>
%    \end{macrocode}
%
% \section{Installation}
%
% \subsection{Download}
%
% \paragraph{Package.} This package is available on
% CTAN\footnote{\url{ftp://ftp.ctan.org/tex-archive/}}:
% \begin{description}
% \item[\CTAN{macros/latex/contrib/oberdiek/hopatch.dtx}] The source file.
% \item[\CTAN{macros/latex/contrib/oberdiek/hopatch.pdf}] Documentation.
% \end{description}
%
%
% \paragraph{Bundle.} All the packages of the bundle `oberdiek'
% are also available in a TDS compliant ZIP archive. There
% the packages are already unpacked and the documentation files
% are generated. The files and directories obey the TDS standard.
% \begin{description}
% \item[\CTAN{install/macros/latex/contrib/oberdiek.tds.zip}]
% \end{description}
% \emph{TDS} refers to the standard ``A Directory Structure
% for \TeX\ Files'' (\CTAN{tds/tds.pdf}). Directories
% with \xfile{texmf} in their name are usually organized this way.
%
% \subsection{Bundle installation}
%
% \paragraph{Unpacking.} Unpack the \xfile{oberdiek.tds.zip} in the
% TDS tree (also known as \xfile{texmf} tree) of your choice.
% Example (linux):
% \begin{quote}
%   |unzip oberdiek.tds.zip -d ~/texmf|
% \end{quote}
%
% \paragraph{Script installation.}
% Check the directory \xfile{TDS:scripts/oberdiek/} for
% scripts that need further installation steps.
% Package \xpackage{attachfile2} comes with the Perl script
% \xfile{pdfatfi.pl} that should be installed in such a way
% that it can be called as \texttt{pdfatfi}.
% Example (linux):
% \begin{quote}
%   |chmod +x scripts/oberdiek/pdfatfi.pl|\\
%   |cp scripts/oberdiek/pdfatfi.pl /usr/local/bin/|
% \end{quote}
%
% \subsection{Package installation}
%
% \paragraph{Unpacking.} The \xfile{.dtx} file is a self-extracting
% \docstrip\ archive. The files are extracted by running the
% \xfile{.dtx} through \plainTeX:
% \begin{quote}
%   \verb|tex hopatch.dtx|
% \end{quote}
%
% \paragraph{TDS.} Now the different files must be moved into
% the different directories in your installation TDS tree
% (also known as \xfile{texmf} tree):
% \begin{quote}
% \def\t{^^A
% \begin{tabular}{@{}>{\ttfamily}l@{ $\rightarrow$ }>{\ttfamily}l@{}}
%   hopatch.sty & tex/latex/oberdiek/hopatch.sty\\
%   hopatch.pdf & doc/latex/oberdiek/hopatch.pdf\\
%   test/hopatch-test1.tex & doc/latex/oberdiek/test/hopatch-test1.tex\\
%   test/hopatch-test2.tex & doc/latex/oberdiek/test/hopatch-test2.tex\\
%   hopatch.dtx & source/latex/oberdiek/hopatch.dtx\\
% \end{tabular}^^A
% }^^A
% \sbox0{\t}^^A
% \ifdim\wd0>\linewidth
%   \begingroup
%     \advance\linewidth by\leftmargin
%     \advance\linewidth by\rightmargin
%   \edef\x{\endgroup
%     \def\noexpand\lw{\the\linewidth}^^A
%   }\x
%   \def\lwbox{^^A
%     \leavevmode
%     \hbox to \linewidth{^^A
%       \kern-\leftmargin\relax
%       \hss
%       \usebox0
%       \hss
%       \kern-\rightmargin\relax
%     }^^A
%   }^^A
%   \ifdim\wd0>\lw
%     \sbox0{\small\t}^^A
%     \ifdim\wd0>\linewidth
%       \ifdim\wd0>\lw
%         \sbox0{\footnotesize\t}^^A
%         \ifdim\wd0>\linewidth
%           \ifdim\wd0>\lw
%             \sbox0{\scriptsize\t}^^A
%             \ifdim\wd0>\linewidth
%               \ifdim\wd0>\lw
%                 \sbox0{\tiny\t}^^A
%                 \ifdim\wd0>\linewidth
%                   \lwbox
%                 \else
%                   \usebox0
%                 \fi
%               \else
%                 \lwbox
%               \fi
%             \else
%               \usebox0
%             \fi
%           \else
%             \lwbox
%           \fi
%         \else
%           \usebox0
%         \fi
%       \else
%         \lwbox
%       \fi
%     \else
%       \usebox0
%     \fi
%   \else
%     \lwbox
%   \fi
% \else
%   \usebox0
% \fi
% \end{quote}
% If you have a \xfile{docstrip.cfg} that configures and enables \docstrip's
% TDS installing feature, then some files can already be in the right
% place, see the documentation of \docstrip.
%
% \subsection{Refresh file name databases}
%
% If your \TeX~distribution
% (\teTeX, \mikTeX, \dots) relies on file name databases, you must refresh
% these. For example, \teTeX\ users run \verb|texhash| or
% \verb|mktexlsr|.
%
% \subsection{Some details for the interested}
%
% \paragraph{Attached source.}
%
% The PDF documentation on CTAN also includes the
% \xfile{.dtx} source file. It can be extracted by
% AcrobatReader 6 or higher. Another option is \textsf{pdftk},
% e.g. unpack the file into the current directory:
% \begin{quote}
%   \verb|pdftk hopatch.pdf unpack_files output .|
% \end{quote}
%
% \paragraph{Unpacking with \LaTeX.}
% The \xfile{.dtx} chooses its action depending on the format:
% \begin{description}
% \item[\plainTeX:] Run \docstrip\ and extract the files.
% \item[\LaTeX:] Generate the documentation.
% \end{description}
% If you insist on using \LaTeX\ for \docstrip\ (really,
% \docstrip\ does not need \LaTeX), then inform the autodetect routine
% about your intention:
% \begin{quote}
%   \verb|latex \let\install=y\input{hopatch.dtx}|
% \end{quote}
% Do not forget to quote the argument according to the demands
% of your shell.
%
% \paragraph{Generating the documentation.}
% You can use both the \xfile{.dtx} or the \xfile{.drv} to generate
% the documentation. The process can be configured by the
% configuration file \xfile{ltxdoc.cfg}. For instance, put this
% line into this file, if you want to have A4 as paper format:
% \begin{quote}
%   \verb|\PassOptionsToClass{a4paper}{article}|
% \end{quote}
% An example follows how to generate the
% documentation with pdf\LaTeX:
% \begin{quote}
%\begin{verbatim}
%pdflatex hopatch.dtx
%makeindex -s gind.ist hopatch.idx
%pdflatex hopatch.dtx
%makeindex -s gind.ist hopatch.idx
%pdflatex hopatch.dtx
%\end{verbatim}
% \end{quote}
%
% \section{Catalogue}
%
% The following XML file can be used as source for the
% \href{http://mirror.ctan.org/help/Catalogue/catalogue.html}{\TeX\ Catalogue}.
% The elements \texttt{caption} and \texttt{description} are imported
% from the original XML file from the Catalogue.
% The name of the XML file in the Catalogue is \xfile{hopatch.xml}.
%    \begin{macrocode}
%<*catalogue>
<?xml version='1.0' encoding='us-ascii'?>
<!DOCTYPE entry SYSTEM 'catalogue.dtd'>
<entry datestamp='$Date$' modifier='$Author$' id='hopatch'>
  <name>hopatch</name>
  <caption>Load patches for packages.</caption>
  <authorref id='auth:oberdiek'/>
  <copyright owner='Heiko Oberdiek' year='2011,2012'/>
  <license type='lppl1.3'/>
  <version number='1.2'/>
  <description>
    The hopatch package provides a command with which the user may
    register a piece of patch code for a particular package.  Hopatch
    will apply the patch immediately, if the relevant package has
    already been loaded; otherwise it will store the patch until the
    package appears.
    <p/>
    The package is part of the <xref refid='oberdiek'>oberdiek</xref> bundle.
  </description>
  <documentation details='Package documentation'
      href='ctan:/macros/latex/contrib/oberdiek/hopatch.pdf'/>
  <ctan file='true' path='/macros/latex/contrib/oberdiek/hopatch.dtx'/>
  <miktex location='oberdiek'/>
  <texlive location='oberdiek'/>
  <install path='/macros/latex/contrib/oberdiek/oberdiek.tds.zip'/>
</entry>
%</catalogue>
%    \end{macrocode}
%
% \begin{thebibliography}{9}
% \raggedright
%
% \bibitem{etoolbox}
% Philipp Lehman:
% \textit{The etoolbox Package}
% 2011-01-03.
% \CTAN{macros/latex/contrib/etoolbox/etoolbox.pdf}
%
% \bibitem{filehook}
% Martin Scharrer:
% \textit{The filehook Package};
% 2011-01-09.
% \CTAN{macros/latex/contrib/filehook/filehook.pdf}
%
% \bibitem{ltxcmds}
% Heiko Oberdiek:
% \textit{The ltxcmds Package};
% 2010-12-12.
% \CTAN{macros/latex/contrib/oberdiek/ltxcmds.pdf}
%
% \bibitem{memoir}
% Peter Wilson, Lars Madsen:
% \textit{The Memoir Class for Configurable Typesetting, User Guide};
% 2010.
% \CTAN{macros/latex/contrib/memoir/memman.pdf}
%
% \bibitem{scrlfile}
% Markus Kohm, Jens-Uwe Morawski:
% \textit{The Guide KOMA-Script};
% 2011-01-20.
% \CTAN{macros/latex/contrib/koma-script/scrguien.pdf}
%
% \end{thebibliography}
%
% \begin{History}
%   \begin{Version}{2011/01/30 v1.0}
%   \item
%     First public version.
%   \end{Version}
%   \begin{Version}{2011/06/24 v1.1}
%   \item
%     Fix the use of \cs{AtEndPreamble} and \cs{AfterEndPreamble}.
%     They are redefined by package \xpackage{etoolbox} after their
%     hooks are used and generate an error message then.
%   \end{Version}
%   \begin{Version}{2012/05/28 v1.2}
%   \item
%     Fix for use without \hologo{eTeX}
%     (thanks Gordon Lee).
%   \end{Version}
% \end{History}
%
% \PrintIndex
%
% \Finale
\endinput

%        (quote the arguments according to the demands of your shell)
%
% Documentation:
%    (a) If hopatch.drv is present:
%           latex hopatch.drv
%    (b) Without hopatch.drv:
%           latex hopatch.dtx; ...
%    The class ltxdoc loads the configuration file ltxdoc.cfg
%    if available. Here you can specify further options, e.g.
%    use A4 as paper format:
%       \PassOptionsToClass{a4paper}{article}
%
%    Programm calls to get the documentation (example):
%       pdflatex hopatch.dtx
%       makeindex -s gind.ist hopatch.idx
%       pdflatex hopatch.dtx
%       makeindex -s gind.ist hopatch.idx
%       pdflatex hopatch.dtx
%
% Installation:
%    TDS:tex/latex/oberdiek/hopatch.sty
%    TDS:doc/latex/oberdiek/hopatch.pdf
%    TDS:doc/latex/oberdiek/test/hopatch-test1.tex
%    TDS:doc/latex/oberdiek/test/hopatch-test2.tex
%    TDS:source/latex/oberdiek/hopatch.dtx
%
%<*ignore>
\begingroup
  \catcode123=1 %
  \catcode125=2 %
  \def\x{LaTeX2e}%
\expandafter\endgroup
\ifcase 0\ifx\install y1\fi\expandafter
         \ifx\csname processbatchFile\endcsname\relax\else1\fi
         \ifx\fmtname\x\else 1\fi\relax
\else\csname fi\endcsname
%</ignore>
%<*install>
\input docstrip.tex
\Msg{************************************************************************}
\Msg{* Installation}
\Msg{* Package: hopatch 2012/05/28 v1.2 Wrapper for package hooks (HO)}
\Msg{************************************************************************}

\keepsilent
\askforoverwritefalse

\let\MetaPrefix\relax
\preamble

This is a generated file.

Project: hopatch
Version: 2012/05/28 v1.2

Copyright (C) 2011, 2012 by
   Heiko Oberdiek <heiko.oberdiek at googlemail.com>

This work may be distributed and/or modified under the
conditions of the LaTeX Project Public License, either
version 1.3c of this license or (at your option) any later
version. This version of this license is in
   http://www.latex-project.org/lppl/lppl-1-3c.txt
and the latest version of this license is in
   http://www.latex-project.org/lppl.txt
and version 1.3 or later is part of all distributions of
LaTeX version 2005/12/01 or later.

This work has the LPPL maintenance status "maintained".

This Current Maintainer of this work is Heiko Oberdiek.

This work consists of the main source file hopatch.dtx
and the derived files
   hopatch.sty, hopatch.pdf, hopatch.ins, hopatch.drv,
   hopatch-test1.tex, hopatch-test2.tex.

\endpreamble
\let\MetaPrefix\DoubleperCent

\generate{%
  \file{hopatch.ins}{\from{hopatch.dtx}{install}}%
  \file{hopatch.drv}{\from{hopatch.dtx}{driver}}%
  \usedir{tex/latex/oberdiek}%
  \file{hopatch.sty}{\from{hopatch.dtx}{package}}%
  \usedir{doc/latex/oberdiek/test}%
  \file{hopatch-test1.tex}{\from{hopatch.dtx}{test1}}%
  \file{hopatch-test2.tex}{\from{hopatch.dtx}{test2}}%
  \nopreamble
  \nopostamble
  \usedir{source/latex/oberdiek/catalogue}%
  \file{hopatch.xml}{\from{hopatch.dtx}{catalogue}}%
}

\catcode32=13\relax% active space
\let =\space%
\Msg{************************************************************************}
\Msg{*}
\Msg{* To finish the installation you have to move the following}
\Msg{* file into a directory searched by TeX:}
\Msg{*}
\Msg{*     hopatch.sty}
\Msg{*}
\Msg{* To produce the documentation run the file `hopatch.drv'}
\Msg{* through LaTeX.}
\Msg{*}
\Msg{* Happy TeXing!}
\Msg{*}
\Msg{************************************************************************}

\endbatchfile
%</install>
%<*ignore>
\fi
%</ignore>
%<*driver>
\NeedsTeXFormat{LaTeX2e}
\ProvidesFile{hopatch.drv}%
  [2012/05/28 v1.2 Wrapper for package hooks (HO)]%
\documentclass{ltxdoc}
\usepackage{holtxdoc}[2011/11/22]
\usepackage{caption}
\usepackage{array}
\setlength{\extrarowheight}{1pt}
\renewcommand*{\arraystretch}{1.1}
\begin{document}
  \DocInput{hopatch.dtx}%
\end{document}
%</driver>
% \fi
%
% \CheckSum{466}
%
% \CharacterTable
%  {Upper-case    \A\B\C\D\E\F\G\H\I\J\K\L\M\N\O\P\Q\R\S\T\U\V\W\X\Y\Z
%   Lower-case    \a\b\c\d\e\f\g\h\i\j\k\l\m\n\o\p\q\r\s\t\u\v\w\x\y\z
%   Digits        \0\1\2\3\4\5\6\7\8\9
%   Exclamation   \!     Double quote  \"     Hash (number) \#
%   Dollar        \$     Percent       \%     Ampersand     \&
%   Acute accent  \'     Left paren    \(     Right paren   \)
%   Asterisk      \*     Plus          \+     Comma         \,
%   Minus         \-     Point         \.     Solidus       \/
%   Colon         \:     Semicolon     \;     Less than     \<
%   Equals        \=     Greater than  \>     Question mark \?
%   Commercial at \@     Left bracket  \[     Backslash     \\
%   Right bracket \]     Circumflex    \^     Underscore    \_
%   Grave accent  \`     Left brace    \{     Vertical bar  \|
%   Right brace   \}     Tilde         \~}
%
% \GetFileInfo{hopatch.drv}
%
% \title{The \xpackage{hopatch} package}
% \date{2012/05/28 v1.2}
% \author{Heiko Oberdiek\\\xemail{heiko.oberdiek at googlemail.com}}
%
% \maketitle
%
% \begin{abstract}
% This packages provides a wrapper to various package hooks
% provided by other packages or classes, but does not define
% own hooks.
% \end{abstract}
%
% \tableofcontents
%
% \section{Documentation}
%
% Sometimes I want to add code right after a package
% has been loaded. Examples are bug fixes, adaptations,
% or added features as needed by package \xpackage{hyperref},
% for instance.
%
% Unhappily \hologo{LaTeX} does not provide this kind of
% hook. \cs{AtEndOfPackage} can be used inside the package
% only, because \hologo{LaTeX} clears the hook right before
% it loads the package.
%
% However, there are already many packages and classes
% that provide hooks that are executed after the package
% is loaded, see table~\ref{tab:hooks}.
%
% \begin{table}
% \caption{After package hooking}
% \label{tab:hooks}
% \begin{tabular}{ll}
% \bfseries Macro & \bfseries Provider\\
% \hline
% \cs{AfterPackage} & package \xpackage{scrlfile} \cite{scrlfile}\\
% \cs{AtEndOfPackageFile} & package \xpackage{filehook} \cite{filehook}\\
% \cs{AtEndPackage} & class \xclass{memoir} \cite{memoir}
% \end{tabular}
% \end{table}
%
% Package \xpackage{hopatch} can be used without the packages
% of table~\ref{tab:hooks}. But for an early executing right
% after a package is loaded, one of the following class or
% packages should be loaded before using \cs{hopatch@AfterPackage}:
% \begin{itemize}
%  \item package \xpackage{filehook}
%  \item package \xpackage{scrlfile}
%  \item class \xclass{memoir}
% \end{itemize}
%
% \begin{table}
% \caption{After begin document hooking}
% \begin{tabular}{ll}
% \bfseries Macro & \bfseries Provider\\
% \hline
% \cs{AtBeginDocument} & \hologo{LaTeX}'s kernel\\
% \cs{AtEndPreamble} & package \xpackage{etoolbox} \cite{etoolbox}\\
% \cs{AfterEndPreamble} & package \xpackage{etoolbox}
% \end{tabular}
% \end{table}
%
% Therefore I skip writing a new package for hooking into
% \hologo{LaTeX}'s package management and use this package
% to provide a wrapper to patch a package after it is loaded.
%
% \begin{declcs}{hopatch@AfterPackage} \M{package} \M{patch code}
% \end{declcs}
% If the package is already loaded, the \meta{patch code} is executed
% immediately.
% Otherwise the \meta{patch code} is stored in a command and tried
% at later locations until the package is available.
%
% The patch is tried in the following order:
% \begin{enumerate}
% \item If the package is already loaded, the patch is
%   applied immediately. Further locations are not tried.
% \item \cs{AtEndPackage}, provided by class \xclass{memoir} \cite{memoir},
%   and \cs{AfterPackage}, provided by package \xpackage{scrlfile} \cite{scrlfile},
%   are called right after the package file is input before
%   the hook of \hologo{LaTeX}'s \cs{AtEndOfPackage}.
% \item \cs{AtEndOfPackageFile}, provided by package \xpackage{filehook} \cite{filehook},
%   is called after the package is loaded and after the hook of
%   \hologo{LaTeX}'s \cs{AtEndOfPackage}.
% \item \cs{AtEndPreamble}, provided by package \xpackage{etoolbox} \cite{etoolbox},
%   is called at the beginning of |\begin{document}| before the hook
%   of \hologo{LaTeX}'s \cs{AtBeginDocument}.
% \item \cs{AtBeginDocument}, provided by \hologo{LaTeX}.
% \item \cs{AfterEndDocument}, provided by package \xpackage{etoolbox} \cite{etoolbox},
%   is called at the very end of |\begin{document}|. Preamble commands are
%   already forbidden there.
% \end{enumerate}
% Because of the various locations the patch code is restricted to
% limitations:
% \begin{itemize}
% \item
% Preamble commands, see \hologo{LaTeX}'s \cs{@onlypreamble}
% throw an error if used after |\begin{document}|. This is already
% the case for \cs{AfterEndDocument}. Therefore preamble commands
% are forbidden in the patching code.
% There are four exceptions \cs{@ifpackageloaded}, \cs{@ifclassloaded},
% \cs{@ifpackagelater} and \cs{@ifclasslater}. They are redefined
% during \cs{AfterEndDocument} using the counterparts of
% package \xpackage{ltxcmds} \cite{ltxcmds}.
% \item
% \cs{AfterPackage} of package \xpackage{scrlfile} and \cs{AtEndPackage}
% of class \xclass{memoir} call the hook before \hologo{LaTeX}'s
% \cs{AtEndOfPackage}.
% \end{itemize}
%
% \StopEventually{
% }
%
% \section{Implementation}
%
%    \begin{macrocode}
%<*package>
%    \end{macrocode}
%
% \subsection{Catcodes and package identification}
%
%    \begin{macrocode}
\begingroup\catcode61\catcode48\catcode32=10\relax%
  \catcode13=5 % ^^M
  \endlinechar=13 %
  \catcode123=1 % {
  \catcode125=2 % }
  \catcode64=11 % @
  \def\x{\endgroup
    \expandafter\edef\csname HOpatch@AtEnd\endcsname{%
      \endlinechar=\the\endlinechar\relax
      \catcode13=\the\catcode13\relax
      \catcode32=\the\catcode32\relax
      \catcode35=\the\catcode35\relax
      \catcode61=\the\catcode61\relax
      \catcode64=\the\catcode64\relax
      \catcode123=\the\catcode123\relax
      \catcode125=\the\catcode125\relax
    }%
  }%
\x\catcode61\catcode48\catcode32=10\relax%
\catcode13=5 % ^^M
\endlinechar=13 %
\catcode35=6 % #
\catcode64=11 % @
\catcode123=1 % {
\catcode125=2 % }
\def\TMP@EnsureCode#1#2{%
  \edef\HOpatch@AtEnd{%
    \HOpatch@AtEnd
    \catcode#1=\the\catcode#1\relax
  }%
  \catcode#1=#2\relax
}
\TMP@EnsureCode{40}{12}% (
\TMP@EnsureCode{41}{12}% )
\TMP@EnsureCode{43}{12}% +
\TMP@EnsureCode{46}{12}% .
\TMP@EnsureCode{47}{12}% /
\TMP@EnsureCode{91}{12}% [
\TMP@EnsureCode{93}{12}% ]
\edef\HOpatch@AtEnd{\HOpatch@AtEnd\noexpand\endinput}
%    \end{macrocode}
%
%    Package identification.
%    \begin{macrocode}
\NeedsTeXFormat{LaTeX2e}
\ProvidesPackage{hopatch}%
  [2012/05/28 v1.2 Wrapper for package hooks (HO)]
%    \end{macrocode}
%
% \subsection{Resources}
%
%    \begin{macrocode}
\begingroup\expandafter\expandafter\expandafter\endgroup
\expandafter\ifx\csname RequirePackage\endcsname\relax
  \def\TMP@RequirePackage#1[#2]{%
    \begingroup\expandafter\expandafter\expandafter\endgroup
    \expandafter\ifx\csname ver@#1.sty\endcsname\relax
      \input #1.sty\relax
    \fi
  }%
  \TMP@RequirePackage{ltxcmds}[2010/12/12]%
\else
  \RequirePackage{ltxcmds}[2010/12/12]%
\fi
%    \end{macrocode}
%
%    \begin{macro}{\HOpatch@counter}
%    \begin{macrocode}
\def\HOpatch@counter{0}%
%    \end{macrocode}
%    \end{macro}
%    \begin{macro}{\HOpatch@StepCounter}
%    \begin{macrocode}
\ltx@ifundefined{numexpr}{%
  \def\HOpatch@StepCounter{%
    \begingroup
      \count@\HOpatch@counter\relax
      \advance\count@\ltx@one\relax
    \edef\x{\endgroup
      \noexpand\def\noexpand\HOpatch@counter{\the\count@}%
    }%
    \x
  }%
}{%
  \def\HOpatch@StepCounter{%
    \edef\HOpatch@counter{%
      \the\numexpr\HOpatch@counter+\ltx@one\relax
    }%
  }%
}
%    \end{macrocode}
%    \end{macro}
%
%    \begin{macro}{\HOpatch@list}
%    \begin{macrocode}
\def\HOpatch@list{}
%    \end{macrocode}
%    \end{macro}
%
%    \begin{macro}{\HOpatch@Add}
%    \begin{macrocode}
\def\HOpatch@Add{%
  \ltx@LocalAppendToMacro\HOpatch@list
}
%    \end{macrocode}
%    \end{macro}
%
% \subsection{Package patching}
%
%    \begin{macro}{\hopatch@AfterPackage}
%    \begin{macrocode}
\def\hopatch@AfterPackage#1{%
  \ltx@ifpackageloaded{#1}{%
    \ltx@firstofone
  }{%
    \HOpatch@AfterPackage{#1}%
  }%
}
%    \end{macrocode}
%    \end{macro}
%    \begin{macro}{\HOpatch@AfterPackage}
%    \begin{macrocode}
\def\HOpatch@AfterPackage#1{%
  \edef\HOpatch@temp{#1}%
  \HOpatch@StepCounter
  \expandafter\HOpatch@@AfterPackage
  \csname HOpatch@\HOpatch@counter\expandafter\endcsname{%
    \HOpatch@temp
  }%
}
%    \end{macrocode}
%    \end{macro}
%
%    \begin{macro}{\HOpatch@@AfterPackage}
%    \begin{macrocode}
\def\HOpatch@@AfterPackage#1#2#3{%
  \begingroup
    \toks@{#3}%
    \xdef\HOpatch@gtemp{%
      \noexpand\ltx@ifpackageloaded{#2}{%
        \noexpand\let\noexpand#1\noexpand\relax
        \the\toks@
      }{}%
    }%
  \endgroup
  \let#1\HOpatch@gtemp
  \HOpatch@Add#1%
  \HOpatch@Try{AfterPackage}{#2}#1%
  \HOpatch@Try{AtEndPackage}{#2}#1%
  \HOpatch@Try{AtEndOfPackageFile}{#2}#1%
}
%    \end{macrocode}
%    \end{macro}
%
%    \begin{macro}{\HOpatch@Try}
%    \begin{macrocode}
\def\HOpatch@Try#1#2#3{%
  \ltx@ifundefined{#1}{}{%
    \csname #1\endcsname{#2}{#3}%
  }%
}
%    \end{macrocode}
%    \end{macro}
%
%    \begin{macrocode}
\AtBeginDocument{\HOpatch@list}
\ltx@ifundefined{AtEndPreamble}{}{%
  \ltx@ifundefined{@endpreamblehook}{}{%
    \AtEndPreamble{\HOpatch@list}%
  }%
}
\ltx@ifundefined{AfterEndPreamble}{}{%
  \ltx@ifundefined{@afterendpreamblehook}{}{%
    \AfterEndPreamble{%
      \let\HOpatch@OrgIfPackageLoaded\@ifpackageloaded
      \let\HOpatch@OrgIfPackageLater\@ifpackagelater
      \let\HOpatch@OrgIfClassLoaded\@ifclassloaded
      \let\HOpatch@OrgIfClassLater\@ifclasslater
      \let\@ifpackageloaded\ltx@ifpackageloaded
      \let\@ifpackagelater\ltx@ifpackagelater
      \let\@ifclassloaded\ltx@ifclassloaded
      \let\@ifclasslater\ltx@ifclasslater
      \HOpatch@list
      \let\@ifpackageloaded\HOpatch@OrgIfPackageLoaded
      \let\@ifpackagelater\HOpatch@OrgIfPackageLater
      \let\@ifclassloaded\HOpatch@OrgIfClassLoaded
      \let\@ifclasslater\HOpatch@OrgIfClassLater
    }%
  }%
}
%    \end{macrocode}
%
%    \begin{macrocode}
\HOpatch@AtEnd%
%    \end{macrocode}
%    \begin{macrocode}
%</package>
%    \end{macrocode}
%
% \section{Test}
%
%    \begin{macrocode}
%<*test1>
\def\LoadCommand{\RequirePackage{hopatch}[2012/05/28]}
%</test1>
%    \end{macrocode}
% \subsection{Catcode checks for loading}
%
%    \begin{macrocode}
%<*test1>
%    \end{macrocode}
%    \begin{macrocode}
\catcode`\{=1 %
\catcode`\}=2 %
\catcode`\#=6 %
\catcode`\@=11 %
\expandafter\ifx\csname count@\endcsname\relax
  \countdef\count@=255 %
\fi
\expandafter\ifx\csname @gobble\endcsname\relax
  \long\def\@gobble#1{}%
\fi
\expandafter\ifx\csname @firstofone\endcsname\relax
  \long\def\@firstofone#1{#1}%
\fi
\expandafter\ifx\csname loop\endcsname\relax
  \expandafter\@firstofone
\else
  \expandafter\@gobble
\fi
{%
  \def\loop#1\repeat{%
    \def\body{#1}%
    \iterate
  }%
  \def\iterate{%
    \body
      \let\next\iterate
    \else
      \let\next\relax
    \fi
    \next
  }%
  \let\repeat=\fi
}%
\def\RestoreCatcodes{}
\count@=0 %
\loop
  \edef\RestoreCatcodes{%
    \RestoreCatcodes
    \catcode\the\count@=\the\catcode\count@\relax
  }%
\ifnum\count@<255 %
  \advance\count@ 1 %
\repeat

\def\RangeCatcodeInvalid#1#2{%
  \count@=#1\relax
  \loop
    \catcode\count@=15 %
  \ifnum\count@<#2\relax
    \advance\count@ 1 %
  \repeat
}
\def\RangeCatcodeCheck#1#2#3{%
  \count@=#1\relax
  \loop
    \ifnum#3=\catcode\count@
    \else
      \errmessage{%
        Character \the\count@\space
        with wrong catcode \the\catcode\count@\space
        instead of \number#3%
      }%
    \fi
  \ifnum\count@<#2\relax
    \advance\count@ 1 %
  \repeat
}
\def\space{ }
\expandafter\ifx\csname LoadCommand\endcsname\relax
  \def\LoadCommand{\input hopatch.sty\relax}%
\fi
\def\Test{%
  \RangeCatcodeInvalid{0}{47}%
  \RangeCatcodeInvalid{58}{64}%
  \RangeCatcodeInvalid{91}{96}%
  \RangeCatcodeInvalid{123}{255}%
  \catcode`\@=12 %
  \catcode`\\=0 %
  \catcode`\%=14 %
  \LoadCommand
  \RangeCatcodeCheck{0}{36}{15}%
  \RangeCatcodeCheck{37}{37}{14}%
  \RangeCatcodeCheck{38}{47}{15}%
  \RangeCatcodeCheck{48}{57}{12}%
  \RangeCatcodeCheck{58}{63}{15}%
  \RangeCatcodeCheck{64}{64}{12}%
  \RangeCatcodeCheck{65}{90}{11}%
  \RangeCatcodeCheck{91}{91}{15}%
  \RangeCatcodeCheck{92}{92}{0}%
  \RangeCatcodeCheck{93}{96}{15}%
  \RangeCatcodeCheck{97}{122}{11}%
  \RangeCatcodeCheck{123}{255}{15}%
  \RestoreCatcodes
}
\Test
\csname @@end\endcsname
\end
%    \end{macrocode}
%    \begin{macrocode}
%</test1>
%    \end{macrocode}
%
%    \begin{macrocode}
%<*test2>
\NeedsTeXFormat{LaTeX2e}
\providecommand\variant{0}
\RequirePackage{filecontents}
\begin{filecontents}{foo.sty}
\ProvidesPackage{foo}
\def\msg#{\immediate\write16}
\def\foo#1{%
  \msg{\fooformat{#1}}%
}
\def\fooformat#1{[#1]}% hash-ok
\foo{* Executing foo at package loading}
\end{filecontents}

\ifnum\variant=1 %
  \documentclass{memoir}%
\else
  \documentclass{article}%
\fi

\ifcase\variant\relax
\or % 1
\or % 2
  \usepackage{etoolbox}%
\or % 3
  \usepackage{scrlfile}%
\or % 4
  \usepackage{filehook}%
\fi

\AtBeginDocument{\foo{* AtBeginDocument before hopatch}}
\usepackage{hopatch}
\AtBeginDocument{\foo{* AtBeginDocument after hopatch}}

\makeatletter
\hopatch@AfterPackage{foo}{%
  \def\fooformat#1{<<#1>>}%
}
\makeatother

\AtBeginDocument{\foo{* AtBeginDocument before foo}}
\usepackage{foo}
\AtBeginDocument{\foo{* AtBeginDocument after foo}}

\foo{* Executing in preamble}

\begin{document}
\foo{* Executing in document}
\end{document}
%</test2>
%    \end{macrocode}
%
% \section{Installation}
%
% \subsection{Download}
%
% \paragraph{Package.} This package is available on
% CTAN\footnote{\url{ftp://ftp.ctan.org/tex-archive/}}:
% \begin{description}
% \item[\CTAN{macros/latex/contrib/oberdiek/hopatch.dtx}] The source file.
% \item[\CTAN{macros/latex/contrib/oberdiek/hopatch.pdf}] Documentation.
% \end{description}
%
%
% \paragraph{Bundle.} All the packages of the bundle `oberdiek'
% are also available in a TDS compliant ZIP archive. There
% the packages are already unpacked and the documentation files
% are generated. The files and directories obey the TDS standard.
% \begin{description}
% \item[\CTAN{install/macros/latex/contrib/oberdiek.tds.zip}]
% \end{description}
% \emph{TDS} refers to the standard ``A Directory Structure
% for \TeX\ Files'' (\CTAN{tds/tds.pdf}). Directories
% with \xfile{texmf} in their name are usually organized this way.
%
% \subsection{Bundle installation}
%
% \paragraph{Unpacking.} Unpack the \xfile{oberdiek.tds.zip} in the
% TDS tree (also known as \xfile{texmf} tree) of your choice.
% Example (linux):
% \begin{quote}
%   |unzip oberdiek.tds.zip -d ~/texmf|
% \end{quote}
%
% \paragraph{Script installation.}
% Check the directory \xfile{TDS:scripts/oberdiek/} for
% scripts that need further installation steps.
% Package \xpackage{attachfile2} comes with the Perl script
% \xfile{pdfatfi.pl} that should be installed in such a way
% that it can be called as \texttt{pdfatfi}.
% Example (linux):
% \begin{quote}
%   |chmod +x scripts/oberdiek/pdfatfi.pl|\\
%   |cp scripts/oberdiek/pdfatfi.pl /usr/local/bin/|
% \end{quote}
%
% \subsection{Package installation}
%
% \paragraph{Unpacking.} The \xfile{.dtx} file is a self-extracting
% \docstrip\ archive. The files are extracted by running the
% \xfile{.dtx} through \plainTeX:
% \begin{quote}
%   \verb|tex hopatch.dtx|
% \end{quote}
%
% \paragraph{TDS.} Now the different files must be moved into
% the different directories in your installation TDS tree
% (also known as \xfile{texmf} tree):
% \begin{quote}
% \def\t{^^A
% \begin{tabular}{@{}>{\ttfamily}l@{ $\rightarrow$ }>{\ttfamily}l@{}}
%   hopatch.sty & tex/latex/oberdiek/hopatch.sty\\
%   hopatch.pdf & doc/latex/oberdiek/hopatch.pdf\\
%   test/hopatch-test1.tex & doc/latex/oberdiek/test/hopatch-test1.tex\\
%   test/hopatch-test2.tex & doc/latex/oberdiek/test/hopatch-test2.tex\\
%   hopatch.dtx & source/latex/oberdiek/hopatch.dtx\\
% \end{tabular}^^A
% }^^A
% \sbox0{\t}^^A
% \ifdim\wd0>\linewidth
%   \begingroup
%     \advance\linewidth by\leftmargin
%     \advance\linewidth by\rightmargin
%   \edef\x{\endgroup
%     \def\noexpand\lw{\the\linewidth}^^A
%   }\x
%   \def\lwbox{^^A
%     \leavevmode
%     \hbox to \linewidth{^^A
%       \kern-\leftmargin\relax
%       \hss
%       \usebox0
%       \hss
%       \kern-\rightmargin\relax
%     }^^A
%   }^^A
%   \ifdim\wd0>\lw
%     \sbox0{\small\t}^^A
%     \ifdim\wd0>\linewidth
%       \ifdim\wd0>\lw
%         \sbox0{\footnotesize\t}^^A
%         \ifdim\wd0>\linewidth
%           \ifdim\wd0>\lw
%             \sbox0{\scriptsize\t}^^A
%             \ifdim\wd0>\linewidth
%               \ifdim\wd0>\lw
%                 \sbox0{\tiny\t}^^A
%                 \ifdim\wd0>\linewidth
%                   \lwbox
%                 \else
%                   \usebox0
%                 \fi
%               \else
%                 \lwbox
%               \fi
%             \else
%               \usebox0
%             \fi
%           \else
%             \lwbox
%           \fi
%         \else
%           \usebox0
%         \fi
%       \else
%         \lwbox
%       \fi
%     \else
%       \usebox0
%     \fi
%   \else
%     \lwbox
%   \fi
% \else
%   \usebox0
% \fi
% \end{quote}
% If you have a \xfile{docstrip.cfg} that configures and enables \docstrip's
% TDS installing feature, then some files can already be in the right
% place, see the documentation of \docstrip.
%
% \subsection{Refresh file name databases}
%
% If your \TeX~distribution
% (\teTeX, \mikTeX, \dots) relies on file name databases, you must refresh
% these. For example, \teTeX\ users run \verb|texhash| or
% \verb|mktexlsr|.
%
% \subsection{Some details for the interested}
%
% \paragraph{Attached source.}
%
% The PDF documentation on CTAN also includes the
% \xfile{.dtx} source file. It can be extracted by
% AcrobatReader 6 or higher. Another option is \textsf{pdftk},
% e.g. unpack the file into the current directory:
% \begin{quote}
%   \verb|pdftk hopatch.pdf unpack_files output .|
% \end{quote}
%
% \paragraph{Unpacking with \LaTeX.}
% The \xfile{.dtx} chooses its action depending on the format:
% \begin{description}
% \item[\plainTeX:] Run \docstrip\ and extract the files.
% \item[\LaTeX:] Generate the documentation.
% \end{description}
% If you insist on using \LaTeX\ for \docstrip\ (really,
% \docstrip\ does not need \LaTeX), then inform the autodetect routine
% about your intention:
% \begin{quote}
%   \verb|latex \let\install=y% \iffalse meta-comment
%
% File: hopatch.dtx
% Version: 2012/05/28 v1.2
% Info: Wrapper for package hooks
%
% Copyright (C) 2011, 2012 by
%    Heiko Oberdiek <heiko.oberdiek at googlemail.com>
%
% This work may be distributed and/or modified under the
% conditions of the LaTeX Project Public License, either
% version 1.3c of this license or (at your option) any later
% version. This version of this license is in
%    http://www.latex-project.org/lppl/lppl-1-3c.txt
% and the latest version of this license is in
%    http://www.latex-project.org/lppl.txt
% and version 1.3 or later is part of all distributions of
% LaTeX version 2005/12/01 or later.
%
% This work has the LPPL maintenance status "maintained".
%
% This Current Maintainer of this work is Heiko Oberdiek.
%
% This work consists of the main source file hopatch.dtx
% and the derived files
%    hopatch.sty, hopatch.pdf, hopatch.ins, hopatch.drv,
%    hopatch-test1.tex, hopatch-test2.tex.
%
% Distribution:
%    CTAN:macros/latex/contrib/oberdiek/hopatch.dtx
%    CTAN:macros/latex/contrib/oberdiek/hopatch.pdf
%
% Unpacking:
%    (a) If hopatch.ins is present:
%           tex hopatch.ins
%    (b) Without hopatch.ins:
%           tex hopatch.dtx
%    (c) If you insist on using LaTeX
%           latex \let\install=y\input{hopatch.dtx}
%        (quote the arguments according to the demands of your shell)
%
% Documentation:
%    (a) If hopatch.drv is present:
%           latex hopatch.drv
%    (b) Without hopatch.drv:
%           latex hopatch.dtx; ...
%    The class ltxdoc loads the configuration file ltxdoc.cfg
%    if available. Here you can specify further options, e.g.
%    use A4 as paper format:
%       \PassOptionsToClass{a4paper}{article}
%
%    Programm calls to get the documentation (example):
%       pdflatex hopatch.dtx
%       makeindex -s gind.ist hopatch.idx
%       pdflatex hopatch.dtx
%       makeindex -s gind.ist hopatch.idx
%       pdflatex hopatch.dtx
%
% Installation:
%    TDS:tex/latex/oberdiek/hopatch.sty
%    TDS:doc/latex/oberdiek/hopatch.pdf
%    TDS:doc/latex/oberdiek/test/hopatch-test1.tex
%    TDS:doc/latex/oberdiek/test/hopatch-test2.tex
%    TDS:source/latex/oberdiek/hopatch.dtx
%
%<*ignore>
\begingroup
  \catcode123=1 %
  \catcode125=2 %
  \def\x{LaTeX2e}%
\expandafter\endgroup
\ifcase 0\ifx\install y1\fi\expandafter
         \ifx\csname processbatchFile\endcsname\relax\else1\fi
         \ifx\fmtname\x\else 1\fi\relax
\else\csname fi\endcsname
%</ignore>
%<*install>
\input docstrip.tex
\Msg{************************************************************************}
\Msg{* Installation}
\Msg{* Package: hopatch 2012/05/28 v1.2 Wrapper for package hooks (HO)}
\Msg{************************************************************************}

\keepsilent
\askforoverwritefalse

\let\MetaPrefix\relax
\preamble

This is a generated file.

Project: hopatch
Version: 2012/05/28 v1.2

Copyright (C) 2011, 2012 by
   Heiko Oberdiek <heiko.oberdiek at googlemail.com>

This work may be distributed and/or modified under the
conditions of the LaTeX Project Public License, either
version 1.3c of this license or (at your option) any later
version. This version of this license is in
   http://www.latex-project.org/lppl/lppl-1-3c.txt
and the latest version of this license is in
   http://www.latex-project.org/lppl.txt
and version 1.3 or later is part of all distributions of
LaTeX version 2005/12/01 or later.

This work has the LPPL maintenance status "maintained".

This Current Maintainer of this work is Heiko Oberdiek.

This work consists of the main source file hopatch.dtx
and the derived files
   hopatch.sty, hopatch.pdf, hopatch.ins, hopatch.drv,
   hopatch-test1.tex, hopatch-test2.tex.

\endpreamble
\let\MetaPrefix\DoubleperCent

\generate{%
  \file{hopatch.ins}{\from{hopatch.dtx}{install}}%
  \file{hopatch.drv}{\from{hopatch.dtx}{driver}}%
  \usedir{tex/latex/oberdiek}%
  \file{hopatch.sty}{\from{hopatch.dtx}{package}}%
  \usedir{doc/latex/oberdiek/test}%
  \file{hopatch-test1.tex}{\from{hopatch.dtx}{test1}}%
  \file{hopatch-test2.tex}{\from{hopatch.dtx}{test2}}%
  \nopreamble
  \nopostamble
  \usedir{source/latex/oberdiek/catalogue}%
  \file{hopatch.xml}{\from{hopatch.dtx}{catalogue}}%
}

\catcode32=13\relax% active space
\let =\space%
\Msg{************************************************************************}
\Msg{*}
\Msg{* To finish the installation you have to move the following}
\Msg{* file into a directory searched by TeX:}
\Msg{*}
\Msg{*     hopatch.sty}
\Msg{*}
\Msg{* To produce the documentation run the file `hopatch.drv'}
\Msg{* through LaTeX.}
\Msg{*}
\Msg{* Happy TeXing!}
\Msg{*}
\Msg{************************************************************************}

\endbatchfile
%</install>
%<*ignore>
\fi
%</ignore>
%<*driver>
\NeedsTeXFormat{LaTeX2e}
\ProvidesFile{hopatch.drv}%
  [2012/05/28 v1.2 Wrapper for package hooks (HO)]%
\documentclass{ltxdoc}
\usepackage{holtxdoc}[2011/11/22]
\usepackage{caption}
\usepackage{array}
\setlength{\extrarowheight}{1pt}
\renewcommand*{\arraystretch}{1.1}
\begin{document}
  \DocInput{hopatch.dtx}%
\end{document}
%</driver>
% \fi
%
% \CheckSum{466}
%
% \CharacterTable
%  {Upper-case    \A\B\C\D\E\F\G\H\I\J\K\L\M\N\O\P\Q\R\S\T\U\V\W\X\Y\Z
%   Lower-case    \a\b\c\d\e\f\g\h\i\j\k\l\m\n\o\p\q\r\s\t\u\v\w\x\y\z
%   Digits        \0\1\2\3\4\5\6\7\8\9
%   Exclamation   \!     Double quote  \"     Hash (number) \#
%   Dollar        \$     Percent       \%     Ampersand     \&
%   Acute accent  \'     Left paren    \(     Right paren   \)
%   Asterisk      \*     Plus          \+     Comma         \,
%   Minus         \-     Point         \.     Solidus       \/
%   Colon         \:     Semicolon     \;     Less than     \<
%   Equals        \=     Greater than  \>     Question mark \?
%   Commercial at \@     Left bracket  \[     Backslash     \\
%   Right bracket \]     Circumflex    \^     Underscore    \_
%   Grave accent  \`     Left brace    \{     Vertical bar  \|
%   Right brace   \}     Tilde         \~}
%
% \GetFileInfo{hopatch.drv}
%
% \title{The \xpackage{hopatch} package}
% \date{2012/05/28 v1.2}
% \author{Heiko Oberdiek\\\xemail{heiko.oberdiek at googlemail.com}}
%
% \maketitle
%
% \begin{abstract}
% This packages provides a wrapper to various package hooks
% provided by other packages or classes, but does not define
% own hooks.
% \end{abstract}
%
% \tableofcontents
%
% \section{Documentation}
%
% Sometimes I want to add code right after a package
% has been loaded. Examples are bug fixes, adaptations,
% or added features as needed by package \xpackage{hyperref},
% for instance.
%
% Unhappily \hologo{LaTeX} does not provide this kind of
% hook. \cs{AtEndOfPackage} can be used inside the package
% only, because \hologo{LaTeX} clears the hook right before
% it loads the package.
%
% However, there are already many packages and classes
% that provide hooks that are executed after the package
% is loaded, see table~\ref{tab:hooks}.
%
% \begin{table}
% \caption{After package hooking}
% \label{tab:hooks}
% \begin{tabular}{ll}
% \bfseries Macro & \bfseries Provider\\
% \hline
% \cs{AfterPackage} & package \xpackage{scrlfile} \cite{scrlfile}\\
% \cs{AtEndOfPackageFile} & package \xpackage{filehook} \cite{filehook}\\
% \cs{AtEndPackage} & class \xclass{memoir} \cite{memoir}
% \end{tabular}
% \end{table}
%
% Package \xpackage{hopatch} can be used without the packages
% of table~\ref{tab:hooks}. But for an early executing right
% after a package is loaded, one of the following class or
% packages should be loaded before using \cs{hopatch@AfterPackage}:
% \begin{itemize}
%  \item package \xpackage{filehook}
%  \item package \xpackage{scrlfile}
%  \item class \xclass{memoir}
% \end{itemize}
%
% \begin{table}
% \caption{After begin document hooking}
% \begin{tabular}{ll}
% \bfseries Macro & \bfseries Provider\\
% \hline
% \cs{AtBeginDocument} & \hologo{LaTeX}'s kernel\\
% \cs{AtEndPreamble} & package \xpackage{etoolbox} \cite{etoolbox}\\
% \cs{AfterEndPreamble} & package \xpackage{etoolbox}
% \end{tabular}
% \end{table}
%
% Therefore I skip writing a new package for hooking into
% \hologo{LaTeX}'s package management and use this package
% to provide a wrapper to patch a package after it is loaded.
%
% \begin{declcs}{hopatch@AfterPackage} \M{package} \M{patch code}
% \end{declcs}
% If the package is already loaded, the \meta{patch code} is executed
% immediately.
% Otherwise the \meta{patch code} is stored in a command and tried
% at later locations until the package is available.
%
% The patch is tried in the following order:
% \begin{enumerate}
% \item If the package is already loaded, the patch is
%   applied immediately. Further locations are not tried.
% \item \cs{AtEndPackage}, provided by class \xclass{memoir} \cite{memoir},
%   and \cs{AfterPackage}, provided by package \xpackage{scrlfile} \cite{scrlfile},
%   are called right after the package file is input before
%   the hook of \hologo{LaTeX}'s \cs{AtEndOfPackage}.
% \item \cs{AtEndOfPackageFile}, provided by package \xpackage{filehook} \cite{filehook},
%   is called after the package is loaded and after the hook of
%   \hologo{LaTeX}'s \cs{AtEndOfPackage}.
% \item \cs{AtEndPreamble}, provided by package \xpackage{etoolbox} \cite{etoolbox},
%   is called at the beginning of |\begin{document}| before the hook
%   of \hologo{LaTeX}'s \cs{AtBeginDocument}.
% \item \cs{AtBeginDocument}, provided by \hologo{LaTeX}.
% \item \cs{AfterEndDocument}, provided by package \xpackage{etoolbox} \cite{etoolbox},
%   is called at the very end of |\begin{document}|. Preamble commands are
%   already forbidden there.
% \end{enumerate}
% Because of the various locations the patch code is restricted to
% limitations:
% \begin{itemize}
% \item
% Preamble commands, see \hologo{LaTeX}'s \cs{@onlypreamble}
% throw an error if used after |\begin{document}|. This is already
% the case for \cs{AfterEndDocument}. Therefore preamble commands
% are forbidden in the patching code.
% There are four exceptions \cs{@ifpackageloaded}, \cs{@ifclassloaded},
% \cs{@ifpackagelater} and \cs{@ifclasslater}. They are redefined
% during \cs{AfterEndDocument} using the counterparts of
% package \xpackage{ltxcmds} \cite{ltxcmds}.
% \item
% \cs{AfterPackage} of package \xpackage{scrlfile} and \cs{AtEndPackage}
% of class \xclass{memoir} call the hook before \hologo{LaTeX}'s
% \cs{AtEndOfPackage}.
% \end{itemize}
%
% \StopEventually{
% }
%
% \section{Implementation}
%
%    \begin{macrocode}
%<*package>
%    \end{macrocode}
%
% \subsection{Catcodes and package identification}
%
%    \begin{macrocode}
\begingroup\catcode61\catcode48\catcode32=10\relax%
  \catcode13=5 % ^^M
  \endlinechar=13 %
  \catcode123=1 % {
  \catcode125=2 % }
  \catcode64=11 % @
  \def\x{\endgroup
    \expandafter\edef\csname HOpatch@AtEnd\endcsname{%
      \endlinechar=\the\endlinechar\relax
      \catcode13=\the\catcode13\relax
      \catcode32=\the\catcode32\relax
      \catcode35=\the\catcode35\relax
      \catcode61=\the\catcode61\relax
      \catcode64=\the\catcode64\relax
      \catcode123=\the\catcode123\relax
      \catcode125=\the\catcode125\relax
    }%
  }%
\x\catcode61\catcode48\catcode32=10\relax%
\catcode13=5 % ^^M
\endlinechar=13 %
\catcode35=6 % #
\catcode64=11 % @
\catcode123=1 % {
\catcode125=2 % }
\def\TMP@EnsureCode#1#2{%
  \edef\HOpatch@AtEnd{%
    \HOpatch@AtEnd
    \catcode#1=\the\catcode#1\relax
  }%
  \catcode#1=#2\relax
}
\TMP@EnsureCode{40}{12}% (
\TMP@EnsureCode{41}{12}% )
\TMP@EnsureCode{43}{12}% +
\TMP@EnsureCode{46}{12}% .
\TMP@EnsureCode{47}{12}% /
\TMP@EnsureCode{91}{12}% [
\TMP@EnsureCode{93}{12}% ]
\edef\HOpatch@AtEnd{\HOpatch@AtEnd\noexpand\endinput}
%    \end{macrocode}
%
%    Package identification.
%    \begin{macrocode}
\NeedsTeXFormat{LaTeX2e}
\ProvidesPackage{hopatch}%
  [2012/05/28 v1.2 Wrapper for package hooks (HO)]
%    \end{macrocode}
%
% \subsection{Resources}
%
%    \begin{macrocode}
\begingroup\expandafter\expandafter\expandafter\endgroup
\expandafter\ifx\csname RequirePackage\endcsname\relax
  \def\TMP@RequirePackage#1[#2]{%
    \begingroup\expandafter\expandafter\expandafter\endgroup
    \expandafter\ifx\csname ver@#1.sty\endcsname\relax
      \input #1.sty\relax
    \fi
  }%
  \TMP@RequirePackage{ltxcmds}[2010/12/12]%
\else
  \RequirePackage{ltxcmds}[2010/12/12]%
\fi
%    \end{macrocode}
%
%    \begin{macro}{\HOpatch@counter}
%    \begin{macrocode}
\def\HOpatch@counter{0}%
%    \end{macrocode}
%    \end{macro}
%    \begin{macro}{\HOpatch@StepCounter}
%    \begin{macrocode}
\ltx@ifundefined{numexpr}{%
  \def\HOpatch@StepCounter{%
    \begingroup
      \count@\HOpatch@counter\relax
      \advance\count@\ltx@one\relax
    \edef\x{\endgroup
      \noexpand\def\noexpand\HOpatch@counter{\the\count@}%
    }%
    \x
  }%
}{%
  \def\HOpatch@StepCounter{%
    \edef\HOpatch@counter{%
      \the\numexpr\HOpatch@counter+\ltx@one\relax
    }%
  }%
}
%    \end{macrocode}
%    \end{macro}
%
%    \begin{macro}{\HOpatch@list}
%    \begin{macrocode}
\def\HOpatch@list{}
%    \end{macrocode}
%    \end{macro}
%
%    \begin{macro}{\HOpatch@Add}
%    \begin{macrocode}
\def\HOpatch@Add{%
  \ltx@LocalAppendToMacro\HOpatch@list
}
%    \end{macrocode}
%    \end{macro}
%
% \subsection{Package patching}
%
%    \begin{macro}{\hopatch@AfterPackage}
%    \begin{macrocode}
\def\hopatch@AfterPackage#1{%
  \ltx@ifpackageloaded{#1}{%
    \ltx@firstofone
  }{%
    \HOpatch@AfterPackage{#1}%
  }%
}
%    \end{macrocode}
%    \end{macro}
%    \begin{macro}{\HOpatch@AfterPackage}
%    \begin{macrocode}
\def\HOpatch@AfterPackage#1{%
  \edef\HOpatch@temp{#1}%
  \HOpatch@StepCounter
  \expandafter\HOpatch@@AfterPackage
  \csname HOpatch@\HOpatch@counter\expandafter\endcsname{%
    \HOpatch@temp
  }%
}
%    \end{macrocode}
%    \end{macro}
%
%    \begin{macro}{\HOpatch@@AfterPackage}
%    \begin{macrocode}
\def\HOpatch@@AfterPackage#1#2#3{%
  \begingroup
    \toks@{#3}%
    \xdef\HOpatch@gtemp{%
      \noexpand\ltx@ifpackageloaded{#2}{%
        \noexpand\let\noexpand#1\noexpand\relax
        \the\toks@
      }{}%
    }%
  \endgroup
  \let#1\HOpatch@gtemp
  \HOpatch@Add#1%
  \HOpatch@Try{AfterPackage}{#2}#1%
  \HOpatch@Try{AtEndPackage}{#2}#1%
  \HOpatch@Try{AtEndOfPackageFile}{#2}#1%
}
%    \end{macrocode}
%    \end{macro}
%
%    \begin{macro}{\HOpatch@Try}
%    \begin{macrocode}
\def\HOpatch@Try#1#2#3{%
  \ltx@ifundefined{#1}{}{%
    \csname #1\endcsname{#2}{#3}%
  }%
}
%    \end{macrocode}
%    \end{macro}
%
%    \begin{macrocode}
\AtBeginDocument{\HOpatch@list}
\ltx@ifundefined{AtEndPreamble}{}{%
  \ltx@ifundefined{@endpreamblehook}{}{%
    \AtEndPreamble{\HOpatch@list}%
  }%
}
\ltx@ifundefined{AfterEndPreamble}{}{%
  \ltx@ifundefined{@afterendpreamblehook}{}{%
    \AfterEndPreamble{%
      \let\HOpatch@OrgIfPackageLoaded\@ifpackageloaded
      \let\HOpatch@OrgIfPackageLater\@ifpackagelater
      \let\HOpatch@OrgIfClassLoaded\@ifclassloaded
      \let\HOpatch@OrgIfClassLater\@ifclasslater
      \let\@ifpackageloaded\ltx@ifpackageloaded
      \let\@ifpackagelater\ltx@ifpackagelater
      \let\@ifclassloaded\ltx@ifclassloaded
      \let\@ifclasslater\ltx@ifclasslater
      \HOpatch@list
      \let\@ifpackageloaded\HOpatch@OrgIfPackageLoaded
      \let\@ifpackagelater\HOpatch@OrgIfPackageLater
      \let\@ifclassloaded\HOpatch@OrgIfClassLoaded
      \let\@ifclasslater\HOpatch@OrgIfClassLater
    }%
  }%
}
%    \end{macrocode}
%
%    \begin{macrocode}
\HOpatch@AtEnd%
%    \end{macrocode}
%    \begin{macrocode}
%</package>
%    \end{macrocode}
%
% \section{Test}
%
%    \begin{macrocode}
%<*test1>
\def\LoadCommand{\RequirePackage{hopatch}[2012/05/28]}
%</test1>
%    \end{macrocode}
% \subsection{Catcode checks for loading}
%
%    \begin{macrocode}
%<*test1>
%    \end{macrocode}
%    \begin{macrocode}
\catcode`\{=1 %
\catcode`\}=2 %
\catcode`\#=6 %
\catcode`\@=11 %
\expandafter\ifx\csname count@\endcsname\relax
  \countdef\count@=255 %
\fi
\expandafter\ifx\csname @gobble\endcsname\relax
  \long\def\@gobble#1{}%
\fi
\expandafter\ifx\csname @firstofone\endcsname\relax
  \long\def\@firstofone#1{#1}%
\fi
\expandafter\ifx\csname loop\endcsname\relax
  \expandafter\@firstofone
\else
  \expandafter\@gobble
\fi
{%
  \def\loop#1\repeat{%
    \def\body{#1}%
    \iterate
  }%
  \def\iterate{%
    \body
      \let\next\iterate
    \else
      \let\next\relax
    \fi
    \next
  }%
  \let\repeat=\fi
}%
\def\RestoreCatcodes{}
\count@=0 %
\loop
  \edef\RestoreCatcodes{%
    \RestoreCatcodes
    \catcode\the\count@=\the\catcode\count@\relax
  }%
\ifnum\count@<255 %
  \advance\count@ 1 %
\repeat

\def\RangeCatcodeInvalid#1#2{%
  \count@=#1\relax
  \loop
    \catcode\count@=15 %
  \ifnum\count@<#2\relax
    \advance\count@ 1 %
  \repeat
}
\def\RangeCatcodeCheck#1#2#3{%
  \count@=#1\relax
  \loop
    \ifnum#3=\catcode\count@
    \else
      \errmessage{%
        Character \the\count@\space
        with wrong catcode \the\catcode\count@\space
        instead of \number#3%
      }%
    \fi
  \ifnum\count@<#2\relax
    \advance\count@ 1 %
  \repeat
}
\def\space{ }
\expandafter\ifx\csname LoadCommand\endcsname\relax
  \def\LoadCommand{\input hopatch.sty\relax}%
\fi
\def\Test{%
  \RangeCatcodeInvalid{0}{47}%
  \RangeCatcodeInvalid{58}{64}%
  \RangeCatcodeInvalid{91}{96}%
  \RangeCatcodeInvalid{123}{255}%
  \catcode`\@=12 %
  \catcode`\\=0 %
  \catcode`\%=14 %
  \LoadCommand
  \RangeCatcodeCheck{0}{36}{15}%
  \RangeCatcodeCheck{37}{37}{14}%
  \RangeCatcodeCheck{38}{47}{15}%
  \RangeCatcodeCheck{48}{57}{12}%
  \RangeCatcodeCheck{58}{63}{15}%
  \RangeCatcodeCheck{64}{64}{12}%
  \RangeCatcodeCheck{65}{90}{11}%
  \RangeCatcodeCheck{91}{91}{15}%
  \RangeCatcodeCheck{92}{92}{0}%
  \RangeCatcodeCheck{93}{96}{15}%
  \RangeCatcodeCheck{97}{122}{11}%
  \RangeCatcodeCheck{123}{255}{15}%
  \RestoreCatcodes
}
\Test
\csname @@end\endcsname
\end
%    \end{macrocode}
%    \begin{macrocode}
%</test1>
%    \end{macrocode}
%
%    \begin{macrocode}
%<*test2>
\NeedsTeXFormat{LaTeX2e}
\providecommand\variant{0}
\RequirePackage{filecontents}
\begin{filecontents}{foo.sty}
\ProvidesPackage{foo}
\def\msg#{\immediate\write16}
\def\foo#1{%
  \msg{\fooformat{#1}}%
}
\def\fooformat#1{[#1]}% hash-ok
\foo{* Executing foo at package loading}
\end{filecontents}

\ifnum\variant=1 %
  \documentclass{memoir}%
\else
  \documentclass{article}%
\fi

\ifcase\variant\relax
\or % 1
\or % 2
  \usepackage{etoolbox}%
\or % 3
  \usepackage{scrlfile}%
\or % 4
  \usepackage{filehook}%
\fi

\AtBeginDocument{\foo{* AtBeginDocument before hopatch}}
\usepackage{hopatch}
\AtBeginDocument{\foo{* AtBeginDocument after hopatch}}

\makeatletter
\hopatch@AfterPackage{foo}{%
  \def\fooformat#1{<<#1>>}%
}
\makeatother

\AtBeginDocument{\foo{* AtBeginDocument before foo}}
\usepackage{foo}
\AtBeginDocument{\foo{* AtBeginDocument after foo}}

\foo{* Executing in preamble}

\begin{document}
\foo{* Executing in document}
\end{document}
%</test2>
%    \end{macrocode}
%
% \section{Installation}
%
% \subsection{Download}
%
% \paragraph{Package.} This package is available on
% CTAN\footnote{\url{ftp://ftp.ctan.org/tex-archive/}}:
% \begin{description}
% \item[\CTAN{macros/latex/contrib/oberdiek/hopatch.dtx}] The source file.
% \item[\CTAN{macros/latex/contrib/oberdiek/hopatch.pdf}] Documentation.
% \end{description}
%
%
% \paragraph{Bundle.} All the packages of the bundle `oberdiek'
% are also available in a TDS compliant ZIP archive. There
% the packages are already unpacked and the documentation files
% are generated. The files and directories obey the TDS standard.
% \begin{description}
% \item[\CTAN{install/macros/latex/contrib/oberdiek.tds.zip}]
% \end{description}
% \emph{TDS} refers to the standard ``A Directory Structure
% for \TeX\ Files'' (\CTAN{tds/tds.pdf}). Directories
% with \xfile{texmf} in their name are usually organized this way.
%
% \subsection{Bundle installation}
%
% \paragraph{Unpacking.} Unpack the \xfile{oberdiek.tds.zip} in the
% TDS tree (also known as \xfile{texmf} tree) of your choice.
% Example (linux):
% \begin{quote}
%   |unzip oberdiek.tds.zip -d ~/texmf|
% \end{quote}
%
% \paragraph{Script installation.}
% Check the directory \xfile{TDS:scripts/oberdiek/} for
% scripts that need further installation steps.
% Package \xpackage{attachfile2} comes with the Perl script
% \xfile{pdfatfi.pl} that should be installed in such a way
% that it can be called as \texttt{pdfatfi}.
% Example (linux):
% \begin{quote}
%   |chmod +x scripts/oberdiek/pdfatfi.pl|\\
%   |cp scripts/oberdiek/pdfatfi.pl /usr/local/bin/|
% \end{quote}
%
% \subsection{Package installation}
%
% \paragraph{Unpacking.} The \xfile{.dtx} file is a self-extracting
% \docstrip\ archive. The files are extracted by running the
% \xfile{.dtx} through \plainTeX:
% \begin{quote}
%   \verb|tex hopatch.dtx|
% \end{quote}
%
% \paragraph{TDS.} Now the different files must be moved into
% the different directories in your installation TDS tree
% (also known as \xfile{texmf} tree):
% \begin{quote}
% \def\t{^^A
% \begin{tabular}{@{}>{\ttfamily}l@{ $\rightarrow$ }>{\ttfamily}l@{}}
%   hopatch.sty & tex/latex/oberdiek/hopatch.sty\\
%   hopatch.pdf & doc/latex/oberdiek/hopatch.pdf\\
%   test/hopatch-test1.tex & doc/latex/oberdiek/test/hopatch-test1.tex\\
%   test/hopatch-test2.tex & doc/latex/oberdiek/test/hopatch-test2.tex\\
%   hopatch.dtx & source/latex/oberdiek/hopatch.dtx\\
% \end{tabular}^^A
% }^^A
% \sbox0{\t}^^A
% \ifdim\wd0>\linewidth
%   \begingroup
%     \advance\linewidth by\leftmargin
%     \advance\linewidth by\rightmargin
%   \edef\x{\endgroup
%     \def\noexpand\lw{\the\linewidth}^^A
%   }\x
%   \def\lwbox{^^A
%     \leavevmode
%     \hbox to \linewidth{^^A
%       \kern-\leftmargin\relax
%       \hss
%       \usebox0
%       \hss
%       \kern-\rightmargin\relax
%     }^^A
%   }^^A
%   \ifdim\wd0>\lw
%     \sbox0{\small\t}^^A
%     \ifdim\wd0>\linewidth
%       \ifdim\wd0>\lw
%         \sbox0{\footnotesize\t}^^A
%         \ifdim\wd0>\linewidth
%           \ifdim\wd0>\lw
%             \sbox0{\scriptsize\t}^^A
%             \ifdim\wd0>\linewidth
%               \ifdim\wd0>\lw
%                 \sbox0{\tiny\t}^^A
%                 \ifdim\wd0>\linewidth
%                   \lwbox
%                 \else
%                   \usebox0
%                 \fi
%               \else
%                 \lwbox
%               \fi
%             \else
%               \usebox0
%             \fi
%           \else
%             \lwbox
%           \fi
%         \else
%           \usebox0
%         \fi
%       \else
%         \lwbox
%       \fi
%     \else
%       \usebox0
%     \fi
%   \else
%     \lwbox
%   \fi
% \else
%   \usebox0
% \fi
% \end{quote}
% If you have a \xfile{docstrip.cfg} that configures and enables \docstrip's
% TDS installing feature, then some files can already be in the right
% place, see the documentation of \docstrip.
%
% \subsection{Refresh file name databases}
%
% If your \TeX~distribution
% (\teTeX, \mikTeX, \dots) relies on file name databases, you must refresh
% these. For example, \teTeX\ users run \verb|texhash| or
% \verb|mktexlsr|.
%
% \subsection{Some details for the interested}
%
% \paragraph{Attached source.}
%
% The PDF documentation on CTAN also includes the
% \xfile{.dtx} source file. It can be extracted by
% AcrobatReader 6 or higher. Another option is \textsf{pdftk},
% e.g. unpack the file into the current directory:
% \begin{quote}
%   \verb|pdftk hopatch.pdf unpack_files output .|
% \end{quote}
%
% \paragraph{Unpacking with \LaTeX.}
% The \xfile{.dtx} chooses its action depending on the format:
% \begin{description}
% \item[\plainTeX:] Run \docstrip\ and extract the files.
% \item[\LaTeX:] Generate the documentation.
% \end{description}
% If you insist on using \LaTeX\ for \docstrip\ (really,
% \docstrip\ does not need \LaTeX), then inform the autodetect routine
% about your intention:
% \begin{quote}
%   \verb|latex \let\install=y\input{hopatch.dtx}|
% \end{quote}
% Do not forget to quote the argument according to the demands
% of your shell.
%
% \paragraph{Generating the documentation.}
% You can use both the \xfile{.dtx} or the \xfile{.drv} to generate
% the documentation. The process can be configured by the
% configuration file \xfile{ltxdoc.cfg}. For instance, put this
% line into this file, if you want to have A4 as paper format:
% \begin{quote}
%   \verb|\PassOptionsToClass{a4paper}{article}|
% \end{quote}
% An example follows how to generate the
% documentation with pdf\LaTeX:
% \begin{quote}
%\begin{verbatim}
%pdflatex hopatch.dtx
%makeindex -s gind.ist hopatch.idx
%pdflatex hopatch.dtx
%makeindex -s gind.ist hopatch.idx
%pdflatex hopatch.dtx
%\end{verbatim}
% \end{quote}
%
% \section{Catalogue}
%
% The following XML file can be used as source for the
% \href{http://mirror.ctan.org/help/Catalogue/catalogue.html}{\TeX\ Catalogue}.
% The elements \texttt{caption} and \texttt{description} are imported
% from the original XML file from the Catalogue.
% The name of the XML file in the Catalogue is \xfile{hopatch.xml}.
%    \begin{macrocode}
%<*catalogue>
<?xml version='1.0' encoding='us-ascii'?>
<!DOCTYPE entry SYSTEM 'catalogue.dtd'>
<entry datestamp='$Date$' modifier='$Author$' id='hopatch'>
  <name>hopatch</name>
  <caption>Load patches for packages.</caption>
  <authorref id='auth:oberdiek'/>
  <copyright owner='Heiko Oberdiek' year='2011,2012'/>
  <license type='lppl1.3'/>
  <version number='1.2'/>
  <description>
    The hopatch package provides a command with which the user may
    register a piece of patch code for a particular package.  Hopatch
    will apply the patch immediately, if the relevant package has
    already been loaded; otherwise it will store the patch until the
    package appears.
    <p/>
    The package is part of the <xref refid='oberdiek'>oberdiek</xref> bundle.
  </description>
  <documentation details='Package documentation'
      href='ctan:/macros/latex/contrib/oberdiek/hopatch.pdf'/>
  <ctan file='true' path='/macros/latex/contrib/oberdiek/hopatch.dtx'/>
  <miktex location='oberdiek'/>
  <texlive location='oberdiek'/>
  <install path='/macros/latex/contrib/oberdiek/oberdiek.tds.zip'/>
</entry>
%</catalogue>
%    \end{macrocode}
%
% \begin{thebibliography}{9}
% \raggedright
%
% \bibitem{etoolbox}
% Philipp Lehman:
% \textit{The etoolbox Package}
% 2011-01-03.
% \CTAN{macros/latex/contrib/etoolbox/etoolbox.pdf}
%
% \bibitem{filehook}
% Martin Scharrer:
% \textit{The filehook Package};
% 2011-01-09.
% \CTAN{macros/latex/contrib/filehook/filehook.pdf}
%
% \bibitem{ltxcmds}
% Heiko Oberdiek:
% \textit{The ltxcmds Package};
% 2010-12-12.
% \CTAN{macros/latex/contrib/oberdiek/ltxcmds.pdf}
%
% \bibitem{memoir}
% Peter Wilson, Lars Madsen:
% \textit{The Memoir Class for Configurable Typesetting, User Guide};
% 2010.
% \CTAN{macros/latex/contrib/memoir/memman.pdf}
%
% \bibitem{scrlfile}
% Markus Kohm, Jens-Uwe Morawski:
% \textit{The Guide KOMA-Script};
% 2011-01-20.
% \CTAN{macros/latex/contrib/koma-script/scrguien.pdf}
%
% \end{thebibliography}
%
% \begin{History}
%   \begin{Version}{2011/01/30 v1.0}
%   \item
%     First public version.
%   \end{Version}
%   \begin{Version}{2011/06/24 v1.1}
%   \item
%     Fix the use of \cs{AtEndPreamble} and \cs{AfterEndPreamble}.
%     They are redefined by package \xpackage{etoolbox} after their
%     hooks are used and generate an error message then.
%   \end{Version}
%   \begin{Version}{2012/05/28 v1.2}
%   \item
%     Fix for use without \hologo{eTeX}
%     (thanks Gordon Lee).
%   \end{Version}
% \end{History}
%
% \PrintIndex
%
% \Finale
\endinput
|
% \end{quote}
% Do not forget to quote the argument according to the demands
% of your shell.
%
% \paragraph{Generating the documentation.}
% You can use both the \xfile{.dtx} or the \xfile{.drv} to generate
% the documentation. The process can be configured by the
% configuration file \xfile{ltxdoc.cfg}. For instance, put this
% line into this file, if you want to have A4 as paper format:
% \begin{quote}
%   \verb|\PassOptionsToClass{a4paper}{article}|
% \end{quote}
% An example follows how to generate the
% documentation with pdf\LaTeX:
% \begin{quote}
%\begin{verbatim}
%pdflatex hopatch.dtx
%makeindex -s gind.ist hopatch.idx
%pdflatex hopatch.dtx
%makeindex -s gind.ist hopatch.idx
%pdflatex hopatch.dtx
%\end{verbatim}
% \end{quote}
%
% \section{Catalogue}
%
% The following XML file can be used as source for the
% \href{http://mirror.ctan.org/help/Catalogue/catalogue.html}{\TeX\ Catalogue}.
% The elements \texttt{caption} and \texttt{description} are imported
% from the original XML file from the Catalogue.
% The name of the XML file in the Catalogue is \xfile{hopatch.xml}.
%    \begin{macrocode}
%<*catalogue>
<?xml version='1.0' encoding='us-ascii'?>
<!DOCTYPE entry SYSTEM 'catalogue.dtd'>
<entry datestamp='$Date$' modifier='$Author$' id='hopatch'>
  <name>hopatch</name>
  <caption>Load patches for packages.</caption>
  <authorref id='auth:oberdiek'/>
  <copyright owner='Heiko Oberdiek' year='2011,2012'/>
  <license type='lppl1.3'/>
  <version number='1.2'/>
  <description>
    The hopatch package provides a command with which the user may
    register a piece of patch code for a particular package.  Hopatch
    will apply the patch immediately, if the relevant package has
    already been loaded; otherwise it will store the patch until the
    package appears.
    <p/>
    The package is part of the <xref refid='oberdiek'>oberdiek</xref> bundle.
  </description>
  <documentation details='Package documentation'
      href='ctan:/macros/latex/contrib/oberdiek/hopatch.pdf'/>
  <ctan file='true' path='/macros/latex/contrib/oberdiek/hopatch.dtx'/>
  <miktex location='oberdiek'/>
  <texlive location='oberdiek'/>
  <install path='/macros/latex/contrib/oberdiek/oberdiek.tds.zip'/>
</entry>
%</catalogue>
%    \end{macrocode}
%
% \begin{thebibliography}{9}
% \raggedright
%
% \bibitem{etoolbox}
% Philipp Lehman:
% \textit{The etoolbox Package}
% 2011-01-03.
% \CTAN{macros/latex/contrib/etoolbox/etoolbox.pdf}
%
% \bibitem{filehook}
% Martin Scharrer:
% \textit{The filehook Package};
% 2011-01-09.
% \CTAN{macros/latex/contrib/filehook/filehook.pdf}
%
% \bibitem{ltxcmds}
% Heiko Oberdiek:
% \textit{The ltxcmds Package};
% 2010-12-12.
% \CTAN{macros/latex/contrib/oberdiek/ltxcmds.pdf}
%
% \bibitem{memoir}
% Peter Wilson, Lars Madsen:
% \textit{The Memoir Class for Configurable Typesetting, User Guide};
% 2010.
% \CTAN{macros/latex/contrib/memoir/memman.pdf}
%
% \bibitem{scrlfile}
% Markus Kohm, Jens-Uwe Morawski:
% \textit{The Guide KOMA-Script};
% 2011-01-20.
% \CTAN{macros/latex/contrib/koma-script/scrguien.pdf}
%
% \end{thebibliography}
%
% \begin{History}
%   \begin{Version}{2011/01/30 v1.0}
%   \item
%     First public version.
%   \end{Version}
%   \begin{Version}{2011/06/24 v1.1}
%   \item
%     Fix the use of \cs{AtEndPreamble} and \cs{AfterEndPreamble}.
%     They are redefined by package \xpackage{etoolbox} after their
%     hooks are used and generate an error message then.
%   \end{Version}
%   \begin{Version}{2012/05/28 v1.2}
%   \item
%     Fix for use without \hologo{eTeX}
%     (thanks Gordon Lee).
%   \end{Version}
% \end{History}
%
% \PrintIndex
%
% \Finale
\endinput
|
% \end{quote}
% Do not forget to quote the argument according to the demands
% of your shell.
%
% \paragraph{Generating the documentation.}
% You can use both the \xfile{.dtx} or the \xfile{.drv} to generate
% the documentation. The process can be configured by the
% configuration file \xfile{ltxdoc.cfg}. For instance, put this
% line into this file, if you want to have A4 as paper format:
% \begin{quote}
%   \verb|\PassOptionsToClass{a4paper}{article}|
% \end{quote}
% An example follows how to generate the
% documentation with pdf\LaTeX:
% \begin{quote}
%\begin{verbatim}
%pdflatex hopatch.dtx
%makeindex -s gind.ist hopatch.idx
%pdflatex hopatch.dtx
%makeindex -s gind.ist hopatch.idx
%pdflatex hopatch.dtx
%\end{verbatim}
% \end{quote}
%
% \section{Catalogue}
%
% The following XML file can be used as source for the
% \href{http://mirror.ctan.org/help/Catalogue/catalogue.html}{\TeX\ Catalogue}.
% The elements \texttt{caption} and \texttt{description} are imported
% from the original XML file from the Catalogue.
% The name of the XML file in the Catalogue is \xfile{hopatch.xml}.
%    \begin{macrocode}
%<*catalogue>
<?xml version='1.0' encoding='us-ascii'?>
<!DOCTYPE entry SYSTEM 'catalogue.dtd'>
<entry datestamp='$Date$' modifier='$Author$' id='hopatch'>
  <name>hopatch</name>
  <caption>Load patches for packages.</caption>
  <authorref id='auth:oberdiek'/>
  <copyright owner='Heiko Oberdiek' year='2011,2012'/>
  <license type='lppl1.3'/>
  <version number='1.2'/>
  <description>
    The hopatch package provides a command with which the user may
    register a piece of patch code for a particular package.  Hopatch
    will apply the patch immediately, if the relevant package has
    already been loaded; otherwise it will store the patch until the
    package appears.
    <p/>
    The package is part of the <xref refid='oberdiek'>oberdiek</xref> bundle.
  </description>
  <documentation details='Package documentation'
      href='ctan:/macros/latex/contrib/oberdiek/hopatch.pdf'/>
  <ctan file='true' path='/macros/latex/contrib/oberdiek/hopatch.dtx'/>
  <miktex location='oberdiek'/>
  <texlive location='oberdiek'/>
  <install path='/macros/latex/contrib/oberdiek/oberdiek.tds.zip'/>
</entry>
%</catalogue>
%    \end{macrocode}
%
% \begin{thebibliography}{9}
% \raggedright
%
% \bibitem{etoolbox}
% Philipp Lehman:
% \textit{The etoolbox Package}
% 2011-01-03.
% \CTAN{macros/latex/contrib/etoolbox/etoolbox.pdf}
%
% \bibitem{filehook}
% Martin Scharrer:
% \textit{The filehook Package};
% 2011-01-09.
% \CTAN{macros/latex/contrib/filehook/filehook.pdf}
%
% \bibitem{ltxcmds}
% Heiko Oberdiek:
% \textit{The ltxcmds Package};
% 2010-12-12.
% \CTAN{macros/latex/contrib/oberdiek/ltxcmds.pdf}
%
% \bibitem{memoir}
% Peter Wilson, Lars Madsen:
% \textit{The Memoir Class for Configurable Typesetting, User Guide};
% 2010.
% \CTAN{macros/latex/contrib/memoir/memman.pdf}
%
% \bibitem{scrlfile}
% Markus Kohm, Jens-Uwe Morawski:
% \textit{The Guide KOMA-Script};
% 2011-01-20.
% \CTAN{macros/latex/contrib/koma-script/scrguien.pdf}
%
% \end{thebibliography}
%
% \begin{History}
%   \begin{Version}{2011/01/30 v1.0}
%   \item
%     First public version.
%   \end{Version}
%   \begin{Version}{2011/06/24 v1.1}
%   \item
%     Fix the use of \cs{AtEndPreamble} and \cs{AfterEndPreamble}.
%     They are redefined by package \xpackage{etoolbox} after their
%     hooks are used and generate an error message then.
%   \end{Version}
%   \begin{Version}{2012/05/28 v1.2}
%   \item
%     Fix for use without \hologo{eTeX}
%     (thanks Gordon Lee).
%   \end{Version}
% \end{History}
%
% \PrintIndex
%
% \Finale
\endinput
|
% \end{quote}
% Do not forget to quote the argument according to the demands
% of your shell.
%
% \paragraph{Generating the documentation.}
% You can use both the \xfile{.dtx} or the \xfile{.drv} to generate
% the documentation. The process can be configured by the
% configuration file \xfile{ltxdoc.cfg}. For instance, put this
% line into this file, if you want to have A4 as paper format:
% \begin{quote}
%   \verb|\PassOptionsToClass{a4paper}{article}|
% \end{quote}
% An example follows how to generate the
% documentation with pdf\LaTeX:
% \begin{quote}
%\begin{verbatim}
%pdflatex hopatch.dtx
%makeindex -s gind.ist hopatch.idx
%pdflatex hopatch.dtx
%makeindex -s gind.ist hopatch.idx
%pdflatex hopatch.dtx
%\end{verbatim}
% \end{quote}
%
% \section{Catalogue}
%
% The following XML file can be used as source for the
% \href{http://mirror.ctan.org/help/Catalogue/catalogue.html}{\TeX\ Catalogue}.
% The elements \texttt{caption} and \texttt{description} are imported
% from the original XML file from the Catalogue.
% The name of the XML file in the Catalogue is \xfile{hopatch.xml}.
%    \begin{macrocode}
%<*catalogue>
<?xml version='1.0' encoding='us-ascii'?>
<!DOCTYPE entry SYSTEM 'catalogue.dtd'>
<entry datestamp='$Date$' modifier='$Author$' id='hopatch'>
  <name>hopatch</name>
  <caption>Load patches for packages.</caption>
  <authorref id='auth:oberdiek'/>
  <copyright owner='Heiko Oberdiek' year='2011,2012'/>
  <license type='lppl1.3'/>
  <version number='1.2'/>
  <description>
    The hopatch package provides a command with which the user may
    register a piece of patch code for a particular package.  Hopatch
    will apply the patch immediately, if the relevant package has
    already been loaded; otherwise it will store the patch until the
    package appears.
    <p/>
    The package is part of the <xref refid='oberdiek'>oberdiek</xref> bundle.
  </description>
  <documentation details='Package documentation'
      href='ctan:/macros/latex/contrib/oberdiek/hopatch.pdf'/>
  <ctan file='true' path='/macros/latex/contrib/oberdiek/hopatch.dtx'/>
  <miktex location='oberdiek'/>
  <texlive location='oberdiek'/>
  <install path='/macros/latex/contrib/oberdiek/oberdiek.tds.zip'/>
</entry>
%</catalogue>
%    \end{macrocode}
%
% \begin{thebibliography}{9}
% \raggedright
%
% \bibitem{etoolbox}
% Philipp Lehman:
% \textit{The etoolbox Package}
% 2011-01-03.
% \CTAN{macros/latex/contrib/etoolbox/etoolbox.pdf}
%
% \bibitem{filehook}
% Martin Scharrer:
% \textit{The filehook Package};
% 2011-01-09.
% \CTAN{macros/latex/contrib/filehook/filehook.pdf}
%
% \bibitem{ltxcmds}
% Heiko Oberdiek:
% \textit{The ltxcmds Package};
% 2010-12-12.
% \CTAN{macros/latex/contrib/oberdiek/ltxcmds.pdf}
%
% \bibitem{memoir}
% Peter Wilson, Lars Madsen:
% \textit{The Memoir Class for Configurable Typesetting, User Guide};
% 2010.
% \CTAN{macros/latex/contrib/memoir/memman.pdf}
%
% \bibitem{scrlfile}
% Markus Kohm, Jens-Uwe Morawski:
% \textit{The Guide KOMA-Script};
% 2011-01-20.
% \CTAN{macros/latex/contrib/koma-script/scrguien.pdf}
%
% \end{thebibliography}
%
% \begin{History}
%   \begin{Version}{2011/01/30 v1.0}
%   \item
%     First public version.
%   \end{Version}
%   \begin{Version}{2011/06/24 v1.1}
%   \item
%     Fix the use of \cs{AtEndPreamble} and \cs{AfterEndPreamble}.
%     They are redefined by package \xpackage{etoolbox} after their
%     hooks are used and generate an error message then.
%   \end{Version}
%   \begin{Version}{2012/05/28 v1.2}
%   \item
%     Fix for use without \hologo{eTeX}
%     (thanks Gordon Lee).
%   \end{Version}
% \end{History}
%
% \PrintIndex
%
% \Finale
\endinput
