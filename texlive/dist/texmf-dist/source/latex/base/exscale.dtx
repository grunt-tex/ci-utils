% \iffalse meta-comment
%
% Copyright 1993-2014
% The LaTeX3 Project and any individual authors listed elsewhere
% in this file.
%
% This file is part of the LaTeX base system.
% -------------------------------------------
%
% It may be distributed and/or modified under the
% conditions of the LaTeX Project Public License, either version 1.3c
% of this license or (at your option) any later version.
% The latest version of this license is in
%    http://www.latex-project.org/lppl.txt
% and version 1.3c or later is part of all distributions of LaTeX
% version 2005/12/01 or later.
%
% This file has the LPPL maintenance status "maintained".
%
% The list of all files belonging to the LaTeX base distribution is
% given in the file `manifest.txt'. See also `legal.txt' for additional
% information.
%
% The list of derived (unpacked) files belonging to the distribution
% and covered by LPPL is defined by the unpacking scripts (with
% extension .ins) which are part of the distribution.
%
% \fi
% \iffalse
%%% File: exscale.dtx
%% Copyright (C) 1992-1997,2007 Frank Mittelbach and Rainer Sch\"opf,
%% all rights reserved.
%<package>\NeedsTeXFormat{LaTeX2e}
%<package>\ProvidesPackage{exscale}
%<fd,    >\ProvidesFile{OMXcmex.fd}
%<package,fd>     [2014/09/29 v2.1h
%<package>                         Standard LaTeX package exscale]
%<fd,    >                         exscale package font definitions]
%<*driver>
\documentclass{ltxdoc}
\usepackage{exscale}
\GetFileInfo{exscale.sty}
\begin{document}
\title{The \texttt{exscale} package\thanks
       {This file has version number \fileversion, dated \filedate.}\\
      for use with \LaTeX2e}
\author{Frank Mittelbach \and Rainer Sch\"opf}
\date{\filedate}
 \maketitle
 \DocInput{exscale.dtx}
\end{document}
%</driver>
% \fi
%
% \CheckSum{41}
%
% \changes{v2.1g}{1997/06/16}{Documentation fix.}
% \changes{v2.1d}{1994/05/09}{Added setting \cs{big}... sizes.}
% \changes{v2.1c}{1994/05/08}{Changed to new documentation standards.}
% \changes{v2.1a}{1993/12/12}{Update for \LaTeXe}
%
% \def\MF{\textsf{Metafont}}
% \def\dst{{\normalfont\scshape docstrip}}
%
%
% \section{Introduction}
%
% On popular request, this package implements scaling of the
% math extension font `cmex'. If this package is used the site needs
% scaled versions of the \texttt{cmex10} in the sizes 10.95pt, 12pt,
% 14.4pt , 17.28pt, 20.74pt, and 24.88pt which corresponds to standard
% magsteps using |\magstephalf|, and |\magstep1| through |\magstep5|.
% Additionally `cmex' variants for the sizes 7pt to 9pt are necessary.
% These fonts are part of the AMS font package and can be found on
% many servers.
%
% We don't think that using a scalable `cmex' font improves the
% appearance of the document very much because the linear scaling of
% the font introduces changes to font dimen parameters of the font
% that are used all over the place in \TeX's math placement routines
% and definitely shouldn't be scaled linearly.  As a positive result,
% however, symbols from this font will come out in approximately the
% right size when used with a large basis font size.
%
% We therefore strongly recommend the generation of designed fonts for
% the larger sizes and hope that the corresponding \MF{} sources will
% be available in the near future.
%
%
% \section{Usage}
%
% This file can be used either as a package by placing its name
% in the argument of
% |\usepackage|, or by generating from it a |.fd| file which then
% replaces the normally used |.fd| file |OMXcmex.fd|.
%
% For the second solution you should prepare a small \dst{} command
% file containing the following statements:
% \begin{quote}
%   |\def\batchfile{|\meta{name-of-your-command-file}|}|  \\
%   |\input docstrip.tex|                         \\
%   |\generateFile{OMXcmex.fd}{t}{\from{exscale.dtx}{fd}}|
% \end{quote}
% and then run this file through \LaTeX.
% This would mean however, that your documents may look different if
% they are printed on another installation whereas when using the
% package file they would come out the same on both sites.
%
% \StopEventually{}
%
%
% \section{Implementation}
%
%
%    In case this should produce a font definition file we write
%    a message to the transcript file only.
%    The explicit spaces are necessary in an
%    |.fd| file and the |\string| guards against situations where |`|,
%    |<| or |>| is active.
% \changes{v2.1f}{1995/11/27}{Use \cs{ProvidesFile}}
%    \begin{macrocode}
%\ProvidesFile{OMXcmex.fd}[]% moved to top of the file
%    \end{macrocode}
%
% We now define to which external fonts the font/shape Computer Modern
% Extension Medium Normal. For the sizes 7--10pt we use fonts that are
% designed for the corresponding size and for the higher sizes we use
% scaled versions of the 10pt font since for these sizes no designed
% fonts are easily available.
%    \begin{macrocode}
%<*fd|package>
\DeclareFontShape{OMX}{cmex}{m}{n}{%
%    \end{macrocode}
%    According to Michael Downes `cmex5' and `cmex6' are not
%    distributed currently. Therefore we use `cmex7' in those size but
%    we omit a warning.
%    \begin{macrocode}
      <-8>sfixed * cmex7
      <8> cmex8
      <9> cmex9
      <10> <10.95> <12> <14.4> <17.28> <20.74> <24.88>cmex10
       }{}
%</fd|package>
%    \end{macrocode}
%
% We also redeclare the symbol font for large symbols. This is
% already done when the base format also uses CM math fonts, but it
% doesn't hurt either.
%    \begin{macrocode}
%<*package>
\DeclareSymbolFont{largesymbols}{OMX}{cmex}{m}{n}
%    \end{macrocode}
%
% Finally we redefine the primitive commands |\big|, |\Big|, |\bigg|
% and |\Bigg| to change size with main size changes. To this end we
% need a new dimen register |\big@size|.
%    \begin{macrocode}
\newdimen\big@size
%    \end{macrocode}
% This dimen is set with every size change to $1.2$ times the height
% of a left parenthesis.
% \changes{v2.1e}{1994/11/06}{Renamed \cs{every@size} to
%                             \cs{every@math@size}}
% \changes{v2.1e}{1994/11/06}{Update \cs{big@size} globally}
% \changes{v2.1h}{2001/10/29}{Use box 0 within group to avoid changing this
%                             scratch register}
%    \begin{macrocode}
\addto@hook\every@math@size{%
  \begingroup
     \setbox\z@\vbox{\hbox{$($}\kern\z@}%
     \global\big@size 1.2\ht\z@
  \endgroup}
%    \end{macrocode}
% Now we can define the macros. Note: |\n@space| sets the internal
% variables |\nulldelimiterspace| and |\mathsurround| to zero length.
%    \begin{macrocode}
\def\bBigg@#1#2{%
   {\hbox{$\left#2\vcenter to#1\big@size{}\right.\n@space$}}}
\def\big{\bBigg@\@ne}
\def\Big{\bBigg@{1.5}}
\def\bigg{\bBigg@\tw@}
\def\Bigg{\bBigg@{2.5}}
%</package>
%    \end{macrocode}
%
% \Finale
%
\endinput
