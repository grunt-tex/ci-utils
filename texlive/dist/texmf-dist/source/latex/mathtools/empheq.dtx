% \iffalse meta-comment
%
% Copyright (C) 2002-2007 by Morten Hoegholm
% Copyright (C) 2012      by Lars Madsen
%
% This work may be distributed and/or modified under the
% conditions of the LaTeX Project Public License, either
% version 1.3 of this license or (at your option) any later
% version. The latest version of this license is in
%    http://www.latex-project.org/lppl.txt
% and version 1.3 or later is part of all distributions of
% LaTeX version 2005/12/01 or later.
%
% This work has the LPPL maintenance status "maintained".
%
% This Current Maintainer of this work is
% Lars Madsen
%
% This work consists of the main source file empheq.dtx
% and the derived files
%    empheq.sty, empheq.pdf, empheq.ins, empheq.drv.
%
% Distribution:
%    CTAN:macros/latex/contrib/mh/empheq.dtx
%    CTAN:macros/latex/contrib/mh/empheq.pdf
%
% Unpacking:
%    (a) If empheq.ins is present:
%           tex empheq.ins
%    (b) Without empheq.ins:
%           tex empheq.dtx
%    (c) If you insist on using LaTeX
%           latex \let\install=y% \iffalse meta-comment
%
% Copyright (C) 2002-2007 by Morten Hoegholm
% Copyright (C) 2012      by Lars Madsen
%
% This work may be distributed and/or modified under the
% conditions of the LaTeX Project Public License, either
% version 1.3 of this license or (at your option) any later
% version. The latest version of this license is in
%    http://www.latex-project.org/lppl.txt
% and version 1.3 or later is part of all distributions of
% LaTeX version 2005/12/01 or later.
%
% This work has the LPPL maintenance status "maintained".
%
% This Current Maintainer of this work is
% Lars Madsen
%
% This work consists of the main source file empheq.dtx
% and the derived files
%    empheq.sty, empheq.pdf, empheq.ins, empheq.drv.
%
% Distribution:
%    CTAN:macros/latex/contrib/mh/empheq.dtx
%    CTAN:macros/latex/contrib/mh/empheq.pdf
%
% Unpacking:
%    (a) If empheq.ins is present:
%           tex empheq.ins
%    (b) Without empheq.ins:
%           tex empheq.dtx
%    (c) If you insist on using LaTeX
%           latex \let\install=y% \iffalse meta-comment
%
% Copyright (C) 2002-2007 by Morten Hoegholm
% Copyright (C) 2012      by Lars Madsen
%
% This work may be distributed and/or modified under the
% conditions of the LaTeX Project Public License, either
% version 1.3 of this license or (at your option) any later
% version. The latest version of this license is in
%    http://www.latex-project.org/lppl.txt
% and version 1.3 or later is part of all distributions of
% LaTeX version 2005/12/01 or later.
%
% This work has the LPPL maintenance status "maintained".
%
% This Current Maintainer of this work is
% Lars Madsen
%
% This work consists of the main source file empheq.dtx
% and the derived files
%    empheq.sty, empheq.pdf, empheq.ins, empheq.drv.
%
% Distribution:
%    CTAN:macros/latex/contrib/mh/empheq.dtx
%    CTAN:macros/latex/contrib/mh/empheq.pdf
%
% Unpacking:
%    (a) If empheq.ins is present:
%           tex empheq.ins
%    (b) Without empheq.ins:
%           tex empheq.dtx
%    (c) If you insist on using LaTeX
%           latex \let\install=y% \iffalse meta-comment
%
% Copyright (C) 2002-2007 by Morten Hoegholm
% Copyright (C) 2012      by Lars Madsen
%
% This work may be distributed and/or modified under the
% conditions of the LaTeX Project Public License, either
% version 1.3 of this license or (at your option) any later
% version. The latest version of this license is in
%    http://www.latex-project.org/lppl.txt
% and version 1.3 or later is part of all distributions of
% LaTeX version 2005/12/01 or later.
%
% This work has the LPPL maintenance status "maintained".
%
% This Current Maintainer of this work is
% Lars Madsen
%
% This work consists of the main source file empheq.dtx
% and the derived files
%    empheq.sty, empheq.pdf, empheq.ins, empheq.drv.
%
% Distribution:
%    CTAN:macros/latex/contrib/mh/empheq.dtx
%    CTAN:macros/latex/contrib/mh/empheq.pdf
%
% Unpacking:
%    (a) If empheq.ins is present:
%           tex empheq.ins
%    (b) Without empheq.ins:
%           tex empheq.dtx
%    (c) If you insist on using LaTeX
%           latex \let\install=y\input{empheq.dtx}
%        (quote the arguments according to the demands of your shell)
%
% Documentation:
%    (a) If empheq.drv is present:
%           latex empheq.drv
%    (b) Without empheq.drv:
%           latex empheq.dtx; ...
%    The class ltxdoc loads the configuration file ltxdoc.cfg
%    if available. Here you can specify further options, e.g.
%    use A4 as paper format:
%       \PassOptionsToClass{a4paper}{article}
%
%    Programm calls to get the documentation (example):
%       pdflatex empheq.dtx
%       makeindex -s gind.ist empheq.idx
%       pdflatex empheq.dtx
%       makeindex -s gind.ist empheq.idx
%       pdflatex empheq.dtx
%
% Installation:
%    TDS:tex/latex/mh/empheq.sty
%    TDS:doc/latex/mh/empheq.pdf
%    TDS:source/latex/mh/empheq.dtx
%
%<*ignore>
\begingroup
  \def\x{LaTeX2e}
\expandafter\endgroup
\ifcase 0\ifx\install y1\fi\expandafter
         \ifx\csname processbatchFile\endcsname\relax\else1\fi
         \ifx\fmtname\x\else 1\fi\relax
\else\csname fi\endcsname
%</ignore>
%<*install>
\input docstrip.tex
\Msg{************************************************************************}
\Msg{* Installation}
\Msg{* Package: empheq 2014/08/04 v2.14 empheq}
\Msg{************************************************************************}

\keepsilent
\askforoverwritefalse

\preamble

This is a generated file.

Copyright (C) 2002-2007 by Morten Hoegholm
Copyright (C) 2012-     by Lars Madsen


This work may be distributed and/or modified under the
conditions of the LaTeX Project Public License, either
version 1.3 of this license or (at your option) any later
version. The latest version of this license is in
   http://www.latex-project.org/lppl.txt
and version 1.3 or later is part of all distributions of
LaTeX version 2005/12/01 or later.

This work has the LPPL maintenance status "maintained".

This Current Maintainer of this work is
Lars Madsen

This work consists of the main source file empheq.dtx
and the derived files
   empheq.sty, empheq.pdf, empheq.ins, empheq.drv.

\endpreamble

\generate{%
  \file{empheq.ins}{\from{empheq.dtx}{install}}%
  \file{empheq.drv}{\from{empheq.dtx}{driver}}%
  \usedir{tex/latex/mh}%
  \file{empheq.sty}{\from{empheq.dtx}{package}}%
}

\obeyspaces
\Msg{************************************************************************}
\Msg{*}
\Msg{* To finish the installation you have to move the following}
\Msg{* file into a directory searched by TeX:}
\Msg{*}
\Msg{*     empheq.sty}
\Msg{*}
\Msg{* To produce the documentation run the file `empheq.drv'}
\Msg{* through LaTeX.}
\Msg{*}
\Msg{* Happy TeXing!}
\Msg{*}
\Msg{************************************************************************}

\endbatchfile
%</install>
%<*ignore>
\fi
%</ignore>
%<*driver>
\NeedsTeXFormat{LaTeX2e}
\ProvidesFile{empheq.drv}%
  [2014/08/04 v2.14 Emphasizing equations]
\documentclass{ltxdoc}
\usepackage{mathpazo}
\usepackage[overload]{empheq}
\usepackage{xcolor,varioref,booktabs,fancybox}
\usepackage[overload]{textcase}

\pagestyle{headings}
\hyphenation{emph-eq}
\makeatletter
\providecommand*\pkg[1]{\textsf{#1}}
\providecommand*\env[1]{\texttt{#1}}
\providecommand*\email[1]{\href{mailto:#1}{\texttt{#1}}}
\providecommand*\mode[1]{\texttt{[#1]}}
\providecommand*\file[1]{\texttt{#1}}
\addtolength\marginparwidth{-30pt}


\newcommand*\AddDescribe[3]{%
  \@namedef{Describe#1}{\leavevmode\@bsphack\begingroup\MakePrivateLetters
    \@nameuse{Describe@#1}}
  \@namedef{Describe@#1}##1{\endgroup
                \marginpar{\raggedleft\@nameuse{PrintDescribe#1}{##1}}%
                \@nameuse{Special#1Index}{##1}\@esphack\ignorespaces}
  \@namedef{Special#1Index}##1{\@bsphack
      \index{##1\actualchar{\protect\ttfamily ##1}
            (#2)\encapchar usage}%
      \index{#3:\levelchar##1\actualchar{\protect\ttfamily ##1}\encapchar
            usage}\@esphack}
  \@namedef{PrintDescribe#1}##1{\strut \MacroFont ##1\ }%
  \@namedef{Print#1Name}##1{\strut \MacroFont ##1\ }%
}
\AddDescribe{Color}{color}{colors}
\AddDescribe{KeyFamily}{key family}{key families}
\newcommand*\DescribeDelimiter[1]{%
  \cs{#1}&\SpecialMkhIndex{empheq#1}\cs{empheq}\texttt{#1} &
  \SpecialMkhIndex{empheqbig#1}\cs{empheqbig}\texttt{#1} &
  $\@nameuse{#1}$\tabularnewline}
\newcommand*\SpecialMkhIndex[1]{%
  \@bsphack
    \index{#1\actualchar{\protect\ttfamily \protect\bslash\space#1}
     (` $\bslash#1$ ')\encapchar usage}%
  \@esphack
}
\newcommand\TabDescribeOption[3][(default)]{%
    \SpecialOptIndex{#2}\opt{#2}  & #3 \tabularnewline
}
\newcommand*\TabDescribeAmsKey[2][]{%
    \SpecialKeyIndex{#2}\env{#2} &
    \texttt{\{#2#1\}} &
    \SpecialKeyIndex{#2*}\env{#2*} &
    \texttt{\{#2*#1\}} \tabularnewline
}
\newcommand*\TabDescribeOptKey[3][]{%
    \SpecialKeyIndex{#2}\key{#2} &
    \texttt{#2=}\meta{#3} & #1 \tabularnewline
}

\newenvironment{codesyntax}
    {\par\small\addvspace{4.5ex plus 1ex}%
     \vskip -\parskip
     \noindent
     \begin{tabular}{|l|}\hline\ignorespaces}%
    {\\\hline\end{tabular}\nobreak\par\nobreak
     \vspace{2.3ex}\vskip -\parskip\noindent\ignorespacesafterend}
\makeatletter

\newcommand*\FeatureRequest[2]{%
  \hskip1sp
  \marginpar{%
    \parbox[b]{\marginparwidth}{\small\sffamily\raggedright
      \strut Feature request by\\#1\\#2%
    }
  }%
}

\newcommand*\cttPosting[2]{%
  \hskip1sp
  \marginpar{%
    \parbox[b]{\marginparwidth}{\small\sffamily\raggedright
     \strut Posted on \texttt{comp.text.tex} \\#1\\#2%
    }%
  }%
}

\expandafter\def\expandafter\MakePrivateLetters\expandafter{%
  \MakePrivateLetters  \catcode`\_=11\relax
}

\providecommand*\SpecialOptIndex[1]{%
  \@bsphack
  \index{#1\actualchar{\protect\ttfamily #1}
          (option)\encapchar usage}%
      \index{options:\levelchar#1\actualchar{\protect\ttfamily #1}\encapchar
            usage}\@esphack}
\providecommand*\opt[1]{\texttt{#1}}

\providecommand*\SpecialKeyIndex[1]{%
  \@bsphack
  \index{#1\actualchar{\protect\ttfamily #1}
          (key)\encapchar usage}%
      \index{keys:\levelchar#1\actualchar{\protect\ttfamily #1}\encapchar
            usage}\@esphack}
\providecommand*\key[1]{\textsf{#1}}
\providecommand*\keyfam[1]{\textsf{#1}}

\providecommand*\eTeX{$\m@th\varepsilon$-\TeX}

\def\MTmeta#1{%
     \ensuremath\langle
     \ifmmode \expandafter \nfss@text \fi
     {%
      \meta@font@select
      \edef\meta@hyphen@restore
        {\hyphenchar\the\font\the\hyphenchar\font}%
      \hyphenchar\font\m@ne
      \language\l@nohyphenation
      #1\/%
      \meta@hyphen@restore
     }\ensuremath\rangle
     \endgroup
}
\makeatother
\DeclareRobustCommand\meta{\begingroup\MakePrivateLetters\MTmeta}%
\def\MToarg#1{{\ttfamily[}\meta{#1}{\ttfamily]}\endgroup}
\DeclareRobustCommand\oarg{\begingroup\MakePrivateLetters\MToarg}%
\def\MHmarg#1{{\ttfamily\char`\{}\meta{#1}{\ttfamily\char`\}}\endgroup}
\DeclareRobustCommand\marg{\begingroup\MakePrivateLetters\MHmarg}%
\def\MHarg#1{{\ttfamily\char`\{#1\ttfamily\char`\}}\endgroup}
\DeclareRobustCommand\arg{\begingroup\MakePrivateLetters\MHarg}%
\def\MHcs#1{\texttt{\char`\\#1}\endgroup}
\DeclareRobustCommand\cs{\begingroup\MakePrivateLetters\MHcs}
\def\MHbool#1{{\ttfamily #1}\endgroup}
\DeclareRobustCommand\bool{\begingroup\MakePrivateLetters\MHbool}%

\def\endverbatim{\if@newlist
\leavevmode\fi\endtrivlist\vspace{-\baselineskip}}
\expandafter\let\csname endverbatim*\endcsname =\endverbatim

\let\MTtheindex\theindex
\def\theindex{\MTtheindex\MakePrivateLetters}

\newcommand*\widefbox[1]{\fbox{\hspace{1em}#1\hspace{1em}}}
\definecolor{myblue}{rgb}{.8, .8, 1}
\newcommand*\mybluebox[1]{%
  \colorbox{myblue}{\hspace{1em}#1\hspace{1em}}}
\newenvironment{important}[2][]{%
  \setkeys{EmphEqEnv}{#2}%
  \setkeys{EmphEqOpt}{box=\mybluebox,#1}%
  \EmphEqMainEnv}%
  {\endEmphEqMainEnv}
\definecolor{shadecolor}{cmyk}{0,0,0.41,0}
\definecolor{light-blue}{cmyk}{0.25,0,0,0}
\newsavebox{\mysaveboxM} % M for math
\newsavebox{\mysaveboxT} % T for text
\newcommand*\Garybox[2][Example]{%
  \sbox{\mysaveboxM}{#2}%
  \sbox{\mysaveboxT}{\fcolorbox{black}{light-blue}{#1}}%
  \sbox{\mysaveboxM}{%
    \parbox[b][\ht\mysaveboxM+.5\ht\mysaveboxT+.5\dp\mysaveboxT][b]{%
      \wd\mysaveboxM}{#2}%
  }%
  \sbox{\mysaveboxM}{%
    \fcolorbox{black}{shadecolor}{%
      \makebox[\linewidth-10em]{\usebox{\mysaveboxM}}%
    }%
  }%
  \usebox{\mysaveboxM}%
  \makebox[0pt][r]{%
    \makebox[\wd\mysaveboxM][c]{%
      \raisebox{\ht\mysaveboxM-0.5\ht\mysaveboxT
                +0.5\dp\mysaveboxT-0.5\fboxrule}{\usebox{\mysaveboxT}}%
    }%
  }%
}

\newcommand\EmphEqpart{%
  \def\partname{\EmphEqpartname}%
  \part}

\def\EmphEqpartname{Part}
\makeatother

\usepackage{graphicx}
\usepackage[final,breaklinks=true,hyperindex=false]{hyperref}
\renewcommand*\usage[1]{\textit{\hyperpage{#1}}}

\OnlyDescription
\begin{document}
\DocInput{empheq.dtx}
\end{document}
%</driver>
%  \fi
%  \changes{v2.14}{2014/07/16}{Updated maintenance and contact info}
%
%  \changes{v2.11}{2004/08/03}{Removed interface to the obsolete empheq
%  package \pkg{empheq07}}
%
%  \changes{v2.10}{2004/07/26}{Rewrote most of the documentation}
%
%  \changes{v2.10}{2004/07/26}{Moved options \texttt{fixamsmath} and
%  \texttt{donotfixamsmathbugs} to \pkg{mathtools}}
%
%  \changes{v2.10}{2004/07/26}{Added support for \pkg{showkeys}}
%
%  \changes{v2.00}{2004/04/14}{Added options \opt{overload},
%  \opt{overload2}, \texttt{fixamsmath}, and
%  \texttt{donotfixamsmathbugs} plus descriptions of their use}
%
%  \changes{v1.06}{2004/03/13}{Upgraded license to LPPL version 1.3}
%
%  \changes{v1.00}{2004/02/15}{Complete rewrite of the package hence
%  no description of the changes made}
%
%  \changes{v0.7a}{2004/02/15}{Made the old package obsolete and
%  changed its name to \pkg{empheq-obsolete}}
%
%  \changes{v0.7}{2003/05/11}{Added further examples and changed
%  existing ones}
%  \changes{v0.6b}{2003/03/04}{Added example for changing margins
%  inside the boxtype}
%  \changes{v0.6a}{2003/01/28}{Minor updates to documentation}
%
%  \changes{v0.5}{2002/12/30}{First experimental and very limited
%  release}
%
%  \GetFileInfo{empheq.drv}
%  \CheckSum{2495}
%
%  \title{The \pkg{empheq} package\thanks{This file has version number
%  \fileversion, last revised \filedate.}\\
%  Emphasizing equations in \LaTeXe}
%  \author{Morten H\o gholm, Lars Madsen}
%  
%  \date{\filedate}
%
%  \maketitle
%
%  \begin{abstract}
%  The \pkg{empheq} package can best be described as a visual markup
%  extension to \pkg{amsmath}. In short it offers a)~a multi line
%  equivalent of \cs{boxed} from \pkg{amsmath} and b)~a way to
%  produce arbitrary delimiters that span entire math displays.
%  \end{abstract}
%
%  \tableofcontents
%
%  \clearpage
%  \EmphEqpart{Basic user's guide}
%
%  There can be little doubt that the de facto standard for
%  mathematical typesetting in \LaTeX{} is the \pkg{amsmath} package.
%  For the creation of \pkg{empheq}, a visual markup package for use
%  in math, it made perfect sense to have \pkg{amsmath} as the
%  backbone.
%
%  The main idea of \pkg{empheq} is to maintain the familiar syntax
%  of the \pkg{amsmath} environments while still providing an easy
%  way of specifying markup instructions. This manual is plastered
%  with examples showing just how.
%
%
%  \section{Basic use of the package}
%
%  So what is it \pkg{empheq} does? Well, it allows you to produce
%  displays like this:
%  \begin{subequations}
%  \begin{empheq}[
%    box=\mybluebox,
%    right={\;\makebox[.9em]{$\raisebox{-.5\totalheight+\fontdimen22\textfont2}
%              {\resizebox{!}{\EmphEqdisplayheight+\EmphEqdisplaydepth}{!}}$}},
%    left={X=Y\Rightarrow\empheqlbrace}
%     ]{alignat=3}
%    A_1&=b_1 & \qquad c_1&=d_1 & \qquad e_1&= f_1
%         \tag*{\_A\raisebox{1ex}{silly}\raisebox{-1ex}{tag}\_}\\
%    A_2&=b_2 & \qquad c_2&=d_2 & \qquad  e_2&= f_2 \\
%    A_3&=b_3 & \qquad c_3&=d_3 & \qquad  e_3&= f_3
%  \end{empheq}
%  \end{subequations}
%  In short \pkg{empheq} enables the user to put things on every side
%  of the display without said user having to worry about what
%  happens to the equation numbers. For example you can now have a
%  display containing multiple lines and still get the effect of the
%  \cs{boxed} command from \pkg{amsmath}.
%
%
%  \subsection{Using the \env{empheq} environment}
%
%  The package defines a single environment
%  \SpecialEnvIndex{empheq}\env{empheq} and the usage is kind of straight
%  forward:
%  \begin{codesyntax}
%    |\begin{empheq}|\oarg{markup instructions}\marg{\AmS{} env_name}\\
%    \mbox{}\quad\meta{contents of \AmS{} environment}\\
%    |\end{empheq}|
%  \end{codesyntax}
%  A first minimal example file would then be something like
%  \begin{verbatim}
%    \documentclass{minimal}
%    \usepackage{empheq}
%    \begin{document}
%    \begin{empheq}{align*}
%      a&=b \tag{*}\\
%      E&=mc^2 + \int_a^a x\, dx
%    \end{empheq}
%    \end{document}
%  \end{verbatim}
%  This gives the following display:
%    \begin{empheq}{align*}
%      a&=b \tag{*}\\
%      E&=mc^2 + \int_a^a x\, dx
%    \end{empheq}
%  Maybe not the most impressive example, but as you can see the
%  contents of the environment is exactly the same as for the regular
%  \env{align*} evironment from \pkg{amsmath}. The rest of the \AmS{}
%  environments are chosen the same way by typing the name as the
%  mandatory argument of \env{empheq} with the exception of the
%  \env{alignat} environment. For this you have to specify the number
%  of columns as shown below.
%  \begin{verbatim}
%  \begin{empheq}{alignat=2}
%    a &= b &\quad c &= d \\
%    \text{this} &= \text{that} &\quad \mathit{fish}&\neq fish
%  \end{empheq}
%  \end{verbatim}
%  \begin{empheq}{alignat=2}
%    a &= b &\quad c &= d \\
%    \text{this} &= \text{that} &\quad \mathit{fish}&\neq fish
%  \end{empheq}
%  To choose the starred version of \env{alignat}, simply type
%  \arg{alignat*=2} instead in the above example.
%
%  The supported \AmS{} environments are listed in Table
%  \vref{tab:environments}. Not supported is the standard \LaTeX{}
%  \env{eqnarray} environment as it is fundamentally
%  deficient.\footnote{See for instance
%  \url{http://www.tex.ac.uk/cgi-bin/texfaq2html?label=eqnarray}}
%
%  \begin{table}
%  \centering
%  \begin{tabular}{llll}\toprule
%    Environment & Usage & Environment & Usage \\ \midrule
%    \TabDescribeAmsKey{equation}
%    \TabDescribeAmsKey{align}
%    \TabDescribeAmsKey{gather}
%    \TabDescribeAmsKey{flalign}
%    \TabDescribeAmsKey[=\textrm{\meta{cols}}]{alignat}
%    \TabDescribeAmsKey{multline} \bottomrule
%  \end{tabular}
%  \caption{The supported \pkg{amsmath}
%  environments}\label{tab:environments}
%  \end{table}
%
%
%  \subsection{Markup instructions}
%
%  The optional argument of the \env{empheq} environment will take
%  markup instructions consisting of a $\meta{key}=\meta{value}$ list
%  of assignments. There are currently five such keys (a sixth is
%  added for naming consistency). They're shown in Table
%  \vref{tab:keys}.
%
%  \begin{table}
%  \centering
%  \begin{tabular}{llp{13em}}\toprule
%    Key & Usage & Additional Info \tabularnewline \midrule
%    \TabDescribeOptKey{box}{box command}
%    \TabDescribeOptKey{innerbox}{box command}
%    \TabDescribeOptKey[Use \cs{text}\arg{...} if you need text
%                       material.]{left}{math material}
%    \TabDescribeOptKey[Use \cs{text}\arg{...} if you need text
%                       material.]{right}{math material}
%    \TabDescribeOptKey[Alias for \key{box}.]{outerbox}{box command}
%    \TabDescribeOptKey[Contents must be horizontally centered;
%                       can only be used in \mode{fleqn} mode.]
%                       {marginbox}{box command}
%    \bottomrule
%  \end{tabular}
%  \caption{The six keys for the optional argument of the
%  \env{empheq} environment}\label{tab:keys}
%  \end{table}
%
%  \begin{codesyntax}
%    \SpecialKeyIndex{left}\key{left}
%  \end{codesyntax}
%  The key \key{left} is for material put on the left side of the
%  display. The material is typeset in math mode and centered
%  vertically.
%  \begin{verbatim}
%  \begin{empheq}[left=L\Rightarrow]{align}
%    a&=b\\
%    E&=mc^2 + \int_a^a x\, dx
%  \end{empheq}
%  \end{verbatim}
%  \begin{empheq}[left=L\Rightarrow]{align}
%    a&=b\\
%    E&=mc^2 + \int_a^a x\, dx
%  \end{empheq}
%
%
%  \begin{codesyntax}
%    \SpecialKeyIndex{right}\key{right}
%  \end{codesyntax}
%  As there is a \key{left} key it hopefully comes as no surprise
%  that there is a companion \key{right} key for typesetting material
%  on the right side of the display.
%  \begin{verbatim}
%  \begin{empheq}[right=\Leftarrow R]{align}
%    a&=b\\
%    E&=mc^2 + \int_a^a x\, dx
%  \end{empheq}
%  \end{verbatim}
%  \begin{empheq}[right=\Leftarrow R]{align}
%    a&=b\\
%    E&=mc^2 + \int_a^a x\, dx
%  \end{empheq}
%
%  \begin{codesyntax}
%    \SpecialKeyIndex{box}\key{box}\\
%    \SpecialKeyIndex{outerbox}\key{outerbox}
%  \end{codesyntax}
%  The key \key{box} specifies the kind of box you would like to put
%  around the display. It can be any kind of box, as long as the
%  contents of the box is situated on the baseline like in a
%  \fbox{\cs{fbox}}.
%  \begin{verbatim}
%  \begin{empheq}[box=\fbox]{align}
%    a&=b\\
%    E&=mc^2 + \int_a^a x\, dx
%  \end{empheq}
%  \end{verbatim}
%  \begin{empheq}[box=\fbox]{align}
%    a&=b\\
%    E&=mc^2 + \int_a^a x\, dx
%  \end{empheq}
%  The key \key{outerbox} is an alias for \key{box} and is added for
%  naming consistency with the key described below.
%
%  \begin{codesyntax}
%    \SpecialKeyIndex{innerbox}\key{innerbox}
%  \end{codesyntax}
%  There is also an \key{innerbox} key. It is not very interesting
%  unless you use one of the other keys.
%  \begin{verbatim}
%  \begin{empheq}[innerbox=\fbox,
%                 left=L\Rightarrow]{align}
%    a&=b\\
%    E&=mc^2 + \int_a^a x\, dx
%  \end{empheq}
%  \end{verbatim}
%  \begin{empheq}[innerbox=\fbox,
%                 left=L\Rightarrow]{align}
%    a&=b\\
%    E&=mc^2 + \int_a^a x\, dx
%  \end{empheq}
%
%  \begin{codesyntax}
%    \SpecialKeyIndex{marginbox}\key{marginbox}
%  \end{codesyntax}
%  \FeatureRequest{Uwe Siart}{2003/12/08}
%  The last key is the \key{marginbox} key. If you typeset your math
%  in \mode{fleqn} mode\footnote{Sorry, but I can't show you because
%  this manual features centered math displays.} you may want the
%  math display and not the outer box to align at the left margin (or
%  rather: the indentation). If you make sure the contents of the
%  outer box is centered horizontally inside the box, \key{marginbox}
%  will align it properly. You shouldn't set both \key{box} and
%  \key{marginbox} at the same time, as this is surely not what you
%  want and the package will silently use the last one in the list.
%
%
%  \begin{quote}
%  \textbf{Warning}: \pkg{keyval} treats commas and equal signs as
%  separator and assignment signs which in turn means that if you
%  want to typeset either of them, they \emph{must}---and believe me,
%  it's really important---be enclosed in braces. In short:
%  \begin{description}
%    \item[{\makebox[3em][l]{Good}}] \texttt{[left=\{(A,B)=(1,0)\}]}
%    \item[{\makebox[3em][l]{Bad}}]  \texttt{[left=(A,B)=(1,0)]}
%  \end{description}
%  \end{quote}
%
%  \begin{codesyntax}
%    \SpecialUsageIndex{\empheqset}\cs{empheqset}\marg{markup instructions}
%  \end{codesyntax}
%  \FeatureRequest{Uwe Siart}{2004/01/13}
%  You can also set the keys globally\footnote{Or rather: outside the
%  scope of the \env{empheq} environment. The settings will still
%  obey scoping rules.} with the command
%  \cs{empheqset}. This means that
%  \begin{verbatim}
%    \empheqset{marginbox=\psframebox}
%  \end{verbatim}
%  will force a |box=\psframebox| in all occurrences of the
%  \env{empheq} environment, but an explicitly given \key{box} or
%  \key{marginbox} will override this setting. You can only use the
%  keys listed in this section as arguments to \cs{empheqset}; not
%  the \AmS\ environments.
%
%
%
%  \subsection{What won't work in the \env{empheq} environment}
%
%  \begin{codesyntax}
%    \SpecialUsageIndex{intertext}\cs{intertext}\marg{text}\\
%    \SpecialUsageIndex{displaybreak}\cs{displaybreak}\oarg{num}
%  \end{codesyntax}
%  Now we've seen some of the things that work in the \env{empheq}
%  environment but we also have to take note of what won't work. As
%  this package provides a way to box multi line math displays it can
%  come as no surprise that using either \cs{intertext} or
%  \cs{displaybreak} inside the \env{empheq} environment makes no
%  sense. Should you however happen to try them anyway, you'll
%  experience that \cs{intertext} issues an error message and
%  \cs{displaybreak} issues a warning. The reason only \cs{intertext}
%  gives an error message is that you'll get output very different
%  from what you expect, and that is not the case (so much at least)
%  with \cs{displaybreak}. But of course, you'll never see either
%  message because you read the manual!
%
%
%
%  \subsection{Special delimiters}
%
%  As you've seen a few pages back, it's possible to add material on
%  both sides of the math display. When doing so you often need a
%  delimiter that scales to fit the entire display, so it comes as no
%  surprise that \pkg{empheq} provides such delimiters.
%  \begin{codesyntax}
%    \cs{empheql}\meta{delim_name}  \cs{empheqbigl}\meta{delim_name}\\
%    \cs{empheqr}\meta{delim_name}  \cs{empheqbigr}\meta{delim_name}
%  \end{codesyntax}
%  For example one might need a large brace like the one in the
%  \env{cases} environment:
%  \begin{verbatim}
%  \begin{empheq}[left=\empheqlbrace, right=\empheqrbrace]{align}
%    E&=mc^2 \\
%    Y&= \sum_{n=1}^\infty \frac{1}{n^2}
%  \end{empheq}
%  \end{verbatim}
%  \begin{empheq}[left=\empheqlbrace, right=\empheqrbrace]{align}
%    E&=mc^2 \\
%    Y&= \sum_{n=1}^\infty \frac{1}{n^2}
%  \end{empheq}
%  The naming scheme is \cs{empheq}\meta{delimiter name} and
%  \cs{empheqbig}\meta{delimiter name}. Thus \cs{empheqrbrace}
%  produces an auto-scaling right brace with same size as the math
%  display while \cs{empheqbigrbrace} produces an even bigger version
%  that spans the inner box plus the math display inside it. For a
%  complete list of supported delimiters see Table
%  \vref{tab:delimiters}.
%
%  \begin{table}
%  \centering
%  \begin{tabular}{lllc} \toprule
%  \multicolumn{3}{c}{\textbf{Delimiter}}\\ \cmidrule{1-3}
%  \multicolumn{1}{c}{Original} & \multicolumn{1}{c}{Normal}  &
%    \multicolumn{1}{c}{Bigger} & Symbol\\ \midrule
%  \DescribeDelimiter{lbrace} \DescribeDelimiter{rbrace}
%  \DescribeDelimiter{lbrack} \DescribeDelimiter{rbrack} \midrule
%  \DescribeDelimiter{langle} \DescribeDelimiter{rangle}
%  \DescribeDelimiter{lparen} \DescribeDelimiter{rparen} \midrule
%  \DescribeDelimiter{lvert} \DescribeDelimiter{rvert}
%  \DescribeDelimiter{lVert} \DescribeDelimiter{rVert} \midrule
%  \DescribeDelimiter{lfloor} \DescribeDelimiter{rfloor}
%  \DescribeDelimiter{lceil} \DescribeDelimiter{rceil} \bottomrule
%  \end{tabular}
%  \caption{The supported auto-scaling delimiters in
%  \pkg{empheq}}\label{tab:delimiters}
%  \end{table}
%
%
%
%  \section{Taking things a little further}
%
%  So far we have covered the basic functions of \pkg{empheq}: the
%  markup instructions and the predefined auto-scaling delimiters.
%  There is more to \pkg{empheq} however, and we'll cover that in
%  this section.
%
%  \subsection{Loading the package}
%
%  The package has two main requirements: One is the
%  \pkg{mathtools}\footnote{By the same author and is distributed
%  with \pkg{empheq}. See its documentation for more information.}
%  package which provides \pkg{empheq} with a lot of necessary tools
%  for doing its thing. The other one is \pkg{amsmath} of course, and
%  to make the loading procedure as easy as possible, you can simply
%  substitute
%  \begin{verbatim}
%  \usepackage[leqno,fleqn,intlimits]{amsmath}
%  \end{verbatim}
%  with
%  \begin{verbatim}
%  \usepackage[leqno,fleqn,intlimits]{empheq}
%  \end{verbatim}
%  \pkg{empheq} makes sure that the \pkg{amsmath} options are passed
%  on and loaded by \pkg{amsmath}. The same goes for the options
%  provided by \pkg{mathtools}.
%
%
%  \subsubsection{Compatibility with the old version}
%
%  The current version (\fileversion) of \pkg{empheq} is incompatible with
%  versions prior to and including~0.7e. If you have documents produced
%  with versions prior to~0.7e, then you have to load the package
%  \pkg{empheq07} instead. \pkg{empheq07} now exists as a separate package
%  package with its own documentation, but I strongly recommend switching
%  to the new version because it is so much better.
%
%  \subsection{Package options}
%
%  In addition to the \pkg{amsmath} options, \pkg{empheq} itself
%  provides a string of options listed in Table \vref{tab:options}.
%
%  \begin{table}[ht]
%    \centering
%    \begin{tabular}{lp{21em}} \toprule
%      Option  & Short description \tabularnewline \midrule
%      \TabDescribeOption[]{overload}
%      {Lets you use the \pkg{empheq} visual markup extensions
%      transparently in an existing document.}
%      \TabDescribeOption[]{overload2}
%      {A wild version of \opt{overload}. Use with care.}
%      \TabDescribeOption[]{ntheorem}
%      {A support module for users of \pkg{ntheorem} and its
%      \opt{thmmarks} and \opt{thref} options.}
%      \TabDescribeOption[]{newmultline}
%      {With v2.10 of \pkg{empheq} the syntax for \env{multline} and
%      \env{multlined} has been improved immensely.}
%      \TabDescribeOption[]{oldmultline}
%      {Support for the somewhat strange syntax for \env{multline} and
%      \env{multlined} in v2.00.}
%      \bottomrule
%    \end{tabular}
%    \caption{Supported options in the \pkg{empheq} package}
%    \label{tab:options}
%  \end{table}
%
%
%  \begin{codesyntax}
%    \SpecialOptIndex{ntheorem}\opt{ntheorem}
%  \end{codesyntax}
%  The \pkg{ntheorem} package is supported by means of the
%  \opt{ntheorem} option. In order for this to work properly, an
%  \pkg{amsmath} bug\footnote{See the \LaTeX{} Bugs Database
%  \url{http://www.latex-project.org/cgi-bin/ltxbugs2html} under
%  \AmS\LaTeX{} problem report 3624.} is fixed. For more information
%  on this option see \S \vref{sec:ntheorem-support}.
%
%  \subsubsection{Using \env{multline}}
%
%  \begin{codesyntax}
%    \SpecialOptIndex{newmultline}\opt{newmultline}\\
%    \SpecialOptIndex{oldmultline}\opt{oldmultline}
%  \end{codesyntax}
%  Those familiar with v2.00 of this package will recognize the
%  somewhat weird syntax for using \env{multline} and
%  \env{multlined}. The good news is that since v2.10 there is now a
%  much improved syntax available. The option \opt{newmultline}
%  (default) selects this better interface, while the old interface
%  is provided for compatibility reasons by the \opt{oldmultline}
%  option.
%
%
%  \begin{codesyntax}
%    \SpecialUsageIndex{\shoveleft}\cs{shoveleft}\oarg{dimen}\marg{arg}\\
%    \SpecialUsageIndex{\shoveright}\cs{shoveright}\oarg{dimen}\marg{arg}
%  \end{codesyntax}
%  With the new interface you also get an extended syntax for
%  \cs{shoveleft} and \cs{shoveright} as shown in the example below.
%  \begin{verbatim}
%  \begin{empheq}{multline}
%    \framebox[.65\columnwidth]{First line} \\
%    \framebox[.5\columnwidth]{Second line} \\
%    \shoveleft{L+E+F+T}         \\
%    \shoveright{R+I+G+H+T}         \\
%    \shoveleft[1cm]{L+E+F+T}         \\
%    \shoveright[\widthof{$R+I+G+H+T$}]{R+I+G+H+T}         \\
%    \framebox[.65\columnwidth]{Last line}
%  \end{empheq}
%  \end{verbatim}
%  \begin{empheq}{multline}
%    \framebox[.65\columnwidth]{First line} \\
%    \framebox[.5\columnwidth]{Second line} \\
%    \shoveleft{L+E+F+T}         \\
%    \shoveright{R+I+G+H+T}         \\
%    \shoveleft[1cm]{L+E+F+T}         \\
%    \shoveright[\widthof{$R+I+G+H+T$}]{R+I+G+H+T}         \\
%    \framebox[.65\columnwidth]{Last line}
%  \end{empheq}
%
%
%  There are however a few differences in the output between the
%  original \pkg{amsmath} version of \env{multline} and the one
%  \pkg{empheq} provides. In \pkg{amsmath} a centered line in
%  \env{multline} is centered on the page without taking into account
%  the \cs{multlinegap}, \cs{multlinetaggap}, and the tag width. Thus
%  \pkg{amsmath} can sometimes give you horrible output without
%  giving you any warning as shown below.
%  \begin{verbatim}
%  \begin{multline}
%    \framebox[.65\columnwidth]{First line} \\
%    \framebox[.9\columnwidth]{Loooong line} \\
%    \framebox[.65\columnwidth]{Last line} \tag{wide tag}
%  \end{multline}
%  \end{verbatim}
%  \begin{AmSmultline}
%    \framebox[.65\columnwidth]{First line} \\
%    \framebox[.9\columnwidth]{Loooong line} \\
%    \framebox[.65\columnwidth]{Last line} \tag{wide tag}
%  \end{AmSmultline}
%  In \pkg{empheq} these parameters are taken into account, so the
%  same input will produce this in \pkg{empheq}:
%  \begin{verbatim}
%  \begin{empheq}{multline}
%    \framebox[.65\columnwidth]{First line} \\
%    \framebox[.9\columnwidth]{Loooong line} \\
%    \framebox[.65\columnwidth]{Last line} \tag{wide tag}
%  \end{empheq}
%  \end{verbatim}
%  \begin{multline*}
%    \framebox[.65\columnwidth]{First line} \\
%    \framebox[.9\columnwidth]{Loooong line} \\
%    \framebox[.65\columnwidth]{Last line} \tag{wide tag}
%  \end{multline*}
%  This results in an
%  \begin{verbatim}
%    Overfull \hbox (30.03783pt too wide) in paragraph ...
%  \end{verbatim}
%  message in the log file, indicating a visual problem. I think this
%  behavior is more sensible than the one the original \env{multline}
%  environment provides.
%
%  \begin{codesyntax}
%    \SpecialEnvIndex{MTmultlined}\cs{begin}\arg{MTmultlined}\oarg{pos}\oarg{width}
%        \meta{contents} \cs{end}\arg{MTmultlined}
%  \end{codesyntax}
%  When you choose the \opt{oldmultline} option you may still want to
%  use the much improved \env{multlined} environment defined in
%  \pkg{mathtools} but unfortunately there will be a name clash.
%  Instead you can access it under the name \env{MTmultlined}.
%
%
%
%  \subsubsection{The \opt{overload} option}\label{sec:overload}
%
%  \begin{codesyntax}
%    \SpecialOptIndex{overload}\opt{overload}\\
%    \SpecialOptIndex{overload2}\opt{overload2}
%  \end{codesyntax}
%  \FeatureRequest{Lars Madsen}{2004/03/25}
%  The \opt{overload} option redefines the original \AmS{}
%  environments so that they take an optional argument.
%  \begin{codesyntax}
%    \cs{begin}\marg{\AmS{} env_name}\oarg{markup instructions}\\
%    \meta{contents of \AmS{} environment}\\
%    \cs{end}\marg{\AmS{} env_name}
%  \end{codesyntax}
%  For example
%  \begin{verbatim}
%    \begin{gather}[box=\widefbox]
%      a = b
%    \end{gather}
%  \end{verbatim}
%  is then actually short for
%  \begin{verbatim}
%    \begin{empheq}[box=\widefbox]{gather}
%      a = b
%    \end{empheq}
%  \end{verbatim}
%  All the \AmS{} environments are supported by this option except
%  for the pseudo environment |\[ ... \]| (it's not really an
%  environment), \footnote{With the option \opt{oldmultline} the
%  \env{multline} and \env{multline*} environments aren't supported
%  either because their syntax in \pkg{empheq} then differ from their
%  syntax in \pkg{amsmath}.} because a)~you don't really need markup
%  instructions for a one line display, and b)~would you really like
%  a syntax like
%  \begin{verbatim}
%    \[[box=\fbox] a=b \]
%  \end{verbatim}
%  where its difficult to see whether or not there is a typo?
%  Choosing \opt{overload} is meant for people who don't want to
%  change their entire document into \pkg{empheq} syntax. I have no
%  problems with that; just be careful when you fiddle with
%  \cs{empheqset} as it will affect \emph{all} math displays!
%
%
%  There is of course a catch (well, two actually): These redefined
%  environments don't run as fast as the regular ones (about three
%  times as slow), but in this day and age I seriously doubt you'll
%  be able to tell the difference unless you have a vast number of
%  equations. The other catch is that you cannot use \cs{intertext}
%  and \cs{displaybreak} as described earlier. If you find yourself
%  wanting to use one of these features in say, an \env{align}
%  environment, you have to use the original \env{align} environment.
%  Luckily it is available if you call it like this:
%  \begin{codesyntax}
%    \cs{begin}\arg{AmS\meta{\rmfamily \AmS{} env_name}} \\
%    \texttt{~~}\meta{contents of \AmS{} environment}\\
%    \cs{end}\arg{AmS\meta{\rmfamily \AmS{} env_name}}
%  \end{codesyntax}
%  For example the original \env{align} environment could be selected
%  with
%  \begin{verbatim}
%    \begin{AmSalign}
%      ...
%    \end{AmSalign}
%  \end{verbatim}
%  These original versions with prefix \texttt{AmS} exist for all the
%  environments.
%
%
%  The option \opt{overload2} activates the overloading feature
%  for the pseudo environment |\[ ... \]|, although I doubt you'll
%  find it useful. Beware that this definition is fragile unless you
%  have \eTeX{} as \LaTeX{} engine.\footnote{\eTeX{} has been the
%  default engine for \LaTeX{} in most major distributions since
%  2003.} Not surprisingly \opt{overload2} activates
%  \opt{overload}.
%
%
%
%  Before you get all excited about these options, you should take
%  note of some aspects of centered math displays in \pkg{amsmath}.
%  In certain circumstances truly centering the display is not always
%  the best solution as in this example:
%  \begin{center}
%    \mbox{}\hfill
%    \framebox[.75\linewidth]{Wide math display; not adjusted}%
%    \hfill\mbox{}%
%    \makebox[0pt][r]{Wide Tag}
%  \end{center}
%  Instead centering the display in the available space works pretty
%  well:
%  \begin{gather*}
%    \framebox[.75\linewidth]{Wide math display; adjusted}%
%    \tag*{Wide Tag}
%  \end{gather*}
%  This is what \pkg{amsmath} does normally, but in the \env{gather}
%  environment it does it on a per line basis. This means that we get
%  results like
%  \begin{AmSgather*}
%    \framebox[.75\linewidth]{Wide math display; adjusted}%
%    \tag*{Wide Tag} \\
%    \framebox[.75\linewidth]{Wide math display; not adjusted} \tag{i}%
%  \end{AmSgather*}
%  whether we like it or not. Inside the \env{empheq} environment we
%  need to have the same adjustment for all lines else the boxing
%  process will not work properly, so when activating the
%  \opt{overload} option the above two-line \env{gather} will
%  instead look like this:
%  \begin{empheq}{gather*}
%    \framebox[.75\linewidth]{Wide math display; adjusted}%
%    \tag*{Wide Tag} \\
%    \framebox[.75\linewidth]{Wide math display; adjusted} \tag*{i}%
%  \end{empheq}
%  I leave it to you to choose whether or not this is better (I think
%  it is better).
%
%
%
%  \subsection{A note on boxed displays}
%
%  When browsing a 400+ pages textbook with at least twice the number
%  of displayed formulae, some of them are surely more important than
%  others. Thus the author of the book (in cooperation with the
%  designer) should make sure that such formulae are easily found
%  again i.e., they should be easily distinguishable from the rest of
%  the pack. One way of doing this is putting the formula into a box
%  which is something we've seen \pkg{empheq} being capable of. There
%  are however a few things to keep in mind:
%  \begin{itemize}
%    \item Don't overdo it. If you do it on half of them there's no
%    point in doing it at all. I don't see much reason for applying
%    this technique to more than 10\% of the formulae.
%    \item Choose the type of box carefully. You want to draw
%    attention to it so it might as well look good.
%  \end{itemize}
%  The latter point can be illustrated by defining a macro similar to
%  \cs{fbox} only with a little more space to the right and left of
%  the argument.
%  \begin{verbatim}
%  \newcommand*\widefbox[1]{\fbox{\hspace{1em}#1\hspace{1em}}}
%  \end{verbatim}
%  If we replace the \cs{fbox} from the \key{box} example before, we
%  get this display:
%  \begin{verbatim}
%  \begin{empheq}[box=\widefbox]{align}
%    a&=b\\
%    E&=mc^2 + \int_a^a x\, dx
%  \end{empheq}
%  \end{verbatim}
%  \begin{empheq}[box=\widefbox]{align}
%    a&=b\\
%    E&=mc^2 + \int_a^a x\, dx
%  \end{empheq}
%  Compare it with
%  \begin{empheq}[box=\fbox]{align}
%    a&=b\\
%    E&=mc^2 + \int_a^a x\, dx
%  \end{empheq}
%  and see which one you prefer.
%
%  Similarly one might be tempted to use colored boxes:
%  \begin{verbatim}
%  \definecolor{myblue}{rgb}{.8, .8, 1}
%  \newcommand*\mybluebox[1]{%
%    \colorbox{myblue}{\hspace{1em}#1\hspace{1em}}}
%  \end{verbatim}
%  We know the drill by now.
%  \begin{verbatim}
%  \begin{empheq}[box=\mybluebox]{align}
%    a&=b\\
%    E&=mc^2 + \int_a^a x\, dx
%  \end{empheq}
%  \end{verbatim}
%  \begin{empheq}[box=\mybluebox]{align}
%    a&=b\\
%    E&=mc^2 + \int_a^a x\, dx
%  \end{empheq}
%  There is more on boxes later in this manual. If you're into the
%  \pkg{fancybox} package then remember to read \S
%  \vref{sec:fancybox_support}.
%
%  \section{Support for other packages}
%
%  With the multitude of packages for \LaTeX, it is not always easy
%  to be a PWWO package (``Plays Well With Others''), but
%  \pkg{empheq} tries really hard to do so. This section lists the
%  packages where \pkg{empheq} has to provide workarounds and they
%  can be divided into two categories.
%  \begin{description}
%    \item[Compatibility] Some packages affect \LaTeX' labelling
%    mechanism and since \pkg{empheq} internally has to turn off
%    labels and related code temporarily, hooks must be provided for
%    these packages. Examples of such packages are \pkg{hyperref} and
%    \pkg{showkeys}.
%
%    \item[Enhancements] Other packages provide useful features that
%    for some reason may not work directly or optimally with
%    \pkg{empheq}. In these cases the problematic commands are
%    redefined so that they not only work with \pkg{empheq}, but
%    also give the same if not better output. An example of this
%    would be the \cs{shadowbox} command from \pkg{fancybox}
%    (described below).
%  \end{description}
%  %
%  \subsection{Support for \pkg{fancybox}}\label{sec:fancybox_support}
%
%  The \pkg{fancybox} package provides various boxes and you can use
%  them with \pkg{empheq} as well. Here's \cs{ovalbox}:
%  \begin{verbatim}
%  \begin{empheq}[box=\ovalbox]{align}
%      E&=mc^2 \\
%      Y&= \sum_{n=1}^\infty \frac{1}{n^2}
%  \end{empheq}
%  \end{verbatim}
%  \begin{empheq}[box=\ovalbox]{align}
%      E&=mc^2 \\
%      Y&= \sum_{n=1}^\infty \frac{1}{n^2}
%  \end{empheq}
%
%
%  \begin{codesyntax}
%  \SpecialUsageIndex{\shadowbox}\SpecialUsageIndex{\shadowbox*}
%    \cs{shadowbox*}\marg{arg} \\
%  \SpecialColorIndex{shadowcolor} \texttt{shadowcolor}
%  \end{codesyntax}
%  The only problem with using the \pkg{fancybox} boxes in
%  conjunction with the \env{empheq} environment is \cs{shadowbox},
%  as this macro typesets the shadow on the baseline and for the
%  equation numbers to be placed correctly, the box command must set
%  its argument on the baseline. Thus the normal \cs{shadowbox}
%  produces \shadowbox{\meta{arg}}, but we need this instead:
%  \shadowbox*{\meta{arg}}. Therefore \pkg{empheq} will detect if
%  \pkg{fancybox} is loaded and in that case it'll enhance
%  \cs{shadowbox} in two ways:
%  \begin{itemize}
%    \item It defines a starred version \cs{shadowbox*} which
%    typesets its argument on the baseline.
%
%    \item The color \texttt{shadowcolor} is introduced. The default
%    color is black.
%  \end{itemize}
%  \begin{verbatim}
%  \definecolor{shadowcolor}{rgb}{0,.5,.5}
%  \setlength\shadowsize{2pt}
%  Line of text for \shadowbox*{testing of \Verb|f#&%\| verbatim} and
%  showing the shadow color.
%  \end{verbatim}
%  \begin{quote}
%  \definecolor{shadowcolor}{rgb}{0,.5,.5}
%  \setlength\shadowsize{2pt}
%  Line of text for \shadowbox*{testing of \Verb|f#&%\| verbatim} and
%  showing the shadow color.
%  \end{quote}
%
%  The point is that if you want a \cs{shadowbox} around your math
%  display then you must use the starred version:
%  \begin{verbatim}
%  \begin{empheq}[box=\shadowbox*]{align}
%      E&=mc^2 \\
%      Y&= \sum_{n=1}^\infty \frac{1}{n^2}
%  \end{empheq}
%  \end{verbatim}
%  \begin{empheq}[box=\shadowbox*]{align}
%      E&=mc^2 \\
%      Y&= \sum_{n=1}^\infty \frac{1}{n^2}
%  \end{empheq}
%
%  See \S \vref{stunning} if you want
%  to make a fancy box yourself.
%
%
%  \subsection{Support for \pkg{ntheorem}}\label{sec:ntheorem-support}
%
% \changes{v2.13}{2012/05/09}{Added caveat}
%  \textbf{Caveat:} Due to an unfortunate interaction between \pkg{empheq} and
%  \pkg{ntheorem}, users may want to add
% \begin{verbatim}
% \usetagform{default}
% \end{verbatim}
%  \emph{after} loading \pkg{ntheorem} otherwise the tags may be
%  placed wrong and any labels within the \env{empheq} environment may
%  be lost. In the future \pkg{ntheorem} should be able to test for
%  this and add \verb|\usetagform{default}| automatically.
%
%  Users who use \verb|\usetagform| to get another tag design, should
%  make sure to postpone this configuration until \emph{after} loading
%  \pkg{ntheorem}. 
%
% \bigskip
%
%  \changes{v2.00}{2004/04/14}{Added description of the
%  \opt{ntheorem} option}
%  \changes{v1.05}{2004/02/16}{Added support for the end-of-theorem
%  marker mechanism of \pkg{ntheorem}}
%
%  \begin{codesyntax}
%    \SpecialOptIndex{ntheorem}\opt{ntheorem}
%  \end{codesyntax}
%  The \pkg{ntheorem} package is supported by the means of the
%  \opt{ntheorem} option of \pkg{empheq}. This loads a set of
%  extra macros which fixes various compatibility problems between
%  \pkg{ntheorem} and \pkg{amsmath} and furthermore introduces
%  special (internal) macros for optimum positioning of
%  end-of-theorem markers, while retaining a user friendly interface.
%
%
%  When you want to use the automatic end-of-theorem marker mechanism
%  from \pkg{ntheorem} you sometimes run into problems as you would
%  want the marker to be placed aligned at the bottom of the math
%  display but still keeping the tags in their proper place. In
%  \mode{leqno} mode this is not that much of a problem as the tags
%  and the marker are set on either side of the math display like
%  in Figure \vref{fig:markerI}.
%  \changes{v1.05a}{2004/02/17}{Added explanatory drawings for
%  \pkg{ntheorem}}
%
%  \begin{figure}[btp]\centering
%  \unitlength2.5pt \thicklines
%  \begin{picture}(90,30)(-30,40)
%    \put(0,50){\framebox(25,20){Math Height}}
%    \put(0,40){\framebox(25,10){Math Depth}}
%    \put(-30,50){\framebox(20,10){Tag height}}
%    \put(-30,45){\framebox(20,5){Tag depth}}
%    \put(40,40){\framebox(20,10){Marker}}
%    \multiput(26,40)(2,0){7}{\line(1,0){1}}
%  \end{picture}
%  \caption{Marker placement in \mode{leqno} mode}
%  \label{fig:markerI}
%  \end{figure}
%
%  \begin{codesyntax}
%    \SpecialUsageIndex{\mintagvsep}\cs{mintagvsep}
%  \end{codesyntax}
%  Unfortunately things are not this easy in \mode{reqno} mode. There
%  are two possible situations as shown in Figure \vref{fig:markerII}.
%  \begin{figure}[btp]
%  \unitlength2.5pt \thicklines
%  \begin{picture}(60,45)(0,25)
%    \put(0,50){\framebox(25,20){Math Height}}
%    \put(0,25){\framebox(25,25){Math Depth}}
%    \put(40,50){\framebox(20,10){Tag height}}
%    \put(40,45){\framebox(20,5){Tag depth}}
%    \put(40,25){\framebox(20,10){Marker}}
%    \put(50,35){\vector(0,1){10}}
%    \put(50,45){\vector(0,-1){10}}
%    \put(50,39){\ \ $>\delta$}
%    \multiput(26,25)(2,0){7}{\line(1,0){1}}
%  \end{picture} \hfill\vrule\hfill
%  \begin{picture}(60,45)(0,25)
%    \put(0,50){\framebox(25,20){Math Height}}
%    \put(0,40){\framebox(25,10){Math Depth}}
%    \put(40,50){\framebox(20,10){Tag height}}
%    \put(40,45){\framebox(20,5){Tag depth}}
%    \put(40,28){\framebox(20,10){Marker}}
%    \put(50,38){\vector(0,1){7}}
%    \put(50,45){\vector(0,-1){7}}
%    \put(50,40){\ \ $=\delta$}
%  \end{picture}
%  \caption{The two possible situations for end marks in \mode{reqno}
%  mode}
%  \label{fig:markerII}
%  \end{figure}
%  If possible we want the bottom of the marker to be aligned with
%  the bottom of the math display, but at the same time we want to
%  ensure a minimum vertical separation $\delta$ between marker and
%  tag. In \pkg{empheq} this is controlled by the length parameter
%  \cs{mintagvsep} which by default is 5\,pt.
%
%  The good news is that this is where \pkg{empheq} sets in.
%  Basically all you need to do is to use the \env{empheq}
%  environment to type set your equations inside the theorem
%  environments:
%  \begin{verbatim}
%    \begin{Theorem}
%    Some text at first and then a math display
%    \begin{empheq}{align}
%      a&=b\\
%      E&=mc^2 + \int_a^a x\, dx
%    \end{empheq}
%    \end{Theorem}
%  \end{verbatim}
%  Then the tag placement and the end-of-theorem marker will be set
%  properly (after a couple of runs as usual).
%
%  To ensure the correct outcome remember to load the packages like
%  this:
%  \begin{verbatim}
%    \usepackage[ntheorem]{empheq} % this loads amsmath as well
%    \usepackage[thmmarks,amsmath]{ntheorem}
%  \end{verbatim}
%  Remember that if you use the \opt{overload} option you
%  can just use the regular math environments to get the desired
%  result with the end mark. The exception is the |\[ ... \]|
%  environment which will only work if you use the option
%  \opt{overload2}.
%
%  \clearpage
%
%  \EmphEqpart{Advanced user's guide}
%
%  The \pkg{empheq} package has more to offer than you have seen, but
%  I found some of the functionality so dangerous it was best to hide
%  it for a little while. The commands you have encountered so far in
%  this manual all have one thing in common: They have only lower
%  case letters in their names. Now it's time to reveal those that
%  have \emph{mixed-case} names, thus implying that you should take
%  great care when using them.
%
%  \section{Delimiters revisited}
%  Let's go back to delimiters, shall we?
%
%  \subsection{Creating your own delimiters}
%  As a convenience for the user, this interface is extended in a
%  more general way so that it is also possible to declare delimiters
%  with the following two commands: \SpecialUsageIndex{\DeclareLeftDelimiter}
%  \SpecialUsageIndex{\DeclareRightDelimiter}
%  \begin{codesyntax}
%    \cs{DeclareLeftDelimiter}\oarg{space adjustment}\marg{delimiter}\\
%    \cs{DeclareRightDelimiter}\oarg{space adjustment}\marg{delimiter}
%  \end{codesyntax}
%  While \pkg{empheq} provides auto-scaling versions of the most
%  common delimiters, you may sometimes want some new ones. Say for
%  instance you have loaded the \pkg{stmaryrd} package and you want
%  to use the double bracket commands \cs{llbracket} and
%  \cs{rrbracket} with \pkg{empheq}. Then you simply do this:
%  \begin{verbatim}
%    \DeclareLeftDelimiter{\llbracket}
%    \DeclareRightDelimiter{\rrbracket}
%  \end{verbatim}
%  This defines the new delimiters \cs{empheqllbracket},
%  \cs{empheqbigllbracket}, \cs{empheqrrbracket} and
%  \cs{empheqbigrrbracket}.
%
%  You can use \cs{big...} delimiters as well if you don't like the
%  automatic scaling. There are however ways to fine-tune if you
%  really want it.
%
%  \subsection{Fine-tuning of delimiters}
%
%  \TeX{} provides two primitives to control the scaling of
%  delimiters produced with \cs{left} and \cs{right}, namely the
%  dimension \cs{delimitershortfall} (denoted by $\delta$ here) and
%  the integer \cs{delimiterfactor} ($f$). The idea is that the
%  sub-formula inside the \cs{left}-\cs{right} pair is to be
%  vertically centered and given its height $h_1$ and its depth $h_2$
%  we want to produce a delimiter with total height $h$, where  $h =
%  2 \max(h_1,h_2)$. \TeX' rules on this is that the minimum
%  delimiter size $h_{\min}$ must meet the requirements
%  \[
%    h_{\min}\geq h \frac{f}{1000} \quad\land \quad h_{\min}\geq h-\delta
%  \]
%  \LaTeX{} sets $\cs{delimitershortfall} = \text{\the\delimitershortfall} $
%  and $\cs{delimiterfactor} = \the\delimiterfactor$, but in our case
%  we will almost always want a delimiter that spans the entire
%  \cs{left}-\cs{right} pair (the math display), thus a change of
%  these settings is needed.
%
%  \begin{codesyntax}
%    \SpecialUsageIndex{\EmphEqdelimitershortfall}
%    \cs{EmphEqdelimitershortfall}\\
%    \SpecialUsageIndex{\EmphEqdelimiterfactor}
%    \cs{EmphEqdelimiterfactor}
%  \end{codesyntax}
%  However it is a bad idea to just change these two settings without
%  thinking of the effect it'll have on the rest of the mathematics
%  in your document. Therefore \pkg{empheq} provides the parameters
%  \cs{EmphEqdelimitershortfall} (default setting is
%  \the\EmphEqdelimitershortfall) and \cs{EmphEqdelimiterfactor}
%  (default is \the\EmphEqdelimiterfactor) to cater for this.
%
%
%  \subsection{Scaling material yourself}
%
%  The attentive reader may have noticed that I still haven't
%  revealed how I managed to get the right size of that big
%  exclamation mark in the first example of this manual, so I guess
%  it's about time.
%  \begin{codesyntax}
%    \SpecialUsageIndex{\EmphEqdisplayheight}\cs{EmphEqdisplayheight} \\
%    \SpecialUsageIndex{\EmphEqdisplaydepth}\cs{EmphEqdisplaydepth}
%  \end{codesyntax}
%  The height of the math display plus the surrounding inner box is
%  given by \cs{EmphEqdisplayheight} and the depth by
%  \cs{EmphEqdisplaydepth}. These can be used when constructing
%  something that should scale to fit the display and can't be done
%  with ``auto-scaling'' commands like \cs{vrule} or \cs{left} or
%  \cs{right}. There is a catch however: If you do any horizontal
%  resizing of the material you want to scale, then you must specify
%  the width \emph{manually}:\footnote{The
%  \cs{fontdimen22}\cs{textfont2} in the \meta{length} argument of
%  \cs{raisebox} is the \emph{math axis} of the font. It is needed in
%  order to get the vertical positioning of the oversized exclamation
%  mark just right.}
%  \begin{verbatim}
%  \begin{empheq}[
%    box=\colorbox{myblue},
%    right={\;
%           \makebox[.9em]{%
%            $\raisebox{-.5\totalheight+\fontdimen22\textfont2}
%              {\resizebox{!}{%
%                \EmphEqdisplayheight+\EmphEqdisplaydepth}{!}}$%
%           }},
%    left={X=Y\Rightarrow}
%     ]{alignat=3}
%    A_1&=b_1 & \qquad c_1&=d_1 & \qquad e_1&= f_1
%         \tag*{\_A\raisebox{1ex}{silly}\raisebox{-1ex}{tag}\_}\\
%    A_2&=b_2 & \qquad c_2&=d_2 & \qquad  e_2&= f_2 \\
%    A_3&=b_3 & \qquad c_3&=d_3 & \qquad  e_3&= f_3
%  \end{empheq}
%  \end{verbatim}
%
%  Here's another example. We want to be able to put a \cs{parbox}
%  with some descriptive text top-aligned on the side of the display.
%  \begin{verbatim}
%  \begin{empheq}[
%    left={\parbox[c][\EmphEqdisplayheight+\EmphEqdisplaydepth][t]
%           {4.5cm}
%           {You may find this kind of description useful.}\enspace}]
%      {align}
%    a&=\int_0^1 x\,dx +\frac{foo + bar}{baz}\\
%    E&= mc^2
%  \end{empheq}
%  \end{verbatim}
%  \begin{empheq}[
%    left={\parbox[c][\EmphEqdisplayheight+\EmphEqdisplaydepth][t]
%           {4.5cm}
%           {You may find this kind of description useful.}\enspace}]
%      {align}
%    a&=\int_0^1 x\,dx +\frac{foo + bar}{baz}\\
%    E&= mc^2
%  \end{empheq}
%
%
%  \section{A few short notes}
%
%  \subsection{About $\cs{eqref}$}
%
%  Internally \env{empheq} separate the displayed math from
%  the corresponding equation numbers such that we can add special
%  delimiters or boxes. This is done by nullifying (or rather
%  conveniently redefining) an internal command called
%  \cs{maketag@@@}. Unfortunately, this component is also used by
%  \cs{eqref} to typeset a reference to an equation number, so in
%  earlier versions, \cs{eqref} would not work inside an \env{empheq}
%  environment.  
%
% In the current version this should now be working, but if you ever
% need to mess with \cs{eqref} it may be handy to know what is being
% done. In essence we do something similar to this:
% \begin{verbatim}
%  \let\empheqeqrefbase\textup
%   ... % next go inside empheq env
%  \renewcommand\eqref[1]{\empheqeqrefbase{%
%       \let\maketag@@@\EQsavedmaketag%
%       \tagform@{\ref{##1}}}}
% \end{verbatim}
% Thus if you need to alter things inside \cs{eqref} and need that to
% work within \env{empheq} as well, you may get away with redefining
% \cs{empheqeqrefbase}.
%
%
%
%  \subsection{About changes to $\cs{baselineskip}$}
%
%  Users should never mess with \cs{baselineskip} directly, it is not
%  the correct manner to get double spacing. Have a look at say the
%  \pkg{setspace} package or similar, or play with
%  \cs{baselinestretch} followed by \cs{normalsize} to initiate. 
%
%  Explanation: We use \cs{parbox} inside to place the display box and
%  the box containing the eqn numbers. To do its stuff \cs{parbox}
%  resets some internal settings, in this case, \cs{baselineskip} is
%  reset to \cs{normalbaselineskip}. So if you don't remember to reset
%  that one as well\dots
%
%
%  \section{Creating something new}
%
%  You can create your own environments and boxes to go with
%  \pkg{empheq}, but there are certain things that must be fulfilled
%  to get it to work properly.
%
%  \subsection{New \env{empheq}-like environments}
%
%  \begin{codesyntax}
%    \SpecialEnvIndex{EmphEqMainEnv}\env{EmphEqMainEnv}\\
%    \SpecialUsageIndex{\EmphEqMainEnv}\cs{EmphEqMainEnv}\\
%    \SpecialUsageIndex{\endEmphEqMainEnv}\cs{endEmphEqMainEnv}
%  \end{codesyntax}
%  Although the real work in \pkg{empheq} is done by an environment
%  (\env{EmphEqMainEnv}), you cannot use it like that yourself when
%  you want to define an \env{empheq}-like environment. Due to
%  technical reasons (see~\cite{Ams99}), you have to use a slightly
%  different syntax:
%  \begin{codesyntax}
%    \cs{newenvironment}
%      \marg{env_name} \oarg{num} \oarg{default}\\
%      \texttt{~~}\arg{\meta{other commands} \cs{EmpheqMainEnv}}\\
%      \texttt{~~}\arg{\cs{endEmpheqMainEnv}}
%  \end{codesyntax}
%  In the above environment the \meta{other commands} part must
%  contain certain things before it'll work.
%  \begin{codesyntax}
%    \SpecialKeyFamilyIndex{EmphEqEnv}\keyfam{EmphEqEnv}\\
%    \SpecialKeyFamilyIndex{EmphEqOpt}\keyfam{EmphEqOpt}
%  \end{codesyntax}
%  Before your homemade environment will work you must set at least
%  one key. The key family \keyfam{EmphEqOpt} is for all the markup
%  instructions like \key{left} and \key{box}, while
%  \keyfam{EmphEqEnv} controls the type of \AmS{} environment. You
%  must set the \keyfam{EmphEqEnv} family before \cs{EmhpEqMainEnv}
%  else you'll get an error message.
%
%  One example could be something like this:
%  \begin{verbatim}
%  \newenvironment{important}[2][]{%
%    \setkeys{EmphEqEnv}{#2}%
%    \setkeys{EmphEqOpt}{box=\mybluebox,#1}%
%    \EmphEqMainEnv}%
%    {\endEmphEqMainEnv}
%  \end{verbatim}
%  Thus
%  \begin{verbatim}
%  \begin{important}{gather}
%    a = b + c +d \\ e = f
%  \end{important}
%  \end{verbatim}
%  produces
%  \begin{important}{gather}
%    a = b + c +d \\ e =  f
%  \end{important}
%  while
%  \begin{verbatim}
%  \begin{important}[left={A=B \Rightarrow \empheqlbrace}]{alignat=2}
%    a &= b &\quad c &= d \\
%    \text{this} &= \text{that} &\quad \mathit{fish}&\neq fish
%  \end{important}
%  \end{verbatim}
%  produces
%  \begin{important}[left={A=B \Rightarrow \empheqlbrace}]{alignat=2}
%    a &= b &\quad c &= d \\
%    \text{this} &= \text{that} &\quad \mathit{fish}&\neq fish
%  \end{important}
%
%
%  \subsection{Creating fancy boxes}\label{stunning}
%
%  As a final example I will show you how to create complicated
%  displays involving (vertically) asymmetrical boxes like
%  \cs{shadowbox*}. In order to get the correct output, the contents
%  of the box must be placed on the baseline as in \fbox{\cs{fbox}}.
%  In this example we want to put a set of equations into a bright
%  yellow box and then add another box with some explanatory text at
%  the top of the yellow box making them overlap.
%
%  First we define the colors used and allocate the boxes. We could
%  probably use scratch boxes, but this is safer.
%  \begin{verbatim}
%  \definecolor{shadecolor}{cmyk}{0,0,0.41,0}
%  \definecolor{light-blue}{cmyk}{0.25,0,0,0}
%  \newsavebox{\mysaveboxM} % M for math
%  \newsavebox{\mysaveboxT} % T for text
%  \end{verbatim}
%  Save the display body in \cs{mysaveboxM} and the text argument in
%  \cs{mysaveboxT}.
%  \begin{verbatim}
%  \newcommand*\Garybox[2][Example]{%
%    \sbox{\mysaveboxM}{#2}%
%    \sbox{\mysaveboxT}{\fcolorbox{black}{light-blue}{#1}}%
%  \end{verbatim}
%  Then typeset the math display in a \cs{parbox} where we control
%  the height and save it in \cs{mysaveboxM}.
%  \begin{verbatim}
%    \sbox{\mysaveboxM}{%
%      \parbox[b][\ht\mysaveboxM+.5\ht\mysaveboxT+.5\dp\mysaveboxT][b]{%
%        \wd\mysaveboxM}{#2}%
%    }%
%  \end{verbatim}
%  We put it into the colored box with the desired width.
%  \begin{verbatim}
%    \sbox{\mysaveboxM}{%
%      \fcolorbox{black}{shadecolor}{%
%        \makebox[\linewidth-10em]{\usebox{\mysaveboxM}}%
%      }%
%    }%
%  \end{verbatim}
%  Then finally we get to the real typesetting. We just insert the
%  math display and then make a box of zero width right next to it.
%  In that box we lift the text argument and center it at the top of
%  the display.
%  \begin{verbatim}
%    \usebox{\mysaveboxM}%
%    \makebox[0pt][r]{%
%      \makebox[\wd\mysaveboxM][c]{%
%        \raisebox{\ht\mysaveboxM-0.5\ht\mysaveboxT
%                  +0.5\dp\mysaveboxT-0.5\fboxrule}{\usebox{\mysaveboxT}}%
%      }%
%    }%
%  }
%  \end{verbatim}
%  Let's see what it looks like.
%  \begin{verbatim}
%  \begin{empheq}[box=\Garybox]{align}
%    \sum \mathbf{F} &= \mathbf{0} \\
%    \sum F_{x} \,\mathbf{i} + \sum F_{y} \,\mathbf{j}
%      + \sum F_{z}  \,\mathbf{k} &= \mathbf{0} \\
%    \sum F_{x} &= 0 \\ \sum F_{y} &= 0 \\ \sum F_{z} &= 0
%  \end{empheq}
%  \end{verbatim}
%  \begin{subequations}
%  \begin{empheq}[box=\Garybox]{align}
%    \sum \mathbf{F} &= \mathbf{0} \\
%    \sum F_{x} \,\mathbf{i} + \sum F_{y} \,\mathbf{j} + \sum F_{z}
%    \,\mathbf{k} &= \mathbf{0}\\
%  \sum F_{x} &= 0 \\ \sum F_{y} &= 0 \\ \sum F_{z} &= 0
%  \end{empheq}
%  If we use the optional argument of \cs{Garybox} we have to enclose
%  the entire argument of \key{box} in braces because we're in
%  \env{empheq}'s optional argument (as described in \cite[page
%  167]{Lamp94}):
%  \begin{verbatim}
%  \begin{empheq}[box={\Garybox[Same old example again]}]{align}
%    a&=b\\
%    E&=mc^2 + \int_a^a x\, dx
%  \end{empheq}
%  \end{verbatim}
%  \begin{empheq}[box={\Garybox[Same old example again]}]{align}
%    a&=b\\
%    E&=mc^2 + \int_a^a x\, dx
%  \end{empheq}
%  \end{subequations}
%
%
%
%
%  \section{Contact information}
%
%  Should you have any feature request, suggestions, or bug reports
%  then feel free to contact Lars Madsen at
%  \begin{quote}
%    \email{daleif@imf.au.dk}
%  \end{quote}
%  Please put `empheq' in the subject line.
%
%  \subsection*{Contributors}
%
%  \begin{itemize}
%    \item In November 2002 Lars Madsen (\email{daleif@imf.au.dk})
%    asked for some features that wasn't readily available with
%    \pkg{amsmath} or any other package. This was the start of
%    \pkg{empheq}.
%
%    \item Gary Gray (\email{gray@engr.psu.edu}) gave me some bug
%    reports on the old version (v0.7) which convinced me to rewrite
%    the package completely. The \cs{Garybox} example is inspired by
%    a wish from Gary.
%
%    \item Uwe Siart (\email{uwe.siart@tum.de}) has been a thorough
%    beta-tester on v1.00 of the package.
%
%    \item Bernard Alfonsi (\email{alfonsi@math.u-psud.fr}) reported
%     a problem with the \opt{ntheorem} option. It turned out to be a
%     bug in the \pkg{ntheorem} package itself but it's fixed in the
%     option.
%
%    \item I have received reports on weird behavior from
%    \pkg{empheq} when used with the \pkg{color} package on some
%    dvi-viewers, which is caused by the lack of color support in
%    those dvi-viewers. Both Andrew B.~Collier (\email{colliera@ukzn.ac.za})
%    and Andr\'e M.~de Roos (\email{aroos@science.uva.nl}) have
%    notified me of this. The solution is to either upgrade you
%    dvi-viewer if possible or convert the document in question to
%    PDF or PS.
%  \end{itemize}
%  Thank you all.
%
%
% \begin{thebibliography}{99}
% \bibitem{Lamp94}
%   Leslie Lamport,
%   \emph{LaTeX User's Guide and Reference Manual}, 2nd edition,
%   Addison-Wesley, Reading, Massachusetts,
%   1994.
% \bibitem{Goos-etal94}
%   Michel Goossens, Frank Mittelbach, and Alexander Samarin,
%   \emph{The \LaTeX\ Companion},
%   Addison-Wesley, Reading, Massachusetts,
%   1994.
% \bibitem{Knut86}
%   Donald Ervin Knuth,
%   \emph{The \TeX book},
%   Addison-Wesley, Reading, Mas\-sa\-chu\-setts,
%   1986.
%  \bibitem{Ams99}
%    American Mathematical Society and Michael Downes,
%    \emph{Technical notes on the \pkg{amsmath} package} Version 2.0,
%    1999/10/29.
%    (Available from CTAN as file \texttt{technote.tex}.)
%  \bibitem{Ams00}
%    Frank Mittelbach, Rainer Sch\"opf, Michael Downes, and David M.~Jones,
%    \emph{The \pkg{amsmath} package} Version 2.13,
%    2000/07/18.
%    (Available from CTAN as file \texttt{amsmath.dtx}.)
%
%  \changes{v1.05b}{2004/02/18}{Added \pkg{fancybox} and
%  \pkg{ntheorem} to the \refname}
%  \bibitem{fancybox}
%    Timothy Van Zandt,
%    \emph{The \pkg{fancybox} package} Version 1.3,
%    2000/09/19
%    (Available from CTAN as file \texttt{fancybox.sty}.)
%
% \bibitem{Raht03}
%   Sebastian Rahtz,
%   \emph{Hypertext marks in \LaTeX} Version v6.74m,
%   2003/11/30.
%   (Available from \url{http://www.tug.org/applications/hyperref} as file
%   \texttt{hyperref.dtx}.)
%
% \bibitem{Carl99}
%   David Carlisle,
%   \emph{The \pkg{keyval} Package},
%   Version 1.13, 1999/03/16.
%   (Available from CTAN as file \texttt{keyval.dtx}.)
%
%  \changes{v1.06}{2004/03/13}{Added \pkg{calc} to the \refname}
%  \bibitem{Thorup:calc:1998}
%  Kresten~Krab Thorup, Frank Jensen, and Chris Rowley.
%  \emph{The {\pkg{calc}} package}, Version 4.1b, 1998/07/07,
%  (Available from CTAN as file \texttt{calc.dtx}.)
%
%  \bibitem{ntheorem}
%    Wolfgang May and Andreas Schlechte,
%    \emph{The \pkg{ntheorem} package} Version 1.203,
%    2002/01/07,
%    (Available from CTAN as file \texttt{ntheorem.dtx}.)
%
%  \changes{v2.10}{2004/07/26}{Added \pkg{showkeys} to the \refname}
%  \bibitem{Carlisle:showkeys:1997}
%  David Carlisle. \emph{The {\pkg{showkeys}} package}, v3.12,
%  1997/06/12, (Available from CTAN as file \texttt{showkeys.dtx}.)
%
%  \end{thebibliography}
%
%  \StopEventually{}
%
%  \EmphEqpart{Implementation}
%
%  \section{Requirements, Options, and Strategy}
%  Lets start the package. We'll need the tools from \pkg{mhsetup}.
%    \begin{macrocode}
%<*package>
\ProvidesPackage{empheq}%
  [2014/07/16 v2.14 Emphasizing equations]
%    \end{macrocode}
%    \begin{macrocode}
\RequirePackage{mhsetup}[2004/10/10]
\MHInternalSyntaxOn
%    \end{macrocode}
%
%
%  \subsection{Requirements, Options, and Allocation}
%
%  The \opt{overload} option allows the user to type
%  \begin{verbatim}
%    \begin{gather}[box=\fbox]
%      a = b
%    \end{gather}
%  \end{verbatim}
%  when they really mean
%  \begin{verbatim}
%    \begin{empheq}[box=\fbox]{gather}
%      a = b
%    \end{empheq}
%  \end{verbatim}
%    \begin{macrocode}
\MH_new_boolean:n {overload_ams}
\DeclareOption{overload}{
  \MH_set_boolean_T:n {overload_ams}
}
\MH_new_boolean:n {overload_amsII}
\DeclareOption{overload2}{
  \MH_set_boolean_T:n {overload_ams}
  \MH_set_boolean_T:n {overload_amsII}
}
%    \end{macrocode}
%  Old crappy \env{multline} interface or the new dazzling?
%    \begin{macrocode}
\MH_new_boolean:n {good_mult}
\DeclareOption{oldmultline}{\MH_set_boolean_F:n {good_mult}}
\DeclareOption{newmultline}{\MH_set_boolean_T:n {good_mult}}
%    \end{macrocode}
%
%
%  Load the \pkg{ntheorem} support file.
%    \begin{macrocode}
\gdef\g@EQ@ntheorem@support@bool{01}
\DeclareOption{ntheorem}{\gdef\g@EQ@ntheorem@support@bool{00}}
%    \end{macrocode}
%  Pass other options to \pkg{mathtools}.
%    \begin{macrocode}
\DeclareOption*{
  \PassOptionsToPackage{\CurrentOption}{mathtools}
}
\ExecuteOptions{newmultline}
\ProcessOptions\relax
%    \end{macrocode}
%  The name of the game is \pkg{amsmath} so we'll need that of
%  course. Since v2.10 that is now taken care of in the
%  \pkg{mathtools} package which'll pass on the options to
%  \pkg{amsmath}.
%    \begin{macrocode}
\MHInternalSyntaxOff
\RequirePackage{mathtools}[2004/10/10]
%    \end{macrocode}
%  Activate the internal syntax.
%    \begin{macrocode}
\MHInternalSyntaxOn
\AtEndOfPackage{\MHInternalSyntaxOff}
%    \end{macrocode}
%  \changes{v2.00}{2004/04/14}{Changed description of naming
%  conventions. Altered \emph{all} private macros; this is not
%  mentioned at each macro as this would just be too troublesome}
%
%  \begin{macro}{\l_EQ_equationtype_int}
%  \begin{macro}{\l_EQ_alignmentmode_int}
%  I'll use the following numbering scheme for the equation types,
%  and store the value in \cs{l_EQ_equationtype_int}:
%  \begin{multicols}{2}
%  \begin{enumerate}\setlength{\itemsep}{0pt}
%    \item align
%    \item align*
%    \item gather
%    \item gather*
%    \item flalign
%    \item flalign*
%    \item alignat
%    \item alignat*
%  \end{enumerate}
%  \end{multicols}
%  As you notice \env{equation} and \env{equation*} are missing. This
%  is because \env{gather} and \env{gather*} work the same way, only
%  they are `nicer'. Cuts down on the amount of code as well. I will
%  add an alias so that a user can use the key \key{equation}, but
%  then \key{gather} will kick in instead behind the scenes. Also the
%  \env{multline} environments are missing here, but again it's
%  because I'm faking it. See \S~\vref{sec:flalign+multline} for
%  further details. The value $0$ isn't used for any of the
%  environments and thus serves as a checkpoint (see
%  page~\pageref{ZeroCheckpoint}).
%
%  Similarly I'll use the counter \cs{l_EQ_alignmentmode_int} to
%  store the standard options for placement of math displays and
%  equation numbering.
%  \begin{multicols}{2}
%  \begin{enumerate}\setlength{\itemsep}{0pt}
%    \item reqno
%    \item leqno
%    \item fleqn, reqno
%    \item fleqn, leqno
%  \end{enumerate}
%  \end{multicols}
%
%    \begin{macrocode}
\newcount\l_EQ_equationtype_int
\newcount\l_EQ_alignmentmode_int
\if@fleqn
  \iftagsleft@
    \l_EQ_alignmentmode_int=4
  \else:
    \l_EQ_alignmentmode_int=\thr@@
  \fi:
\else:
  \iftagsleft@
    \l_EQ_alignmentmode_int=\tw@
  \else:
    \l_EQ_alignmentmode_int=\@ne
  \fi:
\fi:
%    \end{macrocode}
%  \end{macro}
%  \end{macro}
%  \begin{macro}{\l_EQ_totalwidth_dim}
%  \begin{macro}{\l_EQ_displaywidth_dim}
%  \begin{macro}{\l_EQ_temp_linewidth_dim}
%  \begin{macro}{\EmphEqdelimitershortfall}
%  \begin{macro}{\EmphEqdelimiterfactor}
%  \begin{macro}{\g_EQ_toptag_height_dim}
%  \begin{macro}{\g_EQ_bottomtag_depth_dim}
%  \begin{macro}{\g_EQ_toprow_height_dim}
%  \begin{macro}{\g_EQ_bottomrow_depth_dim}
%  \begin{macro}{\g_EQ_widesttag_dim}
%  Some lengths we'll need. Could possibly be scratch registers but we
%  have to make global assignments so that road is a little
%  dangerous.
%    \begin{macrocode}
\newdimen\l_EQ_totalwidth_dim
\newdimen\l_EQ_displaywidth_dim
%    \end{macrocode}
%  Just to make it accessible.
%    \begin{macrocode}
\def\EmphEqdisplaywidth{\l_EQ_displaywidth_dim}
\newdimen\l_EQ_temp_linewidth_dim
\newdimen\EmphEqdelimitershortfall
\newcount\EmphEqdelimiterfactor
\EmphEqdelimiterfactor950
\EmphEqdelimitershortfall\p@
\newdimen\g_EQ_toptag_height_dim
\newdimen\g_EQ_bottomtag_depth_dim
\newdimen\g_EQ_toprow_height_dim
\newdimen\g_EQ_bottomrow_depth_dim
\newdimen\g_EQ_widesttag_dim
\def\EmphEqtagwidth{\g_EQ_widesttag_dim}
%    \end{macrocode}
%  \end{macro}
%  \end{macro}
%  \end{macro}
%  \end{macro}
%  \end{macro}
%  \end{macro}
%  \end{macro}
%  \end{macro}
%  \end{macro}
%  \end{macro}
%
%  \begin{macro}{\EQ_outerbox:}
%  \begin{macro}{\EQ_innerbox:}
%  \begin{macro}{\EQ_left_side:}
%  \begin{macro}{\EQ_right_side:}
%  The two boxes and the material on either side of the math display.
%    \begin{macrocode}
\def\EQ_outerbox:{}
\def\EQ_innerbox:{}
\def\EQ_left_side:{}
\def\EQ_right_side:{}
%    \end{macrocode}
%  \end{macro}
%  \end{macro}
%  \end{macro}
%  \end{macro}
%
% \subsection{Strategy}
%
%  In non-\mode{fleqn} math displays are centered---even when having
%  tags. However this can look rather appalling when either the tag
%  or the math display are very wide like in the following example.
%  \begin{center}
%  \makebox[\linewidth][s]{\mbox{}\hfill \mbox{$\displaystyle
%    a+b+c+d+e+f+g+h+i+j+k+l+m+n+o$} \hfill\mbox{} \llap{(wide tag)}}
%  \end{center}
%  Not exactly pleasing to look at. \pkg{amsmath} takes care to do
%  this adjustment for us, so the user instead obtains the far better
%  \begin{gather*}
%  a+b+c+d+e+f+g+h+i+j+k+l+m+n+o \tag{wide tag}
%  \end{gather*}
%  So \pkg{amsmath} handles this quite well. That's good. It looks
%  like \pkg{amsmath} has an internal threshold that matches the
%  width of the math display against the width of the tags. But what
%  if we have a multiple line display and one of the lines causes an
%  adjustment to kick in while the others don't need it? Then we
%  cannot keep track of the complete display anymore. Furthermore we
%  may want to add material on either side of the math display making
%  the whole thing even more complicated, if not impossible to do
%  without some sort of different approach.
%
%  Luckily we have such an approach. The idea that comes to mind is
%  to see if one can typeset the labels and the math displays
%  \emph{separately}, store them in separate boxes and then put them
%  back together again afterwards. Is this really possible? Yes.
%  Come, I'll show you.
%
%  \subsection{The Main Environment}
%  \begin{macro}{\EQ_mathdisplay_box}
%  \begin{macro}{\EmphEqdisplayheight}
%  \begin{macro}{\EmphEqdisplaydepth}
%  \begin{macro}{\EQ_tag_box}
%  Okay, we'll start by allocating two box registers.
%    \begin{macrocode}
\newsavebox\EQ_mathdisplay_box
\def\EmphEqdisplayheight{\ht\EQ_mathdisplay_box}
\def\EmphEqdisplaydepth{\dp\EQ_mathdisplay_box}
\newsavebox\EQ_tag_box
%    \end{macrocode}
%  \end{macro}
%  \end{macro}
%  \end{macro}
%  \end{macro}
%
%  \begin{macro}{\EQ_complete_mathdisplay:}
%  \begin{macro}{\EQ_getbody:n}
%  \begin{environment}{EmphEqMainEnv}
%  The \env{EmphEqMainEnv} is (as the name implies) the main
%  environment. I decided to go with the environment form because
%  that's what \pkg{amsmath} users are accustomed to. But the code
%  will require the contents of the environment to be handled as an
%  argument to a command, so we use \cs{collect@body} from
%  \pkg{amsmath} itself to collect it.
%    \begin{macrocode}
\def\EQ_complete_mathdisplay:{}
\def\EQ_getbody:n #1{%
  \def\EQ_complete_mathdisplay:{}\def\EQ_mathbody:{#1}}
\newenvironment{EmphEqMainEnv}{\collect@body\EQ_getbody:n}{%
%    \end{macrocode}
%  If the user didn't specify the mandatory argument we will catch it
%  because \cs{l_EQ_equationtype_int} is zero. An error is
%  issued\label{ZeroCheckpoint}.
%    \begin{macrocode}
  \if_num:w \l_EQ_equationtype_int=\z@ %ZeroCheckpoint
    \EQ_upgrade_error:
  \else:
%    \end{macrocode}
%  Those that read the manual get the reward here: The package will
%  actually work!
%    \begin{macrocode}
    \EQ_post_process:
    \ignorespacesafterend
  \fi:}
%    \end{macrocode}
%  \end{environment}
%  \end{macro}
%  \end{macro}
%  \subsection{Post Processing}
%  \begin{macro}{\EQ_post_process:}
%  I know: it almost looked easy for a while didn't it? Sadly the
%  post-processing step of the code is also by far the most
%  complicated.
%    \begin{macrocode}
\def\EQ_post_process:{%
%    \end{macrocode}
%  First a hook where we initialize a few things like storing the
%  line width for later manipulation. Also we want to measure the
%  height of the top row and the depth of the bottom row for best
%  possible placement of the math display and the boxes around it
%  (same with the tags). Those four dimensions are globally set
%  during the code, hence this assignment. Most other stuff is local.
%    \begin{macrocode}
  \EQ_initialize:
%    \end{macrocode}
%  Then we proceed by completing the input. We store it in the
%  control sequence \cs{EQ_complete_mathdisplay}.
%  \cs{EQ_begin_equation:} adds the start of the environment and
%  \cs{EQ_end_equation:} adds the corresponding end.
%    \begin{macrocode}
  \EQ_begin_equation:
%    \end{macrocode}
%  In case we face a \env{multline} we add what's necessary.
%    \begin{macrocode}
  \MH_if_boolean:nT {outer_mult}{
    \setkeys{\MT_options_name:}{
      firstline-afterskip=0pt,
      lastline-preskip=0pt,
      multlined-width=0pt,
    }
    \ifodd \l_EQ_alignmentmode_int
      \setkeys{\MT_options_name:}{multlined-pos=b}
    \else:
      \setkeys{\MT_options_name:}{multlined-pos=t}
    \fi:
      \g@addto@macro\EQ_complete_mathdisplay:{
        \begin{MTmultlined}\relax}
  }
  \g@addto@macro\EQ_complete_mathdisplay:\EQ_mathbody:
  \MH_if_boolean:nT {outer_mult}{
    \g@addto@macro\EQ_complete_mathdisplay:{\end{MTmultlined}}
  }
  \EQ_end_equation:
%    \end{macrocode}
%  Save the value of the \pkg{hyperref} link counter
%  \cs{Hy@linkcounter}.
%    \begin{macrocode}
  \edef\EQ_hyperref_link_fint{\number\Hy@linkcounter}
%    \end{macrocode}
%  Firstly we call \cs{EQ_typeset_tags:} which suppresses all
%  output from the equations with the exception of the \cs{tag}'s.
%  This also allows us to find the widest tag which we'll need for
%  the environments that set's the math displays flush to the
%  margins, namely the \env{flalign} family. We save the counters
%  first though. As an extra treat we're also able to measure the
%  height and depth of the tags and the math display.
%    \begin{macrocode}
  \savecounters@
  \EQ_typeset_tags:
%    \end{macrocode}
%  \changes{v1.05d}{2004/02/26}{Removed one instance of
%  \cs{EQ_typeset_display:}}
%  Then find the widest tag. In an earlier implementation it was
%  necessary to typeset the display twice---a very costly procedure!
%  Now we save roughly 2120 lines of output from \cs{tracingall} on a
%  simple \env{gather} and 2600 lines with \env{align}. That's  a
%  little more than 21\% saved\ldots
%  \changes{v2.10}{2004/07/26}{Removed \cs{find_widesttag:}}
%  \changes{v2.10}{2004/07/26}{Added check for equation number to
%  make sure the labelling/printing mechanism of \pkg{mathtools} work
%  properly}
%  Then we restore the counters and measure the extra material.
%  Save the value of the equation counter for later in case the
%  special equation numbering scheme from \pkg{mathtools} is in
%  effect.
%    \begin{macrocode}
  \edef\l_EQ_equation_fint{\number\c@equation}
  \restorecounters@
%    \end{macrocode}
%  Restore the value of the \pkg{hyperref} link counter
%  \cs{Hy@linkcounter}.
%    \begin{macrocode}
  \Hy@linkcounter=\EQ_hyperref_link_fint
  \settowidth\l_EQ_totalwidth_dim{%
    $\EQ_left_side:\EQ_innerbox:{}\EQ_right_side:$}%
%    \end{macrocode}
%  Then call the appropriate \cs{EQ_modify_linewidth_}\meta{num}\texttt{:}
%  where \meta{num} is taken from the numbering scheme shown earlier.
%    \begin{macrocode}
  \MH_if_boolean:nTF {outer_mult}
    { \def\EQ_next:{_mult} }
    { \def\EQ_next:{} }
  \@nameuse{EQ_modify_linewidth_
            \romannumeral\l_EQ_equationtype_int\EQ_next: :}%
%    \end{macrocode}
%  We can then typeset the display knowing that all modifications
%  have been made correctly. \cs{EQ_typeset_display:} gobbles the
%  \cs{tag}'s, which in turn is equivalent to setting the display
%  centered in non-\mode{fleqn} mode.
%    \begin{macrocode}
  \EQ_typeset_display:
%    \end{macrocode}
%  Call the appropriate \cs{find_displaywidth_}\meta{num}\texttt{:}
%  which extracts the display width (\cs{l_EQ_displaywidth_dim}).
%    \begin{macrocode}
  \@nameuse{find_displaywidth_
           \romannumeral\l_EQ_equationtype_int\EQ_next: :}
  \MH_addtolength:dn
    \l_EQ_totalwidth_dim{\l_EQ_displaywidth_dim}
%    \end{macrocode}
%  We save the math display in a box of the correct size.
%    \begin{macrocode}
  \savebox\EQ_mathdisplay_box[\l_EQ_totalwidth_dim][c]{%
    $\EQ_left_side:
    \EQ_innerbox:{%
      \if_case:w \l_EQ_alignmentmode_int\or:
%    \end{macrocode}
%  If the equation is centered we make the box centered
%    \begin{macrocode}
        \makebox[\l_EQ_displaywidth_dim][c]
          {\usebox{\EQ_mathdisplay_box}}%
      \or:
        \makebox[\l_EQ_displaywidth_dim][c]
          {\usebox{\EQ_mathdisplay_box}}%
      \or:
%    \end{macrocode}
%  Else we set it in a left aligned box and take care of the
%  \cs{@mathmargin} unless we're in \env{flalign} or \env{flalign*}.
%    \begin{macrocode}
        \makebox[\l_EQ_displaywidth_dim][l]{%
          \if_num:w \l_EQ_equationtype_int=5 \else:
            \if_num:w \l_EQ_equationtype_int=6 \else:
              \kern-\@mathmargin
            \fi:
          \fi:
          \usebox{\EQ_mathdisplay_box}}%
      \or:
        \makebox[\l_EQ_displaywidth_dim][l]{%
          \if_num:w \l_EQ_equationtype_int=5 \else:
            \if_num:w \l_EQ_equationtype_int=6 \else:
              \kern-\@mathmargin
            \fi:
          \fi:
          \usebox{\EQ_mathdisplay_box}}%
      \fi:}%
    \EQ_right_side:$}%
%    \end{macrocode}
%  Then we're ready for typesetting. We just use an ordinary
%  |\begin{equation*}| | ... | |\end{equation*}| construction.
%  \changes{v2.00}{2004/04/14}{Used a saved definition of
%  \env{equation*} instead to allow overloading}
%  With v2.00 this changed to using a copy of the original
%  \env{equation*} environment. All this because of the
%  \opt{overload} option.
%    \begin{macrocode}
  \begin{AmSequation*}
  \@nameuse{EQ_place_display_
           \romannumeral\l_EQ_equationtype_int\EQ_next: :}
%    \end{macrocode}
%  \cs{tag} must have a non-zero width argument in order to function
%  properly, so we make sure there is a tag before we try to typeset
%  it. Also make sure that the boolean |show_manual_tags| is true
%  just for this tag.
%    \begin{macrocode}
  \if_dim:w \g_EQ_widesttag_dim>\z@
  \MH_set_boolean_T:n {show_manual_tags}
%    \end{macrocode}
%  Set the tag in a left aligned box when in \mode{leqno} mode and
%  right aligned when in \mode{reqno} mode.
%    \begin{macrocode}
    \if_case:w \l_EQ_alignmentmode_int
    \or: \EQ_place_tag:N r %#1
    \or: \EQ_place_tag:N l %#2
    \or: \EQ_place_tag:N r %#3
    \or: \EQ_place_tag:N l %#4
    \fi:
  \fi:
  \end{AmSequation*}
%    \end{macrocode}
%  \changes{v2.11}{2004/08/03}{Use raw \TeX{} commands to speed it up}
%  Restore the equation counter.
%    \begin{macrocode}
  \global\c@equation=\l_EQ_equation_fint\relax
}
%    \end{macrocode}
%  \end{macro}
%
%  \subsubsection{Initializing the Equations}
%
%  \begin{macro}{\EQ_initialize:}
%  We save all the initializations in this hook:
%    \begin{macrocode}
\def\EQ_initialize:{%
  \m@th
%    \end{macrocode}
%  More support for \pkg{ntheorem}. We use special versions of
%  the tag setting commands in order to produce the correct result.
%  For instance in \mode{leqno} mode you would still want the end of
%  theorem symbol at the bottom of the math display.
%    \begin{macrocode}
  \MH_let:NwN\SetTagPlusEndMark\EQ_tag_plus_endmark:
  \MH_let:NwN\SetOnlyEndMark\EQ_only_endmark:
  \l_EQ_temp_linewidth_dim\linewidth
  \g_EQ_toprow_height_dim\z@
  \g_EQ_bottomrow_depth_dim\z@
  \g_EQ_toptag_height_dim\z@
  \g_EQ_bottomtag_depth_dim\z@
  \g_EQ_widesttag_dim\z@
%    \end{macrocode}
%  \changes{v2.11}{2004/08/03}{Added a \cs{g_EQ_latest_nonzerodepth_row_fint}}
%    \begin{macrocode}
  \xdef\g_EQ_latest_nonzerodepth_row_fint{\z@}
  \edef\EQ_restore_tex_delimiter:{%
    \delimiterfactor\the\delimiterfactor
    \delimitershortfall\the\delimitershortfall\relax}%
  \g@addto@macro\EQ_complete_mathdisplay:
                \EQ_restore_tex_delimiter:
  \delimitershortfall\EmphEqdelimitershortfall
  \delimiterfactor\EmphEqdelimiterfactor
%    \end{macrocode}
%  Disable \cs{intertext} and issue an error message. Using the
%  normal error message for an invalid \cs{intertext} would be less
%  than great because it would say
%  \begin{verbatim}
%  ! Package amsmath Error: Invalid use of \intertext.
%
%  See the amsmath package documentation for explanation.
%  Type  H <return>  for immediate help.
%   ...
%  \end{verbatim}
%  That's not good enough, so we redefine things a bit. In
%  \pkg{amsmath} the command \cs{intertext@} is called inside the
%  math environments and defines \cs{intertext}. We hook into this.
%    \begin{macrocode}
  \MH_let:NwN\intertext@\EQ_intertext:
  \EQ_displaybreak:}
%    \end{macrocode}
%  \end{macro}
%  \begin{macro}{\EQ_intertext:}
%  Then we issue a suitable error message.
%    \begin{macrocode}
\def\EQ_intertext:{%
  \def\intertext##1{\PackageError{empheq}{%
    You~cannot~use~\string\intertext\space inside~the~`empheq'
    \MessageBreak
    environment,~as~`empheq'~creates~an~unbreakable~box}\@eha}}
\def\EQ_displaybreak:{%
  \def\dspbrk@[##1]{\PackageWarning{empheq}{%
    You~cannot~use~\string\displaybreak\space inside~the~`empheq'%
    \MessageBreak
    environment,~as~`empheq'~creates~an~unbreakable~box.}}}
%    \end{macrocode}
%  \end{macro}
%  \begin{macro}{\EQ_new_equationtype:nnn}
%  \begin{macro}{\EQ_new_equationtype_arg:nnn}
%  We have to inform \env{EmphEqMainEnv} which equation type we're
%  using, so we define an interface for that.
%    \begin{macrocode}
\def\EQ_new_equationtype:nnn #1#2#3{%
  \define@key{#1}{#2}[true]{%
    \l_EQ_equationtype_int=#3\relax
    \def\EQ_begin_equation:
      {\g@addto@macro{\EQ_complete_mathdisplay:}{\begin{#2}}}%
    \def\EQ_end_equation:
      {\g@addto@macro{\EQ_complete_mathdisplay:}{\end{#2}}}}}
\def\EQ_new_equationtype_arg:nnn #1#2#3{%
  \define@key{#1}{#2}{%
    \l_EQ_equationtype_int=#3\relax
    \def\EQ_begin_equation:
      {\g@addto@macro{\EQ_complete_mathdisplay:}{\begin{#2}{##1}}}%
    \def\EQ_end_equation:
      {\g@addto@macro{\EQ_complete_mathdisplay:}{\end{#2}}}}}
%    \end{macrocode}
%  \end{macro}
%  \end{macro}
%
%  \subsubsection{Separating Tags and Display}
%
%  \begin{macro}{\EQ_typeset_tags:}
%  Typesetting the tags without the display.
%    \begin{macrocode}
\def\EQ_typeset_tags:{%
  \sbox\EQ_tag_box{%
    \parbox{.5\maxdimen}{%
      \hbadness\@M
%    \end{macrocode}
%  Make sure all skips are zero ands support \pkg{ntheorem}.
%  \changes{v1.05a}{2004/02/17}{Forgot to support \pkg{ntheorem} in
%  both \cs{EQ_typeset_tags:} and \cs{EQ_typeset_display:}.}
%    \begin{macrocode}
      \EQ_prevent_vertical_space:
%    \end{macrocode}
%  Support for \pkg{ntheorem}. This package patches \cs{endgather}
%  and \cs{endalign} with the commands \cs{TagsPlusEndmarks} and
%  \cs{RestoreTags} but we do not want them disturbing our output so
%  we disable them when typesetting the display.
%    \begin{macrocode}
      \EQ_typeset_tags_hook:
%    \end{macrocode}
%  Then use our own versions of \cs{set@gather@field} and
%  \cs{set@field}.
%    \begin{macrocode}
      \MH_let:NwN\set@gather@field\EQ_set_gather_field:
      \MH_let:NwN\set@field\EQ_set_align_field:
%    \end{macrocode}
%  Next two lines are for measuring of the first and last row of a
%  display of the \env{align}, \env{alignat} or \env{flalign} family.
%    \begin{macrocode}
      \MH_set_boolean_T:n {continue_measuring}
      \global\g_EQ_temprow_int\m@ne
      \MH_let:NwN \EQ_saved_maketag:n \maketag@@@
      \def\maketag@@@##1{
        \EQ_maketag:Nn \EQ_saved_maketag:n {##1}}
%    \end{macrocode}
%  Then typeset the display.
%    \begin{macrocode}
      \EQ_complete_mathdisplay:}}}
%    \end{macrocode}
%  \end{macro}
%
%  \begin{macro}{\EQ_predisplay_adjustment:}
%  \changes{v1.05c}{2004/02/25}{Added \cs{EQ_predisplay_adjustment:}}
%
%  \changes{v2.00}{2004/04/14}{Added a missing \cs{normallineskip} in
%  case of extremely big tag height}
%
%  For adjusting vertical spacing in the display.
%    \begin{macrocode}
\def\EQ_predisplay_adjustment:{%
  \if_dim:w \g_EQ_toprow_height_dim<\baselineskip
    \vskip\g_EQ_toprow_height_dim
    \vskip-\baselineskip
  \else:
    \vskip-\normallineskip
  \fi:
  \if_dim:w \g_EQ_toptag_height_dim>\g_EQ_toprow_height_dim
    \if_dim:w \g_EQ_toptag_height_dim>\baselineskip
      \vskip-\g_EQ_toptag_height_dim
      \vskip\baselineskip
      \vskip-\normallineskip
    \fi:
  \fi:}
%    \end{macrocode}
%  \end{macro}
%  \begin{macro}{\EQ_postdisplay_adjustment:}
%  \changes{v1.05c}{2004/02/25}{Added \cs{EQ_postdisplay_adjustment:}}
%  Name says it all.
%    \begin{macrocode}
\def\EQ_postdisplay_adjustment:{%
  \if_dim:w \g_EQ_bottomrow_depth_dim<\dp\strutbox
    \vskip\g_EQ_bottomrow_depth_dim
    \vskip-\dp\strutbox
  \fi:
  \if_dim:w \g_EQ_bottomtag_depth_dim>\g_EQ_bottomrow_depth_dim
    \if_dim:w \g_EQ_bottomtag_depth_dim>\dp\strutbox
      \vskip-\g_EQ_bottomtag_depth_dim
      \vskip\dp\strutbox
    \fi:
  \fi:}
%    \end{macrocode}
%  \end{macro}
%  \begin{macro}{\EQ_typeset_display:}
%  \changes{v1.05c}{2004/02/25}{Now takes two arguments: Either both
%  empty or \cs{EQ_predisplay_adjustment:} \&
%  \cs{EQ_postdisplay_adjustment:}}
%  \changes{v1.05d}{2004/02/26}{Changed again. Uses only
%  \cs{EQ_predisplay_adjustment:} \& \cs{EQ_postdisplay_adjustment:}}
%  We typeset the display without the tags and save it in a box.
%    \begin{macrocode}
\def\EQ_typeset_display:{%
  \sbox\EQ_mathdisplay_box{%
    \parbox{\l_EQ_temp_linewidth_dim}{%
%    \end{macrocode}
%  We adjust the vertical spacing if needed:
%    \begin{macrocode}
      \EQ_predisplay_adjustment:
      \EQ_prevent_vertical_space:
%    \end{macrocode}
%  Support for \pkg{ntheorem}. This package patches \cs{endgather}
%  and \cs{endalign} with the commands \cs{TagsPlusEndmarks} and
%  \cs{RestoreTags} but we do not want them disturbing our output so
%  we disable them when typesetting the display.
%    \begin{macrocode}
      \EQ_typeset_display_hook:
%    \end{macrocode}
%  Instead of gobbling the tags it's better to use \cs{phantom} since
%  a tall tag will ruin the vertical spacing. We use a special
%  version of \cs{phantom} to help us with other things as well.
%    \begin{macrocode}
      \def\maketag@@@##1{\EQ_maketag:Nn \vphantom{##1}}%
      \EQ_complete_mathdisplay:
%    \end{macrocode}
%  More space adjustment.
%    \begin{macrocode}
      \EQ_postdisplay_adjustment:}}}
%    \end{macrocode}
%  \end{macro}
%  We need a few little helpers:
%  \begin{macro}{\EQ_prevent_vertical_space:}
%  \changes{v2.00}{2004/04/14}{Used \cs{nointerlineskip} instead}
%  As we typeset the math display inside a \cs{parbox} the math
%  environment will add \cs{abovedisplayskip} at the top and
%  \cs{belowdisplayskip} at the bottom. Also we suppress interline
%  skip with \cs{nointerlineskip}.
%    \begin{macrocode}
\def\EQ_prevent_vertical_space:{%
  \nointerlineskip
  \abovedisplayskip\z@  \belowdisplayskip\z@\relax}
%    \end{macrocode}
%  \end{macro}
%  \begin{macro}{\EQ_maketag:Nn}
%  \changes{v1.05d}{2004/02/26}{Changed name from
%  \cs{EmphEqphantom} to \cs{EmphEqMaketag}}
%  This takes care of two things: We need to store the height of the
%  top tag and the depth of the bottom tag for later, but we also
%  need to use a \cs{vphantom} of the tag as it may be taller than
%  the line of math it is attached to (say, if you are using node
%  connectors from \pkg{pstricks} or some other weird stuff).
%    \begin{macrocode}
\def\EQ_maketag:Nn #1#2{%
  \setboxz@h{#2}%
  \if_num:w \row@=\@ne
    \global \g_EQ_toptag_height_dim = \ht\z@
  \fi:
%    \end{macrocode}
%  Similarly the depth of the last row is determined by measuring the
%  depth of box zero for each row and then using the last of the
%  values.
%    \begin{macrocode}
  \global \g_EQ_bottomtag_depth_dim=\dp\z@
%    \end{macrocode}
%  \changes{v2.11}{2004/08/03}{Added a check for being in the last row.}
%  Save the row number for later as the \opt{ntheorem} option will need
%  it.
%    \begin{macrocode}
  \xdef\g_EQ_latest_nonzerodepth_row_fint{\number\row@}
  \@tempdima=\wd\z@
  \if_dim:w \@tempdima > \g_EQ_widesttag_dim
    \global \g_EQ_widesttag_dim = \@tempdima
  \fi:
  #1{#2}%
}
%    \end{macrocode}
%  \end{macro}
%
%  \subsubsection{Measuring \env{gather}}
%  \begin{macro}{\EQ_set_gather_field:}
%  Box zero (\cs{z@}) contains the entire line of math in a
%  \env{gather}, so instead of typesetting it we typeset the phantom
%  of it and call \cs{EQ_measure_gather:}.
%    \begin{macrocode}
\def\EQ_set_gather_field:{%
  \iftagsleft@
    \global\lineht@\ht\z@
  \else:
    \global\lineht@\dp\z@
  \fi:
  \kern\eqnshift@
  %\boxz@ % Original amsmath version
  \EQ_measure_gather:
  \phantom{\boxz@}%
  \hfil}
%    \end{macrocode}
%  \end{macro}
%  \begin{macro}{\EQ_measure_gather:}
%  Measuring the height of the individual lines in a \env{gather} is
%  fairly easy. If \cs{row@} is equal to one, we're in the first
%  line, and the height of the top row is simply given by the height
%  of box zero.
%    \begin{macrocode}
\def\EQ_measure_gather:{%
  \if_num:w \row@=\@ne
    \g_EQ_toprow_height_dim=\ht\z@
  \fi:
%    \end{macrocode}
%  Similarly the depth of the last row is determined by measuring the
%  depth of box zero for each row and then using the last of the
%  values.
%    \begin{macrocode}
  \g_EQ_bottomrow_depth_dim=\dp\z@
  \global\g_EQ_toprow_height_dim\g_EQ_toprow_height_dim
  \global\g_EQ_bottomrow_depth_dim\g_EQ_bottomrow_depth_dim}
%    \end{macrocode}
%  \end{macro}
%
%  \subsubsection{Measuring \env{align}, \env{alignat} and \env{flalign}}
%  \begin{macro}{\EQ_set_align_field:}
%  Same technique as for \cs{EQ_set_gather_field}. Only this time
%  we have to stop the underlying \cs{start@align} from going over
%  the measuring phase twice. That's when the conditional
%  \bool{continue_measuring} comes into the picture.
%    \begin{macrocode}
\def\EQ_set_align_field:{%
  \column@plus
  \iftagsleft@
    \if_dim:w \ht\z@>\lineht@
      \global\lineht@\ht\z@
    \fi:
  \else:
    \if_dim:w \dp\z@>\lineht@
      \global\lineht@\dp\z@
    \fi:
  \fi:
  %\boxz@ % Original amsmath version
  \MH_if_boolean:nT {continue_measuring}{ \EQ_measure_align: }
  \phantom{\boxz@}}
%    \end{macrocode}
%  \end{macro}
%  \begin{macro}{\EQ_measure_align:}
%  We have to measure every single field of the \cs{align@} \ldots\@
%  Oh well, here it comes:
%    \begin{macrocode}
\newcount\g_EQ_temprow_int
\MH_new_boolean:n {still_in_samerow}
\MH_new_boolean:n {continue_measuring}
\def\EQ_measure_align:{%
  \if_num:w \g_EQ_temprow_int=\row@
    \MH_set_boolean_T:n {still_in_samerow}
  \else:
    \if_num:w \g_EQ_temprow_int>\row@
      \global\MH_set_boolean_F:n {continue_measuring}
    \fi:
    \MH_set_boolean_F:n {still_in_samerow}
    \global\@tempdimc=\@tempdimb
    \global\@tempdimb=\g_EQ_bottomrow_depth_dim
  \fi:
  \MH_if_boolean:nTF {continue_measuring}{
    \global\g_EQ_temprow_int=\row@
    \@tempcnta=\row@
    \if_num:w \@tempcnta=\z@
      \@tempdima=\g_EQ_toprow_height_dim
      \g_EQ_toprow_height_dim=\ht\z@\relax
      \if_dim:w \g_EQ_toprow_height_dim<\@tempdima
        \g_EQ_toprow_height_dim\@tempdima
      \fi:
    \fi:
    \MH_if_boolean:nTF {still_in_samerow}{
      \@tempdima=\g_EQ_bottomrow_depth_dim
      \g_EQ_bottomrow_depth_dim=\dp\z@\relax
      \if_dim:w \g_EQ_bottomrow_depth_dim<\@tempdima
        \g_EQ_bottomrow_depth_dim\@tempdima
      \fi:
    }{
      \g_EQ_bottomrow_depth_dim\dp\z@\relax
    }
  }{
    \global\g_EQ_bottomrow_depth_dim=\@tempdimc
  }
  \global\g_EQ_toprow_height_dim\g_EQ_toprow_height_dim
  \global\g_EQ_bottomrow_depth_dim\g_EQ_bottomrow_depth_dim}
%    \end{macrocode}
%  \end{macro}
%  \subsubsection{Typesetting the Tags}
%  \begin{macro}{\EQ_place_tag:N}
%  Fool \LaTeX\ into thinking the tag is only as wide as the widest
%  tag (remember the real width of the \cs{EQ_tag_box} is
%  $0.5$\cs{maxdimen}.
%    \begin{macrocode}
\def\EQ_place_tag:N #1{%
  \tag*{\makebox[\g_EQ_widesttag_dim][#1]{%
    \EQ_vertical_tag_adjustment:n {\usebox{\EQ_tag_box}}}}}
%    \end{macrocode}
%  \end{macro}
%  \begin{macro}{\EQ_vertical_tag_adjustment:n}
%  \changes{v2.00}{2004/04/14}{Added a missing \cs{normallineskip} in
%  case of extremely big tag height}
%  This is a little tricky. We have measured the height of the top
%  row and top tag and the depth of the bottom row and bottom tag.
%  Now we have to adjust the vertical spacing so it comes out just
%  right. Remember we have separated the tags and the math display
%  but we would still like to have them line up as they should! We do
%  it all inside a group so we can use the scratch dimensions safely.
%    \begin{macrocode}
\def\EQ_vertical_tag_adjustment:n #1{{%
  \@tempdima\z@\@tempdimb\z@
  \@tempdimc=\dp\strutbox
  \if_dim:w \g_EQ_toprow_height_dim<\baselineskip
    \MH_setlength:dn
      \@tempdima{\g_EQ_toprow_height_dim-\baselineskip}
  \else:
    \MH_setlength:dn \@tempdima{-\normallineskip}
  \fi:
  \if_dim:w \g_EQ_toptag_height_dim>\g_EQ_toprow_height_dim
    \if_dim:w \g_EQ_toptag_height_dim>\baselineskip
      \MH_addtolength:dn \@tempdima
        {-\g_EQ_toptag_height_dim+\baselineskip-\normallineskip}
    \fi:
  \fi:
  \if_dim:w \g_EQ_bottomrow_depth_dim<\@tempdimc
    \MH_setlength:dn \@tempdimb
      {\g_EQ_bottomrow_depth_dim-\@tempdimc}
  \fi:
%    \end{macrocode}
%  \changes{v2.00}{2004/04/14}{Fixed minor bug}
%  If the last condition is fulfilled we also need to cancel the
%  first condition.
%    \begin{macrocode}
  \if_dim:w \g_EQ_bottomtag_depth_dim>\g_EQ_bottomrow_depth_dim
    \if_dim:w \g_EQ_bottomtag_depth_dim<\@tempdimc
    \else:
      \MH_addtolength:dn \@tempdimb
        {-\g_EQ_bottomtag_depth_dim+\@tempdimc}
    \fi:
  \fi:
  \raisebox{-0.5\@tempdima+0.5\@tempdimb}{#1}}}%
%    \end{macrocode}
%  \end{macro}
%  \section{Adding the Equation Types}
%  Adding an equation type is fairly easy. The outline is this:
%  \begin{list}{}{\leftmargin2em}
%    \item[]|\EQ_new_equationtype{EmphEqEnv}|\marg{\AmS\
%    env}\marg{arabic num}\texttt{:}
%    \item[]|\def\EQ_modify_linewidth_|\meta{roman num}\texttt{:}\\
%    \hspace*{2em}\marg{Optionally alter \cs{l_EQ_temp_linewidth_dim}}
%    \item[]|\def\find_displaywidth_|\meta{roman num}\texttt{:}\\
%    \hspace*{2em}\marg{Set \cs{l_EQ_displaywidth_dim}}
%    \item[]|\def\EQ_place_display_|\meta{roman num}\texttt{:}\\
%    \hspace*{2em}\marg{Make use of \cs{EQ_use_mathdisplay:}}
%  \end{list}
%  \begin{macro}{\EQ_use_mathdisplay:}
%  A shorthand.
%    \begin{macrocode}
\def\EQ_use_mathdisplay:{%
  \EQ_outerbox:{\usebox{\EQ_mathdisplay_box}}}
%    \end{macrocode}
%  \end{macro}
%    \begin{macrocode}
\def\EQ_options_name:{EmphEqOpt}
%    \end{macrocode}
%  \subsection{The \env{equation} Family}
%  Just a few aliases because \env{equation} and \env{equation*}
%  behave strangely. Much better to use \env{gather} instead as this
%  is easier to measure. Besides it's less work (for us, not the
%  compiler). ;-)
%    \begin{macrocode}
\MH_keyval_alias_with_addon:nnnn {EmphEqEnv}{equation}{gather}
  {\EQ_disable_linebreak:}
\MH_keyval_alias_with_addon:nnnn {EmphEqEnv}{equation*}{gather*}
  {\EQ_disable_linebreak:}
%    \end{macrocode}
%  \begin{macro}{\EQ_disable_linebreak:}
%  \changes{v1.05c}{2004/02/25}{Introduced
%  \cs{EQ_disable_linebreak:} to prevent multi line
%  \env{equation}'s}
%  \changes{v1.05c}{2004/02/25}{Fixed it so that it works better}
%  Disable the line breaks in \env{equation} as these are actually
%  possible now, since we're in a \env{gather}! \cs{Let@} does it in
%  the environment, so that's the place to hack.
%    \begin{macrocode}
\def\EQ_disable_linebreak:{%
  \MH_let:NwN \EQ_saved_Let: \Let@
  \def\Let@{\def\\{%
    \PackageError{empheq}{You~ cannot~ use~ \string\\~ in~ equation!}
      {\@eha}}
%    \end{macrocode}
%  Then restore the meaning of \cs{Let@} in case there is a
%  \env{gathered} or alike coming up.
%    \begin{macrocode}
    \MH_let:NwN \Let@ \EQ_saved_Let:
  }%
}
%    \end{macrocode}
%  \end{macro}
%
%  \subsection{The \env{align} Family}
%    \begin{macrocode}
\EQ_new_equationtype:nnn {EmphEqEnv}{align}{1}
\EQ_new_equationtype:nnn {EmphEqEnv}{align*}{2}
%    \end{macrocode}
%    \begin{macrocode}
\def\EQ_modify_linewidth_i:{%
%    \end{macrocode}
%  If $\cs{maxfields@}=2$ we're in a single column \env{align} and
%  thus there is no flexible space.
%    \begin{macrocode}
  \if_num:w \maxfields@=\tw@
  \else:
%    \end{macrocode}
%  But if there is any flexible space we would very much like to
%  adjust it because we add material to the right and left of the
%  math display.
%    \begin{macrocode}
    \setlength{\l_EQ_temp_linewidth_dim}{%
      \linewidth-\l_EQ_totalwidth_dim-\widthof{\EQ_outerbox:{}}}%
  \fi:}
\def\EQ_modify_linewidth_ii:{\EQ_modify_linewidth_i:}
\def\find_displaywidth_i:{%
  \MH_setlength:dn \l_EQ_displaywidth_dim
    {\l_EQ_temp_linewidth_dim-\alignsep@}
  \if_case:w \l_EQ_alignmentmode_int
  \or: % #1:
    \MH_addtolength:dn \l_EQ_displaywidth_dim{-\alignsep@}%
  \or: % #2:
    \MH_addtolength:dn \l_EQ_displaywidth_dim{-\alignsep@}%
  \or: % #3:
    \MH_addtolength:dn \l_EQ_displaywidth_dim{-\@mathmargin}%
  \or: % #4:
    \MH_addtolength:dn \l_EQ_displaywidth_dim{-\@mathmargin}%
  \fi:}
\def\find_displaywidth_ii:{\find_displaywidth_i:}
\def\EQ_place_display_i: {\EQ_use_mathdisplay:}
\def\EQ_place_display_ii:{\EQ_place_display_i:}
%    \end{macrocode}
%
%  \subsection{The \env{gather} Family}
%  This is a gem. Notice how easy it is once we got the framework for
%  it.
%    \begin{macrocode}
\EQ_new_equationtype:nnn {EmphEqEnv}{gather}{3}
\EQ_new_equationtype:nnn {EmphEqEnv}{gather*}{4}
\def\EQ_modify_linewidth_iii:{}
\def\EQ_modify_linewidth_iv:{\EQ_modify_linewidth_iii:}
\def\find_displaywidth_iii:{%
  \MH_setlength:dn \l_EQ_displaywidth_dim{\totwidth@}%
  \if_case:w \l_EQ_alignmentmode_int \or: \or: % #1 & #2:
  \or: % #3:
    \MH_addtolength:dn \l_EQ_displaywidth_dim{-\@mathmargin}%
  \or: % #4:
    \MH_addtolength:dn \l_EQ_displaywidth_dim{-\@mathmargin}%
  \fi:}
\def\find_displaywidth_iv:{\find_displaywidth_iii:}
\def\EQ_place_display_iii:{\EQ_use_mathdisplay:}
\def\EQ_place_display_iv: {\EQ_place_display_iii:}
%    \end{macrocode}
%
%  \subsection{The \env{flalign} Family}\label{sec:flalign+multline}
%  This was fairly easy, but then I had to take care of \ldots
%
%  \subsubsection{The \env{multline} Family}
%  This was quite a pickle. Its implementation in \pkg{amsmath} is
%  very strange indeed; for instance the tag is typeset in the
%  measuring phase! So instead I thought I could fake it, but how
%  should I do it?
%
%  The basic idea of \env{multline} is to typeset the first line
%  flush left, the last line flush right and the rest centered.
%  ``Aha,'' I thought, ``what if I make three columns, fool \LaTeX{}
%  into thinking the contents of the columns take no space and then
%  just put each line into a column?'' Thus  the \env{flalign}
%  environment came to mind. I also distinctly remembered to have
%  seen an article about the equivalent of \cs{rlap} and \cs{llap} in
%  math mode, namely an article by Alexander R.~Perlis. This works
%  pretty well, and so does the same trick for producing a
%  \env{multlined} environment. That one uses an \env{alignedat} with
%  three columns.
%
%  \changes{v1.05c}{2004/02/25}{Makes use of
%  \cs{MakeKeyvalAliasAddon} now}
%  Then we make \env{multline} into an alias for {flalign} but with
%  an extra switch \cs{ifEQ_multline:} set to true.
%    \begin{macrocode}
\MH_keyval_alias_with_addon:nnnn {EmphEqEnv}{multline}
  {flalign}{\MH_set_boolean_T:n {multline}}
\MH_keyval_alias_with_addon:nnnn {EmphEqEnv}{multline*}
  {flalign*}{\MH_set_boolean_T:n {multline}}
%    \end{macrocode}
%  \begin{macro}{\if_EQ_multline:}
%  \begin{macro}{\EQ_generic_multline:nnnnn}
%  We make a general command that'll provide the correct spacing and
%  then just define the user commands in terms of that single generic
%  one.
%    \begin{macrocode}
\MH_new_boolean:n {multline}
%    \end{macrocode}
%  \begin{macro}{\EQ_multline_modify_linewidth_tag:}
%  \begin{macro}{\EQ_multline_modify_linewidth_notag:}
%  There is a big difference between the spacing in \env{multline}
%  depending on the presence of a tag. Therefore we split it up in
%  two cases. I made this choice because we need these adjustments
%  inside the adjustments for \env{flalign} and that makes it
%  incredibly difficult to see what happens.
%    \begin{macrocode}
\def\EQ_multline_modify_linewidth_tag:{%
  \if_case:w \l_EQ_alignmentmode_int
  \or: % #1:
    \MH_addtolength:dn \l_EQ_temp_linewidth_dim{%
      -\g_EQ_widesttag_dim-\multlinetaggap-\multlinegap}%
  \or:
    \MH_addtolength:dn \l_EQ_temp_linewidth_dim{%
      -\g_EQ_widesttag_dim-\multlinetaggap-\multlinegap}%
  \or:
    \MH_addtolength:dn \l_EQ_temp_linewidth_dim{%
      -\@mathmargin-\g_EQ_widesttag_dim-\multlinetaggap}%
  \or:
    \MH_addtolength:dn
      \l_EQ_temp_linewidth_dim{-\@mathmargin-\multlinegap}%
  \fi:
}
\def\EQ_multline_modify_linewidth_notag:{%
  \if_case:w \l_EQ_alignmentmode_int\or:
    \MH_addtolength:dn \l_EQ_temp_linewidth_dim{-2\multlinegap}
  \or:
    \MH_addtolength:dn \l_EQ_temp_linewidth_dim{-2\multlinegap}
  \or:
    \MH_addtolength:dn
      \l_EQ_temp_linewidth_dim{-\@mathmargin-\multlinegap}%
  \or:
    \MH_addtolength:dn
      \l_EQ_temp_linewidth_dim{-\@mathmargin-\multlinegap}%
  \fi:
}
%    \end{macrocode}
%  \end{macro}
%  \end{macro}
%
%  \begin{environment}{MTmultlined}
%  Save the superior \env{multlined} from \pkg{mathtools} under a new
%  name.
%    \begin{macrocode}
\MH_let:NwN \MTmultlined \multlined
\MH_let:NwN \endMTmultlined \endmultlined
%    \end{macrocode}
%  \end{environment}
%  We make a choice here: the good \env{multline} or the bad?
%    \begin{macrocode}
\MH_if_boolean:nTF {good_mult}{
%    \end{macrocode}
%  The good:
%    \begin{macrocode}
\MH_keyval_alias_with_addon:nnnn {EmphEqEnv}{multline}{gather}
  {\MH_set_boolean_T:n {outer_mult}}
\MH_keyval_alias_with_addon:nnnn {EmphEqEnv}{multline*}{gather*}
  {\MH_set_boolean_T:n {outer_mult}}
\def\EQ_modify_linewidth_iii_mult:{
  \setlength{\l_EQ_temp_linewidth_dim}{%
    \linewidth-\widthof{\EQ_outerbox:{}}}%
  \if_dim:w \g_EQ_widesttag_dim>\z@
    \EQ_multline_modify_linewidth_tag:
  \else:
    \EQ_multline_modify_linewidth_notag:
  \fi:
  \MH_setlength:dn \l_MT_multwidth_dim{\l_EQ_temp_linewidth_dim}
}
\def\EQ_modify_linewidth_iv_mult:{\EQ_modify_linewidth_iii_mult:}
\def\find_displaywidth_iii_mult:{%
  \MH_setlength:dn \l_EQ_displaywidth_dim{\totwidth@}%
  \if_case:w \l_EQ_alignmentmode_int \or: \or: % #1 & #2:
  \or: % #3:
    \MH_addtolength:dn \l_EQ_displaywidth_dim{-\@mathmargin}%
  \or: % #4:
    \MH_addtolength:dn \l_EQ_displaywidth_dim{-\@mathmargin}%
  \fi:}
\def\find_displaywidth_iv_mult:{\find_displaywidth_iii_mult:}
\def\EQ_place_display_iii_mult:{
  \if_case:w \l_EQ_alignmentmode_int
  \or: % #1:
    \if_dim:w \g_EQ_widesttag_dim>\z@
      \abovedisplayshortskip=\abovedisplayskip
      \rlap{
        \kern-.5\linewidth
        \kern\multlinegap
        \EQ_use_mathdisplay:
      }
    \else:
      \EQ_use_mathdisplay:
    \fi:
  \or: % #2:
    \if_dim:w \g_EQ_widesttag_dim>\z@
      \llap{
        \EQ_use_mathdisplay:
        \kern-.5\linewidth
        \kern\multlinegap
      }%
    \else:
      \EQ_use_mathdisplay:
    \fi:
  \or: % #3:
    \EQ_use_mathdisplay:
  \or: % #4:
    \EQ_use_mathdisplay:
  \fi:
}
\def\EQ_place_display_iv_mult: {\EQ_place_display_iii_mult:}
%    \end{macrocode}
%    \begin{macrocode}
}{
%    \end{macrocode}
%  The bad
%    \begin{macrocode}
\def\EQ_generic_multline:nnnnn #1#2#3#4#5{%
%    \end{macrocode}
%  This is a very tricky little detail. If we're in \mode{leqno} mode
%  we need just a we little width inside the \env{flalign}, so we put
%  in the smallest possible amount. Otherwise we wind up in a
%  \begin{verbatim}
%  ! Arithmetic overflow.
%  <recently read> \count@
%
%  l.304 %  \end{empheq}
%  \end{verbatim}
%  If you scroll past this error it will be typeset correctly, but
%  how many users can fix this by themselves?
%    \begin{macrocode}
  \hskip1sp#2#3{{}#5}#4%
  \if:w *#1\relax\else:\notag\fi:}
%    \end{macrocode}
%  \end{macro}
%  \end{macro}
%  \begin{macro}{\multleft}
%  \begin{macro}{\multcenter}
%  \begin{macro}{\multright}
%  Then the commands that use \cs{EQ_generic_multline:nnnnn}. The
%  first three are the general purpose ones that also have a starred
%  version. The other three are for normal use.
%    \begin{macrocode}
\def\multleft{\@ifstar{%
  \MH_if_boolean:nTF {multlined}{
    \def\EQ_next:{\EQ_multlined_left:n}
  }{
    \def\EQ_next:{
      \EQ_generic_multline:nnnnn {*}{}{\mathrlap}{&&&&&}}
  }
  \EQ_next:}{%
  \MH_if_boolean:nTF {multlined}{
    \def\EQ_next:{\EQ_multlined_left:n}%
  }{
    \def\EQ_next:{
      \EQ_generic_multline:nnnnn {}{}{\mathrlap}{&&&&&}}
  }
  \EQ_next:}}%
\def\multcenter{\@ifstar{%
  \MH_if_boolean:nTF {multlined}{
    \def\EQ_next:{\EQ_multlined_center:n}%
  }{
    \def\EQ_next:{
      \EQ_generic_multline:nnnnn {*}{&&}{\mathclap}{&&&}}
  }
  \EQ_next:}{%
  \MH_if_boolean:nTF {multlined}{
    \def\EQ_next:{\EQ_multlined_center:n }%
  }{
    \def\EQ_next:{
      \EQ_generic_multline:nnnnn {}{&&}{\mathclap}{&&&}}
  }
  \EQ_next:}}%
\def\multright{\@ifstar{%
  \MH_if_boolean:nTF {multlined}{
    \def\EQ_next:{\EQ_multlined_right:n }%
  }{
    \def\EQ_next:{
      \EQ_generic_multline:nnnnn {*}{&&&&&}{\mathllap}{}}
  }
  \EQ_next:}{%
  \MH_if_boolean:nTF {multlined}{
    \def\EQ_next:{\EQ_multlined_right:n }%
  }{
    \def\EQ_next:{
      \EQ_generic_multline:nnnnn {}{&&&&&}{\mathllap}{}}
  }
  \EQ_next:}}
%    \end{macrocode}
%  \end{macro}
%  \end{macro}
%  \end{macro}
%  \begin{macro}{\multfirst}
%  \begin{macro}{\multmiddle}
%  \begin{macro}{\multlast}
%  If we're using the \mode{leqno} option, the equation number should
%  be placed in the first line.
%    \begin{macrocode}
\def\multfirst{%
  \MH_if_boolean:nTF {multlined}{
     \def\EQ_next:{\EQ_multlined_left:n }%
  }{
    \if_case:w \l_EQ_alignmentmode_int
    \or: % #1:
      \def\EQ_next:{\multleft}%
    \or: % #2:
      \def\EQ_next:{\multleft*}%
    \or: % #3:
      \def\EQ_next:{\multleft}%
    \or: % #4:
      \def\EQ_next:{\multleft*}%
    \fi:
  }
  \EQ_next:}
%    \end{macrocode}
%  Pretend the star was never there.
%    \begin{macrocode}
\def\multmiddle{%
  \def\EQ_next:{%
    \MH_if_boolean:nTF {multlined}{
      \expandafter\EQ_multlined_center:n
    }{
      \expandafter\multcenter
    }
  }%
  \@ifstar{\EQ_next:}{\EQ_next:}}
%    \end{macrocode}
%  If we're  using the \mode{reqno} option, the equation number
%  should be placed in the last line.
%    \begin{macrocode}
\def\multlast{%
  \MH_if_boolean:nTF {multlined}{
     \def\EQ_next:{\EQ_multlined_right:n }%
  }{
    \if_case:w \l_EQ_alignmentmode_int
    \or: % #1:
      \def\EQ_next:{\multright*}%
    \or: % #2:
      \def\EQ_next:{\multright}%
    \or: % #3:
      \def\EQ_next:{\multright*}%
    \or: % #4:
      \def\EQ_next:{\multright}%
    \fi:
  }
  \EQ_next:}
%    \end{macrocode}
%  \end{macro}
%  \end{macro}
%  \end{macro}
%  \begin{macro}{\multlinedwidth}
%  Here begins the code for \env{multlined}.
%    \begin{macrocode}
\MH_new_boolean:n {multlined}
\newdimen\multlinedwidth
\MH_setlength:dn \multlinedwidth{.7\linewidth}
%    \end{macrocode}
%  \end{macro}
%  \begin{environment}{multlined}
%  \changes{v2.10}{2004/07/26}{Disallow spaces before optional
%  arguments.}
%  \begin{macro}{\EQ_multlined_I:w}
%  \begin{macro}{\EQ_multlined_II:w}
%  \begin{macro}{\EQ_multlined_III:nn}
%  Some trickery as we want to scan for two optional arguments that
%  may be interchanged.
%    \begin{macrocode}
\renewenvironment{multlined}%
  {\MH_nospace_ifnextchar:Nnn [  {\EQ_multlined_I:w}
                                 {\EQ_multlined_I:w[c]}
  }
  { \endaligned }
\def\EQ_multlined_I:w[#1]{%
  \MH_nospace_ifnextchar:Nnn[  {\EQ_multlined_II:w[#1]}
                               {\EQ_multlined_II:w[#1][\multlinedwidth]}}
\def\EQ_multlined_II:w[#1][#2]{%
%    \end{macrocode}
%  We tell \LaTeX\ that we're in \env{multlined}.
%    \begin{macrocode}
  \MH_set_boolean_T:n {multlined}
%    \end{macrocode}
%  \changes{v2.10}{2004/07/26}{Fixed restoration of \cmd{\\}}
%  Restore the meaning of \cmd{\\} inside \env{multlined}, else it
%  wouldn't work in the \env{equation} environment.
%    \begin{macrocode}
  \MH_let:NwN \Let@\MT_saved_Let:
%    \end{macrocode}
%  Then check the first optional argument and call
%  \cs{EQ_multlined_III:w}, which sets \cs{multlinewidth} if
%  specified and calls \cs{start@aligned} (an \env{alignedat} with
%  three columns).
%    \begin{macrocode}
  \if:w t#1\relax \EQ_multlined_III:nn {#1}{#2}%
  \else:
    \if:w b#1\relax \EQ_multlined_III:nn {#1}{#2}%
    \else:
      \if:w c#1\relax \EQ_multlined_III:nn {#1}{#2}%
      \else:
        \EQ_multlined_III:nn {#2}{#1}%
      \fi:
    \fi:
  \fi:
  \mkern-\thinmuskip
  \EQ_next:}
\def\EQ_multlined_III:nn #1#2{%
  \def\EQ_next:{\setlength\multlinedwidth{#2}%
                          \start@aligned{#1}{\thr@@}}}
%    \end{macrocode}
%  \end{macro}
%  \end{macro}
%  \end{macro}
%  \end{environment}
%  \begin{macro}{\EQ_multlined_left:n}
%  \begin{macro}{\EQ_multlined_center:n}
%  \begin{macro}{\EQ_multlined_right:n}
%  The internal versions of \cs{multleft} etc.\ adapted for
%  \env{multlined}.
%    \begin{macrocode}
\def\EQ_multlined_left:n #1{%
  \mathrlap{#1}&\hskip.5\multlinedwidth&&\hskip.5\multlinedwidth&&}
\def\EQ_multlined_center:n #1{%
  &\hskip.5\multlinedwidth&&\mathclap{#1}\hskip.5\multlinedwidth&&}
\def\EQ_multlined_right:n #1{%
  &\hskip.5\multlinedwidth&&\hskip.5\multlinedwidth&&\mathllap{#1}}
%    \end{macrocode}
%  \end{macro}
%  \end{macro}
%  \end{macro}
%
%  End of bad multline.
%    \begin{macrocode}
}
%    \end{macrocode}
%
%  \subsubsection{The Code for \env{flalign} \& \env{multline}}
%    \begin{macrocode}
\EQ_new_equationtype:nnn {EmphEqEnv}{flalign}{5}
\EQ_new_equationtype:nnn {EmphEqEnv}{flalign*}{6}
\def\EQ_modify_linewidth_v:{%
  \setlength{\l_EQ_temp_linewidth_dim}{%
    \linewidth-\l_EQ_totalwidth_dim-\widthof{\EQ_outerbox:{}}}%
  \if_dim:w \g_EQ_widesttag_dim>\z@
    \MH_if_boolean:nTF {multline}{
      \EQ_multline_modify_linewidth_tag:
    }{
      \if_case:w \l_EQ_alignmentmode_int
      \or: % #1:
        \MH_addtolength:dn  \l_EQ_temp_linewidth_dim
          {-\g_EQ_widesttag_dim-\mintagsep}
      \or: % #2:
        \MH_addtolength:dn \l_EQ_temp_linewidth_dim
          {-\g_EQ_widesttag_dim-\mintagsep}%
      \or: % #3:
        \MH_addtolength:dn \l_EQ_temp_linewidth_dim
          {-\g_EQ_widesttag_dim-\mintagsep}
      \or: % #4:
        \MH_addtolength:dn \l_EQ_temp_linewidth_dim
          {-\@mathmargin}
      \fi:
    }
  \else:
    \MH_if_boolean:nT {multline}{
      \EQ_multline_modify_linewidth_notag:
    }
  \fi:}
\def\EQ_modify_linewidth_vi:{\EQ_modify_linewidth_v:}
\def\find_displaywidth_v:{%
  \MH_setlength:dn
    \l_EQ_displaywidth_dim{\l_EQ_temp_linewidth_dim}}
\def\find_displaywidth_vi:{\find_displaywidth_v:}
\def\EQ_place_display_v:{%
  \if_case:w \l_EQ_alignmentmode_int
  \or: % #1:
    \if_dim:w \g_EQ_widesttag_dim>\z@
      \MH_if_boolean:nTF {multline}{
        \rlap{\kern\multlinegap  \kern-.5\linewidth
          \EQ_use_mathdisplay:
          \kern-\mintagsep  \kern\multlinetaggap}%
      }{
        \rlap{\kern-.5\linewidth\EQ_use_mathdisplay:}%
      }
    \else:
      \EQ_use_mathdisplay:
    \fi:
  \or: % #2:
    \if_dim:w \g_EQ_widesttag_dim>\z@
      \MH_if_boolean:nTF {multline}{
        \llap{\kern-\mintagsep \kern\multlinetaggap
          \EQ_use_mathdisplay:
          \kern\multlinegap  \kern-.5\linewidth}%
      }{
        \llap{\EQ_use_mathdisplay:\kern-.5\linewidth}%
      }
    \else:
      \EQ_use_mathdisplay:
    \fi:
  \or: % #3:
    \MH_if_boolean:nF {multline}{ \kern-\multlinetaggap }
    \EQ_use_mathdisplay:
  \or: % #4:
    \if_dim:w \g_EQ_widesttag_dim>\z@
    \else:
      \MH_if_boolean:nF {multline}{ \kern-\@mathmargin }
    \fi:
    \EQ_use_mathdisplay:
  \fi:
  \MH_set_boolean_F:n {multline}}
\def\EQ_place_display_vi:{\EQ_place_display_v:}
%    \end{macrocode}
%
%  \subsection{The \env{alignat} Family}
%  This is just as easy as \env{gather}. Actually only one line is
%  different from \env{gather}!
%    \begin{macrocode}
\EQ_new_equationtype_arg:nnn {EmphEqEnv}{alignat}{7}
\EQ_new_equationtype_arg:nnn {EmphEqEnv}{alignat*}{8}
\def\EQ_modify_linewidth_vii:{}
\def\EQ_modify_linewidth_viii:{\EQ_modify_linewidth_vii:}
\def\find_displaywidth_vii:{%
  \MH_setlength:dn \l_EQ_displaywidth_dim{\totwidth@}%
  \if_case:w \l_EQ_alignmentmode_int
  \or: % #1:
  \or: % #2:
  \or: % #3:
    \MH_addtolength:dn \l_EQ_displaywidth_dim{-\eqnshift@}%
  \or: % #4:
    \MH_addtolength:dn \l_EQ_displaywidth_dim{-\@mathmargin}%
  \fi:}
\def\find_displaywidth_viii:{\find_displaywidth_vii:}
\def\EQ_place_display_vii:{\EQ_use_mathdisplay:}
\def\EQ_place_display_viii:{\EQ_place_display_vii:}
%    \end{macrocode}
%  \section{The User Interface}
%  The different keys.
%    \begin{macrocode}
\define@key{\EQ_options_name:}{box}{\def\EQ_outerbox:{#1}}
\MH_keyval_alias:nnn {\EQ_options_name:}{outerbox}{box}
%    \end{macrocode}
%  \begin{macro}{\EQ_marginbox:nn}
%  The margin box. Uwe Siart wanted it.
%    \begin{macrocode}
\def\EQ_marginbox:nn #1#2{%
  \settowidth{\@tempdima}{#2}%
  \makebox[\z@]{\hspace{\@tempdima}#1{#2}}}
\define@key{\EQ_options_name:}{marginbox}{%
  \def\EQ_outerbox:{\EQ_marginbox:nn {#1}}}
%    \end{macrocode}
%  \end{macro}
%  Almost self-explanatory, isn't?
%    \begin{macrocode}
\define@key{\EQ_options_name:}{innerbox}{\def\EQ_innerbox:{#1}}
\define@key{\EQ_options_name:}{left}{\def\EQ_left_side:{#1}}
\define@key{\EQ_options_name:}{right}{\def\EQ_right_side:{#1}}
%    \end{macrocode}
%  \begin{macro}{\empheqset}
%  Just a user friendly way to set keys. Uwe Siart again.
%    \begin{macrocode}
\newcommand*\empheqset[1]{\setkeys{\EQ_options_name:}{#1}}
%    \end{macrocode}
%  \end{macro}
%  The old keys from the original version are no longer supported.
%  Instead they give an error message (RM):
%    \begin{macrocode}
\def\EQ_upgrade_error:{%
  \PackageError{empheq}
    {You~ are~ using~ the~ old~ syntax!\MessageBreak
    The~ `empheq'~ environment~ takes~ a~ mandatory~
    argument~ now.\MessageBreak
    You~ need~ to~ exit~ and~ change~ your~ source~ file.}
    {It~ won't~ work,~ trust me.~ Press~ `X'~ now.}}
\define@key{\EQ_options_name:}{boxtype}{\EQ_upgrade_error:}
\define@key{\EQ_options_name:}{Left}{\EQ_upgrade_error:}
\define@key{\EQ_options_name:}{Right}{\EQ_upgrade_error:}
%    \end{macrocode}
% \begin{macro}{\empheqeqrefbase}
%   This macro is used to locally redefine \cs{eqref}, it will just
%   point at \cs{textup}, the user can then redefine it if nessary.
%    \begin{macrocode}
\let\empheqeqrefbase\textup
%    \end{macrocode}
%   
% \end{macro}
%  \begin{environment}{empheq}
%  The environment.
%    \begin{macrocode}
\newenvironment{empheq}[2][]{%
%    \end{macrocode}
% \begin{macro}{\EQ_saved_maketag}
% \changes{v2.13}{2013/02/11}{Fixing \cs{eqref}}
% We start with a small caveat: During processing \cs{maketag@@@} is
% redefined, enabling us to separate math and eqn numbers. But the
% user may have used \cs{eqref} in a comment in the math, and
% \cs{eqref} use \cs{maketag@@@}\dots{} First save the original \cs{maketag@@@}:
%    \begin{macrocode}
  \MH_let:NwN\EQ_saved_maketag: \maketag@@@%
%    \end{macrocode}
% Next redefine \cs{eqref} to locally bring back the original:
%    \begin{macrocode}
  \renewcommand\eqref[1]{\empheqeqrefbase{%
       \let\maketag@@@\EQ_saved_maketag:%
       \tagform@{\ref{##1}}}}
  \setkeys{EmphEqEnv}{#2}\setkeys{\EQ_options_name:}{#1}%
  \EmphEqMainEnv}{\endEmphEqMainEnv}
%    \end{macrocode}
%  \end{macro}
% \end{environment}
%  \begin{macro}{\DeclareLeftDelimiter}
%  \begin{macro}{\EQ_declare_left_delimiter:nnnn}
%  \begin{macro}{\DeclareRightDelimiter}
%  \begin{macro}{\EQ_declare_right_delimiter:nnnn}
%  The delimiter interface.
%  \changes{v0.7}{2003/05/11}{Introduced delimiter interface}
%  \changes{v1.05d}{2004/02/26}{Changed the definitions.}
%  We now use the \cs{EQ_tag_box} to calculate the height but
%  adjust the vertical spacing. All this to prevent running
%  \cs{EQ_typeset_display:} twice.
%    \begin{macrocode}
\newcommand*\DeclareLeftDelimiter[2][]{%
  \EQ_declare_left_delimiter:nnnn {#1}{#2}{}{}%
  \EQ_declare_left_delimiter:nnnn {#1}{#2}{big}{\EQ_innerbox:}}
\def\EQ_declare_left_delimiter:nnnn #1#2#3#4{%
  \@namedef{empheq#3\expandafter\@gobble\string#2}{%
    \left#2%
    \mkern-\thinmuskip
    \vphantom{#4{\parbox{.5\maxdimen}{%
      \EQ_predisplay_adjustment:
      \EQ_prevent_vertical_space:
      \usebox{\EQ_tag_box}%
      \EQ_postdisplay_adjustment:}}}%
    \right.%
    \kern-\nulldelimiterspace #1}}
\newcommand*\DeclareRightDelimiter[2][]{%
  \EQ_declare_right_delimiter:nnnn {#1}{#2}{}{}%
  \EQ_declare_right_delimiter:nnnn {#1}{#2}{big}{\EQ_innerbox:}}
\def\EQ_declare_right_delimiter:nnnn #1#2#3#4{%
  \@namedef{empheq#3\expandafter\@gobble\string#2}{%
    \kern-\nulldelimiterspace
    #1%
    \left.%
    \vphantom{#4{\parbox{.5\maxdimen}{%
      \EQ_predisplay_adjustment:
      \EQ_prevent_vertical_space:
      \usebox{\EQ_tag_box}%
      \EQ_postdisplay_adjustment:}}}%
    \mkern-\thinmuskip
    \right#2}}
%    \end{macrocode}
%  \end{macro}
%  \end{macro}
%  \end{macro}
%  \end{macro}
%  The actual declarations.
%    \begin{macrocode}
\DeclareLeftDelimiter{\lbrace}
\DeclareRightDelimiter{\rbrace}
\DeclareLeftDelimiter{\lbrack}
\DeclareRightDelimiter{\rbrack}
\DeclareLeftDelimiter{\langle}
\DeclareRightDelimiter{\rangle}
\DeclareLeftDelimiter{\lvert}
\DeclareRightDelimiter{\rvert}
\DeclareLeftDelimiter{\lVert}
\DeclareRightDelimiter{\rVert}
\DeclareLeftDelimiter{\lfloor}
\DeclareRightDelimiter{\rfloor}
\DeclareLeftDelimiter{\lceil}
\DeclareRightDelimiter{\rceil}
\DeclareLeftDelimiter{\lparen}
\DeclareRightDelimiter{\rparen}
%    \end{macrocode}
%
%  \section{Support for Other Packages}
%
%  Support for various other packages is found in this section. The
%  \pkg{ntheorem} package now has a dedicated section (\S
%  \vref{sec:impl:ntheorem}), as it now a module.
%
%
%
%    \begin{macrocode}
\def\EQ_typeset_tags_hook:{}
\def\EQ_typeset_display_hook:{}
%    \end{macrocode}
%
%  \subsection{\pkg{showkeys}}
%  \changes{v2.10}{2004/07/26}{Added support for the \pkg{showkeys}
%  package}
%  We just disable \pkg{showkeys} when typesetting the labels.
%    \begin{macrocode}
\g@addto@macro\EQ_typeset_display_hook:{
  \def\SK@@label#1>#2\SK@{}
}
%    \end{macrocode}
%  If we overload the \AmS{} equation types we should still be able
%  to use \pkg{showkeys}.
%    \begin{macrocode}
\@namedef{SK@AmSequation}  {\SK@equationtrue}
\@namedef{SK@AmSequation*} {\SK@equationtrue}
\@namedef{SK@AmSalign}     {\SK@equationtrue}
\@namedef{SK@AmSalign*}    {\SK@equationtrue}
\@namedef{SK@AmSalignat}   {\SK@equationtrue}
\@namedef{SK@AmSalignat*}  {\SK@equationtrue}
\@namedef{SK@AmSgather}    {\SK@equationtrue}
\@namedef{SK@AmSgather*}   {\SK@equationtrue}
\@namedef{SK@AmSmultline}  {\SK@equationtrue}
\@namedef{SK@AmSmultline*} {\SK@equationtrue}
\@namedef{SK@AmSflalign}   {\SK@equationtrue}
\@namedef{SK@AmSflalign*}  {\SK@equationtrue}
%    \end{macrocode}
%
%  \subsection{\pkg{hyperref}}
%
%
%  A little trick to avoid the (in)famous warning
%  \begin{verbatim}
%  ! pdfTeX warning (ext4): destination with the same identifier
%  (name{equation.1a }) has been already used, duplicate ignored
%  <to be read again>
%                     \endgroup \set@typeset@protect
%  \end{verbatim}
%  This doesn't require \pkg{hyperref}.
%    \begin{macrocode}
\AtBeginDocument{
  \@ifundefined{Hy@linkcounter}{\newcount\Hy@linkcounter}{}
}
\g@addto@macro\EQ_typeset_display_hook:{
  \MH_let:NwN \hyper@refstepcounter\@gobble
}
%    \end{macrocode}
% 
% \subsection{\pkg{nccmath}}
%
% Must disable this macro, otherwise the spacing is wrong.
%    \begin{macrocode}
 \g@addto@macro\EQ_typeset_display_hook:{
   \let\NCC@ignorepar\relax
 }
 \g@addto@macro\EQ_typeset_tags_hook:{
   \let\NCC@ignorepar\relax
 }
%    \end{macrocode}
%
%  \subsection{Basic \pkg{ntheorem} support}
%    \begin{macrocode}
\g@addto@macro\EQ_typeset_tags_hook:{
  \MH_let:NwN\TagsPlusEndmarks\relax
  \MH_let:NwN\RestoreTags\relax
}
\g@addto@macro\EQ_typeset_display_hook:{
  \MH_let:NwN\TagsPlusEndmarks\relax
  \MH_let:NwN\RestoreTags\relax
}
%    \end{macrocode}
%  \subsection{Basic \LaTeXe{} support}
%  Gobble the \cs{ltx@label} and disable most error messages.
%  \changes{v2.00}{2004/04/14}{Make a special label-gobbling command
%  to support \pkg{ntheorem} even more}
%    \begin{macrocode}
\g@addto@macro\EQ_typeset_display_hook:{
  \MH_let:NwN\ltx@label\MT_gobblelabel:w
  \MH_let:NwN \GenericError \@gobblefour
}
%    \end{macrode}
%  \subsection{Support for \pkg{fancybox}}
%  First check if \pkg{fancybox} is loaded and then see if
%  \pkg{color} is loaded. If the latter is loaded we define
%  \texttt{shadowcolor} else we let \cs{textcolor} gobble the first
%  argument (the color).
%    \begin{macrocode}
\AtBeginDocument{%
  \@ifpackageloaded{fancybox}{%
    \@ifpackageloaded{color}{\definecolor{shadowcolor}{rgb}{0,0,0}}%
    {\newcommand*\textcolor[2]{\normalcolor{#2}}}%
%    \end{macrocode}
%  \begin{macro}{\shadowbox}
%  Then define the \cs{shadowbox}. Use a conditional to let us know if
%  it was starred or not.
%    \begin{macrocode}
    \MH_new_boolean:n {shadowbox_star}
    \renewcommand*\shadowbox{%
      \@ifstar{\MH_set_boolean_T:n {shadowbox_star}
               \VerbBox\EQ_shadowbox:n}
              {\MH_set_boolean_F:n {shadowbox_star}
               \VerbBox\EQ_shadowbox:n}}
%    \end{macrocode}
%  \end{macro}
%  \begin{macro}{\EQ_shadowbox:n}
%  The internal version of \cs{shadowbox}.
%    \begin{macrocode}
    \def\EQ_shadowbox:n #1{%
      \sbox{\z@}{\fbox{#1}}%
%    \end{macrocode}
%  The difference between the two versions is that \cs{shadowbox}
%  aligns the bottom of the shadow at the baseline whereas
%  \cs{shadowbox*} aligns the argument on the baseline, so for the
%  starred version the trick is to raise the box by the depth of the
%  argument plus the size of the shadow.
%    \begin{macrocode}
      \MH_if_boolean:nTF {shadowbox_star}{
        \MH_setlength:dn \@tempdima{\z@}%
      }{
        \MH_setlength:dn \@tempdima{\dp\z@+\shadowsize}%
      }
      \raisebox{\@tempdima}{%
        \makebox[\z@][l]{%
          \kern\shadowsize \textcolor{shadowcolor}{%
            \rule[-\dp\z@-\shadowsize]{\wd\z@}{\shadowsize}}%
        }%
        \usebox\z@\textcolor{shadowcolor}{%
          \rule[-\dp\z@-\shadowsize]{\shadowsize}{\ht\z@+\dp\z@}}}}%
  }{}%
}
%    \end{macrocode}
%  \end{macro}
%
%
%  \section{The \opt{overload} option}
%  \begin{macro}{\EQ_overload_ams_begin:n}
%  A helper for saving snapshots:
%    \begin{macrocode}
\def\EQ_overload_ams_begin:n #1{%
  \MH_let:cc {AmS#1}{#1}%
  \MH_let:cc {AmS#1*}{#1*}
}
%    \end{macrocode}
%  \end{macro}
%  \begin{macro}{\EQ_overload_ams_end:n}
%  All but one of the aligned variations must be set to the original
%    \begin{macrocode}
\def\EQ_overload_ams_end:n #1{%
  \MH_let:cc {endAmS#1}{endAmSalign}
}
%    \end{macrocode}
%  \end{macro}
%  \begin{environment}{AmSequation}
%  \begin{environment}{AmSequation*}
%  \changes{v2.10}{2004/07/26}{Moved redefinitions to
%  \cs{AtBeginDocument} to give \pkg{hyperref} a chance}
%  Before we start redefining environments, we'll need a basic
%  \env{equation} environment that is not redefined.
%    \begin{macrocode}
\AtBeginDocument{
  \EQ_overload_ams_begin:n {equation}
  \MH_let:cc {endAmSequation}{endequation}
  \MH_let:cc {endAmSequation*}{endequation*}
%    \end{macrocode}
%  \end{environment}
%  \end{environment}
%  Lets start overloading.
%    \begin{macrocode}
  \MH_if_boolean:nT {overload_ams}{
%    \end{macrocode}
%  \begin{macro}{\EQ_renew_ams_environment:n}
%  \changes{v2.10}{2004/07/26}{Disallow spaces before the optional
%  argument.}
%  Then we can renew the environments. Syntax is
%  \cs{begin}\marg{\AmS{} env}\oarg{markup}. Exception to this rule
%  is \env{alignat} (see further down).
%    \begin{macrocode}
    \def\EQ_renew_ams_environment:n #1{%
      \MaybeMHPrecedingSpacesOff
      \renewenvironment{#1}[1][]{%
%    \end{macrocode}
% \changes{v2.13}{2013/02/11}{Adding \cs{eqref} fix}
% Remember to add \cs{eqref} fix to the overloaded envs as well.
%    \begin{macrocode}
        \MH_let:NwN\EQ_saved_maketag: \maketag@@@%
        \renewcommand\eqref[1]{\empheqeqrefbase{%
            \let\maketag@@@\EQ_saved_maketag:%
            \tagform@{\ref{####1}}}}
        \setkeys{\EQ_options_name:}{##1}%
        \setkeys{EmphEqEnv}{AmS#1}%
        \EmphEqMainEnv
      }{\endEmphEqMainEnv}
      \MHPrecedingSpacesOn
    }
%    \end{macrocode}
%  \end{macro}
%  Add equation type: \emph{Must} have same number as original type!
%    \begin{macrocode}
    \EQ_new_equationtype:nnn     {EmphEqEnv}{AmSalign}   {1}
    \EQ_new_equationtype:nnn     {EmphEqEnv}{AmSalign*}  {2}
    \EQ_new_equationtype:nnn     {EmphEqEnv}{AmSgather}  {3}
    \EQ_new_equationtype:nnn     {EmphEqEnv}{AmSgather*} {4}
    \EQ_new_equationtype:nnn     {EmphEqEnv}{AmSflalign} {5}
    \EQ_new_equationtype:nnn     {EmphEqEnv}{AmSflalign*}{6}
    \EQ_new_equationtype_arg:nnn {EmphEqEnv}{AmSalignat} {7}
    \EQ_new_equationtype_arg:nnn {EmphEqEnv}{AmSalignat*}{8}
%    \end{macrocode}
%  Then overload by saving the snapshots of the old definitions. We
%  do it via \cs{AtBeginDocument}, because a package like
%  \pkg{ntheorem} redefines \cs{endgather} etc.
%
%  First the \env{gather} and \env{equation} families.
%    \begin{macrocode}
    \EQ_overload_ams_begin:n {gather}
    \MH_let:cc {endAmSgather}{endgather}
    \MH_let:cc {endAmSgather*}{endAmSgather}
    \EQ_renew_ams_environment:n {gather}
    \EQ_renew_ams_environment:n {gather*}
    \EQ_overload_ams_begin:n {equation}
    \MH_keyval_alias_with_addon:nnnn {EmphEqEnv} {AmSequation}
      {AmSgather}{\EQ_disable_linebreak:}
    \MH_keyval_alias_with_addon:nnnn {EmphEqEnv}{AmSequation*}
      {AmSgather*}{\EQ_disable_linebreak:}
    \EQ_renew_ams_environment:n {equation}
    \EQ_renew_ams_environment:n {equation*}
%    \end{macrocode}
%  The \env{align} family.
%    \begin{macrocode}
    \EQ_overload_ams_begin:n {align}
    \MH_let:cc {endAmSalign}{endalign}
    \EQ_overload_ams_end:n {align*}
    \EQ_renew_ams_environment:n {align}
    \EQ_renew_ams_environment:n {align*}
%    \end{macrocode}
%  The \env{flalign} family.
%    \begin{macrocode}
    \EQ_overload_ams_begin:n {flalign}
    \EQ_overload_ams_end:n {flalign}
    \EQ_overload_ams_end:n {flalign*}
    \EQ_renew_ams_environment:n {flalign}
    \EQ_renew_ams_environment:n {flalign*}
%    \end{macrocode}
%  The \env{alignat} family.
%    \begin{macrocode}
    \EQ_overload_ams_begin:n {alignat}
    \EQ_overload_ams_end:n {alignat}
    \EQ_overload_ams_end:n {alignat*}
%    \end{macrocode}
%  \begin{macro}{\EQ_alignat_optional:w}
%  \changes{v2.10}{2004/07/26}{Disallow spaces before the optional
%  argument.}
%  \begin{environment}{alignat}
%  \begin{environment}{alignat*}
%  This one is a little different because we want a syntax like
%  \cs{begin\{alignat\}}\linebreak[0]\marg{cols}\oarg{markup}, so we
%  have to make a separate command for testing the trailing optional
%  argument.
%    \begin{macrocode}
    \MaybeMHPrecedingSpacesOff
    \newcommand*\EQ_alignat_optional:w [1][]{
      \setkeys{\EQ_options_name:}{#1}
      \EmphEqMainEnv}
    \MHPrecedingSpacesOn
%    \end{macrocode}
% \changes{v2.13}{2013/02/11}{Adding \cs{eqref} fix}
% Remember to add \cs{eqref} fix to the overloaded envs as well.
%    \begin{macrocode}
    \renewenvironment{alignat}[1]{
      \MH_let:NwN\EQ_saved_maketag: \maketag@@@%
      \renewcommand\eqref[1]{\empheqeqrefbase{%
          \let\maketag@@@\EQ_saved_maketag:%
          \tagform@{\ref{##1}}}}
      \setkeys{EmphEqEnv}{AmSalignat=#1}
      \EQ_alignat_optional:w}
      {\endEmphEqMainEnv}
    \renewenvironment{alignat*}[1]{%
      \MH_let:NwN\EQ_saved_maketag: \maketag@@@%
      \renewcommand\eqref[1]{\empheqeqrefbase{%
          \let\maketag@@@\EQ_saved_maketag:%
          \tagform@{\ref{##1}}}}
      \setkeys{EmphEqEnv}{AmSalignat*=#1}%
      \EQ_alignat_optional:w}%
      {\endEmphEqMainEnv}
%    \end{macrocode}
%  \end{environment}
%  \end{environment}
%  \end{macro}
%  \begin{macro}{\[}
%  \begin{macro}{\]}
%  Then restore \cs{[} and \cs{]}.
%    \begin{macrocode}
    \DeclareRobustCommand{\[}{\begin{AmSequation*}}
    \DeclareRobustCommand{\]}{\end{AmSequation*}}
%    \end{macrocode}
%  \end{macro}
%  \end{macro}
%  Finally a string of aliases so that we can blend \opt{overload}
%  syntax with regular syntax.
%    \begin{macrocode}
    \MH_keyval_alias:nnn {EmphEqEnv}{gather}  {AmSgather}
    \MH_keyval_alias:nnn {EmphEqEnv}{gather*} {AmSgather*}
    \MH_keyval_alias:nnn {EmphEqEnv}{align}   {AmSalign}
    \MH_keyval_alias:nnn {EmphEqEnv}{align*}  {AmSalign*}
    \MH_keyval_alias:nnn {EmphEqEnv}{flalign} {AmSflalign}
    \MH_keyval_alias:nnn {EmphEqEnv}{flalign*}{AmSflalign*}
    \MH_keyval_alias:nnn {EmphEqEnv}{alignat} {AmSalignat}
    \MH_keyval_alias:nnn {EmphEqEnv}{alignat*}{AmSalignat*}
    \MH_if_boolean:nT {good_mult}{
      \EQ_overload_ams_begin:n {multline}
      \MH_let:cc {endAmSmultline}{endmultline}
      \MH_let:cc {endAmSmultline*}{endAmSmultline}
      \MH_keyval_alias_with_addon:nnnn {EmphEqEnv}{AmSmultline}
        {AmSgather}{\MH_set_boolean_T:n {outer_mult}}
      \MH_keyval_alias_with_addon:nnnn {EmphEqEnv}{AmSmultline*}
        {AmSgather*}{\MH_set_boolean_T:n {outer_mult}}
      \EQ_renew_ams_environment:n {multline}
      \EQ_renew_ams_environment:n {multline*}
    }
%    \end{macrocode}
%  End of \opt{overload}. We go into \opt{overload2}\ldots This
%  is dangerous, but will probably work. The macro \cs{[} is now
%  fragile except if you're running \eTeX{} as engine for \LaTeX.
%    \begin{macrocode}
    \MH_if_boolean:nT {overload_amsII}{
      \MH_protected:\def\[#1\]{\begin{equation*} #1\end{equation*}}
%    \end{macrocode}
%  Then we add the \env{multline} family.
%    \begin{macrocode}
      \MH_if_boolean:nF {good_mult}{
        \EQ_overload_ams_begin:n {multline}
        \MH_keyval_alias_with_addon:nnnn {EmphEqEnv}{AmSmultline}
          {AmSflalign}{\MH_set_boolean_T:n {multline}}
        \MH_keyval_alias_with_addon:nnnn {EmphEqEnv}{AmSmultline*}
          {AmSflalign*}{\MH_set_boolean_T:n {multline}}
        \EQ_renew_ams_environment:n {multline}
        \EQ_renew_ams_environment:n {multline*}
      }
    }
  }
}
%    \end{macrocode}
%  End of \cs{AtBeginDocument} additions.
%
%  If we need to load \pkg{ntheorem} support this is where it
%  happens. Otherwise we just stop here.
%    \begin{macrocode}
\if\g@EQ@ntheorem@support@bool
\else 
  \expandafter\endinput
\fi
%    \end{macrocode}
%
%  \section{Patching Things up for \pkg{ntheorem}}
%  \label{sec:impl:ntheorem}
%
%  Although \pkg{ntheorem} provides a lot of useful features it
%  does however ruin things in the \pkg{amsmath} environments,
%  because of the alternative syntax of its \cs{label} command.
%  \begin{macro}{\default_raisetag}
%  \begin{macro}{\raisetag}
%  \begin{macro}{\raise@tag}
%  \changes{v1.06}{2004/03/13}{Fixed \pkg{amsmath} bug \# 3624}
%  This corrects a bug\footnote{See
%  \url{http://www.latex-project.org/cgi-bin/ltxbugs2html?pr=amslatex/3624}}
%  in \pkg{amsmath} that could cause tags to cross page boundaries
%  without warning. The \cs{nobreak} is the addition.
%    \begin{macrocode}
\def\default_raisetag:{\nobreak}
\def\raisetag#1{
  \skip@#1\relax
  \xdef\raise@tag{\default_raisetag:
                  \vskip\iftagsleft@\else:-\fi:\the\skip@\relax}}
\MH_let:NwN\raise@tag\default_raisetag:
%    \end{macrocode}
%  \end{macro}
%  \end{macro}
%  \end{macro}
%  \begin{macro}{\displ@y@}
%  Patch it so that it doesn't set \cs{raise@tag} to \cs{@empty}
%  globally.
%    \begin{macrocode}
\def\displ@y@{\@display@init{%
  \global\column@\z@ \global\dspbrk@lvl\m@ne
  \global\tag@false \global\MH_let:NwN\raise@tag\default_raisetag:
}}
%    \end{macrocode}
%  \end{macro}
%  \begin{macro}{\MT_gobblelabel:w}
%  \changes{v2.00}{2004/04/14}{Added \cs{MT_gobblelabel:w}}
%  \begin{macro}{\EQ_gobbleoptional:w}
%  \changes{v2.00}{2004/04/14}{Added \cs{EQ_gobbleoptional:w}}
%  Macros for gobbling the `unorthodox' \cs{label} syntax of
%  \pkg{ntheorem}. Also overrides the normal \cs{label}-gobbling.
%    \begin{macrocode}
\def\MT_gobblelabel:w #1{\@ifnextchar[{\EQ_gobbleoptional:w}{}}
\def\EQ_gobbleoptional:w [#1]{}
%    \end{macrocode}
%  \end{macro}
%  \end{macro}
%  \subsection{Fixing \env{gather}}
%  The measuring phase of \env{gather} is patched by letting
%  \cs{label} be a copy of the internal \cs{MT_gobblelabel:w}
%  instead of \cs{@gobble}.
%    \begin{macrocode}
\def\gmeasure@#1{%
  \begingroup
    \measuring@true
    \totwidth@\z@
    \global\MH_let:NwN\tag@lengths\@empty
    \savecounters@
    \setbox\@ne\vbox{%
      \everycr{\noalign{\global\tag@false
%    \end{macrocode}
%  The patch for \cs{raisetag}.
%    \begin{macrocode}
        \global\MH_let:NwN\raise@tag\default_raisetag:
        \global\column@\z@}}%
%    \end{macrocode}
%  The patch for \cs{label}.
%    \begin{macrocode}
      \MH_let:NwN\label\MT_gobblelabel:w
      \halign{%
        \setboxz@h{$\m@th\displaystyle{##}$}%
        \if_dim:w \wdz@>\totwidth@
          \global\totwidth@\wdz@
        \fi:
        &\setboxz@h{\strut@{##}}%
        \savetaglength@
        \crcr
        #1%
        \math@cr@@@
      }%
    }%
    \restorecounters@
    \if@fleqn
        \global\advance\totwidth@\@mathmargin
    \fi:
    \iftagsleft@
        \if_dim:w \totwidth@>\displaywidth
            \global\MH_let:NwN\gdisplaywidth@\totwidth@
        \else:
            \global\MH_let:NwN\gdisplaywidth@\displaywidth
        \fi:
    \fi:
  \endgroup
}
%    \end{macrocode}
%
%  \subsection{Fixing \env{align}, \env{alignat}, and \env{flalign}}
%  Just copy the definition of \cs{measure@} from \pkg{amsmath} and
%  change one (1!) line like we did for \env{gather}.
%    \begin{macrocode}
\def\measure@#1{%
  \begingroup
    \measuring@true
    \global\eqnshift@\z@
    \global\alignsep@\z@
    \global\MH_let:NwN\tag@lengths\@empty
    \global\MH_let:NwN\field@lengths\@empty
    \savecounters@
    \global\setbox0\vbox{%
      \MH_let:NwN\math@cr@@@\math@cr@@@align@measure
      \everycr{\noalign{\global\tag@false
%    \end{macrocode}
%  The patch for \cs{raisetag}.
%    \begin{macrocode}
        \global\MH_let:NwN\raise@tag\default_raisetag:
        \global\column@\z@}}%
%    \end{macrocode}
%  The patch.
%    \begin{macrocode}
      \MH_let:NwN\label\MT_gobblelabel:w
      \global\row@\z@
      \tabskip\z@
      \halign{\span\align@preamble\crcr
        #1%
        \math@cr@@@
        \global\column@\z@
        \add@amps\maxfields@\cr
      }%
    }%
    \restorecounters@
    \ifodd\maxfields@
      \global\advance\maxfields@\@ne
    \fi:
    \if_num:w \xatlevel@=\tw@
      \if_num:w \maxfields@<\thr@@
        \MH_let:NwN\xatlevel@\z@
      \fi:
    \fi:
    \setbox\z@\vbox{%
      \unvbox\z@ \unpenalty \global\setbox\@ne\lastbox
    }%
    \global\totwidth@\wd\@ne
    \if@fleqn \global\advance\totwidth@\@mathmargin \fi:
    \global\MH_let:NwN\maxcolumn@widths\@empty
    \begingroup
      \MH_let:NwN\or:\relax
      \loop
        \global\setbox\@ne\hbox{%
          \unhbox\@ne \unskip \global\setbox\thr@@\lastbox
        }%
      \ifhbox\thr@@
        \xdef\maxcolumn@widths{ \or: \the\wd\thr@@ \maxcolumn@widths}%
      \repeat
    \endgroup
    \dimen@\displaywidth
    \advance\dimen@-\totwidth@
    \if_case:w \xatlevel@
      \global\alignsep@\z@
      \MH_let:NwN\minalignsep\z@
      \@tempcntb\z@
      \if@fleqn
        \@tempcnta\@ne
        \global\eqnshift@\@mathmargin
      \else:
        \@tempcnta\tw@
        \global\eqnshift@\dimen@
        \global\divide\eqnshift@\@tempcnta
      \fi:
    \or:
      \@tempcntb\maxfields@
      \divide\@tempcntb\tw@
      \@tempcnta\@tempcntb
      \advance\@tempcntb\m@ne
      \if@fleqn
        \global\eqnshift@\@mathmargin
        \global\alignsep@\dimen@
        \global\divide\alignsep@\@tempcnta
      \else:
        \global\advance\@tempcnta\@ne
        \global\eqnshift@\dimen@
        \global\divide\eqnshift@\@tempcnta
        \global\alignsep@\eqnshift@
      \fi:
    \or:
      \@tempcntb\maxfields@
      \divide\@tempcntb\tw@
      \global\advance\@tempcntb\m@ne
      \global\@tempcnta\@tempcntb
      \global\eqnshift@\z@
      \global\alignsep@\dimen@
      \if@fleqn
        \global\advance\alignsep@\@mathmargin\relax
      \fi:
      \global\divide\alignsep@\@tempcntb
    \fi:
    \if_dim:w \alignsep@<\minalignsep\relax
      \global\alignsep@\minalignsep\relax
      \if_dim:w \eqnshift@>\z@
        \if@fleqn\else:
          \global\eqnshift@\displaywidth
          \global\advance\eqnshift@-\totwidth@
          \global\advance\eqnshift@-\@tempcntb\alignsep@
          \global\divide\eqnshift@\tw@
        \fi:
      \fi:
    \fi:
    \if_dim:w \eqnshift@<\z@
      \global\eqnshift@\z@
    \fi:
    \calc@shift@align
    \global\tagshift@\totwidth@
    \global\advance\tagshift@\@tempcntb\alignsep@
    \if@fleqn
      \if_num:w \xatlevel@=\tw@
        \global\advance\tagshift@-\@mathmargin\relax
      \fi:
    \else:
      \global\advance\tagshift@\eqnshift@
    \fi:
    \iftagsleft@ \else:
      \global\advance\tagshift@-\displaywidth
    \fi:
    \dimen@\minalignsep\relax
    \global\advance\totwidth@\@tempcntb\dimen@
    \if_dim:w \totwidth@>\displaywidth
      \global\MH_let:NwN\displaywidth@\totwidth@
    \else:
      \global\MH_let:NwN\displaywidth@\displaywidth
    \fi:
  \endgroup
}
%    \end{macrocode}
%
%  \subsection{Fixing \env{multline}}
%  \begin{macro}{\EQ_multline_labelhack:n}
%  \begin{macro}{\EQ_multline_labelhack_opt:nw}
%  As \pkg{ntheorem} defines \cs{label} to have the syntax
%  \cs{label}\marg{label}\oarg{name}, we need to make sure that it
%  has that inside \env{multline} as well. Especially since the
%  measuring phase of \env{multline} \emph{defines} \cs{label} as
%  well.
%    \begin{macrocode}
\def\EQ_multline_labelhack:n #1{%
  \begingroup\measuring@false\label@in@display{#1}\endgroup}
\def\EQ_multline_labelhack_opt:nw #1[#2]{%
  \begingroup\measuring@false\label@in@display{#1}[#2]\endgroup}
%    \end{macrocode}
%  \end{macro}
%  \end{macro}
%  \begin{macro}{\multline@}
%  Patching \env{multline} takes more work because it does its
%  processing differently, so we need to apply hooks twice: One in
%  the measuring phase \cs{mmeasure@} where \cs{label} is redefined
%  and one in the production phase \cs{multline@} where the ordinary
%  \cs{label} is gobbled (we use \cs{MT_gobblelabel:w} as
%  before). First the typesetting phase.
%    \begin{macrocode}
\def\multline@#1{%
  \Let@
  \@display@init{\global\advance\row@\@ne \global\dspbrk@lvl\m@ne}%
  \chardef\dspbrk@context\z@
  \restore@math@cr
  \MH_let:NwN\tag\tag@in@align
%    \end{macrocode}
%  The patch for \cs{raisetag}.
%    \begin{macrocode}
  \global\tag@false \global\MH_let:NwN\raise@tag\default_raisetag:
  \mmeasure@{#1}%
  \MH_let:NwN\tag\gobble@tag
%    \end{macrocode}
%  The patch.
%    \begin{macrocode}
  \MH_let:NwN\label\MT_gobblelabel:w
  \tabskip \if@fleqn \@mathmargin \else: \z@skip \fi:
  \totwidth@\displaywidth
  \if@fleqn
    \advance\totwidth@-\@mathmargin
  \fi:
  \halign\bgroup
    \hbox to\totwidth@{%
      \if@fleqn
        \hskip \@centering \relax
      \else:
        \hfil
      \fi:
      \strut@
      $\m@th\displaystyle{}##\endmultline@math%$
      \hfil
    }%
    \crcr
    \if@fleqn
      \hskip-\@mathmargin
      \def\multline@indent{\hskip\@mathmargin}% put it back
    \else:
      \hfilneg
      \def\multline@indent{\hskip\multlinegap}%
    \fi:
    \iftagsleft@
      \iftag@
        \begingroup
          \ifshifttag@
            \rlap{\vbox{%
              \normalbaselines
              \hbox{%
                \strut@
                \make@display@tag
              }%
              \vbox to\lineht@{}%
              \raise@tag
            }}%
              \multline@indent
          \else:
            \setbox\z@\hbox{\make@display@tag}%
            \dimen@\@mathmargin \advance\dimen@-\wd\z@
            \if_dim:w \dimen@<\multlinetaggap
              \dimen@\multlinetaggap
            \fi:
            \box\z@ \hskip\dimen@\relax
          \fi:
        \endgroup
      \else:
        \multline@indent
      \fi:
    \else:
      \multline@indent
    \fi:
  #1%
}
%    \end{macrocode}
%  \end{macro}
%  \begin{macro}{\mmeasure@}
%  Then redefine the label correctly.
%    \begin{macrocode}
\def\mmeasure@#1{%
  \begingroup
    \measuring@true
%    \end{macrocode}
%  The patch.
%    \begin{macrocode}
    \def\label##1{%
      \@ifnextchar[
        {\EQ_multline_labelhack_opt:nw {##1}}
        {\EQ_multline_labelhack:n {##1}}
    }
    \def\math@cr@@@{\cr}%
    \MH_let:NwN\shoveleft\@iden \MH_let:NwN\shoveright\@iden
    \savecounters@
    \global\row@\z@
    \setbox\@ne\vbox{%
      \global\MH_let:NwN\df@tag\@empty
      \halign{%
        \setboxz@h{\@lign$\m@th\displaystyle{}##$}%
        \iftagsleft@
          \if_num:w \row@=\@ne
            \global\totwidth@\wdz@
            \global\lineht@\ht\z@
          \fi:
        \else:
          \global\totwidth@\wdz@
          \global\lineht@\dp\z@
        \fi:
        \crcr
        #1%
        \crcr
      }%
    }%
    \if_meaning:NN
      \df@tag\@empty
    \else:
      \global\tag@true
    \fi:
    \if@eqnsw\global\tag@true\fi:
    \iftag@
      \setboxz@h{%
        \if@eqnsw
          \stepcounter{equation}%
          \tagform@\theequation
        \else:
          \df@tag
        \fi:
      }%
      \global\tagwidth@\wdz@
      \dimen@\totwidth@
      \advance\dimen@\tagwidth@
      \advance\dimen@\multlinetaggap
      \iftagsleft@\else:
        \if@fleqn
          \advance\dimen@\@mathmargin
        \fi:
      \fi:
      \if_dim:w \dimen@>\displaywidth
        \global\shifttag@true
      \else:
        \global\shifttag@false
      \fi:
    \fi:
    \restorecounters@
  \endgroup
}
%    \end{macrocode}
%  \end{macro}
%
%  \subsection{Setting End Marks Correctly}
%
%  \begin{macro}{\mintagvsep}
%  \changes{v1.05}{2004/02/16}{Introduced \cs{mintagvsep} to
%  support the \pkg{ntheorem} package}
%  \begin{macro}{\EQ_only_endmark:}
%  \changes{v1.05}{2004/02/16}{Introduced \cs{EQ_only_endmark:}
%  to support the \pkg{ntheorem} package}
%  \changes{v2.11}{2004/08/03}{Make sure the tag is printed even with the
%  \pkg{mathtools} key \key{showonlyrefs} activated}
%  \begin{macro}{\EQ_tag_plus_endmark:}
%  \changes{v1.05}{2004/02/16}{Introduced
%  \cs{EQ_tag_plus_endmark:} to support the \pkg{ntheorem} package}
%  \pkg{ntheorem} provides an extension to the theorem environment
%  and also supports automatic end of theorem marks. To do so it must
%  redefine parts of the \pkg{amsmath} code and it eventually ends up
%  in problems when the tags and the marker are all typeset at the
%  right margin.
%
%  There are basically two different possibilities:
%  \begin{center}\hfill
%  \unitlength2.5pt \thicklines
%  \begin{picture}(60,45)(0,25)
%    \put(0,50){\framebox(25,20){Math Height}}
%    \put(0,25){\framebox(25,25){Math Depth}}
%    \put(40,50){\framebox(20,10){Tag height}}
%    \put(40,45){\framebox(20,5){Tag depth}}
%    \put(40,25){\framebox(20,10){Marker}}
%    \put(50,35){\vector(0,1){10}}
%    \put(50,45){\vector(0,-1){10}}
%    \put(50,39){\ \ $\delta>0$}
%    \multiput(26,25)(2,0){7}{\line(1,0){1}}
%  \end{picture} \hfill\vrule\hfill
%  \begin{picture}(60,45)(0,25)
%    \put(0,50){\framebox(25,20){Math Height}}
%    \put(0,40){\framebox(25,10){Math Depth}}
%    \put(40,50){\framebox(20,10){Tag height}}
%    \put(40,45){\framebox(20,5){Tag depth}}
%    \put(40,28){\framebox(20,10){Marker}}
%    \put(50,38){\vector(0,1){7}}
%    \put(50,45){\vector(0,-1){7}}
%    \put(50,40){\ \ $\delta=0$}
%  \end{picture}\hfill\mbox{}
%  \end{center}
%  Let's look closer at this. $h_M$ is the height of the end of
%  theorem marker, which we'll try to place so that its baseline
%  aligns with the bottom of the math display. $d_D$ is the depth of
%  the the last line of the math display including the depth of the
%  surrounding boxes while $d_T$ is the depth of the tag. Also we let
%  $m$ denote the minimum vertical distance between tag and marker
%  and $\delta$ is some positive length that may be added to $m$
%  depending on the other values. To ensure a ``proper'' placement of
%  the marker we simply test the inequality
%  \begin{gather}
%     h_M + d_T + m > d_D \label{eq:tag_placement}
%  \end{gather}
%  If the test is true we're done and just have to set the marker so
%  that the top of it is placed at a distance of $m$ from the bottom
%  of the tag. If it's not we can safely put the marker at the bottom
%  of the display.
%    \begin{macrocode}
\newlength\mintagvsep
\mintagvsep5pt
\def\EQ_only_endmark:{%
  \MH_set_boolean_T:n {show_manual_tags}
  \global\tag@true
  \iftagsleft@
    \gdef\df@tag{%
      \hbox to \displaywidth{%
        \hss
%    \end{macrocode}
%  If the tags are set to the left we just have to lower the end mark
%  by the total depth of the display including its surrounding box.
%  Also we have to take care of \cs{normallineskip}.
%    \begin{macrocode}
        \raisebox{-\normallineskip
                  -\depthof{\EQ_outerbox:{}}
                  -\heightof{\EQ_outerbox:{}}
                  -\dp\EQ_mathdisplay_box
                  -\ht\EQ_mathdisplay_box}
                 [0pt][0pt]{\PotEndMark{\maketag@@@}}}}
  \else:
    \gdef\df@tag{%
      \raisebox{-\dp\EQ_mathdisplay_box
                -\depthof{\EQ_outerbox:{}}}
               {\PotEndMark{\maketag@@@}}}%
  \fi:}
%    \end{macrocode}
%  \changes{v2.00}{2004/04/14}{Fixed another bug in placement of the
%  marker}
%  \changes{v2.10}{2004/07/26}{Fixed horizontal spacing in \env{equation}}
%  In v2.00 I fixed another bug I had overlooked. This is getting
%  embarrassing. The outer box must be taken into account as well.
%    \begin{macrocode}
\def\EQ_tag_plus_endmark:{%
  \iftagsleft@
    \gdef\maketag@@@##1{%
%    \end{macrocode}
%  The following fixes a bug when there is a wide tag in an
%  \env{equation} environment and was reported by Bernard
%  Alfonsi.\footnote{\email{alfonsi@math.u-psud.fr}} The problem
%  arises when \pkg{ntheorem} tries to typeset a tag and an end mark,
%  because the tag is set at the left margin and the end mark at the
%  right margin and \pkg{ntheorem} over-simplifies this and forgets
%  to take the displayed math into account! The right solution is to
%  wrap the tag in a box of the correct size, namely |\widthof{##1}|
%  and then push the end mark to the right margin. Then the displayed
%  math is set correcly.
%    \begin{macrocode}
      \hfuzz\displaywidth
      \makebox[\widthof{##1}][l]{%
        \hbox to \displaywidth{\m@th\normalfont##1\hss
          \raisebox{-\dp\EQ_mathdisplay_box
                    -\depthof{\EQ_outerbox:{}}}
                   {\PotEndMark{\hss}}}}}%
  \else:
    \gdef\maketag@@@##1{%
      \hbox{\m@th\normalfont##1\llap{\hss\PotEndMark{%
        \settoheight{\@tempdima}{\@nameuse{\InTheoType Symbol}}%
        \MH_addtolength:dn \@tempdima
          {\g_EQ_bottomtag_depth_dim+\mintagvsep}
        \setlength{\@tempdimb}
          {\g_EQ_bottomrow_depth_dim+\depthof{\EQ_outerbox:{}}
            +\depthof{\EQ_innerbox:{}}}
        \if_dim:w \@tempdima>\@tempdimb
          \MH_setlength:dn \@tempdima{-\dp\EQ_mathdisplay_box
            -\@tempdima+\g_EQ_bottomrow_depth_dim}
          \addtolength{\@tempdima}{\depthof{\EQ_innerbox:{}}}%
          \def\EQ_next:{\raisebox{\@tempdima}}%
        \else:
%    \end{macrocode}
%  \changes{v1.06}{2004/03/13}{Fixed bug in placement of the marker
%  and re-arranged some code}
%  In v1.06 I fixed a stupid bug regarding placement of the marker.
%  If I had done a proper test I would have discovered it
%  immediately!
%    \begin{macrocode}
          \def\EQ_next:{
            \raisebox{-\dp\EQ_mathdisplay_box
                      -\depthof{\EQ_outerbox:{}}}}%
        \fi:\EQ_next:}}}}%
  \fi:}
%    \end{macrocode}
%  \end{macro}
%  \end{macro}
%  \end{macro}
%
%  \begin{macro}{\MT_nonumber:}
%    \begin{macrocode}
\MH_let:NwN \MT_nonumber: \nonumber
%    \end{macrocode}
%  \end{macro}
%  \begin{macro}{\nonumber}
%  Redefine \cs{nonumber} to reset depth of bottom tag.
%    \begin{macrocode}
\def\g_EQ_latest_nonzerodepth_row_fint{\z@}
\renewcommand*\nonumber{
  \@tempcnta=\row@
  \if_num:w \l_EQ_equationtype_int=\thr@@
    \advance\@tempcnta\m@ne
  \else:
    \if_num:w \l_EQ_equationtype_int=4\relax
      \advance\@tempcnta\m@ne
    \fi:
  \fi:
  \if_num:w \g_EQ_latest_nonzerodepth_row_fint=\@tempcnta
  \else:
    \global\g_EQ_bottomtag_depth_dim=-\baselineskip
  \fi:
  \MT_nonumber:
}
%    \end{macrocode}
%  \end{macro}
%
%  End of \pkg{ntheorem} patches and package.
%    \begin{macrocode}
%</package>
%    \end{macrocode}
% Finale
\endinput

%        (quote the arguments according to the demands of your shell)
%
% Documentation:
%    (a) If empheq.drv is present:
%           latex empheq.drv
%    (b) Without empheq.drv:
%           latex empheq.dtx; ...
%    The class ltxdoc loads the configuration file ltxdoc.cfg
%    if available. Here you can specify further options, e.g.
%    use A4 as paper format:
%       \PassOptionsToClass{a4paper}{article}
%
%    Programm calls to get the documentation (example):
%       pdflatex empheq.dtx
%       makeindex -s gind.ist empheq.idx
%       pdflatex empheq.dtx
%       makeindex -s gind.ist empheq.idx
%       pdflatex empheq.dtx
%
% Installation:
%    TDS:tex/latex/mh/empheq.sty
%    TDS:doc/latex/mh/empheq.pdf
%    TDS:source/latex/mh/empheq.dtx
%
%<*ignore>
\begingroup
  \def\x{LaTeX2e}
\expandafter\endgroup
\ifcase 0\ifx\install y1\fi\expandafter
         \ifx\csname processbatchFile\endcsname\relax\else1\fi
         \ifx\fmtname\x\else 1\fi\relax
\else\csname fi\endcsname
%</ignore>
%<*install>
\input docstrip.tex
\Msg{************************************************************************}
\Msg{* Installation}
\Msg{* Package: empheq 2014/08/04 v2.14 empheq}
\Msg{************************************************************************}

\keepsilent
\askforoverwritefalse

\preamble

This is a generated file.

Copyright (C) 2002-2007 by Morten Hoegholm
Copyright (C) 2012-     by Lars Madsen


This work may be distributed and/or modified under the
conditions of the LaTeX Project Public License, either
version 1.3 of this license or (at your option) any later
version. The latest version of this license is in
   http://www.latex-project.org/lppl.txt
and version 1.3 or later is part of all distributions of
LaTeX version 2005/12/01 or later.

This work has the LPPL maintenance status "maintained".

This Current Maintainer of this work is
Lars Madsen

This work consists of the main source file empheq.dtx
and the derived files
   empheq.sty, empheq.pdf, empheq.ins, empheq.drv.

\endpreamble

\generate{%
  \file{empheq.ins}{\from{empheq.dtx}{install}}%
  \file{empheq.drv}{\from{empheq.dtx}{driver}}%
  \usedir{tex/latex/mh}%
  \file{empheq.sty}{\from{empheq.dtx}{package}}%
}

\obeyspaces
\Msg{************************************************************************}
\Msg{*}
\Msg{* To finish the installation you have to move the following}
\Msg{* file into a directory searched by TeX:}
\Msg{*}
\Msg{*     empheq.sty}
\Msg{*}
\Msg{* To produce the documentation run the file `empheq.drv'}
\Msg{* through LaTeX.}
\Msg{*}
\Msg{* Happy TeXing!}
\Msg{*}
\Msg{************************************************************************}

\endbatchfile
%</install>
%<*ignore>
\fi
%</ignore>
%<*driver>
\NeedsTeXFormat{LaTeX2e}
\ProvidesFile{empheq.drv}%
  [2014/08/04 v2.14 Emphasizing equations]
\documentclass{ltxdoc}
\usepackage{mathpazo}
\usepackage[overload]{empheq}
\usepackage{xcolor,varioref,booktabs,fancybox}
\usepackage[overload]{textcase}

\pagestyle{headings}
\hyphenation{emph-eq}
\makeatletter
\providecommand*\pkg[1]{\textsf{#1}}
\providecommand*\env[1]{\texttt{#1}}
\providecommand*\email[1]{\href{mailto:#1}{\texttt{#1}}}
\providecommand*\mode[1]{\texttt{[#1]}}
\providecommand*\file[1]{\texttt{#1}}
\addtolength\marginparwidth{-30pt}


\newcommand*\AddDescribe[3]{%
  \@namedef{Describe#1}{\leavevmode\@bsphack\begingroup\MakePrivateLetters
    \@nameuse{Describe@#1}}
  \@namedef{Describe@#1}##1{\endgroup
                \marginpar{\raggedleft\@nameuse{PrintDescribe#1}{##1}}%
                \@nameuse{Special#1Index}{##1}\@esphack\ignorespaces}
  \@namedef{Special#1Index}##1{\@bsphack
      \index{##1\actualchar{\protect\ttfamily ##1}
            (#2)\encapchar usage}%
      \index{#3:\levelchar##1\actualchar{\protect\ttfamily ##1}\encapchar
            usage}\@esphack}
  \@namedef{PrintDescribe#1}##1{\strut \MacroFont ##1\ }%
  \@namedef{Print#1Name}##1{\strut \MacroFont ##1\ }%
}
\AddDescribe{Color}{color}{colors}
\AddDescribe{KeyFamily}{key family}{key families}
\newcommand*\DescribeDelimiter[1]{%
  \cs{#1}&\SpecialMkhIndex{empheq#1}\cs{empheq}\texttt{#1} &
  \SpecialMkhIndex{empheqbig#1}\cs{empheqbig}\texttt{#1} &
  $\@nameuse{#1}$\tabularnewline}
\newcommand*\SpecialMkhIndex[1]{%
  \@bsphack
    \index{#1\actualchar{\protect\ttfamily \protect\bslash\space#1}
     (` $\bslash#1$ ')\encapchar usage}%
  \@esphack
}
\newcommand\TabDescribeOption[3][(default)]{%
    \SpecialOptIndex{#2}\opt{#2}  & #3 \tabularnewline
}
\newcommand*\TabDescribeAmsKey[2][]{%
    \SpecialKeyIndex{#2}\env{#2} &
    \texttt{\{#2#1\}} &
    \SpecialKeyIndex{#2*}\env{#2*} &
    \texttt{\{#2*#1\}} \tabularnewline
}
\newcommand*\TabDescribeOptKey[3][]{%
    \SpecialKeyIndex{#2}\key{#2} &
    \texttt{#2=}\meta{#3} & #1 \tabularnewline
}

\newenvironment{codesyntax}
    {\par\small\addvspace{4.5ex plus 1ex}%
     \vskip -\parskip
     \noindent
     \begin{tabular}{|l|}\hline\ignorespaces}%
    {\\\hline\end{tabular}\nobreak\par\nobreak
     \vspace{2.3ex}\vskip -\parskip\noindent\ignorespacesafterend}
\makeatletter

\newcommand*\FeatureRequest[2]{%
  \hskip1sp
  \marginpar{%
    \parbox[b]{\marginparwidth}{\small\sffamily\raggedright
      \strut Feature request by\\#1\\#2%
    }
  }%
}

\newcommand*\cttPosting[2]{%
  \hskip1sp
  \marginpar{%
    \parbox[b]{\marginparwidth}{\small\sffamily\raggedright
     \strut Posted on \texttt{comp.text.tex} \\#1\\#2%
    }%
  }%
}

\expandafter\def\expandafter\MakePrivateLetters\expandafter{%
  \MakePrivateLetters  \catcode`\_=11\relax
}

\providecommand*\SpecialOptIndex[1]{%
  \@bsphack
  \index{#1\actualchar{\protect\ttfamily #1}
          (option)\encapchar usage}%
      \index{options:\levelchar#1\actualchar{\protect\ttfamily #1}\encapchar
            usage}\@esphack}
\providecommand*\opt[1]{\texttt{#1}}

\providecommand*\SpecialKeyIndex[1]{%
  \@bsphack
  \index{#1\actualchar{\protect\ttfamily #1}
          (key)\encapchar usage}%
      \index{keys:\levelchar#1\actualchar{\protect\ttfamily #1}\encapchar
            usage}\@esphack}
\providecommand*\key[1]{\textsf{#1}}
\providecommand*\keyfam[1]{\textsf{#1}}

\providecommand*\eTeX{$\m@th\varepsilon$-\TeX}

\def\MTmeta#1{%
     \ensuremath\langle
     \ifmmode \expandafter \nfss@text \fi
     {%
      \meta@font@select
      \edef\meta@hyphen@restore
        {\hyphenchar\the\font\the\hyphenchar\font}%
      \hyphenchar\font\m@ne
      \language\l@nohyphenation
      #1\/%
      \meta@hyphen@restore
     }\ensuremath\rangle
     \endgroup
}
\makeatother
\DeclareRobustCommand\meta{\begingroup\MakePrivateLetters\MTmeta}%
\def\MToarg#1{{\ttfamily[}\meta{#1}{\ttfamily]}\endgroup}
\DeclareRobustCommand\oarg{\begingroup\MakePrivateLetters\MToarg}%
\def\MHmarg#1{{\ttfamily\char`\{}\meta{#1}{\ttfamily\char`\}}\endgroup}
\DeclareRobustCommand\marg{\begingroup\MakePrivateLetters\MHmarg}%
\def\MHarg#1{{\ttfamily\char`\{#1\ttfamily\char`\}}\endgroup}
\DeclareRobustCommand\arg{\begingroup\MakePrivateLetters\MHarg}%
\def\MHcs#1{\texttt{\char`\\#1}\endgroup}
\DeclareRobustCommand\cs{\begingroup\MakePrivateLetters\MHcs}
\def\MHbool#1{{\ttfamily #1}\endgroup}
\DeclareRobustCommand\bool{\begingroup\MakePrivateLetters\MHbool}%

\def\endverbatim{\if@newlist
\leavevmode\fi\endtrivlist\vspace{-\baselineskip}}
\expandafter\let\csname endverbatim*\endcsname =\endverbatim

\let\MTtheindex\theindex
\def\theindex{\MTtheindex\MakePrivateLetters}

\newcommand*\widefbox[1]{\fbox{\hspace{1em}#1\hspace{1em}}}
\definecolor{myblue}{rgb}{.8, .8, 1}
\newcommand*\mybluebox[1]{%
  \colorbox{myblue}{\hspace{1em}#1\hspace{1em}}}
\newenvironment{important}[2][]{%
  \setkeys{EmphEqEnv}{#2}%
  \setkeys{EmphEqOpt}{box=\mybluebox,#1}%
  \EmphEqMainEnv}%
  {\endEmphEqMainEnv}
\definecolor{shadecolor}{cmyk}{0,0,0.41,0}
\definecolor{light-blue}{cmyk}{0.25,0,0,0}
\newsavebox{\mysaveboxM} % M for math
\newsavebox{\mysaveboxT} % T for text
\newcommand*\Garybox[2][Example]{%
  \sbox{\mysaveboxM}{#2}%
  \sbox{\mysaveboxT}{\fcolorbox{black}{light-blue}{#1}}%
  \sbox{\mysaveboxM}{%
    \parbox[b][\ht\mysaveboxM+.5\ht\mysaveboxT+.5\dp\mysaveboxT][b]{%
      \wd\mysaveboxM}{#2}%
  }%
  \sbox{\mysaveboxM}{%
    \fcolorbox{black}{shadecolor}{%
      \makebox[\linewidth-10em]{\usebox{\mysaveboxM}}%
    }%
  }%
  \usebox{\mysaveboxM}%
  \makebox[0pt][r]{%
    \makebox[\wd\mysaveboxM][c]{%
      \raisebox{\ht\mysaveboxM-0.5\ht\mysaveboxT
                +0.5\dp\mysaveboxT-0.5\fboxrule}{\usebox{\mysaveboxT}}%
    }%
  }%
}

\newcommand\EmphEqpart{%
  \def\partname{\EmphEqpartname}%
  \part}

\def\EmphEqpartname{Part}
\makeatother

\usepackage{graphicx}
\usepackage[final,breaklinks=true,hyperindex=false]{hyperref}
\renewcommand*\usage[1]{\textit{\hyperpage{#1}}}

\OnlyDescription
\begin{document}
\DocInput{empheq.dtx}
\end{document}
%</driver>
%  \fi
%  \changes{v2.14}{2014/07/16}{Updated maintenance and contact info}
%
%  \changes{v2.11}{2004/08/03}{Removed interface to the obsolete empheq
%  package \pkg{empheq07}}
%
%  \changes{v2.10}{2004/07/26}{Rewrote most of the documentation}
%
%  \changes{v2.10}{2004/07/26}{Moved options \texttt{fixamsmath} and
%  \texttt{donotfixamsmathbugs} to \pkg{mathtools}}
%
%  \changes{v2.10}{2004/07/26}{Added support for \pkg{showkeys}}
%
%  \changes{v2.00}{2004/04/14}{Added options \opt{overload},
%  \opt{overload2}, \texttt{fixamsmath}, and
%  \texttt{donotfixamsmathbugs} plus descriptions of their use}
%
%  \changes{v1.06}{2004/03/13}{Upgraded license to LPPL version 1.3}
%
%  \changes{v1.00}{2004/02/15}{Complete rewrite of the package hence
%  no description of the changes made}
%
%  \changes{v0.7a}{2004/02/15}{Made the old package obsolete and
%  changed its name to \pkg{empheq-obsolete}}
%
%  \changes{v0.7}{2003/05/11}{Added further examples and changed
%  existing ones}
%  \changes{v0.6b}{2003/03/04}{Added example for changing margins
%  inside the boxtype}
%  \changes{v0.6a}{2003/01/28}{Minor updates to documentation}
%
%  \changes{v0.5}{2002/12/30}{First experimental and very limited
%  release}
%
%  \GetFileInfo{empheq.drv}
%  \CheckSum{2495}
%
%  \title{The \pkg{empheq} package\thanks{This file has version number
%  \fileversion, last revised \filedate.}\\
%  Emphasizing equations in \LaTeXe}
%  \author{Morten H\o gholm, Lars Madsen}
%  
%  \date{\filedate}
%
%  \maketitle
%
%  \begin{abstract}
%  The \pkg{empheq} package can best be described as a visual markup
%  extension to \pkg{amsmath}. In short it offers a)~a multi line
%  equivalent of \cs{boxed} from \pkg{amsmath} and b)~a way to
%  produce arbitrary delimiters that span entire math displays.
%  \end{abstract}
%
%  \tableofcontents
%
%  \clearpage
%  \EmphEqpart{Basic user's guide}
%
%  There can be little doubt that the de facto standard for
%  mathematical typesetting in \LaTeX{} is the \pkg{amsmath} package.
%  For the creation of \pkg{empheq}, a visual markup package for use
%  in math, it made perfect sense to have \pkg{amsmath} as the
%  backbone.
%
%  The main idea of \pkg{empheq} is to maintain the familiar syntax
%  of the \pkg{amsmath} environments while still providing an easy
%  way of specifying markup instructions. This manual is plastered
%  with examples showing just how.
%
%
%  \section{Basic use of the package}
%
%  So what is it \pkg{empheq} does? Well, it allows you to produce
%  displays like this:
%  \begin{subequations}
%  \begin{empheq}[
%    box=\mybluebox,
%    right={\;\makebox[.9em]{$\raisebox{-.5\totalheight+\fontdimen22\textfont2}
%              {\resizebox{!}{\EmphEqdisplayheight+\EmphEqdisplaydepth}{!}}$}},
%    left={X=Y\Rightarrow\empheqlbrace}
%     ]{alignat=3}
%    A_1&=b_1 & \qquad c_1&=d_1 & \qquad e_1&= f_1
%         \tag*{\_A\raisebox{1ex}{silly}\raisebox{-1ex}{tag}\_}\\
%    A_2&=b_2 & \qquad c_2&=d_2 & \qquad  e_2&= f_2 \\
%    A_3&=b_3 & \qquad c_3&=d_3 & \qquad  e_3&= f_3
%  \end{empheq}
%  \end{subequations}
%  In short \pkg{empheq} enables the user to put things on every side
%  of the display without said user having to worry about what
%  happens to the equation numbers. For example you can now have a
%  display containing multiple lines and still get the effect of the
%  \cs{boxed} command from \pkg{amsmath}.
%
%
%  \subsection{Using the \env{empheq} environment}
%
%  The package defines a single environment
%  \SpecialEnvIndex{empheq}\env{empheq} and the usage is kind of straight
%  forward:
%  \begin{codesyntax}
%    |\begin{empheq}|\oarg{markup instructions}\marg{\AmS{} env_name}\\
%    \mbox{}\quad\meta{contents of \AmS{} environment}\\
%    |\end{empheq}|
%  \end{codesyntax}
%  A first minimal example file would then be something like
%  \begin{verbatim}
%    \documentclass{minimal}
%    \usepackage{empheq}
%    \begin{document}
%    \begin{empheq}{align*}
%      a&=b \tag{*}\\
%      E&=mc^2 + \int_a^a x\, dx
%    \end{empheq}
%    \end{document}
%  \end{verbatim}
%  This gives the following display:
%    \begin{empheq}{align*}
%      a&=b \tag{*}\\
%      E&=mc^2 + \int_a^a x\, dx
%    \end{empheq}
%  Maybe not the most impressive example, but as you can see the
%  contents of the environment is exactly the same as for the regular
%  \env{align*} evironment from \pkg{amsmath}. The rest of the \AmS{}
%  environments are chosen the same way by typing the name as the
%  mandatory argument of \env{empheq} with the exception of the
%  \env{alignat} environment. For this you have to specify the number
%  of columns as shown below.
%  \begin{verbatim}
%  \begin{empheq}{alignat=2}
%    a &= b &\quad c &= d \\
%    \text{this} &= \text{that} &\quad \mathit{fish}&\neq fish
%  \end{empheq}
%  \end{verbatim}
%  \begin{empheq}{alignat=2}
%    a &= b &\quad c &= d \\
%    \text{this} &= \text{that} &\quad \mathit{fish}&\neq fish
%  \end{empheq}
%  To choose the starred version of \env{alignat}, simply type
%  \arg{alignat*=2} instead in the above example.
%
%  The supported \AmS{} environments are listed in Table
%  \vref{tab:environments}. Not supported is the standard \LaTeX{}
%  \env{eqnarray} environment as it is fundamentally
%  deficient.\footnote{See for instance
%  \url{http://www.tex.ac.uk/cgi-bin/texfaq2html?label=eqnarray}}
%
%  \begin{table}
%  \centering
%  \begin{tabular}{llll}\toprule
%    Environment & Usage & Environment & Usage \\ \midrule
%    \TabDescribeAmsKey{equation}
%    \TabDescribeAmsKey{align}
%    \TabDescribeAmsKey{gather}
%    \TabDescribeAmsKey{flalign}
%    \TabDescribeAmsKey[=\textrm{\meta{cols}}]{alignat}
%    \TabDescribeAmsKey{multline} \bottomrule
%  \end{tabular}
%  \caption{The supported \pkg{amsmath}
%  environments}\label{tab:environments}
%  \end{table}
%
%
%  \subsection{Markup instructions}
%
%  The optional argument of the \env{empheq} environment will take
%  markup instructions consisting of a $\meta{key}=\meta{value}$ list
%  of assignments. There are currently five such keys (a sixth is
%  added for naming consistency). They're shown in Table
%  \vref{tab:keys}.
%
%  \begin{table}
%  \centering
%  \begin{tabular}{llp{13em}}\toprule
%    Key & Usage & Additional Info \tabularnewline \midrule
%    \TabDescribeOptKey{box}{box command}
%    \TabDescribeOptKey{innerbox}{box command}
%    \TabDescribeOptKey[Use \cs{text}\arg{...} if you need text
%                       material.]{left}{math material}
%    \TabDescribeOptKey[Use \cs{text}\arg{...} if you need text
%                       material.]{right}{math material}
%    \TabDescribeOptKey[Alias for \key{box}.]{outerbox}{box command}
%    \TabDescribeOptKey[Contents must be horizontally centered;
%                       can only be used in \mode{fleqn} mode.]
%                       {marginbox}{box command}
%    \bottomrule
%  \end{tabular}
%  \caption{The six keys for the optional argument of the
%  \env{empheq} environment}\label{tab:keys}
%  \end{table}
%
%  \begin{codesyntax}
%    \SpecialKeyIndex{left}\key{left}
%  \end{codesyntax}
%  The key \key{left} is for material put on the left side of the
%  display. The material is typeset in math mode and centered
%  vertically.
%  \begin{verbatim}
%  \begin{empheq}[left=L\Rightarrow]{align}
%    a&=b\\
%    E&=mc^2 + \int_a^a x\, dx
%  \end{empheq}
%  \end{verbatim}
%  \begin{empheq}[left=L\Rightarrow]{align}
%    a&=b\\
%    E&=mc^2 + \int_a^a x\, dx
%  \end{empheq}
%
%
%  \begin{codesyntax}
%    \SpecialKeyIndex{right}\key{right}
%  \end{codesyntax}
%  As there is a \key{left} key it hopefully comes as no surprise
%  that there is a companion \key{right} key for typesetting material
%  on the right side of the display.
%  \begin{verbatim}
%  \begin{empheq}[right=\Leftarrow R]{align}
%    a&=b\\
%    E&=mc^2 + \int_a^a x\, dx
%  \end{empheq}
%  \end{verbatim}
%  \begin{empheq}[right=\Leftarrow R]{align}
%    a&=b\\
%    E&=mc^2 + \int_a^a x\, dx
%  \end{empheq}
%
%  \begin{codesyntax}
%    \SpecialKeyIndex{box}\key{box}\\
%    \SpecialKeyIndex{outerbox}\key{outerbox}
%  \end{codesyntax}
%  The key \key{box} specifies the kind of box you would like to put
%  around the display. It can be any kind of box, as long as the
%  contents of the box is situated on the baseline like in a
%  \fbox{\cs{fbox}}.
%  \begin{verbatim}
%  \begin{empheq}[box=\fbox]{align}
%    a&=b\\
%    E&=mc^2 + \int_a^a x\, dx
%  \end{empheq}
%  \end{verbatim}
%  \begin{empheq}[box=\fbox]{align}
%    a&=b\\
%    E&=mc^2 + \int_a^a x\, dx
%  \end{empheq}
%  The key \key{outerbox} is an alias for \key{box} and is added for
%  naming consistency with the key described below.
%
%  \begin{codesyntax}
%    \SpecialKeyIndex{innerbox}\key{innerbox}
%  \end{codesyntax}
%  There is also an \key{innerbox} key. It is not very interesting
%  unless you use one of the other keys.
%  \begin{verbatim}
%  \begin{empheq}[innerbox=\fbox,
%                 left=L\Rightarrow]{align}
%    a&=b\\
%    E&=mc^2 + \int_a^a x\, dx
%  \end{empheq}
%  \end{verbatim}
%  \begin{empheq}[innerbox=\fbox,
%                 left=L\Rightarrow]{align}
%    a&=b\\
%    E&=mc^2 + \int_a^a x\, dx
%  \end{empheq}
%
%  \begin{codesyntax}
%    \SpecialKeyIndex{marginbox}\key{marginbox}
%  \end{codesyntax}
%  \FeatureRequest{Uwe Siart}{2003/12/08}
%  The last key is the \key{marginbox} key. If you typeset your math
%  in \mode{fleqn} mode\footnote{Sorry, but I can't show you because
%  this manual features centered math displays.} you may want the
%  math display and not the outer box to align at the left margin (or
%  rather: the indentation). If you make sure the contents of the
%  outer box is centered horizontally inside the box, \key{marginbox}
%  will align it properly. You shouldn't set both \key{box} and
%  \key{marginbox} at the same time, as this is surely not what you
%  want and the package will silently use the last one in the list.
%
%
%  \begin{quote}
%  \textbf{Warning}: \pkg{keyval} treats commas and equal signs as
%  separator and assignment signs which in turn means that if you
%  want to typeset either of them, they \emph{must}---and believe me,
%  it's really important---be enclosed in braces. In short:
%  \begin{description}
%    \item[{\makebox[3em][l]{Good}}] \texttt{[left=\{(A,B)=(1,0)\}]}
%    \item[{\makebox[3em][l]{Bad}}]  \texttt{[left=(A,B)=(1,0)]}
%  \end{description}
%  \end{quote}
%
%  \begin{codesyntax}
%    \SpecialUsageIndex{\empheqset}\cs{empheqset}\marg{markup instructions}
%  \end{codesyntax}
%  \FeatureRequest{Uwe Siart}{2004/01/13}
%  You can also set the keys globally\footnote{Or rather: outside the
%  scope of the \env{empheq} environment. The settings will still
%  obey scoping rules.} with the command
%  \cs{empheqset}. This means that
%  \begin{verbatim}
%    \empheqset{marginbox=\psframebox}
%  \end{verbatim}
%  will force a |box=\psframebox| in all occurrences of the
%  \env{empheq} environment, but an explicitly given \key{box} or
%  \key{marginbox} will override this setting. You can only use the
%  keys listed in this section as arguments to \cs{empheqset}; not
%  the \AmS\ environments.
%
%
%
%  \subsection{What won't work in the \env{empheq} environment}
%
%  \begin{codesyntax}
%    \SpecialUsageIndex{intertext}\cs{intertext}\marg{text}\\
%    \SpecialUsageIndex{displaybreak}\cs{displaybreak}\oarg{num}
%  \end{codesyntax}
%  Now we've seen some of the things that work in the \env{empheq}
%  environment but we also have to take note of what won't work. As
%  this package provides a way to box multi line math displays it can
%  come as no surprise that using either \cs{intertext} or
%  \cs{displaybreak} inside the \env{empheq} environment makes no
%  sense. Should you however happen to try them anyway, you'll
%  experience that \cs{intertext} issues an error message and
%  \cs{displaybreak} issues a warning. The reason only \cs{intertext}
%  gives an error message is that you'll get output very different
%  from what you expect, and that is not the case (so much at least)
%  with \cs{displaybreak}. But of course, you'll never see either
%  message because you read the manual!
%
%
%
%  \subsection{Special delimiters}
%
%  As you've seen a few pages back, it's possible to add material on
%  both sides of the math display. When doing so you often need a
%  delimiter that scales to fit the entire display, so it comes as no
%  surprise that \pkg{empheq} provides such delimiters.
%  \begin{codesyntax}
%    \cs{empheql}\meta{delim_name}  \cs{empheqbigl}\meta{delim_name}\\
%    \cs{empheqr}\meta{delim_name}  \cs{empheqbigr}\meta{delim_name}
%  \end{codesyntax}
%  For example one might need a large brace like the one in the
%  \env{cases} environment:
%  \begin{verbatim}
%  \begin{empheq}[left=\empheqlbrace, right=\empheqrbrace]{align}
%    E&=mc^2 \\
%    Y&= \sum_{n=1}^\infty \frac{1}{n^2}
%  \end{empheq}
%  \end{verbatim}
%  \begin{empheq}[left=\empheqlbrace, right=\empheqrbrace]{align}
%    E&=mc^2 \\
%    Y&= \sum_{n=1}^\infty \frac{1}{n^2}
%  \end{empheq}
%  The naming scheme is \cs{empheq}\meta{delimiter name} and
%  \cs{empheqbig}\meta{delimiter name}. Thus \cs{empheqrbrace}
%  produces an auto-scaling right brace with same size as the math
%  display while \cs{empheqbigrbrace} produces an even bigger version
%  that spans the inner box plus the math display inside it. For a
%  complete list of supported delimiters see Table
%  \vref{tab:delimiters}.
%
%  \begin{table}
%  \centering
%  \begin{tabular}{lllc} \toprule
%  \multicolumn{3}{c}{\textbf{Delimiter}}\\ \cmidrule{1-3}
%  \multicolumn{1}{c}{Original} & \multicolumn{1}{c}{Normal}  &
%    \multicolumn{1}{c}{Bigger} & Symbol\\ \midrule
%  \DescribeDelimiter{lbrace} \DescribeDelimiter{rbrace}
%  \DescribeDelimiter{lbrack} \DescribeDelimiter{rbrack} \midrule
%  \DescribeDelimiter{langle} \DescribeDelimiter{rangle}
%  \DescribeDelimiter{lparen} \DescribeDelimiter{rparen} \midrule
%  \DescribeDelimiter{lvert} \DescribeDelimiter{rvert}
%  \DescribeDelimiter{lVert} \DescribeDelimiter{rVert} \midrule
%  \DescribeDelimiter{lfloor} \DescribeDelimiter{rfloor}
%  \DescribeDelimiter{lceil} \DescribeDelimiter{rceil} \bottomrule
%  \end{tabular}
%  \caption{The supported auto-scaling delimiters in
%  \pkg{empheq}}\label{tab:delimiters}
%  \end{table}
%
%
%
%  \section{Taking things a little further}
%
%  So far we have covered the basic functions of \pkg{empheq}: the
%  markup instructions and the predefined auto-scaling delimiters.
%  There is more to \pkg{empheq} however, and we'll cover that in
%  this section.
%
%  \subsection{Loading the package}
%
%  The package has two main requirements: One is the
%  \pkg{mathtools}\footnote{By the same author and is distributed
%  with \pkg{empheq}. See its documentation for more information.}
%  package which provides \pkg{empheq} with a lot of necessary tools
%  for doing its thing. The other one is \pkg{amsmath} of course, and
%  to make the loading procedure as easy as possible, you can simply
%  substitute
%  \begin{verbatim}
%  \usepackage[leqno,fleqn,intlimits]{amsmath}
%  \end{verbatim}
%  with
%  \begin{verbatim}
%  \usepackage[leqno,fleqn,intlimits]{empheq}
%  \end{verbatim}
%  \pkg{empheq} makes sure that the \pkg{amsmath} options are passed
%  on and loaded by \pkg{amsmath}. The same goes for the options
%  provided by \pkg{mathtools}.
%
%
%  \subsubsection{Compatibility with the old version}
%
%  The current version (\fileversion) of \pkg{empheq} is incompatible with
%  versions prior to and including~0.7e. If you have documents produced
%  with versions prior to~0.7e, then you have to load the package
%  \pkg{empheq07} instead. \pkg{empheq07} now exists as a separate package
%  package with its own documentation, but I strongly recommend switching
%  to the new version because it is so much better.
%
%  \subsection{Package options}
%
%  In addition to the \pkg{amsmath} options, \pkg{empheq} itself
%  provides a string of options listed in Table \vref{tab:options}.
%
%  \begin{table}[ht]
%    \centering
%    \begin{tabular}{lp{21em}} \toprule
%      Option  & Short description \tabularnewline \midrule
%      \TabDescribeOption[]{overload}
%      {Lets you use the \pkg{empheq} visual markup extensions
%      transparently in an existing document.}
%      \TabDescribeOption[]{overload2}
%      {A wild version of \opt{overload}. Use with care.}
%      \TabDescribeOption[]{ntheorem}
%      {A support module for users of \pkg{ntheorem} and its
%      \opt{thmmarks} and \opt{thref} options.}
%      \TabDescribeOption[]{newmultline}
%      {With v2.10 of \pkg{empheq} the syntax for \env{multline} and
%      \env{multlined} has been improved immensely.}
%      \TabDescribeOption[]{oldmultline}
%      {Support for the somewhat strange syntax for \env{multline} and
%      \env{multlined} in v2.00.}
%      \bottomrule
%    \end{tabular}
%    \caption{Supported options in the \pkg{empheq} package}
%    \label{tab:options}
%  \end{table}
%
%
%  \begin{codesyntax}
%    \SpecialOptIndex{ntheorem}\opt{ntheorem}
%  \end{codesyntax}
%  The \pkg{ntheorem} package is supported by means of the
%  \opt{ntheorem} option. In order for this to work properly, an
%  \pkg{amsmath} bug\footnote{See the \LaTeX{} Bugs Database
%  \url{http://www.latex-project.org/cgi-bin/ltxbugs2html} under
%  \AmS\LaTeX{} problem report 3624.} is fixed. For more information
%  on this option see \S \vref{sec:ntheorem-support}.
%
%  \subsubsection{Using \env{multline}}
%
%  \begin{codesyntax}
%    \SpecialOptIndex{newmultline}\opt{newmultline}\\
%    \SpecialOptIndex{oldmultline}\opt{oldmultline}
%  \end{codesyntax}
%  Those familiar with v2.00 of this package will recognize the
%  somewhat weird syntax for using \env{multline} and
%  \env{multlined}. The good news is that since v2.10 there is now a
%  much improved syntax available. The option \opt{newmultline}
%  (default) selects this better interface, while the old interface
%  is provided for compatibility reasons by the \opt{oldmultline}
%  option.
%
%
%  \begin{codesyntax}
%    \SpecialUsageIndex{\shoveleft}\cs{shoveleft}\oarg{dimen}\marg{arg}\\
%    \SpecialUsageIndex{\shoveright}\cs{shoveright}\oarg{dimen}\marg{arg}
%  \end{codesyntax}
%  With the new interface you also get an extended syntax for
%  \cs{shoveleft} and \cs{shoveright} as shown in the example below.
%  \begin{verbatim}
%  \begin{empheq}{multline}
%    \framebox[.65\columnwidth]{First line} \\
%    \framebox[.5\columnwidth]{Second line} \\
%    \shoveleft{L+E+F+T}         \\
%    \shoveright{R+I+G+H+T}         \\
%    \shoveleft[1cm]{L+E+F+T}         \\
%    \shoveright[\widthof{$R+I+G+H+T$}]{R+I+G+H+T}         \\
%    \framebox[.65\columnwidth]{Last line}
%  \end{empheq}
%  \end{verbatim}
%  \begin{empheq}{multline}
%    \framebox[.65\columnwidth]{First line} \\
%    \framebox[.5\columnwidth]{Second line} \\
%    \shoveleft{L+E+F+T}         \\
%    \shoveright{R+I+G+H+T}         \\
%    \shoveleft[1cm]{L+E+F+T}         \\
%    \shoveright[\widthof{$R+I+G+H+T$}]{R+I+G+H+T}         \\
%    \framebox[.65\columnwidth]{Last line}
%  \end{empheq}
%
%
%  There are however a few differences in the output between the
%  original \pkg{amsmath} version of \env{multline} and the one
%  \pkg{empheq} provides. In \pkg{amsmath} a centered line in
%  \env{multline} is centered on the page without taking into account
%  the \cs{multlinegap}, \cs{multlinetaggap}, and the tag width. Thus
%  \pkg{amsmath} can sometimes give you horrible output without
%  giving you any warning as shown below.
%  \begin{verbatim}
%  \begin{multline}
%    \framebox[.65\columnwidth]{First line} \\
%    \framebox[.9\columnwidth]{Loooong line} \\
%    \framebox[.65\columnwidth]{Last line} \tag{wide tag}
%  \end{multline}
%  \end{verbatim}
%  \begin{AmSmultline}
%    \framebox[.65\columnwidth]{First line} \\
%    \framebox[.9\columnwidth]{Loooong line} \\
%    \framebox[.65\columnwidth]{Last line} \tag{wide tag}
%  \end{AmSmultline}
%  In \pkg{empheq} these parameters are taken into account, so the
%  same input will produce this in \pkg{empheq}:
%  \begin{verbatim}
%  \begin{empheq}{multline}
%    \framebox[.65\columnwidth]{First line} \\
%    \framebox[.9\columnwidth]{Loooong line} \\
%    \framebox[.65\columnwidth]{Last line} \tag{wide tag}
%  \end{empheq}
%  \end{verbatim}
%  \begin{multline*}
%    \framebox[.65\columnwidth]{First line} \\
%    \framebox[.9\columnwidth]{Loooong line} \\
%    \framebox[.65\columnwidth]{Last line} \tag{wide tag}
%  \end{multline*}
%  This results in an
%  \begin{verbatim}
%    Overfull \hbox (30.03783pt too wide) in paragraph ...
%  \end{verbatim}
%  message in the log file, indicating a visual problem. I think this
%  behavior is more sensible than the one the original \env{multline}
%  environment provides.
%
%  \begin{codesyntax}
%    \SpecialEnvIndex{MTmultlined}\cs{begin}\arg{MTmultlined}\oarg{pos}\oarg{width}
%        \meta{contents} \cs{end}\arg{MTmultlined}
%  \end{codesyntax}
%  When you choose the \opt{oldmultline} option you may still want to
%  use the much improved \env{multlined} environment defined in
%  \pkg{mathtools} but unfortunately there will be a name clash.
%  Instead you can access it under the name \env{MTmultlined}.
%
%
%
%  \subsubsection{The \opt{overload} option}\label{sec:overload}
%
%  \begin{codesyntax}
%    \SpecialOptIndex{overload}\opt{overload}\\
%    \SpecialOptIndex{overload2}\opt{overload2}
%  \end{codesyntax}
%  \FeatureRequest{Lars Madsen}{2004/03/25}
%  The \opt{overload} option redefines the original \AmS{}
%  environments so that they take an optional argument.
%  \begin{codesyntax}
%    \cs{begin}\marg{\AmS{} env_name}\oarg{markup instructions}\\
%    \meta{contents of \AmS{} environment}\\
%    \cs{end}\marg{\AmS{} env_name}
%  \end{codesyntax}
%  For example
%  \begin{verbatim}
%    \begin{gather}[box=\widefbox]
%      a = b
%    \end{gather}
%  \end{verbatim}
%  is then actually short for
%  \begin{verbatim}
%    \begin{empheq}[box=\widefbox]{gather}
%      a = b
%    \end{empheq}
%  \end{verbatim}
%  All the \AmS{} environments are supported by this option except
%  for the pseudo environment |\[ ... \]| (it's not really an
%  environment), \footnote{With the option \opt{oldmultline} the
%  \env{multline} and \env{multline*} environments aren't supported
%  either because their syntax in \pkg{empheq} then differ from their
%  syntax in \pkg{amsmath}.} because a)~you don't really need markup
%  instructions for a one line display, and b)~would you really like
%  a syntax like
%  \begin{verbatim}
%    \[[box=\fbox] a=b \]
%  \end{verbatim}
%  where its difficult to see whether or not there is a typo?
%  Choosing \opt{overload} is meant for people who don't want to
%  change their entire document into \pkg{empheq} syntax. I have no
%  problems with that; just be careful when you fiddle with
%  \cs{empheqset} as it will affect \emph{all} math displays!
%
%
%  There is of course a catch (well, two actually): These redefined
%  environments don't run as fast as the regular ones (about three
%  times as slow), but in this day and age I seriously doubt you'll
%  be able to tell the difference unless you have a vast number of
%  equations. The other catch is that you cannot use \cs{intertext}
%  and \cs{displaybreak} as described earlier. If you find yourself
%  wanting to use one of these features in say, an \env{align}
%  environment, you have to use the original \env{align} environment.
%  Luckily it is available if you call it like this:
%  \begin{codesyntax}
%    \cs{begin}\arg{AmS\meta{\rmfamily \AmS{} env_name}} \\
%    \texttt{~~}\meta{contents of \AmS{} environment}\\
%    \cs{end}\arg{AmS\meta{\rmfamily \AmS{} env_name}}
%  \end{codesyntax}
%  For example the original \env{align} environment could be selected
%  with
%  \begin{verbatim}
%    \begin{AmSalign}
%      ...
%    \end{AmSalign}
%  \end{verbatim}
%  These original versions with prefix \texttt{AmS} exist for all the
%  environments.
%
%
%  The option \opt{overload2} activates the overloading feature
%  for the pseudo environment |\[ ... \]|, although I doubt you'll
%  find it useful. Beware that this definition is fragile unless you
%  have \eTeX{} as \LaTeX{} engine.\footnote{\eTeX{} has been the
%  default engine for \LaTeX{} in most major distributions since
%  2003.} Not surprisingly \opt{overload2} activates
%  \opt{overload}.
%
%
%
%  Before you get all excited about these options, you should take
%  note of some aspects of centered math displays in \pkg{amsmath}.
%  In certain circumstances truly centering the display is not always
%  the best solution as in this example:
%  \begin{center}
%    \mbox{}\hfill
%    \framebox[.75\linewidth]{Wide math display; not adjusted}%
%    \hfill\mbox{}%
%    \makebox[0pt][r]{Wide Tag}
%  \end{center}
%  Instead centering the display in the available space works pretty
%  well:
%  \begin{gather*}
%    \framebox[.75\linewidth]{Wide math display; adjusted}%
%    \tag*{Wide Tag}
%  \end{gather*}
%  This is what \pkg{amsmath} does normally, but in the \env{gather}
%  environment it does it on a per line basis. This means that we get
%  results like
%  \begin{AmSgather*}
%    \framebox[.75\linewidth]{Wide math display; adjusted}%
%    \tag*{Wide Tag} \\
%    \framebox[.75\linewidth]{Wide math display; not adjusted} \tag{i}%
%  \end{AmSgather*}
%  whether we like it or not. Inside the \env{empheq} environment we
%  need to have the same adjustment for all lines else the boxing
%  process will not work properly, so when activating the
%  \opt{overload} option the above two-line \env{gather} will
%  instead look like this:
%  \begin{empheq}{gather*}
%    \framebox[.75\linewidth]{Wide math display; adjusted}%
%    \tag*{Wide Tag} \\
%    \framebox[.75\linewidth]{Wide math display; adjusted} \tag*{i}%
%  \end{empheq}
%  I leave it to you to choose whether or not this is better (I think
%  it is better).
%
%
%
%  \subsection{A note on boxed displays}
%
%  When browsing a 400+ pages textbook with at least twice the number
%  of displayed formulae, some of them are surely more important than
%  others. Thus the author of the book (in cooperation with the
%  designer) should make sure that such formulae are easily found
%  again i.e., they should be easily distinguishable from the rest of
%  the pack. One way of doing this is putting the formula into a box
%  which is something we've seen \pkg{empheq} being capable of. There
%  are however a few things to keep in mind:
%  \begin{itemize}
%    \item Don't overdo it. If you do it on half of them there's no
%    point in doing it at all. I don't see much reason for applying
%    this technique to more than 10\% of the formulae.
%    \item Choose the type of box carefully. You want to draw
%    attention to it so it might as well look good.
%  \end{itemize}
%  The latter point can be illustrated by defining a macro similar to
%  \cs{fbox} only with a little more space to the right and left of
%  the argument.
%  \begin{verbatim}
%  \newcommand*\widefbox[1]{\fbox{\hspace{1em}#1\hspace{1em}}}
%  \end{verbatim}
%  If we replace the \cs{fbox} from the \key{box} example before, we
%  get this display:
%  \begin{verbatim}
%  \begin{empheq}[box=\widefbox]{align}
%    a&=b\\
%    E&=mc^2 + \int_a^a x\, dx
%  \end{empheq}
%  \end{verbatim}
%  \begin{empheq}[box=\widefbox]{align}
%    a&=b\\
%    E&=mc^2 + \int_a^a x\, dx
%  \end{empheq}
%  Compare it with
%  \begin{empheq}[box=\fbox]{align}
%    a&=b\\
%    E&=mc^2 + \int_a^a x\, dx
%  \end{empheq}
%  and see which one you prefer.
%
%  Similarly one might be tempted to use colored boxes:
%  \begin{verbatim}
%  \definecolor{myblue}{rgb}{.8, .8, 1}
%  \newcommand*\mybluebox[1]{%
%    \colorbox{myblue}{\hspace{1em}#1\hspace{1em}}}
%  \end{verbatim}
%  We know the drill by now.
%  \begin{verbatim}
%  \begin{empheq}[box=\mybluebox]{align}
%    a&=b\\
%    E&=mc^2 + \int_a^a x\, dx
%  \end{empheq}
%  \end{verbatim}
%  \begin{empheq}[box=\mybluebox]{align}
%    a&=b\\
%    E&=mc^2 + \int_a^a x\, dx
%  \end{empheq}
%  There is more on boxes later in this manual. If you're into the
%  \pkg{fancybox} package then remember to read \S
%  \vref{sec:fancybox_support}.
%
%  \section{Support for other packages}
%
%  With the multitude of packages for \LaTeX, it is not always easy
%  to be a PWWO package (``Plays Well With Others''), but
%  \pkg{empheq} tries really hard to do so. This section lists the
%  packages where \pkg{empheq} has to provide workarounds and they
%  can be divided into two categories.
%  \begin{description}
%    \item[Compatibility] Some packages affect \LaTeX' labelling
%    mechanism and since \pkg{empheq} internally has to turn off
%    labels and related code temporarily, hooks must be provided for
%    these packages. Examples of such packages are \pkg{hyperref} and
%    \pkg{showkeys}.
%
%    \item[Enhancements] Other packages provide useful features that
%    for some reason may not work directly or optimally with
%    \pkg{empheq}. In these cases the problematic commands are
%    redefined so that they not only work with \pkg{empheq}, but
%    also give the same if not better output. An example of this
%    would be the \cs{shadowbox} command from \pkg{fancybox}
%    (described below).
%  \end{description}
%  %
%  \subsection{Support for \pkg{fancybox}}\label{sec:fancybox_support}
%
%  The \pkg{fancybox} package provides various boxes and you can use
%  them with \pkg{empheq} as well. Here's \cs{ovalbox}:
%  \begin{verbatim}
%  \begin{empheq}[box=\ovalbox]{align}
%      E&=mc^2 \\
%      Y&= \sum_{n=1}^\infty \frac{1}{n^2}
%  \end{empheq}
%  \end{verbatim}
%  \begin{empheq}[box=\ovalbox]{align}
%      E&=mc^2 \\
%      Y&= \sum_{n=1}^\infty \frac{1}{n^2}
%  \end{empheq}
%
%
%  \begin{codesyntax}
%  \SpecialUsageIndex{\shadowbox}\SpecialUsageIndex{\shadowbox*}
%    \cs{shadowbox*}\marg{arg} \\
%  \SpecialColorIndex{shadowcolor} \texttt{shadowcolor}
%  \end{codesyntax}
%  The only problem with using the \pkg{fancybox} boxes in
%  conjunction with the \env{empheq} environment is \cs{shadowbox},
%  as this macro typesets the shadow on the baseline and for the
%  equation numbers to be placed correctly, the box command must set
%  its argument on the baseline. Thus the normal \cs{shadowbox}
%  produces \shadowbox{\meta{arg}}, but we need this instead:
%  \shadowbox*{\meta{arg}}. Therefore \pkg{empheq} will detect if
%  \pkg{fancybox} is loaded and in that case it'll enhance
%  \cs{shadowbox} in two ways:
%  \begin{itemize}
%    \item It defines a starred version \cs{shadowbox*} which
%    typesets its argument on the baseline.
%
%    \item The color \texttt{shadowcolor} is introduced. The default
%    color is black.
%  \end{itemize}
%  \begin{verbatim}
%  \definecolor{shadowcolor}{rgb}{0,.5,.5}
%  \setlength\shadowsize{2pt}
%  Line of text for \shadowbox*{testing of \Verb|f#&%\| verbatim} and
%  showing the shadow color.
%  \end{verbatim}
%  \begin{quote}
%  \definecolor{shadowcolor}{rgb}{0,.5,.5}
%  \setlength\shadowsize{2pt}
%  Line of text for \shadowbox*{testing of \Verb|f#&%\| verbatim} and
%  showing the shadow color.
%  \end{quote}
%
%  The point is that if you want a \cs{shadowbox} around your math
%  display then you must use the starred version:
%  \begin{verbatim}
%  \begin{empheq}[box=\shadowbox*]{align}
%      E&=mc^2 \\
%      Y&= \sum_{n=1}^\infty \frac{1}{n^2}
%  \end{empheq}
%  \end{verbatim}
%  \begin{empheq}[box=\shadowbox*]{align}
%      E&=mc^2 \\
%      Y&= \sum_{n=1}^\infty \frac{1}{n^2}
%  \end{empheq}
%
%  See \S \vref{stunning} if you want
%  to make a fancy box yourself.
%
%
%  \subsection{Support for \pkg{ntheorem}}\label{sec:ntheorem-support}
%
% \changes{v2.13}{2012/05/09}{Added caveat}
%  \textbf{Caveat:} Due to an unfortunate interaction between \pkg{empheq} and
%  \pkg{ntheorem}, users may want to add
% \begin{verbatim}
% \usetagform{default}
% \end{verbatim}
%  \emph{after} loading \pkg{ntheorem} otherwise the tags may be
%  placed wrong and any labels within the \env{empheq} environment may
%  be lost. In the future \pkg{ntheorem} should be able to test for
%  this and add \verb|\usetagform{default}| automatically.
%
%  Users who use \verb|\usetagform| to get another tag design, should
%  make sure to postpone this configuration until \emph{after} loading
%  \pkg{ntheorem}. 
%
% \bigskip
%
%  \changes{v2.00}{2004/04/14}{Added description of the
%  \opt{ntheorem} option}
%  \changes{v1.05}{2004/02/16}{Added support for the end-of-theorem
%  marker mechanism of \pkg{ntheorem}}
%
%  \begin{codesyntax}
%    \SpecialOptIndex{ntheorem}\opt{ntheorem}
%  \end{codesyntax}
%  The \pkg{ntheorem} package is supported by the means of the
%  \opt{ntheorem} option of \pkg{empheq}. This loads a set of
%  extra macros which fixes various compatibility problems between
%  \pkg{ntheorem} and \pkg{amsmath} and furthermore introduces
%  special (internal) macros for optimum positioning of
%  end-of-theorem markers, while retaining a user friendly interface.
%
%
%  When you want to use the automatic end-of-theorem marker mechanism
%  from \pkg{ntheorem} you sometimes run into problems as you would
%  want the marker to be placed aligned at the bottom of the math
%  display but still keeping the tags in their proper place. In
%  \mode{leqno} mode this is not that much of a problem as the tags
%  and the marker are set on either side of the math display like
%  in Figure \vref{fig:markerI}.
%  \changes{v1.05a}{2004/02/17}{Added explanatory drawings for
%  \pkg{ntheorem}}
%
%  \begin{figure}[btp]\centering
%  \unitlength2.5pt \thicklines
%  \begin{picture}(90,30)(-30,40)
%    \put(0,50){\framebox(25,20){Math Height}}
%    \put(0,40){\framebox(25,10){Math Depth}}
%    \put(-30,50){\framebox(20,10){Tag height}}
%    \put(-30,45){\framebox(20,5){Tag depth}}
%    \put(40,40){\framebox(20,10){Marker}}
%    \multiput(26,40)(2,0){7}{\line(1,0){1}}
%  \end{picture}
%  \caption{Marker placement in \mode{leqno} mode}
%  \label{fig:markerI}
%  \end{figure}
%
%  \begin{codesyntax}
%    \SpecialUsageIndex{\mintagvsep}\cs{mintagvsep}
%  \end{codesyntax}
%  Unfortunately things are not this easy in \mode{reqno} mode. There
%  are two possible situations as shown in Figure \vref{fig:markerII}.
%  \begin{figure}[btp]
%  \unitlength2.5pt \thicklines
%  \begin{picture}(60,45)(0,25)
%    \put(0,50){\framebox(25,20){Math Height}}
%    \put(0,25){\framebox(25,25){Math Depth}}
%    \put(40,50){\framebox(20,10){Tag height}}
%    \put(40,45){\framebox(20,5){Tag depth}}
%    \put(40,25){\framebox(20,10){Marker}}
%    \put(50,35){\vector(0,1){10}}
%    \put(50,45){\vector(0,-1){10}}
%    \put(50,39){\ \ $>\delta$}
%    \multiput(26,25)(2,0){7}{\line(1,0){1}}
%  \end{picture} \hfill\vrule\hfill
%  \begin{picture}(60,45)(0,25)
%    \put(0,50){\framebox(25,20){Math Height}}
%    \put(0,40){\framebox(25,10){Math Depth}}
%    \put(40,50){\framebox(20,10){Tag height}}
%    \put(40,45){\framebox(20,5){Tag depth}}
%    \put(40,28){\framebox(20,10){Marker}}
%    \put(50,38){\vector(0,1){7}}
%    \put(50,45){\vector(0,-1){7}}
%    \put(50,40){\ \ $=\delta$}
%  \end{picture}
%  \caption{The two possible situations for end marks in \mode{reqno}
%  mode}
%  \label{fig:markerII}
%  \end{figure}
%  If possible we want the bottom of the marker to be aligned with
%  the bottom of the math display, but at the same time we want to
%  ensure a minimum vertical separation $\delta$ between marker and
%  tag. In \pkg{empheq} this is controlled by the length parameter
%  \cs{mintagvsep} which by default is 5\,pt.
%
%  The good news is that this is where \pkg{empheq} sets in.
%  Basically all you need to do is to use the \env{empheq}
%  environment to type set your equations inside the theorem
%  environments:
%  \begin{verbatim}
%    \begin{Theorem}
%    Some text at first and then a math display
%    \begin{empheq}{align}
%      a&=b\\
%      E&=mc^2 + \int_a^a x\, dx
%    \end{empheq}
%    \end{Theorem}
%  \end{verbatim}
%  Then the tag placement and the end-of-theorem marker will be set
%  properly (after a couple of runs as usual).
%
%  To ensure the correct outcome remember to load the packages like
%  this:
%  \begin{verbatim}
%    \usepackage[ntheorem]{empheq} % this loads amsmath as well
%    \usepackage[thmmarks,amsmath]{ntheorem}
%  \end{verbatim}
%  Remember that if you use the \opt{overload} option you
%  can just use the regular math environments to get the desired
%  result with the end mark. The exception is the |\[ ... \]|
%  environment which will only work if you use the option
%  \opt{overload2}.
%
%  \clearpage
%
%  \EmphEqpart{Advanced user's guide}
%
%  The \pkg{empheq} package has more to offer than you have seen, but
%  I found some of the functionality so dangerous it was best to hide
%  it for a little while. The commands you have encountered so far in
%  this manual all have one thing in common: They have only lower
%  case letters in their names. Now it's time to reveal those that
%  have \emph{mixed-case} names, thus implying that you should take
%  great care when using them.
%
%  \section{Delimiters revisited}
%  Let's go back to delimiters, shall we?
%
%  \subsection{Creating your own delimiters}
%  As a convenience for the user, this interface is extended in a
%  more general way so that it is also possible to declare delimiters
%  with the following two commands: \SpecialUsageIndex{\DeclareLeftDelimiter}
%  \SpecialUsageIndex{\DeclareRightDelimiter}
%  \begin{codesyntax}
%    \cs{DeclareLeftDelimiter}\oarg{space adjustment}\marg{delimiter}\\
%    \cs{DeclareRightDelimiter}\oarg{space adjustment}\marg{delimiter}
%  \end{codesyntax}
%  While \pkg{empheq} provides auto-scaling versions of the most
%  common delimiters, you may sometimes want some new ones. Say for
%  instance you have loaded the \pkg{stmaryrd} package and you want
%  to use the double bracket commands \cs{llbracket} and
%  \cs{rrbracket} with \pkg{empheq}. Then you simply do this:
%  \begin{verbatim}
%    \DeclareLeftDelimiter{\llbracket}
%    \DeclareRightDelimiter{\rrbracket}
%  \end{verbatim}
%  This defines the new delimiters \cs{empheqllbracket},
%  \cs{empheqbigllbracket}, \cs{empheqrrbracket} and
%  \cs{empheqbigrrbracket}.
%
%  You can use \cs{big...} delimiters as well if you don't like the
%  automatic scaling. There are however ways to fine-tune if you
%  really want it.
%
%  \subsection{Fine-tuning of delimiters}
%
%  \TeX{} provides two primitives to control the scaling of
%  delimiters produced with \cs{left} and \cs{right}, namely the
%  dimension \cs{delimitershortfall} (denoted by $\delta$ here) and
%  the integer \cs{delimiterfactor} ($f$). The idea is that the
%  sub-formula inside the \cs{left}-\cs{right} pair is to be
%  vertically centered and given its height $h_1$ and its depth $h_2$
%  we want to produce a delimiter with total height $h$, where  $h =
%  2 \max(h_1,h_2)$. \TeX' rules on this is that the minimum
%  delimiter size $h_{\min}$ must meet the requirements
%  \[
%    h_{\min}\geq h \frac{f}{1000} \quad\land \quad h_{\min}\geq h-\delta
%  \]
%  \LaTeX{} sets $\cs{delimitershortfall} = \text{\the\delimitershortfall} $
%  and $\cs{delimiterfactor} = \the\delimiterfactor$, but in our case
%  we will almost always want a delimiter that spans the entire
%  \cs{left}-\cs{right} pair (the math display), thus a change of
%  these settings is needed.
%
%  \begin{codesyntax}
%    \SpecialUsageIndex{\EmphEqdelimitershortfall}
%    \cs{EmphEqdelimitershortfall}\\
%    \SpecialUsageIndex{\EmphEqdelimiterfactor}
%    \cs{EmphEqdelimiterfactor}
%  \end{codesyntax}
%  However it is a bad idea to just change these two settings without
%  thinking of the effect it'll have on the rest of the mathematics
%  in your document. Therefore \pkg{empheq} provides the parameters
%  \cs{EmphEqdelimitershortfall} (default setting is
%  \the\EmphEqdelimitershortfall) and \cs{EmphEqdelimiterfactor}
%  (default is \the\EmphEqdelimiterfactor) to cater for this.
%
%
%  \subsection{Scaling material yourself}
%
%  The attentive reader may have noticed that I still haven't
%  revealed how I managed to get the right size of that big
%  exclamation mark in the first example of this manual, so I guess
%  it's about time.
%  \begin{codesyntax}
%    \SpecialUsageIndex{\EmphEqdisplayheight}\cs{EmphEqdisplayheight} \\
%    \SpecialUsageIndex{\EmphEqdisplaydepth}\cs{EmphEqdisplaydepth}
%  \end{codesyntax}
%  The height of the math display plus the surrounding inner box is
%  given by \cs{EmphEqdisplayheight} and the depth by
%  \cs{EmphEqdisplaydepth}. These can be used when constructing
%  something that should scale to fit the display and can't be done
%  with ``auto-scaling'' commands like \cs{vrule} or \cs{left} or
%  \cs{right}. There is a catch however: If you do any horizontal
%  resizing of the material you want to scale, then you must specify
%  the width \emph{manually}:\footnote{The
%  \cs{fontdimen22}\cs{textfont2} in the \meta{length} argument of
%  \cs{raisebox} is the \emph{math axis} of the font. It is needed in
%  order to get the vertical positioning of the oversized exclamation
%  mark just right.}
%  \begin{verbatim}
%  \begin{empheq}[
%    box=\colorbox{myblue},
%    right={\;
%           \makebox[.9em]{%
%            $\raisebox{-.5\totalheight+\fontdimen22\textfont2}
%              {\resizebox{!}{%
%                \EmphEqdisplayheight+\EmphEqdisplaydepth}{!}}$%
%           }},
%    left={X=Y\Rightarrow}
%     ]{alignat=3}
%    A_1&=b_1 & \qquad c_1&=d_1 & \qquad e_1&= f_1
%         \tag*{\_A\raisebox{1ex}{silly}\raisebox{-1ex}{tag}\_}\\
%    A_2&=b_2 & \qquad c_2&=d_2 & \qquad  e_2&= f_2 \\
%    A_3&=b_3 & \qquad c_3&=d_3 & \qquad  e_3&= f_3
%  \end{empheq}
%  \end{verbatim}
%
%  Here's another example. We want to be able to put a \cs{parbox}
%  with some descriptive text top-aligned on the side of the display.
%  \begin{verbatim}
%  \begin{empheq}[
%    left={\parbox[c][\EmphEqdisplayheight+\EmphEqdisplaydepth][t]
%           {4.5cm}
%           {You may find this kind of description useful.}\enspace}]
%      {align}
%    a&=\int_0^1 x\,dx +\frac{foo + bar}{baz}\\
%    E&= mc^2
%  \end{empheq}
%  \end{verbatim}
%  \begin{empheq}[
%    left={\parbox[c][\EmphEqdisplayheight+\EmphEqdisplaydepth][t]
%           {4.5cm}
%           {You may find this kind of description useful.}\enspace}]
%      {align}
%    a&=\int_0^1 x\,dx +\frac{foo + bar}{baz}\\
%    E&= mc^2
%  \end{empheq}
%
%
%  \section{A few short notes}
%
%  \subsection{About $\cs{eqref}$}
%
%  Internally \env{empheq} separate the displayed math from
%  the corresponding equation numbers such that we can add special
%  delimiters or boxes. This is done by nullifying (or rather
%  conveniently redefining) an internal command called
%  \cs{maketag@@@}. Unfortunately, this component is also used by
%  \cs{eqref} to typeset a reference to an equation number, so in
%  earlier versions, \cs{eqref} would not work inside an \env{empheq}
%  environment.  
%
% In the current version this should now be working, but if you ever
% need to mess with \cs{eqref} it may be handy to know what is being
% done. In essence we do something similar to this:
% \begin{verbatim}
%  \let\empheqeqrefbase\textup
%   ... % next go inside empheq env
%  \renewcommand\eqref[1]{\empheqeqrefbase{%
%       \let\maketag@@@\EQsavedmaketag%
%       \tagform@{\ref{##1}}}}
% \end{verbatim}
% Thus if you need to alter things inside \cs{eqref} and need that to
% work within \env{empheq} as well, you may get away with redefining
% \cs{empheqeqrefbase}.
%
%
%
%  \subsection{About changes to $\cs{baselineskip}$}
%
%  Users should never mess with \cs{baselineskip} directly, it is not
%  the correct manner to get double spacing. Have a look at say the
%  \pkg{setspace} package or similar, or play with
%  \cs{baselinestretch} followed by \cs{normalsize} to initiate. 
%
%  Explanation: We use \cs{parbox} inside to place the display box and
%  the box containing the eqn numbers. To do its stuff \cs{parbox}
%  resets some internal settings, in this case, \cs{baselineskip} is
%  reset to \cs{normalbaselineskip}. So if you don't remember to reset
%  that one as well\dots
%
%
%  \section{Creating something new}
%
%  You can create your own environments and boxes to go with
%  \pkg{empheq}, but there are certain things that must be fulfilled
%  to get it to work properly.
%
%  \subsection{New \env{empheq}-like environments}
%
%  \begin{codesyntax}
%    \SpecialEnvIndex{EmphEqMainEnv}\env{EmphEqMainEnv}\\
%    \SpecialUsageIndex{\EmphEqMainEnv}\cs{EmphEqMainEnv}\\
%    \SpecialUsageIndex{\endEmphEqMainEnv}\cs{endEmphEqMainEnv}
%  \end{codesyntax}
%  Although the real work in \pkg{empheq} is done by an environment
%  (\env{EmphEqMainEnv}), you cannot use it like that yourself when
%  you want to define an \env{empheq}-like environment. Due to
%  technical reasons (see~\cite{Ams99}), you have to use a slightly
%  different syntax:
%  \begin{codesyntax}
%    \cs{newenvironment}
%      \marg{env_name} \oarg{num} \oarg{default}\\
%      \texttt{~~}\arg{\meta{other commands} \cs{EmpheqMainEnv}}\\
%      \texttt{~~}\arg{\cs{endEmpheqMainEnv}}
%  \end{codesyntax}
%  In the above environment the \meta{other commands} part must
%  contain certain things before it'll work.
%  \begin{codesyntax}
%    \SpecialKeyFamilyIndex{EmphEqEnv}\keyfam{EmphEqEnv}\\
%    \SpecialKeyFamilyIndex{EmphEqOpt}\keyfam{EmphEqOpt}
%  \end{codesyntax}
%  Before your homemade environment will work you must set at least
%  one key. The key family \keyfam{EmphEqOpt} is for all the markup
%  instructions like \key{left} and \key{box}, while
%  \keyfam{EmphEqEnv} controls the type of \AmS{} environment. You
%  must set the \keyfam{EmphEqEnv} family before \cs{EmhpEqMainEnv}
%  else you'll get an error message.
%
%  One example could be something like this:
%  \begin{verbatim}
%  \newenvironment{important}[2][]{%
%    \setkeys{EmphEqEnv}{#2}%
%    \setkeys{EmphEqOpt}{box=\mybluebox,#1}%
%    \EmphEqMainEnv}%
%    {\endEmphEqMainEnv}
%  \end{verbatim}
%  Thus
%  \begin{verbatim}
%  \begin{important}{gather}
%    a = b + c +d \\ e = f
%  \end{important}
%  \end{verbatim}
%  produces
%  \begin{important}{gather}
%    a = b + c +d \\ e =  f
%  \end{important}
%  while
%  \begin{verbatim}
%  \begin{important}[left={A=B \Rightarrow \empheqlbrace}]{alignat=2}
%    a &= b &\quad c &= d \\
%    \text{this} &= \text{that} &\quad \mathit{fish}&\neq fish
%  \end{important}
%  \end{verbatim}
%  produces
%  \begin{important}[left={A=B \Rightarrow \empheqlbrace}]{alignat=2}
%    a &= b &\quad c &= d \\
%    \text{this} &= \text{that} &\quad \mathit{fish}&\neq fish
%  \end{important}
%
%
%  \subsection{Creating fancy boxes}\label{stunning}
%
%  As a final example I will show you how to create complicated
%  displays involving (vertically) asymmetrical boxes like
%  \cs{shadowbox*}. In order to get the correct output, the contents
%  of the box must be placed on the baseline as in \fbox{\cs{fbox}}.
%  In this example we want to put a set of equations into a bright
%  yellow box and then add another box with some explanatory text at
%  the top of the yellow box making them overlap.
%
%  First we define the colors used and allocate the boxes. We could
%  probably use scratch boxes, but this is safer.
%  \begin{verbatim}
%  \definecolor{shadecolor}{cmyk}{0,0,0.41,0}
%  \definecolor{light-blue}{cmyk}{0.25,0,0,0}
%  \newsavebox{\mysaveboxM} % M for math
%  \newsavebox{\mysaveboxT} % T for text
%  \end{verbatim}
%  Save the display body in \cs{mysaveboxM} and the text argument in
%  \cs{mysaveboxT}.
%  \begin{verbatim}
%  \newcommand*\Garybox[2][Example]{%
%    \sbox{\mysaveboxM}{#2}%
%    \sbox{\mysaveboxT}{\fcolorbox{black}{light-blue}{#1}}%
%  \end{verbatim}
%  Then typeset the math display in a \cs{parbox} where we control
%  the height and save it in \cs{mysaveboxM}.
%  \begin{verbatim}
%    \sbox{\mysaveboxM}{%
%      \parbox[b][\ht\mysaveboxM+.5\ht\mysaveboxT+.5\dp\mysaveboxT][b]{%
%        \wd\mysaveboxM}{#2}%
%    }%
%  \end{verbatim}
%  We put it into the colored box with the desired width.
%  \begin{verbatim}
%    \sbox{\mysaveboxM}{%
%      \fcolorbox{black}{shadecolor}{%
%        \makebox[\linewidth-10em]{\usebox{\mysaveboxM}}%
%      }%
%    }%
%  \end{verbatim}
%  Then finally we get to the real typesetting. We just insert the
%  math display and then make a box of zero width right next to it.
%  In that box we lift the text argument and center it at the top of
%  the display.
%  \begin{verbatim}
%    \usebox{\mysaveboxM}%
%    \makebox[0pt][r]{%
%      \makebox[\wd\mysaveboxM][c]{%
%        \raisebox{\ht\mysaveboxM-0.5\ht\mysaveboxT
%                  +0.5\dp\mysaveboxT-0.5\fboxrule}{\usebox{\mysaveboxT}}%
%      }%
%    }%
%  }
%  \end{verbatim}
%  Let's see what it looks like.
%  \begin{verbatim}
%  \begin{empheq}[box=\Garybox]{align}
%    \sum \mathbf{F} &= \mathbf{0} \\
%    \sum F_{x} \,\mathbf{i} + \sum F_{y} \,\mathbf{j}
%      + \sum F_{z}  \,\mathbf{k} &= \mathbf{0} \\
%    \sum F_{x} &= 0 \\ \sum F_{y} &= 0 \\ \sum F_{z} &= 0
%  \end{empheq}
%  \end{verbatim}
%  \begin{subequations}
%  \begin{empheq}[box=\Garybox]{align}
%    \sum \mathbf{F} &= \mathbf{0} \\
%    \sum F_{x} \,\mathbf{i} + \sum F_{y} \,\mathbf{j} + \sum F_{z}
%    \,\mathbf{k} &= \mathbf{0}\\
%  \sum F_{x} &= 0 \\ \sum F_{y} &= 0 \\ \sum F_{z} &= 0
%  \end{empheq}
%  If we use the optional argument of \cs{Garybox} we have to enclose
%  the entire argument of \key{box} in braces because we're in
%  \env{empheq}'s optional argument (as described in \cite[page
%  167]{Lamp94}):
%  \begin{verbatim}
%  \begin{empheq}[box={\Garybox[Same old example again]}]{align}
%    a&=b\\
%    E&=mc^2 + \int_a^a x\, dx
%  \end{empheq}
%  \end{verbatim}
%  \begin{empheq}[box={\Garybox[Same old example again]}]{align}
%    a&=b\\
%    E&=mc^2 + \int_a^a x\, dx
%  \end{empheq}
%  \end{subequations}
%
%
%
%
%  \section{Contact information}
%
%  Should you have any feature request, suggestions, or bug reports
%  then feel free to contact Lars Madsen at
%  \begin{quote}
%    \email{daleif@imf.au.dk}
%  \end{quote}
%  Please put `empheq' in the subject line.
%
%  \subsection*{Contributors}
%
%  \begin{itemize}
%    \item In November 2002 Lars Madsen (\email{daleif@imf.au.dk})
%    asked for some features that wasn't readily available with
%    \pkg{amsmath} or any other package. This was the start of
%    \pkg{empheq}.
%
%    \item Gary Gray (\email{gray@engr.psu.edu}) gave me some bug
%    reports on the old version (v0.7) which convinced me to rewrite
%    the package completely. The \cs{Garybox} example is inspired by
%    a wish from Gary.
%
%    \item Uwe Siart (\email{uwe.siart@tum.de}) has been a thorough
%    beta-tester on v1.00 of the package.
%
%    \item Bernard Alfonsi (\email{alfonsi@math.u-psud.fr}) reported
%     a problem with the \opt{ntheorem} option. It turned out to be a
%     bug in the \pkg{ntheorem} package itself but it's fixed in the
%     option.
%
%    \item I have received reports on weird behavior from
%    \pkg{empheq} when used with the \pkg{color} package on some
%    dvi-viewers, which is caused by the lack of color support in
%    those dvi-viewers. Both Andrew B.~Collier (\email{colliera@ukzn.ac.za})
%    and Andr\'e M.~de Roos (\email{aroos@science.uva.nl}) have
%    notified me of this. The solution is to either upgrade you
%    dvi-viewer if possible or convert the document in question to
%    PDF or PS.
%  \end{itemize}
%  Thank you all.
%
%
% \begin{thebibliography}{99}
% \bibitem{Lamp94}
%   Leslie Lamport,
%   \emph{LaTeX User's Guide and Reference Manual}, 2nd edition,
%   Addison-Wesley, Reading, Massachusetts,
%   1994.
% \bibitem{Goos-etal94}
%   Michel Goossens, Frank Mittelbach, and Alexander Samarin,
%   \emph{The \LaTeX\ Companion},
%   Addison-Wesley, Reading, Massachusetts,
%   1994.
% \bibitem{Knut86}
%   Donald Ervin Knuth,
%   \emph{The \TeX book},
%   Addison-Wesley, Reading, Mas\-sa\-chu\-setts,
%   1986.
%  \bibitem{Ams99}
%    American Mathematical Society and Michael Downes,
%    \emph{Technical notes on the \pkg{amsmath} package} Version 2.0,
%    1999/10/29.
%    (Available from CTAN as file \texttt{technote.tex}.)
%  \bibitem{Ams00}
%    Frank Mittelbach, Rainer Sch\"opf, Michael Downes, and David M.~Jones,
%    \emph{The \pkg{amsmath} package} Version 2.13,
%    2000/07/18.
%    (Available from CTAN as file \texttt{amsmath.dtx}.)
%
%  \changes{v1.05b}{2004/02/18}{Added \pkg{fancybox} and
%  \pkg{ntheorem} to the \refname}
%  \bibitem{fancybox}
%    Timothy Van Zandt,
%    \emph{The \pkg{fancybox} package} Version 1.3,
%    2000/09/19
%    (Available from CTAN as file \texttt{fancybox.sty}.)
%
% \bibitem{Raht03}
%   Sebastian Rahtz,
%   \emph{Hypertext marks in \LaTeX} Version v6.74m,
%   2003/11/30.
%   (Available from \url{http://www.tug.org/applications/hyperref} as file
%   \texttt{hyperref.dtx}.)
%
% \bibitem{Carl99}
%   David Carlisle,
%   \emph{The \pkg{keyval} Package},
%   Version 1.13, 1999/03/16.
%   (Available from CTAN as file \texttt{keyval.dtx}.)
%
%  \changes{v1.06}{2004/03/13}{Added \pkg{calc} to the \refname}
%  \bibitem{Thorup:calc:1998}
%  Kresten~Krab Thorup, Frank Jensen, and Chris Rowley.
%  \emph{The {\pkg{calc}} package}, Version 4.1b, 1998/07/07,
%  (Available from CTAN as file \texttt{calc.dtx}.)
%
%  \bibitem{ntheorem}
%    Wolfgang May and Andreas Schlechte,
%    \emph{The \pkg{ntheorem} package} Version 1.203,
%    2002/01/07,
%    (Available from CTAN as file \texttt{ntheorem.dtx}.)
%
%  \changes{v2.10}{2004/07/26}{Added \pkg{showkeys} to the \refname}
%  \bibitem{Carlisle:showkeys:1997}
%  David Carlisle. \emph{The {\pkg{showkeys}} package}, v3.12,
%  1997/06/12, (Available from CTAN as file \texttt{showkeys.dtx}.)
%
%  \end{thebibliography}
%
%  \StopEventually{}
%
%  \EmphEqpart{Implementation}
%
%  \section{Requirements, Options, and Strategy}
%  Lets start the package. We'll need the tools from \pkg{mhsetup}.
%    \begin{macrocode}
%<*package>
\ProvidesPackage{empheq}%
  [2014/07/16 v2.14 Emphasizing equations]
%    \end{macrocode}
%    \begin{macrocode}
\RequirePackage{mhsetup}[2004/10/10]
\MHInternalSyntaxOn
%    \end{macrocode}
%
%
%  \subsection{Requirements, Options, and Allocation}
%
%  The \opt{overload} option allows the user to type
%  \begin{verbatim}
%    \begin{gather}[box=\fbox]
%      a = b
%    \end{gather}
%  \end{verbatim}
%  when they really mean
%  \begin{verbatim}
%    \begin{empheq}[box=\fbox]{gather}
%      a = b
%    \end{empheq}
%  \end{verbatim}
%    \begin{macrocode}
\MH_new_boolean:n {overload_ams}
\DeclareOption{overload}{
  \MH_set_boolean_T:n {overload_ams}
}
\MH_new_boolean:n {overload_amsII}
\DeclareOption{overload2}{
  \MH_set_boolean_T:n {overload_ams}
  \MH_set_boolean_T:n {overload_amsII}
}
%    \end{macrocode}
%  Old crappy \env{multline} interface or the new dazzling?
%    \begin{macrocode}
\MH_new_boolean:n {good_mult}
\DeclareOption{oldmultline}{\MH_set_boolean_F:n {good_mult}}
\DeclareOption{newmultline}{\MH_set_boolean_T:n {good_mult}}
%    \end{macrocode}
%
%
%  Load the \pkg{ntheorem} support file.
%    \begin{macrocode}
\gdef\g@EQ@ntheorem@support@bool{01}
\DeclareOption{ntheorem}{\gdef\g@EQ@ntheorem@support@bool{00}}
%    \end{macrocode}
%  Pass other options to \pkg{mathtools}.
%    \begin{macrocode}
\DeclareOption*{
  \PassOptionsToPackage{\CurrentOption}{mathtools}
}
\ExecuteOptions{newmultline}
\ProcessOptions\relax
%    \end{macrocode}
%  The name of the game is \pkg{amsmath} so we'll need that of
%  course. Since v2.10 that is now taken care of in the
%  \pkg{mathtools} package which'll pass on the options to
%  \pkg{amsmath}.
%    \begin{macrocode}
\MHInternalSyntaxOff
\RequirePackage{mathtools}[2004/10/10]
%    \end{macrocode}
%  Activate the internal syntax.
%    \begin{macrocode}
\MHInternalSyntaxOn
\AtEndOfPackage{\MHInternalSyntaxOff}
%    \end{macrocode}
%  \changes{v2.00}{2004/04/14}{Changed description of naming
%  conventions. Altered \emph{all} private macros; this is not
%  mentioned at each macro as this would just be too troublesome}
%
%  \begin{macro}{\l_EQ_equationtype_int}
%  \begin{macro}{\l_EQ_alignmentmode_int}
%  I'll use the following numbering scheme for the equation types,
%  and store the value in \cs{l_EQ_equationtype_int}:
%  \begin{multicols}{2}
%  \begin{enumerate}\setlength{\itemsep}{0pt}
%    \item align
%    \item align*
%    \item gather
%    \item gather*
%    \item flalign
%    \item flalign*
%    \item alignat
%    \item alignat*
%  \end{enumerate}
%  \end{multicols}
%  As you notice \env{equation} and \env{equation*} are missing. This
%  is because \env{gather} and \env{gather*} work the same way, only
%  they are `nicer'. Cuts down on the amount of code as well. I will
%  add an alias so that a user can use the key \key{equation}, but
%  then \key{gather} will kick in instead behind the scenes. Also the
%  \env{multline} environments are missing here, but again it's
%  because I'm faking it. See \S~\vref{sec:flalign+multline} for
%  further details. The value $0$ isn't used for any of the
%  environments and thus serves as a checkpoint (see
%  page~\pageref{ZeroCheckpoint}).
%
%  Similarly I'll use the counter \cs{l_EQ_alignmentmode_int} to
%  store the standard options for placement of math displays and
%  equation numbering.
%  \begin{multicols}{2}
%  \begin{enumerate}\setlength{\itemsep}{0pt}
%    \item reqno
%    \item leqno
%    \item fleqn, reqno
%    \item fleqn, leqno
%  \end{enumerate}
%  \end{multicols}
%
%    \begin{macrocode}
\newcount\l_EQ_equationtype_int
\newcount\l_EQ_alignmentmode_int
\if@fleqn
  \iftagsleft@
    \l_EQ_alignmentmode_int=4
  \else:
    \l_EQ_alignmentmode_int=\thr@@
  \fi:
\else:
  \iftagsleft@
    \l_EQ_alignmentmode_int=\tw@
  \else:
    \l_EQ_alignmentmode_int=\@ne
  \fi:
\fi:
%    \end{macrocode}
%  \end{macro}
%  \end{macro}
%  \begin{macro}{\l_EQ_totalwidth_dim}
%  \begin{macro}{\l_EQ_displaywidth_dim}
%  \begin{macro}{\l_EQ_temp_linewidth_dim}
%  \begin{macro}{\EmphEqdelimitershortfall}
%  \begin{macro}{\EmphEqdelimiterfactor}
%  \begin{macro}{\g_EQ_toptag_height_dim}
%  \begin{macro}{\g_EQ_bottomtag_depth_dim}
%  \begin{macro}{\g_EQ_toprow_height_dim}
%  \begin{macro}{\g_EQ_bottomrow_depth_dim}
%  \begin{macro}{\g_EQ_widesttag_dim}
%  Some lengths we'll need. Could possibly be scratch registers but we
%  have to make global assignments so that road is a little
%  dangerous.
%    \begin{macrocode}
\newdimen\l_EQ_totalwidth_dim
\newdimen\l_EQ_displaywidth_dim
%    \end{macrocode}
%  Just to make it accessible.
%    \begin{macrocode}
\def\EmphEqdisplaywidth{\l_EQ_displaywidth_dim}
\newdimen\l_EQ_temp_linewidth_dim
\newdimen\EmphEqdelimitershortfall
\newcount\EmphEqdelimiterfactor
\EmphEqdelimiterfactor950
\EmphEqdelimitershortfall\p@
\newdimen\g_EQ_toptag_height_dim
\newdimen\g_EQ_bottomtag_depth_dim
\newdimen\g_EQ_toprow_height_dim
\newdimen\g_EQ_bottomrow_depth_dim
\newdimen\g_EQ_widesttag_dim
\def\EmphEqtagwidth{\g_EQ_widesttag_dim}
%    \end{macrocode}
%  \end{macro}
%  \end{macro}
%  \end{macro}
%  \end{macro}
%  \end{macro}
%  \end{macro}
%  \end{macro}
%  \end{macro}
%  \end{macro}
%  \end{macro}
%
%  \begin{macro}{\EQ_outerbox:}
%  \begin{macro}{\EQ_innerbox:}
%  \begin{macro}{\EQ_left_side:}
%  \begin{macro}{\EQ_right_side:}
%  The two boxes and the material on either side of the math display.
%    \begin{macrocode}
\def\EQ_outerbox:{}
\def\EQ_innerbox:{}
\def\EQ_left_side:{}
\def\EQ_right_side:{}
%    \end{macrocode}
%  \end{macro}
%  \end{macro}
%  \end{macro}
%  \end{macro}
%
% \subsection{Strategy}
%
%  In non-\mode{fleqn} math displays are centered---even when having
%  tags. However this can look rather appalling when either the tag
%  or the math display are very wide like in the following example.
%  \begin{center}
%  \makebox[\linewidth][s]{\mbox{}\hfill \mbox{$\displaystyle
%    a+b+c+d+e+f+g+h+i+j+k+l+m+n+o$} \hfill\mbox{} \llap{(wide tag)}}
%  \end{center}
%  Not exactly pleasing to look at. \pkg{amsmath} takes care to do
%  this adjustment for us, so the user instead obtains the far better
%  \begin{gather*}
%  a+b+c+d+e+f+g+h+i+j+k+l+m+n+o \tag{wide tag}
%  \end{gather*}
%  So \pkg{amsmath} handles this quite well. That's good. It looks
%  like \pkg{amsmath} has an internal threshold that matches the
%  width of the math display against the width of the tags. But what
%  if we have a multiple line display and one of the lines causes an
%  adjustment to kick in while the others don't need it? Then we
%  cannot keep track of the complete display anymore. Furthermore we
%  may want to add material on either side of the math display making
%  the whole thing even more complicated, if not impossible to do
%  without some sort of different approach.
%
%  Luckily we have such an approach. The idea that comes to mind is
%  to see if one can typeset the labels and the math displays
%  \emph{separately}, store them in separate boxes and then put them
%  back together again afterwards. Is this really possible? Yes.
%  Come, I'll show you.
%
%  \subsection{The Main Environment}
%  \begin{macro}{\EQ_mathdisplay_box}
%  \begin{macro}{\EmphEqdisplayheight}
%  \begin{macro}{\EmphEqdisplaydepth}
%  \begin{macro}{\EQ_tag_box}
%  Okay, we'll start by allocating two box registers.
%    \begin{macrocode}
\newsavebox\EQ_mathdisplay_box
\def\EmphEqdisplayheight{\ht\EQ_mathdisplay_box}
\def\EmphEqdisplaydepth{\dp\EQ_mathdisplay_box}
\newsavebox\EQ_tag_box
%    \end{macrocode}
%  \end{macro}
%  \end{macro}
%  \end{macro}
%  \end{macro}
%
%  \begin{macro}{\EQ_complete_mathdisplay:}
%  \begin{macro}{\EQ_getbody:n}
%  \begin{environment}{EmphEqMainEnv}
%  The \env{EmphEqMainEnv} is (as the name implies) the main
%  environment. I decided to go with the environment form because
%  that's what \pkg{amsmath} users are accustomed to. But the code
%  will require the contents of the environment to be handled as an
%  argument to a command, so we use \cs{collect@body} from
%  \pkg{amsmath} itself to collect it.
%    \begin{macrocode}
\def\EQ_complete_mathdisplay:{}
\def\EQ_getbody:n #1{%
  \def\EQ_complete_mathdisplay:{}\def\EQ_mathbody:{#1}}
\newenvironment{EmphEqMainEnv}{\collect@body\EQ_getbody:n}{%
%    \end{macrocode}
%  If the user didn't specify the mandatory argument we will catch it
%  because \cs{l_EQ_equationtype_int} is zero. An error is
%  issued\label{ZeroCheckpoint}.
%    \begin{macrocode}
  \if_num:w \l_EQ_equationtype_int=\z@ %ZeroCheckpoint
    \EQ_upgrade_error:
  \else:
%    \end{macrocode}
%  Those that read the manual get the reward here: The package will
%  actually work!
%    \begin{macrocode}
    \EQ_post_process:
    \ignorespacesafterend
  \fi:}
%    \end{macrocode}
%  \end{environment}
%  \end{macro}
%  \end{macro}
%  \subsection{Post Processing}
%  \begin{macro}{\EQ_post_process:}
%  I know: it almost looked easy for a while didn't it? Sadly the
%  post-processing step of the code is also by far the most
%  complicated.
%    \begin{macrocode}
\def\EQ_post_process:{%
%    \end{macrocode}
%  First a hook where we initialize a few things like storing the
%  line width for later manipulation. Also we want to measure the
%  height of the top row and the depth of the bottom row for best
%  possible placement of the math display and the boxes around it
%  (same with the tags). Those four dimensions are globally set
%  during the code, hence this assignment. Most other stuff is local.
%    \begin{macrocode}
  \EQ_initialize:
%    \end{macrocode}
%  Then we proceed by completing the input. We store it in the
%  control sequence \cs{EQ_complete_mathdisplay}.
%  \cs{EQ_begin_equation:} adds the start of the environment and
%  \cs{EQ_end_equation:} adds the corresponding end.
%    \begin{macrocode}
  \EQ_begin_equation:
%    \end{macrocode}
%  In case we face a \env{multline} we add what's necessary.
%    \begin{macrocode}
  \MH_if_boolean:nT {outer_mult}{
    \setkeys{\MT_options_name:}{
      firstline-afterskip=0pt,
      lastline-preskip=0pt,
      multlined-width=0pt,
    }
    \ifodd \l_EQ_alignmentmode_int
      \setkeys{\MT_options_name:}{multlined-pos=b}
    \else:
      \setkeys{\MT_options_name:}{multlined-pos=t}
    \fi:
      \g@addto@macro\EQ_complete_mathdisplay:{
        \begin{MTmultlined}\relax}
  }
  \g@addto@macro\EQ_complete_mathdisplay:\EQ_mathbody:
  \MH_if_boolean:nT {outer_mult}{
    \g@addto@macro\EQ_complete_mathdisplay:{\end{MTmultlined}}
  }
  \EQ_end_equation:
%    \end{macrocode}
%  Save the value of the \pkg{hyperref} link counter
%  \cs{Hy@linkcounter}.
%    \begin{macrocode}
  \edef\EQ_hyperref_link_fint{\number\Hy@linkcounter}
%    \end{macrocode}
%  Firstly we call \cs{EQ_typeset_tags:} which suppresses all
%  output from the equations with the exception of the \cs{tag}'s.
%  This also allows us to find the widest tag which we'll need for
%  the environments that set's the math displays flush to the
%  margins, namely the \env{flalign} family. We save the counters
%  first though. As an extra treat we're also able to measure the
%  height and depth of the tags and the math display.
%    \begin{macrocode}
  \savecounters@
  \EQ_typeset_tags:
%    \end{macrocode}
%  \changes{v1.05d}{2004/02/26}{Removed one instance of
%  \cs{EQ_typeset_display:}}
%  Then find the widest tag. In an earlier implementation it was
%  necessary to typeset the display twice---a very costly procedure!
%  Now we save roughly 2120 lines of output from \cs{tracingall} on a
%  simple \env{gather} and 2600 lines with \env{align}. That's  a
%  little more than 21\% saved\ldots
%  \changes{v2.10}{2004/07/26}{Removed \cs{find_widesttag:}}
%  \changes{v2.10}{2004/07/26}{Added check for equation number to
%  make sure the labelling/printing mechanism of \pkg{mathtools} work
%  properly}
%  Then we restore the counters and measure the extra material.
%  Save the value of the equation counter for later in case the
%  special equation numbering scheme from \pkg{mathtools} is in
%  effect.
%    \begin{macrocode}
  \edef\l_EQ_equation_fint{\number\c@equation}
  \restorecounters@
%    \end{macrocode}
%  Restore the value of the \pkg{hyperref} link counter
%  \cs{Hy@linkcounter}.
%    \begin{macrocode}
  \Hy@linkcounter=\EQ_hyperref_link_fint
  \settowidth\l_EQ_totalwidth_dim{%
    $\EQ_left_side:\EQ_innerbox:{}\EQ_right_side:$}%
%    \end{macrocode}
%  Then call the appropriate \cs{EQ_modify_linewidth_}\meta{num}\texttt{:}
%  where \meta{num} is taken from the numbering scheme shown earlier.
%    \begin{macrocode}
  \MH_if_boolean:nTF {outer_mult}
    { \def\EQ_next:{_mult} }
    { \def\EQ_next:{} }
  \@nameuse{EQ_modify_linewidth_
            \romannumeral\l_EQ_equationtype_int\EQ_next: :}%
%    \end{macrocode}
%  We can then typeset the display knowing that all modifications
%  have been made correctly. \cs{EQ_typeset_display:} gobbles the
%  \cs{tag}'s, which in turn is equivalent to setting the display
%  centered in non-\mode{fleqn} mode.
%    \begin{macrocode}
  \EQ_typeset_display:
%    \end{macrocode}
%  Call the appropriate \cs{find_displaywidth_}\meta{num}\texttt{:}
%  which extracts the display width (\cs{l_EQ_displaywidth_dim}).
%    \begin{macrocode}
  \@nameuse{find_displaywidth_
           \romannumeral\l_EQ_equationtype_int\EQ_next: :}
  \MH_addtolength:dn
    \l_EQ_totalwidth_dim{\l_EQ_displaywidth_dim}
%    \end{macrocode}
%  We save the math display in a box of the correct size.
%    \begin{macrocode}
  \savebox\EQ_mathdisplay_box[\l_EQ_totalwidth_dim][c]{%
    $\EQ_left_side:
    \EQ_innerbox:{%
      \if_case:w \l_EQ_alignmentmode_int\or:
%    \end{macrocode}
%  If the equation is centered we make the box centered
%    \begin{macrocode}
        \makebox[\l_EQ_displaywidth_dim][c]
          {\usebox{\EQ_mathdisplay_box}}%
      \or:
        \makebox[\l_EQ_displaywidth_dim][c]
          {\usebox{\EQ_mathdisplay_box}}%
      \or:
%    \end{macrocode}
%  Else we set it in a left aligned box and take care of the
%  \cs{@mathmargin} unless we're in \env{flalign} or \env{flalign*}.
%    \begin{macrocode}
        \makebox[\l_EQ_displaywidth_dim][l]{%
          \if_num:w \l_EQ_equationtype_int=5 \else:
            \if_num:w \l_EQ_equationtype_int=6 \else:
              \kern-\@mathmargin
            \fi:
          \fi:
          \usebox{\EQ_mathdisplay_box}}%
      \or:
        \makebox[\l_EQ_displaywidth_dim][l]{%
          \if_num:w \l_EQ_equationtype_int=5 \else:
            \if_num:w \l_EQ_equationtype_int=6 \else:
              \kern-\@mathmargin
            \fi:
          \fi:
          \usebox{\EQ_mathdisplay_box}}%
      \fi:}%
    \EQ_right_side:$}%
%    \end{macrocode}
%  Then we're ready for typesetting. We just use an ordinary
%  |\begin{equation*}| | ... | |\end{equation*}| construction.
%  \changes{v2.00}{2004/04/14}{Used a saved definition of
%  \env{equation*} instead to allow overloading}
%  With v2.00 this changed to using a copy of the original
%  \env{equation*} environment. All this because of the
%  \opt{overload} option.
%    \begin{macrocode}
  \begin{AmSequation*}
  \@nameuse{EQ_place_display_
           \romannumeral\l_EQ_equationtype_int\EQ_next: :}
%    \end{macrocode}
%  \cs{tag} must have a non-zero width argument in order to function
%  properly, so we make sure there is a tag before we try to typeset
%  it. Also make sure that the boolean |show_manual_tags| is true
%  just for this tag.
%    \begin{macrocode}
  \if_dim:w \g_EQ_widesttag_dim>\z@
  \MH_set_boolean_T:n {show_manual_tags}
%    \end{macrocode}
%  Set the tag in a left aligned box when in \mode{leqno} mode and
%  right aligned when in \mode{reqno} mode.
%    \begin{macrocode}
    \if_case:w \l_EQ_alignmentmode_int
    \or: \EQ_place_tag:N r %#1
    \or: \EQ_place_tag:N l %#2
    \or: \EQ_place_tag:N r %#3
    \or: \EQ_place_tag:N l %#4
    \fi:
  \fi:
  \end{AmSequation*}
%    \end{macrocode}
%  \changes{v2.11}{2004/08/03}{Use raw \TeX{} commands to speed it up}
%  Restore the equation counter.
%    \begin{macrocode}
  \global\c@equation=\l_EQ_equation_fint\relax
}
%    \end{macrocode}
%  \end{macro}
%
%  \subsubsection{Initializing the Equations}
%
%  \begin{macro}{\EQ_initialize:}
%  We save all the initializations in this hook:
%    \begin{macrocode}
\def\EQ_initialize:{%
  \m@th
%    \end{macrocode}
%  More support for \pkg{ntheorem}. We use special versions of
%  the tag setting commands in order to produce the correct result.
%  For instance in \mode{leqno} mode you would still want the end of
%  theorem symbol at the bottom of the math display.
%    \begin{macrocode}
  \MH_let:NwN\SetTagPlusEndMark\EQ_tag_plus_endmark:
  \MH_let:NwN\SetOnlyEndMark\EQ_only_endmark:
  \l_EQ_temp_linewidth_dim\linewidth
  \g_EQ_toprow_height_dim\z@
  \g_EQ_bottomrow_depth_dim\z@
  \g_EQ_toptag_height_dim\z@
  \g_EQ_bottomtag_depth_dim\z@
  \g_EQ_widesttag_dim\z@
%    \end{macrocode}
%  \changes{v2.11}{2004/08/03}{Added a \cs{g_EQ_latest_nonzerodepth_row_fint}}
%    \begin{macrocode}
  \xdef\g_EQ_latest_nonzerodepth_row_fint{\z@}
  \edef\EQ_restore_tex_delimiter:{%
    \delimiterfactor\the\delimiterfactor
    \delimitershortfall\the\delimitershortfall\relax}%
  \g@addto@macro\EQ_complete_mathdisplay:
                \EQ_restore_tex_delimiter:
  \delimitershortfall\EmphEqdelimitershortfall
  \delimiterfactor\EmphEqdelimiterfactor
%    \end{macrocode}
%  Disable \cs{intertext} and issue an error message. Using the
%  normal error message for an invalid \cs{intertext} would be less
%  than great because it would say
%  \begin{verbatim}
%  ! Package amsmath Error: Invalid use of \intertext.
%
%  See the amsmath package documentation for explanation.
%  Type  H <return>  for immediate help.
%   ...
%  \end{verbatim}
%  That's not good enough, so we redefine things a bit. In
%  \pkg{amsmath} the command \cs{intertext@} is called inside the
%  math environments and defines \cs{intertext}. We hook into this.
%    \begin{macrocode}
  \MH_let:NwN\intertext@\EQ_intertext:
  \EQ_displaybreak:}
%    \end{macrocode}
%  \end{macro}
%  \begin{macro}{\EQ_intertext:}
%  Then we issue a suitable error message.
%    \begin{macrocode}
\def\EQ_intertext:{%
  \def\intertext##1{\PackageError{empheq}{%
    You~cannot~use~\string\intertext\space inside~the~`empheq'
    \MessageBreak
    environment,~as~`empheq'~creates~an~unbreakable~box}\@eha}}
\def\EQ_displaybreak:{%
  \def\dspbrk@[##1]{\PackageWarning{empheq}{%
    You~cannot~use~\string\displaybreak\space inside~the~`empheq'%
    \MessageBreak
    environment,~as~`empheq'~creates~an~unbreakable~box.}}}
%    \end{macrocode}
%  \end{macro}
%  \begin{macro}{\EQ_new_equationtype:nnn}
%  \begin{macro}{\EQ_new_equationtype_arg:nnn}
%  We have to inform \env{EmphEqMainEnv} which equation type we're
%  using, so we define an interface for that.
%    \begin{macrocode}
\def\EQ_new_equationtype:nnn #1#2#3{%
  \define@key{#1}{#2}[true]{%
    \l_EQ_equationtype_int=#3\relax
    \def\EQ_begin_equation:
      {\g@addto@macro{\EQ_complete_mathdisplay:}{\begin{#2}}}%
    \def\EQ_end_equation:
      {\g@addto@macro{\EQ_complete_mathdisplay:}{\end{#2}}}}}
\def\EQ_new_equationtype_arg:nnn #1#2#3{%
  \define@key{#1}{#2}{%
    \l_EQ_equationtype_int=#3\relax
    \def\EQ_begin_equation:
      {\g@addto@macro{\EQ_complete_mathdisplay:}{\begin{#2}{##1}}}%
    \def\EQ_end_equation:
      {\g@addto@macro{\EQ_complete_mathdisplay:}{\end{#2}}}}}
%    \end{macrocode}
%  \end{macro}
%  \end{macro}
%
%  \subsubsection{Separating Tags and Display}
%
%  \begin{macro}{\EQ_typeset_tags:}
%  Typesetting the tags without the display.
%    \begin{macrocode}
\def\EQ_typeset_tags:{%
  \sbox\EQ_tag_box{%
    \parbox{.5\maxdimen}{%
      \hbadness\@M
%    \end{macrocode}
%  Make sure all skips are zero ands support \pkg{ntheorem}.
%  \changes{v1.05a}{2004/02/17}{Forgot to support \pkg{ntheorem} in
%  both \cs{EQ_typeset_tags:} and \cs{EQ_typeset_display:}.}
%    \begin{macrocode}
      \EQ_prevent_vertical_space:
%    \end{macrocode}
%  Support for \pkg{ntheorem}. This package patches \cs{endgather}
%  and \cs{endalign} with the commands \cs{TagsPlusEndmarks} and
%  \cs{RestoreTags} but we do not want them disturbing our output so
%  we disable them when typesetting the display.
%    \begin{macrocode}
      \EQ_typeset_tags_hook:
%    \end{macrocode}
%  Then use our own versions of \cs{set@gather@field} and
%  \cs{set@field}.
%    \begin{macrocode}
      \MH_let:NwN\set@gather@field\EQ_set_gather_field:
      \MH_let:NwN\set@field\EQ_set_align_field:
%    \end{macrocode}
%  Next two lines are for measuring of the first and last row of a
%  display of the \env{align}, \env{alignat} or \env{flalign} family.
%    \begin{macrocode}
      \MH_set_boolean_T:n {continue_measuring}
      \global\g_EQ_temprow_int\m@ne
      \MH_let:NwN \EQ_saved_maketag:n \maketag@@@
      \def\maketag@@@##1{
        \EQ_maketag:Nn \EQ_saved_maketag:n {##1}}
%    \end{macrocode}
%  Then typeset the display.
%    \begin{macrocode}
      \EQ_complete_mathdisplay:}}}
%    \end{macrocode}
%  \end{macro}
%
%  \begin{macro}{\EQ_predisplay_adjustment:}
%  \changes{v1.05c}{2004/02/25}{Added \cs{EQ_predisplay_adjustment:}}
%
%  \changes{v2.00}{2004/04/14}{Added a missing \cs{normallineskip} in
%  case of extremely big tag height}
%
%  For adjusting vertical spacing in the display.
%    \begin{macrocode}
\def\EQ_predisplay_adjustment:{%
  \if_dim:w \g_EQ_toprow_height_dim<\baselineskip
    \vskip\g_EQ_toprow_height_dim
    \vskip-\baselineskip
  \else:
    \vskip-\normallineskip
  \fi:
  \if_dim:w \g_EQ_toptag_height_dim>\g_EQ_toprow_height_dim
    \if_dim:w \g_EQ_toptag_height_dim>\baselineskip
      \vskip-\g_EQ_toptag_height_dim
      \vskip\baselineskip
      \vskip-\normallineskip
    \fi:
  \fi:}
%    \end{macrocode}
%  \end{macro}
%  \begin{macro}{\EQ_postdisplay_adjustment:}
%  \changes{v1.05c}{2004/02/25}{Added \cs{EQ_postdisplay_adjustment:}}
%  Name says it all.
%    \begin{macrocode}
\def\EQ_postdisplay_adjustment:{%
  \if_dim:w \g_EQ_bottomrow_depth_dim<\dp\strutbox
    \vskip\g_EQ_bottomrow_depth_dim
    \vskip-\dp\strutbox
  \fi:
  \if_dim:w \g_EQ_bottomtag_depth_dim>\g_EQ_bottomrow_depth_dim
    \if_dim:w \g_EQ_bottomtag_depth_dim>\dp\strutbox
      \vskip-\g_EQ_bottomtag_depth_dim
      \vskip\dp\strutbox
    \fi:
  \fi:}
%    \end{macrocode}
%  \end{macro}
%  \begin{macro}{\EQ_typeset_display:}
%  \changes{v1.05c}{2004/02/25}{Now takes two arguments: Either both
%  empty or \cs{EQ_predisplay_adjustment:} \&
%  \cs{EQ_postdisplay_adjustment:}}
%  \changes{v1.05d}{2004/02/26}{Changed again. Uses only
%  \cs{EQ_predisplay_adjustment:} \& \cs{EQ_postdisplay_adjustment:}}
%  We typeset the display without the tags and save it in a box.
%    \begin{macrocode}
\def\EQ_typeset_display:{%
  \sbox\EQ_mathdisplay_box{%
    \parbox{\l_EQ_temp_linewidth_dim}{%
%    \end{macrocode}
%  We adjust the vertical spacing if needed:
%    \begin{macrocode}
      \EQ_predisplay_adjustment:
      \EQ_prevent_vertical_space:
%    \end{macrocode}
%  Support for \pkg{ntheorem}. This package patches \cs{endgather}
%  and \cs{endalign} with the commands \cs{TagsPlusEndmarks} and
%  \cs{RestoreTags} but we do not want them disturbing our output so
%  we disable them when typesetting the display.
%    \begin{macrocode}
      \EQ_typeset_display_hook:
%    \end{macrocode}
%  Instead of gobbling the tags it's better to use \cs{phantom} since
%  a tall tag will ruin the vertical spacing. We use a special
%  version of \cs{phantom} to help us with other things as well.
%    \begin{macrocode}
      \def\maketag@@@##1{\EQ_maketag:Nn \vphantom{##1}}%
      \EQ_complete_mathdisplay:
%    \end{macrocode}
%  More space adjustment.
%    \begin{macrocode}
      \EQ_postdisplay_adjustment:}}}
%    \end{macrocode}
%  \end{macro}
%  We need a few little helpers:
%  \begin{macro}{\EQ_prevent_vertical_space:}
%  \changes{v2.00}{2004/04/14}{Used \cs{nointerlineskip} instead}
%  As we typeset the math display inside a \cs{parbox} the math
%  environment will add \cs{abovedisplayskip} at the top and
%  \cs{belowdisplayskip} at the bottom. Also we suppress interline
%  skip with \cs{nointerlineskip}.
%    \begin{macrocode}
\def\EQ_prevent_vertical_space:{%
  \nointerlineskip
  \abovedisplayskip\z@  \belowdisplayskip\z@\relax}
%    \end{macrocode}
%  \end{macro}
%  \begin{macro}{\EQ_maketag:Nn}
%  \changes{v1.05d}{2004/02/26}{Changed name from
%  \cs{EmphEqphantom} to \cs{EmphEqMaketag}}
%  This takes care of two things: We need to store the height of the
%  top tag and the depth of the bottom tag for later, but we also
%  need to use a \cs{vphantom} of the tag as it may be taller than
%  the line of math it is attached to (say, if you are using node
%  connectors from \pkg{pstricks} or some other weird stuff).
%    \begin{macrocode}
\def\EQ_maketag:Nn #1#2{%
  \setboxz@h{#2}%
  \if_num:w \row@=\@ne
    \global \g_EQ_toptag_height_dim = \ht\z@
  \fi:
%    \end{macrocode}
%  Similarly the depth of the last row is determined by measuring the
%  depth of box zero for each row and then using the last of the
%  values.
%    \begin{macrocode}
  \global \g_EQ_bottomtag_depth_dim=\dp\z@
%    \end{macrocode}
%  \changes{v2.11}{2004/08/03}{Added a check for being in the last row.}
%  Save the row number for later as the \opt{ntheorem} option will need
%  it.
%    \begin{macrocode}
  \xdef\g_EQ_latest_nonzerodepth_row_fint{\number\row@}
  \@tempdima=\wd\z@
  \if_dim:w \@tempdima > \g_EQ_widesttag_dim
    \global \g_EQ_widesttag_dim = \@tempdima
  \fi:
  #1{#2}%
}
%    \end{macrocode}
%  \end{macro}
%
%  \subsubsection{Measuring \env{gather}}
%  \begin{macro}{\EQ_set_gather_field:}
%  Box zero (\cs{z@}) contains the entire line of math in a
%  \env{gather}, so instead of typesetting it we typeset the phantom
%  of it and call \cs{EQ_measure_gather:}.
%    \begin{macrocode}
\def\EQ_set_gather_field:{%
  \iftagsleft@
    \global\lineht@\ht\z@
  \else:
    \global\lineht@\dp\z@
  \fi:
  \kern\eqnshift@
  %\boxz@ % Original amsmath version
  \EQ_measure_gather:
  \phantom{\boxz@}%
  \hfil}
%    \end{macrocode}
%  \end{macro}
%  \begin{macro}{\EQ_measure_gather:}
%  Measuring the height of the individual lines in a \env{gather} is
%  fairly easy. If \cs{row@} is equal to one, we're in the first
%  line, and the height of the top row is simply given by the height
%  of box zero.
%    \begin{macrocode}
\def\EQ_measure_gather:{%
  \if_num:w \row@=\@ne
    \g_EQ_toprow_height_dim=\ht\z@
  \fi:
%    \end{macrocode}
%  Similarly the depth of the last row is determined by measuring the
%  depth of box zero for each row and then using the last of the
%  values.
%    \begin{macrocode}
  \g_EQ_bottomrow_depth_dim=\dp\z@
  \global\g_EQ_toprow_height_dim\g_EQ_toprow_height_dim
  \global\g_EQ_bottomrow_depth_dim\g_EQ_bottomrow_depth_dim}
%    \end{macrocode}
%  \end{macro}
%
%  \subsubsection{Measuring \env{align}, \env{alignat} and \env{flalign}}
%  \begin{macro}{\EQ_set_align_field:}
%  Same technique as for \cs{EQ_set_gather_field}. Only this time
%  we have to stop the underlying \cs{start@align} from going over
%  the measuring phase twice. That's when the conditional
%  \bool{continue_measuring} comes into the picture.
%    \begin{macrocode}
\def\EQ_set_align_field:{%
  \column@plus
  \iftagsleft@
    \if_dim:w \ht\z@>\lineht@
      \global\lineht@\ht\z@
    \fi:
  \else:
    \if_dim:w \dp\z@>\lineht@
      \global\lineht@\dp\z@
    \fi:
  \fi:
  %\boxz@ % Original amsmath version
  \MH_if_boolean:nT {continue_measuring}{ \EQ_measure_align: }
  \phantom{\boxz@}}
%    \end{macrocode}
%  \end{macro}
%  \begin{macro}{\EQ_measure_align:}
%  We have to measure every single field of the \cs{align@} \ldots\@
%  Oh well, here it comes:
%    \begin{macrocode}
\newcount\g_EQ_temprow_int
\MH_new_boolean:n {still_in_samerow}
\MH_new_boolean:n {continue_measuring}
\def\EQ_measure_align:{%
  \if_num:w \g_EQ_temprow_int=\row@
    \MH_set_boolean_T:n {still_in_samerow}
  \else:
    \if_num:w \g_EQ_temprow_int>\row@
      \global\MH_set_boolean_F:n {continue_measuring}
    \fi:
    \MH_set_boolean_F:n {still_in_samerow}
    \global\@tempdimc=\@tempdimb
    \global\@tempdimb=\g_EQ_bottomrow_depth_dim
  \fi:
  \MH_if_boolean:nTF {continue_measuring}{
    \global\g_EQ_temprow_int=\row@
    \@tempcnta=\row@
    \if_num:w \@tempcnta=\z@
      \@tempdima=\g_EQ_toprow_height_dim
      \g_EQ_toprow_height_dim=\ht\z@\relax
      \if_dim:w \g_EQ_toprow_height_dim<\@tempdima
        \g_EQ_toprow_height_dim\@tempdima
      \fi:
    \fi:
    \MH_if_boolean:nTF {still_in_samerow}{
      \@tempdima=\g_EQ_bottomrow_depth_dim
      \g_EQ_bottomrow_depth_dim=\dp\z@\relax
      \if_dim:w \g_EQ_bottomrow_depth_dim<\@tempdima
        \g_EQ_bottomrow_depth_dim\@tempdima
      \fi:
    }{
      \g_EQ_bottomrow_depth_dim\dp\z@\relax
    }
  }{
    \global\g_EQ_bottomrow_depth_dim=\@tempdimc
  }
  \global\g_EQ_toprow_height_dim\g_EQ_toprow_height_dim
  \global\g_EQ_bottomrow_depth_dim\g_EQ_bottomrow_depth_dim}
%    \end{macrocode}
%  \end{macro}
%  \subsubsection{Typesetting the Tags}
%  \begin{macro}{\EQ_place_tag:N}
%  Fool \LaTeX\ into thinking the tag is only as wide as the widest
%  tag (remember the real width of the \cs{EQ_tag_box} is
%  $0.5$\cs{maxdimen}.
%    \begin{macrocode}
\def\EQ_place_tag:N #1{%
  \tag*{\makebox[\g_EQ_widesttag_dim][#1]{%
    \EQ_vertical_tag_adjustment:n {\usebox{\EQ_tag_box}}}}}
%    \end{macrocode}
%  \end{macro}
%  \begin{macro}{\EQ_vertical_tag_adjustment:n}
%  \changes{v2.00}{2004/04/14}{Added a missing \cs{normallineskip} in
%  case of extremely big tag height}
%  This is a little tricky. We have measured the height of the top
%  row and top tag and the depth of the bottom row and bottom tag.
%  Now we have to adjust the vertical spacing so it comes out just
%  right. Remember we have separated the tags and the math display
%  but we would still like to have them line up as they should! We do
%  it all inside a group so we can use the scratch dimensions safely.
%    \begin{macrocode}
\def\EQ_vertical_tag_adjustment:n #1{{%
  \@tempdima\z@\@tempdimb\z@
  \@tempdimc=\dp\strutbox
  \if_dim:w \g_EQ_toprow_height_dim<\baselineskip
    \MH_setlength:dn
      \@tempdima{\g_EQ_toprow_height_dim-\baselineskip}
  \else:
    \MH_setlength:dn \@tempdima{-\normallineskip}
  \fi:
  \if_dim:w \g_EQ_toptag_height_dim>\g_EQ_toprow_height_dim
    \if_dim:w \g_EQ_toptag_height_dim>\baselineskip
      \MH_addtolength:dn \@tempdima
        {-\g_EQ_toptag_height_dim+\baselineskip-\normallineskip}
    \fi:
  \fi:
  \if_dim:w \g_EQ_bottomrow_depth_dim<\@tempdimc
    \MH_setlength:dn \@tempdimb
      {\g_EQ_bottomrow_depth_dim-\@tempdimc}
  \fi:
%    \end{macrocode}
%  \changes{v2.00}{2004/04/14}{Fixed minor bug}
%  If the last condition is fulfilled we also need to cancel the
%  first condition.
%    \begin{macrocode}
  \if_dim:w \g_EQ_bottomtag_depth_dim>\g_EQ_bottomrow_depth_dim
    \if_dim:w \g_EQ_bottomtag_depth_dim<\@tempdimc
    \else:
      \MH_addtolength:dn \@tempdimb
        {-\g_EQ_bottomtag_depth_dim+\@tempdimc}
    \fi:
  \fi:
  \raisebox{-0.5\@tempdima+0.5\@tempdimb}{#1}}}%
%    \end{macrocode}
%  \end{macro}
%  \section{Adding the Equation Types}
%  Adding an equation type is fairly easy. The outline is this:
%  \begin{list}{}{\leftmargin2em}
%    \item[]|\EQ_new_equationtype{EmphEqEnv}|\marg{\AmS\
%    env}\marg{arabic num}\texttt{:}
%    \item[]|\def\EQ_modify_linewidth_|\meta{roman num}\texttt{:}\\
%    \hspace*{2em}\marg{Optionally alter \cs{l_EQ_temp_linewidth_dim}}
%    \item[]|\def\find_displaywidth_|\meta{roman num}\texttt{:}\\
%    \hspace*{2em}\marg{Set \cs{l_EQ_displaywidth_dim}}
%    \item[]|\def\EQ_place_display_|\meta{roman num}\texttt{:}\\
%    \hspace*{2em}\marg{Make use of \cs{EQ_use_mathdisplay:}}
%  \end{list}
%  \begin{macro}{\EQ_use_mathdisplay:}
%  A shorthand.
%    \begin{macrocode}
\def\EQ_use_mathdisplay:{%
  \EQ_outerbox:{\usebox{\EQ_mathdisplay_box}}}
%    \end{macrocode}
%  \end{macro}
%    \begin{macrocode}
\def\EQ_options_name:{EmphEqOpt}
%    \end{macrocode}
%  \subsection{The \env{equation} Family}
%  Just a few aliases because \env{equation} and \env{equation*}
%  behave strangely. Much better to use \env{gather} instead as this
%  is easier to measure. Besides it's less work (for us, not the
%  compiler). ;-)
%    \begin{macrocode}
\MH_keyval_alias_with_addon:nnnn {EmphEqEnv}{equation}{gather}
  {\EQ_disable_linebreak:}
\MH_keyval_alias_with_addon:nnnn {EmphEqEnv}{equation*}{gather*}
  {\EQ_disable_linebreak:}
%    \end{macrocode}
%  \begin{macro}{\EQ_disable_linebreak:}
%  \changes{v1.05c}{2004/02/25}{Introduced
%  \cs{EQ_disable_linebreak:} to prevent multi line
%  \env{equation}'s}
%  \changes{v1.05c}{2004/02/25}{Fixed it so that it works better}
%  Disable the line breaks in \env{equation} as these are actually
%  possible now, since we're in a \env{gather}! \cs{Let@} does it in
%  the environment, so that's the place to hack.
%    \begin{macrocode}
\def\EQ_disable_linebreak:{%
  \MH_let:NwN \EQ_saved_Let: \Let@
  \def\Let@{\def\\{%
    \PackageError{empheq}{You~ cannot~ use~ \string\\~ in~ equation!}
      {\@eha}}
%    \end{macrocode}
%  Then restore the meaning of \cs{Let@} in case there is a
%  \env{gathered} or alike coming up.
%    \begin{macrocode}
    \MH_let:NwN \Let@ \EQ_saved_Let:
  }%
}
%    \end{macrocode}
%  \end{macro}
%
%  \subsection{The \env{align} Family}
%    \begin{macrocode}
\EQ_new_equationtype:nnn {EmphEqEnv}{align}{1}
\EQ_new_equationtype:nnn {EmphEqEnv}{align*}{2}
%    \end{macrocode}
%    \begin{macrocode}
\def\EQ_modify_linewidth_i:{%
%    \end{macrocode}
%  If $\cs{maxfields@}=2$ we're in a single column \env{align} and
%  thus there is no flexible space.
%    \begin{macrocode}
  \if_num:w \maxfields@=\tw@
  \else:
%    \end{macrocode}
%  But if there is any flexible space we would very much like to
%  adjust it because we add material to the right and left of the
%  math display.
%    \begin{macrocode}
    \setlength{\l_EQ_temp_linewidth_dim}{%
      \linewidth-\l_EQ_totalwidth_dim-\widthof{\EQ_outerbox:{}}}%
  \fi:}
\def\EQ_modify_linewidth_ii:{\EQ_modify_linewidth_i:}
\def\find_displaywidth_i:{%
  \MH_setlength:dn \l_EQ_displaywidth_dim
    {\l_EQ_temp_linewidth_dim-\alignsep@}
  \if_case:w \l_EQ_alignmentmode_int
  \or: % #1:
    \MH_addtolength:dn \l_EQ_displaywidth_dim{-\alignsep@}%
  \or: % #2:
    \MH_addtolength:dn \l_EQ_displaywidth_dim{-\alignsep@}%
  \or: % #3:
    \MH_addtolength:dn \l_EQ_displaywidth_dim{-\@mathmargin}%
  \or: % #4:
    \MH_addtolength:dn \l_EQ_displaywidth_dim{-\@mathmargin}%
  \fi:}
\def\find_displaywidth_ii:{\find_displaywidth_i:}
\def\EQ_place_display_i: {\EQ_use_mathdisplay:}
\def\EQ_place_display_ii:{\EQ_place_display_i:}
%    \end{macrocode}
%
%  \subsection{The \env{gather} Family}
%  This is a gem. Notice how easy it is once we got the framework for
%  it.
%    \begin{macrocode}
\EQ_new_equationtype:nnn {EmphEqEnv}{gather}{3}
\EQ_new_equationtype:nnn {EmphEqEnv}{gather*}{4}
\def\EQ_modify_linewidth_iii:{}
\def\EQ_modify_linewidth_iv:{\EQ_modify_linewidth_iii:}
\def\find_displaywidth_iii:{%
  \MH_setlength:dn \l_EQ_displaywidth_dim{\totwidth@}%
  \if_case:w \l_EQ_alignmentmode_int \or: \or: % #1 & #2:
  \or: % #3:
    \MH_addtolength:dn \l_EQ_displaywidth_dim{-\@mathmargin}%
  \or: % #4:
    \MH_addtolength:dn \l_EQ_displaywidth_dim{-\@mathmargin}%
  \fi:}
\def\find_displaywidth_iv:{\find_displaywidth_iii:}
\def\EQ_place_display_iii:{\EQ_use_mathdisplay:}
\def\EQ_place_display_iv: {\EQ_place_display_iii:}
%    \end{macrocode}
%
%  \subsection{The \env{flalign} Family}\label{sec:flalign+multline}
%  This was fairly easy, but then I had to take care of \ldots
%
%  \subsubsection{The \env{multline} Family}
%  This was quite a pickle. Its implementation in \pkg{amsmath} is
%  very strange indeed; for instance the tag is typeset in the
%  measuring phase! So instead I thought I could fake it, but how
%  should I do it?
%
%  The basic idea of \env{multline} is to typeset the first line
%  flush left, the last line flush right and the rest centered.
%  ``Aha,'' I thought, ``what if I make three columns, fool \LaTeX{}
%  into thinking the contents of the columns take no space and then
%  just put each line into a column?'' Thus  the \env{flalign}
%  environment came to mind. I also distinctly remembered to have
%  seen an article about the equivalent of \cs{rlap} and \cs{llap} in
%  math mode, namely an article by Alexander R.~Perlis. This works
%  pretty well, and so does the same trick for producing a
%  \env{multlined} environment. That one uses an \env{alignedat} with
%  three columns.
%
%  \changes{v1.05c}{2004/02/25}{Makes use of
%  \cs{MakeKeyvalAliasAddon} now}
%  Then we make \env{multline} into an alias for {flalign} but with
%  an extra switch \cs{ifEQ_multline:} set to true.
%    \begin{macrocode}
\MH_keyval_alias_with_addon:nnnn {EmphEqEnv}{multline}
  {flalign}{\MH_set_boolean_T:n {multline}}
\MH_keyval_alias_with_addon:nnnn {EmphEqEnv}{multline*}
  {flalign*}{\MH_set_boolean_T:n {multline}}
%    \end{macrocode}
%  \begin{macro}{\if_EQ_multline:}
%  \begin{macro}{\EQ_generic_multline:nnnnn}
%  We make a general command that'll provide the correct spacing and
%  then just define the user commands in terms of that single generic
%  one.
%    \begin{macrocode}
\MH_new_boolean:n {multline}
%    \end{macrocode}
%  \begin{macro}{\EQ_multline_modify_linewidth_tag:}
%  \begin{macro}{\EQ_multline_modify_linewidth_notag:}
%  There is a big difference between the spacing in \env{multline}
%  depending on the presence of a tag. Therefore we split it up in
%  two cases. I made this choice because we need these adjustments
%  inside the adjustments for \env{flalign} and that makes it
%  incredibly difficult to see what happens.
%    \begin{macrocode}
\def\EQ_multline_modify_linewidth_tag:{%
  \if_case:w \l_EQ_alignmentmode_int
  \or: % #1:
    \MH_addtolength:dn \l_EQ_temp_linewidth_dim{%
      -\g_EQ_widesttag_dim-\multlinetaggap-\multlinegap}%
  \or:
    \MH_addtolength:dn \l_EQ_temp_linewidth_dim{%
      -\g_EQ_widesttag_dim-\multlinetaggap-\multlinegap}%
  \or:
    \MH_addtolength:dn \l_EQ_temp_linewidth_dim{%
      -\@mathmargin-\g_EQ_widesttag_dim-\multlinetaggap}%
  \or:
    \MH_addtolength:dn
      \l_EQ_temp_linewidth_dim{-\@mathmargin-\multlinegap}%
  \fi:
}
\def\EQ_multline_modify_linewidth_notag:{%
  \if_case:w \l_EQ_alignmentmode_int\or:
    \MH_addtolength:dn \l_EQ_temp_linewidth_dim{-2\multlinegap}
  \or:
    \MH_addtolength:dn \l_EQ_temp_linewidth_dim{-2\multlinegap}
  \or:
    \MH_addtolength:dn
      \l_EQ_temp_linewidth_dim{-\@mathmargin-\multlinegap}%
  \or:
    \MH_addtolength:dn
      \l_EQ_temp_linewidth_dim{-\@mathmargin-\multlinegap}%
  \fi:
}
%    \end{macrocode}
%  \end{macro}
%  \end{macro}
%
%  \begin{environment}{MTmultlined}
%  Save the superior \env{multlined} from \pkg{mathtools} under a new
%  name.
%    \begin{macrocode}
\MH_let:NwN \MTmultlined \multlined
\MH_let:NwN \endMTmultlined \endmultlined
%    \end{macrocode}
%  \end{environment}
%  We make a choice here: the good \env{multline} or the bad?
%    \begin{macrocode}
\MH_if_boolean:nTF {good_mult}{
%    \end{macrocode}
%  The good:
%    \begin{macrocode}
\MH_keyval_alias_with_addon:nnnn {EmphEqEnv}{multline}{gather}
  {\MH_set_boolean_T:n {outer_mult}}
\MH_keyval_alias_with_addon:nnnn {EmphEqEnv}{multline*}{gather*}
  {\MH_set_boolean_T:n {outer_mult}}
\def\EQ_modify_linewidth_iii_mult:{
  \setlength{\l_EQ_temp_linewidth_dim}{%
    \linewidth-\widthof{\EQ_outerbox:{}}}%
  \if_dim:w \g_EQ_widesttag_dim>\z@
    \EQ_multline_modify_linewidth_tag:
  \else:
    \EQ_multline_modify_linewidth_notag:
  \fi:
  \MH_setlength:dn \l_MT_multwidth_dim{\l_EQ_temp_linewidth_dim}
}
\def\EQ_modify_linewidth_iv_mult:{\EQ_modify_linewidth_iii_mult:}
\def\find_displaywidth_iii_mult:{%
  \MH_setlength:dn \l_EQ_displaywidth_dim{\totwidth@}%
  \if_case:w \l_EQ_alignmentmode_int \or: \or: % #1 & #2:
  \or: % #3:
    \MH_addtolength:dn \l_EQ_displaywidth_dim{-\@mathmargin}%
  \or: % #4:
    \MH_addtolength:dn \l_EQ_displaywidth_dim{-\@mathmargin}%
  \fi:}
\def\find_displaywidth_iv_mult:{\find_displaywidth_iii_mult:}
\def\EQ_place_display_iii_mult:{
  \if_case:w \l_EQ_alignmentmode_int
  \or: % #1:
    \if_dim:w \g_EQ_widesttag_dim>\z@
      \abovedisplayshortskip=\abovedisplayskip
      \rlap{
        \kern-.5\linewidth
        \kern\multlinegap
        \EQ_use_mathdisplay:
      }
    \else:
      \EQ_use_mathdisplay:
    \fi:
  \or: % #2:
    \if_dim:w \g_EQ_widesttag_dim>\z@
      \llap{
        \EQ_use_mathdisplay:
        \kern-.5\linewidth
        \kern\multlinegap
      }%
    \else:
      \EQ_use_mathdisplay:
    \fi:
  \or: % #3:
    \EQ_use_mathdisplay:
  \or: % #4:
    \EQ_use_mathdisplay:
  \fi:
}
\def\EQ_place_display_iv_mult: {\EQ_place_display_iii_mult:}
%    \end{macrocode}
%    \begin{macrocode}
}{
%    \end{macrocode}
%  The bad
%    \begin{macrocode}
\def\EQ_generic_multline:nnnnn #1#2#3#4#5{%
%    \end{macrocode}
%  This is a very tricky little detail. If we're in \mode{leqno} mode
%  we need just a we little width inside the \env{flalign}, so we put
%  in the smallest possible amount. Otherwise we wind up in a
%  \begin{verbatim}
%  ! Arithmetic overflow.
%  <recently read> \count@
%
%  l.304 %  \end{empheq}
%  \end{verbatim}
%  If you scroll past this error it will be typeset correctly, but
%  how many users can fix this by themselves?
%    \begin{macrocode}
  \hskip1sp#2#3{{}#5}#4%
  \if:w *#1\relax\else:\notag\fi:}
%    \end{macrocode}
%  \end{macro}
%  \end{macro}
%  \begin{macro}{\multleft}
%  \begin{macro}{\multcenter}
%  \begin{macro}{\multright}
%  Then the commands that use \cs{EQ_generic_multline:nnnnn}. The
%  first three are the general purpose ones that also have a starred
%  version. The other three are for normal use.
%    \begin{macrocode}
\def\multleft{\@ifstar{%
  \MH_if_boolean:nTF {multlined}{
    \def\EQ_next:{\EQ_multlined_left:n}
  }{
    \def\EQ_next:{
      \EQ_generic_multline:nnnnn {*}{}{\mathrlap}{&&&&&}}
  }
  \EQ_next:}{%
  \MH_if_boolean:nTF {multlined}{
    \def\EQ_next:{\EQ_multlined_left:n}%
  }{
    \def\EQ_next:{
      \EQ_generic_multline:nnnnn {}{}{\mathrlap}{&&&&&}}
  }
  \EQ_next:}}%
\def\multcenter{\@ifstar{%
  \MH_if_boolean:nTF {multlined}{
    \def\EQ_next:{\EQ_multlined_center:n}%
  }{
    \def\EQ_next:{
      \EQ_generic_multline:nnnnn {*}{&&}{\mathclap}{&&&}}
  }
  \EQ_next:}{%
  \MH_if_boolean:nTF {multlined}{
    \def\EQ_next:{\EQ_multlined_center:n }%
  }{
    \def\EQ_next:{
      \EQ_generic_multline:nnnnn {}{&&}{\mathclap}{&&&}}
  }
  \EQ_next:}}%
\def\multright{\@ifstar{%
  \MH_if_boolean:nTF {multlined}{
    \def\EQ_next:{\EQ_multlined_right:n }%
  }{
    \def\EQ_next:{
      \EQ_generic_multline:nnnnn {*}{&&&&&}{\mathllap}{}}
  }
  \EQ_next:}{%
  \MH_if_boolean:nTF {multlined}{
    \def\EQ_next:{\EQ_multlined_right:n }%
  }{
    \def\EQ_next:{
      \EQ_generic_multline:nnnnn {}{&&&&&}{\mathllap}{}}
  }
  \EQ_next:}}
%    \end{macrocode}
%  \end{macro}
%  \end{macro}
%  \end{macro}
%  \begin{macro}{\multfirst}
%  \begin{macro}{\multmiddle}
%  \begin{macro}{\multlast}
%  If we're using the \mode{leqno} option, the equation number should
%  be placed in the first line.
%    \begin{macrocode}
\def\multfirst{%
  \MH_if_boolean:nTF {multlined}{
     \def\EQ_next:{\EQ_multlined_left:n }%
  }{
    \if_case:w \l_EQ_alignmentmode_int
    \or: % #1:
      \def\EQ_next:{\multleft}%
    \or: % #2:
      \def\EQ_next:{\multleft*}%
    \or: % #3:
      \def\EQ_next:{\multleft}%
    \or: % #4:
      \def\EQ_next:{\multleft*}%
    \fi:
  }
  \EQ_next:}
%    \end{macrocode}
%  Pretend the star was never there.
%    \begin{macrocode}
\def\multmiddle{%
  \def\EQ_next:{%
    \MH_if_boolean:nTF {multlined}{
      \expandafter\EQ_multlined_center:n
    }{
      \expandafter\multcenter
    }
  }%
  \@ifstar{\EQ_next:}{\EQ_next:}}
%    \end{macrocode}
%  If we're  using the \mode{reqno} option, the equation number
%  should be placed in the last line.
%    \begin{macrocode}
\def\multlast{%
  \MH_if_boolean:nTF {multlined}{
     \def\EQ_next:{\EQ_multlined_right:n }%
  }{
    \if_case:w \l_EQ_alignmentmode_int
    \or: % #1:
      \def\EQ_next:{\multright*}%
    \or: % #2:
      \def\EQ_next:{\multright}%
    \or: % #3:
      \def\EQ_next:{\multright*}%
    \or: % #4:
      \def\EQ_next:{\multright}%
    \fi:
  }
  \EQ_next:}
%    \end{macrocode}
%  \end{macro}
%  \end{macro}
%  \end{macro}
%  \begin{macro}{\multlinedwidth}
%  Here begins the code for \env{multlined}.
%    \begin{macrocode}
\MH_new_boolean:n {multlined}
\newdimen\multlinedwidth
\MH_setlength:dn \multlinedwidth{.7\linewidth}
%    \end{macrocode}
%  \end{macro}
%  \begin{environment}{multlined}
%  \changes{v2.10}{2004/07/26}{Disallow spaces before optional
%  arguments.}
%  \begin{macro}{\EQ_multlined_I:w}
%  \begin{macro}{\EQ_multlined_II:w}
%  \begin{macro}{\EQ_multlined_III:nn}
%  Some trickery as we want to scan for two optional arguments that
%  may be interchanged.
%    \begin{macrocode}
\renewenvironment{multlined}%
  {\MH_nospace_ifnextchar:Nnn [  {\EQ_multlined_I:w}
                                 {\EQ_multlined_I:w[c]}
  }
  { \endaligned }
\def\EQ_multlined_I:w[#1]{%
  \MH_nospace_ifnextchar:Nnn[  {\EQ_multlined_II:w[#1]}
                               {\EQ_multlined_II:w[#1][\multlinedwidth]}}
\def\EQ_multlined_II:w[#1][#2]{%
%    \end{macrocode}
%  We tell \LaTeX\ that we're in \env{multlined}.
%    \begin{macrocode}
  \MH_set_boolean_T:n {multlined}
%    \end{macrocode}
%  \changes{v2.10}{2004/07/26}{Fixed restoration of \cmd{\\}}
%  Restore the meaning of \cmd{\\} inside \env{multlined}, else it
%  wouldn't work in the \env{equation} environment.
%    \begin{macrocode}
  \MH_let:NwN \Let@\MT_saved_Let:
%    \end{macrocode}
%  Then check the first optional argument and call
%  \cs{EQ_multlined_III:w}, which sets \cs{multlinewidth} if
%  specified and calls \cs{start@aligned} (an \env{alignedat} with
%  three columns).
%    \begin{macrocode}
  \if:w t#1\relax \EQ_multlined_III:nn {#1}{#2}%
  \else:
    \if:w b#1\relax \EQ_multlined_III:nn {#1}{#2}%
    \else:
      \if:w c#1\relax \EQ_multlined_III:nn {#1}{#2}%
      \else:
        \EQ_multlined_III:nn {#2}{#1}%
      \fi:
    \fi:
  \fi:
  \mkern-\thinmuskip
  \EQ_next:}
\def\EQ_multlined_III:nn #1#2{%
  \def\EQ_next:{\setlength\multlinedwidth{#2}%
                          \start@aligned{#1}{\thr@@}}}
%    \end{macrocode}
%  \end{macro}
%  \end{macro}
%  \end{macro}
%  \end{environment}
%  \begin{macro}{\EQ_multlined_left:n}
%  \begin{macro}{\EQ_multlined_center:n}
%  \begin{macro}{\EQ_multlined_right:n}
%  The internal versions of \cs{multleft} etc.\ adapted for
%  \env{multlined}.
%    \begin{macrocode}
\def\EQ_multlined_left:n #1{%
  \mathrlap{#1}&\hskip.5\multlinedwidth&&\hskip.5\multlinedwidth&&}
\def\EQ_multlined_center:n #1{%
  &\hskip.5\multlinedwidth&&\mathclap{#1}\hskip.5\multlinedwidth&&}
\def\EQ_multlined_right:n #1{%
  &\hskip.5\multlinedwidth&&\hskip.5\multlinedwidth&&\mathllap{#1}}
%    \end{macrocode}
%  \end{macro}
%  \end{macro}
%  \end{macro}
%
%  End of bad multline.
%    \begin{macrocode}
}
%    \end{macrocode}
%
%  \subsubsection{The Code for \env{flalign} \& \env{multline}}
%    \begin{macrocode}
\EQ_new_equationtype:nnn {EmphEqEnv}{flalign}{5}
\EQ_new_equationtype:nnn {EmphEqEnv}{flalign*}{6}
\def\EQ_modify_linewidth_v:{%
  \setlength{\l_EQ_temp_linewidth_dim}{%
    \linewidth-\l_EQ_totalwidth_dim-\widthof{\EQ_outerbox:{}}}%
  \if_dim:w \g_EQ_widesttag_dim>\z@
    \MH_if_boolean:nTF {multline}{
      \EQ_multline_modify_linewidth_tag:
    }{
      \if_case:w \l_EQ_alignmentmode_int
      \or: % #1:
        \MH_addtolength:dn  \l_EQ_temp_linewidth_dim
          {-\g_EQ_widesttag_dim-\mintagsep}
      \or: % #2:
        \MH_addtolength:dn \l_EQ_temp_linewidth_dim
          {-\g_EQ_widesttag_dim-\mintagsep}%
      \or: % #3:
        \MH_addtolength:dn \l_EQ_temp_linewidth_dim
          {-\g_EQ_widesttag_dim-\mintagsep}
      \or: % #4:
        \MH_addtolength:dn \l_EQ_temp_linewidth_dim
          {-\@mathmargin}
      \fi:
    }
  \else:
    \MH_if_boolean:nT {multline}{
      \EQ_multline_modify_linewidth_notag:
    }
  \fi:}
\def\EQ_modify_linewidth_vi:{\EQ_modify_linewidth_v:}
\def\find_displaywidth_v:{%
  \MH_setlength:dn
    \l_EQ_displaywidth_dim{\l_EQ_temp_linewidth_dim}}
\def\find_displaywidth_vi:{\find_displaywidth_v:}
\def\EQ_place_display_v:{%
  \if_case:w \l_EQ_alignmentmode_int
  \or: % #1:
    \if_dim:w \g_EQ_widesttag_dim>\z@
      \MH_if_boolean:nTF {multline}{
        \rlap{\kern\multlinegap  \kern-.5\linewidth
          \EQ_use_mathdisplay:
          \kern-\mintagsep  \kern\multlinetaggap}%
      }{
        \rlap{\kern-.5\linewidth\EQ_use_mathdisplay:}%
      }
    \else:
      \EQ_use_mathdisplay:
    \fi:
  \or: % #2:
    \if_dim:w \g_EQ_widesttag_dim>\z@
      \MH_if_boolean:nTF {multline}{
        \llap{\kern-\mintagsep \kern\multlinetaggap
          \EQ_use_mathdisplay:
          \kern\multlinegap  \kern-.5\linewidth}%
      }{
        \llap{\EQ_use_mathdisplay:\kern-.5\linewidth}%
      }
    \else:
      \EQ_use_mathdisplay:
    \fi:
  \or: % #3:
    \MH_if_boolean:nF {multline}{ \kern-\multlinetaggap }
    \EQ_use_mathdisplay:
  \or: % #4:
    \if_dim:w \g_EQ_widesttag_dim>\z@
    \else:
      \MH_if_boolean:nF {multline}{ \kern-\@mathmargin }
    \fi:
    \EQ_use_mathdisplay:
  \fi:
  \MH_set_boolean_F:n {multline}}
\def\EQ_place_display_vi:{\EQ_place_display_v:}
%    \end{macrocode}
%
%  \subsection{The \env{alignat} Family}
%  This is just as easy as \env{gather}. Actually only one line is
%  different from \env{gather}!
%    \begin{macrocode}
\EQ_new_equationtype_arg:nnn {EmphEqEnv}{alignat}{7}
\EQ_new_equationtype_arg:nnn {EmphEqEnv}{alignat*}{8}
\def\EQ_modify_linewidth_vii:{}
\def\EQ_modify_linewidth_viii:{\EQ_modify_linewidth_vii:}
\def\find_displaywidth_vii:{%
  \MH_setlength:dn \l_EQ_displaywidth_dim{\totwidth@}%
  \if_case:w \l_EQ_alignmentmode_int
  \or: % #1:
  \or: % #2:
  \or: % #3:
    \MH_addtolength:dn \l_EQ_displaywidth_dim{-\eqnshift@}%
  \or: % #4:
    \MH_addtolength:dn \l_EQ_displaywidth_dim{-\@mathmargin}%
  \fi:}
\def\find_displaywidth_viii:{\find_displaywidth_vii:}
\def\EQ_place_display_vii:{\EQ_use_mathdisplay:}
\def\EQ_place_display_viii:{\EQ_place_display_vii:}
%    \end{macrocode}
%  \section{The User Interface}
%  The different keys.
%    \begin{macrocode}
\define@key{\EQ_options_name:}{box}{\def\EQ_outerbox:{#1}}
\MH_keyval_alias:nnn {\EQ_options_name:}{outerbox}{box}
%    \end{macrocode}
%  \begin{macro}{\EQ_marginbox:nn}
%  The margin box. Uwe Siart wanted it.
%    \begin{macrocode}
\def\EQ_marginbox:nn #1#2{%
  \settowidth{\@tempdima}{#2}%
  \makebox[\z@]{\hspace{\@tempdima}#1{#2}}}
\define@key{\EQ_options_name:}{marginbox}{%
  \def\EQ_outerbox:{\EQ_marginbox:nn {#1}}}
%    \end{macrocode}
%  \end{macro}
%  Almost self-explanatory, isn't?
%    \begin{macrocode}
\define@key{\EQ_options_name:}{innerbox}{\def\EQ_innerbox:{#1}}
\define@key{\EQ_options_name:}{left}{\def\EQ_left_side:{#1}}
\define@key{\EQ_options_name:}{right}{\def\EQ_right_side:{#1}}
%    \end{macrocode}
%  \begin{macro}{\empheqset}
%  Just a user friendly way to set keys. Uwe Siart again.
%    \begin{macrocode}
\newcommand*\empheqset[1]{\setkeys{\EQ_options_name:}{#1}}
%    \end{macrocode}
%  \end{macro}
%  The old keys from the original version are no longer supported.
%  Instead they give an error message (RM):
%    \begin{macrocode}
\def\EQ_upgrade_error:{%
  \PackageError{empheq}
    {You~ are~ using~ the~ old~ syntax!\MessageBreak
    The~ `empheq'~ environment~ takes~ a~ mandatory~
    argument~ now.\MessageBreak
    You~ need~ to~ exit~ and~ change~ your~ source~ file.}
    {It~ won't~ work,~ trust me.~ Press~ `X'~ now.}}
\define@key{\EQ_options_name:}{boxtype}{\EQ_upgrade_error:}
\define@key{\EQ_options_name:}{Left}{\EQ_upgrade_error:}
\define@key{\EQ_options_name:}{Right}{\EQ_upgrade_error:}
%    \end{macrocode}
% \begin{macro}{\empheqeqrefbase}
%   This macro is used to locally redefine \cs{eqref}, it will just
%   point at \cs{textup}, the user can then redefine it if nessary.
%    \begin{macrocode}
\let\empheqeqrefbase\textup
%    \end{macrocode}
%   
% \end{macro}
%  \begin{environment}{empheq}
%  The environment.
%    \begin{macrocode}
\newenvironment{empheq}[2][]{%
%    \end{macrocode}
% \begin{macro}{\EQ_saved_maketag}
% \changes{v2.13}{2013/02/11}{Fixing \cs{eqref}}
% We start with a small caveat: During processing \cs{maketag@@@} is
% redefined, enabling us to separate math and eqn numbers. But the
% user may have used \cs{eqref} in a comment in the math, and
% \cs{eqref} use \cs{maketag@@@}\dots{} First save the original \cs{maketag@@@}:
%    \begin{macrocode}
  \MH_let:NwN\EQ_saved_maketag: \maketag@@@%
%    \end{macrocode}
% Next redefine \cs{eqref} to locally bring back the original:
%    \begin{macrocode}
  \renewcommand\eqref[1]{\empheqeqrefbase{%
       \let\maketag@@@\EQ_saved_maketag:%
       \tagform@{\ref{##1}}}}
  \setkeys{EmphEqEnv}{#2}\setkeys{\EQ_options_name:}{#1}%
  \EmphEqMainEnv}{\endEmphEqMainEnv}
%    \end{macrocode}
%  \end{macro}
% \end{environment}
%  \begin{macro}{\DeclareLeftDelimiter}
%  \begin{macro}{\EQ_declare_left_delimiter:nnnn}
%  \begin{macro}{\DeclareRightDelimiter}
%  \begin{macro}{\EQ_declare_right_delimiter:nnnn}
%  The delimiter interface.
%  \changes{v0.7}{2003/05/11}{Introduced delimiter interface}
%  \changes{v1.05d}{2004/02/26}{Changed the definitions.}
%  We now use the \cs{EQ_tag_box} to calculate the height but
%  adjust the vertical spacing. All this to prevent running
%  \cs{EQ_typeset_display:} twice.
%    \begin{macrocode}
\newcommand*\DeclareLeftDelimiter[2][]{%
  \EQ_declare_left_delimiter:nnnn {#1}{#2}{}{}%
  \EQ_declare_left_delimiter:nnnn {#1}{#2}{big}{\EQ_innerbox:}}
\def\EQ_declare_left_delimiter:nnnn #1#2#3#4{%
  \@namedef{empheq#3\expandafter\@gobble\string#2}{%
    \left#2%
    \mkern-\thinmuskip
    \vphantom{#4{\parbox{.5\maxdimen}{%
      \EQ_predisplay_adjustment:
      \EQ_prevent_vertical_space:
      \usebox{\EQ_tag_box}%
      \EQ_postdisplay_adjustment:}}}%
    \right.%
    \kern-\nulldelimiterspace #1}}
\newcommand*\DeclareRightDelimiter[2][]{%
  \EQ_declare_right_delimiter:nnnn {#1}{#2}{}{}%
  \EQ_declare_right_delimiter:nnnn {#1}{#2}{big}{\EQ_innerbox:}}
\def\EQ_declare_right_delimiter:nnnn #1#2#3#4{%
  \@namedef{empheq#3\expandafter\@gobble\string#2}{%
    \kern-\nulldelimiterspace
    #1%
    \left.%
    \vphantom{#4{\parbox{.5\maxdimen}{%
      \EQ_predisplay_adjustment:
      \EQ_prevent_vertical_space:
      \usebox{\EQ_tag_box}%
      \EQ_postdisplay_adjustment:}}}%
    \mkern-\thinmuskip
    \right#2}}
%    \end{macrocode}
%  \end{macro}
%  \end{macro}
%  \end{macro}
%  \end{macro}
%  The actual declarations.
%    \begin{macrocode}
\DeclareLeftDelimiter{\lbrace}
\DeclareRightDelimiter{\rbrace}
\DeclareLeftDelimiter{\lbrack}
\DeclareRightDelimiter{\rbrack}
\DeclareLeftDelimiter{\langle}
\DeclareRightDelimiter{\rangle}
\DeclareLeftDelimiter{\lvert}
\DeclareRightDelimiter{\rvert}
\DeclareLeftDelimiter{\lVert}
\DeclareRightDelimiter{\rVert}
\DeclareLeftDelimiter{\lfloor}
\DeclareRightDelimiter{\rfloor}
\DeclareLeftDelimiter{\lceil}
\DeclareRightDelimiter{\rceil}
\DeclareLeftDelimiter{\lparen}
\DeclareRightDelimiter{\rparen}
%    \end{macrocode}
%
%  \section{Support for Other Packages}
%
%  Support for various other packages is found in this section. The
%  \pkg{ntheorem} package now has a dedicated section (\S
%  \vref{sec:impl:ntheorem}), as it now a module.
%
%
%
%    \begin{macrocode}
\def\EQ_typeset_tags_hook:{}
\def\EQ_typeset_display_hook:{}
%    \end{macrocode}
%
%  \subsection{\pkg{showkeys}}
%  \changes{v2.10}{2004/07/26}{Added support for the \pkg{showkeys}
%  package}
%  We just disable \pkg{showkeys} when typesetting the labels.
%    \begin{macrocode}
\g@addto@macro\EQ_typeset_display_hook:{
  \def\SK@@label#1>#2\SK@{}
}
%    \end{macrocode}
%  If we overload the \AmS{} equation types we should still be able
%  to use \pkg{showkeys}.
%    \begin{macrocode}
\@namedef{SK@AmSequation}  {\SK@equationtrue}
\@namedef{SK@AmSequation*} {\SK@equationtrue}
\@namedef{SK@AmSalign}     {\SK@equationtrue}
\@namedef{SK@AmSalign*}    {\SK@equationtrue}
\@namedef{SK@AmSalignat}   {\SK@equationtrue}
\@namedef{SK@AmSalignat*}  {\SK@equationtrue}
\@namedef{SK@AmSgather}    {\SK@equationtrue}
\@namedef{SK@AmSgather*}   {\SK@equationtrue}
\@namedef{SK@AmSmultline}  {\SK@equationtrue}
\@namedef{SK@AmSmultline*} {\SK@equationtrue}
\@namedef{SK@AmSflalign}   {\SK@equationtrue}
\@namedef{SK@AmSflalign*}  {\SK@equationtrue}
%    \end{macrocode}
%
%  \subsection{\pkg{hyperref}}
%
%
%  A little trick to avoid the (in)famous warning
%  \begin{verbatim}
%  ! pdfTeX warning (ext4): destination with the same identifier
%  (name{equation.1a }) has been already used, duplicate ignored
%  <to be read again>
%                     \endgroup \set@typeset@protect
%  \end{verbatim}
%  This doesn't require \pkg{hyperref}.
%    \begin{macrocode}
\AtBeginDocument{
  \@ifundefined{Hy@linkcounter}{\newcount\Hy@linkcounter}{}
}
\g@addto@macro\EQ_typeset_display_hook:{
  \MH_let:NwN \hyper@refstepcounter\@gobble
}
%    \end{macrocode}
% 
% \subsection{\pkg{nccmath}}
%
% Must disable this macro, otherwise the spacing is wrong.
%    \begin{macrocode}
 \g@addto@macro\EQ_typeset_display_hook:{
   \let\NCC@ignorepar\relax
 }
 \g@addto@macro\EQ_typeset_tags_hook:{
   \let\NCC@ignorepar\relax
 }
%    \end{macrocode}
%
%  \subsection{Basic \pkg{ntheorem} support}
%    \begin{macrocode}
\g@addto@macro\EQ_typeset_tags_hook:{
  \MH_let:NwN\TagsPlusEndmarks\relax
  \MH_let:NwN\RestoreTags\relax
}
\g@addto@macro\EQ_typeset_display_hook:{
  \MH_let:NwN\TagsPlusEndmarks\relax
  \MH_let:NwN\RestoreTags\relax
}
%    \end{macrocode}
%  \subsection{Basic \LaTeXe{} support}
%  Gobble the \cs{ltx@label} and disable most error messages.
%  \changes{v2.00}{2004/04/14}{Make a special label-gobbling command
%  to support \pkg{ntheorem} even more}
%    \begin{macrocode}
\g@addto@macro\EQ_typeset_display_hook:{
  \MH_let:NwN\ltx@label\MT_gobblelabel:w
  \MH_let:NwN \GenericError \@gobblefour
}
%    \end{macrode}
%  \subsection{Support for \pkg{fancybox}}
%  First check if \pkg{fancybox} is loaded and then see if
%  \pkg{color} is loaded. If the latter is loaded we define
%  \texttt{shadowcolor} else we let \cs{textcolor} gobble the first
%  argument (the color).
%    \begin{macrocode}
\AtBeginDocument{%
  \@ifpackageloaded{fancybox}{%
    \@ifpackageloaded{color}{\definecolor{shadowcolor}{rgb}{0,0,0}}%
    {\newcommand*\textcolor[2]{\normalcolor{#2}}}%
%    \end{macrocode}
%  \begin{macro}{\shadowbox}
%  Then define the \cs{shadowbox}. Use a conditional to let us know if
%  it was starred or not.
%    \begin{macrocode}
    \MH_new_boolean:n {shadowbox_star}
    \renewcommand*\shadowbox{%
      \@ifstar{\MH_set_boolean_T:n {shadowbox_star}
               \VerbBox\EQ_shadowbox:n}
              {\MH_set_boolean_F:n {shadowbox_star}
               \VerbBox\EQ_shadowbox:n}}
%    \end{macrocode}
%  \end{macro}
%  \begin{macro}{\EQ_shadowbox:n}
%  The internal version of \cs{shadowbox}.
%    \begin{macrocode}
    \def\EQ_shadowbox:n #1{%
      \sbox{\z@}{\fbox{#1}}%
%    \end{macrocode}
%  The difference between the two versions is that \cs{shadowbox}
%  aligns the bottom of the shadow at the baseline whereas
%  \cs{shadowbox*} aligns the argument on the baseline, so for the
%  starred version the trick is to raise the box by the depth of the
%  argument plus the size of the shadow.
%    \begin{macrocode}
      \MH_if_boolean:nTF {shadowbox_star}{
        \MH_setlength:dn \@tempdima{\z@}%
      }{
        \MH_setlength:dn \@tempdima{\dp\z@+\shadowsize}%
      }
      \raisebox{\@tempdima}{%
        \makebox[\z@][l]{%
          \kern\shadowsize \textcolor{shadowcolor}{%
            \rule[-\dp\z@-\shadowsize]{\wd\z@}{\shadowsize}}%
        }%
        \usebox\z@\textcolor{shadowcolor}{%
          \rule[-\dp\z@-\shadowsize]{\shadowsize}{\ht\z@+\dp\z@}}}}%
  }{}%
}
%    \end{macrocode}
%  \end{macro}
%
%
%  \section{The \opt{overload} option}
%  \begin{macro}{\EQ_overload_ams_begin:n}
%  A helper for saving snapshots:
%    \begin{macrocode}
\def\EQ_overload_ams_begin:n #1{%
  \MH_let:cc {AmS#1}{#1}%
  \MH_let:cc {AmS#1*}{#1*}
}
%    \end{macrocode}
%  \end{macro}
%  \begin{macro}{\EQ_overload_ams_end:n}
%  All but one of the aligned variations must be set to the original
%    \begin{macrocode}
\def\EQ_overload_ams_end:n #1{%
  \MH_let:cc {endAmS#1}{endAmSalign}
}
%    \end{macrocode}
%  \end{macro}
%  \begin{environment}{AmSequation}
%  \begin{environment}{AmSequation*}
%  \changes{v2.10}{2004/07/26}{Moved redefinitions to
%  \cs{AtBeginDocument} to give \pkg{hyperref} a chance}
%  Before we start redefining environments, we'll need a basic
%  \env{equation} environment that is not redefined.
%    \begin{macrocode}
\AtBeginDocument{
  \EQ_overload_ams_begin:n {equation}
  \MH_let:cc {endAmSequation}{endequation}
  \MH_let:cc {endAmSequation*}{endequation*}
%    \end{macrocode}
%  \end{environment}
%  \end{environment}
%  Lets start overloading.
%    \begin{macrocode}
  \MH_if_boolean:nT {overload_ams}{
%    \end{macrocode}
%  \begin{macro}{\EQ_renew_ams_environment:n}
%  \changes{v2.10}{2004/07/26}{Disallow spaces before the optional
%  argument.}
%  Then we can renew the environments. Syntax is
%  \cs{begin}\marg{\AmS{} env}\oarg{markup}. Exception to this rule
%  is \env{alignat} (see further down).
%    \begin{macrocode}
    \def\EQ_renew_ams_environment:n #1{%
      \MaybeMHPrecedingSpacesOff
      \renewenvironment{#1}[1][]{%
%    \end{macrocode}
% \changes{v2.13}{2013/02/11}{Adding \cs{eqref} fix}
% Remember to add \cs{eqref} fix to the overloaded envs as well.
%    \begin{macrocode}
        \MH_let:NwN\EQ_saved_maketag: \maketag@@@%
        \renewcommand\eqref[1]{\empheqeqrefbase{%
            \let\maketag@@@\EQ_saved_maketag:%
            \tagform@{\ref{####1}}}}
        \setkeys{\EQ_options_name:}{##1}%
        \setkeys{EmphEqEnv}{AmS#1}%
        \EmphEqMainEnv
      }{\endEmphEqMainEnv}
      \MHPrecedingSpacesOn
    }
%    \end{macrocode}
%  \end{macro}
%  Add equation type: \emph{Must} have same number as original type!
%    \begin{macrocode}
    \EQ_new_equationtype:nnn     {EmphEqEnv}{AmSalign}   {1}
    \EQ_new_equationtype:nnn     {EmphEqEnv}{AmSalign*}  {2}
    \EQ_new_equationtype:nnn     {EmphEqEnv}{AmSgather}  {3}
    \EQ_new_equationtype:nnn     {EmphEqEnv}{AmSgather*} {4}
    \EQ_new_equationtype:nnn     {EmphEqEnv}{AmSflalign} {5}
    \EQ_new_equationtype:nnn     {EmphEqEnv}{AmSflalign*}{6}
    \EQ_new_equationtype_arg:nnn {EmphEqEnv}{AmSalignat} {7}
    \EQ_new_equationtype_arg:nnn {EmphEqEnv}{AmSalignat*}{8}
%    \end{macrocode}
%  Then overload by saving the snapshots of the old definitions. We
%  do it via \cs{AtBeginDocument}, because a package like
%  \pkg{ntheorem} redefines \cs{endgather} etc.
%
%  First the \env{gather} and \env{equation} families.
%    \begin{macrocode}
    \EQ_overload_ams_begin:n {gather}
    \MH_let:cc {endAmSgather}{endgather}
    \MH_let:cc {endAmSgather*}{endAmSgather}
    \EQ_renew_ams_environment:n {gather}
    \EQ_renew_ams_environment:n {gather*}
    \EQ_overload_ams_begin:n {equation}
    \MH_keyval_alias_with_addon:nnnn {EmphEqEnv} {AmSequation}
      {AmSgather}{\EQ_disable_linebreak:}
    \MH_keyval_alias_with_addon:nnnn {EmphEqEnv}{AmSequation*}
      {AmSgather*}{\EQ_disable_linebreak:}
    \EQ_renew_ams_environment:n {equation}
    \EQ_renew_ams_environment:n {equation*}
%    \end{macrocode}
%  The \env{align} family.
%    \begin{macrocode}
    \EQ_overload_ams_begin:n {align}
    \MH_let:cc {endAmSalign}{endalign}
    \EQ_overload_ams_end:n {align*}
    \EQ_renew_ams_environment:n {align}
    \EQ_renew_ams_environment:n {align*}
%    \end{macrocode}
%  The \env{flalign} family.
%    \begin{macrocode}
    \EQ_overload_ams_begin:n {flalign}
    \EQ_overload_ams_end:n {flalign}
    \EQ_overload_ams_end:n {flalign*}
    \EQ_renew_ams_environment:n {flalign}
    \EQ_renew_ams_environment:n {flalign*}
%    \end{macrocode}
%  The \env{alignat} family.
%    \begin{macrocode}
    \EQ_overload_ams_begin:n {alignat}
    \EQ_overload_ams_end:n {alignat}
    \EQ_overload_ams_end:n {alignat*}
%    \end{macrocode}
%  \begin{macro}{\EQ_alignat_optional:w}
%  \changes{v2.10}{2004/07/26}{Disallow spaces before the optional
%  argument.}
%  \begin{environment}{alignat}
%  \begin{environment}{alignat*}
%  This one is a little different because we want a syntax like
%  \cs{begin\{alignat\}}\linebreak[0]\marg{cols}\oarg{markup}, so we
%  have to make a separate command for testing the trailing optional
%  argument.
%    \begin{macrocode}
    \MaybeMHPrecedingSpacesOff
    \newcommand*\EQ_alignat_optional:w [1][]{
      \setkeys{\EQ_options_name:}{#1}
      \EmphEqMainEnv}
    \MHPrecedingSpacesOn
%    \end{macrocode}
% \changes{v2.13}{2013/02/11}{Adding \cs{eqref} fix}
% Remember to add \cs{eqref} fix to the overloaded envs as well.
%    \begin{macrocode}
    \renewenvironment{alignat}[1]{
      \MH_let:NwN\EQ_saved_maketag: \maketag@@@%
      \renewcommand\eqref[1]{\empheqeqrefbase{%
          \let\maketag@@@\EQ_saved_maketag:%
          \tagform@{\ref{##1}}}}
      \setkeys{EmphEqEnv}{AmSalignat=#1}
      \EQ_alignat_optional:w}
      {\endEmphEqMainEnv}
    \renewenvironment{alignat*}[1]{%
      \MH_let:NwN\EQ_saved_maketag: \maketag@@@%
      \renewcommand\eqref[1]{\empheqeqrefbase{%
          \let\maketag@@@\EQ_saved_maketag:%
          \tagform@{\ref{##1}}}}
      \setkeys{EmphEqEnv}{AmSalignat*=#1}%
      \EQ_alignat_optional:w}%
      {\endEmphEqMainEnv}
%    \end{macrocode}
%  \end{environment}
%  \end{environment}
%  \end{macro}
%  \begin{macro}{\[}
%  \begin{macro}{\]}
%  Then restore \cs{[} and \cs{]}.
%    \begin{macrocode}
    \DeclareRobustCommand{\[}{\begin{AmSequation*}}
    \DeclareRobustCommand{\]}{\end{AmSequation*}}
%    \end{macrocode}
%  \end{macro}
%  \end{macro}
%  Finally a string of aliases so that we can blend \opt{overload}
%  syntax with regular syntax.
%    \begin{macrocode}
    \MH_keyval_alias:nnn {EmphEqEnv}{gather}  {AmSgather}
    \MH_keyval_alias:nnn {EmphEqEnv}{gather*} {AmSgather*}
    \MH_keyval_alias:nnn {EmphEqEnv}{align}   {AmSalign}
    \MH_keyval_alias:nnn {EmphEqEnv}{align*}  {AmSalign*}
    \MH_keyval_alias:nnn {EmphEqEnv}{flalign} {AmSflalign}
    \MH_keyval_alias:nnn {EmphEqEnv}{flalign*}{AmSflalign*}
    \MH_keyval_alias:nnn {EmphEqEnv}{alignat} {AmSalignat}
    \MH_keyval_alias:nnn {EmphEqEnv}{alignat*}{AmSalignat*}
    \MH_if_boolean:nT {good_mult}{
      \EQ_overload_ams_begin:n {multline}
      \MH_let:cc {endAmSmultline}{endmultline}
      \MH_let:cc {endAmSmultline*}{endAmSmultline}
      \MH_keyval_alias_with_addon:nnnn {EmphEqEnv}{AmSmultline}
        {AmSgather}{\MH_set_boolean_T:n {outer_mult}}
      \MH_keyval_alias_with_addon:nnnn {EmphEqEnv}{AmSmultline*}
        {AmSgather*}{\MH_set_boolean_T:n {outer_mult}}
      \EQ_renew_ams_environment:n {multline}
      \EQ_renew_ams_environment:n {multline*}
    }
%    \end{macrocode}
%  End of \opt{overload}. We go into \opt{overload2}\ldots This
%  is dangerous, but will probably work. The macro \cs{[} is now
%  fragile except if you're running \eTeX{} as engine for \LaTeX.
%    \begin{macrocode}
    \MH_if_boolean:nT {overload_amsII}{
      \MH_protected:\def\[#1\]{\begin{equation*} #1\end{equation*}}
%    \end{macrocode}
%  Then we add the \env{multline} family.
%    \begin{macrocode}
      \MH_if_boolean:nF {good_mult}{
        \EQ_overload_ams_begin:n {multline}
        \MH_keyval_alias_with_addon:nnnn {EmphEqEnv}{AmSmultline}
          {AmSflalign}{\MH_set_boolean_T:n {multline}}
        \MH_keyval_alias_with_addon:nnnn {EmphEqEnv}{AmSmultline*}
          {AmSflalign*}{\MH_set_boolean_T:n {multline}}
        \EQ_renew_ams_environment:n {multline}
        \EQ_renew_ams_environment:n {multline*}
      }
    }
  }
}
%    \end{macrocode}
%  End of \cs{AtBeginDocument} additions.
%
%  If we need to load \pkg{ntheorem} support this is where it
%  happens. Otherwise we just stop here.
%    \begin{macrocode}
\if\g@EQ@ntheorem@support@bool
\else 
  \expandafter\endinput
\fi
%    \end{macrocode}
%
%  \section{Patching Things up for \pkg{ntheorem}}
%  \label{sec:impl:ntheorem}
%
%  Although \pkg{ntheorem} provides a lot of useful features it
%  does however ruin things in the \pkg{amsmath} environments,
%  because of the alternative syntax of its \cs{label} command.
%  \begin{macro}{\default_raisetag}
%  \begin{macro}{\raisetag}
%  \begin{macro}{\raise@tag}
%  \changes{v1.06}{2004/03/13}{Fixed \pkg{amsmath} bug \# 3624}
%  This corrects a bug\footnote{See
%  \url{http://www.latex-project.org/cgi-bin/ltxbugs2html?pr=amslatex/3624}}
%  in \pkg{amsmath} that could cause tags to cross page boundaries
%  without warning. The \cs{nobreak} is the addition.
%    \begin{macrocode}
\def\default_raisetag:{\nobreak}
\def\raisetag#1{
  \skip@#1\relax
  \xdef\raise@tag{\default_raisetag:
                  \vskip\iftagsleft@\else:-\fi:\the\skip@\relax}}
\MH_let:NwN\raise@tag\default_raisetag:
%    \end{macrocode}
%  \end{macro}
%  \end{macro}
%  \end{macro}
%  \begin{macro}{\displ@y@}
%  Patch it so that it doesn't set \cs{raise@tag} to \cs{@empty}
%  globally.
%    \begin{macrocode}
\def\displ@y@{\@display@init{%
  \global\column@\z@ \global\dspbrk@lvl\m@ne
  \global\tag@false \global\MH_let:NwN\raise@tag\default_raisetag:
}}
%    \end{macrocode}
%  \end{macro}
%  \begin{macro}{\MT_gobblelabel:w}
%  \changes{v2.00}{2004/04/14}{Added \cs{MT_gobblelabel:w}}
%  \begin{macro}{\EQ_gobbleoptional:w}
%  \changes{v2.00}{2004/04/14}{Added \cs{EQ_gobbleoptional:w}}
%  Macros for gobbling the `unorthodox' \cs{label} syntax of
%  \pkg{ntheorem}. Also overrides the normal \cs{label}-gobbling.
%    \begin{macrocode}
\def\MT_gobblelabel:w #1{\@ifnextchar[{\EQ_gobbleoptional:w}{}}
\def\EQ_gobbleoptional:w [#1]{}
%    \end{macrocode}
%  \end{macro}
%  \end{macro}
%  \subsection{Fixing \env{gather}}
%  The measuring phase of \env{gather} is patched by letting
%  \cs{label} be a copy of the internal \cs{MT_gobblelabel:w}
%  instead of \cs{@gobble}.
%    \begin{macrocode}
\def\gmeasure@#1{%
  \begingroup
    \measuring@true
    \totwidth@\z@
    \global\MH_let:NwN\tag@lengths\@empty
    \savecounters@
    \setbox\@ne\vbox{%
      \everycr{\noalign{\global\tag@false
%    \end{macrocode}
%  The patch for \cs{raisetag}.
%    \begin{macrocode}
        \global\MH_let:NwN\raise@tag\default_raisetag:
        \global\column@\z@}}%
%    \end{macrocode}
%  The patch for \cs{label}.
%    \begin{macrocode}
      \MH_let:NwN\label\MT_gobblelabel:w
      \halign{%
        \setboxz@h{$\m@th\displaystyle{##}$}%
        \if_dim:w \wdz@>\totwidth@
          \global\totwidth@\wdz@
        \fi:
        &\setboxz@h{\strut@{##}}%
        \savetaglength@
        \crcr
        #1%
        \math@cr@@@
      }%
    }%
    \restorecounters@
    \if@fleqn
        \global\advance\totwidth@\@mathmargin
    \fi:
    \iftagsleft@
        \if_dim:w \totwidth@>\displaywidth
            \global\MH_let:NwN\gdisplaywidth@\totwidth@
        \else:
            \global\MH_let:NwN\gdisplaywidth@\displaywidth
        \fi:
    \fi:
  \endgroup
}
%    \end{macrocode}
%
%  \subsection{Fixing \env{align}, \env{alignat}, and \env{flalign}}
%  Just copy the definition of \cs{measure@} from \pkg{amsmath} and
%  change one (1!) line like we did for \env{gather}.
%    \begin{macrocode}
\def\measure@#1{%
  \begingroup
    \measuring@true
    \global\eqnshift@\z@
    \global\alignsep@\z@
    \global\MH_let:NwN\tag@lengths\@empty
    \global\MH_let:NwN\field@lengths\@empty
    \savecounters@
    \global\setbox0\vbox{%
      \MH_let:NwN\math@cr@@@\math@cr@@@align@measure
      \everycr{\noalign{\global\tag@false
%    \end{macrocode}
%  The patch for \cs{raisetag}.
%    \begin{macrocode}
        \global\MH_let:NwN\raise@tag\default_raisetag:
        \global\column@\z@}}%
%    \end{macrocode}
%  The patch.
%    \begin{macrocode}
      \MH_let:NwN\label\MT_gobblelabel:w
      \global\row@\z@
      \tabskip\z@
      \halign{\span\align@preamble\crcr
        #1%
        \math@cr@@@
        \global\column@\z@
        \add@amps\maxfields@\cr
      }%
    }%
    \restorecounters@
    \ifodd\maxfields@
      \global\advance\maxfields@\@ne
    \fi:
    \if_num:w \xatlevel@=\tw@
      \if_num:w \maxfields@<\thr@@
        \MH_let:NwN\xatlevel@\z@
      \fi:
    \fi:
    \setbox\z@\vbox{%
      \unvbox\z@ \unpenalty \global\setbox\@ne\lastbox
    }%
    \global\totwidth@\wd\@ne
    \if@fleqn \global\advance\totwidth@\@mathmargin \fi:
    \global\MH_let:NwN\maxcolumn@widths\@empty
    \begingroup
      \MH_let:NwN\or:\relax
      \loop
        \global\setbox\@ne\hbox{%
          \unhbox\@ne \unskip \global\setbox\thr@@\lastbox
        }%
      \ifhbox\thr@@
        \xdef\maxcolumn@widths{ \or: \the\wd\thr@@ \maxcolumn@widths}%
      \repeat
    \endgroup
    \dimen@\displaywidth
    \advance\dimen@-\totwidth@
    \if_case:w \xatlevel@
      \global\alignsep@\z@
      \MH_let:NwN\minalignsep\z@
      \@tempcntb\z@
      \if@fleqn
        \@tempcnta\@ne
        \global\eqnshift@\@mathmargin
      \else:
        \@tempcnta\tw@
        \global\eqnshift@\dimen@
        \global\divide\eqnshift@\@tempcnta
      \fi:
    \or:
      \@tempcntb\maxfields@
      \divide\@tempcntb\tw@
      \@tempcnta\@tempcntb
      \advance\@tempcntb\m@ne
      \if@fleqn
        \global\eqnshift@\@mathmargin
        \global\alignsep@\dimen@
        \global\divide\alignsep@\@tempcnta
      \else:
        \global\advance\@tempcnta\@ne
        \global\eqnshift@\dimen@
        \global\divide\eqnshift@\@tempcnta
        \global\alignsep@\eqnshift@
      \fi:
    \or:
      \@tempcntb\maxfields@
      \divide\@tempcntb\tw@
      \global\advance\@tempcntb\m@ne
      \global\@tempcnta\@tempcntb
      \global\eqnshift@\z@
      \global\alignsep@\dimen@
      \if@fleqn
        \global\advance\alignsep@\@mathmargin\relax
      \fi:
      \global\divide\alignsep@\@tempcntb
    \fi:
    \if_dim:w \alignsep@<\minalignsep\relax
      \global\alignsep@\minalignsep\relax
      \if_dim:w \eqnshift@>\z@
        \if@fleqn\else:
          \global\eqnshift@\displaywidth
          \global\advance\eqnshift@-\totwidth@
          \global\advance\eqnshift@-\@tempcntb\alignsep@
          \global\divide\eqnshift@\tw@
        \fi:
      \fi:
    \fi:
    \if_dim:w \eqnshift@<\z@
      \global\eqnshift@\z@
    \fi:
    \calc@shift@align
    \global\tagshift@\totwidth@
    \global\advance\tagshift@\@tempcntb\alignsep@
    \if@fleqn
      \if_num:w \xatlevel@=\tw@
        \global\advance\tagshift@-\@mathmargin\relax
      \fi:
    \else:
      \global\advance\tagshift@\eqnshift@
    \fi:
    \iftagsleft@ \else:
      \global\advance\tagshift@-\displaywidth
    \fi:
    \dimen@\minalignsep\relax
    \global\advance\totwidth@\@tempcntb\dimen@
    \if_dim:w \totwidth@>\displaywidth
      \global\MH_let:NwN\displaywidth@\totwidth@
    \else:
      \global\MH_let:NwN\displaywidth@\displaywidth
    \fi:
  \endgroup
}
%    \end{macrocode}
%
%  \subsection{Fixing \env{multline}}
%  \begin{macro}{\EQ_multline_labelhack:n}
%  \begin{macro}{\EQ_multline_labelhack_opt:nw}
%  As \pkg{ntheorem} defines \cs{label} to have the syntax
%  \cs{label}\marg{label}\oarg{name}, we need to make sure that it
%  has that inside \env{multline} as well. Especially since the
%  measuring phase of \env{multline} \emph{defines} \cs{label} as
%  well.
%    \begin{macrocode}
\def\EQ_multline_labelhack:n #1{%
  \begingroup\measuring@false\label@in@display{#1}\endgroup}
\def\EQ_multline_labelhack_opt:nw #1[#2]{%
  \begingroup\measuring@false\label@in@display{#1}[#2]\endgroup}
%    \end{macrocode}
%  \end{macro}
%  \end{macro}
%  \begin{macro}{\multline@}
%  Patching \env{multline} takes more work because it does its
%  processing differently, so we need to apply hooks twice: One in
%  the measuring phase \cs{mmeasure@} where \cs{label} is redefined
%  and one in the production phase \cs{multline@} where the ordinary
%  \cs{label} is gobbled (we use \cs{MT_gobblelabel:w} as
%  before). First the typesetting phase.
%    \begin{macrocode}
\def\multline@#1{%
  \Let@
  \@display@init{\global\advance\row@\@ne \global\dspbrk@lvl\m@ne}%
  \chardef\dspbrk@context\z@
  \restore@math@cr
  \MH_let:NwN\tag\tag@in@align
%    \end{macrocode}
%  The patch for \cs{raisetag}.
%    \begin{macrocode}
  \global\tag@false \global\MH_let:NwN\raise@tag\default_raisetag:
  \mmeasure@{#1}%
  \MH_let:NwN\tag\gobble@tag
%    \end{macrocode}
%  The patch.
%    \begin{macrocode}
  \MH_let:NwN\label\MT_gobblelabel:w
  \tabskip \if@fleqn \@mathmargin \else: \z@skip \fi:
  \totwidth@\displaywidth
  \if@fleqn
    \advance\totwidth@-\@mathmargin
  \fi:
  \halign\bgroup
    \hbox to\totwidth@{%
      \if@fleqn
        \hskip \@centering \relax
      \else:
        \hfil
      \fi:
      \strut@
      $\m@th\displaystyle{}##\endmultline@math%$
      \hfil
    }%
    \crcr
    \if@fleqn
      \hskip-\@mathmargin
      \def\multline@indent{\hskip\@mathmargin}% put it back
    \else:
      \hfilneg
      \def\multline@indent{\hskip\multlinegap}%
    \fi:
    \iftagsleft@
      \iftag@
        \begingroup
          \ifshifttag@
            \rlap{\vbox{%
              \normalbaselines
              \hbox{%
                \strut@
                \make@display@tag
              }%
              \vbox to\lineht@{}%
              \raise@tag
            }}%
              \multline@indent
          \else:
            \setbox\z@\hbox{\make@display@tag}%
            \dimen@\@mathmargin \advance\dimen@-\wd\z@
            \if_dim:w \dimen@<\multlinetaggap
              \dimen@\multlinetaggap
            \fi:
            \box\z@ \hskip\dimen@\relax
          \fi:
        \endgroup
      \else:
        \multline@indent
      \fi:
    \else:
      \multline@indent
    \fi:
  #1%
}
%    \end{macrocode}
%  \end{macro}
%  \begin{macro}{\mmeasure@}
%  Then redefine the label correctly.
%    \begin{macrocode}
\def\mmeasure@#1{%
  \begingroup
    \measuring@true
%    \end{macrocode}
%  The patch.
%    \begin{macrocode}
    \def\label##1{%
      \@ifnextchar[
        {\EQ_multline_labelhack_opt:nw {##1}}
        {\EQ_multline_labelhack:n {##1}}
    }
    \def\math@cr@@@{\cr}%
    \MH_let:NwN\shoveleft\@iden \MH_let:NwN\shoveright\@iden
    \savecounters@
    \global\row@\z@
    \setbox\@ne\vbox{%
      \global\MH_let:NwN\df@tag\@empty
      \halign{%
        \setboxz@h{\@lign$\m@th\displaystyle{}##$}%
        \iftagsleft@
          \if_num:w \row@=\@ne
            \global\totwidth@\wdz@
            \global\lineht@\ht\z@
          \fi:
        \else:
          \global\totwidth@\wdz@
          \global\lineht@\dp\z@
        \fi:
        \crcr
        #1%
        \crcr
      }%
    }%
    \if_meaning:NN
      \df@tag\@empty
    \else:
      \global\tag@true
    \fi:
    \if@eqnsw\global\tag@true\fi:
    \iftag@
      \setboxz@h{%
        \if@eqnsw
          \stepcounter{equation}%
          \tagform@\theequation
        \else:
          \df@tag
        \fi:
      }%
      \global\tagwidth@\wdz@
      \dimen@\totwidth@
      \advance\dimen@\tagwidth@
      \advance\dimen@\multlinetaggap
      \iftagsleft@\else:
        \if@fleqn
          \advance\dimen@\@mathmargin
        \fi:
      \fi:
      \if_dim:w \dimen@>\displaywidth
        \global\shifttag@true
      \else:
        \global\shifttag@false
      \fi:
    \fi:
    \restorecounters@
  \endgroup
}
%    \end{macrocode}
%  \end{macro}
%
%  \subsection{Setting End Marks Correctly}
%
%  \begin{macro}{\mintagvsep}
%  \changes{v1.05}{2004/02/16}{Introduced \cs{mintagvsep} to
%  support the \pkg{ntheorem} package}
%  \begin{macro}{\EQ_only_endmark:}
%  \changes{v1.05}{2004/02/16}{Introduced \cs{EQ_only_endmark:}
%  to support the \pkg{ntheorem} package}
%  \changes{v2.11}{2004/08/03}{Make sure the tag is printed even with the
%  \pkg{mathtools} key \key{showonlyrefs} activated}
%  \begin{macro}{\EQ_tag_plus_endmark:}
%  \changes{v1.05}{2004/02/16}{Introduced
%  \cs{EQ_tag_plus_endmark:} to support the \pkg{ntheorem} package}
%  \pkg{ntheorem} provides an extension to the theorem environment
%  and also supports automatic end of theorem marks. To do so it must
%  redefine parts of the \pkg{amsmath} code and it eventually ends up
%  in problems when the tags and the marker are all typeset at the
%  right margin.
%
%  There are basically two different possibilities:
%  \begin{center}\hfill
%  \unitlength2.5pt \thicklines
%  \begin{picture}(60,45)(0,25)
%    \put(0,50){\framebox(25,20){Math Height}}
%    \put(0,25){\framebox(25,25){Math Depth}}
%    \put(40,50){\framebox(20,10){Tag height}}
%    \put(40,45){\framebox(20,5){Tag depth}}
%    \put(40,25){\framebox(20,10){Marker}}
%    \put(50,35){\vector(0,1){10}}
%    \put(50,45){\vector(0,-1){10}}
%    \put(50,39){\ \ $\delta>0$}
%    \multiput(26,25)(2,0){7}{\line(1,0){1}}
%  \end{picture} \hfill\vrule\hfill
%  \begin{picture}(60,45)(0,25)
%    \put(0,50){\framebox(25,20){Math Height}}
%    \put(0,40){\framebox(25,10){Math Depth}}
%    \put(40,50){\framebox(20,10){Tag height}}
%    \put(40,45){\framebox(20,5){Tag depth}}
%    \put(40,28){\framebox(20,10){Marker}}
%    \put(50,38){\vector(0,1){7}}
%    \put(50,45){\vector(0,-1){7}}
%    \put(50,40){\ \ $\delta=0$}
%  \end{picture}\hfill\mbox{}
%  \end{center}
%  Let's look closer at this. $h_M$ is the height of the end of
%  theorem marker, which we'll try to place so that its baseline
%  aligns with the bottom of the math display. $d_D$ is the depth of
%  the the last line of the math display including the depth of the
%  surrounding boxes while $d_T$ is the depth of the tag. Also we let
%  $m$ denote the minimum vertical distance between tag and marker
%  and $\delta$ is some positive length that may be added to $m$
%  depending on the other values. To ensure a ``proper'' placement of
%  the marker we simply test the inequality
%  \begin{gather}
%     h_M + d_T + m > d_D \label{eq:tag_placement}
%  \end{gather}
%  If the test is true we're done and just have to set the marker so
%  that the top of it is placed at a distance of $m$ from the bottom
%  of the tag. If it's not we can safely put the marker at the bottom
%  of the display.
%    \begin{macrocode}
\newlength\mintagvsep
\mintagvsep5pt
\def\EQ_only_endmark:{%
  \MH_set_boolean_T:n {show_manual_tags}
  \global\tag@true
  \iftagsleft@
    \gdef\df@tag{%
      \hbox to \displaywidth{%
        \hss
%    \end{macrocode}
%  If the tags are set to the left we just have to lower the end mark
%  by the total depth of the display including its surrounding box.
%  Also we have to take care of \cs{normallineskip}.
%    \begin{macrocode}
        \raisebox{-\normallineskip
                  -\depthof{\EQ_outerbox:{}}
                  -\heightof{\EQ_outerbox:{}}
                  -\dp\EQ_mathdisplay_box
                  -\ht\EQ_mathdisplay_box}
                 [0pt][0pt]{\PotEndMark{\maketag@@@}}}}
  \else:
    \gdef\df@tag{%
      \raisebox{-\dp\EQ_mathdisplay_box
                -\depthof{\EQ_outerbox:{}}}
               {\PotEndMark{\maketag@@@}}}%
  \fi:}
%    \end{macrocode}
%  \changes{v2.00}{2004/04/14}{Fixed another bug in placement of the
%  marker}
%  \changes{v2.10}{2004/07/26}{Fixed horizontal spacing in \env{equation}}
%  In v2.00 I fixed another bug I had overlooked. This is getting
%  embarrassing. The outer box must be taken into account as well.
%    \begin{macrocode}
\def\EQ_tag_plus_endmark:{%
  \iftagsleft@
    \gdef\maketag@@@##1{%
%    \end{macrocode}
%  The following fixes a bug when there is a wide tag in an
%  \env{equation} environment and was reported by Bernard
%  Alfonsi.\footnote{\email{alfonsi@math.u-psud.fr}} The problem
%  arises when \pkg{ntheorem} tries to typeset a tag and an end mark,
%  because the tag is set at the left margin and the end mark at the
%  right margin and \pkg{ntheorem} over-simplifies this and forgets
%  to take the displayed math into account! The right solution is to
%  wrap the tag in a box of the correct size, namely |\widthof{##1}|
%  and then push the end mark to the right margin. Then the displayed
%  math is set correcly.
%    \begin{macrocode}
      \hfuzz\displaywidth
      \makebox[\widthof{##1}][l]{%
        \hbox to \displaywidth{\m@th\normalfont##1\hss
          \raisebox{-\dp\EQ_mathdisplay_box
                    -\depthof{\EQ_outerbox:{}}}
                   {\PotEndMark{\hss}}}}}%
  \else:
    \gdef\maketag@@@##1{%
      \hbox{\m@th\normalfont##1\llap{\hss\PotEndMark{%
        \settoheight{\@tempdima}{\@nameuse{\InTheoType Symbol}}%
        \MH_addtolength:dn \@tempdima
          {\g_EQ_bottomtag_depth_dim+\mintagvsep}
        \setlength{\@tempdimb}
          {\g_EQ_bottomrow_depth_dim+\depthof{\EQ_outerbox:{}}
            +\depthof{\EQ_innerbox:{}}}
        \if_dim:w \@tempdima>\@tempdimb
          \MH_setlength:dn \@tempdima{-\dp\EQ_mathdisplay_box
            -\@tempdima+\g_EQ_bottomrow_depth_dim}
          \addtolength{\@tempdima}{\depthof{\EQ_innerbox:{}}}%
          \def\EQ_next:{\raisebox{\@tempdima}}%
        \else:
%    \end{macrocode}
%  \changes{v1.06}{2004/03/13}{Fixed bug in placement of the marker
%  and re-arranged some code}
%  In v1.06 I fixed a stupid bug regarding placement of the marker.
%  If I had done a proper test I would have discovered it
%  immediately!
%    \begin{macrocode}
          \def\EQ_next:{
            \raisebox{-\dp\EQ_mathdisplay_box
                      -\depthof{\EQ_outerbox:{}}}}%
        \fi:\EQ_next:}}}}%
  \fi:}
%    \end{macrocode}
%  \end{macro}
%  \end{macro}
%  \end{macro}
%
%  \begin{macro}{\MT_nonumber:}
%    \begin{macrocode}
\MH_let:NwN \MT_nonumber: \nonumber
%    \end{macrocode}
%  \end{macro}
%  \begin{macro}{\nonumber}
%  Redefine \cs{nonumber} to reset depth of bottom tag.
%    \begin{macrocode}
\def\g_EQ_latest_nonzerodepth_row_fint{\z@}
\renewcommand*\nonumber{
  \@tempcnta=\row@
  \if_num:w \l_EQ_equationtype_int=\thr@@
    \advance\@tempcnta\m@ne
  \else:
    \if_num:w \l_EQ_equationtype_int=4\relax
      \advance\@tempcnta\m@ne
    \fi:
  \fi:
  \if_num:w \g_EQ_latest_nonzerodepth_row_fint=\@tempcnta
  \else:
    \global\g_EQ_bottomtag_depth_dim=-\baselineskip
  \fi:
  \MT_nonumber:
}
%    \end{macrocode}
%  \end{macro}
%
%  End of \pkg{ntheorem} patches and package.
%    \begin{macrocode}
%</package>
%    \end{macrocode}
% Finale
\endinput

%        (quote the arguments according to the demands of your shell)
%
% Documentation:
%    (a) If empheq.drv is present:
%           latex empheq.drv
%    (b) Without empheq.drv:
%           latex empheq.dtx; ...
%    The class ltxdoc loads the configuration file ltxdoc.cfg
%    if available. Here you can specify further options, e.g.
%    use A4 as paper format:
%       \PassOptionsToClass{a4paper}{article}
%
%    Programm calls to get the documentation (example):
%       pdflatex empheq.dtx
%       makeindex -s gind.ist empheq.idx
%       pdflatex empheq.dtx
%       makeindex -s gind.ist empheq.idx
%       pdflatex empheq.dtx
%
% Installation:
%    TDS:tex/latex/mh/empheq.sty
%    TDS:doc/latex/mh/empheq.pdf
%    TDS:source/latex/mh/empheq.dtx
%
%<*ignore>
\begingroup
  \def\x{LaTeX2e}
\expandafter\endgroup
\ifcase 0\ifx\install y1\fi\expandafter
         \ifx\csname processbatchFile\endcsname\relax\else1\fi
         \ifx\fmtname\x\else 1\fi\relax
\else\csname fi\endcsname
%</ignore>
%<*install>
\input docstrip.tex
\Msg{************************************************************************}
\Msg{* Installation}
\Msg{* Package: empheq 2014/08/04 v2.14 empheq}
\Msg{************************************************************************}

\keepsilent
\askforoverwritefalse

\preamble

This is a generated file.

Copyright (C) 2002-2007 by Morten Hoegholm
Copyright (C) 2012-     by Lars Madsen


This work may be distributed and/or modified under the
conditions of the LaTeX Project Public License, either
version 1.3 of this license or (at your option) any later
version. The latest version of this license is in
   http://www.latex-project.org/lppl.txt
and version 1.3 or later is part of all distributions of
LaTeX version 2005/12/01 or later.

This work has the LPPL maintenance status "maintained".

This Current Maintainer of this work is
Lars Madsen

This work consists of the main source file empheq.dtx
and the derived files
   empheq.sty, empheq.pdf, empheq.ins, empheq.drv.

\endpreamble

\generate{%
  \file{empheq.ins}{\from{empheq.dtx}{install}}%
  \file{empheq.drv}{\from{empheq.dtx}{driver}}%
  \usedir{tex/latex/mh}%
  \file{empheq.sty}{\from{empheq.dtx}{package}}%
}

\obeyspaces
\Msg{************************************************************************}
\Msg{*}
\Msg{* To finish the installation you have to move the following}
\Msg{* file into a directory searched by TeX:}
\Msg{*}
\Msg{*     empheq.sty}
\Msg{*}
\Msg{* To produce the documentation run the file `empheq.drv'}
\Msg{* through LaTeX.}
\Msg{*}
\Msg{* Happy TeXing!}
\Msg{*}
\Msg{************************************************************************}

\endbatchfile
%</install>
%<*ignore>
\fi
%</ignore>
%<*driver>
\NeedsTeXFormat{LaTeX2e}
\ProvidesFile{empheq.drv}%
  [2014/08/04 v2.14 Emphasizing equations]
\documentclass{ltxdoc}
\usepackage{mathpazo}
\usepackage[overload]{empheq}
\usepackage{xcolor,varioref,booktabs,fancybox}
\usepackage[overload]{textcase}

\pagestyle{headings}
\hyphenation{emph-eq}
\makeatletter
\providecommand*\pkg[1]{\textsf{#1}}
\providecommand*\env[1]{\texttt{#1}}
\providecommand*\email[1]{\href{mailto:#1}{\texttt{#1}}}
\providecommand*\mode[1]{\texttt{[#1]}}
\providecommand*\file[1]{\texttt{#1}}
\addtolength\marginparwidth{-30pt}


\newcommand*\AddDescribe[3]{%
  \@namedef{Describe#1}{\leavevmode\@bsphack\begingroup\MakePrivateLetters
    \@nameuse{Describe@#1}}
  \@namedef{Describe@#1}##1{\endgroup
                \marginpar{\raggedleft\@nameuse{PrintDescribe#1}{##1}}%
                \@nameuse{Special#1Index}{##1}\@esphack\ignorespaces}
  \@namedef{Special#1Index}##1{\@bsphack
      \index{##1\actualchar{\protect\ttfamily ##1}
            (#2)\encapchar usage}%
      \index{#3:\levelchar##1\actualchar{\protect\ttfamily ##1}\encapchar
            usage}\@esphack}
  \@namedef{PrintDescribe#1}##1{\strut \MacroFont ##1\ }%
  \@namedef{Print#1Name}##1{\strut \MacroFont ##1\ }%
}
\AddDescribe{Color}{color}{colors}
\AddDescribe{KeyFamily}{key family}{key families}
\newcommand*\DescribeDelimiter[1]{%
  \cs{#1}&\SpecialMkhIndex{empheq#1}\cs{empheq}\texttt{#1} &
  \SpecialMkhIndex{empheqbig#1}\cs{empheqbig}\texttt{#1} &
  $\@nameuse{#1}$\tabularnewline}
\newcommand*\SpecialMkhIndex[1]{%
  \@bsphack
    \index{#1\actualchar{\protect\ttfamily \protect\bslash\space#1}
     (` $\bslash#1$ ')\encapchar usage}%
  \@esphack
}
\newcommand\TabDescribeOption[3][(default)]{%
    \SpecialOptIndex{#2}\opt{#2}  & #3 \tabularnewline
}
\newcommand*\TabDescribeAmsKey[2][]{%
    \SpecialKeyIndex{#2}\env{#2} &
    \texttt{\{#2#1\}} &
    \SpecialKeyIndex{#2*}\env{#2*} &
    \texttt{\{#2*#1\}} \tabularnewline
}
\newcommand*\TabDescribeOptKey[3][]{%
    \SpecialKeyIndex{#2}\key{#2} &
    \texttt{#2=}\meta{#3} & #1 \tabularnewline
}

\newenvironment{codesyntax}
    {\par\small\addvspace{4.5ex plus 1ex}%
     \vskip -\parskip
     \noindent
     \begin{tabular}{|l|}\hline\ignorespaces}%
    {\\\hline\end{tabular}\nobreak\par\nobreak
     \vspace{2.3ex}\vskip -\parskip\noindent\ignorespacesafterend}
\makeatletter

\newcommand*\FeatureRequest[2]{%
  \hskip1sp
  \marginpar{%
    \parbox[b]{\marginparwidth}{\small\sffamily\raggedright
      \strut Feature request by\\#1\\#2%
    }
  }%
}

\newcommand*\cttPosting[2]{%
  \hskip1sp
  \marginpar{%
    \parbox[b]{\marginparwidth}{\small\sffamily\raggedright
     \strut Posted on \texttt{comp.text.tex} \\#1\\#2%
    }%
  }%
}

\expandafter\def\expandafter\MakePrivateLetters\expandafter{%
  \MakePrivateLetters  \catcode`\_=11\relax
}

\providecommand*\SpecialOptIndex[1]{%
  \@bsphack
  \index{#1\actualchar{\protect\ttfamily #1}
          (option)\encapchar usage}%
      \index{options:\levelchar#1\actualchar{\protect\ttfamily #1}\encapchar
            usage}\@esphack}
\providecommand*\opt[1]{\texttt{#1}}

\providecommand*\SpecialKeyIndex[1]{%
  \@bsphack
  \index{#1\actualchar{\protect\ttfamily #1}
          (key)\encapchar usage}%
      \index{keys:\levelchar#1\actualchar{\protect\ttfamily #1}\encapchar
            usage}\@esphack}
\providecommand*\key[1]{\textsf{#1}}
\providecommand*\keyfam[1]{\textsf{#1}}

\providecommand*\eTeX{$\m@th\varepsilon$-\TeX}

\def\MTmeta#1{%
     \ensuremath\langle
     \ifmmode \expandafter \nfss@text \fi
     {%
      \meta@font@select
      \edef\meta@hyphen@restore
        {\hyphenchar\the\font\the\hyphenchar\font}%
      \hyphenchar\font\m@ne
      \language\l@nohyphenation
      #1\/%
      \meta@hyphen@restore
     }\ensuremath\rangle
     \endgroup
}
\makeatother
\DeclareRobustCommand\meta{\begingroup\MakePrivateLetters\MTmeta}%
\def\MToarg#1{{\ttfamily[}\meta{#1}{\ttfamily]}\endgroup}
\DeclareRobustCommand\oarg{\begingroup\MakePrivateLetters\MToarg}%
\def\MHmarg#1{{\ttfamily\char`\{}\meta{#1}{\ttfamily\char`\}}\endgroup}
\DeclareRobustCommand\marg{\begingroup\MakePrivateLetters\MHmarg}%
\def\MHarg#1{{\ttfamily\char`\{#1\ttfamily\char`\}}\endgroup}
\DeclareRobustCommand\arg{\begingroup\MakePrivateLetters\MHarg}%
\def\MHcs#1{\texttt{\char`\\#1}\endgroup}
\DeclareRobustCommand\cs{\begingroup\MakePrivateLetters\MHcs}
\def\MHbool#1{{\ttfamily #1}\endgroup}
\DeclareRobustCommand\bool{\begingroup\MakePrivateLetters\MHbool}%

\def\endverbatim{\if@newlist
\leavevmode\fi\endtrivlist\vspace{-\baselineskip}}
\expandafter\let\csname endverbatim*\endcsname =\endverbatim

\let\MTtheindex\theindex
\def\theindex{\MTtheindex\MakePrivateLetters}

\newcommand*\widefbox[1]{\fbox{\hspace{1em}#1\hspace{1em}}}
\definecolor{myblue}{rgb}{.8, .8, 1}
\newcommand*\mybluebox[1]{%
  \colorbox{myblue}{\hspace{1em}#1\hspace{1em}}}
\newenvironment{important}[2][]{%
  \setkeys{EmphEqEnv}{#2}%
  \setkeys{EmphEqOpt}{box=\mybluebox,#1}%
  \EmphEqMainEnv}%
  {\endEmphEqMainEnv}
\definecolor{shadecolor}{cmyk}{0,0,0.41,0}
\definecolor{light-blue}{cmyk}{0.25,0,0,0}
\newsavebox{\mysaveboxM} % M for math
\newsavebox{\mysaveboxT} % T for text
\newcommand*\Garybox[2][Example]{%
  \sbox{\mysaveboxM}{#2}%
  \sbox{\mysaveboxT}{\fcolorbox{black}{light-blue}{#1}}%
  \sbox{\mysaveboxM}{%
    \parbox[b][\ht\mysaveboxM+.5\ht\mysaveboxT+.5\dp\mysaveboxT][b]{%
      \wd\mysaveboxM}{#2}%
  }%
  \sbox{\mysaveboxM}{%
    \fcolorbox{black}{shadecolor}{%
      \makebox[\linewidth-10em]{\usebox{\mysaveboxM}}%
    }%
  }%
  \usebox{\mysaveboxM}%
  \makebox[0pt][r]{%
    \makebox[\wd\mysaveboxM][c]{%
      \raisebox{\ht\mysaveboxM-0.5\ht\mysaveboxT
                +0.5\dp\mysaveboxT-0.5\fboxrule}{\usebox{\mysaveboxT}}%
    }%
  }%
}

\newcommand\EmphEqpart{%
  \def\partname{\EmphEqpartname}%
  \part}

\def\EmphEqpartname{Part}
\makeatother

\usepackage{graphicx}
\usepackage[final,breaklinks=true,hyperindex=false]{hyperref}
\renewcommand*\usage[1]{\textit{\hyperpage{#1}}}

\OnlyDescription
\begin{document}
\DocInput{empheq.dtx}
\end{document}
%</driver>
%  \fi
%  \changes{v2.14}{2014/07/16}{Updated maintenance and contact info}
%
%  \changes{v2.11}{2004/08/03}{Removed interface to the obsolete empheq
%  package \pkg{empheq07}}
%
%  \changes{v2.10}{2004/07/26}{Rewrote most of the documentation}
%
%  \changes{v2.10}{2004/07/26}{Moved options \texttt{fixamsmath} and
%  \texttt{donotfixamsmathbugs} to \pkg{mathtools}}
%
%  \changes{v2.10}{2004/07/26}{Added support for \pkg{showkeys}}
%
%  \changes{v2.00}{2004/04/14}{Added options \opt{overload},
%  \opt{overload2}, \texttt{fixamsmath}, and
%  \texttt{donotfixamsmathbugs} plus descriptions of their use}
%
%  \changes{v1.06}{2004/03/13}{Upgraded license to LPPL version 1.3}
%
%  \changes{v1.00}{2004/02/15}{Complete rewrite of the package hence
%  no description of the changes made}
%
%  \changes{v0.7a}{2004/02/15}{Made the old package obsolete and
%  changed its name to \pkg{empheq-obsolete}}
%
%  \changes{v0.7}{2003/05/11}{Added further examples and changed
%  existing ones}
%  \changes{v0.6b}{2003/03/04}{Added example for changing margins
%  inside the boxtype}
%  \changes{v0.6a}{2003/01/28}{Minor updates to documentation}
%
%  \changes{v0.5}{2002/12/30}{First experimental and very limited
%  release}
%
%  \GetFileInfo{empheq.drv}
%  \CheckSum{2495}
%
%  \title{The \pkg{empheq} package\thanks{This file has version number
%  \fileversion, last revised \filedate.}\\
%  Emphasizing equations in \LaTeXe}
%  \author{Morten H\o gholm, Lars Madsen}
%  
%  \date{\filedate}
%
%  \maketitle
%
%  \begin{abstract}
%  The \pkg{empheq} package can best be described as a visual markup
%  extension to \pkg{amsmath}. In short it offers a)~a multi line
%  equivalent of \cs{boxed} from \pkg{amsmath} and b)~a way to
%  produce arbitrary delimiters that span entire math displays.
%  \end{abstract}
%
%  \tableofcontents
%
%  \clearpage
%  \EmphEqpart{Basic user's guide}
%
%  There can be little doubt that the de facto standard for
%  mathematical typesetting in \LaTeX{} is the \pkg{amsmath} package.
%  For the creation of \pkg{empheq}, a visual markup package for use
%  in math, it made perfect sense to have \pkg{amsmath} as the
%  backbone.
%
%  The main idea of \pkg{empheq} is to maintain the familiar syntax
%  of the \pkg{amsmath} environments while still providing an easy
%  way of specifying markup instructions. This manual is plastered
%  with examples showing just how.
%
%
%  \section{Basic use of the package}
%
%  So what is it \pkg{empheq} does? Well, it allows you to produce
%  displays like this:
%  \begin{subequations}
%  \begin{empheq}[
%    box=\mybluebox,
%    right={\;\makebox[.9em]{$\raisebox{-.5\totalheight+\fontdimen22\textfont2}
%              {\resizebox{!}{\EmphEqdisplayheight+\EmphEqdisplaydepth}{!}}$}},
%    left={X=Y\Rightarrow\empheqlbrace}
%     ]{alignat=3}
%    A_1&=b_1 & \qquad c_1&=d_1 & \qquad e_1&= f_1
%         \tag*{\_A\raisebox{1ex}{silly}\raisebox{-1ex}{tag}\_}\\
%    A_2&=b_2 & \qquad c_2&=d_2 & \qquad  e_2&= f_2 \\
%    A_3&=b_3 & \qquad c_3&=d_3 & \qquad  e_3&= f_3
%  \end{empheq}
%  \end{subequations}
%  In short \pkg{empheq} enables the user to put things on every side
%  of the display without said user having to worry about what
%  happens to the equation numbers. For example you can now have a
%  display containing multiple lines and still get the effect of the
%  \cs{boxed} command from \pkg{amsmath}.
%
%
%  \subsection{Using the \env{empheq} environment}
%
%  The package defines a single environment
%  \SpecialEnvIndex{empheq}\env{empheq} and the usage is kind of straight
%  forward:
%  \begin{codesyntax}
%    |\begin{empheq}|\oarg{markup instructions}\marg{\AmS{} env_name}\\
%    \mbox{}\quad\meta{contents of \AmS{} environment}\\
%    |\end{empheq}|
%  \end{codesyntax}
%  A first minimal example file would then be something like
%  \begin{verbatim}
%    \documentclass{minimal}
%    \usepackage{empheq}
%    \begin{document}
%    \begin{empheq}{align*}
%      a&=b \tag{*}\\
%      E&=mc^2 + \int_a^a x\, dx
%    \end{empheq}
%    \end{document}
%  \end{verbatim}
%  This gives the following display:
%    \begin{empheq}{align*}
%      a&=b \tag{*}\\
%      E&=mc^2 + \int_a^a x\, dx
%    \end{empheq}
%  Maybe not the most impressive example, but as you can see the
%  contents of the environment is exactly the same as for the regular
%  \env{align*} evironment from \pkg{amsmath}. The rest of the \AmS{}
%  environments are chosen the same way by typing the name as the
%  mandatory argument of \env{empheq} with the exception of the
%  \env{alignat} environment. For this you have to specify the number
%  of columns as shown below.
%  \begin{verbatim}
%  \begin{empheq}{alignat=2}
%    a &= b &\quad c &= d \\
%    \text{this} &= \text{that} &\quad \mathit{fish}&\neq fish
%  \end{empheq}
%  \end{verbatim}
%  \begin{empheq}{alignat=2}
%    a &= b &\quad c &= d \\
%    \text{this} &= \text{that} &\quad \mathit{fish}&\neq fish
%  \end{empheq}
%  To choose the starred version of \env{alignat}, simply type
%  \arg{alignat*=2} instead in the above example.
%
%  The supported \AmS{} environments are listed in Table
%  \vref{tab:environments}. Not supported is the standard \LaTeX{}
%  \env{eqnarray} environment as it is fundamentally
%  deficient.\footnote{See for instance
%  \url{http://www.tex.ac.uk/cgi-bin/texfaq2html?label=eqnarray}}
%
%  \begin{table}
%  \centering
%  \begin{tabular}{llll}\toprule
%    Environment & Usage & Environment & Usage \\ \midrule
%    \TabDescribeAmsKey{equation}
%    \TabDescribeAmsKey{align}
%    \TabDescribeAmsKey{gather}
%    \TabDescribeAmsKey{flalign}
%    \TabDescribeAmsKey[=\textrm{\meta{cols}}]{alignat}
%    \TabDescribeAmsKey{multline} \bottomrule
%  \end{tabular}
%  \caption{The supported \pkg{amsmath}
%  environments}\label{tab:environments}
%  \end{table}
%
%
%  \subsection{Markup instructions}
%
%  The optional argument of the \env{empheq} environment will take
%  markup instructions consisting of a $\meta{key}=\meta{value}$ list
%  of assignments. There are currently five such keys (a sixth is
%  added for naming consistency). They're shown in Table
%  \vref{tab:keys}.
%
%  \begin{table}
%  \centering
%  \begin{tabular}{llp{13em}}\toprule
%    Key & Usage & Additional Info \tabularnewline \midrule
%    \TabDescribeOptKey{box}{box command}
%    \TabDescribeOptKey{innerbox}{box command}
%    \TabDescribeOptKey[Use \cs{text}\arg{...} if you need text
%                       material.]{left}{math material}
%    \TabDescribeOptKey[Use \cs{text}\arg{...} if you need text
%                       material.]{right}{math material}
%    \TabDescribeOptKey[Alias for \key{box}.]{outerbox}{box command}
%    \TabDescribeOptKey[Contents must be horizontally centered;
%                       can only be used in \mode{fleqn} mode.]
%                       {marginbox}{box command}
%    \bottomrule
%  \end{tabular}
%  \caption{The six keys for the optional argument of the
%  \env{empheq} environment}\label{tab:keys}
%  \end{table}
%
%  \begin{codesyntax}
%    \SpecialKeyIndex{left}\key{left}
%  \end{codesyntax}
%  The key \key{left} is for material put on the left side of the
%  display. The material is typeset in math mode and centered
%  vertically.
%  \begin{verbatim}
%  \begin{empheq}[left=L\Rightarrow]{align}
%    a&=b\\
%    E&=mc^2 + \int_a^a x\, dx
%  \end{empheq}
%  \end{verbatim}
%  \begin{empheq}[left=L\Rightarrow]{align}
%    a&=b\\
%    E&=mc^2 + \int_a^a x\, dx
%  \end{empheq}
%
%
%  \begin{codesyntax}
%    \SpecialKeyIndex{right}\key{right}
%  \end{codesyntax}
%  As there is a \key{left} key it hopefully comes as no surprise
%  that there is a companion \key{right} key for typesetting material
%  on the right side of the display.
%  \begin{verbatim}
%  \begin{empheq}[right=\Leftarrow R]{align}
%    a&=b\\
%    E&=mc^2 + \int_a^a x\, dx
%  \end{empheq}
%  \end{verbatim}
%  \begin{empheq}[right=\Leftarrow R]{align}
%    a&=b\\
%    E&=mc^2 + \int_a^a x\, dx
%  \end{empheq}
%
%  \begin{codesyntax}
%    \SpecialKeyIndex{box}\key{box}\\
%    \SpecialKeyIndex{outerbox}\key{outerbox}
%  \end{codesyntax}
%  The key \key{box} specifies the kind of box you would like to put
%  around the display. It can be any kind of box, as long as the
%  contents of the box is situated on the baseline like in a
%  \fbox{\cs{fbox}}.
%  \begin{verbatim}
%  \begin{empheq}[box=\fbox]{align}
%    a&=b\\
%    E&=mc^2 + \int_a^a x\, dx
%  \end{empheq}
%  \end{verbatim}
%  \begin{empheq}[box=\fbox]{align}
%    a&=b\\
%    E&=mc^2 + \int_a^a x\, dx
%  \end{empheq}
%  The key \key{outerbox} is an alias for \key{box} and is added for
%  naming consistency with the key described below.
%
%  \begin{codesyntax}
%    \SpecialKeyIndex{innerbox}\key{innerbox}
%  \end{codesyntax}
%  There is also an \key{innerbox} key. It is not very interesting
%  unless you use one of the other keys.
%  \begin{verbatim}
%  \begin{empheq}[innerbox=\fbox,
%                 left=L\Rightarrow]{align}
%    a&=b\\
%    E&=mc^2 + \int_a^a x\, dx
%  \end{empheq}
%  \end{verbatim}
%  \begin{empheq}[innerbox=\fbox,
%                 left=L\Rightarrow]{align}
%    a&=b\\
%    E&=mc^2 + \int_a^a x\, dx
%  \end{empheq}
%
%  \begin{codesyntax}
%    \SpecialKeyIndex{marginbox}\key{marginbox}
%  \end{codesyntax}
%  \FeatureRequest{Uwe Siart}{2003/12/08}
%  The last key is the \key{marginbox} key. If you typeset your math
%  in \mode{fleqn} mode\footnote{Sorry, but I can't show you because
%  this manual features centered math displays.} you may want the
%  math display and not the outer box to align at the left margin (or
%  rather: the indentation). If you make sure the contents of the
%  outer box is centered horizontally inside the box, \key{marginbox}
%  will align it properly. You shouldn't set both \key{box} and
%  \key{marginbox} at the same time, as this is surely not what you
%  want and the package will silently use the last one in the list.
%
%
%  \begin{quote}
%  \textbf{Warning}: \pkg{keyval} treats commas and equal signs as
%  separator and assignment signs which in turn means that if you
%  want to typeset either of them, they \emph{must}---and believe me,
%  it's really important---be enclosed in braces. In short:
%  \begin{description}
%    \item[{\makebox[3em][l]{Good}}] \texttt{[left=\{(A,B)=(1,0)\}]}
%    \item[{\makebox[3em][l]{Bad}}]  \texttt{[left=(A,B)=(1,0)]}
%  \end{description}
%  \end{quote}
%
%  \begin{codesyntax}
%    \SpecialUsageIndex{\empheqset}\cs{empheqset}\marg{markup instructions}
%  \end{codesyntax}
%  \FeatureRequest{Uwe Siart}{2004/01/13}
%  You can also set the keys globally\footnote{Or rather: outside the
%  scope of the \env{empheq} environment. The settings will still
%  obey scoping rules.} with the command
%  \cs{empheqset}. This means that
%  \begin{verbatim}
%    \empheqset{marginbox=\psframebox}
%  \end{verbatim}
%  will force a |box=\psframebox| in all occurrences of the
%  \env{empheq} environment, but an explicitly given \key{box} or
%  \key{marginbox} will override this setting. You can only use the
%  keys listed in this section as arguments to \cs{empheqset}; not
%  the \AmS\ environments.
%
%
%
%  \subsection{What won't work in the \env{empheq} environment}
%
%  \begin{codesyntax}
%    \SpecialUsageIndex{intertext}\cs{intertext}\marg{text}\\
%    \SpecialUsageIndex{displaybreak}\cs{displaybreak}\oarg{num}
%  \end{codesyntax}
%  Now we've seen some of the things that work in the \env{empheq}
%  environment but we also have to take note of what won't work. As
%  this package provides a way to box multi line math displays it can
%  come as no surprise that using either \cs{intertext} or
%  \cs{displaybreak} inside the \env{empheq} environment makes no
%  sense. Should you however happen to try them anyway, you'll
%  experience that \cs{intertext} issues an error message and
%  \cs{displaybreak} issues a warning. The reason only \cs{intertext}
%  gives an error message is that you'll get output very different
%  from what you expect, and that is not the case (so much at least)
%  with \cs{displaybreak}. But of course, you'll never see either
%  message because you read the manual!
%
%
%
%  \subsection{Special delimiters}
%
%  As you've seen a few pages back, it's possible to add material on
%  both sides of the math display. When doing so you often need a
%  delimiter that scales to fit the entire display, so it comes as no
%  surprise that \pkg{empheq} provides such delimiters.
%  \begin{codesyntax}
%    \cs{empheql}\meta{delim_name}  \cs{empheqbigl}\meta{delim_name}\\
%    \cs{empheqr}\meta{delim_name}  \cs{empheqbigr}\meta{delim_name}
%  \end{codesyntax}
%  For example one might need a large brace like the one in the
%  \env{cases} environment:
%  \begin{verbatim}
%  \begin{empheq}[left=\empheqlbrace, right=\empheqrbrace]{align}
%    E&=mc^2 \\
%    Y&= \sum_{n=1}^\infty \frac{1}{n^2}
%  \end{empheq}
%  \end{verbatim}
%  \begin{empheq}[left=\empheqlbrace, right=\empheqrbrace]{align}
%    E&=mc^2 \\
%    Y&= \sum_{n=1}^\infty \frac{1}{n^2}
%  \end{empheq}
%  The naming scheme is \cs{empheq}\meta{delimiter name} and
%  \cs{empheqbig}\meta{delimiter name}. Thus \cs{empheqrbrace}
%  produces an auto-scaling right brace with same size as the math
%  display while \cs{empheqbigrbrace} produces an even bigger version
%  that spans the inner box plus the math display inside it. For a
%  complete list of supported delimiters see Table
%  \vref{tab:delimiters}.
%
%  \begin{table}
%  \centering
%  \begin{tabular}{lllc} \toprule
%  \multicolumn{3}{c}{\textbf{Delimiter}}\\ \cmidrule{1-3}
%  \multicolumn{1}{c}{Original} & \multicolumn{1}{c}{Normal}  &
%    \multicolumn{1}{c}{Bigger} & Symbol\\ \midrule
%  \DescribeDelimiter{lbrace} \DescribeDelimiter{rbrace}
%  \DescribeDelimiter{lbrack} \DescribeDelimiter{rbrack} \midrule
%  \DescribeDelimiter{langle} \DescribeDelimiter{rangle}
%  \DescribeDelimiter{lparen} \DescribeDelimiter{rparen} \midrule
%  \DescribeDelimiter{lvert} \DescribeDelimiter{rvert}
%  \DescribeDelimiter{lVert} \DescribeDelimiter{rVert} \midrule
%  \DescribeDelimiter{lfloor} \DescribeDelimiter{rfloor}
%  \DescribeDelimiter{lceil} \DescribeDelimiter{rceil} \bottomrule
%  \end{tabular}
%  \caption{The supported auto-scaling delimiters in
%  \pkg{empheq}}\label{tab:delimiters}
%  \end{table}
%
%
%
%  \section{Taking things a little further}
%
%  So far we have covered the basic functions of \pkg{empheq}: the
%  markup instructions and the predefined auto-scaling delimiters.
%  There is more to \pkg{empheq} however, and we'll cover that in
%  this section.
%
%  \subsection{Loading the package}
%
%  The package has two main requirements: One is the
%  \pkg{mathtools}\footnote{By the same author and is distributed
%  with \pkg{empheq}. See its documentation for more information.}
%  package which provides \pkg{empheq} with a lot of necessary tools
%  for doing its thing. The other one is \pkg{amsmath} of course, and
%  to make the loading procedure as easy as possible, you can simply
%  substitute
%  \begin{verbatim}
%  \usepackage[leqno,fleqn,intlimits]{amsmath}
%  \end{verbatim}
%  with
%  \begin{verbatim}
%  \usepackage[leqno,fleqn,intlimits]{empheq}
%  \end{verbatim}
%  \pkg{empheq} makes sure that the \pkg{amsmath} options are passed
%  on and loaded by \pkg{amsmath}. The same goes for the options
%  provided by \pkg{mathtools}.
%
%
%  \subsubsection{Compatibility with the old version}
%
%  The current version (\fileversion) of \pkg{empheq} is incompatible with
%  versions prior to and including~0.7e. If you have documents produced
%  with versions prior to~0.7e, then you have to load the package
%  \pkg{empheq07} instead. \pkg{empheq07} now exists as a separate package
%  package with its own documentation, but I strongly recommend switching
%  to the new version because it is so much better.
%
%  \subsection{Package options}
%
%  In addition to the \pkg{amsmath} options, \pkg{empheq} itself
%  provides a string of options listed in Table \vref{tab:options}.
%
%  \begin{table}[ht]
%    \centering
%    \begin{tabular}{lp{21em}} \toprule
%      Option  & Short description \tabularnewline \midrule
%      \TabDescribeOption[]{overload}
%      {Lets you use the \pkg{empheq} visual markup extensions
%      transparently in an existing document.}
%      \TabDescribeOption[]{overload2}
%      {A wild version of \opt{overload}. Use with care.}
%      \TabDescribeOption[]{ntheorem}
%      {A support module for users of \pkg{ntheorem} and its
%      \opt{thmmarks} and \opt{thref} options.}
%      \TabDescribeOption[]{newmultline}
%      {With v2.10 of \pkg{empheq} the syntax for \env{multline} and
%      \env{multlined} has been improved immensely.}
%      \TabDescribeOption[]{oldmultline}
%      {Support for the somewhat strange syntax for \env{multline} and
%      \env{multlined} in v2.00.}
%      \bottomrule
%    \end{tabular}
%    \caption{Supported options in the \pkg{empheq} package}
%    \label{tab:options}
%  \end{table}
%
%
%  \begin{codesyntax}
%    \SpecialOptIndex{ntheorem}\opt{ntheorem}
%  \end{codesyntax}
%  The \pkg{ntheorem} package is supported by means of the
%  \opt{ntheorem} option. In order for this to work properly, an
%  \pkg{amsmath} bug\footnote{See the \LaTeX{} Bugs Database
%  \url{http://www.latex-project.org/cgi-bin/ltxbugs2html} under
%  \AmS\LaTeX{} problem report 3624.} is fixed. For more information
%  on this option see \S \vref{sec:ntheorem-support}.
%
%  \subsubsection{Using \env{multline}}
%
%  \begin{codesyntax}
%    \SpecialOptIndex{newmultline}\opt{newmultline}\\
%    \SpecialOptIndex{oldmultline}\opt{oldmultline}
%  \end{codesyntax}
%  Those familiar with v2.00 of this package will recognize the
%  somewhat weird syntax for using \env{multline} and
%  \env{multlined}. The good news is that since v2.10 there is now a
%  much improved syntax available. The option \opt{newmultline}
%  (default) selects this better interface, while the old interface
%  is provided for compatibility reasons by the \opt{oldmultline}
%  option.
%
%
%  \begin{codesyntax}
%    \SpecialUsageIndex{\shoveleft}\cs{shoveleft}\oarg{dimen}\marg{arg}\\
%    \SpecialUsageIndex{\shoveright}\cs{shoveright}\oarg{dimen}\marg{arg}
%  \end{codesyntax}
%  With the new interface you also get an extended syntax for
%  \cs{shoveleft} and \cs{shoveright} as shown in the example below.
%  \begin{verbatim}
%  \begin{empheq}{multline}
%    \framebox[.65\columnwidth]{First line} \\
%    \framebox[.5\columnwidth]{Second line} \\
%    \shoveleft{L+E+F+T}         \\
%    \shoveright{R+I+G+H+T}         \\
%    \shoveleft[1cm]{L+E+F+T}         \\
%    \shoveright[\widthof{$R+I+G+H+T$}]{R+I+G+H+T}         \\
%    \framebox[.65\columnwidth]{Last line}
%  \end{empheq}
%  \end{verbatim}
%  \begin{empheq}{multline}
%    \framebox[.65\columnwidth]{First line} \\
%    \framebox[.5\columnwidth]{Second line} \\
%    \shoveleft{L+E+F+T}         \\
%    \shoveright{R+I+G+H+T}         \\
%    \shoveleft[1cm]{L+E+F+T}         \\
%    \shoveright[\widthof{$R+I+G+H+T$}]{R+I+G+H+T}         \\
%    \framebox[.65\columnwidth]{Last line}
%  \end{empheq}
%
%
%  There are however a few differences in the output between the
%  original \pkg{amsmath} version of \env{multline} and the one
%  \pkg{empheq} provides. In \pkg{amsmath} a centered line in
%  \env{multline} is centered on the page without taking into account
%  the \cs{multlinegap}, \cs{multlinetaggap}, and the tag width. Thus
%  \pkg{amsmath} can sometimes give you horrible output without
%  giving you any warning as shown below.
%  \begin{verbatim}
%  \begin{multline}
%    \framebox[.65\columnwidth]{First line} \\
%    \framebox[.9\columnwidth]{Loooong line} \\
%    \framebox[.65\columnwidth]{Last line} \tag{wide tag}
%  \end{multline}
%  \end{verbatim}
%  \begin{AmSmultline}
%    \framebox[.65\columnwidth]{First line} \\
%    \framebox[.9\columnwidth]{Loooong line} \\
%    \framebox[.65\columnwidth]{Last line} \tag{wide tag}
%  \end{AmSmultline}
%  In \pkg{empheq} these parameters are taken into account, so the
%  same input will produce this in \pkg{empheq}:
%  \begin{verbatim}
%  \begin{empheq}{multline}
%    \framebox[.65\columnwidth]{First line} \\
%    \framebox[.9\columnwidth]{Loooong line} \\
%    \framebox[.65\columnwidth]{Last line} \tag{wide tag}
%  \end{empheq}
%  \end{verbatim}
%  \begin{multline*}
%    \framebox[.65\columnwidth]{First line} \\
%    \framebox[.9\columnwidth]{Loooong line} \\
%    \framebox[.65\columnwidth]{Last line} \tag{wide tag}
%  \end{multline*}
%  This results in an
%  \begin{verbatim}
%    Overfull \hbox (30.03783pt too wide) in paragraph ...
%  \end{verbatim}
%  message in the log file, indicating a visual problem. I think this
%  behavior is more sensible than the one the original \env{multline}
%  environment provides.
%
%  \begin{codesyntax}
%    \SpecialEnvIndex{MTmultlined}\cs{begin}\arg{MTmultlined}\oarg{pos}\oarg{width}
%        \meta{contents} \cs{end}\arg{MTmultlined}
%  \end{codesyntax}
%  When you choose the \opt{oldmultline} option you may still want to
%  use the much improved \env{multlined} environment defined in
%  \pkg{mathtools} but unfortunately there will be a name clash.
%  Instead you can access it under the name \env{MTmultlined}.
%
%
%
%  \subsubsection{The \opt{overload} option}\label{sec:overload}
%
%  \begin{codesyntax}
%    \SpecialOptIndex{overload}\opt{overload}\\
%    \SpecialOptIndex{overload2}\opt{overload2}
%  \end{codesyntax}
%  \FeatureRequest{Lars Madsen}{2004/03/25}
%  The \opt{overload} option redefines the original \AmS{}
%  environments so that they take an optional argument.
%  \begin{codesyntax}
%    \cs{begin}\marg{\AmS{} env_name}\oarg{markup instructions}\\
%    \meta{contents of \AmS{} environment}\\
%    \cs{end}\marg{\AmS{} env_name}
%  \end{codesyntax}
%  For example
%  \begin{verbatim}
%    \begin{gather}[box=\widefbox]
%      a = b
%    \end{gather}
%  \end{verbatim}
%  is then actually short for
%  \begin{verbatim}
%    \begin{empheq}[box=\widefbox]{gather}
%      a = b
%    \end{empheq}
%  \end{verbatim}
%  All the \AmS{} environments are supported by this option except
%  for the pseudo environment |\[ ... \]| (it's not really an
%  environment), \footnote{With the option \opt{oldmultline} the
%  \env{multline} and \env{multline*} environments aren't supported
%  either because their syntax in \pkg{empheq} then differ from their
%  syntax in \pkg{amsmath}.} because a)~you don't really need markup
%  instructions for a one line display, and b)~would you really like
%  a syntax like
%  \begin{verbatim}
%    \[[box=\fbox] a=b \]
%  \end{verbatim}
%  where its difficult to see whether or not there is a typo?
%  Choosing \opt{overload} is meant for people who don't want to
%  change their entire document into \pkg{empheq} syntax. I have no
%  problems with that; just be careful when you fiddle with
%  \cs{empheqset} as it will affect \emph{all} math displays!
%
%
%  There is of course a catch (well, two actually): These redefined
%  environments don't run as fast as the regular ones (about three
%  times as slow), but in this day and age I seriously doubt you'll
%  be able to tell the difference unless you have a vast number of
%  equations. The other catch is that you cannot use \cs{intertext}
%  and \cs{displaybreak} as described earlier. If you find yourself
%  wanting to use one of these features in say, an \env{align}
%  environment, you have to use the original \env{align} environment.
%  Luckily it is available if you call it like this:
%  \begin{codesyntax}
%    \cs{begin}\arg{AmS\meta{\rmfamily \AmS{} env_name}} \\
%    \texttt{~~}\meta{contents of \AmS{} environment}\\
%    \cs{end}\arg{AmS\meta{\rmfamily \AmS{} env_name}}
%  \end{codesyntax}
%  For example the original \env{align} environment could be selected
%  with
%  \begin{verbatim}
%    \begin{AmSalign}
%      ...
%    \end{AmSalign}
%  \end{verbatim}
%  These original versions with prefix \texttt{AmS} exist for all the
%  environments.
%
%
%  The option \opt{overload2} activates the overloading feature
%  for the pseudo environment |\[ ... \]|, although I doubt you'll
%  find it useful. Beware that this definition is fragile unless you
%  have \eTeX{} as \LaTeX{} engine.\footnote{\eTeX{} has been the
%  default engine for \LaTeX{} in most major distributions since
%  2003.} Not surprisingly \opt{overload2} activates
%  \opt{overload}.
%
%
%
%  Before you get all excited about these options, you should take
%  note of some aspects of centered math displays in \pkg{amsmath}.
%  In certain circumstances truly centering the display is not always
%  the best solution as in this example:
%  \begin{center}
%    \mbox{}\hfill
%    \framebox[.75\linewidth]{Wide math display; not adjusted}%
%    \hfill\mbox{}%
%    \makebox[0pt][r]{Wide Tag}
%  \end{center}
%  Instead centering the display in the available space works pretty
%  well:
%  \begin{gather*}
%    \framebox[.75\linewidth]{Wide math display; adjusted}%
%    \tag*{Wide Tag}
%  \end{gather*}
%  This is what \pkg{amsmath} does normally, but in the \env{gather}
%  environment it does it on a per line basis. This means that we get
%  results like
%  \begin{AmSgather*}
%    \framebox[.75\linewidth]{Wide math display; adjusted}%
%    \tag*{Wide Tag} \\
%    \framebox[.75\linewidth]{Wide math display; not adjusted} \tag{i}%
%  \end{AmSgather*}
%  whether we like it or not. Inside the \env{empheq} environment we
%  need to have the same adjustment for all lines else the boxing
%  process will not work properly, so when activating the
%  \opt{overload} option the above two-line \env{gather} will
%  instead look like this:
%  \begin{empheq}{gather*}
%    \framebox[.75\linewidth]{Wide math display; adjusted}%
%    \tag*{Wide Tag} \\
%    \framebox[.75\linewidth]{Wide math display; adjusted} \tag*{i}%
%  \end{empheq}
%  I leave it to you to choose whether or not this is better (I think
%  it is better).
%
%
%
%  \subsection{A note on boxed displays}
%
%  When browsing a 400+ pages textbook with at least twice the number
%  of displayed formulae, some of them are surely more important than
%  others. Thus the author of the book (in cooperation with the
%  designer) should make sure that such formulae are easily found
%  again i.e., they should be easily distinguishable from the rest of
%  the pack. One way of doing this is putting the formula into a box
%  which is something we've seen \pkg{empheq} being capable of. There
%  are however a few things to keep in mind:
%  \begin{itemize}
%    \item Don't overdo it. If you do it on half of them there's no
%    point in doing it at all. I don't see much reason for applying
%    this technique to more than 10\% of the formulae.
%    \item Choose the type of box carefully. You want to draw
%    attention to it so it might as well look good.
%  \end{itemize}
%  The latter point can be illustrated by defining a macro similar to
%  \cs{fbox} only with a little more space to the right and left of
%  the argument.
%  \begin{verbatim}
%  \newcommand*\widefbox[1]{\fbox{\hspace{1em}#1\hspace{1em}}}
%  \end{verbatim}
%  If we replace the \cs{fbox} from the \key{box} example before, we
%  get this display:
%  \begin{verbatim}
%  \begin{empheq}[box=\widefbox]{align}
%    a&=b\\
%    E&=mc^2 + \int_a^a x\, dx
%  \end{empheq}
%  \end{verbatim}
%  \begin{empheq}[box=\widefbox]{align}
%    a&=b\\
%    E&=mc^2 + \int_a^a x\, dx
%  \end{empheq}
%  Compare it with
%  \begin{empheq}[box=\fbox]{align}
%    a&=b\\
%    E&=mc^2 + \int_a^a x\, dx
%  \end{empheq}
%  and see which one you prefer.
%
%  Similarly one might be tempted to use colored boxes:
%  \begin{verbatim}
%  \definecolor{myblue}{rgb}{.8, .8, 1}
%  \newcommand*\mybluebox[1]{%
%    \colorbox{myblue}{\hspace{1em}#1\hspace{1em}}}
%  \end{verbatim}
%  We know the drill by now.
%  \begin{verbatim}
%  \begin{empheq}[box=\mybluebox]{align}
%    a&=b\\
%    E&=mc^2 + \int_a^a x\, dx
%  \end{empheq}
%  \end{verbatim}
%  \begin{empheq}[box=\mybluebox]{align}
%    a&=b\\
%    E&=mc^2 + \int_a^a x\, dx
%  \end{empheq}
%  There is more on boxes later in this manual. If you're into the
%  \pkg{fancybox} package then remember to read \S
%  \vref{sec:fancybox_support}.
%
%  \section{Support for other packages}
%
%  With the multitude of packages for \LaTeX, it is not always easy
%  to be a PWWO package (``Plays Well With Others''), but
%  \pkg{empheq} tries really hard to do so. This section lists the
%  packages where \pkg{empheq} has to provide workarounds and they
%  can be divided into two categories.
%  \begin{description}
%    \item[Compatibility] Some packages affect \LaTeX' labelling
%    mechanism and since \pkg{empheq} internally has to turn off
%    labels and related code temporarily, hooks must be provided for
%    these packages. Examples of such packages are \pkg{hyperref} and
%    \pkg{showkeys}.
%
%    \item[Enhancements] Other packages provide useful features that
%    for some reason may not work directly or optimally with
%    \pkg{empheq}. In these cases the problematic commands are
%    redefined so that they not only work with \pkg{empheq}, but
%    also give the same if not better output. An example of this
%    would be the \cs{shadowbox} command from \pkg{fancybox}
%    (described below).
%  \end{description}
%  %
%  \subsection{Support for \pkg{fancybox}}\label{sec:fancybox_support}
%
%  The \pkg{fancybox} package provides various boxes and you can use
%  them with \pkg{empheq} as well. Here's \cs{ovalbox}:
%  \begin{verbatim}
%  \begin{empheq}[box=\ovalbox]{align}
%      E&=mc^2 \\
%      Y&= \sum_{n=1}^\infty \frac{1}{n^2}
%  \end{empheq}
%  \end{verbatim}
%  \begin{empheq}[box=\ovalbox]{align}
%      E&=mc^2 \\
%      Y&= \sum_{n=1}^\infty \frac{1}{n^2}
%  \end{empheq}
%
%
%  \begin{codesyntax}
%  \SpecialUsageIndex{\shadowbox}\SpecialUsageIndex{\shadowbox*}
%    \cs{shadowbox*}\marg{arg} \\
%  \SpecialColorIndex{shadowcolor} \texttt{shadowcolor}
%  \end{codesyntax}
%  The only problem with using the \pkg{fancybox} boxes in
%  conjunction with the \env{empheq} environment is \cs{shadowbox},
%  as this macro typesets the shadow on the baseline and for the
%  equation numbers to be placed correctly, the box command must set
%  its argument on the baseline. Thus the normal \cs{shadowbox}
%  produces \shadowbox{\meta{arg}}, but we need this instead:
%  \shadowbox*{\meta{arg}}. Therefore \pkg{empheq} will detect if
%  \pkg{fancybox} is loaded and in that case it'll enhance
%  \cs{shadowbox} in two ways:
%  \begin{itemize}
%    \item It defines a starred version \cs{shadowbox*} which
%    typesets its argument on the baseline.
%
%    \item The color \texttt{shadowcolor} is introduced. The default
%    color is black.
%  \end{itemize}
%  \begin{verbatim}
%  \definecolor{shadowcolor}{rgb}{0,.5,.5}
%  \setlength\shadowsize{2pt}
%  Line of text for \shadowbox*{testing of \Verb|f#&%\| verbatim} and
%  showing the shadow color.
%  \end{verbatim}
%  \begin{quote}
%  \definecolor{shadowcolor}{rgb}{0,.5,.5}
%  \setlength\shadowsize{2pt}
%  Line of text for \shadowbox*{testing of \Verb|f#&%\| verbatim} and
%  showing the shadow color.
%  \end{quote}
%
%  The point is that if you want a \cs{shadowbox} around your math
%  display then you must use the starred version:
%  \begin{verbatim}
%  \begin{empheq}[box=\shadowbox*]{align}
%      E&=mc^2 \\
%      Y&= \sum_{n=1}^\infty \frac{1}{n^2}
%  \end{empheq}
%  \end{verbatim}
%  \begin{empheq}[box=\shadowbox*]{align}
%      E&=mc^2 \\
%      Y&= \sum_{n=1}^\infty \frac{1}{n^2}
%  \end{empheq}
%
%  See \S \vref{stunning} if you want
%  to make a fancy box yourself.
%
%
%  \subsection{Support for \pkg{ntheorem}}\label{sec:ntheorem-support}
%
% \changes{v2.13}{2012/05/09}{Added caveat}
%  \textbf{Caveat:} Due to an unfortunate interaction between \pkg{empheq} and
%  \pkg{ntheorem}, users may want to add
% \begin{verbatim}
% \usetagform{default}
% \end{verbatim}
%  \emph{after} loading \pkg{ntheorem} otherwise the tags may be
%  placed wrong and any labels within the \env{empheq} environment may
%  be lost. In the future \pkg{ntheorem} should be able to test for
%  this and add \verb|\usetagform{default}| automatically.
%
%  Users who use \verb|\usetagform| to get another tag design, should
%  make sure to postpone this configuration until \emph{after} loading
%  \pkg{ntheorem}. 
%
% \bigskip
%
%  \changes{v2.00}{2004/04/14}{Added description of the
%  \opt{ntheorem} option}
%  \changes{v1.05}{2004/02/16}{Added support for the end-of-theorem
%  marker mechanism of \pkg{ntheorem}}
%
%  \begin{codesyntax}
%    \SpecialOptIndex{ntheorem}\opt{ntheorem}
%  \end{codesyntax}
%  The \pkg{ntheorem} package is supported by the means of the
%  \opt{ntheorem} option of \pkg{empheq}. This loads a set of
%  extra macros which fixes various compatibility problems between
%  \pkg{ntheorem} and \pkg{amsmath} and furthermore introduces
%  special (internal) macros for optimum positioning of
%  end-of-theorem markers, while retaining a user friendly interface.
%
%
%  When you want to use the automatic end-of-theorem marker mechanism
%  from \pkg{ntheorem} you sometimes run into problems as you would
%  want the marker to be placed aligned at the bottom of the math
%  display but still keeping the tags in their proper place. In
%  \mode{leqno} mode this is not that much of a problem as the tags
%  and the marker are set on either side of the math display like
%  in Figure \vref{fig:markerI}.
%  \changes{v1.05a}{2004/02/17}{Added explanatory drawings for
%  \pkg{ntheorem}}
%
%  \begin{figure}[btp]\centering
%  \unitlength2.5pt \thicklines
%  \begin{picture}(90,30)(-30,40)
%    \put(0,50){\framebox(25,20){Math Height}}
%    \put(0,40){\framebox(25,10){Math Depth}}
%    \put(-30,50){\framebox(20,10){Tag height}}
%    \put(-30,45){\framebox(20,5){Tag depth}}
%    \put(40,40){\framebox(20,10){Marker}}
%    \multiput(26,40)(2,0){7}{\line(1,0){1}}
%  \end{picture}
%  \caption{Marker placement in \mode{leqno} mode}
%  \label{fig:markerI}
%  \end{figure}
%
%  \begin{codesyntax}
%    \SpecialUsageIndex{\mintagvsep}\cs{mintagvsep}
%  \end{codesyntax}
%  Unfortunately things are not this easy in \mode{reqno} mode. There
%  are two possible situations as shown in Figure \vref{fig:markerII}.
%  \begin{figure}[btp]
%  \unitlength2.5pt \thicklines
%  \begin{picture}(60,45)(0,25)
%    \put(0,50){\framebox(25,20){Math Height}}
%    \put(0,25){\framebox(25,25){Math Depth}}
%    \put(40,50){\framebox(20,10){Tag height}}
%    \put(40,45){\framebox(20,5){Tag depth}}
%    \put(40,25){\framebox(20,10){Marker}}
%    \put(50,35){\vector(0,1){10}}
%    \put(50,45){\vector(0,-1){10}}
%    \put(50,39){\ \ $>\delta$}
%    \multiput(26,25)(2,0){7}{\line(1,0){1}}
%  \end{picture} \hfill\vrule\hfill
%  \begin{picture}(60,45)(0,25)
%    \put(0,50){\framebox(25,20){Math Height}}
%    \put(0,40){\framebox(25,10){Math Depth}}
%    \put(40,50){\framebox(20,10){Tag height}}
%    \put(40,45){\framebox(20,5){Tag depth}}
%    \put(40,28){\framebox(20,10){Marker}}
%    \put(50,38){\vector(0,1){7}}
%    \put(50,45){\vector(0,-1){7}}
%    \put(50,40){\ \ $=\delta$}
%  \end{picture}
%  \caption{The two possible situations for end marks in \mode{reqno}
%  mode}
%  \label{fig:markerII}
%  \end{figure}
%  If possible we want the bottom of the marker to be aligned with
%  the bottom of the math display, but at the same time we want to
%  ensure a minimum vertical separation $\delta$ between marker and
%  tag. In \pkg{empheq} this is controlled by the length parameter
%  \cs{mintagvsep} which by default is 5\,pt.
%
%  The good news is that this is where \pkg{empheq} sets in.
%  Basically all you need to do is to use the \env{empheq}
%  environment to type set your equations inside the theorem
%  environments:
%  \begin{verbatim}
%    \begin{Theorem}
%    Some text at first and then a math display
%    \begin{empheq}{align}
%      a&=b\\
%      E&=mc^2 + \int_a^a x\, dx
%    \end{empheq}
%    \end{Theorem}
%  \end{verbatim}
%  Then the tag placement and the end-of-theorem marker will be set
%  properly (after a couple of runs as usual).
%
%  To ensure the correct outcome remember to load the packages like
%  this:
%  \begin{verbatim}
%    \usepackage[ntheorem]{empheq} % this loads amsmath as well
%    \usepackage[thmmarks,amsmath]{ntheorem}
%  \end{verbatim}
%  Remember that if you use the \opt{overload} option you
%  can just use the regular math environments to get the desired
%  result with the end mark. The exception is the |\[ ... \]|
%  environment which will only work if you use the option
%  \opt{overload2}.
%
%  \clearpage
%
%  \EmphEqpart{Advanced user's guide}
%
%  The \pkg{empheq} package has more to offer than you have seen, but
%  I found some of the functionality so dangerous it was best to hide
%  it for a little while. The commands you have encountered so far in
%  this manual all have one thing in common: They have only lower
%  case letters in their names. Now it's time to reveal those that
%  have \emph{mixed-case} names, thus implying that you should take
%  great care when using them.
%
%  \section{Delimiters revisited}
%  Let's go back to delimiters, shall we?
%
%  \subsection{Creating your own delimiters}
%  As a convenience for the user, this interface is extended in a
%  more general way so that it is also possible to declare delimiters
%  with the following two commands: \SpecialUsageIndex{\DeclareLeftDelimiter}
%  \SpecialUsageIndex{\DeclareRightDelimiter}
%  \begin{codesyntax}
%    \cs{DeclareLeftDelimiter}\oarg{space adjustment}\marg{delimiter}\\
%    \cs{DeclareRightDelimiter}\oarg{space adjustment}\marg{delimiter}
%  \end{codesyntax}
%  While \pkg{empheq} provides auto-scaling versions of the most
%  common delimiters, you may sometimes want some new ones. Say for
%  instance you have loaded the \pkg{stmaryrd} package and you want
%  to use the double bracket commands \cs{llbracket} and
%  \cs{rrbracket} with \pkg{empheq}. Then you simply do this:
%  \begin{verbatim}
%    \DeclareLeftDelimiter{\llbracket}
%    \DeclareRightDelimiter{\rrbracket}
%  \end{verbatim}
%  This defines the new delimiters \cs{empheqllbracket},
%  \cs{empheqbigllbracket}, \cs{empheqrrbracket} and
%  \cs{empheqbigrrbracket}.
%
%  You can use \cs{big...} delimiters as well if you don't like the
%  automatic scaling. There are however ways to fine-tune if you
%  really want it.
%
%  \subsection{Fine-tuning of delimiters}
%
%  \TeX{} provides two primitives to control the scaling of
%  delimiters produced with \cs{left} and \cs{right}, namely the
%  dimension \cs{delimitershortfall} (denoted by $\delta$ here) and
%  the integer \cs{delimiterfactor} ($f$). The idea is that the
%  sub-formula inside the \cs{left}-\cs{right} pair is to be
%  vertically centered and given its height $h_1$ and its depth $h_2$
%  we want to produce a delimiter with total height $h$, where  $h =
%  2 \max(h_1,h_2)$. \TeX' rules on this is that the minimum
%  delimiter size $h_{\min}$ must meet the requirements
%  \[
%    h_{\min}\geq h \frac{f}{1000} \quad\land \quad h_{\min}\geq h-\delta
%  \]
%  \LaTeX{} sets $\cs{delimitershortfall} = \text{\the\delimitershortfall} $
%  and $\cs{delimiterfactor} = \the\delimiterfactor$, but in our case
%  we will almost always want a delimiter that spans the entire
%  \cs{left}-\cs{right} pair (the math display), thus a change of
%  these settings is needed.
%
%  \begin{codesyntax}
%    \SpecialUsageIndex{\EmphEqdelimitershortfall}
%    \cs{EmphEqdelimitershortfall}\\
%    \SpecialUsageIndex{\EmphEqdelimiterfactor}
%    \cs{EmphEqdelimiterfactor}
%  \end{codesyntax}
%  However it is a bad idea to just change these two settings without
%  thinking of the effect it'll have on the rest of the mathematics
%  in your document. Therefore \pkg{empheq} provides the parameters
%  \cs{EmphEqdelimitershortfall} (default setting is
%  \the\EmphEqdelimitershortfall) and \cs{EmphEqdelimiterfactor}
%  (default is \the\EmphEqdelimiterfactor) to cater for this.
%
%
%  \subsection{Scaling material yourself}
%
%  The attentive reader may have noticed that I still haven't
%  revealed how I managed to get the right size of that big
%  exclamation mark in the first example of this manual, so I guess
%  it's about time.
%  \begin{codesyntax}
%    \SpecialUsageIndex{\EmphEqdisplayheight}\cs{EmphEqdisplayheight} \\
%    \SpecialUsageIndex{\EmphEqdisplaydepth}\cs{EmphEqdisplaydepth}
%  \end{codesyntax}
%  The height of the math display plus the surrounding inner box is
%  given by \cs{EmphEqdisplayheight} and the depth by
%  \cs{EmphEqdisplaydepth}. These can be used when constructing
%  something that should scale to fit the display and can't be done
%  with ``auto-scaling'' commands like \cs{vrule} or \cs{left} or
%  \cs{right}. There is a catch however: If you do any horizontal
%  resizing of the material you want to scale, then you must specify
%  the width \emph{manually}:\footnote{The
%  \cs{fontdimen22}\cs{textfont2} in the \meta{length} argument of
%  \cs{raisebox} is the \emph{math axis} of the font. It is needed in
%  order to get the vertical positioning of the oversized exclamation
%  mark just right.}
%  \begin{verbatim}
%  \begin{empheq}[
%    box=\colorbox{myblue},
%    right={\;
%           \makebox[.9em]{%
%            $\raisebox{-.5\totalheight+\fontdimen22\textfont2}
%              {\resizebox{!}{%
%                \EmphEqdisplayheight+\EmphEqdisplaydepth}{!}}$%
%           }},
%    left={X=Y\Rightarrow}
%     ]{alignat=3}
%    A_1&=b_1 & \qquad c_1&=d_1 & \qquad e_1&= f_1
%         \tag*{\_A\raisebox{1ex}{silly}\raisebox{-1ex}{tag}\_}\\
%    A_2&=b_2 & \qquad c_2&=d_2 & \qquad  e_2&= f_2 \\
%    A_3&=b_3 & \qquad c_3&=d_3 & \qquad  e_3&= f_3
%  \end{empheq}
%  \end{verbatim}
%
%  Here's another example. We want to be able to put a \cs{parbox}
%  with some descriptive text top-aligned on the side of the display.
%  \begin{verbatim}
%  \begin{empheq}[
%    left={\parbox[c][\EmphEqdisplayheight+\EmphEqdisplaydepth][t]
%           {4.5cm}
%           {You may find this kind of description useful.}\enspace}]
%      {align}
%    a&=\int_0^1 x\,dx +\frac{foo + bar}{baz}\\
%    E&= mc^2
%  \end{empheq}
%  \end{verbatim}
%  \begin{empheq}[
%    left={\parbox[c][\EmphEqdisplayheight+\EmphEqdisplaydepth][t]
%           {4.5cm}
%           {You may find this kind of description useful.}\enspace}]
%      {align}
%    a&=\int_0^1 x\,dx +\frac{foo + bar}{baz}\\
%    E&= mc^2
%  \end{empheq}
%
%
%  \section{A few short notes}
%
%  \subsection{About $\cs{eqref}$}
%
%  Internally \env{empheq} separate the displayed math from
%  the corresponding equation numbers such that we can add special
%  delimiters or boxes. This is done by nullifying (or rather
%  conveniently redefining) an internal command called
%  \cs{maketag@@@}. Unfortunately, this component is also used by
%  \cs{eqref} to typeset a reference to an equation number, so in
%  earlier versions, \cs{eqref} would not work inside an \env{empheq}
%  environment.  
%
% In the current version this should now be working, but if you ever
% need to mess with \cs{eqref} it may be handy to know what is being
% done. In essence we do something similar to this:
% \begin{verbatim}
%  \let\empheqeqrefbase\textup
%   ... % next go inside empheq env
%  \renewcommand\eqref[1]{\empheqeqrefbase{%
%       \let\maketag@@@\EQsavedmaketag%
%       \tagform@{\ref{##1}}}}
% \end{verbatim}
% Thus if you need to alter things inside \cs{eqref} and need that to
% work within \env{empheq} as well, you may get away with redefining
% \cs{empheqeqrefbase}.
%
%
%
%  \subsection{About changes to $\cs{baselineskip}$}
%
%  Users should never mess with \cs{baselineskip} directly, it is not
%  the correct manner to get double spacing. Have a look at say the
%  \pkg{setspace} package or similar, or play with
%  \cs{baselinestretch} followed by \cs{normalsize} to initiate. 
%
%  Explanation: We use \cs{parbox} inside to place the display box and
%  the box containing the eqn numbers. To do its stuff \cs{parbox}
%  resets some internal settings, in this case, \cs{baselineskip} is
%  reset to \cs{normalbaselineskip}. So if you don't remember to reset
%  that one as well\dots
%
%
%  \section{Creating something new}
%
%  You can create your own environments and boxes to go with
%  \pkg{empheq}, but there are certain things that must be fulfilled
%  to get it to work properly.
%
%  \subsection{New \env{empheq}-like environments}
%
%  \begin{codesyntax}
%    \SpecialEnvIndex{EmphEqMainEnv}\env{EmphEqMainEnv}\\
%    \SpecialUsageIndex{\EmphEqMainEnv}\cs{EmphEqMainEnv}\\
%    \SpecialUsageIndex{\endEmphEqMainEnv}\cs{endEmphEqMainEnv}
%  \end{codesyntax}
%  Although the real work in \pkg{empheq} is done by an environment
%  (\env{EmphEqMainEnv}), you cannot use it like that yourself when
%  you want to define an \env{empheq}-like environment. Due to
%  technical reasons (see~\cite{Ams99}), you have to use a slightly
%  different syntax:
%  \begin{codesyntax}
%    \cs{newenvironment}
%      \marg{env_name} \oarg{num} \oarg{default}\\
%      \texttt{~~}\arg{\meta{other commands} \cs{EmpheqMainEnv}}\\
%      \texttt{~~}\arg{\cs{endEmpheqMainEnv}}
%  \end{codesyntax}
%  In the above environment the \meta{other commands} part must
%  contain certain things before it'll work.
%  \begin{codesyntax}
%    \SpecialKeyFamilyIndex{EmphEqEnv}\keyfam{EmphEqEnv}\\
%    \SpecialKeyFamilyIndex{EmphEqOpt}\keyfam{EmphEqOpt}
%  \end{codesyntax}
%  Before your homemade environment will work you must set at least
%  one key. The key family \keyfam{EmphEqOpt} is for all the markup
%  instructions like \key{left} and \key{box}, while
%  \keyfam{EmphEqEnv} controls the type of \AmS{} environment. You
%  must set the \keyfam{EmphEqEnv} family before \cs{EmhpEqMainEnv}
%  else you'll get an error message.
%
%  One example could be something like this:
%  \begin{verbatim}
%  \newenvironment{important}[2][]{%
%    \setkeys{EmphEqEnv}{#2}%
%    \setkeys{EmphEqOpt}{box=\mybluebox,#1}%
%    \EmphEqMainEnv}%
%    {\endEmphEqMainEnv}
%  \end{verbatim}
%  Thus
%  \begin{verbatim}
%  \begin{important}{gather}
%    a = b + c +d \\ e = f
%  \end{important}
%  \end{verbatim}
%  produces
%  \begin{important}{gather}
%    a = b + c +d \\ e =  f
%  \end{important}
%  while
%  \begin{verbatim}
%  \begin{important}[left={A=B \Rightarrow \empheqlbrace}]{alignat=2}
%    a &= b &\quad c &= d \\
%    \text{this} &= \text{that} &\quad \mathit{fish}&\neq fish
%  \end{important}
%  \end{verbatim}
%  produces
%  \begin{important}[left={A=B \Rightarrow \empheqlbrace}]{alignat=2}
%    a &= b &\quad c &= d \\
%    \text{this} &= \text{that} &\quad \mathit{fish}&\neq fish
%  \end{important}
%
%
%  \subsection{Creating fancy boxes}\label{stunning}
%
%  As a final example I will show you how to create complicated
%  displays involving (vertically) asymmetrical boxes like
%  \cs{shadowbox*}. In order to get the correct output, the contents
%  of the box must be placed on the baseline as in \fbox{\cs{fbox}}.
%  In this example we want to put a set of equations into a bright
%  yellow box and then add another box with some explanatory text at
%  the top of the yellow box making them overlap.
%
%  First we define the colors used and allocate the boxes. We could
%  probably use scratch boxes, but this is safer.
%  \begin{verbatim}
%  \definecolor{shadecolor}{cmyk}{0,0,0.41,0}
%  \definecolor{light-blue}{cmyk}{0.25,0,0,0}
%  \newsavebox{\mysaveboxM} % M for math
%  \newsavebox{\mysaveboxT} % T for text
%  \end{verbatim}
%  Save the display body in \cs{mysaveboxM} and the text argument in
%  \cs{mysaveboxT}.
%  \begin{verbatim}
%  \newcommand*\Garybox[2][Example]{%
%    \sbox{\mysaveboxM}{#2}%
%    \sbox{\mysaveboxT}{\fcolorbox{black}{light-blue}{#1}}%
%  \end{verbatim}
%  Then typeset the math display in a \cs{parbox} where we control
%  the height and save it in \cs{mysaveboxM}.
%  \begin{verbatim}
%    \sbox{\mysaveboxM}{%
%      \parbox[b][\ht\mysaveboxM+.5\ht\mysaveboxT+.5\dp\mysaveboxT][b]{%
%        \wd\mysaveboxM}{#2}%
%    }%
%  \end{verbatim}
%  We put it into the colored box with the desired width.
%  \begin{verbatim}
%    \sbox{\mysaveboxM}{%
%      \fcolorbox{black}{shadecolor}{%
%        \makebox[\linewidth-10em]{\usebox{\mysaveboxM}}%
%      }%
%    }%
%  \end{verbatim}
%  Then finally we get to the real typesetting. We just insert the
%  math display and then make a box of zero width right next to it.
%  In that box we lift the text argument and center it at the top of
%  the display.
%  \begin{verbatim}
%    \usebox{\mysaveboxM}%
%    \makebox[0pt][r]{%
%      \makebox[\wd\mysaveboxM][c]{%
%        \raisebox{\ht\mysaveboxM-0.5\ht\mysaveboxT
%                  +0.5\dp\mysaveboxT-0.5\fboxrule}{\usebox{\mysaveboxT}}%
%      }%
%    }%
%  }
%  \end{verbatim}
%  Let's see what it looks like.
%  \begin{verbatim}
%  \begin{empheq}[box=\Garybox]{align}
%    \sum \mathbf{F} &= \mathbf{0} \\
%    \sum F_{x} \,\mathbf{i} + \sum F_{y} \,\mathbf{j}
%      + \sum F_{z}  \,\mathbf{k} &= \mathbf{0} \\
%    \sum F_{x} &= 0 \\ \sum F_{y} &= 0 \\ \sum F_{z} &= 0
%  \end{empheq}
%  \end{verbatim}
%  \begin{subequations}
%  \begin{empheq}[box=\Garybox]{align}
%    \sum \mathbf{F} &= \mathbf{0} \\
%    \sum F_{x} \,\mathbf{i} + \sum F_{y} \,\mathbf{j} + \sum F_{z}
%    \,\mathbf{k} &= \mathbf{0}\\
%  \sum F_{x} &= 0 \\ \sum F_{y} &= 0 \\ \sum F_{z} &= 0
%  \end{empheq}
%  If we use the optional argument of \cs{Garybox} we have to enclose
%  the entire argument of \key{box} in braces because we're in
%  \env{empheq}'s optional argument (as described in \cite[page
%  167]{Lamp94}):
%  \begin{verbatim}
%  \begin{empheq}[box={\Garybox[Same old example again]}]{align}
%    a&=b\\
%    E&=mc^2 + \int_a^a x\, dx
%  \end{empheq}
%  \end{verbatim}
%  \begin{empheq}[box={\Garybox[Same old example again]}]{align}
%    a&=b\\
%    E&=mc^2 + \int_a^a x\, dx
%  \end{empheq}
%  \end{subequations}
%
%
%
%
%  \section{Contact information}
%
%  Should you have any feature request, suggestions, or bug reports
%  then feel free to contact Lars Madsen at
%  \begin{quote}
%    \email{daleif@imf.au.dk}
%  \end{quote}
%  Please put `empheq' in the subject line.
%
%  \subsection*{Contributors}
%
%  \begin{itemize}
%    \item In November 2002 Lars Madsen (\email{daleif@imf.au.dk})
%    asked for some features that wasn't readily available with
%    \pkg{amsmath} or any other package. This was the start of
%    \pkg{empheq}.
%
%    \item Gary Gray (\email{gray@engr.psu.edu}) gave me some bug
%    reports on the old version (v0.7) which convinced me to rewrite
%    the package completely. The \cs{Garybox} example is inspired by
%    a wish from Gary.
%
%    \item Uwe Siart (\email{uwe.siart@tum.de}) has been a thorough
%    beta-tester on v1.00 of the package.
%
%    \item Bernard Alfonsi (\email{alfonsi@math.u-psud.fr}) reported
%     a problem with the \opt{ntheorem} option. It turned out to be a
%     bug in the \pkg{ntheorem} package itself but it's fixed in the
%     option.
%
%    \item I have received reports on weird behavior from
%    \pkg{empheq} when used with the \pkg{color} package on some
%    dvi-viewers, which is caused by the lack of color support in
%    those dvi-viewers. Both Andrew B.~Collier (\email{colliera@ukzn.ac.za})
%    and Andr\'e M.~de Roos (\email{aroos@science.uva.nl}) have
%    notified me of this. The solution is to either upgrade you
%    dvi-viewer if possible or convert the document in question to
%    PDF or PS.
%  \end{itemize}
%  Thank you all.
%
%
% \begin{thebibliography}{99}
% \bibitem{Lamp94}
%   Leslie Lamport,
%   \emph{LaTeX User's Guide and Reference Manual}, 2nd edition,
%   Addison-Wesley, Reading, Massachusetts,
%   1994.
% \bibitem{Goos-etal94}
%   Michel Goossens, Frank Mittelbach, and Alexander Samarin,
%   \emph{The \LaTeX\ Companion},
%   Addison-Wesley, Reading, Massachusetts,
%   1994.
% \bibitem{Knut86}
%   Donald Ervin Knuth,
%   \emph{The \TeX book},
%   Addison-Wesley, Reading, Mas\-sa\-chu\-setts,
%   1986.
%  \bibitem{Ams99}
%    American Mathematical Society and Michael Downes,
%    \emph{Technical notes on the \pkg{amsmath} package} Version 2.0,
%    1999/10/29.
%    (Available from CTAN as file \texttt{technote.tex}.)
%  \bibitem{Ams00}
%    Frank Mittelbach, Rainer Sch\"opf, Michael Downes, and David M.~Jones,
%    \emph{The \pkg{amsmath} package} Version 2.13,
%    2000/07/18.
%    (Available from CTAN as file \texttt{amsmath.dtx}.)
%
%  \changes{v1.05b}{2004/02/18}{Added \pkg{fancybox} and
%  \pkg{ntheorem} to the \refname}
%  \bibitem{fancybox}
%    Timothy Van Zandt,
%    \emph{The \pkg{fancybox} package} Version 1.3,
%    2000/09/19
%    (Available from CTAN as file \texttt{fancybox.sty}.)
%
% \bibitem{Raht03}
%   Sebastian Rahtz,
%   \emph{Hypertext marks in \LaTeX} Version v6.74m,
%   2003/11/30.
%   (Available from \url{http://www.tug.org/applications/hyperref} as file
%   \texttt{hyperref.dtx}.)
%
% \bibitem{Carl99}
%   David Carlisle,
%   \emph{The \pkg{keyval} Package},
%   Version 1.13, 1999/03/16.
%   (Available from CTAN as file \texttt{keyval.dtx}.)
%
%  \changes{v1.06}{2004/03/13}{Added \pkg{calc} to the \refname}
%  \bibitem{Thorup:calc:1998}
%  Kresten~Krab Thorup, Frank Jensen, and Chris Rowley.
%  \emph{The {\pkg{calc}} package}, Version 4.1b, 1998/07/07,
%  (Available from CTAN as file \texttt{calc.dtx}.)
%
%  \bibitem{ntheorem}
%    Wolfgang May and Andreas Schlechte,
%    \emph{The \pkg{ntheorem} package} Version 1.203,
%    2002/01/07,
%    (Available from CTAN as file \texttt{ntheorem.dtx}.)
%
%  \changes{v2.10}{2004/07/26}{Added \pkg{showkeys} to the \refname}
%  \bibitem{Carlisle:showkeys:1997}
%  David Carlisle. \emph{The {\pkg{showkeys}} package}, v3.12,
%  1997/06/12, (Available from CTAN as file \texttt{showkeys.dtx}.)
%
%  \end{thebibliography}
%
%  \StopEventually{}
%
%  \EmphEqpart{Implementation}
%
%  \section{Requirements, Options, and Strategy}
%  Lets start the package. We'll need the tools from \pkg{mhsetup}.
%    \begin{macrocode}
%<*package>
\ProvidesPackage{empheq}%
  [2014/07/16 v2.14 Emphasizing equations]
%    \end{macrocode}
%    \begin{macrocode}
\RequirePackage{mhsetup}[2004/10/10]
\MHInternalSyntaxOn
%    \end{macrocode}
%
%
%  \subsection{Requirements, Options, and Allocation}
%
%  The \opt{overload} option allows the user to type
%  \begin{verbatim}
%    \begin{gather}[box=\fbox]
%      a = b
%    \end{gather}
%  \end{verbatim}
%  when they really mean
%  \begin{verbatim}
%    \begin{empheq}[box=\fbox]{gather}
%      a = b
%    \end{empheq}
%  \end{verbatim}
%    \begin{macrocode}
\MH_new_boolean:n {overload_ams}
\DeclareOption{overload}{
  \MH_set_boolean_T:n {overload_ams}
}
\MH_new_boolean:n {overload_amsII}
\DeclareOption{overload2}{
  \MH_set_boolean_T:n {overload_ams}
  \MH_set_boolean_T:n {overload_amsII}
}
%    \end{macrocode}
%  Old crappy \env{multline} interface or the new dazzling?
%    \begin{macrocode}
\MH_new_boolean:n {good_mult}
\DeclareOption{oldmultline}{\MH_set_boolean_F:n {good_mult}}
\DeclareOption{newmultline}{\MH_set_boolean_T:n {good_mult}}
%    \end{macrocode}
%
%
%  Load the \pkg{ntheorem} support file.
%    \begin{macrocode}
\gdef\g@EQ@ntheorem@support@bool{01}
\DeclareOption{ntheorem}{\gdef\g@EQ@ntheorem@support@bool{00}}
%    \end{macrocode}
%  Pass other options to \pkg{mathtools}.
%    \begin{macrocode}
\DeclareOption*{
  \PassOptionsToPackage{\CurrentOption}{mathtools}
}
\ExecuteOptions{newmultline}
\ProcessOptions\relax
%    \end{macrocode}
%  The name of the game is \pkg{amsmath} so we'll need that of
%  course. Since v2.10 that is now taken care of in the
%  \pkg{mathtools} package which'll pass on the options to
%  \pkg{amsmath}.
%    \begin{macrocode}
\MHInternalSyntaxOff
\RequirePackage{mathtools}[2004/10/10]
%    \end{macrocode}
%  Activate the internal syntax.
%    \begin{macrocode}
\MHInternalSyntaxOn
\AtEndOfPackage{\MHInternalSyntaxOff}
%    \end{macrocode}
%  \changes{v2.00}{2004/04/14}{Changed description of naming
%  conventions. Altered \emph{all} private macros; this is not
%  mentioned at each macro as this would just be too troublesome}
%
%  \begin{macro}{\l_EQ_equationtype_int}
%  \begin{macro}{\l_EQ_alignmentmode_int}
%  I'll use the following numbering scheme for the equation types,
%  and store the value in \cs{l_EQ_equationtype_int}:
%  \begin{multicols}{2}
%  \begin{enumerate}\setlength{\itemsep}{0pt}
%    \item align
%    \item align*
%    \item gather
%    \item gather*
%    \item flalign
%    \item flalign*
%    \item alignat
%    \item alignat*
%  \end{enumerate}
%  \end{multicols}
%  As you notice \env{equation} and \env{equation*} are missing. This
%  is because \env{gather} and \env{gather*} work the same way, only
%  they are `nicer'. Cuts down on the amount of code as well. I will
%  add an alias so that a user can use the key \key{equation}, but
%  then \key{gather} will kick in instead behind the scenes. Also the
%  \env{multline} environments are missing here, but again it's
%  because I'm faking it. See \S~\vref{sec:flalign+multline} for
%  further details. The value $0$ isn't used for any of the
%  environments and thus serves as a checkpoint (see
%  page~\pageref{ZeroCheckpoint}).
%
%  Similarly I'll use the counter \cs{l_EQ_alignmentmode_int} to
%  store the standard options for placement of math displays and
%  equation numbering.
%  \begin{multicols}{2}
%  \begin{enumerate}\setlength{\itemsep}{0pt}
%    \item reqno
%    \item leqno
%    \item fleqn, reqno
%    \item fleqn, leqno
%  \end{enumerate}
%  \end{multicols}
%
%    \begin{macrocode}
\newcount\l_EQ_equationtype_int
\newcount\l_EQ_alignmentmode_int
\if@fleqn
  \iftagsleft@
    \l_EQ_alignmentmode_int=4
  \else:
    \l_EQ_alignmentmode_int=\thr@@
  \fi:
\else:
  \iftagsleft@
    \l_EQ_alignmentmode_int=\tw@
  \else:
    \l_EQ_alignmentmode_int=\@ne
  \fi:
\fi:
%    \end{macrocode}
%  \end{macro}
%  \end{macro}
%  \begin{macro}{\l_EQ_totalwidth_dim}
%  \begin{macro}{\l_EQ_displaywidth_dim}
%  \begin{macro}{\l_EQ_temp_linewidth_dim}
%  \begin{macro}{\EmphEqdelimitershortfall}
%  \begin{macro}{\EmphEqdelimiterfactor}
%  \begin{macro}{\g_EQ_toptag_height_dim}
%  \begin{macro}{\g_EQ_bottomtag_depth_dim}
%  \begin{macro}{\g_EQ_toprow_height_dim}
%  \begin{macro}{\g_EQ_bottomrow_depth_dim}
%  \begin{macro}{\g_EQ_widesttag_dim}
%  Some lengths we'll need. Could possibly be scratch registers but we
%  have to make global assignments so that road is a little
%  dangerous.
%    \begin{macrocode}
\newdimen\l_EQ_totalwidth_dim
\newdimen\l_EQ_displaywidth_dim
%    \end{macrocode}
%  Just to make it accessible.
%    \begin{macrocode}
\def\EmphEqdisplaywidth{\l_EQ_displaywidth_dim}
\newdimen\l_EQ_temp_linewidth_dim
\newdimen\EmphEqdelimitershortfall
\newcount\EmphEqdelimiterfactor
\EmphEqdelimiterfactor950
\EmphEqdelimitershortfall\p@
\newdimen\g_EQ_toptag_height_dim
\newdimen\g_EQ_bottomtag_depth_dim
\newdimen\g_EQ_toprow_height_dim
\newdimen\g_EQ_bottomrow_depth_dim
\newdimen\g_EQ_widesttag_dim
\def\EmphEqtagwidth{\g_EQ_widesttag_dim}
%    \end{macrocode}
%  \end{macro}
%  \end{macro}
%  \end{macro}
%  \end{macro}
%  \end{macro}
%  \end{macro}
%  \end{macro}
%  \end{macro}
%  \end{macro}
%  \end{macro}
%
%  \begin{macro}{\EQ_outerbox:}
%  \begin{macro}{\EQ_innerbox:}
%  \begin{macro}{\EQ_left_side:}
%  \begin{macro}{\EQ_right_side:}
%  The two boxes and the material on either side of the math display.
%    \begin{macrocode}
\def\EQ_outerbox:{}
\def\EQ_innerbox:{}
\def\EQ_left_side:{}
\def\EQ_right_side:{}
%    \end{macrocode}
%  \end{macro}
%  \end{macro}
%  \end{macro}
%  \end{macro}
%
% \subsection{Strategy}
%
%  In non-\mode{fleqn} math displays are centered---even when having
%  tags. However this can look rather appalling when either the tag
%  or the math display are very wide like in the following example.
%  \begin{center}
%  \makebox[\linewidth][s]{\mbox{}\hfill \mbox{$\displaystyle
%    a+b+c+d+e+f+g+h+i+j+k+l+m+n+o$} \hfill\mbox{} \llap{(wide tag)}}
%  \end{center}
%  Not exactly pleasing to look at. \pkg{amsmath} takes care to do
%  this adjustment for us, so the user instead obtains the far better
%  \begin{gather*}
%  a+b+c+d+e+f+g+h+i+j+k+l+m+n+o \tag{wide tag}
%  \end{gather*}
%  So \pkg{amsmath} handles this quite well. That's good. It looks
%  like \pkg{amsmath} has an internal threshold that matches the
%  width of the math display against the width of the tags. But what
%  if we have a multiple line display and one of the lines causes an
%  adjustment to kick in while the others don't need it? Then we
%  cannot keep track of the complete display anymore. Furthermore we
%  may want to add material on either side of the math display making
%  the whole thing even more complicated, if not impossible to do
%  without some sort of different approach.
%
%  Luckily we have such an approach. The idea that comes to mind is
%  to see if one can typeset the labels and the math displays
%  \emph{separately}, store them in separate boxes and then put them
%  back together again afterwards. Is this really possible? Yes.
%  Come, I'll show you.
%
%  \subsection{The Main Environment}
%  \begin{macro}{\EQ_mathdisplay_box}
%  \begin{macro}{\EmphEqdisplayheight}
%  \begin{macro}{\EmphEqdisplaydepth}
%  \begin{macro}{\EQ_tag_box}
%  Okay, we'll start by allocating two box registers.
%    \begin{macrocode}
\newsavebox\EQ_mathdisplay_box
\def\EmphEqdisplayheight{\ht\EQ_mathdisplay_box}
\def\EmphEqdisplaydepth{\dp\EQ_mathdisplay_box}
\newsavebox\EQ_tag_box
%    \end{macrocode}
%  \end{macro}
%  \end{macro}
%  \end{macro}
%  \end{macro}
%
%  \begin{macro}{\EQ_complete_mathdisplay:}
%  \begin{macro}{\EQ_getbody:n}
%  \begin{environment}{EmphEqMainEnv}
%  The \env{EmphEqMainEnv} is (as the name implies) the main
%  environment. I decided to go with the environment form because
%  that's what \pkg{amsmath} users are accustomed to. But the code
%  will require the contents of the environment to be handled as an
%  argument to a command, so we use \cs{collect@body} from
%  \pkg{amsmath} itself to collect it.
%    \begin{macrocode}
\def\EQ_complete_mathdisplay:{}
\def\EQ_getbody:n #1{%
  \def\EQ_complete_mathdisplay:{}\def\EQ_mathbody:{#1}}
\newenvironment{EmphEqMainEnv}{\collect@body\EQ_getbody:n}{%
%    \end{macrocode}
%  If the user didn't specify the mandatory argument we will catch it
%  because \cs{l_EQ_equationtype_int} is zero. An error is
%  issued\label{ZeroCheckpoint}.
%    \begin{macrocode}
  \if_num:w \l_EQ_equationtype_int=\z@ %ZeroCheckpoint
    \EQ_upgrade_error:
  \else:
%    \end{macrocode}
%  Those that read the manual get the reward here: The package will
%  actually work!
%    \begin{macrocode}
    \EQ_post_process:
    \ignorespacesafterend
  \fi:}
%    \end{macrocode}
%  \end{environment}
%  \end{macro}
%  \end{macro}
%  \subsection{Post Processing}
%  \begin{macro}{\EQ_post_process:}
%  I know: it almost looked easy for a while didn't it? Sadly the
%  post-processing step of the code is also by far the most
%  complicated.
%    \begin{macrocode}
\def\EQ_post_process:{%
%    \end{macrocode}
%  First a hook where we initialize a few things like storing the
%  line width for later manipulation. Also we want to measure the
%  height of the top row and the depth of the bottom row for best
%  possible placement of the math display and the boxes around it
%  (same with the tags). Those four dimensions are globally set
%  during the code, hence this assignment. Most other stuff is local.
%    \begin{macrocode}
  \EQ_initialize:
%    \end{macrocode}
%  Then we proceed by completing the input. We store it in the
%  control sequence \cs{EQ_complete_mathdisplay}.
%  \cs{EQ_begin_equation:} adds the start of the environment and
%  \cs{EQ_end_equation:} adds the corresponding end.
%    \begin{macrocode}
  \EQ_begin_equation:
%    \end{macrocode}
%  In case we face a \env{multline} we add what's necessary.
%    \begin{macrocode}
  \MH_if_boolean:nT {outer_mult}{
    \setkeys{\MT_options_name:}{
      firstline-afterskip=0pt,
      lastline-preskip=0pt,
      multlined-width=0pt,
    }
    \ifodd \l_EQ_alignmentmode_int
      \setkeys{\MT_options_name:}{multlined-pos=b}
    \else:
      \setkeys{\MT_options_name:}{multlined-pos=t}
    \fi:
      \g@addto@macro\EQ_complete_mathdisplay:{
        \begin{MTmultlined}\relax}
  }
  \g@addto@macro\EQ_complete_mathdisplay:\EQ_mathbody:
  \MH_if_boolean:nT {outer_mult}{
    \g@addto@macro\EQ_complete_mathdisplay:{\end{MTmultlined}}
  }
  \EQ_end_equation:
%    \end{macrocode}
%  Save the value of the \pkg{hyperref} link counter
%  \cs{Hy@linkcounter}.
%    \begin{macrocode}
  \edef\EQ_hyperref_link_fint{\number\Hy@linkcounter}
%    \end{macrocode}
%  Firstly we call \cs{EQ_typeset_tags:} which suppresses all
%  output from the equations with the exception of the \cs{tag}'s.
%  This also allows us to find the widest tag which we'll need for
%  the environments that set's the math displays flush to the
%  margins, namely the \env{flalign} family. We save the counters
%  first though. As an extra treat we're also able to measure the
%  height and depth of the tags and the math display.
%    \begin{macrocode}
  \savecounters@
  \EQ_typeset_tags:
%    \end{macrocode}
%  \changes{v1.05d}{2004/02/26}{Removed one instance of
%  \cs{EQ_typeset_display:}}
%  Then find the widest tag. In an earlier implementation it was
%  necessary to typeset the display twice---a very costly procedure!
%  Now we save roughly 2120 lines of output from \cs{tracingall} on a
%  simple \env{gather} and 2600 lines with \env{align}. That's  a
%  little more than 21\% saved\ldots
%  \changes{v2.10}{2004/07/26}{Removed \cs{find_widesttag:}}
%  \changes{v2.10}{2004/07/26}{Added check for equation number to
%  make sure the labelling/printing mechanism of \pkg{mathtools} work
%  properly}
%  Then we restore the counters and measure the extra material.
%  Save the value of the equation counter for later in case the
%  special equation numbering scheme from \pkg{mathtools} is in
%  effect.
%    \begin{macrocode}
  \edef\l_EQ_equation_fint{\number\c@equation}
  \restorecounters@
%    \end{macrocode}
%  Restore the value of the \pkg{hyperref} link counter
%  \cs{Hy@linkcounter}.
%    \begin{macrocode}
  \Hy@linkcounter=\EQ_hyperref_link_fint
  \settowidth\l_EQ_totalwidth_dim{%
    $\EQ_left_side:\EQ_innerbox:{}\EQ_right_side:$}%
%    \end{macrocode}
%  Then call the appropriate \cs{EQ_modify_linewidth_}\meta{num}\texttt{:}
%  where \meta{num} is taken from the numbering scheme shown earlier.
%    \begin{macrocode}
  \MH_if_boolean:nTF {outer_mult}
    { \def\EQ_next:{_mult} }
    { \def\EQ_next:{} }
  \@nameuse{EQ_modify_linewidth_
            \romannumeral\l_EQ_equationtype_int\EQ_next: :}%
%    \end{macrocode}
%  We can then typeset the display knowing that all modifications
%  have been made correctly. \cs{EQ_typeset_display:} gobbles the
%  \cs{tag}'s, which in turn is equivalent to setting the display
%  centered in non-\mode{fleqn} mode.
%    \begin{macrocode}
  \EQ_typeset_display:
%    \end{macrocode}
%  Call the appropriate \cs{find_displaywidth_}\meta{num}\texttt{:}
%  which extracts the display width (\cs{l_EQ_displaywidth_dim}).
%    \begin{macrocode}
  \@nameuse{find_displaywidth_
           \romannumeral\l_EQ_equationtype_int\EQ_next: :}
  \MH_addtolength:dn
    \l_EQ_totalwidth_dim{\l_EQ_displaywidth_dim}
%    \end{macrocode}
%  We save the math display in a box of the correct size.
%    \begin{macrocode}
  \savebox\EQ_mathdisplay_box[\l_EQ_totalwidth_dim][c]{%
    $\EQ_left_side:
    \EQ_innerbox:{%
      \if_case:w \l_EQ_alignmentmode_int\or:
%    \end{macrocode}
%  If the equation is centered we make the box centered
%    \begin{macrocode}
        \makebox[\l_EQ_displaywidth_dim][c]
          {\usebox{\EQ_mathdisplay_box}}%
      \or:
        \makebox[\l_EQ_displaywidth_dim][c]
          {\usebox{\EQ_mathdisplay_box}}%
      \or:
%    \end{macrocode}
%  Else we set it in a left aligned box and take care of the
%  \cs{@mathmargin} unless we're in \env{flalign} or \env{flalign*}.
%    \begin{macrocode}
        \makebox[\l_EQ_displaywidth_dim][l]{%
          \if_num:w \l_EQ_equationtype_int=5 \else:
            \if_num:w \l_EQ_equationtype_int=6 \else:
              \kern-\@mathmargin
            \fi:
          \fi:
          \usebox{\EQ_mathdisplay_box}}%
      \or:
        \makebox[\l_EQ_displaywidth_dim][l]{%
          \if_num:w \l_EQ_equationtype_int=5 \else:
            \if_num:w \l_EQ_equationtype_int=6 \else:
              \kern-\@mathmargin
            \fi:
          \fi:
          \usebox{\EQ_mathdisplay_box}}%
      \fi:}%
    \EQ_right_side:$}%
%    \end{macrocode}
%  Then we're ready for typesetting. We just use an ordinary
%  |\begin{equation*}| | ... | |\end{equation*}| construction.
%  \changes{v2.00}{2004/04/14}{Used a saved definition of
%  \env{equation*} instead to allow overloading}
%  With v2.00 this changed to using a copy of the original
%  \env{equation*} environment. All this because of the
%  \opt{overload} option.
%    \begin{macrocode}
  \begin{AmSequation*}
  \@nameuse{EQ_place_display_
           \romannumeral\l_EQ_equationtype_int\EQ_next: :}
%    \end{macrocode}
%  \cs{tag} must have a non-zero width argument in order to function
%  properly, so we make sure there is a tag before we try to typeset
%  it. Also make sure that the boolean |show_manual_tags| is true
%  just for this tag.
%    \begin{macrocode}
  \if_dim:w \g_EQ_widesttag_dim>\z@
  \MH_set_boolean_T:n {show_manual_tags}
%    \end{macrocode}
%  Set the tag in a left aligned box when in \mode{leqno} mode and
%  right aligned when in \mode{reqno} mode.
%    \begin{macrocode}
    \if_case:w \l_EQ_alignmentmode_int
    \or: \EQ_place_tag:N r %#1
    \or: \EQ_place_tag:N l %#2
    \or: \EQ_place_tag:N r %#3
    \or: \EQ_place_tag:N l %#4
    \fi:
  \fi:
  \end{AmSequation*}
%    \end{macrocode}
%  \changes{v2.11}{2004/08/03}{Use raw \TeX{} commands to speed it up}
%  Restore the equation counter.
%    \begin{macrocode}
  \global\c@equation=\l_EQ_equation_fint\relax
}
%    \end{macrocode}
%  \end{macro}
%
%  \subsubsection{Initializing the Equations}
%
%  \begin{macro}{\EQ_initialize:}
%  We save all the initializations in this hook:
%    \begin{macrocode}
\def\EQ_initialize:{%
  \m@th
%    \end{macrocode}
%  More support for \pkg{ntheorem}. We use special versions of
%  the tag setting commands in order to produce the correct result.
%  For instance in \mode{leqno} mode you would still want the end of
%  theorem symbol at the bottom of the math display.
%    \begin{macrocode}
  \MH_let:NwN\SetTagPlusEndMark\EQ_tag_plus_endmark:
  \MH_let:NwN\SetOnlyEndMark\EQ_only_endmark:
  \l_EQ_temp_linewidth_dim\linewidth
  \g_EQ_toprow_height_dim\z@
  \g_EQ_bottomrow_depth_dim\z@
  \g_EQ_toptag_height_dim\z@
  \g_EQ_bottomtag_depth_dim\z@
  \g_EQ_widesttag_dim\z@
%    \end{macrocode}
%  \changes{v2.11}{2004/08/03}{Added a \cs{g_EQ_latest_nonzerodepth_row_fint}}
%    \begin{macrocode}
  \xdef\g_EQ_latest_nonzerodepth_row_fint{\z@}
  \edef\EQ_restore_tex_delimiter:{%
    \delimiterfactor\the\delimiterfactor
    \delimitershortfall\the\delimitershortfall\relax}%
  \g@addto@macro\EQ_complete_mathdisplay:
                \EQ_restore_tex_delimiter:
  \delimitershortfall\EmphEqdelimitershortfall
  \delimiterfactor\EmphEqdelimiterfactor
%    \end{macrocode}
%  Disable \cs{intertext} and issue an error message. Using the
%  normal error message for an invalid \cs{intertext} would be less
%  than great because it would say
%  \begin{verbatim}
%  ! Package amsmath Error: Invalid use of \intertext.
%
%  See the amsmath package documentation for explanation.
%  Type  H <return>  for immediate help.
%   ...
%  \end{verbatim}
%  That's not good enough, so we redefine things a bit. In
%  \pkg{amsmath} the command \cs{intertext@} is called inside the
%  math environments and defines \cs{intertext}. We hook into this.
%    \begin{macrocode}
  \MH_let:NwN\intertext@\EQ_intertext:
  \EQ_displaybreak:}
%    \end{macrocode}
%  \end{macro}
%  \begin{macro}{\EQ_intertext:}
%  Then we issue a suitable error message.
%    \begin{macrocode}
\def\EQ_intertext:{%
  \def\intertext##1{\PackageError{empheq}{%
    You~cannot~use~\string\intertext\space inside~the~`empheq'
    \MessageBreak
    environment,~as~`empheq'~creates~an~unbreakable~box}\@eha}}
\def\EQ_displaybreak:{%
  \def\dspbrk@[##1]{\PackageWarning{empheq}{%
    You~cannot~use~\string\displaybreak\space inside~the~`empheq'%
    \MessageBreak
    environment,~as~`empheq'~creates~an~unbreakable~box.}}}
%    \end{macrocode}
%  \end{macro}
%  \begin{macro}{\EQ_new_equationtype:nnn}
%  \begin{macro}{\EQ_new_equationtype_arg:nnn}
%  We have to inform \env{EmphEqMainEnv} which equation type we're
%  using, so we define an interface for that.
%    \begin{macrocode}
\def\EQ_new_equationtype:nnn #1#2#3{%
  \define@key{#1}{#2}[true]{%
    \l_EQ_equationtype_int=#3\relax
    \def\EQ_begin_equation:
      {\g@addto@macro{\EQ_complete_mathdisplay:}{\begin{#2}}}%
    \def\EQ_end_equation:
      {\g@addto@macro{\EQ_complete_mathdisplay:}{\end{#2}}}}}
\def\EQ_new_equationtype_arg:nnn #1#2#3{%
  \define@key{#1}{#2}{%
    \l_EQ_equationtype_int=#3\relax
    \def\EQ_begin_equation:
      {\g@addto@macro{\EQ_complete_mathdisplay:}{\begin{#2}{##1}}}%
    \def\EQ_end_equation:
      {\g@addto@macro{\EQ_complete_mathdisplay:}{\end{#2}}}}}
%    \end{macrocode}
%  \end{macro}
%  \end{macro}
%
%  \subsubsection{Separating Tags and Display}
%
%  \begin{macro}{\EQ_typeset_tags:}
%  Typesetting the tags without the display.
%    \begin{macrocode}
\def\EQ_typeset_tags:{%
  \sbox\EQ_tag_box{%
    \parbox{.5\maxdimen}{%
      \hbadness\@M
%    \end{macrocode}
%  Make sure all skips are zero ands support \pkg{ntheorem}.
%  \changes{v1.05a}{2004/02/17}{Forgot to support \pkg{ntheorem} in
%  both \cs{EQ_typeset_tags:} and \cs{EQ_typeset_display:}.}
%    \begin{macrocode}
      \EQ_prevent_vertical_space:
%    \end{macrocode}
%  Support for \pkg{ntheorem}. This package patches \cs{endgather}
%  and \cs{endalign} with the commands \cs{TagsPlusEndmarks} and
%  \cs{RestoreTags} but we do not want them disturbing our output so
%  we disable them when typesetting the display.
%    \begin{macrocode}
      \EQ_typeset_tags_hook:
%    \end{macrocode}
%  Then use our own versions of \cs{set@gather@field} and
%  \cs{set@field}.
%    \begin{macrocode}
      \MH_let:NwN\set@gather@field\EQ_set_gather_field:
      \MH_let:NwN\set@field\EQ_set_align_field:
%    \end{macrocode}
%  Next two lines are for measuring of the first and last row of a
%  display of the \env{align}, \env{alignat} or \env{flalign} family.
%    \begin{macrocode}
      \MH_set_boolean_T:n {continue_measuring}
      \global\g_EQ_temprow_int\m@ne
      \MH_let:NwN \EQ_saved_maketag:n \maketag@@@
      \def\maketag@@@##1{
        \EQ_maketag:Nn \EQ_saved_maketag:n {##1}}
%    \end{macrocode}
%  Then typeset the display.
%    \begin{macrocode}
      \EQ_complete_mathdisplay:}}}
%    \end{macrocode}
%  \end{macro}
%
%  \begin{macro}{\EQ_predisplay_adjustment:}
%  \changes{v1.05c}{2004/02/25}{Added \cs{EQ_predisplay_adjustment:}}
%
%  \changes{v2.00}{2004/04/14}{Added a missing \cs{normallineskip} in
%  case of extremely big tag height}
%
%  For adjusting vertical spacing in the display.
%    \begin{macrocode}
\def\EQ_predisplay_adjustment:{%
  \if_dim:w \g_EQ_toprow_height_dim<\baselineskip
    \vskip\g_EQ_toprow_height_dim
    \vskip-\baselineskip
  \else:
    \vskip-\normallineskip
  \fi:
  \if_dim:w \g_EQ_toptag_height_dim>\g_EQ_toprow_height_dim
    \if_dim:w \g_EQ_toptag_height_dim>\baselineskip
      \vskip-\g_EQ_toptag_height_dim
      \vskip\baselineskip
      \vskip-\normallineskip
    \fi:
  \fi:}
%    \end{macrocode}
%  \end{macro}
%  \begin{macro}{\EQ_postdisplay_adjustment:}
%  \changes{v1.05c}{2004/02/25}{Added \cs{EQ_postdisplay_adjustment:}}
%  Name says it all.
%    \begin{macrocode}
\def\EQ_postdisplay_adjustment:{%
  \if_dim:w \g_EQ_bottomrow_depth_dim<\dp\strutbox
    \vskip\g_EQ_bottomrow_depth_dim
    \vskip-\dp\strutbox
  \fi:
  \if_dim:w \g_EQ_bottomtag_depth_dim>\g_EQ_bottomrow_depth_dim
    \if_dim:w \g_EQ_bottomtag_depth_dim>\dp\strutbox
      \vskip-\g_EQ_bottomtag_depth_dim
      \vskip\dp\strutbox
    \fi:
  \fi:}
%    \end{macrocode}
%  \end{macro}
%  \begin{macro}{\EQ_typeset_display:}
%  \changes{v1.05c}{2004/02/25}{Now takes two arguments: Either both
%  empty or \cs{EQ_predisplay_adjustment:} \&
%  \cs{EQ_postdisplay_adjustment:}}
%  \changes{v1.05d}{2004/02/26}{Changed again. Uses only
%  \cs{EQ_predisplay_adjustment:} \& \cs{EQ_postdisplay_adjustment:}}
%  We typeset the display without the tags and save it in a box.
%    \begin{macrocode}
\def\EQ_typeset_display:{%
  \sbox\EQ_mathdisplay_box{%
    \parbox{\l_EQ_temp_linewidth_dim}{%
%    \end{macrocode}
%  We adjust the vertical spacing if needed:
%    \begin{macrocode}
      \EQ_predisplay_adjustment:
      \EQ_prevent_vertical_space:
%    \end{macrocode}
%  Support for \pkg{ntheorem}. This package patches \cs{endgather}
%  and \cs{endalign} with the commands \cs{TagsPlusEndmarks} and
%  \cs{RestoreTags} but we do not want them disturbing our output so
%  we disable them when typesetting the display.
%    \begin{macrocode}
      \EQ_typeset_display_hook:
%    \end{macrocode}
%  Instead of gobbling the tags it's better to use \cs{phantom} since
%  a tall tag will ruin the vertical spacing. We use a special
%  version of \cs{phantom} to help us with other things as well.
%    \begin{macrocode}
      \def\maketag@@@##1{\EQ_maketag:Nn \vphantom{##1}}%
      \EQ_complete_mathdisplay:
%    \end{macrocode}
%  More space adjustment.
%    \begin{macrocode}
      \EQ_postdisplay_adjustment:}}}
%    \end{macrocode}
%  \end{macro}
%  We need a few little helpers:
%  \begin{macro}{\EQ_prevent_vertical_space:}
%  \changes{v2.00}{2004/04/14}{Used \cs{nointerlineskip} instead}
%  As we typeset the math display inside a \cs{parbox} the math
%  environment will add \cs{abovedisplayskip} at the top and
%  \cs{belowdisplayskip} at the bottom. Also we suppress interline
%  skip with \cs{nointerlineskip}.
%    \begin{macrocode}
\def\EQ_prevent_vertical_space:{%
  \nointerlineskip
  \abovedisplayskip\z@  \belowdisplayskip\z@\relax}
%    \end{macrocode}
%  \end{macro}
%  \begin{macro}{\EQ_maketag:Nn}
%  \changes{v1.05d}{2004/02/26}{Changed name from
%  \cs{EmphEqphantom} to \cs{EmphEqMaketag}}
%  This takes care of two things: We need to store the height of the
%  top tag and the depth of the bottom tag for later, but we also
%  need to use a \cs{vphantom} of the tag as it may be taller than
%  the line of math it is attached to (say, if you are using node
%  connectors from \pkg{pstricks} or some other weird stuff).
%    \begin{macrocode}
\def\EQ_maketag:Nn #1#2{%
  \setboxz@h{#2}%
  \if_num:w \row@=\@ne
    \global \g_EQ_toptag_height_dim = \ht\z@
  \fi:
%    \end{macrocode}
%  Similarly the depth of the last row is determined by measuring the
%  depth of box zero for each row and then using the last of the
%  values.
%    \begin{macrocode}
  \global \g_EQ_bottomtag_depth_dim=\dp\z@
%    \end{macrocode}
%  \changes{v2.11}{2004/08/03}{Added a check for being in the last row.}
%  Save the row number for later as the \opt{ntheorem} option will need
%  it.
%    \begin{macrocode}
  \xdef\g_EQ_latest_nonzerodepth_row_fint{\number\row@}
  \@tempdima=\wd\z@
  \if_dim:w \@tempdima > \g_EQ_widesttag_dim
    \global \g_EQ_widesttag_dim = \@tempdima
  \fi:
  #1{#2}%
}
%    \end{macrocode}
%  \end{macro}
%
%  \subsubsection{Measuring \env{gather}}
%  \begin{macro}{\EQ_set_gather_field:}
%  Box zero (\cs{z@}) contains the entire line of math in a
%  \env{gather}, so instead of typesetting it we typeset the phantom
%  of it and call \cs{EQ_measure_gather:}.
%    \begin{macrocode}
\def\EQ_set_gather_field:{%
  \iftagsleft@
    \global\lineht@\ht\z@
  \else:
    \global\lineht@\dp\z@
  \fi:
  \kern\eqnshift@
  %\boxz@ % Original amsmath version
  \EQ_measure_gather:
  \phantom{\boxz@}%
  \hfil}
%    \end{macrocode}
%  \end{macro}
%  \begin{macro}{\EQ_measure_gather:}
%  Measuring the height of the individual lines in a \env{gather} is
%  fairly easy. If \cs{row@} is equal to one, we're in the first
%  line, and the height of the top row is simply given by the height
%  of box zero.
%    \begin{macrocode}
\def\EQ_measure_gather:{%
  \if_num:w \row@=\@ne
    \g_EQ_toprow_height_dim=\ht\z@
  \fi:
%    \end{macrocode}
%  Similarly the depth of the last row is determined by measuring the
%  depth of box zero for each row and then using the last of the
%  values.
%    \begin{macrocode}
  \g_EQ_bottomrow_depth_dim=\dp\z@
  \global\g_EQ_toprow_height_dim\g_EQ_toprow_height_dim
  \global\g_EQ_bottomrow_depth_dim\g_EQ_bottomrow_depth_dim}
%    \end{macrocode}
%  \end{macro}
%
%  \subsubsection{Measuring \env{align}, \env{alignat} and \env{flalign}}
%  \begin{macro}{\EQ_set_align_field:}
%  Same technique as for \cs{EQ_set_gather_field}. Only this time
%  we have to stop the underlying \cs{start@align} from going over
%  the measuring phase twice. That's when the conditional
%  \bool{continue_measuring} comes into the picture.
%    \begin{macrocode}
\def\EQ_set_align_field:{%
  \column@plus
  \iftagsleft@
    \if_dim:w \ht\z@>\lineht@
      \global\lineht@\ht\z@
    \fi:
  \else:
    \if_dim:w \dp\z@>\lineht@
      \global\lineht@\dp\z@
    \fi:
  \fi:
  %\boxz@ % Original amsmath version
  \MH_if_boolean:nT {continue_measuring}{ \EQ_measure_align: }
  \phantom{\boxz@}}
%    \end{macrocode}
%  \end{macro}
%  \begin{macro}{\EQ_measure_align:}
%  We have to measure every single field of the \cs{align@} \ldots\@
%  Oh well, here it comes:
%    \begin{macrocode}
\newcount\g_EQ_temprow_int
\MH_new_boolean:n {still_in_samerow}
\MH_new_boolean:n {continue_measuring}
\def\EQ_measure_align:{%
  \if_num:w \g_EQ_temprow_int=\row@
    \MH_set_boolean_T:n {still_in_samerow}
  \else:
    \if_num:w \g_EQ_temprow_int>\row@
      \global\MH_set_boolean_F:n {continue_measuring}
    \fi:
    \MH_set_boolean_F:n {still_in_samerow}
    \global\@tempdimc=\@tempdimb
    \global\@tempdimb=\g_EQ_bottomrow_depth_dim
  \fi:
  \MH_if_boolean:nTF {continue_measuring}{
    \global\g_EQ_temprow_int=\row@
    \@tempcnta=\row@
    \if_num:w \@tempcnta=\z@
      \@tempdima=\g_EQ_toprow_height_dim
      \g_EQ_toprow_height_dim=\ht\z@\relax
      \if_dim:w \g_EQ_toprow_height_dim<\@tempdima
        \g_EQ_toprow_height_dim\@tempdima
      \fi:
    \fi:
    \MH_if_boolean:nTF {still_in_samerow}{
      \@tempdima=\g_EQ_bottomrow_depth_dim
      \g_EQ_bottomrow_depth_dim=\dp\z@\relax
      \if_dim:w \g_EQ_bottomrow_depth_dim<\@tempdima
        \g_EQ_bottomrow_depth_dim\@tempdima
      \fi:
    }{
      \g_EQ_bottomrow_depth_dim\dp\z@\relax
    }
  }{
    \global\g_EQ_bottomrow_depth_dim=\@tempdimc
  }
  \global\g_EQ_toprow_height_dim\g_EQ_toprow_height_dim
  \global\g_EQ_bottomrow_depth_dim\g_EQ_bottomrow_depth_dim}
%    \end{macrocode}
%  \end{macro}
%  \subsubsection{Typesetting the Tags}
%  \begin{macro}{\EQ_place_tag:N}
%  Fool \LaTeX\ into thinking the tag is only as wide as the widest
%  tag (remember the real width of the \cs{EQ_tag_box} is
%  $0.5$\cs{maxdimen}.
%    \begin{macrocode}
\def\EQ_place_tag:N #1{%
  \tag*{\makebox[\g_EQ_widesttag_dim][#1]{%
    \EQ_vertical_tag_adjustment:n {\usebox{\EQ_tag_box}}}}}
%    \end{macrocode}
%  \end{macro}
%  \begin{macro}{\EQ_vertical_tag_adjustment:n}
%  \changes{v2.00}{2004/04/14}{Added a missing \cs{normallineskip} in
%  case of extremely big tag height}
%  This is a little tricky. We have measured the height of the top
%  row and top tag and the depth of the bottom row and bottom tag.
%  Now we have to adjust the vertical spacing so it comes out just
%  right. Remember we have separated the tags and the math display
%  but we would still like to have them line up as they should! We do
%  it all inside a group so we can use the scratch dimensions safely.
%    \begin{macrocode}
\def\EQ_vertical_tag_adjustment:n #1{{%
  \@tempdima\z@\@tempdimb\z@
  \@tempdimc=\dp\strutbox
  \if_dim:w \g_EQ_toprow_height_dim<\baselineskip
    \MH_setlength:dn
      \@tempdima{\g_EQ_toprow_height_dim-\baselineskip}
  \else:
    \MH_setlength:dn \@tempdima{-\normallineskip}
  \fi:
  \if_dim:w \g_EQ_toptag_height_dim>\g_EQ_toprow_height_dim
    \if_dim:w \g_EQ_toptag_height_dim>\baselineskip
      \MH_addtolength:dn \@tempdima
        {-\g_EQ_toptag_height_dim+\baselineskip-\normallineskip}
    \fi:
  \fi:
  \if_dim:w \g_EQ_bottomrow_depth_dim<\@tempdimc
    \MH_setlength:dn \@tempdimb
      {\g_EQ_bottomrow_depth_dim-\@tempdimc}
  \fi:
%    \end{macrocode}
%  \changes{v2.00}{2004/04/14}{Fixed minor bug}
%  If the last condition is fulfilled we also need to cancel the
%  first condition.
%    \begin{macrocode}
  \if_dim:w \g_EQ_bottomtag_depth_dim>\g_EQ_bottomrow_depth_dim
    \if_dim:w \g_EQ_bottomtag_depth_dim<\@tempdimc
    \else:
      \MH_addtolength:dn \@tempdimb
        {-\g_EQ_bottomtag_depth_dim+\@tempdimc}
    \fi:
  \fi:
  \raisebox{-0.5\@tempdima+0.5\@tempdimb}{#1}}}%
%    \end{macrocode}
%  \end{macro}
%  \section{Adding the Equation Types}
%  Adding an equation type is fairly easy. The outline is this:
%  \begin{list}{}{\leftmargin2em}
%    \item[]|\EQ_new_equationtype{EmphEqEnv}|\marg{\AmS\
%    env}\marg{arabic num}\texttt{:}
%    \item[]|\def\EQ_modify_linewidth_|\meta{roman num}\texttt{:}\\
%    \hspace*{2em}\marg{Optionally alter \cs{l_EQ_temp_linewidth_dim}}
%    \item[]|\def\find_displaywidth_|\meta{roman num}\texttt{:}\\
%    \hspace*{2em}\marg{Set \cs{l_EQ_displaywidth_dim}}
%    \item[]|\def\EQ_place_display_|\meta{roman num}\texttt{:}\\
%    \hspace*{2em}\marg{Make use of \cs{EQ_use_mathdisplay:}}
%  \end{list}
%  \begin{macro}{\EQ_use_mathdisplay:}
%  A shorthand.
%    \begin{macrocode}
\def\EQ_use_mathdisplay:{%
  \EQ_outerbox:{\usebox{\EQ_mathdisplay_box}}}
%    \end{macrocode}
%  \end{macro}
%    \begin{macrocode}
\def\EQ_options_name:{EmphEqOpt}
%    \end{macrocode}
%  \subsection{The \env{equation} Family}
%  Just a few aliases because \env{equation} and \env{equation*}
%  behave strangely. Much better to use \env{gather} instead as this
%  is easier to measure. Besides it's less work (for us, not the
%  compiler). ;-)
%    \begin{macrocode}
\MH_keyval_alias_with_addon:nnnn {EmphEqEnv}{equation}{gather}
  {\EQ_disable_linebreak:}
\MH_keyval_alias_with_addon:nnnn {EmphEqEnv}{equation*}{gather*}
  {\EQ_disable_linebreak:}
%    \end{macrocode}
%  \begin{macro}{\EQ_disable_linebreak:}
%  \changes{v1.05c}{2004/02/25}{Introduced
%  \cs{EQ_disable_linebreak:} to prevent multi line
%  \env{equation}'s}
%  \changes{v1.05c}{2004/02/25}{Fixed it so that it works better}
%  Disable the line breaks in \env{equation} as these are actually
%  possible now, since we're in a \env{gather}! \cs{Let@} does it in
%  the environment, so that's the place to hack.
%    \begin{macrocode}
\def\EQ_disable_linebreak:{%
  \MH_let:NwN \EQ_saved_Let: \Let@
  \def\Let@{\def\\{%
    \PackageError{empheq}{You~ cannot~ use~ \string\\~ in~ equation!}
      {\@eha}}
%    \end{macrocode}
%  Then restore the meaning of \cs{Let@} in case there is a
%  \env{gathered} or alike coming up.
%    \begin{macrocode}
    \MH_let:NwN \Let@ \EQ_saved_Let:
  }%
}
%    \end{macrocode}
%  \end{macro}
%
%  \subsection{The \env{align} Family}
%    \begin{macrocode}
\EQ_new_equationtype:nnn {EmphEqEnv}{align}{1}
\EQ_new_equationtype:nnn {EmphEqEnv}{align*}{2}
%    \end{macrocode}
%    \begin{macrocode}
\def\EQ_modify_linewidth_i:{%
%    \end{macrocode}
%  If $\cs{maxfields@}=2$ we're in a single column \env{align} and
%  thus there is no flexible space.
%    \begin{macrocode}
  \if_num:w \maxfields@=\tw@
  \else:
%    \end{macrocode}
%  But if there is any flexible space we would very much like to
%  adjust it because we add material to the right and left of the
%  math display.
%    \begin{macrocode}
    \setlength{\l_EQ_temp_linewidth_dim}{%
      \linewidth-\l_EQ_totalwidth_dim-\widthof{\EQ_outerbox:{}}}%
  \fi:}
\def\EQ_modify_linewidth_ii:{\EQ_modify_linewidth_i:}
\def\find_displaywidth_i:{%
  \MH_setlength:dn \l_EQ_displaywidth_dim
    {\l_EQ_temp_linewidth_dim-\alignsep@}
  \if_case:w \l_EQ_alignmentmode_int
  \or: % #1:
    \MH_addtolength:dn \l_EQ_displaywidth_dim{-\alignsep@}%
  \or: % #2:
    \MH_addtolength:dn \l_EQ_displaywidth_dim{-\alignsep@}%
  \or: % #3:
    \MH_addtolength:dn \l_EQ_displaywidth_dim{-\@mathmargin}%
  \or: % #4:
    \MH_addtolength:dn \l_EQ_displaywidth_dim{-\@mathmargin}%
  \fi:}
\def\find_displaywidth_ii:{\find_displaywidth_i:}
\def\EQ_place_display_i: {\EQ_use_mathdisplay:}
\def\EQ_place_display_ii:{\EQ_place_display_i:}
%    \end{macrocode}
%
%  \subsection{The \env{gather} Family}
%  This is a gem. Notice how easy it is once we got the framework for
%  it.
%    \begin{macrocode}
\EQ_new_equationtype:nnn {EmphEqEnv}{gather}{3}
\EQ_new_equationtype:nnn {EmphEqEnv}{gather*}{4}
\def\EQ_modify_linewidth_iii:{}
\def\EQ_modify_linewidth_iv:{\EQ_modify_linewidth_iii:}
\def\find_displaywidth_iii:{%
  \MH_setlength:dn \l_EQ_displaywidth_dim{\totwidth@}%
  \if_case:w \l_EQ_alignmentmode_int \or: \or: % #1 & #2:
  \or: % #3:
    \MH_addtolength:dn \l_EQ_displaywidth_dim{-\@mathmargin}%
  \or: % #4:
    \MH_addtolength:dn \l_EQ_displaywidth_dim{-\@mathmargin}%
  \fi:}
\def\find_displaywidth_iv:{\find_displaywidth_iii:}
\def\EQ_place_display_iii:{\EQ_use_mathdisplay:}
\def\EQ_place_display_iv: {\EQ_place_display_iii:}
%    \end{macrocode}
%
%  \subsection{The \env{flalign} Family}\label{sec:flalign+multline}
%  This was fairly easy, but then I had to take care of \ldots
%
%  \subsubsection{The \env{multline} Family}
%  This was quite a pickle. Its implementation in \pkg{amsmath} is
%  very strange indeed; for instance the tag is typeset in the
%  measuring phase! So instead I thought I could fake it, but how
%  should I do it?
%
%  The basic idea of \env{multline} is to typeset the first line
%  flush left, the last line flush right and the rest centered.
%  ``Aha,'' I thought, ``what if I make three columns, fool \LaTeX{}
%  into thinking the contents of the columns take no space and then
%  just put each line into a column?'' Thus  the \env{flalign}
%  environment came to mind. I also distinctly remembered to have
%  seen an article about the equivalent of \cs{rlap} and \cs{llap} in
%  math mode, namely an article by Alexander R.~Perlis. This works
%  pretty well, and so does the same trick for producing a
%  \env{multlined} environment. That one uses an \env{alignedat} with
%  three columns.
%
%  \changes{v1.05c}{2004/02/25}{Makes use of
%  \cs{MakeKeyvalAliasAddon} now}
%  Then we make \env{multline} into an alias for {flalign} but with
%  an extra switch \cs{ifEQ_multline:} set to true.
%    \begin{macrocode}
\MH_keyval_alias_with_addon:nnnn {EmphEqEnv}{multline}
  {flalign}{\MH_set_boolean_T:n {multline}}
\MH_keyval_alias_with_addon:nnnn {EmphEqEnv}{multline*}
  {flalign*}{\MH_set_boolean_T:n {multline}}
%    \end{macrocode}
%  \begin{macro}{\if_EQ_multline:}
%  \begin{macro}{\EQ_generic_multline:nnnnn}
%  We make a general command that'll provide the correct spacing and
%  then just define the user commands in terms of that single generic
%  one.
%    \begin{macrocode}
\MH_new_boolean:n {multline}
%    \end{macrocode}
%  \begin{macro}{\EQ_multline_modify_linewidth_tag:}
%  \begin{macro}{\EQ_multline_modify_linewidth_notag:}
%  There is a big difference between the spacing in \env{multline}
%  depending on the presence of a tag. Therefore we split it up in
%  two cases. I made this choice because we need these adjustments
%  inside the adjustments for \env{flalign} and that makes it
%  incredibly difficult to see what happens.
%    \begin{macrocode}
\def\EQ_multline_modify_linewidth_tag:{%
  \if_case:w \l_EQ_alignmentmode_int
  \or: % #1:
    \MH_addtolength:dn \l_EQ_temp_linewidth_dim{%
      -\g_EQ_widesttag_dim-\multlinetaggap-\multlinegap}%
  \or:
    \MH_addtolength:dn \l_EQ_temp_linewidth_dim{%
      -\g_EQ_widesttag_dim-\multlinetaggap-\multlinegap}%
  \or:
    \MH_addtolength:dn \l_EQ_temp_linewidth_dim{%
      -\@mathmargin-\g_EQ_widesttag_dim-\multlinetaggap}%
  \or:
    \MH_addtolength:dn
      \l_EQ_temp_linewidth_dim{-\@mathmargin-\multlinegap}%
  \fi:
}
\def\EQ_multline_modify_linewidth_notag:{%
  \if_case:w \l_EQ_alignmentmode_int\or:
    \MH_addtolength:dn \l_EQ_temp_linewidth_dim{-2\multlinegap}
  \or:
    \MH_addtolength:dn \l_EQ_temp_linewidth_dim{-2\multlinegap}
  \or:
    \MH_addtolength:dn
      \l_EQ_temp_linewidth_dim{-\@mathmargin-\multlinegap}%
  \or:
    \MH_addtolength:dn
      \l_EQ_temp_linewidth_dim{-\@mathmargin-\multlinegap}%
  \fi:
}
%    \end{macrocode}
%  \end{macro}
%  \end{macro}
%
%  \begin{environment}{MTmultlined}
%  Save the superior \env{multlined} from \pkg{mathtools} under a new
%  name.
%    \begin{macrocode}
\MH_let:NwN \MTmultlined \multlined
\MH_let:NwN \endMTmultlined \endmultlined
%    \end{macrocode}
%  \end{environment}
%  We make a choice here: the good \env{multline} or the bad?
%    \begin{macrocode}
\MH_if_boolean:nTF {good_mult}{
%    \end{macrocode}
%  The good:
%    \begin{macrocode}
\MH_keyval_alias_with_addon:nnnn {EmphEqEnv}{multline}{gather}
  {\MH_set_boolean_T:n {outer_mult}}
\MH_keyval_alias_with_addon:nnnn {EmphEqEnv}{multline*}{gather*}
  {\MH_set_boolean_T:n {outer_mult}}
\def\EQ_modify_linewidth_iii_mult:{
  \setlength{\l_EQ_temp_linewidth_dim}{%
    \linewidth-\widthof{\EQ_outerbox:{}}}%
  \if_dim:w \g_EQ_widesttag_dim>\z@
    \EQ_multline_modify_linewidth_tag:
  \else:
    \EQ_multline_modify_linewidth_notag:
  \fi:
  \MH_setlength:dn \l_MT_multwidth_dim{\l_EQ_temp_linewidth_dim}
}
\def\EQ_modify_linewidth_iv_mult:{\EQ_modify_linewidth_iii_mult:}
\def\find_displaywidth_iii_mult:{%
  \MH_setlength:dn \l_EQ_displaywidth_dim{\totwidth@}%
  \if_case:w \l_EQ_alignmentmode_int \or: \or: % #1 & #2:
  \or: % #3:
    \MH_addtolength:dn \l_EQ_displaywidth_dim{-\@mathmargin}%
  \or: % #4:
    \MH_addtolength:dn \l_EQ_displaywidth_dim{-\@mathmargin}%
  \fi:}
\def\find_displaywidth_iv_mult:{\find_displaywidth_iii_mult:}
\def\EQ_place_display_iii_mult:{
  \if_case:w \l_EQ_alignmentmode_int
  \or: % #1:
    \if_dim:w \g_EQ_widesttag_dim>\z@
      \abovedisplayshortskip=\abovedisplayskip
      \rlap{
        \kern-.5\linewidth
        \kern\multlinegap
        \EQ_use_mathdisplay:
      }
    \else:
      \EQ_use_mathdisplay:
    \fi:
  \or: % #2:
    \if_dim:w \g_EQ_widesttag_dim>\z@
      \llap{
        \EQ_use_mathdisplay:
        \kern-.5\linewidth
        \kern\multlinegap
      }%
    \else:
      \EQ_use_mathdisplay:
    \fi:
  \or: % #3:
    \EQ_use_mathdisplay:
  \or: % #4:
    \EQ_use_mathdisplay:
  \fi:
}
\def\EQ_place_display_iv_mult: {\EQ_place_display_iii_mult:}
%    \end{macrocode}
%    \begin{macrocode}
}{
%    \end{macrocode}
%  The bad
%    \begin{macrocode}
\def\EQ_generic_multline:nnnnn #1#2#3#4#5{%
%    \end{macrocode}
%  This is a very tricky little detail. If we're in \mode{leqno} mode
%  we need just a we little width inside the \env{flalign}, so we put
%  in the smallest possible amount. Otherwise we wind up in a
%  \begin{verbatim}
%  ! Arithmetic overflow.
%  <recently read> \count@
%
%  l.304 %  \end{empheq}
%  \end{verbatim}
%  If you scroll past this error it will be typeset correctly, but
%  how many users can fix this by themselves?
%    \begin{macrocode}
  \hskip1sp#2#3{{}#5}#4%
  \if:w *#1\relax\else:\notag\fi:}
%    \end{macrocode}
%  \end{macro}
%  \end{macro}
%  \begin{macro}{\multleft}
%  \begin{macro}{\multcenter}
%  \begin{macro}{\multright}
%  Then the commands that use \cs{EQ_generic_multline:nnnnn}. The
%  first three are the general purpose ones that also have a starred
%  version. The other three are for normal use.
%    \begin{macrocode}
\def\multleft{\@ifstar{%
  \MH_if_boolean:nTF {multlined}{
    \def\EQ_next:{\EQ_multlined_left:n}
  }{
    \def\EQ_next:{
      \EQ_generic_multline:nnnnn {*}{}{\mathrlap}{&&&&&}}
  }
  \EQ_next:}{%
  \MH_if_boolean:nTF {multlined}{
    \def\EQ_next:{\EQ_multlined_left:n}%
  }{
    \def\EQ_next:{
      \EQ_generic_multline:nnnnn {}{}{\mathrlap}{&&&&&}}
  }
  \EQ_next:}}%
\def\multcenter{\@ifstar{%
  \MH_if_boolean:nTF {multlined}{
    \def\EQ_next:{\EQ_multlined_center:n}%
  }{
    \def\EQ_next:{
      \EQ_generic_multline:nnnnn {*}{&&}{\mathclap}{&&&}}
  }
  \EQ_next:}{%
  \MH_if_boolean:nTF {multlined}{
    \def\EQ_next:{\EQ_multlined_center:n }%
  }{
    \def\EQ_next:{
      \EQ_generic_multline:nnnnn {}{&&}{\mathclap}{&&&}}
  }
  \EQ_next:}}%
\def\multright{\@ifstar{%
  \MH_if_boolean:nTF {multlined}{
    \def\EQ_next:{\EQ_multlined_right:n }%
  }{
    \def\EQ_next:{
      \EQ_generic_multline:nnnnn {*}{&&&&&}{\mathllap}{}}
  }
  \EQ_next:}{%
  \MH_if_boolean:nTF {multlined}{
    \def\EQ_next:{\EQ_multlined_right:n }%
  }{
    \def\EQ_next:{
      \EQ_generic_multline:nnnnn {}{&&&&&}{\mathllap}{}}
  }
  \EQ_next:}}
%    \end{macrocode}
%  \end{macro}
%  \end{macro}
%  \end{macro}
%  \begin{macro}{\multfirst}
%  \begin{macro}{\multmiddle}
%  \begin{macro}{\multlast}
%  If we're using the \mode{leqno} option, the equation number should
%  be placed in the first line.
%    \begin{macrocode}
\def\multfirst{%
  \MH_if_boolean:nTF {multlined}{
     \def\EQ_next:{\EQ_multlined_left:n }%
  }{
    \if_case:w \l_EQ_alignmentmode_int
    \or: % #1:
      \def\EQ_next:{\multleft}%
    \or: % #2:
      \def\EQ_next:{\multleft*}%
    \or: % #3:
      \def\EQ_next:{\multleft}%
    \or: % #4:
      \def\EQ_next:{\multleft*}%
    \fi:
  }
  \EQ_next:}
%    \end{macrocode}
%  Pretend the star was never there.
%    \begin{macrocode}
\def\multmiddle{%
  \def\EQ_next:{%
    \MH_if_boolean:nTF {multlined}{
      \expandafter\EQ_multlined_center:n
    }{
      \expandafter\multcenter
    }
  }%
  \@ifstar{\EQ_next:}{\EQ_next:}}
%    \end{macrocode}
%  If we're  using the \mode{reqno} option, the equation number
%  should be placed in the last line.
%    \begin{macrocode}
\def\multlast{%
  \MH_if_boolean:nTF {multlined}{
     \def\EQ_next:{\EQ_multlined_right:n }%
  }{
    \if_case:w \l_EQ_alignmentmode_int
    \or: % #1:
      \def\EQ_next:{\multright*}%
    \or: % #2:
      \def\EQ_next:{\multright}%
    \or: % #3:
      \def\EQ_next:{\multright*}%
    \or: % #4:
      \def\EQ_next:{\multright}%
    \fi:
  }
  \EQ_next:}
%    \end{macrocode}
%  \end{macro}
%  \end{macro}
%  \end{macro}
%  \begin{macro}{\multlinedwidth}
%  Here begins the code for \env{multlined}.
%    \begin{macrocode}
\MH_new_boolean:n {multlined}
\newdimen\multlinedwidth
\MH_setlength:dn \multlinedwidth{.7\linewidth}
%    \end{macrocode}
%  \end{macro}
%  \begin{environment}{multlined}
%  \changes{v2.10}{2004/07/26}{Disallow spaces before optional
%  arguments.}
%  \begin{macro}{\EQ_multlined_I:w}
%  \begin{macro}{\EQ_multlined_II:w}
%  \begin{macro}{\EQ_multlined_III:nn}
%  Some trickery as we want to scan for two optional arguments that
%  may be interchanged.
%    \begin{macrocode}
\renewenvironment{multlined}%
  {\MH_nospace_ifnextchar:Nnn [  {\EQ_multlined_I:w}
                                 {\EQ_multlined_I:w[c]}
  }
  { \endaligned }
\def\EQ_multlined_I:w[#1]{%
  \MH_nospace_ifnextchar:Nnn[  {\EQ_multlined_II:w[#1]}
                               {\EQ_multlined_II:w[#1][\multlinedwidth]}}
\def\EQ_multlined_II:w[#1][#2]{%
%    \end{macrocode}
%  We tell \LaTeX\ that we're in \env{multlined}.
%    \begin{macrocode}
  \MH_set_boolean_T:n {multlined}
%    \end{macrocode}
%  \changes{v2.10}{2004/07/26}{Fixed restoration of \cmd{\\}}
%  Restore the meaning of \cmd{\\} inside \env{multlined}, else it
%  wouldn't work in the \env{equation} environment.
%    \begin{macrocode}
  \MH_let:NwN \Let@\MT_saved_Let:
%    \end{macrocode}
%  Then check the first optional argument and call
%  \cs{EQ_multlined_III:w}, which sets \cs{multlinewidth} if
%  specified and calls \cs{start@aligned} (an \env{alignedat} with
%  three columns).
%    \begin{macrocode}
  \if:w t#1\relax \EQ_multlined_III:nn {#1}{#2}%
  \else:
    \if:w b#1\relax \EQ_multlined_III:nn {#1}{#2}%
    \else:
      \if:w c#1\relax \EQ_multlined_III:nn {#1}{#2}%
      \else:
        \EQ_multlined_III:nn {#2}{#1}%
      \fi:
    \fi:
  \fi:
  \mkern-\thinmuskip
  \EQ_next:}
\def\EQ_multlined_III:nn #1#2{%
  \def\EQ_next:{\setlength\multlinedwidth{#2}%
                          \start@aligned{#1}{\thr@@}}}
%    \end{macrocode}
%  \end{macro}
%  \end{macro}
%  \end{macro}
%  \end{environment}
%  \begin{macro}{\EQ_multlined_left:n}
%  \begin{macro}{\EQ_multlined_center:n}
%  \begin{macro}{\EQ_multlined_right:n}
%  The internal versions of \cs{multleft} etc.\ adapted for
%  \env{multlined}.
%    \begin{macrocode}
\def\EQ_multlined_left:n #1{%
  \mathrlap{#1}&\hskip.5\multlinedwidth&&\hskip.5\multlinedwidth&&}
\def\EQ_multlined_center:n #1{%
  &\hskip.5\multlinedwidth&&\mathclap{#1}\hskip.5\multlinedwidth&&}
\def\EQ_multlined_right:n #1{%
  &\hskip.5\multlinedwidth&&\hskip.5\multlinedwidth&&\mathllap{#1}}
%    \end{macrocode}
%  \end{macro}
%  \end{macro}
%  \end{macro}
%
%  End of bad multline.
%    \begin{macrocode}
}
%    \end{macrocode}
%
%  \subsubsection{The Code for \env{flalign} \& \env{multline}}
%    \begin{macrocode}
\EQ_new_equationtype:nnn {EmphEqEnv}{flalign}{5}
\EQ_new_equationtype:nnn {EmphEqEnv}{flalign*}{6}
\def\EQ_modify_linewidth_v:{%
  \setlength{\l_EQ_temp_linewidth_dim}{%
    \linewidth-\l_EQ_totalwidth_dim-\widthof{\EQ_outerbox:{}}}%
  \if_dim:w \g_EQ_widesttag_dim>\z@
    \MH_if_boolean:nTF {multline}{
      \EQ_multline_modify_linewidth_tag:
    }{
      \if_case:w \l_EQ_alignmentmode_int
      \or: % #1:
        \MH_addtolength:dn  \l_EQ_temp_linewidth_dim
          {-\g_EQ_widesttag_dim-\mintagsep}
      \or: % #2:
        \MH_addtolength:dn \l_EQ_temp_linewidth_dim
          {-\g_EQ_widesttag_dim-\mintagsep}%
      \or: % #3:
        \MH_addtolength:dn \l_EQ_temp_linewidth_dim
          {-\g_EQ_widesttag_dim-\mintagsep}
      \or: % #4:
        \MH_addtolength:dn \l_EQ_temp_linewidth_dim
          {-\@mathmargin}
      \fi:
    }
  \else:
    \MH_if_boolean:nT {multline}{
      \EQ_multline_modify_linewidth_notag:
    }
  \fi:}
\def\EQ_modify_linewidth_vi:{\EQ_modify_linewidth_v:}
\def\find_displaywidth_v:{%
  \MH_setlength:dn
    \l_EQ_displaywidth_dim{\l_EQ_temp_linewidth_dim}}
\def\find_displaywidth_vi:{\find_displaywidth_v:}
\def\EQ_place_display_v:{%
  \if_case:w \l_EQ_alignmentmode_int
  \or: % #1:
    \if_dim:w \g_EQ_widesttag_dim>\z@
      \MH_if_boolean:nTF {multline}{
        \rlap{\kern\multlinegap  \kern-.5\linewidth
          \EQ_use_mathdisplay:
          \kern-\mintagsep  \kern\multlinetaggap}%
      }{
        \rlap{\kern-.5\linewidth\EQ_use_mathdisplay:}%
      }
    \else:
      \EQ_use_mathdisplay:
    \fi:
  \or: % #2:
    \if_dim:w \g_EQ_widesttag_dim>\z@
      \MH_if_boolean:nTF {multline}{
        \llap{\kern-\mintagsep \kern\multlinetaggap
          \EQ_use_mathdisplay:
          \kern\multlinegap  \kern-.5\linewidth}%
      }{
        \llap{\EQ_use_mathdisplay:\kern-.5\linewidth}%
      }
    \else:
      \EQ_use_mathdisplay:
    \fi:
  \or: % #3:
    \MH_if_boolean:nF {multline}{ \kern-\multlinetaggap }
    \EQ_use_mathdisplay:
  \or: % #4:
    \if_dim:w \g_EQ_widesttag_dim>\z@
    \else:
      \MH_if_boolean:nF {multline}{ \kern-\@mathmargin }
    \fi:
    \EQ_use_mathdisplay:
  \fi:
  \MH_set_boolean_F:n {multline}}
\def\EQ_place_display_vi:{\EQ_place_display_v:}
%    \end{macrocode}
%
%  \subsection{The \env{alignat} Family}
%  This is just as easy as \env{gather}. Actually only one line is
%  different from \env{gather}!
%    \begin{macrocode}
\EQ_new_equationtype_arg:nnn {EmphEqEnv}{alignat}{7}
\EQ_new_equationtype_arg:nnn {EmphEqEnv}{alignat*}{8}
\def\EQ_modify_linewidth_vii:{}
\def\EQ_modify_linewidth_viii:{\EQ_modify_linewidth_vii:}
\def\find_displaywidth_vii:{%
  \MH_setlength:dn \l_EQ_displaywidth_dim{\totwidth@}%
  \if_case:w \l_EQ_alignmentmode_int
  \or: % #1:
  \or: % #2:
  \or: % #3:
    \MH_addtolength:dn \l_EQ_displaywidth_dim{-\eqnshift@}%
  \or: % #4:
    \MH_addtolength:dn \l_EQ_displaywidth_dim{-\@mathmargin}%
  \fi:}
\def\find_displaywidth_viii:{\find_displaywidth_vii:}
\def\EQ_place_display_vii:{\EQ_use_mathdisplay:}
\def\EQ_place_display_viii:{\EQ_place_display_vii:}
%    \end{macrocode}
%  \section{The User Interface}
%  The different keys.
%    \begin{macrocode}
\define@key{\EQ_options_name:}{box}{\def\EQ_outerbox:{#1}}
\MH_keyval_alias:nnn {\EQ_options_name:}{outerbox}{box}
%    \end{macrocode}
%  \begin{macro}{\EQ_marginbox:nn}
%  The margin box. Uwe Siart wanted it.
%    \begin{macrocode}
\def\EQ_marginbox:nn #1#2{%
  \settowidth{\@tempdima}{#2}%
  \makebox[\z@]{\hspace{\@tempdima}#1{#2}}}
\define@key{\EQ_options_name:}{marginbox}{%
  \def\EQ_outerbox:{\EQ_marginbox:nn {#1}}}
%    \end{macrocode}
%  \end{macro}
%  Almost self-explanatory, isn't?
%    \begin{macrocode}
\define@key{\EQ_options_name:}{innerbox}{\def\EQ_innerbox:{#1}}
\define@key{\EQ_options_name:}{left}{\def\EQ_left_side:{#1}}
\define@key{\EQ_options_name:}{right}{\def\EQ_right_side:{#1}}
%    \end{macrocode}
%  \begin{macro}{\empheqset}
%  Just a user friendly way to set keys. Uwe Siart again.
%    \begin{macrocode}
\newcommand*\empheqset[1]{\setkeys{\EQ_options_name:}{#1}}
%    \end{macrocode}
%  \end{macro}
%  The old keys from the original version are no longer supported.
%  Instead they give an error message (RM):
%    \begin{macrocode}
\def\EQ_upgrade_error:{%
  \PackageError{empheq}
    {You~ are~ using~ the~ old~ syntax!\MessageBreak
    The~ `empheq'~ environment~ takes~ a~ mandatory~
    argument~ now.\MessageBreak
    You~ need~ to~ exit~ and~ change~ your~ source~ file.}
    {It~ won't~ work,~ trust me.~ Press~ `X'~ now.}}
\define@key{\EQ_options_name:}{boxtype}{\EQ_upgrade_error:}
\define@key{\EQ_options_name:}{Left}{\EQ_upgrade_error:}
\define@key{\EQ_options_name:}{Right}{\EQ_upgrade_error:}
%    \end{macrocode}
% \begin{macro}{\empheqeqrefbase}
%   This macro is used to locally redefine \cs{eqref}, it will just
%   point at \cs{textup}, the user can then redefine it if nessary.
%    \begin{macrocode}
\let\empheqeqrefbase\textup
%    \end{macrocode}
%   
% \end{macro}
%  \begin{environment}{empheq}
%  The environment.
%    \begin{macrocode}
\newenvironment{empheq}[2][]{%
%    \end{macrocode}
% \begin{macro}{\EQ_saved_maketag}
% \changes{v2.13}{2013/02/11}{Fixing \cs{eqref}}
% We start with a small caveat: During processing \cs{maketag@@@} is
% redefined, enabling us to separate math and eqn numbers. But the
% user may have used \cs{eqref} in a comment in the math, and
% \cs{eqref} use \cs{maketag@@@}\dots{} First save the original \cs{maketag@@@}:
%    \begin{macrocode}
  \MH_let:NwN\EQ_saved_maketag: \maketag@@@%
%    \end{macrocode}
% Next redefine \cs{eqref} to locally bring back the original:
%    \begin{macrocode}
  \renewcommand\eqref[1]{\empheqeqrefbase{%
       \let\maketag@@@\EQ_saved_maketag:%
       \tagform@{\ref{##1}}}}
  \setkeys{EmphEqEnv}{#2}\setkeys{\EQ_options_name:}{#1}%
  \EmphEqMainEnv}{\endEmphEqMainEnv}
%    \end{macrocode}
%  \end{macro}
% \end{environment}
%  \begin{macro}{\DeclareLeftDelimiter}
%  \begin{macro}{\EQ_declare_left_delimiter:nnnn}
%  \begin{macro}{\DeclareRightDelimiter}
%  \begin{macro}{\EQ_declare_right_delimiter:nnnn}
%  The delimiter interface.
%  \changes{v0.7}{2003/05/11}{Introduced delimiter interface}
%  \changes{v1.05d}{2004/02/26}{Changed the definitions.}
%  We now use the \cs{EQ_tag_box} to calculate the height but
%  adjust the vertical spacing. All this to prevent running
%  \cs{EQ_typeset_display:} twice.
%    \begin{macrocode}
\newcommand*\DeclareLeftDelimiter[2][]{%
  \EQ_declare_left_delimiter:nnnn {#1}{#2}{}{}%
  \EQ_declare_left_delimiter:nnnn {#1}{#2}{big}{\EQ_innerbox:}}
\def\EQ_declare_left_delimiter:nnnn #1#2#3#4{%
  \@namedef{empheq#3\expandafter\@gobble\string#2}{%
    \left#2%
    \mkern-\thinmuskip
    \vphantom{#4{\parbox{.5\maxdimen}{%
      \EQ_predisplay_adjustment:
      \EQ_prevent_vertical_space:
      \usebox{\EQ_tag_box}%
      \EQ_postdisplay_adjustment:}}}%
    \right.%
    \kern-\nulldelimiterspace #1}}
\newcommand*\DeclareRightDelimiter[2][]{%
  \EQ_declare_right_delimiter:nnnn {#1}{#2}{}{}%
  \EQ_declare_right_delimiter:nnnn {#1}{#2}{big}{\EQ_innerbox:}}
\def\EQ_declare_right_delimiter:nnnn #1#2#3#4{%
  \@namedef{empheq#3\expandafter\@gobble\string#2}{%
    \kern-\nulldelimiterspace
    #1%
    \left.%
    \vphantom{#4{\parbox{.5\maxdimen}{%
      \EQ_predisplay_adjustment:
      \EQ_prevent_vertical_space:
      \usebox{\EQ_tag_box}%
      \EQ_postdisplay_adjustment:}}}%
    \mkern-\thinmuskip
    \right#2}}
%    \end{macrocode}
%  \end{macro}
%  \end{macro}
%  \end{macro}
%  \end{macro}
%  The actual declarations.
%    \begin{macrocode}
\DeclareLeftDelimiter{\lbrace}
\DeclareRightDelimiter{\rbrace}
\DeclareLeftDelimiter{\lbrack}
\DeclareRightDelimiter{\rbrack}
\DeclareLeftDelimiter{\langle}
\DeclareRightDelimiter{\rangle}
\DeclareLeftDelimiter{\lvert}
\DeclareRightDelimiter{\rvert}
\DeclareLeftDelimiter{\lVert}
\DeclareRightDelimiter{\rVert}
\DeclareLeftDelimiter{\lfloor}
\DeclareRightDelimiter{\rfloor}
\DeclareLeftDelimiter{\lceil}
\DeclareRightDelimiter{\rceil}
\DeclareLeftDelimiter{\lparen}
\DeclareRightDelimiter{\rparen}
%    \end{macrocode}
%
%  \section{Support for Other Packages}
%
%  Support for various other packages is found in this section. The
%  \pkg{ntheorem} package now has a dedicated section (\S
%  \vref{sec:impl:ntheorem}), as it now a module.
%
%
%
%    \begin{macrocode}
\def\EQ_typeset_tags_hook:{}
\def\EQ_typeset_display_hook:{}
%    \end{macrocode}
%
%  \subsection{\pkg{showkeys}}
%  \changes{v2.10}{2004/07/26}{Added support for the \pkg{showkeys}
%  package}
%  We just disable \pkg{showkeys} when typesetting the labels.
%    \begin{macrocode}
\g@addto@macro\EQ_typeset_display_hook:{
  \def\SK@@label#1>#2\SK@{}
}
%    \end{macrocode}
%  If we overload the \AmS{} equation types we should still be able
%  to use \pkg{showkeys}.
%    \begin{macrocode}
\@namedef{SK@AmSequation}  {\SK@equationtrue}
\@namedef{SK@AmSequation*} {\SK@equationtrue}
\@namedef{SK@AmSalign}     {\SK@equationtrue}
\@namedef{SK@AmSalign*}    {\SK@equationtrue}
\@namedef{SK@AmSalignat}   {\SK@equationtrue}
\@namedef{SK@AmSalignat*}  {\SK@equationtrue}
\@namedef{SK@AmSgather}    {\SK@equationtrue}
\@namedef{SK@AmSgather*}   {\SK@equationtrue}
\@namedef{SK@AmSmultline}  {\SK@equationtrue}
\@namedef{SK@AmSmultline*} {\SK@equationtrue}
\@namedef{SK@AmSflalign}   {\SK@equationtrue}
\@namedef{SK@AmSflalign*}  {\SK@equationtrue}
%    \end{macrocode}
%
%  \subsection{\pkg{hyperref}}
%
%
%  A little trick to avoid the (in)famous warning
%  \begin{verbatim}
%  ! pdfTeX warning (ext4): destination with the same identifier
%  (name{equation.1a }) has been already used, duplicate ignored
%  <to be read again>
%                     \endgroup \set@typeset@protect
%  \end{verbatim}
%  This doesn't require \pkg{hyperref}.
%    \begin{macrocode}
\AtBeginDocument{
  \@ifundefined{Hy@linkcounter}{\newcount\Hy@linkcounter}{}
}
\g@addto@macro\EQ_typeset_display_hook:{
  \MH_let:NwN \hyper@refstepcounter\@gobble
}
%    \end{macrocode}
% 
% \subsection{\pkg{nccmath}}
%
% Must disable this macro, otherwise the spacing is wrong.
%    \begin{macrocode}
 \g@addto@macro\EQ_typeset_display_hook:{
   \let\NCC@ignorepar\relax
 }
 \g@addto@macro\EQ_typeset_tags_hook:{
   \let\NCC@ignorepar\relax
 }
%    \end{macrocode}
%
%  \subsection{Basic \pkg{ntheorem} support}
%    \begin{macrocode}
\g@addto@macro\EQ_typeset_tags_hook:{
  \MH_let:NwN\TagsPlusEndmarks\relax
  \MH_let:NwN\RestoreTags\relax
}
\g@addto@macro\EQ_typeset_display_hook:{
  \MH_let:NwN\TagsPlusEndmarks\relax
  \MH_let:NwN\RestoreTags\relax
}
%    \end{macrocode}
%  \subsection{Basic \LaTeXe{} support}
%  Gobble the \cs{ltx@label} and disable most error messages.
%  \changes{v2.00}{2004/04/14}{Make a special label-gobbling command
%  to support \pkg{ntheorem} even more}
%    \begin{macrocode}
\g@addto@macro\EQ_typeset_display_hook:{
  \MH_let:NwN\ltx@label\MT_gobblelabel:w
  \MH_let:NwN \GenericError \@gobblefour
}
%    \end{macrode}
%  \subsection{Support for \pkg{fancybox}}
%  First check if \pkg{fancybox} is loaded and then see if
%  \pkg{color} is loaded. If the latter is loaded we define
%  \texttt{shadowcolor} else we let \cs{textcolor} gobble the first
%  argument (the color).
%    \begin{macrocode}
\AtBeginDocument{%
  \@ifpackageloaded{fancybox}{%
    \@ifpackageloaded{color}{\definecolor{shadowcolor}{rgb}{0,0,0}}%
    {\newcommand*\textcolor[2]{\normalcolor{#2}}}%
%    \end{macrocode}
%  \begin{macro}{\shadowbox}
%  Then define the \cs{shadowbox}. Use a conditional to let us know if
%  it was starred or not.
%    \begin{macrocode}
    \MH_new_boolean:n {shadowbox_star}
    \renewcommand*\shadowbox{%
      \@ifstar{\MH_set_boolean_T:n {shadowbox_star}
               \VerbBox\EQ_shadowbox:n}
              {\MH_set_boolean_F:n {shadowbox_star}
               \VerbBox\EQ_shadowbox:n}}
%    \end{macrocode}
%  \end{macro}
%  \begin{macro}{\EQ_shadowbox:n}
%  The internal version of \cs{shadowbox}.
%    \begin{macrocode}
    \def\EQ_shadowbox:n #1{%
      \sbox{\z@}{\fbox{#1}}%
%    \end{macrocode}
%  The difference between the two versions is that \cs{shadowbox}
%  aligns the bottom of the shadow at the baseline whereas
%  \cs{shadowbox*} aligns the argument on the baseline, so for the
%  starred version the trick is to raise the box by the depth of the
%  argument plus the size of the shadow.
%    \begin{macrocode}
      \MH_if_boolean:nTF {shadowbox_star}{
        \MH_setlength:dn \@tempdima{\z@}%
      }{
        \MH_setlength:dn \@tempdima{\dp\z@+\shadowsize}%
      }
      \raisebox{\@tempdima}{%
        \makebox[\z@][l]{%
          \kern\shadowsize \textcolor{shadowcolor}{%
            \rule[-\dp\z@-\shadowsize]{\wd\z@}{\shadowsize}}%
        }%
        \usebox\z@\textcolor{shadowcolor}{%
          \rule[-\dp\z@-\shadowsize]{\shadowsize}{\ht\z@+\dp\z@}}}}%
  }{}%
}
%    \end{macrocode}
%  \end{macro}
%
%
%  \section{The \opt{overload} option}
%  \begin{macro}{\EQ_overload_ams_begin:n}
%  A helper for saving snapshots:
%    \begin{macrocode}
\def\EQ_overload_ams_begin:n #1{%
  \MH_let:cc {AmS#1}{#1}%
  \MH_let:cc {AmS#1*}{#1*}
}
%    \end{macrocode}
%  \end{macro}
%  \begin{macro}{\EQ_overload_ams_end:n}
%  All but one of the aligned variations must be set to the original
%    \begin{macrocode}
\def\EQ_overload_ams_end:n #1{%
  \MH_let:cc {endAmS#1}{endAmSalign}
}
%    \end{macrocode}
%  \end{macro}
%  \begin{environment}{AmSequation}
%  \begin{environment}{AmSequation*}
%  \changes{v2.10}{2004/07/26}{Moved redefinitions to
%  \cs{AtBeginDocument} to give \pkg{hyperref} a chance}
%  Before we start redefining environments, we'll need a basic
%  \env{equation} environment that is not redefined.
%    \begin{macrocode}
\AtBeginDocument{
  \EQ_overload_ams_begin:n {equation}
  \MH_let:cc {endAmSequation}{endequation}
  \MH_let:cc {endAmSequation*}{endequation*}
%    \end{macrocode}
%  \end{environment}
%  \end{environment}
%  Lets start overloading.
%    \begin{macrocode}
  \MH_if_boolean:nT {overload_ams}{
%    \end{macrocode}
%  \begin{macro}{\EQ_renew_ams_environment:n}
%  \changes{v2.10}{2004/07/26}{Disallow spaces before the optional
%  argument.}
%  Then we can renew the environments. Syntax is
%  \cs{begin}\marg{\AmS{} env}\oarg{markup}. Exception to this rule
%  is \env{alignat} (see further down).
%    \begin{macrocode}
    \def\EQ_renew_ams_environment:n #1{%
      \MaybeMHPrecedingSpacesOff
      \renewenvironment{#1}[1][]{%
%    \end{macrocode}
% \changes{v2.13}{2013/02/11}{Adding \cs{eqref} fix}
% Remember to add \cs{eqref} fix to the overloaded envs as well.
%    \begin{macrocode}
        \MH_let:NwN\EQ_saved_maketag: \maketag@@@%
        \renewcommand\eqref[1]{\empheqeqrefbase{%
            \let\maketag@@@\EQ_saved_maketag:%
            \tagform@{\ref{####1}}}}
        \setkeys{\EQ_options_name:}{##1}%
        \setkeys{EmphEqEnv}{AmS#1}%
        \EmphEqMainEnv
      }{\endEmphEqMainEnv}
      \MHPrecedingSpacesOn
    }
%    \end{macrocode}
%  \end{macro}
%  Add equation type: \emph{Must} have same number as original type!
%    \begin{macrocode}
    \EQ_new_equationtype:nnn     {EmphEqEnv}{AmSalign}   {1}
    \EQ_new_equationtype:nnn     {EmphEqEnv}{AmSalign*}  {2}
    \EQ_new_equationtype:nnn     {EmphEqEnv}{AmSgather}  {3}
    \EQ_new_equationtype:nnn     {EmphEqEnv}{AmSgather*} {4}
    \EQ_new_equationtype:nnn     {EmphEqEnv}{AmSflalign} {5}
    \EQ_new_equationtype:nnn     {EmphEqEnv}{AmSflalign*}{6}
    \EQ_new_equationtype_arg:nnn {EmphEqEnv}{AmSalignat} {7}
    \EQ_new_equationtype_arg:nnn {EmphEqEnv}{AmSalignat*}{8}
%    \end{macrocode}
%  Then overload by saving the snapshots of the old definitions. We
%  do it via \cs{AtBeginDocument}, because a package like
%  \pkg{ntheorem} redefines \cs{endgather} etc.
%
%  First the \env{gather} and \env{equation} families.
%    \begin{macrocode}
    \EQ_overload_ams_begin:n {gather}
    \MH_let:cc {endAmSgather}{endgather}
    \MH_let:cc {endAmSgather*}{endAmSgather}
    \EQ_renew_ams_environment:n {gather}
    \EQ_renew_ams_environment:n {gather*}
    \EQ_overload_ams_begin:n {equation}
    \MH_keyval_alias_with_addon:nnnn {EmphEqEnv} {AmSequation}
      {AmSgather}{\EQ_disable_linebreak:}
    \MH_keyval_alias_with_addon:nnnn {EmphEqEnv}{AmSequation*}
      {AmSgather*}{\EQ_disable_linebreak:}
    \EQ_renew_ams_environment:n {equation}
    \EQ_renew_ams_environment:n {equation*}
%    \end{macrocode}
%  The \env{align} family.
%    \begin{macrocode}
    \EQ_overload_ams_begin:n {align}
    \MH_let:cc {endAmSalign}{endalign}
    \EQ_overload_ams_end:n {align*}
    \EQ_renew_ams_environment:n {align}
    \EQ_renew_ams_environment:n {align*}
%    \end{macrocode}
%  The \env{flalign} family.
%    \begin{macrocode}
    \EQ_overload_ams_begin:n {flalign}
    \EQ_overload_ams_end:n {flalign}
    \EQ_overload_ams_end:n {flalign*}
    \EQ_renew_ams_environment:n {flalign}
    \EQ_renew_ams_environment:n {flalign*}
%    \end{macrocode}
%  The \env{alignat} family.
%    \begin{macrocode}
    \EQ_overload_ams_begin:n {alignat}
    \EQ_overload_ams_end:n {alignat}
    \EQ_overload_ams_end:n {alignat*}
%    \end{macrocode}
%  \begin{macro}{\EQ_alignat_optional:w}
%  \changes{v2.10}{2004/07/26}{Disallow spaces before the optional
%  argument.}
%  \begin{environment}{alignat}
%  \begin{environment}{alignat*}
%  This one is a little different because we want a syntax like
%  \cs{begin\{alignat\}}\linebreak[0]\marg{cols}\oarg{markup}, so we
%  have to make a separate command for testing the trailing optional
%  argument.
%    \begin{macrocode}
    \MaybeMHPrecedingSpacesOff
    \newcommand*\EQ_alignat_optional:w [1][]{
      \setkeys{\EQ_options_name:}{#1}
      \EmphEqMainEnv}
    \MHPrecedingSpacesOn
%    \end{macrocode}
% \changes{v2.13}{2013/02/11}{Adding \cs{eqref} fix}
% Remember to add \cs{eqref} fix to the overloaded envs as well.
%    \begin{macrocode}
    \renewenvironment{alignat}[1]{
      \MH_let:NwN\EQ_saved_maketag: \maketag@@@%
      \renewcommand\eqref[1]{\empheqeqrefbase{%
          \let\maketag@@@\EQ_saved_maketag:%
          \tagform@{\ref{##1}}}}
      \setkeys{EmphEqEnv}{AmSalignat=#1}
      \EQ_alignat_optional:w}
      {\endEmphEqMainEnv}
    \renewenvironment{alignat*}[1]{%
      \MH_let:NwN\EQ_saved_maketag: \maketag@@@%
      \renewcommand\eqref[1]{\empheqeqrefbase{%
          \let\maketag@@@\EQ_saved_maketag:%
          \tagform@{\ref{##1}}}}
      \setkeys{EmphEqEnv}{AmSalignat*=#1}%
      \EQ_alignat_optional:w}%
      {\endEmphEqMainEnv}
%    \end{macrocode}
%  \end{environment}
%  \end{environment}
%  \end{macro}
%  \begin{macro}{\[}
%  \begin{macro}{\]}
%  Then restore \cs{[} and \cs{]}.
%    \begin{macrocode}
    \DeclareRobustCommand{\[}{\begin{AmSequation*}}
    \DeclareRobustCommand{\]}{\end{AmSequation*}}
%    \end{macrocode}
%  \end{macro}
%  \end{macro}
%  Finally a string of aliases so that we can blend \opt{overload}
%  syntax with regular syntax.
%    \begin{macrocode}
    \MH_keyval_alias:nnn {EmphEqEnv}{gather}  {AmSgather}
    \MH_keyval_alias:nnn {EmphEqEnv}{gather*} {AmSgather*}
    \MH_keyval_alias:nnn {EmphEqEnv}{align}   {AmSalign}
    \MH_keyval_alias:nnn {EmphEqEnv}{align*}  {AmSalign*}
    \MH_keyval_alias:nnn {EmphEqEnv}{flalign} {AmSflalign}
    \MH_keyval_alias:nnn {EmphEqEnv}{flalign*}{AmSflalign*}
    \MH_keyval_alias:nnn {EmphEqEnv}{alignat} {AmSalignat}
    \MH_keyval_alias:nnn {EmphEqEnv}{alignat*}{AmSalignat*}
    \MH_if_boolean:nT {good_mult}{
      \EQ_overload_ams_begin:n {multline}
      \MH_let:cc {endAmSmultline}{endmultline}
      \MH_let:cc {endAmSmultline*}{endAmSmultline}
      \MH_keyval_alias_with_addon:nnnn {EmphEqEnv}{AmSmultline}
        {AmSgather}{\MH_set_boolean_T:n {outer_mult}}
      \MH_keyval_alias_with_addon:nnnn {EmphEqEnv}{AmSmultline*}
        {AmSgather*}{\MH_set_boolean_T:n {outer_mult}}
      \EQ_renew_ams_environment:n {multline}
      \EQ_renew_ams_environment:n {multline*}
    }
%    \end{macrocode}
%  End of \opt{overload}. We go into \opt{overload2}\ldots This
%  is dangerous, but will probably work. The macro \cs{[} is now
%  fragile except if you're running \eTeX{} as engine for \LaTeX.
%    \begin{macrocode}
    \MH_if_boolean:nT {overload_amsII}{
      \MH_protected:\def\[#1\]{\begin{equation*} #1\end{equation*}}
%    \end{macrocode}
%  Then we add the \env{multline} family.
%    \begin{macrocode}
      \MH_if_boolean:nF {good_mult}{
        \EQ_overload_ams_begin:n {multline}
        \MH_keyval_alias_with_addon:nnnn {EmphEqEnv}{AmSmultline}
          {AmSflalign}{\MH_set_boolean_T:n {multline}}
        \MH_keyval_alias_with_addon:nnnn {EmphEqEnv}{AmSmultline*}
          {AmSflalign*}{\MH_set_boolean_T:n {multline}}
        \EQ_renew_ams_environment:n {multline}
        \EQ_renew_ams_environment:n {multline*}
      }
    }
  }
}
%    \end{macrocode}
%  End of \cs{AtBeginDocument} additions.
%
%  If we need to load \pkg{ntheorem} support this is where it
%  happens. Otherwise we just stop here.
%    \begin{macrocode}
\if\g@EQ@ntheorem@support@bool
\else 
  \expandafter\endinput
\fi
%    \end{macrocode}
%
%  \section{Patching Things up for \pkg{ntheorem}}
%  \label{sec:impl:ntheorem}
%
%  Although \pkg{ntheorem} provides a lot of useful features it
%  does however ruin things in the \pkg{amsmath} environments,
%  because of the alternative syntax of its \cs{label} command.
%  \begin{macro}{\default_raisetag}
%  \begin{macro}{\raisetag}
%  \begin{macro}{\raise@tag}
%  \changes{v1.06}{2004/03/13}{Fixed \pkg{amsmath} bug \# 3624}
%  This corrects a bug\footnote{See
%  \url{http://www.latex-project.org/cgi-bin/ltxbugs2html?pr=amslatex/3624}}
%  in \pkg{amsmath} that could cause tags to cross page boundaries
%  without warning. The \cs{nobreak} is the addition.
%    \begin{macrocode}
\def\default_raisetag:{\nobreak}
\def\raisetag#1{
  \skip@#1\relax
  \xdef\raise@tag{\default_raisetag:
                  \vskip\iftagsleft@\else:-\fi:\the\skip@\relax}}
\MH_let:NwN\raise@tag\default_raisetag:
%    \end{macrocode}
%  \end{macro}
%  \end{macro}
%  \end{macro}
%  \begin{macro}{\displ@y@}
%  Patch it so that it doesn't set \cs{raise@tag} to \cs{@empty}
%  globally.
%    \begin{macrocode}
\def\displ@y@{\@display@init{%
  \global\column@\z@ \global\dspbrk@lvl\m@ne
  \global\tag@false \global\MH_let:NwN\raise@tag\default_raisetag:
}}
%    \end{macrocode}
%  \end{macro}
%  \begin{macro}{\MT_gobblelabel:w}
%  \changes{v2.00}{2004/04/14}{Added \cs{MT_gobblelabel:w}}
%  \begin{macro}{\EQ_gobbleoptional:w}
%  \changes{v2.00}{2004/04/14}{Added \cs{EQ_gobbleoptional:w}}
%  Macros for gobbling the `unorthodox' \cs{label} syntax of
%  \pkg{ntheorem}. Also overrides the normal \cs{label}-gobbling.
%    \begin{macrocode}
\def\MT_gobblelabel:w #1{\@ifnextchar[{\EQ_gobbleoptional:w}{}}
\def\EQ_gobbleoptional:w [#1]{}
%    \end{macrocode}
%  \end{macro}
%  \end{macro}
%  \subsection{Fixing \env{gather}}
%  The measuring phase of \env{gather} is patched by letting
%  \cs{label} be a copy of the internal \cs{MT_gobblelabel:w}
%  instead of \cs{@gobble}.
%    \begin{macrocode}
\def\gmeasure@#1{%
  \begingroup
    \measuring@true
    \totwidth@\z@
    \global\MH_let:NwN\tag@lengths\@empty
    \savecounters@
    \setbox\@ne\vbox{%
      \everycr{\noalign{\global\tag@false
%    \end{macrocode}
%  The patch for \cs{raisetag}.
%    \begin{macrocode}
        \global\MH_let:NwN\raise@tag\default_raisetag:
        \global\column@\z@}}%
%    \end{macrocode}
%  The patch for \cs{label}.
%    \begin{macrocode}
      \MH_let:NwN\label\MT_gobblelabel:w
      \halign{%
        \setboxz@h{$\m@th\displaystyle{##}$}%
        \if_dim:w \wdz@>\totwidth@
          \global\totwidth@\wdz@
        \fi:
        &\setboxz@h{\strut@{##}}%
        \savetaglength@
        \crcr
        #1%
        \math@cr@@@
      }%
    }%
    \restorecounters@
    \if@fleqn
        \global\advance\totwidth@\@mathmargin
    \fi:
    \iftagsleft@
        \if_dim:w \totwidth@>\displaywidth
            \global\MH_let:NwN\gdisplaywidth@\totwidth@
        \else:
            \global\MH_let:NwN\gdisplaywidth@\displaywidth
        \fi:
    \fi:
  \endgroup
}
%    \end{macrocode}
%
%  \subsection{Fixing \env{align}, \env{alignat}, and \env{flalign}}
%  Just copy the definition of \cs{measure@} from \pkg{amsmath} and
%  change one (1!) line like we did for \env{gather}.
%    \begin{macrocode}
\def\measure@#1{%
  \begingroup
    \measuring@true
    \global\eqnshift@\z@
    \global\alignsep@\z@
    \global\MH_let:NwN\tag@lengths\@empty
    \global\MH_let:NwN\field@lengths\@empty
    \savecounters@
    \global\setbox0\vbox{%
      \MH_let:NwN\math@cr@@@\math@cr@@@align@measure
      \everycr{\noalign{\global\tag@false
%    \end{macrocode}
%  The patch for \cs{raisetag}.
%    \begin{macrocode}
        \global\MH_let:NwN\raise@tag\default_raisetag:
        \global\column@\z@}}%
%    \end{macrocode}
%  The patch.
%    \begin{macrocode}
      \MH_let:NwN\label\MT_gobblelabel:w
      \global\row@\z@
      \tabskip\z@
      \halign{\span\align@preamble\crcr
        #1%
        \math@cr@@@
        \global\column@\z@
        \add@amps\maxfields@\cr
      }%
    }%
    \restorecounters@
    \ifodd\maxfields@
      \global\advance\maxfields@\@ne
    \fi:
    \if_num:w \xatlevel@=\tw@
      \if_num:w \maxfields@<\thr@@
        \MH_let:NwN\xatlevel@\z@
      \fi:
    \fi:
    \setbox\z@\vbox{%
      \unvbox\z@ \unpenalty \global\setbox\@ne\lastbox
    }%
    \global\totwidth@\wd\@ne
    \if@fleqn \global\advance\totwidth@\@mathmargin \fi:
    \global\MH_let:NwN\maxcolumn@widths\@empty
    \begingroup
      \MH_let:NwN\or:\relax
      \loop
        \global\setbox\@ne\hbox{%
          \unhbox\@ne \unskip \global\setbox\thr@@\lastbox
        }%
      \ifhbox\thr@@
        \xdef\maxcolumn@widths{ \or: \the\wd\thr@@ \maxcolumn@widths}%
      \repeat
    \endgroup
    \dimen@\displaywidth
    \advance\dimen@-\totwidth@
    \if_case:w \xatlevel@
      \global\alignsep@\z@
      \MH_let:NwN\minalignsep\z@
      \@tempcntb\z@
      \if@fleqn
        \@tempcnta\@ne
        \global\eqnshift@\@mathmargin
      \else:
        \@tempcnta\tw@
        \global\eqnshift@\dimen@
        \global\divide\eqnshift@\@tempcnta
      \fi:
    \or:
      \@tempcntb\maxfields@
      \divide\@tempcntb\tw@
      \@tempcnta\@tempcntb
      \advance\@tempcntb\m@ne
      \if@fleqn
        \global\eqnshift@\@mathmargin
        \global\alignsep@\dimen@
        \global\divide\alignsep@\@tempcnta
      \else:
        \global\advance\@tempcnta\@ne
        \global\eqnshift@\dimen@
        \global\divide\eqnshift@\@tempcnta
        \global\alignsep@\eqnshift@
      \fi:
    \or:
      \@tempcntb\maxfields@
      \divide\@tempcntb\tw@
      \global\advance\@tempcntb\m@ne
      \global\@tempcnta\@tempcntb
      \global\eqnshift@\z@
      \global\alignsep@\dimen@
      \if@fleqn
        \global\advance\alignsep@\@mathmargin\relax
      \fi:
      \global\divide\alignsep@\@tempcntb
    \fi:
    \if_dim:w \alignsep@<\minalignsep\relax
      \global\alignsep@\minalignsep\relax
      \if_dim:w \eqnshift@>\z@
        \if@fleqn\else:
          \global\eqnshift@\displaywidth
          \global\advance\eqnshift@-\totwidth@
          \global\advance\eqnshift@-\@tempcntb\alignsep@
          \global\divide\eqnshift@\tw@
        \fi:
      \fi:
    \fi:
    \if_dim:w \eqnshift@<\z@
      \global\eqnshift@\z@
    \fi:
    \calc@shift@align
    \global\tagshift@\totwidth@
    \global\advance\tagshift@\@tempcntb\alignsep@
    \if@fleqn
      \if_num:w \xatlevel@=\tw@
        \global\advance\tagshift@-\@mathmargin\relax
      \fi:
    \else:
      \global\advance\tagshift@\eqnshift@
    \fi:
    \iftagsleft@ \else:
      \global\advance\tagshift@-\displaywidth
    \fi:
    \dimen@\minalignsep\relax
    \global\advance\totwidth@\@tempcntb\dimen@
    \if_dim:w \totwidth@>\displaywidth
      \global\MH_let:NwN\displaywidth@\totwidth@
    \else:
      \global\MH_let:NwN\displaywidth@\displaywidth
    \fi:
  \endgroup
}
%    \end{macrocode}
%
%  \subsection{Fixing \env{multline}}
%  \begin{macro}{\EQ_multline_labelhack:n}
%  \begin{macro}{\EQ_multline_labelhack_opt:nw}
%  As \pkg{ntheorem} defines \cs{label} to have the syntax
%  \cs{label}\marg{label}\oarg{name}, we need to make sure that it
%  has that inside \env{multline} as well. Especially since the
%  measuring phase of \env{multline} \emph{defines} \cs{label} as
%  well.
%    \begin{macrocode}
\def\EQ_multline_labelhack:n #1{%
  \begingroup\measuring@false\label@in@display{#1}\endgroup}
\def\EQ_multline_labelhack_opt:nw #1[#2]{%
  \begingroup\measuring@false\label@in@display{#1}[#2]\endgroup}
%    \end{macrocode}
%  \end{macro}
%  \end{macro}
%  \begin{macro}{\multline@}
%  Patching \env{multline} takes more work because it does its
%  processing differently, so we need to apply hooks twice: One in
%  the measuring phase \cs{mmeasure@} where \cs{label} is redefined
%  and one in the production phase \cs{multline@} where the ordinary
%  \cs{label} is gobbled (we use \cs{MT_gobblelabel:w} as
%  before). First the typesetting phase.
%    \begin{macrocode}
\def\multline@#1{%
  \Let@
  \@display@init{\global\advance\row@\@ne \global\dspbrk@lvl\m@ne}%
  \chardef\dspbrk@context\z@
  \restore@math@cr
  \MH_let:NwN\tag\tag@in@align
%    \end{macrocode}
%  The patch for \cs{raisetag}.
%    \begin{macrocode}
  \global\tag@false \global\MH_let:NwN\raise@tag\default_raisetag:
  \mmeasure@{#1}%
  \MH_let:NwN\tag\gobble@tag
%    \end{macrocode}
%  The patch.
%    \begin{macrocode}
  \MH_let:NwN\label\MT_gobblelabel:w
  \tabskip \if@fleqn \@mathmargin \else: \z@skip \fi:
  \totwidth@\displaywidth
  \if@fleqn
    \advance\totwidth@-\@mathmargin
  \fi:
  \halign\bgroup
    \hbox to\totwidth@{%
      \if@fleqn
        \hskip \@centering \relax
      \else:
        \hfil
      \fi:
      \strut@
      $\m@th\displaystyle{}##\endmultline@math%$
      \hfil
    }%
    \crcr
    \if@fleqn
      \hskip-\@mathmargin
      \def\multline@indent{\hskip\@mathmargin}% put it back
    \else:
      \hfilneg
      \def\multline@indent{\hskip\multlinegap}%
    \fi:
    \iftagsleft@
      \iftag@
        \begingroup
          \ifshifttag@
            \rlap{\vbox{%
              \normalbaselines
              \hbox{%
                \strut@
                \make@display@tag
              }%
              \vbox to\lineht@{}%
              \raise@tag
            }}%
              \multline@indent
          \else:
            \setbox\z@\hbox{\make@display@tag}%
            \dimen@\@mathmargin \advance\dimen@-\wd\z@
            \if_dim:w \dimen@<\multlinetaggap
              \dimen@\multlinetaggap
            \fi:
            \box\z@ \hskip\dimen@\relax
          \fi:
        \endgroup
      \else:
        \multline@indent
      \fi:
    \else:
      \multline@indent
    \fi:
  #1%
}
%    \end{macrocode}
%  \end{macro}
%  \begin{macro}{\mmeasure@}
%  Then redefine the label correctly.
%    \begin{macrocode}
\def\mmeasure@#1{%
  \begingroup
    \measuring@true
%    \end{macrocode}
%  The patch.
%    \begin{macrocode}
    \def\label##1{%
      \@ifnextchar[
        {\EQ_multline_labelhack_opt:nw {##1}}
        {\EQ_multline_labelhack:n {##1}}
    }
    \def\math@cr@@@{\cr}%
    \MH_let:NwN\shoveleft\@iden \MH_let:NwN\shoveright\@iden
    \savecounters@
    \global\row@\z@
    \setbox\@ne\vbox{%
      \global\MH_let:NwN\df@tag\@empty
      \halign{%
        \setboxz@h{\@lign$\m@th\displaystyle{}##$}%
        \iftagsleft@
          \if_num:w \row@=\@ne
            \global\totwidth@\wdz@
            \global\lineht@\ht\z@
          \fi:
        \else:
          \global\totwidth@\wdz@
          \global\lineht@\dp\z@
        \fi:
        \crcr
        #1%
        \crcr
      }%
    }%
    \if_meaning:NN
      \df@tag\@empty
    \else:
      \global\tag@true
    \fi:
    \if@eqnsw\global\tag@true\fi:
    \iftag@
      \setboxz@h{%
        \if@eqnsw
          \stepcounter{equation}%
          \tagform@\theequation
        \else:
          \df@tag
        \fi:
      }%
      \global\tagwidth@\wdz@
      \dimen@\totwidth@
      \advance\dimen@\tagwidth@
      \advance\dimen@\multlinetaggap
      \iftagsleft@\else:
        \if@fleqn
          \advance\dimen@\@mathmargin
        \fi:
      \fi:
      \if_dim:w \dimen@>\displaywidth
        \global\shifttag@true
      \else:
        \global\shifttag@false
      \fi:
    \fi:
    \restorecounters@
  \endgroup
}
%    \end{macrocode}
%  \end{macro}
%
%  \subsection{Setting End Marks Correctly}
%
%  \begin{macro}{\mintagvsep}
%  \changes{v1.05}{2004/02/16}{Introduced \cs{mintagvsep} to
%  support the \pkg{ntheorem} package}
%  \begin{macro}{\EQ_only_endmark:}
%  \changes{v1.05}{2004/02/16}{Introduced \cs{EQ_only_endmark:}
%  to support the \pkg{ntheorem} package}
%  \changes{v2.11}{2004/08/03}{Make sure the tag is printed even with the
%  \pkg{mathtools} key \key{showonlyrefs} activated}
%  \begin{macro}{\EQ_tag_plus_endmark:}
%  \changes{v1.05}{2004/02/16}{Introduced
%  \cs{EQ_tag_plus_endmark:} to support the \pkg{ntheorem} package}
%  \pkg{ntheorem} provides an extension to the theorem environment
%  and also supports automatic end of theorem marks. To do so it must
%  redefine parts of the \pkg{amsmath} code and it eventually ends up
%  in problems when the tags and the marker are all typeset at the
%  right margin.
%
%  There are basically two different possibilities:
%  \begin{center}\hfill
%  \unitlength2.5pt \thicklines
%  \begin{picture}(60,45)(0,25)
%    \put(0,50){\framebox(25,20){Math Height}}
%    \put(0,25){\framebox(25,25){Math Depth}}
%    \put(40,50){\framebox(20,10){Tag height}}
%    \put(40,45){\framebox(20,5){Tag depth}}
%    \put(40,25){\framebox(20,10){Marker}}
%    \put(50,35){\vector(0,1){10}}
%    \put(50,45){\vector(0,-1){10}}
%    \put(50,39){\ \ $\delta>0$}
%    \multiput(26,25)(2,0){7}{\line(1,0){1}}
%  \end{picture} \hfill\vrule\hfill
%  \begin{picture}(60,45)(0,25)
%    \put(0,50){\framebox(25,20){Math Height}}
%    \put(0,40){\framebox(25,10){Math Depth}}
%    \put(40,50){\framebox(20,10){Tag height}}
%    \put(40,45){\framebox(20,5){Tag depth}}
%    \put(40,28){\framebox(20,10){Marker}}
%    \put(50,38){\vector(0,1){7}}
%    \put(50,45){\vector(0,-1){7}}
%    \put(50,40){\ \ $\delta=0$}
%  \end{picture}\hfill\mbox{}
%  \end{center}
%  Let's look closer at this. $h_M$ is the height of the end of
%  theorem marker, which we'll try to place so that its baseline
%  aligns with the bottom of the math display. $d_D$ is the depth of
%  the the last line of the math display including the depth of the
%  surrounding boxes while $d_T$ is the depth of the tag. Also we let
%  $m$ denote the minimum vertical distance between tag and marker
%  and $\delta$ is some positive length that may be added to $m$
%  depending on the other values. To ensure a ``proper'' placement of
%  the marker we simply test the inequality
%  \begin{gather}
%     h_M + d_T + m > d_D \label{eq:tag_placement}
%  \end{gather}
%  If the test is true we're done and just have to set the marker so
%  that the top of it is placed at a distance of $m$ from the bottom
%  of the tag. If it's not we can safely put the marker at the bottom
%  of the display.
%    \begin{macrocode}
\newlength\mintagvsep
\mintagvsep5pt
\def\EQ_only_endmark:{%
  \MH_set_boolean_T:n {show_manual_tags}
  \global\tag@true
  \iftagsleft@
    \gdef\df@tag{%
      \hbox to \displaywidth{%
        \hss
%    \end{macrocode}
%  If the tags are set to the left we just have to lower the end mark
%  by the total depth of the display including its surrounding box.
%  Also we have to take care of \cs{normallineskip}.
%    \begin{macrocode}
        \raisebox{-\normallineskip
                  -\depthof{\EQ_outerbox:{}}
                  -\heightof{\EQ_outerbox:{}}
                  -\dp\EQ_mathdisplay_box
                  -\ht\EQ_mathdisplay_box}
                 [0pt][0pt]{\PotEndMark{\maketag@@@}}}}
  \else:
    \gdef\df@tag{%
      \raisebox{-\dp\EQ_mathdisplay_box
                -\depthof{\EQ_outerbox:{}}}
               {\PotEndMark{\maketag@@@}}}%
  \fi:}
%    \end{macrocode}
%  \changes{v2.00}{2004/04/14}{Fixed another bug in placement of the
%  marker}
%  \changes{v2.10}{2004/07/26}{Fixed horizontal spacing in \env{equation}}
%  In v2.00 I fixed another bug I had overlooked. This is getting
%  embarrassing. The outer box must be taken into account as well.
%    \begin{macrocode}
\def\EQ_tag_plus_endmark:{%
  \iftagsleft@
    \gdef\maketag@@@##1{%
%    \end{macrocode}
%  The following fixes a bug when there is a wide tag in an
%  \env{equation} environment and was reported by Bernard
%  Alfonsi.\footnote{\email{alfonsi@math.u-psud.fr}} The problem
%  arises when \pkg{ntheorem} tries to typeset a tag and an end mark,
%  because the tag is set at the left margin and the end mark at the
%  right margin and \pkg{ntheorem} over-simplifies this and forgets
%  to take the displayed math into account! The right solution is to
%  wrap the tag in a box of the correct size, namely |\widthof{##1}|
%  and then push the end mark to the right margin. Then the displayed
%  math is set correcly.
%    \begin{macrocode}
      \hfuzz\displaywidth
      \makebox[\widthof{##1}][l]{%
        \hbox to \displaywidth{\m@th\normalfont##1\hss
          \raisebox{-\dp\EQ_mathdisplay_box
                    -\depthof{\EQ_outerbox:{}}}
                   {\PotEndMark{\hss}}}}}%
  \else:
    \gdef\maketag@@@##1{%
      \hbox{\m@th\normalfont##1\llap{\hss\PotEndMark{%
        \settoheight{\@tempdima}{\@nameuse{\InTheoType Symbol}}%
        \MH_addtolength:dn \@tempdima
          {\g_EQ_bottomtag_depth_dim+\mintagvsep}
        \setlength{\@tempdimb}
          {\g_EQ_bottomrow_depth_dim+\depthof{\EQ_outerbox:{}}
            +\depthof{\EQ_innerbox:{}}}
        \if_dim:w \@tempdima>\@tempdimb
          \MH_setlength:dn \@tempdima{-\dp\EQ_mathdisplay_box
            -\@tempdima+\g_EQ_bottomrow_depth_dim}
          \addtolength{\@tempdima}{\depthof{\EQ_innerbox:{}}}%
          \def\EQ_next:{\raisebox{\@tempdima}}%
        \else:
%    \end{macrocode}
%  \changes{v1.06}{2004/03/13}{Fixed bug in placement of the marker
%  and re-arranged some code}
%  In v1.06 I fixed a stupid bug regarding placement of the marker.
%  If I had done a proper test I would have discovered it
%  immediately!
%    \begin{macrocode}
          \def\EQ_next:{
            \raisebox{-\dp\EQ_mathdisplay_box
                      -\depthof{\EQ_outerbox:{}}}}%
        \fi:\EQ_next:}}}}%
  \fi:}
%    \end{macrocode}
%  \end{macro}
%  \end{macro}
%  \end{macro}
%
%  \begin{macro}{\MT_nonumber:}
%    \begin{macrocode}
\MH_let:NwN \MT_nonumber: \nonumber
%    \end{macrocode}
%  \end{macro}
%  \begin{macro}{\nonumber}
%  Redefine \cs{nonumber} to reset depth of bottom tag.
%    \begin{macrocode}
\def\g_EQ_latest_nonzerodepth_row_fint{\z@}
\renewcommand*\nonumber{
  \@tempcnta=\row@
  \if_num:w \l_EQ_equationtype_int=\thr@@
    \advance\@tempcnta\m@ne
  \else:
    \if_num:w \l_EQ_equationtype_int=4\relax
      \advance\@tempcnta\m@ne
    \fi:
  \fi:
  \if_num:w \g_EQ_latest_nonzerodepth_row_fint=\@tempcnta
  \else:
    \global\g_EQ_bottomtag_depth_dim=-\baselineskip
  \fi:
  \MT_nonumber:
}
%    \end{macrocode}
%  \end{macro}
%
%  End of \pkg{ntheorem} patches and package.
%    \begin{macrocode}
%</package>
%    \end{macrocode}
% Finale
\endinput

%        (quote the arguments according to the demands of your shell)
%
% Documentation:
%    (a) If empheq.drv is present:
%           latex empheq.drv
%    (b) Without empheq.drv:
%           latex empheq.dtx; ...
%    The class ltxdoc loads the configuration file ltxdoc.cfg
%    if available. Here you can specify further options, e.g.
%    use A4 as paper format:
%       \PassOptionsToClass{a4paper}{article}
%
%    Programm calls to get the documentation (example):
%       pdflatex empheq.dtx
%       makeindex -s gind.ist empheq.idx
%       pdflatex empheq.dtx
%       makeindex -s gind.ist empheq.idx
%       pdflatex empheq.dtx
%
% Installation:
%    TDS:tex/latex/mh/empheq.sty
%    TDS:doc/latex/mh/empheq.pdf
%    TDS:source/latex/mh/empheq.dtx
%
%<*ignore>
\begingroup
  \def\x{LaTeX2e}
\expandafter\endgroup
\ifcase 0\ifx\install y1\fi\expandafter
         \ifx\csname processbatchFile\endcsname\relax\else1\fi
         \ifx\fmtname\x\else 1\fi\relax
\else\csname fi\endcsname
%</ignore>
%<*install>
\input docstrip.tex
\Msg{************************************************************************}
\Msg{* Installation}
\Msg{* Package: empheq 2014/08/04 v2.14 empheq}
\Msg{************************************************************************}

\keepsilent
\askforoverwritefalse

\preamble

This is a generated file.

Copyright (C) 2002-2007 by Morten Hoegholm
Copyright (C) 2012-     by Lars Madsen


This work may be distributed and/or modified under the
conditions of the LaTeX Project Public License, either
version 1.3 of this license or (at your option) any later
version. The latest version of this license is in
   http://www.latex-project.org/lppl.txt
and version 1.3 or later is part of all distributions of
LaTeX version 2005/12/01 or later.

This work has the LPPL maintenance status "maintained".

This Current Maintainer of this work is
Lars Madsen

This work consists of the main source file empheq.dtx
and the derived files
   empheq.sty, empheq.pdf, empheq.ins, empheq.drv.

\endpreamble

\generate{%
  \file{empheq.ins}{\from{empheq.dtx}{install}}%
  \file{empheq.drv}{\from{empheq.dtx}{driver}}%
  \usedir{tex/latex/mh}%
  \file{empheq.sty}{\from{empheq.dtx}{package}}%
}

\obeyspaces
\Msg{************************************************************************}
\Msg{*}
\Msg{* To finish the installation you have to move the following}
\Msg{* file into a directory searched by TeX:}
\Msg{*}
\Msg{*     empheq.sty}
\Msg{*}
\Msg{* To produce the documentation run the file `empheq.drv'}
\Msg{* through LaTeX.}
\Msg{*}
\Msg{* Happy TeXing!}
\Msg{*}
\Msg{************************************************************************}

\endbatchfile
%</install>
%<*ignore>
\fi
%</ignore>
%<*driver>
\NeedsTeXFormat{LaTeX2e}
\ProvidesFile{empheq.drv}%
  [2014/08/04 v2.14 Emphasizing equations]
\documentclass{ltxdoc}
\usepackage{mathpazo}
\usepackage[overload]{empheq}
\usepackage{xcolor,varioref,booktabs,fancybox}
\usepackage[overload]{textcase}

\pagestyle{headings}
\hyphenation{emph-eq}
\makeatletter
\providecommand*\pkg[1]{\textsf{#1}}
\providecommand*\env[1]{\texttt{#1}}
\providecommand*\email[1]{\href{mailto:#1}{\texttt{#1}}}
\providecommand*\mode[1]{\texttt{[#1]}}
\providecommand*\file[1]{\texttt{#1}}
\addtolength\marginparwidth{-30pt}


\newcommand*\AddDescribe[3]{%
  \@namedef{Describe#1}{\leavevmode\@bsphack\begingroup\MakePrivateLetters
    \@nameuse{Describe@#1}}
  \@namedef{Describe@#1}##1{\endgroup
                \marginpar{\raggedleft\@nameuse{PrintDescribe#1}{##1}}%
                \@nameuse{Special#1Index}{##1}\@esphack\ignorespaces}
  \@namedef{Special#1Index}##1{\@bsphack
      \index{##1\actualchar{\protect\ttfamily ##1}
            (#2)\encapchar usage}%
      \index{#3:\levelchar##1\actualchar{\protect\ttfamily ##1}\encapchar
            usage}\@esphack}
  \@namedef{PrintDescribe#1}##1{\strut \MacroFont ##1\ }%
  \@namedef{Print#1Name}##1{\strut \MacroFont ##1\ }%
}
\AddDescribe{Color}{color}{colors}
\AddDescribe{KeyFamily}{key family}{key families}
\newcommand*\DescribeDelimiter[1]{%
  \cs{#1}&\SpecialMkhIndex{empheq#1}\cs{empheq}\texttt{#1} &
  \SpecialMkhIndex{empheqbig#1}\cs{empheqbig}\texttt{#1} &
  $\@nameuse{#1}$\tabularnewline}
\newcommand*\SpecialMkhIndex[1]{%
  \@bsphack
    \index{#1\actualchar{\protect\ttfamily \protect\bslash\space#1}
     (` $\bslash#1$ ')\encapchar usage}%
  \@esphack
}
\newcommand\TabDescribeOption[3][(default)]{%
    \SpecialOptIndex{#2}\opt{#2}  & #3 \tabularnewline
}
\newcommand*\TabDescribeAmsKey[2][]{%
    \SpecialKeyIndex{#2}\env{#2} &
    \texttt{\{#2#1\}} &
    \SpecialKeyIndex{#2*}\env{#2*} &
    \texttt{\{#2*#1\}} \tabularnewline
}
\newcommand*\TabDescribeOptKey[3][]{%
    \SpecialKeyIndex{#2}\key{#2} &
    \texttt{#2=}\meta{#3} & #1 \tabularnewline
}

\newenvironment{codesyntax}
    {\par\small\addvspace{4.5ex plus 1ex}%
     \vskip -\parskip
     \noindent
     \begin{tabular}{|l|}\hline\ignorespaces}%
    {\\\hline\end{tabular}\nobreak\par\nobreak
     \vspace{2.3ex}\vskip -\parskip\noindent\ignorespacesafterend}
\makeatletter

\newcommand*\FeatureRequest[2]{%
  \hskip1sp
  \marginpar{%
    \parbox[b]{\marginparwidth}{\small\sffamily\raggedright
      \strut Feature request by\\#1\\#2%
    }
  }%
}

\newcommand*\cttPosting[2]{%
  \hskip1sp
  \marginpar{%
    \parbox[b]{\marginparwidth}{\small\sffamily\raggedright
     \strut Posted on \texttt{comp.text.tex} \\#1\\#2%
    }%
  }%
}

\expandafter\def\expandafter\MakePrivateLetters\expandafter{%
  \MakePrivateLetters  \catcode`\_=11\relax
}

\providecommand*\SpecialOptIndex[1]{%
  \@bsphack
  \index{#1\actualchar{\protect\ttfamily #1}
          (option)\encapchar usage}%
      \index{options:\levelchar#1\actualchar{\protect\ttfamily #1}\encapchar
            usage}\@esphack}
\providecommand*\opt[1]{\texttt{#1}}

\providecommand*\SpecialKeyIndex[1]{%
  \@bsphack
  \index{#1\actualchar{\protect\ttfamily #1}
          (key)\encapchar usage}%
      \index{keys:\levelchar#1\actualchar{\protect\ttfamily #1}\encapchar
            usage}\@esphack}
\providecommand*\key[1]{\textsf{#1}}
\providecommand*\keyfam[1]{\textsf{#1}}

\providecommand*\eTeX{$\m@th\varepsilon$-\TeX}

\def\MTmeta#1{%
     \ensuremath\langle
     \ifmmode \expandafter \nfss@text \fi
     {%
      \meta@font@select
      \edef\meta@hyphen@restore
        {\hyphenchar\the\font\the\hyphenchar\font}%
      \hyphenchar\font\m@ne
      \language\l@nohyphenation
      #1\/%
      \meta@hyphen@restore
     }\ensuremath\rangle
     \endgroup
}
\makeatother
\DeclareRobustCommand\meta{\begingroup\MakePrivateLetters\MTmeta}%
\def\MToarg#1{{\ttfamily[}\meta{#1}{\ttfamily]}\endgroup}
\DeclareRobustCommand\oarg{\begingroup\MakePrivateLetters\MToarg}%
\def\MHmarg#1{{\ttfamily\char`\{}\meta{#1}{\ttfamily\char`\}}\endgroup}
\DeclareRobustCommand\marg{\begingroup\MakePrivateLetters\MHmarg}%
\def\MHarg#1{{\ttfamily\char`\{#1\ttfamily\char`\}}\endgroup}
\DeclareRobustCommand\arg{\begingroup\MakePrivateLetters\MHarg}%
\def\MHcs#1{\texttt{\char`\\#1}\endgroup}
\DeclareRobustCommand\cs{\begingroup\MakePrivateLetters\MHcs}
\def\MHbool#1{{\ttfamily #1}\endgroup}
\DeclareRobustCommand\bool{\begingroup\MakePrivateLetters\MHbool}%

\def\endverbatim{\if@newlist
\leavevmode\fi\endtrivlist\vspace{-\baselineskip}}
\expandafter\let\csname endverbatim*\endcsname =\endverbatim

\let\MTtheindex\theindex
\def\theindex{\MTtheindex\MakePrivateLetters}

\newcommand*\widefbox[1]{\fbox{\hspace{1em}#1\hspace{1em}}}
\definecolor{myblue}{rgb}{.8, .8, 1}
\newcommand*\mybluebox[1]{%
  \colorbox{myblue}{\hspace{1em}#1\hspace{1em}}}
\newenvironment{important}[2][]{%
  \setkeys{EmphEqEnv}{#2}%
  \setkeys{EmphEqOpt}{box=\mybluebox,#1}%
  \EmphEqMainEnv}%
  {\endEmphEqMainEnv}
\definecolor{shadecolor}{cmyk}{0,0,0.41,0}
\definecolor{light-blue}{cmyk}{0.25,0,0,0}
\newsavebox{\mysaveboxM} % M for math
\newsavebox{\mysaveboxT} % T for text
\newcommand*\Garybox[2][Example]{%
  \sbox{\mysaveboxM}{#2}%
  \sbox{\mysaveboxT}{\fcolorbox{black}{light-blue}{#1}}%
  \sbox{\mysaveboxM}{%
    \parbox[b][\ht\mysaveboxM+.5\ht\mysaveboxT+.5\dp\mysaveboxT][b]{%
      \wd\mysaveboxM}{#2}%
  }%
  \sbox{\mysaveboxM}{%
    \fcolorbox{black}{shadecolor}{%
      \makebox[\linewidth-10em]{\usebox{\mysaveboxM}}%
    }%
  }%
  \usebox{\mysaveboxM}%
  \makebox[0pt][r]{%
    \makebox[\wd\mysaveboxM][c]{%
      \raisebox{\ht\mysaveboxM-0.5\ht\mysaveboxT
                +0.5\dp\mysaveboxT-0.5\fboxrule}{\usebox{\mysaveboxT}}%
    }%
  }%
}

\newcommand\EmphEqpart{%
  \def\partname{\EmphEqpartname}%
  \part}

\def\EmphEqpartname{Part}
\makeatother

\usepackage{graphicx}
\usepackage[final,breaklinks=true,hyperindex=false]{hyperref}
\renewcommand*\usage[1]{\textit{\hyperpage{#1}}}

\OnlyDescription
\begin{document}
\DocInput{empheq.dtx}
\end{document}
%</driver>
%  \fi
%  \changes{v2.14}{2014/07/16}{Updated maintenance and contact info}
%
%  \changes{v2.11}{2004/08/03}{Removed interface to the obsolete empheq
%  package \pkg{empheq07}}
%
%  \changes{v2.10}{2004/07/26}{Rewrote most of the documentation}
%
%  \changes{v2.10}{2004/07/26}{Moved options \texttt{fixamsmath} and
%  \texttt{donotfixamsmathbugs} to \pkg{mathtools}}
%
%  \changes{v2.10}{2004/07/26}{Added support for \pkg{showkeys}}
%
%  \changes{v2.00}{2004/04/14}{Added options \opt{overload},
%  \opt{overload2}, \texttt{fixamsmath}, and
%  \texttt{donotfixamsmathbugs} plus descriptions of their use}
%
%  \changes{v1.06}{2004/03/13}{Upgraded license to LPPL version 1.3}
%
%  \changes{v1.00}{2004/02/15}{Complete rewrite of the package hence
%  no description of the changes made}
%
%  \changes{v0.7a}{2004/02/15}{Made the old package obsolete and
%  changed its name to \pkg{empheq-obsolete}}
%
%  \changes{v0.7}{2003/05/11}{Added further examples and changed
%  existing ones}
%  \changes{v0.6b}{2003/03/04}{Added example for changing margins
%  inside the boxtype}
%  \changes{v0.6a}{2003/01/28}{Minor updates to documentation}
%
%  \changes{v0.5}{2002/12/30}{First experimental and very limited
%  release}
%
%  \GetFileInfo{empheq.drv}
%  \CheckSum{2495}
%
%  \title{The \pkg{empheq} package\thanks{This file has version number
%  \fileversion, last revised \filedate.}\\
%  Emphasizing equations in \LaTeXe}
%  \author{Morten H\o gholm, Lars Madsen}
%  
%  \date{\filedate}
%
%  \maketitle
%
%  \begin{abstract}
%  The \pkg{empheq} package can best be described as a visual markup
%  extension to \pkg{amsmath}. In short it offers a)~a multi line
%  equivalent of \cs{boxed} from \pkg{amsmath} and b)~a way to
%  produce arbitrary delimiters that span entire math displays.
%  \end{abstract}
%
%  \tableofcontents
%
%  \clearpage
%  \EmphEqpart{Basic user's guide}
%
%  There can be little doubt that the de facto standard for
%  mathematical typesetting in \LaTeX{} is the \pkg{amsmath} package.
%  For the creation of \pkg{empheq}, a visual markup package for use
%  in math, it made perfect sense to have \pkg{amsmath} as the
%  backbone.
%
%  The main idea of \pkg{empheq} is to maintain the familiar syntax
%  of the \pkg{amsmath} environments while still providing an easy
%  way of specifying markup instructions. This manual is plastered
%  with examples showing just how.
%
%
%  \section{Basic use of the package}
%
%  So what is it \pkg{empheq} does? Well, it allows you to produce
%  displays like this:
%  \begin{subequations}
%  \begin{empheq}[
%    box=\mybluebox,
%    right={\;\makebox[.9em]{$\raisebox{-.5\totalheight+\fontdimen22\textfont2}
%              {\resizebox{!}{\EmphEqdisplayheight+\EmphEqdisplaydepth}{!}}$}},
%    left={X=Y\Rightarrow\empheqlbrace}
%     ]{alignat=3}
%    A_1&=b_1 & \qquad c_1&=d_1 & \qquad e_1&= f_1
%         \tag*{\_A\raisebox{1ex}{silly}\raisebox{-1ex}{tag}\_}\\
%    A_2&=b_2 & \qquad c_2&=d_2 & \qquad  e_2&= f_2 \\
%    A_3&=b_3 & \qquad c_3&=d_3 & \qquad  e_3&= f_3
%  \end{empheq}
%  \end{subequations}
%  In short \pkg{empheq} enables the user to put things on every side
%  of the display without said user having to worry about what
%  happens to the equation numbers. For example you can now have a
%  display containing multiple lines and still get the effect of the
%  \cs{boxed} command from \pkg{amsmath}.
%
%
%  \subsection{Using the \env{empheq} environment}
%
%  The package defines a single environment
%  \SpecialEnvIndex{empheq}\env{empheq} and the usage is kind of straight
%  forward:
%  \begin{codesyntax}
%    |\begin{empheq}|\oarg{markup instructions}\marg{\AmS{} env_name}\\
%    \mbox{}\quad\meta{contents of \AmS{} environment}\\
%    |\end{empheq}|
%  \end{codesyntax}
%  A first minimal example file would then be something like
%  \begin{verbatim}
%    \documentclass{minimal}
%    \usepackage{empheq}
%    \begin{document}
%    \begin{empheq}{align*}
%      a&=b \tag{*}\\
%      E&=mc^2 + \int_a^a x\, dx
%    \end{empheq}
%    \end{document}
%  \end{verbatim}
%  This gives the following display:
%    \begin{empheq}{align*}
%      a&=b \tag{*}\\
%      E&=mc^2 + \int_a^a x\, dx
%    \end{empheq}
%  Maybe not the most impressive example, but as you can see the
%  contents of the environment is exactly the same as for the regular
%  \env{align*} evironment from \pkg{amsmath}. The rest of the \AmS{}
%  environments are chosen the same way by typing the name as the
%  mandatory argument of \env{empheq} with the exception of the
%  \env{alignat} environment. For this you have to specify the number
%  of columns as shown below.
%  \begin{verbatim}
%  \begin{empheq}{alignat=2}
%    a &= b &\quad c &= d \\
%    \text{this} &= \text{that} &\quad \mathit{fish}&\neq fish
%  \end{empheq}
%  \end{verbatim}
%  \begin{empheq}{alignat=2}
%    a &= b &\quad c &= d \\
%    \text{this} &= \text{that} &\quad \mathit{fish}&\neq fish
%  \end{empheq}
%  To choose the starred version of \env{alignat}, simply type
%  \arg{alignat*=2} instead in the above example.
%
%  The supported \AmS{} environments are listed in Table
%  \vref{tab:environments}. Not supported is the standard \LaTeX{}
%  \env{eqnarray} environment as it is fundamentally
%  deficient.\footnote{See for instance
%  \url{http://www.tex.ac.uk/cgi-bin/texfaq2html?label=eqnarray}}
%
%  \begin{table}
%  \centering
%  \begin{tabular}{llll}\toprule
%    Environment & Usage & Environment & Usage \\ \midrule
%    \TabDescribeAmsKey{equation}
%    \TabDescribeAmsKey{align}
%    \TabDescribeAmsKey{gather}
%    \TabDescribeAmsKey{flalign}
%    \TabDescribeAmsKey[=\textrm{\meta{cols}}]{alignat}
%    \TabDescribeAmsKey{multline} \bottomrule
%  \end{tabular}
%  \caption{The supported \pkg{amsmath}
%  environments}\label{tab:environments}
%  \end{table}
%
%
%  \subsection{Markup instructions}
%
%  The optional argument of the \env{empheq} environment will take
%  markup instructions consisting of a $\meta{key}=\meta{value}$ list
%  of assignments. There are currently five such keys (a sixth is
%  added for naming consistency). They're shown in Table
%  \vref{tab:keys}.
%
%  \begin{table}
%  \centering
%  \begin{tabular}{llp{13em}}\toprule
%    Key & Usage & Additional Info \tabularnewline \midrule
%    \TabDescribeOptKey{box}{box command}
%    \TabDescribeOptKey{innerbox}{box command}
%    \TabDescribeOptKey[Use \cs{text}\arg{...} if you need text
%                       material.]{left}{math material}
%    \TabDescribeOptKey[Use \cs{text}\arg{...} if you need text
%                       material.]{right}{math material}
%    \TabDescribeOptKey[Alias for \key{box}.]{outerbox}{box command}
%    \TabDescribeOptKey[Contents must be horizontally centered;
%                       can only be used in \mode{fleqn} mode.]
%                       {marginbox}{box command}
%    \bottomrule
%  \end{tabular}
%  \caption{The six keys for the optional argument of the
%  \env{empheq} environment}\label{tab:keys}
%  \end{table}
%
%  \begin{codesyntax}
%    \SpecialKeyIndex{left}\key{left}
%  \end{codesyntax}
%  The key \key{left} is for material put on the left side of the
%  display. The material is typeset in math mode and centered
%  vertically.
%  \begin{verbatim}
%  \begin{empheq}[left=L\Rightarrow]{align}
%    a&=b\\
%    E&=mc^2 + \int_a^a x\, dx
%  \end{empheq}
%  \end{verbatim}
%  \begin{empheq}[left=L\Rightarrow]{align}
%    a&=b\\
%    E&=mc^2 + \int_a^a x\, dx
%  \end{empheq}
%
%
%  \begin{codesyntax}
%    \SpecialKeyIndex{right}\key{right}
%  \end{codesyntax}
%  As there is a \key{left} key it hopefully comes as no surprise
%  that there is a companion \key{right} key for typesetting material
%  on the right side of the display.
%  \begin{verbatim}
%  \begin{empheq}[right=\Leftarrow R]{align}
%    a&=b\\
%    E&=mc^2 + \int_a^a x\, dx
%  \end{empheq}
%  \end{verbatim}
%  \begin{empheq}[right=\Leftarrow R]{align}
%    a&=b\\
%    E&=mc^2 + \int_a^a x\, dx
%  \end{empheq}
%
%  \begin{codesyntax}
%    \SpecialKeyIndex{box}\key{box}\\
%    \SpecialKeyIndex{outerbox}\key{outerbox}
%  \end{codesyntax}
%  The key \key{box} specifies the kind of box you would like to put
%  around the display. It can be any kind of box, as long as the
%  contents of the box is situated on the baseline like in a
%  \fbox{\cs{fbox}}.
%  \begin{verbatim}
%  \begin{empheq}[box=\fbox]{align}
%    a&=b\\
%    E&=mc^2 + \int_a^a x\, dx
%  \end{empheq}
%  \end{verbatim}
%  \begin{empheq}[box=\fbox]{align}
%    a&=b\\
%    E&=mc^2 + \int_a^a x\, dx
%  \end{empheq}
%  The key \key{outerbox} is an alias for \key{box} and is added for
%  naming consistency with the key described below.
%
%  \begin{codesyntax}
%    \SpecialKeyIndex{innerbox}\key{innerbox}
%  \end{codesyntax}
%  There is also an \key{innerbox} key. It is not very interesting
%  unless you use one of the other keys.
%  \begin{verbatim}
%  \begin{empheq}[innerbox=\fbox,
%                 left=L\Rightarrow]{align}
%    a&=b\\
%    E&=mc^2 + \int_a^a x\, dx
%  \end{empheq}
%  \end{verbatim}
%  \begin{empheq}[innerbox=\fbox,
%                 left=L\Rightarrow]{align}
%    a&=b\\
%    E&=mc^2 + \int_a^a x\, dx
%  \end{empheq}
%
%  \begin{codesyntax}
%    \SpecialKeyIndex{marginbox}\key{marginbox}
%  \end{codesyntax}
%  \FeatureRequest{Uwe Siart}{2003/12/08}
%  The last key is the \key{marginbox} key. If you typeset your math
%  in \mode{fleqn} mode\footnote{Sorry, but I can't show you because
%  this manual features centered math displays.} you may want the
%  math display and not the outer box to align at the left margin (or
%  rather: the indentation). If you make sure the contents of the
%  outer box is centered horizontally inside the box, \key{marginbox}
%  will align it properly. You shouldn't set both \key{box} and
%  \key{marginbox} at the same time, as this is surely not what you
%  want and the package will silently use the last one in the list.
%
%
%  \begin{quote}
%  \textbf{Warning}: \pkg{keyval} treats commas and equal signs as
%  separator and assignment signs which in turn means that if you
%  want to typeset either of them, they \emph{must}---and believe me,
%  it's really important---be enclosed in braces. In short:
%  \begin{description}
%    \item[{\makebox[3em][l]{Good}}] \texttt{[left=\{(A,B)=(1,0)\}]}
%    \item[{\makebox[3em][l]{Bad}}]  \texttt{[left=(A,B)=(1,0)]}
%  \end{description}
%  \end{quote}
%
%  \begin{codesyntax}
%    \SpecialUsageIndex{\empheqset}\cs{empheqset}\marg{markup instructions}
%  \end{codesyntax}
%  \FeatureRequest{Uwe Siart}{2004/01/13}
%  You can also set the keys globally\footnote{Or rather: outside the
%  scope of the \env{empheq} environment. The settings will still
%  obey scoping rules.} with the command
%  \cs{empheqset}. This means that
%  \begin{verbatim}
%    \empheqset{marginbox=\psframebox}
%  \end{verbatim}
%  will force a |box=\psframebox| in all occurrences of the
%  \env{empheq} environment, but an explicitly given \key{box} or
%  \key{marginbox} will override this setting. You can only use the
%  keys listed in this section as arguments to \cs{empheqset}; not
%  the \AmS\ environments.
%
%
%
%  \subsection{What won't work in the \env{empheq} environment}
%
%  \begin{codesyntax}
%    \SpecialUsageIndex{intertext}\cs{intertext}\marg{text}\\
%    \SpecialUsageIndex{displaybreak}\cs{displaybreak}\oarg{num}
%  \end{codesyntax}
%  Now we've seen some of the things that work in the \env{empheq}
%  environment but we also have to take note of what won't work. As
%  this package provides a way to box multi line math displays it can
%  come as no surprise that using either \cs{intertext} or
%  \cs{displaybreak} inside the \env{empheq} environment makes no
%  sense. Should you however happen to try them anyway, you'll
%  experience that \cs{intertext} issues an error message and
%  \cs{displaybreak} issues a warning. The reason only \cs{intertext}
%  gives an error message is that you'll get output very different
%  from what you expect, and that is not the case (so much at least)
%  with \cs{displaybreak}. But of course, you'll never see either
%  message because you read the manual!
%
%
%
%  \subsection{Special delimiters}
%
%  As you've seen a few pages back, it's possible to add material on
%  both sides of the math display. When doing so you often need a
%  delimiter that scales to fit the entire display, so it comes as no
%  surprise that \pkg{empheq} provides such delimiters.
%  \begin{codesyntax}
%    \cs{empheql}\meta{delim_name}  \cs{empheqbigl}\meta{delim_name}\\
%    \cs{empheqr}\meta{delim_name}  \cs{empheqbigr}\meta{delim_name}
%  \end{codesyntax}
%  For example one might need a large brace like the one in the
%  \env{cases} environment:
%  \begin{verbatim}
%  \begin{empheq}[left=\empheqlbrace, right=\empheqrbrace]{align}
%    E&=mc^2 \\
%    Y&= \sum_{n=1}^\infty \frac{1}{n^2}
%  \end{empheq}
%  \end{verbatim}
%  \begin{empheq}[left=\empheqlbrace, right=\empheqrbrace]{align}
%    E&=mc^2 \\
%    Y&= \sum_{n=1}^\infty \frac{1}{n^2}
%  \end{empheq}
%  The naming scheme is \cs{empheq}\meta{delimiter name} and
%  \cs{empheqbig}\meta{delimiter name}. Thus \cs{empheqrbrace}
%  produces an auto-scaling right brace with same size as the math
%  display while \cs{empheqbigrbrace} produces an even bigger version
%  that spans the inner box plus the math display inside it. For a
%  complete list of supported delimiters see Table
%  \vref{tab:delimiters}.
%
%  \begin{table}
%  \centering
%  \begin{tabular}{lllc} \toprule
%  \multicolumn{3}{c}{\textbf{Delimiter}}\\ \cmidrule{1-3}
%  \multicolumn{1}{c}{Original} & \multicolumn{1}{c}{Normal}  &
%    \multicolumn{1}{c}{Bigger} & Symbol\\ \midrule
%  \DescribeDelimiter{lbrace} \DescribeDelimiter{rbrace}
%  \DescribeDelimiter{lbrack} \DescribeDelimiter{rbrack} \midrule
%  \DescribeDelimiter{langle} \DescribeDelimiter{rangle}
%  \DescribeDelimiter{lparen} \DescribeDelimiter{rparen} \midrule
%  \DescribeDelimiter{lvert} \DescribeDelimiter{rvert}
%  \DescribeDelimiter{lVert} \DescribeDelimiter{rVert} \midrule
%  \DescribeDelimiter{lfloor} \DescribeDelimiter{rfloor}
%  \DescribeDelimiter{lceil} \DescribeDelimiter{rceil} \bottomrule
%  \end{tabular}
%  \caption{The supported auto-scaling delimiters in
%  \pkg{empheq}}\label{tab:delimiters}
%  \end{table}
%
%
%
%  \section{Taking things a little further}
%
%  So far we have covered the basic functions of \pkg{empheq}: the
%  markup instructions and the predefined auto-scaling delimiters.
%  There is more to \pkg{empheq} however, and we'll cover that in
%  this section.
%
%  \subsection{Loading the package}
%
%  The package has two main requirements: One is the
%  \pkg{mathtools}\footnote{By the same author and is distributed
%  with \pkg{empheq}. See its documentation for more information.}
%  package which provides \pkg{empheq} with a lot of necessary tools
%  for doing its thing. The other one is \pkg{amsmath} of course, and
%  to make the loading procedure as easy as possible, you can simply
%  substitute
%  \begin{verbatim}
%  \usepackage[leqno,fleqn,intlimits]{amsmath}
%  \end{verbatim}
%  with
%  \begin{verbatim}
%  \usepackage[leqno,fleqn,intlimits]{empheq}
%  \end{verbatim}
%  \pkg{empheq} makes sure that the \pkg{amsmath} options are passed
%  on and loaded by \pkg{amsmath}. The same goes for the options
%  provided by \pkg{mathtools}.
%
%
%  \subsubsection{Compatibility with the old version}
%
%  The current version (\fileversion) of \pkg{empheq} is incompatible with
%  versions prior to and including~0.7e. If you have documents produced
%  with versions prior to~0.7e, then you have to load the package
%  \pkg{empheq07} instead. \pkg{empheq07} now exists as a separate package
%  package with its own documentation, but I strongly recommend switching
%  to the new version because it is so much better.
%
%  \subsection{Package options}
%
%  In addition to the \pkg{amsmath} options, \pkg{empheq} itself
%  provides a string of options listed in Table \vref{tab:options}.
%
%  \begin{table}[ht]
%    \centering
%    \begin{tabular}{lp{21em}} \toprule
%      Option  & Short description \tabularnewline \midrule
%      \TabDescribeOption[]{overload}
%      {Lets you use the \pkg{empheq} visual markup extensions
%      transparently in an existing document.}
%      \TabDescribeOption[]{overload2}
%      {A wild version of \opt{overload}. Use with care.}
%      \TabDescribeOption[]{ntheorem}
%      {A support module for users of \pkg{ntheorem} and its
%      \opt{thmmarks} and \opt{thref} options.}
%      \TabDescribeOption[]{newmultline}
%      {With v2.10 of \pkg{empheq} the syntax for \env{multline} and
%      \env{multlined} has been improved immensely.}
%      \TabDescribeOption[]{oldmultline}
%      {Support for the somewhat strange syntax for \env{multline} and
%      \env{multlined} in v2.00.}
%      \bottomrule
%    \end{tabular}
%    \caption{Supported options in the \pkg{empheq} package}
%    \label{tab:options}
%  \end{table}
%
%
%  \begin{codesyntax}
%    \SpecialOptIndex{ntheorem}\opt{ntheorem}
%  \end{codesyntax}
%  The \pkg{ntheorem} package is supported by means of the
%  \opt{ntheorem} option. In order for this to work properly, an
%  \pkg{amsmath} bug\footnote{See the \LaTeX{} Bugs Database
%  \url{http://www.latex-project.org/cgi-bin/ltxbugs2html} under
%  \AmS\LaTeX{} problem report 3624.} is fixed. For more information
%  on this option see \S \vref{sec:ntheorem-support}.
%
%  \subsubsection{Using \env{multline}}
%
%  \begin{codesyntax}
%    \SpecialOptIndex{newmultline}\opt{newmultline}\\
%    \SpecialOptIndex{oldmultline}\opt{oldmultline}
%  \end{codesyntax}
%  Those familiar with v2.00 of this package will recognize the
%  somewhat weird syntax for using \env{multline} and
%  \env{multlined}. The good news is that since v2.10 there is now a
%  much improved syntax available. The option \opt{newmultline}
%  (default) selects this better interface, while the old interface
%  is provided for compatibility reasons by the \opt{oldmultline}
%  option.
%
%
%  \begin{codesyntax}
%    \SpecialUsageIndex{\shoveleft}\cs{shoveleft}\oarg{dimen}\marg{arg}\\
%    \SpecialUsageIndex{\shoveright}\cs{shoveright}\oarg{dimen}\marg{arg}
%  \end{codesyntax}
%  With the new interface you also get an extended syntax for
%  \cs{shoveleft} and \cs{shoveright} as shown in the example below.
%  \begin{verbatim}
%  \begin{empheq}{multline}
%    \framebox[.65\columnwidth]{First line} \\
%    \framebox[.5\columnwidth]{Second line} \\
%    \shoveleft{L+E+F+T}         \\
%    \shoveright{R+I+G+H+T}         \\
%    \shoveleft[1cm]{L+E+F+T}         \\
%    \shoveright[\widthof{$R+I+G+H+T$}]{R+I+G+H+T}         \\
%    \framebox[.65\columnwidth]{Last line}
%  \end{empheq}
%  \end{verbatim}
%  \begin{empheq}{multline}
%    \framebox[.65\columnwidth]{First line} \\
%    \framebox[.5\columnwidth]{Second line} \\
%    \shoveleft{L+E+F+T}         \\
%    \shoveright{R+I+G+H+T}         \\
%    \shoveleft[1cm]{L+E+F+T}         \\
%    \shoveright[\widthof{$R+I+G+H+T$}]{R+I+G+H+T}         \\
%    \framebox[.65\columnwidth]{Last line}
%  \end{empheq}
%
%
%  There are however a few differences in the output between the
%  original \pkg{amsmath} version of \env{multline} and the one
%  \pkg{empheq} provides. In \pkg{amsmath} a centered line in
%  \env{multline} is centered on the page without taking into account
%  the \cs{multlinegap}, \cs{multlinetaggap}, and the tag width. Thus
%  \pkg{amsmath} can sometimes give you horrible output without
%  giving you any warning as shown below.
%  \begin{verbatim}
%  \begin{multline}
%    \framebox[.65\columnwidth]{First line} \\
%    \framebox[.9\columnwidth]{Loooong line} \\
%    \framebox[.65\columnwidth]{Last line} \tag{wide tag}
%  \end{multline}
%  \end{verbatim}
%  \begin{AmSmultline}
%    \framebox[.65\columnwidth]{First line} \\
%    \framebox[.9\columnwidth]{Loooong line} \\
%    \framebox[.65\columnwidth]{Last line} \tag{wide tag}
%  \end{AmSmultline}
%  In \pkg{empheq} these parameters are taken into account, so the
%  same input will produce this in \pkg{empheq}:
%  \begin{verbatim}
%  \begin{empheq}{multline}
%    \framebox[.65\columnwidth]{First line} \\
%    \framebox[.9\columnwidth]{Loooong line} \\
%    \framebox[.65\columnwidth]{Last line} \tag{wide tag}
%  \end{empheq}
%  \end{verbatim}
%  \begin{multline*}
%    \framebox[.65\columnwidth]{First line} \\
%    \framebox[.9\columnwidth]{Loooong line} \\
%    \framebox[.65\columnwidth]{Last line} \tag{wide tag}
%  \end{multline*}
%  This results in an
%  \begin{verbatim}
%    Overfull \hbox (30.03783pt too wide) in paragraph ...
%  \end{verbatim}
%  message in the log file, indicating a visual problem. I think this
%  behavior is more sensible than the one the original \env{multline}
%  environment provides.
%
%  \begin{codesyntax}
%    \SpecialEnvIndex{MTmultlined}\cs{begin}\arg{MTmultlined}\oarg{pos}\oarg{width}
%        \meta{contents} \cs{end}\arg{MTmultlined}
%  \end{codesyntax}
%  When you choose the \opt{oldmultline} option you may still want to
%  use the much improved \env{multlined} environment defined in
%  \pkg{mathtools} but unfortunately there will be a name clash.
%  Instead you can access it under the name \env{MTmultlined}.
%
%
%
%  \subsubsection{The \opt{overload} option}\label{sec:overload}
%
%  \begin{codesyntax}
%    \SpecialOptIndex{overload}\opt{overload}\\
%    \SpecialOptIndex{overload2}\opt{overload2}
%  \end{codesyntax}
%  \FeatureRequest{Lars Madsen}{2004/03/25}
%  The \opt{overload} option redefines the original \AmS{}
%  environments so that they take an optional argument.
%  \begin{codesyntax}
%    \cs{begin}\marg{\AmS{} env_name}\oarg{markup instructions}\\
%    \meta{contents of \AmS{} environment}\\
%    \cs{end}\marg{\AmS{} env_name}
%  \end{codesyntax}
%  For example
%  \begin{verbatim}
%    \begin{gather}[box=\widefbox]
%      a = b
%    \end{gather}
%  \end{verbatim}
%  is then actually short for
%  \begin{verbatim}
%    \begin{empheq}[box=\widefbox]{gather}
%      a = b
%    \end{empheq}
%  \end{verbatim}
%  All the \AmS{} environments are supported by this option except
%  for the pseudo environment |\[ ... \]| (it's not really an
%  environment), \footnote{With the option \opt{oldmultline} the
%  \env{multline} and \env{multline*} environments aren't supported
%  either because their syntax in \pkg{empheq} then differ from their
%  syntax in \pkg{amsmath}.} because a)~you don't really need markup
%  instructions for a one line display, and b)~would you really like
%  a syntax like
%  \begin{verbatim}
%    \[[box=\fbox] a=b \]
%  \end{verbatim}
%  where its difficult to see whether or not there is a typo?
%  Choosing \opt{overload} is meant for people who don't want to
%  change their entire document into \pkg{empheq} syntax. I have no
%  problems with that; just be careful when you fiddle with
%  \cs{empheqset} as it will affect \emph{all} math displays!
%
%
%  There is of course a catch (well, two actually): These redefined
%  environments don't run as fast as the regular ones (about three
%  times as slow), but in this day and age I seriously doubt you'll
%  be able to tell the difference unless you have a vast number of
%  equations. The other catch is that you cannot use \cs{intertext}
%  and \cs{displaybreak} as described earlier. If you find yourself
%  wanting to use one of these features in say, an \env{align}
%  environment, you have to use the original \env{align} environment.
%  Luckily it is available if you call it like this:
%  \begin{codesyntax}
%    \cs{begin}\arg{AmS\meta{\rmfamily \AmS{} env_name}} \\
%    \texttt{~~}\meta{contents of \AmS{} environment}\\
%    \cs{end}\arg{AmS\meta{\rmfamily \AmS{} env_name}}
%  \end{codesyntax}
%  For example the original \env{align} environment could be selected
%  with
%  \begin{verbatim}
%    \begin{AmSalign}
%      ...
%    \end{AmSalign}
%  \end{verbatim}
%  These original versions with prefix \texttt{AmS} exist for all the
%  environments.
%
%
%  The option \opt{overload2} activates the overloading feature
%  for the pseudo environment |\[ ... \]|, although I doubt you'll
%  find it useful. Beware that this definition is fragile unless you
%  have \eTeX{} as \LaTeX{} engine.\footnote{\eTeX{} has been the
%  default engine for \LaTeX{} in most major distributions since
%  2003.} Not surprisingly \opt{overload2} activates
%  \opt{overload}.
%
%
%
%  Before you get all excited about these options, you should take
%  note of some aspects of centered math displays in \pkg{amsmath}.
%  In certain circumstances truly centering the display is not always
%  the best solution as in this example:
%  \begin{center}
%    \mbox{}\hfill
%    \framebox[.75\linewidth]{Wide math display; not adjusted}%
%    \hfill\mbox{}%
%    \makebox[0pt][r]{Wide Tag}
%  \end{center}
%  Instead centering the display in the available space works pretty
%  well:
%  \begin{gather*}
%    \framebox[.75\linewidth]{Wide math display; adjusted}%
%    \tag*{Wide Tag}
%  \end{gather*}
%  This is what \pkg{amsmath} does normally, but in the \env{gather}
%  environment it does it on a per line basis. This means that we get
%  results like
%  \begin{AmSgather*}
%    \framebox[.75\linewidth]{Wide math display; adjusted}%
%    \tag*{Wide Tag} \\
%    \framebox[.75\linewidth]{Wide math display; not adjusted} \tag{i}%
%  \end{AmSgather*}
%  whether we like it or not. Inside the \env{empheq} environment we
%  need to have the same adjustment for all lines else the boxing
%  process will not work properly, so when activating the
%  \opt{overload} option the above two-line \env{gather} will
%  instead look like this:
%  \begin{empheq}{gather*}
%    \framebox[.75\linewidth]{Wide math display; adjusted}%
%    \tag*{Wide Tag} \\
%    \framebox[.75\linewidth]{Wide math display; adjusted} \tag*{i}%
%  \end{empheq}
%  I leave it to you to choose whether or not this is better (I think
%  it is better).
%
%
%
%  \subsection{A note on boxed displays}
%
%  When browsing a 400+ pages textbook with at least twice the number
%  of displayed formulae, some of them are surely more important than
%  others. Thus the author of the book (in cooperation with the
%  designer) should make sure that such formulae are easily found
%  again i.e., they should be easily distinguishable from the rest of
%  the pack. One way of doing this is putting the formula into a box
%  which is something we've seen \pkg{empheq} being capable of. There
%  are however a few things to keep in mind:
%  \begin{itemize}
%    \item Don't overdo it. If you do it on half of them there's no
%    point in doing it at all. I don't see much reason for applying
%    this technique to more than 10\% of the formulae.
%    \item Choose the type of box carefully. You want to draw
%    attention to it so it might as well look good.
%  \end{itemize}
%  The latter point can be illustrated by defining a macro similar to
%  \cs{fbox} only with a little more space to the right and left of
%  the argument.
%  \begin{verbatim}
%  \newcommand*\widefbox[1]{\fbox{\hspace{1em}#1\hspace{1em}}}
%  \end{verbatim}
%  If we replace the \cs{fbox} from the \key{box} example before, we
%  get this display:
%  \begin{verbatim}
%  \begin{empheq}[box=\widefbox]{align}
%    a&=b\\
%    E&=mc^2 + \int_a^a x\, dx
%  \end{empheq}
%  \end{verbatim}
%  \begin{empheq}[box=\widefbox]{align}
%    a&=b\\
%    E&=mc^2 + \int_a^a x\, dx
%  \end{empheq}
%  Compare it with
%  \begin{empheq}[box=\fbox]{align}
%    a&=b\\
%    E&=mc^2 + \int_a^a x\, dx
%  \end{empheq}
%  and see which one you prefer.
%
%  Similarly one might be tempted to use colored boxes:
%  \begin{verbatim}
%  \definecolor{myblue}{rgb}{.8, .8, 1}
%  \newcommand*\mybluebox[1]{%
%    \colorbox{myblue}{\hspace{1em}#1\hspace{1em}}}
%  \end{verbatim}
%  We know the drill by now.
%  \begin{verbatim}
%  \begin{empheq}[box=\mybluebox]{align}
%    a&=b\\
%    E&=mc^2 + \int_a^a x\, dx
%  \end{empheq}
%  \end{verbatim}
%  \begin{empheq}[box=\mybluebox]{align}
%    a&=b\\
%    E&=mc^2 + \int_a^a x\, dx
%  \end{empheq}
%  There is more on boxes later in this manual. If you're into the
%  \pkg{fancybox} package then remember to read \S
%  \vref{sec:fancybox_support}.
%
%  \section{Support for other packages}
%
%  With the multitude of packages for \LaTeX, it is not always easy
%  to be a PWWO package (``Plays Well With Others''), but
%  \pkg{empheq} tries really hard to do so. This section lists the
%  packages where \pkg{empheq} has to provide workarounds and they
%  can be divided into two categories.
%  \begin{description}
%    \item[Compatibility] Some packages affect \LaTeX' labelling
%    mechanism and since \pkg{empheq} internally has to turn off
%    labels and related code temporarily, hooks must be provided for
%    these packages. Examples of such packages are \pkg{hyperref} and
%    \pkg{showkeys}.
%
%    \item[Enhancements] Other packages provide useful features that
%    for some reason may not work directly or optimally with
%    \pkg{empheq}. In these cases the problematic commands are
%    redefined so that they not only work with \pkg{empheq}, but
%    also give the same if not better output. An example of this
%    would be the \cs{shadowbox} command from \pkg{fancybox}
%    (described below).
%  \end{description}
%  %
%  \subsection{Support for \pkg{fancybox}}\label{sec:fancybox_support}
%
%  The \pkg{fancybox} package provides various boxes and you can use
%  them with \pkg{empheq} as well. Here's \cs{ovalbox}:
%  \begin{verbatim}
%  \begin{empheq}[box=\ovalbox]{align}
%      E&=mc^2 \\
%      Y&= \sum_{n=1}^\infty \frac{1}{n^2}
%  \end{empheq}
%  \end{verbatim}
%  \begin{empheq}[box=\ovalbox]{align}
%      E&=mc^2 \\
%      Y&= \sum_{n=1}^\infty \frac{1}{n^2}
%  \end{empheq}
%
%
%  \begin{codesyntax}
%  \SpecialUsageIndex{\shadowbox}\SpecialUsageIndex{\shadowbox*}
%    \cs{shadowbox*}\marg{arg} \\
%  \SpecialColorIndex{shadowcolor} \texttt{shadowcolor}
%  \end{codesyntax}
%  The only problem with using the \pkg{fancybox} boxes in
%  conjunction with the \env{empheq} environment is \cs{shadowbox},
%  as this macro typesets the shadow on the baseline and for the
%  equation numbers to be placed correctly, the box command must set
%  its argument on the baseline. Thus the normal \cs{shadowbox}
%  produces \shadowbox{\meta{arg}}, but we need this instead:
%  \shadowbox*{\meta{arg}}. Therefore \pkg{empheq} will detect if
%  \pkg{fancybox} is loaded and in that case it'll enhance
%  \cs{shadowbox} in two ways:
%  \begin{itemize}
%    \item It defines a starred version \cs{shadowbox*} which
%    typesets its argument on the baseline.
%
%    \item The color \texttt{shadowcolor} is introduced. The default
%    color is black.
%  \end{itemize}
%  \begin{verbatim}
%  \definecolor{shadowcolor}{rgb}{0,.5,.5}
%  \setlength\shadowsize{2pt}
%  Line of text for \shadowbox*{testing of \Verb|f#&%\| verbatim} and
%  showing the shadow color.
%  \end{verbatim}
%  \begin{quote}
%  \definecolor{shadowcolor}{rgb}{0,.5,.5}
%  \setlength\shadowsize{2pt}
%  Line of text for \shadowbox*{testing of \Verb|f#&%\| verbatim} and
%  showing the shadow color.
%  \end{quote}
%
%  The point is that if you want a \cs{shadowbox} around your math
%  display then you must use the starred version:
%  \begin{verbatim}
%  \begin{empheq}[box=\shadowbox*]{align}
%      E&=mc^2 \\
%      Y&= \sum_{n=1}^\infty \frac{1}{n^2}
%  \end{empheq}
%  \end{verbatim}
%  \begin{empheq}[box=\shadowbox*]{align}
%      E&=mc^2 \\
%      Y&= \sum_{n=1}^\infty \frac{1}{n^2}
%  \end{empheq}
%
%  See \S \vref{stunning} if you want
%  to make a fancy box yourself.
%
%
%  \subsection{Support for \pkg{ntheorem}}\label{sec:ntheorem-support}
%
% \changes{v2.13}{2012/05/09}{Added caveat}
%  \textbf{Caveat:} Due to an unfortunate interaction between \pkg{empheq} and
%  \pkg{ntheorem}, users may want to add
% \begin{verbatim}
% \usetagform{default}
% \end{verbatim}
%  \emph{after} loading \pkg{ntheorem} otherwise the tags may be
%  placed wrong and any labels within the \env{empheq} environment may
%  be lost. In the future \pkg{ntheorem} should be able to test for
%  this and add \verb|\usetagform{default}| automatically.
%
%  Users who use \verb|\usetagform| to get another tag design, should
%  make sure to postpone this configuration until \emph{after} loading
%  \pkg{ntheorem}. 
%
% \bigskip
%
%  \changes{v2.00}{2004/04/14}{Added description of the
%  \opt{ntheorem} option}
%  \changes{v1.05}{2004/02/16}{Added support for the end-of-theorem
%  marker mechanism of \pkg{ntheorem}}
%
%  \begin{codesyntax}
%    \SpecialOptIndex{ntheorem}\opt{ntheorem}
%  \end{codesyntax}
%  The \pkg{ntheorem} package is supported by the means of the
%  \opt{ntheorem} option of \pkg{empheq}. This loads a set of
%  extra macros which fixes various compatibility problems between
%  \pkg{ntheorem} and \pkg{amsmath} and furthermore introduces
%  special (internal) macros for optimum positioning of
%  end-of-theorem markers, while retaining a user friendly interface.
%
%
%  When you want to use the automatic end-of-theorem marker mechanism
%  from \pkg{ntheorem} you sometimes run into problems as you would
%  want the marker to be placed aligned at the bottom of the math
%  display but still keeping the tags in their proper place. In
%  \mode{leqno} mode this is not that much of a problem as the tags
%  and the marker are set on either side of the math display like
%  in Figure \vref{fig:markerI}.
%  \changes{v1.05a}{2004/02/17}{Added explanatory drawings for
%  \pkg{ntheorem}}
%
%  \begin{figure}[btp]\centering
%  \unitlength2.5pt \thicklines
%  \begin{picture}(90,30)(-30,40)
%    \put(0,50){\framebox(25,20){Math Height}}
%    \put(0,40){\framebox(25,10){Math Depth}}
%    \put(-30,50){\framebox(20,10){Tag height}}
%    \put(-30,45){\framebox(20,5){Tag depth}}
%    \put(40,40){\framebox(20,10){Marker}}
%    \multiput(26,40)(2,0){7}{\line(1,0){1}}
%  \end{picture}
%  \caption{Marker placement in \mode{leqno} mode}
%  \label{fig:markerI}
%  \end{figure}
%
%  \begin{codesyntax}
%    \SpecialUsageIndex{\mintagvsep}\cs{mintagvsep}
%  \end{codesyntax}
%  Unfortunately things are not this easy in \mode{reqno} mode. There
%  are two possible situations as shown in Figure \vref{fig:markerII}.
%  \begin{figure}[btp]
%  \unitlength2.5pt \thicklines
%  \begin{picture}(60,45)(0,25)
%    \put(0,50){\framebox(25,20){Math Height}}
%    \put(0,25){\framebox(25,25){Math Depth}}
%    \put(40,50){\framebox(20,10){Tag height}}
%    \put(40,45){\framebox(20,5){Tag depth}}
%    \put(40,25){\framebox(20,10){Marker}}
%    \put(50,35){\vector(0,1){10}}
%    \put(50,45){\vector(0,-1){10}}
%    \put(50,39){\ \ $>\delta$}
%    \multiput(26,25)(2,0){7}{\line(1,0){1}}
%  \end{picture} \hfill\vrule\hfill
%  \begin{picture}(60,45)(0,25)
%    \put(0,50){\framebox(25,20){Math Height}}
%    \put(0,40){\framebox(25,10){Math Depth}}
%    \put(40,50){\framebox(20,10){Tag height}}
%    \put(40,45){\framebox(20,5){Tag depth}}
%    \put(40,28){\framebox(20,10){Marker}}
%    \put(50,38){\vector(0,1){7}}
%    \put(50,45){\vector(0,-1){7}}
%    \put(50,40){\ \ $=\delta$}
%  \end{picture}
%  \caption{The two possible situations for end marks in \mode{reqno}
%  mode}
%  \label{fig:markerII}
%  \end{figure}
%  If possible we want the bottom of the marker to be aligned with
%  the bottom of the math display, but at the same time we want to
%  ensure a minimum vertical separation $\delta$ between marker and
%  tag. In \pkg{empheq} this is controlled by the length parameter
%  \cs{mintagvsep} which by default is 5\,pt.
%
%  The good news is that this is where \pkg{empheq} sets in.
%  Basically all you need to do is to use the \env{empheq}
%  environment to type set your equations inside the theorem
%  environments:
%  \begin{verbatim}
%    \begin{Theorem}
%    Some text at first and then a math display
%    \begin{empheq}{align}
%      a&=b\\
%      E&=mc^2 + \int_a^a x\, dx
%    \end{empheq}
%    \end{Theorem}
%  \end{verbatim}
%  Then the tag placement and the end-of-theorem marker will be set
%  properly (after a couple of runs as usual).
%
%  To ensure the correct outcome remember to load the packages like
%  this:
%  \begin{verbatim}
%    \usepackage[ntheorem]{empheq} % this loads amsmath as well
%    \usepackage[thmmarks,amsmath]{ntheorem}
%  \end{verbatim}
%  Remember that if you use the \opt{overload} option you
%  can just use the regular math environments to get the desired
%  result with the end mark. The exception is the |\[ ... \]|
%  environment which will only work if you use the option
%  \opt{overload2}.
%
%  \clearpage
%
%  \EmphEqpart{Advanced user's guide}
%
%  The \pkg{empheq} package has more to offer than you have seen, but
%  I found some of the functionality so dangerous it was best to hide
%  it for a little while. The commands you have encountered so far in
%  this manual all have one thing in common: They have only lower
%  case letters in their names. Now it's time to reveal those that
%  have \emph{mixed-case} names, thus implying that you should take
%  great care when using them.
%
%  \section{Delimiters revisited}
%  Let's go back to delimiters, shall we?
%
%  \subsection{Creating your own delimiters}
%  As a convenience for the user, this interface is extended in a
%  more general way so that it is also possible to declare delimiters
%  with the following two commands: \SpecialUsageIndex{\DeclareLeftDelimiter}
%  \SpecialUsageIndex{\DeclareRightDelimiter}
%  \begin{codesyntax}
%    \cs{DeclareLeftDelimiter}\oarg{space adjustment}\marg{delimiter}\\
%    \cs{DeclareRightDelimiter}\oarg{space adjustment}\marg{delimiter}
%  \end{codesyntax}
%  While \pkg{empheq} provides auto-scaling versions of the most
%  common delimiters, you may sometimes want some new ones. Say for
%  instance you have loaded the \pkg{stmaryrd} package and you want
%  to use the double bracket commands \cs{llbracket} and
%  \cs{rrbracket} with \pkg{empheq}. Then you simply do this:
%  \begin{verbatim}
%    \DeclareLeftDelimiter{\llbracket}
%    \DeclareRightDelimiter{\rrbracket}
%  \end{verbatim}
%  This defines the new delimiters \cs{empheqllbracket},
%  \cs{empheqbigllbracket}, \cs{empheqrrbracket} and
%  \cs{empheqbigrrbracket}.
%
%  You can use \cs{big...} delimiters as well if you don't like the
%  automatic scaling. There are however ways to fine-tune if you
%  really want it.
%
%  \subsection{Fine-tuning of delimiters}
%
%  \TeX{} provides two primitives to control the scaling of
%  delimiters produced with \cs{left} and \cs{right}, namely the
%  dimension \cs{delimitershortfall} (denoted by $\delta$ here) and
%  the integer \cs{delimiterfactor} ($f$). The idea is that the
%  sub-formula inside the \cs{left}-\cs{right} pair is to be
%  vertically centered and given its height $h_1$ and its depth $h_2$
%  we want to produce a delimiter with total height $h$, where  $h =
%  2 \max(h_1,h_2)$. \TeX' rules on this is that the minimum
%  delimiter size $h_{\min}$ must meet the requirements
%  \[
%    h_{\min}\geq h \frac{f}{1000} \quad\land \quad h_{\min}\geq h-\delta
%  \]
%  \LaTeX{} sets $\cs{delimitershortfall} = \text{\the\delimitershortfall} $
%  and $\cs{delimiterfactor} = \the\delimiterfactor$, but in our case
%  we will almost always want a delimiter that spans the entire
%  \cs{left}-\cs{right} pair (the math display), thus a change of
%  these settings is needed.
%
%  \begin{codesyntax}
%    \SpecialUsageIndex{\EmphEqdelimitershortfall}
%    \cs{EmphEqdelimitershortfall}\\
%    \SpecialUsageIndex{\EmphEqdelimiterfactor}
%    \cs{EmphEqdelimiterfactor}
%  \end{codesyntax}
%  However it is a bad idea to just change these two settings without
%  thinking of the effect it'll have on the rest of the mathematics
%  in your document. Therefore \pkg{empheq} provides the parameters
%  \cs{EmphEqdelimitershortfall} (default setting is
%  \the\EmphEqdelimitershortfall) and \cs{EmphEqdelimiterfactor}
%  (default is \the\EmphEqdelimiterfactor) to cater for this.
%
%
%  \subsection{Scaling material yourself}
%
%  The attentive reader may have noticed that I still haven't
%  revealed how I managed to get the right size of that big
%  exclamation mark in the first example of this manual, so I guess
%  it's about time.
%  \begin{codesyntax}
%    \SpecialUsageIndex{\EmphEqdisplayheight}\cs{EmphEqdisplayheight} \\
%    \SpecialUsageIndex{\EmphEqdisplaydepth}\cs{EmphEqdisplaydepth}
%  \end{codesyntax}
%  The height of the math display plus the surrounding inner box is
%  given by \cs{EmphEqdisplayheight} and the depth by
%  \cs{EmphEqdisplaydepth}. These can be used when constructing
%  something that should scale to fit the display and can't be done
%  with ``auto-scaling'' commands like \cs{vrule} or \cs{left} or
%  \cs{right}. There is a catch however: If you do any horizontal
%  resizing of the material you want to scale, then you must specify
%  the width \emph{manually}:\footnote{The
%  \cs{fontdimen22}\cs{textfont2} in the \meta{length} argument of
%  \cs{raisebox} is the \emph{math axis} of the font. It is needed in
%  order to get the vertical positioning of the oversized exclamation
%  mark just right.}
%  \begin{verbatim}
%  \begin{empheq}[
%    box=\colorbox{myblue},
%    right={\;
%           \makebox[.9em]{%
%            $\raisebox{-.5\totalheight+\fontdimen22\textfont2}
%              {\resizebox{!}{%
%                \EmphEqdisplayheight+\EmphEqdisplaydepth}{!}}$%
%           }},
%    left={X=Y\Rightarrow}
%     ]{alignat=3}
%    A_1&=b_1 & \qquad c_1&=d_1 & \qquad e_1&= f_1
%         \tag*{\_A\raisebox{1ex}{silly}\raisebox{-1ex}{tag}\_}\\
%    A_2&=b_2 & \qquad c_2&=d_2 & \qquad  e_2&= f_2 \\
%    A_3&=b_3 & \qquad c_3&=d_3 & \qquad  e_3&= f_3
%  \end{empheq}
%  \end{verbatim}
%
%  Here's another example. We want to be able to put a \cs{parbox}
%  with some descriptive text top-aligned on the side of the display.
%  \begin{verbatim}
%  \begin{empheq}[
%    left={\parbox[c][\EmphEqdisplayheight+\EmphEqdisplaydepth][t]
%           {4.5cm}
%           {You may find this kind of description useful.}\enspace}]
%      {align}
%    a&=\int_0^1 x\,dx +\frac{foo + bar}{baz}\\
%    E&= mc^2
%  \end{empheq}
%  \end{verbatim}
%  \begin{empheq}[
%    left={\parbox[c][\EmphEqdisplayheight+\EmphEqdisplaydepth][t]
%           {4.5cm}
%           {You may find this kind of description useful.}\enspace}]
%      {align}
%    a&=\int_0^1 x\,dx +\frac{foo + bar}{baz}\\
%    E&= mc^2
%  \end{empheq}
%
%
%  \section{A few short notes}
%
%  \subsection{About $\cs{eqref}$}
%
%  Internally \env{empheq} separate the displayed math from
%  the corresponding equation numbers such that we can add special
%  delimiters or boxes. This is done by nullifying (or rather
%  conveniently redefining) an internal command called
%  \cs{maketag@@@}. Unfortunately, this component is also used by
%  \cs{eqref} to typeset a reference to an equation number, so in
%  earlier versions, \cs{eqref} would not work inside an \env{empheq}
%  environment.  
%
% In the current version this should now be working, but if you ever
% need to mess with \cs{eqref} it may be handy to know what is being
% done. In essence we do something similar to this:
% \begin{verbatim}
%  \let\empheqeqrefbase\textup
%   ... % next go inside empheq env
%  \renewcommand\eqref[1]{\empheqeqrefbase{%
%       \let\maketag@@@\EQsavedmaketag%
%       \tagform@{\ref{##1}}}}
% \end{verbatim}
% Thus if you need to alter things inside \cs{eqref} and need that to
% work within \env{empheq} as well, you may get away with redefining
% \cs{empheqeqrefbase}.
%
%
%
%  \subsection{About changes to $\cs{baselineskip}$}
%
%  Users should never mess with \cs{baselineskip} directly, it is not
%  the correct manner to get double spacing. Have a look at say the
%  \pkg{setspace} package or similar, or play with
%  \cs{baselinestretch} followed by \cs{normalsize} to initiate. 
%
%  Explanation: We use \cs{parbox} inside to place the display box and
%  the box containing the eqn numbers. To do its stuff \cs{parbox}
%  resets some internal settings, in this case, \cs{baselineskip} is
%  reset to \cs{normalbaselineskip}. So if you don't remember to reset
%  that one as well\dots
%
%
%  \section{Creating something new}
%
%  You can create your own environments and boxes to go with
%  \pkg{empheq}, but there are certain things that must be fulfilled
%  to get it to work properly.
%
%  \subsection{New \env{empheq}-like environments}
%
%  \begin{codesyntax}
%    \SpecialEnvIndex{EmphEqMainEnv}\env{EmphEqMainEnv}\\
%    \SpecialUsageIndex{\EmphEqMainEnv}\cs{EmphEqMainEnv}\\
%    \SpecialUsageIndex{\endEmphEqMainEnv}\cs{endEmphEqMainEnv}
%  \end{codesyntax}
%  Although the real work in \pkg{empheq} is done by an environment
%  (\env{EmphEqMainEnv}), you cannot use it like that yourself when
%  you want to define an \env{empheq}-like environment. Due to
%  technical reasons (see~\cite{Ams99}), you have to use a slightly
%  different syntax:
%  \begin{codesyntax}
%    \cs{newenvironment}
%      \marg{env_name} \oarg{num} \oarg{default}\\
%      \texttt{~~}\arg{\meta{other commands} \cs{EmpheqMainEnv}}\\
%      \texttt{~~}\arg{\cs{endEmpheqMainEnv}}
%  \end{codesyntax}
%  In the above environment the \meta{other commands} part must
%  contain certain things before it'll work.
%  \begin{codesyntax}
%    \SpecialKeyFamilyIndex{EmphEqEnv}\keyfam{EmphEqEnv}\\
%    \SpecialKeyFamilyIndex{EmphEqOpt}\keyfam{EmphEqOpt}
%  \end{codesyntax}
%  Before your homemade environment will work you must set at least
%  one key. The key family \keyfam{EmphEqOpt} is for all the markup
%  instructions like \key{left} and \key{box}, while
%  \keyfam{EmphEqEnv} controls the type of \AmS{} environment. You
%  must set the \keyfam{EmphEqEnv} family before \cs{EmhpEqMainEnv}
%  else you'll get an error message.
%
%  One example could be something like this:
%  \begin{verbatim}
%  \newenvironment{important}[2][]{%
%    \setkeys{EmphEqEnv}{#2}%
%    \setkeys{EmphEqOpt}{box=\mybluebox,#1}%
%    \EmphEqMainEnv}%
%    {\endEmphEqMainEnv}
%  \end{verbatim}
%  Thus
%  \begin{verbatim}
%  \begin{important}{gather}
%    a = b + c +d \\ e = f
%  \end{important}
%  \end{verbatim}
%  produces
%  \begin{important}{gather}
%    a = b + c +d \\ e =  f
%  \end{important}
%  while
%  \begin{verbatim}
%  \begin{important}[left={A=B \Rightarrow \empheqlbrace}]{alignat=2}
%    a &= b &\quad c &= d \\
%    \text{this} &= \text{that} &\quad \mathit{fish}&\neq fish
%  \end{important}
%  \end{verbatim}
%  produces
%  \begin{important}[left={A=B \Rightarrow \empheqlbrace}]{alignat=2}
%    a &= b &\quad c &= d \\
%    \text{this} &= \text{that} &\quad \mathit{fish}&\neq fish
%  \end{important}
%
%
%  \subsection{Creating fancy boxes}\label{stunning}
%
%  As a final example I will show you how to create complicated
%  displays involving (vertically) asymmetrical boxes like
%  \cs{shadowbox*}. In order to get the correct output, the contents
%  of the box must be placed on the baseline as in \fbox{\cs{fbox}}.
%  In this example we want to put a set of equations into a bright
%  yellow box and then add another box with some explanatory text at
%  the top of the yellow box making them overlap.
%
%  First we define the colors used and allocate the boxes. We could
%  probably use scratch boxes, but this is safer.
%  \begin{verbatim}
%  \definecolor{shadecolor}{cmyk}{0,0,0.41,0}
%  \definecolor{light-blue}{cmyk}{0.25,0,0,0}
%  \newsavebox{\mysaveboxM} % M for math
%  \newsavebox{\mysaveboxT} % T for text
%  \end{verbatim}
%  Save the display body in \cs{mysaveboxM} and the text argument in
%  \cs{mysaveboxT}.
%  \begin{verbatim}
%  \newcommand*\Garybox[2][Example]{%
%    \sbox{\mysaveboxM}{#2}%
%    \sbox{\mysaveboxT}{\fcolorbox{black}{light-blue}{#1}}%
%  \end{verbatim}
%  Then typeset the math display in a \cs{parbox} where we control
%  the height and save it in \cs{mysaveboxM}.
%  \begin{verbatim}
%    \sbox{\mysaveboxM}{%
%      \parbox[b][\ht\mysaveboxM+.5\ht\mysaveboxT+.5\dp\mysaveboxT][b]{%
%        \wd\mysaveboxM}{#2}%
%    }%
%  \end{verbatim}
%  We put it into the colored box with the desired width.
%  \begin{verbatim}
%    \sbox{\mysaveboxM}{%
%      \fcolorbox{black}{shadecolor}{%
%        \makebox[\linewidth-10em]{\usebox{\mysaveboxM}}%
%      }%
%    }%
%  \end{verbatim}
%  Then finally we get to the real typesetting. We just insert the
%  math display and then make a box of zero width right next to it.
%  In that box we lift the text argument and center it at the top of
%  the display.
%  \begin{verbatim}
%    \usebox{\mysaveboxM}%
%    \makebox[0pt][r]{%
%      \makebox[\wd\mysaveboxM][c]{%
%        \raisebox{\ht\mysaveboxM-0.5\ht\mysaveboxT
%                  +0.5\dp\mysaveboxT-0.5\fboxrule}{\usebox{\mysaveboxT}}%
%      }%
%    }%
%  }
%  \end{verbatim}
%  Let's see what it looks like.
%  \begin{verbatim}
%  \begin{empheq}[box=\Garybox]{align}
%    \sum \mathbf{F} &= \mathbf{0} \\
%    \sum F_{x} \,\mathbf{i} + \sum F_{y} \,\mathbf{j}
%      + \sum F_{z}  \,\mathbf{k} &= \mathbf{0} \\
%    \sum F_{x} &= 0 \\ \sum F_{y} &= 0 \\ \sum F_{z} &= 0
%  \end{empheq}
%  \end{verbatim}
%  \begin{subequations}
%  \begin{empheq}[box=\Garybox]{align}
%    \sum \mathbf{F} &= \mathbf{0} \\
%    \sum F_{x} \,\mathbf{i} + \sum F_{y} \,\mathbf{j} + \sum F_{z}
%    \,\mathbf{k} &= \mathbf{0}\\
%  \sum F_{x} &= 0 \\ \sum F_{y} &= 0 \\ \sum F_{z} &= 0
%  \end{empheq}
%  If we use the optional argument of \cs{Garybox} we have to enclose
%  the entire argument of \key{box} in braces because we're in
%  \env{empheq}'s optional argument (as described in \cite[page
%  167]{Lamp94}):
%  \begin{verbatim}
%  \begin{empheq}[box={\Garybox[Same old example again]}]{align}
%    a&=b\\
%    E&=mc^2 + \int_a^a x\, dx
%  \end{empheq}
%  \end{verbatim}
%  \begin{empheq}[box={\Garybox[Same old example again]}]{align}
%    a&=b\\
%    E&=mc^2 + \int_a^a x\, dx
%  \end{empheq}
%  \end{subequations}
%
%
%
%
%  \section{Contact information}
%
%  Should you have any feature request, suggestions, or bug reports
%  then feel free to contact Lars Madsen at
%  \begin{quote}
%    \email{daleif@imf.au.dk}
%  \end{quote}
%  Please put `empheq' in the subject line.
%
%  \subsection*{Contributors}
%
%  \begin{itemize}
%    \item In November 2002 Lars Madsen (\email{daleif@imf.au.dk})
%    asked for some features that wasn't readily available with
%    \pkg{amsmath} or any other package. This was the start of
%    \pkg{empheq}.
%
%    \item Gary Gray (\email{gray@engr.psu.edu}) gave me some bug
%    reports on the old version (v0.7) which convinced me to rewrite
%    the package completely. The \cs{Garybox} example is inspired by
%    a wish from Gary.
%
%    \item Uwe Siart (\email{uwe.siart@tum.de}) has been a thorough
%    beta-tester on v1.00 of the package.
%
%    \item Bernard Alfonsi (\email{alfonsi@math.u-psud.fr}) reported
%     a problem with the \opt{ntheorem} option. It turned out to be a
%     bug in the \pkg{ntheorem} package itself but it's fixed in the
%     option.
%
%    \item I have received reports on weird behavior from
%    \pkg{empheq} when used with the \pkg{color} package on some
%    dvi-viewers, which is caused by the lack of color support in
%    those dvi-viewers. Both Andrew B.~Collier (\email{colliera@ukzn.ac.za})
%    and Andr\'e M.~de Roos (\email{aroos@science.uva.nl}) have
%    notified me of this. The solution is to either upgrade you
%    dvi-viewer if possible or convert the document in question to
%    PDF or PS.
%  \end{itemize}
%  Thank you all.
%
%
% \begin{thebibliography}{99}
% \bibitem{Lamp94}
%   Leslie Lamport,
%   \emph{LaTeX User's Guide and Reference Manual}, 2nd edition,
%   Addison-Wesley, Reading, Massachusetts,
%   1994.
% \bibitem{Goos-etal94}
%   Michel Goossens, Frank Mittelbach, and Alexander Samarin,
%   \emph{The \LaTeX\ Companion},
%   Addison-Wesley, Reading, Massachusetts,
%   1994.
% \bibitem{Knut86}
%   Donald Ervin Knuth,
%   \emph{The \TeX book},
%   Addison-Wesley, Reading, Mas\-sa\-chu\-setts,
%   1986.
%  \bibitem{Ams99}
%    American Mathematical Society and Michael Downes,
%    \emph{Technical notes on the \pkg{amsmath} package} Version 2.0,
%    1999/10/29.
%    (Available from CTAN as file \texttt{technote.tex}.)
%  \bibitem{Ams00}
%    Frank Mittelbach, Rainer Sch\"opf, Michael Downes, and David M.~Jones,
%    \emph{The \pkg{amsmath} package} Version 2.13,
%    2000/07/18.
%    (Available from CTAN as file \texttt{amsmath.dtx}.)
%
%  \changes{v1.05b}{2004/02/18}{Added \pkg{fancybox} and
%  \pkg{ntheorem} to the \refname}
%  \bibitem{fancybox}
%    Timothy Van Zandt,
%    \emph{The \pkg{fancybox} package} Version 1.3,
%    2000/09/19
%    (Available from CTAN as file \texttt{fancybox.sty}.)
%
% \bibitem{Raht03}
%   Sebastian Rahtz,
%   \emph{Hypertext marks in \LaTeX} Version v6.74m,
%   2003/11/30.
%   (Available from \url{http://www.tug.org/applications/hyperref} as file
%   \texttt{hyperref.dtx}.)
%
% \bibitem{Carl99}
%   David Carlisle,
%   \emph{The \pkg{keyval} Package},
%   Version 1.13, 1999/03/16.
%   (Available from CTAN as file \texttt{keyval.dtx}.)
%
%  \changes{v1.06}{2004/03/13}{Added \pkg{calc} to the \refname}
%  \bibitem{Thorup:calc:1998}
%  Kresten~Krab Thorup, Frank Jensen, and Chris Rowley.
%  \emph{The {\pkg{calc}} package}, Version 4.1b, 1998/07/07,
%  (Available from CTAN as file \texttt{calc.dtx}.)
%
%  \bibitem{ntheorem}
%    Wolfgang May and Andreas Schlechte,
%    \emph{The \pkg{ntheorem} package} Version 1.203,
%    2002/01/07,
%    (Available from CTAN as file \texttt{ntheorem.dtx}.)
%
%  \changes{v2.10}{2004/07/26}{Added \pkg{showkeys} to the \refname}
%  \bibitem{Carlisle:showkeys:1997}
%  David Carlisle. \emph{The {\pkg{showkeys}} package}, v3.12,
%  1997/06/12, (Available from CTAN as file \texttt{showkeys.dtx}.)
%
%  \end{thebibliography}
%
%  \StopEventually{}
%
%  \EmphEqpart{Implementation}
%
%  \section{Requirements, Options, and Strategy}
%  Lets start the package. We'll need the tools from \pkg{mhsetup}.
%    \begin{macrocode}
%<*package>
\ProvidesPackage{empheq}%
  [2014/07/16 v2.14 Emphasizing equations]
%    \end{macrocode}
%    \begin{macrocode}
\RequirePackage{mhsetup}[2004/10/10]
\MHInternalSyntaxOn
%    \end{macrocode}
%
%
%  \subsection{Requirements, Options, and Allocation}
%
%  The \opt{overload} option allows the user to type
%  \begin{verbatim}
%    \begin{gather}[box=\fbox]
%      a = b
%    \end{gather}
%  \end{verbatim}
%  when they really mean
%  \begin{verbatim}
%    \begin{empheq}[box=\fbox]{gather}
%      a = b
%    \end{empheq}
%  \end{verbatim}
%    \begin{macrocode}
\MH_new_boolean:n {overload_ams}
\DeclareOption{overload}{
  \MH_set_boolean_T:n {overload_ams}
}
\MH_new_boolean:n {overload_amsII}
\DeclareOption{overload2}{
  \MH_set_boolean_T:n {overload_ams}
  \MH_set_boolean_T:n {overload_amsII}
}
%    \end{macrocode}
%  Old crappy \env{multline} interface or the new dazzling?
%    \begin{macrocode}
\MH_new_boolean:n {good_mult}
\DeclareOption{oldmultline}{\MH_set_boolean_F:n {good_mult}}
\DeclareOption{newmultline}{\MH_set_boolean_T:n {good_mult}}
%    \end{macrocode}
%
%
%  Load the \pkg{ntheorem} support file.
%    \begin{macrocode}
\gdef\g@EQ@ntheorem@support@bool{01}
\DeclareOption{ntheorem}{\gdef\g@EQ@ntheorem@support@bool{00}}
%    \end{macrocode}
%  Pass other options to \pkg{mathtools}.
%    \begin{macrocode}
\DeclareOption*{
  \PassOptionsToPackage{\CurrentOption}{mathtools}
}
\ExecuteOptions{newmultline}
\ProcessOptions\relax
%    \end{macrocode}
%  The name of the game is \pkg{amsmath} so we'll need that of
%  course. Since v2.10 that is now taken care of in the
%  \pkg{mathtools} package which'll pass on the options to
%  \pkg{amsmath}.
%    \begin{macrocode}
\MHInternalSyntaxOff
\RequirePackage{mathtools}[2004/10/10]
%    \end{macrocode}
%  Activate the internal syntax.
%    \begin{macrocode}
\MHInternalSyntaxOn
\AtEndOfPackage{\MHInternalSyntaxOff}
%    \end{macrocode}
%  \changes{v2.00}{2004/04/14}{Changed description of naming
%  conventions. Altered \emph{all} private macros; this is not
%  mentioned at each macro as this would just be too troublesome}
%
%  \begin{macro}{\l_EQ_equationtype_int}
%  \begin{macro}{\l_EQ_alignmentmode_int}
%  I'll use the following numbering scheme for the equation types,
%  and store the value in \cs{l_EQ_equationtype_int}:
%  \begin{multicols}{2}
%  \begin{enumerate}\setlength{\itemsep}{0pt}
%    \item align
%    \item align*
%    \item gather
%    \item gather*
%    \item flalign
%    \item flalign*
%    \item alignat
%    \item alignat*
%  \end{enumerate}
%  \end{multicols}
%  As you notice \env{equation} and \env{equation*} are missing. This
%  is because \env{gather} and \env{gather*} work the same way, only
%  they are `nicer'. Cuts down on the amount of code as well. I will
%  add an alias so that a user can use the key \key{equation}, but
%  then \key{gather} will kick in instead behind the scenes. Also the
%  \env{multline} environments are missing here, but again it's
%  because I'm faking it. See \S~\vref{sec:flalign+multline} for
%  further details. The value $0$ isn't used for any of the
%  environments and thus serves as a checkpoint (see
%  page~\pageref{ZeroCheckpoint}).
%
%  Similarly I'll use the counter \cs{l_EQ_alignmentmode_int} to
%  store the standard options for placement of math displays and
%  equation numbering.
%  \begin{multicols}{2}
%  \begin{enumerate}\setlength{\itemsep}{0pt}
%    \item reqno
%    \item leqno
%    \item fleqn, reqno
%    \item fleqn, leqno
%  \end{enumerate}
%  \end{multicols}
%
%    \begin{macrocode}
\newcount\l_EQ_equationtype_int
\newcount\l_EQ_alignmentmode_int
\if@fleqn
  \iftagsleft@
    \l_EQ_alignmentmode_int=4
  \else:
    \l_EQ_alignmentmode_int=\thr@@
  \fi:
\else:
  \iftagsleft@
    \l_EQ_alignmentmode_int=\tw@
  \else:
    \l_EQ_alignmentmode_int=\@ne
  \fi:
\fi:
%    \end{macrocode}
%  \end{macro}
%  \end{macro}
%  \begin{macro}{\l_EQ_totalwidth_dim}
%  \begin{macro}{\l_EQ_displaywidth_dim}
%  \begin{macro}{\l_EQ_temp_linewidth_dim}
%  \begin{macro}{\EmphEqdelimitershortfall}
%  \begin{macro}{\EmphEqdelimiterfactor}
%  \begin{macro}{\g_EQ_toptag_height_dim}
%  \begin{macro}{\g_EQ_bottomtag_depth_dim}
%  \begin{macro}{\g_EQ_toprow_height_dim}
%  \begin{macro}{\g_EQ_bottomrow_depth_dim}
%  \begin{macro}{\g_EQ_widesttag_dim}
%  Some lengths we'll need. Could possibly be scratch registers but we
%  have to make global assignments so that road is a little
%  dangerous.
%    \begin{macrocode}
\newdimen\l_EQ_totalwidth_dim
\newdimen\l_EQ_displaywidth_dim
%    \end{macrocode}
%  Just to make it accessible.
%    \begin{macrocode}
\def\EmphEqdisplaywidth{\l_EQ_displaywidth_dim}
\newdimen\l_EQ_temp_linewidth_dim
\newdimen\EmphEqdelimitershortfall
\newcount\EmphEqdelimiterfactor
\EmphEqdelimiterfactor950
\EmphEqdelimitershortfall\p@
\newdimen\g_EQ_toptag_height_dim
\newdimen\g_EQ_bottomtag_depth_dim
\newdimen\g_EQ_toprow_height_dim
\newdimen\g_EQ_bottomrow_depth_dim
\newdimen\g_EQ_widesttag_dim
\def\EmphEqtagwidth{\g_EQ_widesttag_dim}
%    \end{macrocode}
%  \end{macro}
%  \end{macro}
%  \end{macro}
%  \end{macro}
%  \end{macro}
%  \end{macro}
%  \end{macro}
%  \end{macro}
%  \end{macro}
%  \end{macro}
%
%  \begin{macro}{\EQ_outerbox:}
%  \begin{macro}{\EQ_innerbox:}
%  \begin{macro}{\EQ_left_side:}
%  \begin{macro}{\EQ_right_side:}
%  The two boxes and the material on either side of the math display.
%    \begin{macrocode}
\def\EQ_outerbox:{}
\def\EQ_innerbox:{}
\def\EQ_left_side:{}
\def\EQ_right_side:{}
%    \end{macrocode}
%  \end{macro}
%  \end{macro}
%  \end{macro}
%  \end{macro}
%
% \subsection{Strategy}
%
%  In non-\mode{fleqn} math displays are centered---even when having
%  tags. However this can look rather appalling when either the tag
%  or the math display are very wide like in the following example.
%  \begin{center}
%  \makebox[\linewidth][s]{\mbox{}\hfill \mbox{$\displaystyle
%    a+b+c+d+e+f+g+h+i+j+k+l+m+n+o$} \hfill\mbox{} \llap{(wide tag)}}
%  \end{center}
%  Not exactly pleasing to look at. \pkg{amsmath} takes care to do
%  this adjustment for us, so the user instead obtains the far better
%  \begin{gather*}
%  a+b+c+d+e+f+g+h+i+j+k+l+m+n+o \tag{wide tag}
%  \end{gather*}
%  So \pkg{amsmath} handles this quite well. That's good. It looks
%  like \pkg{amsmath} has an internal threshold that matches the
%  width of the math display against the width of the tags. But what
%  if we have a multiple line display and one of the lines causes an
%  adjustment to kick in while the others don't need it? Then we
%  cannot keep track of the complete display anymore. Furthermore we
%  may want to add material on either side of the math display making
%  the whole thing even more complicated, if not impossible to do
%  without some sort of different approach.
%
%  Luckily we have such an approach. The idea that comes to mind is
%  to see if one can typeset the labels and the math displays
%  \emph{separately}, store them in separate boxes and then put them
%  back together again afterwards. Is this really possible? Yes.
%  Come, I'll show you.
%
%  \subsection{The Main Environment}
%  \begin{macro}{\EQ_mathdisplay_box}
%  \begin{macro}{\EmphEqdisplayheight}
%  \begin{macro}{\EmphEqdisplaydepth}
%  \begin{macro}{\EQ_tag_box}
%  Okay, we'll start by allocating two box registers.
%    \begin{macrocode}
\newsavebox\EQ_mathdisplay_box
\def\EmphEqdisplayheight{\ht\EQ_mathdisplay_box}
\def\EmphEqdisplaydepth{\dp\EQ_mathdisplay_box}
\newsavebox\EQ_tag_box
%    \end{macrocode}
%  \end{macro}
%  \end{macro}
%  \end{macro}
%  \end{macro}
%
%  \begin{macro}{\EQ_complete_mathdisplay:}
%  \begin{macro}{\EQ_getbody:n}
%  \begin{environment}{EmphEqMainEnv}
%  The \env{EmphEqMainEnv} is (as the name implies) the main
%  environment. I decided to go with the environment form because
%  that's what \pkg{amsmath} users are accustomed to. But the code
%  will require the contents of the environment to be handled as an
%  argument to a command, so we use \cs{collect@body} from
%  \pkg{amsmath} itself to collect it.
%    \begin{macrocode}
\def\EQ_complete_mathdisplay:{}
\def\EQ_getbody:n #1{%
  \def\EQ_complete_mathdisplay:{}\def\EQ_mathbody:{#1}}
\newenvironment{EmphEqMainEnv}{\collect@body\EQ_getbody:n}{%
%    \end{macrocode}
%  If the user didn't specify the mandatory argument we will catch it
%  because \cs{l_EQ_equationtype_int} is zero. An error is
%  issued\label{ZeroCheckpoint}.
%    \begin{macrocode}
  \if_num:w \l_EQ_equationtype_int=\z@ %ZeroCheckpoint
    \EQ_upgrade_error:
  \else:
%    \end{macrocode}
%  Those that read the manual get the reward here: The package will
%  actually work!
%    \begin{macrocode}
    \EQ_post_process:
    \ignorespacesafterend
  \fi:}
%    \end{macrocode}
%  \end{environment}
%  \end{macro}
%  \end{macro}
%  \subsection{Post Processing}
%  \begin{macro}{\EQ_post_process:}
%  I know: it almost looked easy for a while didn't it? Sadly the
%  post-processing step of the code is also by far the most
%  complicated.
%    \begin{macrocode}
\def\EQ_post_process:{%
%    \end{macrocode}
%  First a hook where we initialize a few things like storing the
%  line width for later manipulation. Also we want to measure the
%  height of the top row and the depth of the bottom row for best
%  possible placement of the math display and the boxes around it
%  (same with the tags). Those four dimensions are globally set
%  during the code, hence this assignment. Most other stuff is local.
%    \begin{macrocode}
  \EQ_initialize:
%    \end{macrocode}
%  Then we proceed by completing the input. We store it in the
%  control sequence \cs{EQ_complete_mathdisplay}.
%  \cs{EQ_begin_equation:} adds the start of the environment and
%  \cs{EQ_end_equation:} adds the corresponding end.
%    \begin{macrocode}
  \EQ_begin_equation:
%    \end{macrocode}
%  In case we face a \env{multline} we add what's necessary.
%    \begin{macrocode}
  \MH_if_boolean:nT {outer_mult}{
    \setkeys{\MT_options_name:}{
      firstline-afterskip=0pt,
      lastline-preskip=0pt,
      multlined-width=0pt,
    }
    \ifodd \l_EQ_alignmentmode_int
      \setkeys{\MT_options_name:}{multlined-pos=b}
    \else:
      \setkeys{\MT_options_name:}{multlined-pos=t}
    \fi:
      \g@addto@macro\EQ_complete_mathdisplay:{
        \begin{MTmultlined}\relax}
  }
  \g@addto@macro\EQ_complete_mathdisplay:\EQ_mathbody:
  \MH_if_boolean:nT {outer_mult}{
    \g@addto@macro\EQ_complete_mathdisplay:{\end{MTmultlined}}
  }
  \EQ_end_equation:
%    \end{macrocode}
%  Save the value of the \pkg{hyperref} link counter
%  \cs{Hy@linkcounter}.
%    \begin{macrocode}
  \edef\EQ_hyperref_link_fint{\number\Hy@linkcounter}
%    \end{macrocode}
%  Firstly we call \cs{EQ_typeset_tags:} which suppresses all
%  output from the equations with the exception of the \cs{tag}'s.
%  This also allows us to find the widest tag which we'll need for
%  the environments that set's the math displays flush to the
%  margins, namely the \env{flalign} family. We save the counters
%  first though. As an extra treat we're also able to measure the
%  height and depth of the tags and the math display.
%    \begin{macrocode}
  \savecounters@
  \EQ_typeset_tags:
%    \end{macrocode}
%  \changes{v1.05d}{2004/02/26}{Removed one instance of
%  \cs{EQ_typeset_display:}}
%  Then find the widest tag. In an earlier implementation it was
%  necessary to typeset the display twice---a very costly procedure!
%  Now we save roughly 2120 lines of output from \cs{tracingall} on a
%  simple \env{gather} and 2600 lines with \env{align}. That's  a
%  little more than 21\% saved\ldots
%  \changes{v2.10}{2004/07/26}{Removed \cs{find_widesttag:}}
%  \changes{v2.10}{2004/07/26}{Added check for equation number to
%  make sure the labelling/printing mechanism of \pkg{mathtools} work
%  properly}
%  Then we restore the counters and measure the extra material.
%  Save the value of the equation counter for later in case the
%  special equation numbering scheme from \pkg{mathtools} is in
%  effect.
%    \begin{macrocode}
  \edef\l_EQ_equation_fint{\number\c@equation}
  \restorecounters@
%    \end{macrocode}
%  Restore the value of the \pkg{hyperref} link counter
%  \cs{Hy@linkcounter}.
%    \begin{macrocode}
  \Hy@linkcounter=\EQ_hyperref_link_fint
  \settowidth\l_EQ_totalwidth_dim{%
    $\EQ_left_side:\EQ_innerbox:{}\EQ_right_side:$}%
%    \end{macrocode}
%  Then call the appropriate \cs{EQ_modify_linewidth_}\meta{num}\texttt{:}
%  where \meta{num} is taken from the numbering scheme shown earlier.
%    \begin{macrocode}
  \MH_if_boolean:nTF {outer_mult}
    { \def\EQ_next:{_mult} }
    { \def\EQ_next:{} }
  \@nameuse{EQ_modify_linewidth_
            \romannumeral\l_EQ_equationtype_int\EQ_next: :}%
%    \end{macrocode}
%  We can then typeset the display knowing that all modifications
%  have been made correctly. \cs{EQ_typeset_display:} gobbles the
%  \cs{tag}'s, which in turn is equivalent to setting the display
%  centered in non-\mode{fleqn} mode.
%    \begin{macrocode}
  \EQ_typeset_display:
%    \end{macrocode}
%  Call the appropriate \cs{find_displaywidth_}\meta{num}\texttt{:}
%  which extracts the display width (\cs{l_EQ_displaywidth_dim}).
%    \begin{macrocode}
  \@nameuse{find_displaywidth_
           \romannumeral\l_EQ_equationtype_int\EQ_next: :}
  \MH_addtolength:dn
    \l_EQ_totalwidth_dim{\l_EQ_displaywidth_dim}
%    \end{macrocode}
%  We save the math display in a box of the correct size.
%    \begin{macrocode}
  \savebox\EQ_mathdisplay_box[\l_EQ_totalwidth_dim][c]{%
    $\EQ_left_side:
    \EQ_innerbox:{%
      \if_case:w \l_EQ_alignmentmode_int\or:
%    \end{macrocode}
%  If the equation is centered we make the box centered
%    \begin{macrocode}
        \makebox[\l_EQ_displaywidth_dim][c]
          {\usebox{\EQ_mathdisplay_box}}%
      \or:
        \makebox[\l_EQ_displaywidth_dim][c]
          {\usebox{\EQ_mathdisplay_box}}%
      \or:
%    \end{macrocode}
%  Else we set it in a left aligned box and take care of the
%  \cs{@mathmargin} unless we're in \env{flalign} or \env{flalign*}.
%    \begin{macrocode}
        \makebox[\l_EQ_displaywidth_dim][l]{%
          \if_num:w \l_EQ_equationtype_int=5 \else:
            \if_num:w \l_EQ_equationtype_int=6 \else:
              \kern-\@mathmargin
            \fi:
          \fi:
          \usebox{\EQ_mathdisplay_box}}%
      \or:
        \makebox[\l_EQ_displaywidth_dim][l]{%
          \if_num:w \l_EQ_equationtype_int=5 \else:
            \if_num:w \l_EQ_equationtype_int=6 \else:
              \kern-\@mathmargin
            \fi:
          \fi:
          \usebox{\EQ_mathdisplay_box}}%
      \fi:}%
    \EQ_right_side:$}%
%    \end{macrocode}
%  Then we're ready for typesetting. We just use an ordinary
%  |\begin{equation*}| | ... | |\end{equation*}| construction.
%  \changes{v2.00}{2004/04/14}{Used a saved definition of
%  \env{equation*} instead to allow overloading}
%  With v2.00 this changed to using a copy of the original
%  \env{equation*} environment. All this because of the
%  \opt{overload} option.
%    \begin{macrocode}
  \begin{AmSequation*}
  \@nameuse{EQ_place_display_
           \romannumeral\l_EQ_equationtype_int\EQ_next: :}
%    \end{macrocode}
%  \cs{tag} must have a non-zero width argument in order to function
%  properly, so we make sure there is a tag before we try to typeset
%  it. Also make sure that the boolean |show_manual_tags| is true
%  just for this tag.
%    \begin{macrocode}
  \if_dim:w \g_EQ_widesttag_dim>\z@
  \MH_set_boolean_T:n {show_manual_tags}
%    \end{macrocode}
%  Set the tag in a left aligned box when in \mode{leqno} mode and
%  right aligned when in \mode{reqno} mode.
%    \begin{macrocode}
    \if_case:w \l_EQ_alignmentmode_int
    \or: \EQ_place_tag:N r %#1
    \or: \EQ_place_tag:N l %#2
    \or: \EQ_place_tag:N r %#3
    \or: \EQ_place_tag:N l %#4
    \fi:
  \fi:
  \end{AmSequation*}
%    \end{macrocode}
%  \changes{v2.11}{2004/08/03}{Use raw \TeX{} commands to speed it up}
%  Restore the equation counter.
%    \begin{macrocode}
  \global\c@equation=\l_EQ_equation_fint\relax
}
%    \end{macrocode}
%  \end{macro}
%
%  \subsubsection{Initializing the Equations}
%
%  \begin{macro}{\EQ_initialize:}
%  We save all the initializations in this hook:
%    \begin{macrocode}
\def\EQ_initialize:{%
  \m@th
%    \end{macrocode}
%  More support for \pkg{ntheorem}. We use special versions of
%  the tag setting commands in order to produce the correct result.
%  For instance in \mode{leqno} mode you would still want the end of
%  theorem symbol at the bottom of the math display.
%    \begin{macrocode}
  \MH_let:NwN\SetTagPlusEndMark\EQ_tag_plus_endmark:
  \MH_let:NwN\SetOnlyEndMark\EQ_only_endmark:
  \l_EQ_temp_linewidth_dim\linewidth
  \g_EQ_toprow_height_dim\z@
  \g_EQ_bottomrow_depth_dim\z@
  \g_EQ_toptag_height_dim\z@
  \g_EQ_bottomtag_depth_dim\z@
  \g_EQ_widesttag_dim\z@
%    \end{macrocode}
%  \changes{v2.11}{2004/08/03}{Added a \cs{g_EQ_latest_nonzerodepth_row_fint}}
%    \begin{macrocode}
  \xdef\g_EQ_latest_nonzerodepth_row_fint{\z@}
  \edef\EQ_restore_tex_delimiter:{%
    \delimiterfactor\the\delimiterfactor
    \delimitershortfall\the\delimitershortfall\relax}%
  \g@addto@macro\EQ_complete_mathdisplay:
                \EQ_restore_tex_delimiter:
  \delimitershortfall\EmphEqdelimitershortfall
  \delimiterfactor\EmphEqdelimiterfactor
%    \end{macrocode}
%  Disable \cs{intertext} and issue an error message. Using the
%  normal error message for an invalid \cs{intertext} would be less
%  than great because it would say
%  \begin{verbatim}
%  ! Package amsmath Error: Invalid use of \intertext.
%
%  See the amsmath package documentation for explanation.
%  Type  H <return>  for immediate help.
%   ...
%  \end{verbatim}
%  That's not good enough, so we redefine things a bit. In
%  \pkg{amsmath} the command \cs{intertext@} is called inside the
%  math environments and defines \cs{intertext}. We hook into this.
%    \begin{macrocode}
  \MH_let:NwN\intertext@\EQ_intertext:
  \EQ_displaybreak:}
%    \end{macrocode}
%  \end{macro}
%  \begin{macro}{\EQ_intertext:}
%  Then we issue a suitable error message.
%    \begin{macrocode}
\def\EQ_intertext:{%
  \def\intertext##1{\PackageError{empheq}{%
    You~cannot~use~\string\intertext\space inside~the~`empheq'
    \MessageBreak
    environment,~as~`empheq'~creates~an~unbreakable~box}\@eha}}
\def\EQ_displaybreak:{%
  \def\dspbrk@[##1]{\PackageWarning{empheq}{%
    You~cannot~use~\string\displaybreak\space inside~the~`empheq'%
    \MessageBreak
    environment,~as~`empheq'~creates~an~unbreakable~box.}}}
%    \end{macrocode}
%  \end{macro}
%  \begin{macro}{\EQ_new_equationtype:nnn}
%  \begin{macro}{\EQ_new_equationtype_arg:nnn}
%  We have to inform \env{EmphEqMainEnv} which equation type we're
%  using, so we define an interface for that.
%    \begin{macrocode}
\def\EQ_new_equationtype:nnn #1#2#3{%
  \define@key{#1}{#2}[true]{%
    \l_EQ_equationtype_int=#3\relax
    \def\EQ_begin_equation:
      {\g@addto@macro{\EQ_complete_mathdisplay:}{\begin{#2}}}%
    \def\EQ_end_equation:
      {\g@addto@macro{\EQ_complete_mathdisplay:}{\end{#2}}}}}
\def\EQ_new_equationtype_arg:nnn #1#2#3{%
  \define@key{#1}{#2}{%
    \l_EQ_equationtype_int=#3\relax
    \def\EQ_begin_equation:
      {\g@addto@macro{\EQ_complete_mathdisplay:}{\begin{#2}{##1}}}%
    \def\EQ_end_equation:
      {\g@addto@macro{\EQ_complete_mathdisplay:}{\end{#2}}}}}
%    \end{macrocode}
%  \end{macro}
%  \end{macro}
%
%  \subsubsection{Separating Tags and Display}
%
%  \begin{macro}{\EQ_typeset_tags:}
%  Typesetting the tags without the display.
%    \begin{macrocode}
\def\EQ_typeset_tags:{%
  \sbox\EQ_tag_box{%
    \parbox{.5\maxdimen}{%
      \hbadness\@M
%    \end{macrocode}
%  Make sure all skips are zero ands support \pkg{ntheorem}.
%  \changes{v1.05a}{2004/02/17}{Forgot to support \pkg{ntheorem} in
%  both \cs{EQ_typeset_tags:} and \cs{EQ_typeset_display:}.}
%    \begin{macrocode}
      \EQ_prevent_vertical_space:
%    \end{macrocode}
%  Support for \pkg{ntheorem}. This package patches \cs{endgather}
%  and \cs{endalign} with the commands \cs{TagsPlusEndmarks} and
%  \cs{RestoreTags} but we do not want them disturbing our output so
%  we disable them when typesetting the display.
%    \begin{macrocode}
      \EQ_typeset_tags_hook:
%    \end{macrocode}
%  Then use our own versions of \cs{set@gather@field} and
%  \cs{set@field}.
%    \begin{macrocode}
      \MH_let:NwN\set@gather@field\EQ_set_gather_field:
      \MH_let:NwN\set@field\EQ_set_align_field:
%    \end{macrocode}
%  Next two lines are for measuring of the first and last row of a
%  display of the \env{align}, \env{alignat} or \env{flalign} family.
%    \begin{macrocode}
      \MH_set_boolean_T:n {continue_measuring}
      \global\g_EQ_temprow_int\m@ne
      \MH_let:NwN \EQ_saved_maketag:n \maketag@@@
      \def\maketag@@@##1{
        \EQ_maketag:Nn \EQ_saved_maketag:n {##1}}
%    \end{macrocode}
%  Then typeset the display.
%    \begin{macrocode}
      \EQ_complete_mathdisplay:}}}
%    \end{macrocode}
%  \end{macro}
%
%  \begin{macro}{\EQ_predisplay_adjustment:}
%  \changes{v1.05c}{2004/02/25}{Added \cs{EQ_predisplay_adjustment:}}
%
%  \changes{v2.00}{2004/04/14}{Added a missing \cs{normallineskip} in
%  case of extremely big tag height}
%
%  For adjusting vertical spacing in the display.
%    \begin{macrocode}
\def\EQ_predisplay_adjustment:{%
  \if_dim:w \g_EQ_toprow_height_dim<\baselineskip
    \vskip\g_EQ_toprow_height_dim
    \vskip-\baselineskip
  \else:
    \vskip-\normallineskip
  \fi:
  \if_dim:w \g_EQ_toptag_height_dim>\g_EQ_toprow_height_dim
    \if_dim:w \g_EQ_toptag_height_dim>\baselineskip
      \vskip-\g_EQ_toptag_height_dim
      \vskip\baselineskip
      \vskip-\normallineskip
    \fi:
  \fi:}
%    \end{macrocode}
%  \end{macro}
%  \begin{macro}{\EQ_postdisplay_adjustment:}
%  \changes{v1.05c}{2004/02/25}{Added \cs{EQ_postdisplay_adjustment:}}
%  Name says it all.
%    \begin{macrocode}
\def\EQ_postdisplay_adjustment:{%
  \if_dim:w \g_EQ_bottomrow_depth_dim<\dp\strutbox
    \vskip\g_EQ_bottomrow_depth_dim
    \vskip-\dp\strutbox
  \fi:
  \if_dim:w \g_EQ_bottomtag_depth_dim>\g_EQ_bottomrow_depth_dim
    \if_dim:w \g_EQ_bottomtag_depth_dim>\dp\strutbox
      \vskip-\g_EQ_bottomtag_depth_dim
      \vskip\dp\strutbox
    \fi:
  \fi:}
%    \end{macrocode}
%  \end{macro}
%  \begin{macro}{\EQ_typeset_display:}
%  \changes{v1.05c}{2004/02/25}{Now takes two arguments: Either both
%  empty or \cs{EQ_predisplay_adjustment:} \&
%  \cs{EQ_postdisplay_adjustment:}}
%  \changes{v1.05d}{2004/02/26}{Changed again. Uses only
%  \cs{EQ_predisplay_adjustment:} \& \cs{EQ_postdisplay_adjustment:}}
%  We typeset the display without the tags and save it in a box.
%    \begin{macrocode}
\def\EQ_typeset_display:{%
  \sbox\EQ_mathdisplay_box{%
    \parbox{\l_EQ_temp_linewidth_dim}{%
%    \end{macrocode}
%  We adjust the vertical spacing if needed:
%    \begin{macrocode}
      \EQ_predisplay_adjustment:
      \EQ_prevent_vertical_space:
%    \end{macrocode}
%  Support for \pkg{ntheorem}. This package patches \cs{endgather}
%  and \cs{endalign} with the commands \cs{TagsPlusEndmarks} and
%  \cs{RestoreTags} but we do not want them disturbing our output so
%  we disable them when typesetting the display.
%    \begin{macrocode}
      \EQ_typeset_display_hook:
%    \end{macrocode}
%  Instead of gobbling the tags it's better to use \cs{phantom} since
%  a tall tag will ruin the vertical spacing. We use a special
%  version of \cs{phantom} to help us with other things as well.
%    \begin{macrocode}
      \def\maketag@@@##1{\EQ_maketag:Nn \vphantom{##1}}%
      \EQ_complete_mathdisplay:
%    \end{macrocode}
%  More space adjustment.
%    \begin{macrocode}
      \EQ_postdisplay_adjustment:}}}
%    \end{macrocode}
%  \end{macro}
%  We need a few little helpers:
%  \begin{macro}{\EQ_prevent_vertical_space:}
%  \changes{v2.00}{2004/04/14}{Used \cs{nointerlineskip} instead}
%  As we typeset the math display inside a \cs{parbox} the math
%  environment will add \cs{abovedisplayskip} at the top and
%  \cs{belowdisplayskip} at the bottom. Also we suppress interline
%  skip with \cs{nointerlineskip}.
%    \begin{macrocode}
\def\EQ_prevent_vertical_space:{%
  \nointerlineskip
  \abovedisplayskip\z@  \belowdisplayskip\z@\relax}
%    \end{macrocode}
%  \end{macro}
%  \begin{macro}{\EQ_maketag:Nn}
%  \changes{v1.05d}{2004/02/26}{Changed name from
%  \cs{EmphEqphantom} to \cs{EmphEqMaketag}}
%  This takes care of two things: We need to store the height of the
%  top tag and the depth of the bottom tag for later, but we also
%  need to use a \cs{vphantom} of the tag as it may be taller than
%  the line of math it is attached to (say, if you are using node
%  connectors from \pkg{pstricks} or some other weird stuff).
%    \begin{macrocode}
\def\EQ_maketag:Nn #1#2{%
  \setboxz@h{#2}%
  \if_num:w \row@=\@ne
    \global \g_EQ_toptag_height_dim = \ht\z@
  \fi:
%    \end{macrocode}
%  Similarly the depth of the last row is determined by measuring the
%  depth of box zero for each row and then using the last of the
%  values.
%    \begin{macrocode}
  \global \g_EQ_bottomtag_depth_dim=\dp\z@
%    \end{macrocode}
%  \changes{v2.11}{2004/08/03}{Added a check for being in the last row.}
%  Save the row number for later as the \opt{ntheorem} option will need
%  it.
%    \begin{macrocode}
  \xdef\g_EQ_latest_nonzerodepth_row_fint{\number\row@}
  \@tempdima=\wd\z@
  \if_dim:w \@tempdima > \g_EQ_widesttag_dim
    \global \g_EQ_widesttag_dim = \@tempdima
  \fi:
  #1{#2}%
}
%    \end{macrocode}
%  \end{macro}
%
%  \subsubsection{Measuring \env{gather}}
%  \begin{macro}{\EQ_set_gather_field:}
%  Box zero (\cs{z@}) contains the entire line of math in a
%  \env{gather}, so instead of typesetting it we typeset the phantom
%  of it and call \cs{EQ_measure_gather:}.
%    \begin{macrocode}
\def\EQ_set_gather_field:{%
  \iftagsleft@
    \global\lineht@\ht\z@
  \else:
    \global\lineht@\dp\z@
  \fi:
  \kern\eqnshift@
  %\boxz@ % Original amsmath version
  \EQ_measure_gather:
  \phantom{\boxz@}%
  \hfil}
%    \end{macrocode}
%  \end{macro}
%  \begin{macro}{\EQ_measure_gather:}
%  Measuring the height of the individual lines in a \env{gather} is
%  fairly easy. If \cs{row@} is equal to one, we're in the first
%  line, and the height of the top row is simply given by the height
%  of box zero.
%    \begin{macrocode}
\def\EQ_measure_gather:{%
  \if_num:w \row@=\@ne
    \g_EQ_toprow_height_dim=\ht\z@
  \fi:
%    \end{macrocode}
%  Similarly the depth of the last row is determined by measuring the
%  depth of box zero for each row and then using the last of the
%  values.
%    \begin{macrocode}
  \g_EQ_bottomrow_depth_dim=\dp\z@
  \global\g_EQ_toprow_height_dim\g_EQ_toprow_height_dim
  \global\g_EQ_bottomrow_depth_dim\g_EQ_bottomrow_depth_dim}
%    \end{macrocode}
%  \end{macro}
%
%  \subsubsection{Measuring \env{align}, \env{alignat} and \env{flalign}}
%  \begin{macro}{\EQ_set_align_field:}
%  Same technique as for \cs{EQ_set_gather_field}. Only this time
%  we have to stop the underlying \cs{start@align} from going over
%  the measuring phase twice. That's when the conditional
%  \bool{continue_measuring} comes into the picture.
%    \begin{macrocode}
\def\EQ_set_align_field:{%
  \column@plus
  \iftagsleft@
    \if_dim:w \ht\z@>\lineht@
      \global\lineht@\ht\z@
    \fi:
  \else:
    \if_dim:w \dp\z@>\lineht@
      \global\lineht@\dp\z@
    \fi:
  \fi:
  %\boxz@ % Original amsmath version
  \MH_if_boolean:nT {continue_measuring}{ \EQ_measure_align: }
  \phantom{\boxz@}}
%    \end{macrocode}
%  \end{macro}
%  \begin{macro}{\EQ_measure_align:}
%  We have to measure every single field of the \cs{align@} \ldots\@
%  Oh well, here it comes:
%    \begin{macrocode}
\newcount\g_EQ_temprow_int
\MH_new_boolean:n {still_in_samerow}
\MH_new_boolean:n {continue_measuring}
\def\EQ_measure_align:{%
  \if_num:w \g_EQ_temprow_int=\row@
    \MH_set_boolean_T:n {still_in_samerow}
  \else:
    \if_num:w \g_EQ_temprow_int>\row@
      \global\MH_set_boolean_F:n {continue_measuring}
    \fi:
    \MH_set_boolean_F:n {still_in_samerow}
    \global\@tempdimc=\@tempdimb
    \global\@tempdimb=\g_EQ_bottomrow_depth_dim
  \fi:
  \MH_if_boolean:nTF {continue_measuring}{
    \global\g_EQ_temprow_int=\row@
    \@tempcnta=\row@
    \if_num:w \@tempcnta=\z@
      \@tempdima=\g_EQ_toprow_height_dim
      \g_EQ_toprow_height_dim=\ht\z@\relax
      \if_dim:w \g_EQ_toprow_height_dim<\@tempdima
        \g_EQ_toprow_height_dim\@tempdima
      \fi:
    \fi:
    \MH_if_boolean:nTF {still_in_samerow}{
      \@tempdima=\g_EQ_bottomrow_depth_dim
      \g_EQ_bottomrow_depth_dim=\dp\z@\relax
      \if_dim:w \g_EQ_bottomrow_depth_dim<\@tempdima
        \g_EQ_bottomrow_depth_dim\@tempdima
      \fi:
    }{
      \g_EQ_bottomrow_depth_dim\dp\z@\relax
    }
  }{
    \global\g_EQ_bottomrow_depth_dim=\@tempdimc
  }
  \global\g_EQ_toprow_height_dim\g_EQ_toprow_height_dim
  \global\g_EQ_bottomrow_depth_dim\g_EQ_bottomrow_depth_dim}
%    \end{macrocode}
%  \end{macro}
%  \subsubsection{Typesetting the Tags}
%  \begin{macro}{\EQ_place_tag:N}
%  Fool \LaTeX\ into thinking the tag is only as wide as the widest
%  tag (remember the real width of the \cs{EQ_tag_box} is
%  $0.5$\cs{maxdimen}.
%    \begin{macrocode}
\def\EQ_place_tag:N #1{%
  \tag*{\makebox[\g_EQ_widesttag_dim][#1]{%
    \EQ_vertical_tag_adjustment:n {\usebox{\EQ_tag_box}}}}}
%    \end{macrocode}
%  \end{macro}
%  \begin{macro}{\EQ_vertical_tag_adjustment:n}
%  \changes{v2.00}{2004/04/14}{Added a missing \cs{normallineskip} in
%  case of extremely big tag height}
%  This is a little tricky. We have measured the height of the top
%  row and top tag and the depth of the bottom row and bottom tag.
%  Now we have to adjust the vertical spacing so it comes out just
%  right. Remember we have separated the tags and the math display
%  but we would still like to have them line up as they should! We do
%  it all inside a group so we can use the scratch dimensions safely.
%    \begin{macrocode}
\def\EQ_vertical_tag_adjustment:n #1{{%
  \@tempdima\z@\@tempdimb\z@
  \@tempdimc=\dp\strutbox
  \if_dim:w \g_EQ_toprow_height_dim<\baselineskip
    \MH_setlength:dn
      \@tempdima{\g_EQ_toprow_height_dim-\baselineskip}
  \else:
    \MH_setlength:dn \@tempdima{-\normallineskip}
  \fi:
  \if_dim:w \g_EQ_toptag_height_dim>\g_EQ_toprow_height_dim
    \if_dim:w \g_EQ_toptag_height_dim>\baselineskip
      \MH_addtolength:dn \@tempdima
        {-\g_EQ_toptag_height_dim+\baselineskip-\normallineskip}
    \fi:
  \fi:
  \if_dim:w \g_EQ_bottomrow_depth_dim<\@tempdimc
    \MH_setlength:dn \@tempdimb
      {\g_EQ_bottomrow_depth_dim-\@tempdimc}
  \fi:
%    \end{macrocode}
%  \changes{v2.00}{2004/04/14}{Fixed minor bug}
%  If the last condition is fulfilled we also need to cancel the
%  first condition.
%    \begin{macrocode}
  \if_dim:w \g_EQ_bottomtag_depth_dim>\g_EQ_bottomrow_depth_dim
    \if_dim:w \g_EQ_bottomtag_depth_dim<\@tempdimc
    \else:
      \MH_addtolength:dn \@tempdimb
        {-\g_EQ_bottomtag_depth_dim+\@tempdimc}
    \fi:
  \fi:
  \raisebox{-0.5\@tempdima+0.5\@tempdimb}{#1}}}%
%    \end{macrocode}
%  \end{macro}
%  \section{Adding the Equation Types}
%  Adding an equation type is fairly easy. The outline is this:
%  \begin{list}{}{\leftmargin2em}
%    \item[]|\EQ_new_equationtype{EmphEqEnv}|\marg{\AmS\
%    env}\marg{arabic num}\texttt{:}
%    \item[]|\def\EQ_modify_linewidth_|\meta{roman num}\texttt{:}\\
%    \hspace*{2em}\marg{Optionally alter \cs{l_EQ_temp_linewidth_dim}}
%    \item[]|\def\find_displaywidth_|\meta{roman num}\texttt{:}\\
%    \hspace*{2em}\marg{Set \cs{l_EQ_displaywidth_dim}}
%    \item[]|\def\EQ_place_display_|\meta{roman num}\texttt{:}\\
%    \hspace*{2em}\marg{Make use of \cs{EQ_use_mathdisplay:}}
%  \end{list}
%  \begin{macro}{\EQ_use_mathdisplay:}
%  A shorthand.
%    \begin{macrocode}
\def\EQ_use_mathdisplay:{%
  \EQ_outerbox:{\usebox{\EQ_mathdisplay_box}}}
%    \end{macrocode}
%  \end{macro}
%    \begin{macrocode}
\def\EQ_options_name:{EmphEqOpt}
%    \end{macrocode}
%  \subsection{The \env{equation} Family}
%  Just a few aliases because \env{equation} and \env{equation*}
%  behave strangely. Much better to use \env{gather} instead as this
%  is easier to measure. Besides it's less work (for us, not the
%  compiler). ;-)
%    \begin{macrocode}
\MH_keyval_alias_with_addon:nnnn {EmphEqEnv}{equation}{gather}
  {\EQ_disable_linebreak:}
\MH_keyval_alias_with_addon:nnnn {EmphEqEnv}{equation*}{gather*}
  {\EQ_disable_linebreak:}
%    \end{macrocode}
%  \begin{macro}{\EQ_disable_linebreak:}
%  \changes{v1.05c}{2004/02/25}{Introduced
%  \cs{EQ_disable_linebreak:} to prevent multi line
%  \env{equation}'s}
%  \changes{v1.05c}{2004/02/25}{Fixed it so that it works better}
%  Disable the line breaks in \env{equation} as these are actually
%  possible now, since we're in a \env{gather}! \cs{Let@} does it in
%  the environment, so that's the place to hack.
%    \begin{macrocode}
\def\EQ_disable_linebreak:{%
  \MH_let:NwN \EQ_saved_Let: \Let@
  \def\Let@{\def\\{%
    \PackageError{empheq}{You~ cannot~ use~ \string\\~ in~ equation!}
      {\@eha}}
%    \end{macrocode}
%  Then restore the meaning of \cs{Let@} in case there is a
%  \env{gathered} or alike coming up.
%    \begin{macrocode}
    \MH_let:NwN \Let@ \EQ_saved_Let:
  }%
}
%    \end{macrocode}
%  \end{macro}
%
%  \subsection{The \env{align} Family}
%    \begin{macrocode}
\EQ_new_equationtype:nnn {EmphEqEnv}{align}{1}
\EQ_new_equationtype:nnn {EmphEqEnv}{align*}{2}
%    \end{macrocode}
%    \begin{macrocode}
\def\EQ_modify_linewidth_i:{%
%    \end{macrocode}
%  If $\cs{maxfields@}=2$ we're in a single column \env{align} and
%  thus there is no flexible space.
%    \begin{macrocode}
  \if_num:w \maxfields@=\tw@
  \else:
%    \end{macrocode}
%  But if there is any flexible space we would very much like to
%  adjust it because we add material to the right and left of the
%  math display.
%    \begin{macrocode}
    \setlength{\l_EQ_temp_linewidth_dim}{%
      \linewidth-\l_EQ_totalwidth_dim-\widthof{\EQ_outerbox:{}}}%
  \fi:}
\def\EQ_modify_linewidth_ii:{\EQ_modify_linewidth_i:}
\def\find_displaywidth_i:{%
  \MH_setlength:dn \l_EQ_displaywidth_dim
    {\l_EQ_temp_linewidth_dim-\alignsep@}
  \if_case:w \l_EQ_alignmentmode_int
  \or: % #1:
    \MH_addtolength:dn \l_EQ_displaywidth_dim{-\alignsep@}%
  \or: % #2:
    \MH_addtolength:dn \l_EQ_displaywidth_dim{-\alignsep@}%
  \or: % #3:
    \MH_addtolength:dn \l_EQ_displaywidth_dim{-\@mathmargin}%
  \or: % #4:
    \MH_addtolength:dn \l_EQ_displaywidth_dim{-\@mathmargin}%
  \fi:}
\def\find_displaywidth_ii:{\find_displaywidth_i:}
\def\EQ_place_display_i: {\EQ_use_mathdisplay:}
\def\EQ_place_display_ii:{\EQ_place_display_i:}
%    \end{macrocode}
%
%  \subsection{The \env{gather} Family}
%  This is a gem. Notice how easy it is once we got the framework for
%  it.
%    \begin{macrocode}
\EQ_new_equationtype:nnn {EmphEqEnv}{gather}{3}
\EQ_new_equationtype:nnn {EmphEqEnv}{gather*}{4}
\def\EQ_modify_linewidth_iii:{}
\def\EQ_modify_linewidth_iv:{\EQ_modify_linewidth_iii:}
\def\find_displaywidth_iii:{%
  \MH_setlength:dn \l_EQ_displaywidth_dim{\totwidth@}%
  \if_case:w \l_EQ_alignmentmode_int \or: \or: % #1 & #2:
  \or: % #3:
    \MH_addtolength:dn \l_EQ_displaywidth_dim{-\@mathmargin}%
  \or: % #4:
    \MH_addtolength:dn \l_EQ_displaywidth_dim{-\@mathmargin}%
  \fi:}
\def\find_displaywidth_iv:{\find_displaywidth_iii:}
\def\EQ_place_display_iii:{\EQ_use_mathdisplay:}
\def\EQ_place_display_iv: {\EQ_place_display_iii:}
%    \end{macrocode}
%
%  \subsection{The \env{flalign} Family}\label{sec:flalign+multline}
%  This was fairly easy, but then I had to take care of \ldots
%
%  \subsubsection{The \env{multline} Family}
%  This was quite a pickle. Its implementation in \pkg{amsmath} is
%  very strange indeed; for instance the tag is typeset in the
%  measuring phase! So instead I thought I could fake it, but how
%  should I do it?
%
%  The basic idea of \env{multline} is to typeset the first line
%  flush left, the last line flush right and the rest centered.
%  ``Aha,'' I thought, ``what if I make three columns, fool \LaTeX{}
%  into thinking the contents of the columns take no space and then
%  just put each line into a column?'' Thus  the \env{flalign}
%  environment came to mind. I also distinctly remembered to have
%  seen an article about the equivalent of \cs{rlap} and \cs{llap} in
%  math mode, namely an article by Alexander R.~Perlis. This works
%  pretty well, and so does the same trick for producing a
%  \env{multlined} environment. That one uses an \env{alignedat} with
%  three columns.
%
%  \changes{v1.05c}{2004/02/25}{Makes use of
%  \cs{MakeKeyvalAliasAddon} now}
%  Then we make \env{multline} into an alias for {flalign} but with
%  an extra switch \cs{ifEQ_multline:} set to true.
%    \begin{macrocode}
\MH_keyval_alias_with_addon:nnnn {EmphEqEnv}{multline}
  {flalign}{\MH_set_boolean_T:n {multline}}
\MH_keyval_alias_with_addon:nnnn {EmphEqEnv}{multline*}
  {flalign*}{\MH_set_boolean_T:n {multline}}
%    \end{macrocode}
%  \begin{macro}{\if_EQ_multline:}
%  \begin{macro}{\EQ_generic_multline:nnnnn}
%  We make a general command that'll provide the correct spacing and
%  then just define the user commands in terms of that single generic
%  one.
%    \begin{macrocode}
\MH_new_boolean:n {multline}
%    \end{macrocode}
%  \begin{macro}{\EQ_multline_modify_linewidth_tag:}
%  \begin{macro}{\EQ_multline_modify_linewidth_notag:}
%  There is a big difference between the spacing in \env{multline}
%  depending on the presence of a tag. Therefore we split it up in
%  two cases. I made this choice because we need these adjustments
%  inside the adjustments for \env{flalign} and that makes it
%  incredibly difficult to see what happens.
%    \begin{macrocode}
\def\EQ_multline_modify_linewidth_tag:{%
  \if_case:w \l_EQ_alignmentmode_int
  \or: % #1:
    \MH_addtolength:dn \l_EQ_temp_linewidth_dim{%
      -\g_EQ_widesttag_dim-\multlinetaggap-\multlinegap}%
  \or:
    \MH_addtolength:dn \l_EQ_temp_linewidth_dim{%
      -\g_EQ_widesttag_dim-\multlinetaggap-\multlinegap}%
  \or:
    \MH_addtolength:dn \l_EQ_temp_linewidth_dim{%
      -\@mathmargin-\g_EQ_widesttag_dim-\multlinetaggap}%
  \or:
    \MH_addtolength:dn
      \l_EQ_temp_linewidth_dim{-\@mathmargin-\multlinegap}%
  \fi:
}
\def\EQ_multline_modify_linewidth_notag:{%
  \if_case:w \l_EQ_alignmentmode_int\or:
    \MH_addtolength:dn \l_EQ_temp_linewidth_dim{-2\multlinegap}
  \or:
    \MH_addtolength:dn \l_EQ_temp_linewidth_dim{-2\multlinegap}
  \or:
    \MH_addtolength:dn
      \l_EQ_temp_linewidth_dim{-\@mathmargin-\multlinegap}%
  \or:
    \MH_addtolength:dn
      \l_EQ_temp_linewidth_dim{-\@mathmargin-\multlinegap}%
  \fi:
}
%    \end{macrocode}
%  \end{macro}
%  \end{macro}
%
%  \begin{environment}{MTmultlined}
%  Save the superior \env{multlined} from \pkg{mathtools} under a new
%  name.
%    \begin{macrocode}
\MH_let:NwN \MTmultlined \multlined
\MH_let:NwN \endMTmultlined \endmultlined
%    \end{macrocode}
%  \end{environment}
%  We make a choice here: the good \env{multline} or the bad?
%    \begin{macrocode}
\MH_if_boolean:nTF {good_mult}{
%    \end{macrocode}
%  The good:
%    \begin{macrocode}
\MH_keyval_alias_with_addon:nnnn {EmphEqEnv}{multline}{gather}
  {\MH_set_boolean_T:n {outer_mult}}
\MH_keyval_alias_with_addon:nnnn {EmphEqEnv}{multline*}{gather*}
  {\MH_set_boolean_T:n {outer_mult}}
\def\EQ_modify_linewidth_iii_mult:{
  \setlength{\l_EQ_temp_linewidth_dim}{%
    \linewidth-\widthof{\EQ_outerbox:{}}}%
  \if_dim:w \g_EQ_widesttag_dim>\z@
    \EQ_multline_modify_linewidth_tag:
  \else:
    \EQ_multline_modify_linewidth_notag:
  \fi:
  \MH_setlength:dn \l_MT_multwidth_dim{\l_EQ_temp_linewidth_dim}
}
\def\EQ_modify_linewidth_iv_mult:{\EQ_modify_linewidth_iii_mult:}
\def\find_displaywidth_iii_mult:{%
  \MH_setlength:dn \l_EQ_displaywidth_dim{\totwidth@}%
  \if_case:w \l_EQ_alignmentmode_int \or: \or: % #1 & #2:
  \or: % #3:
    \MH_addtolength:dn \l_EQ_displaywidth_dim{-\@mathmargin}%
  \or: % #4:
    \MH_addtolength:dn \l_EQ_displaywidth_dim{-\@mathmargin}%
  \fi:}
\def\find_displaywidth_iv_mult:{\find_displaywidth_iii_mult:}
\def\EQ_place_display_iii_mult:{
  \if_case:w \l_EQ_alignmentmode_int
  \or: % #1:
    \if_dim:w \g_EQ_widesttag_dim>\z@
      \abovedisplayshortskip=\abovedisplayskip
      \rlap{
        \kern-.5\linewidth
        \kern\multlinegap
        \EQ_use_mathdisplay:
      }
    \else:
      \EQ_use_mathdisplay:
    \fi:
  \or: % #2:
    \if_dim:w \g_EQ_widesttag_dim>\z@
      \llap{
        \EQ_use_mathdisplay:
        \kern-.5\linewidth
        \kern\multlinegap
      }%
    \else:
      \EQ_use_mathdisplay:
    \fi:
  \or: % #3:
    \EQ_use_mathdisplay:
  \or: % #4:
    \EQ_use_mathdisplay:
  \fi:
}
\def\EQ_place_display_iv_mult: {\EQ_place_display_iii_mult:}
%    \end{macrocode}
%    \begin{macrocode}
}{
%    \end{macrocode}
%  The bad
%    \begin{macrocode}
\def\EQ_generic_multline:nnnnn #1#2#3#4#5{%
%    \end{macrocode}
%  This is a very tricky little detail. If we're in \mode{leqno} mode
%  we need just a we little width inside the \env{flalign}, so we put
%  in the smallest possible amount. Otherwise we wind up in a
%  \begin{verbatim}
%  ! Arithmetic overflow.
%  <recently read> \count@
%
%  l.304 %  \end{empheq}
%  \end{verbatim}
%  If you scroll past this error it will be typeset correctly, but
%  how many users can fix this by themselves?
%    \begin{macrocode}
  \hskip1sp#2#3{{}#5}#4%
  \if:w *#1\relax\else:\notag\fi:}
%    \end{macrocode}
%  \end{macro}
%  \end{macro}
%  \begin{macro}{\multleft}
%  \begin{macro}{\multcenter}
%  \begin{macro}{\multright}
%  Then the commands that use \cs{EQ_generic_multline:nnnnn}. The
%  first three are the general purpose ones that also have a starred
%  version. The other three are for normal use.
%    \begin{macrocode}
\def\multleft{\@ifstar{%
  \MH_if_boolean:nTF {multlined}{
    \def\EQ_next:{\EQ_multlined_left:n}
  }{
    \def\EQ_next:{
      \EQ_generic_multline:nnnnn {*}{}{\mathrlap}{&&&&&}}
  }
  \EQ_next:}{%
  \MH_if_boolean:nTF {multlined}{
    \def\EQ_next:{\EQ_multlined_left:n}%
  }{
    \def\EQ_next:{
      \EQ_generic_multline:nnnnn {}{}{\mathrlap}{&&&&&}}
  }
  \EQ_next:}}%
\def\multcenter{\@ifstar{%
  \MH_if_boolean:nTF {multlined}{
    \def\EQ_next:{\EQ_multlined_center:n}%
  }{
    \def\EQ_next:{
      \EQ_generic_multline:nnnnn {*}{&&}{\mathclap}{&&&}}
  }
  \EQ_next:}{%
  \MH_if_boolean:nTF {multlined}{
    \def\EQ_next:{\EQ_multlined_center:n }%
  }{
    \def\EQ_next:{
      \EQ_generic_multline:nnnnn {}{&&}{\mathclap}{&&&}}
  }
  \EQ_next:}}%
\def\multright{\@ifstar{%
  \MH_if_boolean:nTF {multlined}{
    \def\EQ_next:{\EQ_multlined_right:n }%
  }{
    \def\EQ_next:{
      \EQ_generic_multline:nnnnn {*}{&&&&&}{\mathllap}{}}
  }
  \EQ_next:}{%
  \MH_if_boolean:nTF {multlined}{
    \def\EQ_next:{\EQ_multlined_right:n }%
  }{
    \def\EQ_next:{
      \EQ_generic_multline:nnnnn {}{&&&&&}{\mathllap}{}}
  }
  \EQ_next:}}
%    \end{macrocode}
%  \end{macro}
%  \end{macro}
%  \end{macro}
%  \begin{macro}{\multfirst}
%  \begin{macro}{\multmiddle}
%  \begin{macro}{\multlast}
%  If we're using the \mode{leqno} option, the equation number should
%  be placed in the first line.
%    \begin{macrocode}
\def\multfirst{%
  \MH_if_boolean:nTF {multlined}{
     \def\EQ_next:{\EQ_multlined_left:n }%
  }{
    \if_case:w \l_EQ_alignmentmode_int
    \or: % #1:
      \def\EQ_next:{\multleft}%
    \or: % #2:
      \def\EQ_next:{\multleft*}%
    \or: % #3:
      \def\EQ_next:{\multleft}%
    \or: % #4:
      \def\EQ_next:{\multleft*}%
    \fi:
  }
  \EQ_next:}
%    \end{macrocode}
%  Pretend the star was never there.
%    \begin{macrocode}
\def\multmiddle{%
  \def\EQ_next:{%
    \MH_if_boolean:nTF {multlined}{
      \expandafter\EQ_multlined_center:n
    }{
      \expandafter\multcenter
    }
  }%
  \@ifstar{\EQ_next:}{\EQ_next:}}
%    \end{macrocode}
%  If we're  using the \mode{reqno} option, the equation number
%  should be placed in the last line.
%    \begin{macrocode}
\def\multlast{%
  \MH_if_boolean:nTF {multlined}{
     \def\EQ_next:{\EQ_multlined_right:n }%
  }{
    \if_case:w \l_EQ_alignmentmode_int
    \or: % #1:
      \def\EQ_next:{\multright*}%
    \or: % #2:
      \def\EQ_next:{\multright}%
    \or: % #3:
      \def\EQ_next:{\multright*}%
    \or: % #4:
      \def\EQ_next:{\multright}%
    \fi:
  }
  \EQ_next:}
%    \end{macrocode}
%  \end{macro}
%  \end{macro}
%  \end{macro}
%  \begin{macro}{\multlinedwidth}
%  Here begins the code for \env{multlined}.
%    \begin{macrocode}
\MH_new_boolean:n {multlined}
\newdimen\multlinedwidth
\MH_setlength:dn \multlinedwidth{.7\linewidth}
%    \end{macrocode}
%  \end{macro}
%  \begin{environment}{multlined}
%  \changes{v2.10}{2004/07/26}{Disallow spaces before optional
%  arguments.}
%  \begin{macro}{\EQ_multlined_I:w}
%  \begin{macro}{\EQ_multlined_II:w}
%  \begin{macro}{\EQ_multlined_III:nn}
%  Some trickery as we want to scan for two optional arguments that
%  may be interchanged.
%    \begin{macrocode}
\renewenvironment{multlined}%
  {\MH_nospace_ifnextchar:Nnn [  {\EQ_multlined_I:w}
                                 {\EQ_multlined_I:w[c]}
  }
  { \endaligned }
\def\EQ_multlined_I:w[#1]{%
  \MH_nospace_ifnextchar:Nnn[  {\EQ_multlined_II:w[#1]}
                               {\EQ_multlined_II:w[#1][\multlinedwidth]}}
\def\EQ_multlined_II:w[#1][#2]{%
%    \end{macrocode}
%  We tell \LaTeX\ that we're in \env{multlined}.
%    \begin{macrocode}
  \MH_set_boolean_T:n {multlined}
%    \end{macrocode}
%  \changes{v2.10}{2004/07/26}{Fixed restoration of \cmd{\\}}
%  Restore the meaning of \cmd{\\} inside \env{multlined}, else it
%  wouldn't work in the \env{equation} environment.
%    \begin{macrocode}
  \MH_let:NwN \Let@\MT_saved_Let:
%    \end{macrocode}
%  Then check the first optional argument and call
%  \cs{EQ_multlined_III:w}, which sets \cs{multlinewidth} if
%  specified and calls \cs{start@aligned} (an \env{alignedat} with
%  three columns).
%    \begin{macrocode}
  \if:w t#1\relax \EQ_multlined_III:nn {#1}{#2}%
  \else:
    \if:w b#1\relax \EQ_multlined_III:nn {#1}{#2}%
    \else:
      \if:w c#1\relax \EQ_multlined_III:nn {#1}{#2}%
      \else:
        \EQ_multlined_III:nn {#2}{#1}%
      \fi:
    \fi:
  \fi:
  \mkern-\thinmuskip
  \EQ_next:}
\def\EQ_multlined_III:nn #1#2{%
  \def\EQ_next:{\setlength\multlinedwidth{#2}%
                          \start@aligned{#1}{\thr@@}}}
%    \end{macrocode}
%  \end{macro}
%  \end{macro}
%  \end{macro}
%  \end{environment}
%  \begin{macro}{\EQ_multlined_left:n}
%  \begin{macro}{\EQ_multlined_center:n}
%  \begin{macro}{\EQ_multlined_right:n}
%  The internal versions of \cs{multleft} etc.\ adapted for
%  \env{multlined}.
%    \begin{macrocode}
\def\EQ_multlined_left:n #1{%
  \mathrlap{#1}&\hskip.5\multlinedwidth&&\hskip.5\multlinedwidth&&}
\def\EQ_multlined_center:n #1{%
  &\hskip.5\multlinedwidth&&\mathclap{#1}\hskip.5\multlinedwidth&&}
\def\EQ_multlined_right:n #1{%
  &\hskip.5\multlinedwidth&&\hskip.5\multlinedwidth&&\mathllap{#1}}
%    \end{macrocode}
%  \end{macro}
%  \end{macro}
%  \end{macro}
%
%  End of bad multline.
%    \begin{macrocode}
}
%    \end{macrocode}
%
%  \subsubsection{The Code for \env{flalign} \& \env{multline}}
%    \begin{macrocode}
\EQ_new_equationtype:nnn {EmphEqEnv}{flalign}{5}
\EQ_new_equationtype:nnn {EmphEqEnv}{flalign*}{6}
\def\EQ_modify_linewidth_v:{%
  \setlength{\l_EQ_temp_linewidth_dim}{%
    \linewidth-\l_EQ_totalwidth_dim-\widthof{\EQ_outerbox:{}}}%
  \if_dim:w \g_EQ_widesttag_dim>\z@
    \MH_if_boolean:nTF {multline}{
      \EQ_multline_modify_linewidth_tag:
    }{
      \if_case:w \l_EQ_alignmentmode_int
      \or: % #1:
        \MH_addtolength:dn  \l_EQ_temp_linewidth_dim
          {-\g_EQ_widesttag_dim-\mintagsep}
      \or: % #2:
        \MH_addtolength:dn \l_EQ_temp_linewidth_dim
          {-\g_EQ_widesttag_dim-\mintagsep}%
      \or: % #3:
        \MH_addtolength:dn \l_EQ_temp_linewidth_dim
          {-\g_EQ_widesttag_dim-\mintagsep}
      \or: % #4:
        \MH_addtolength:dn \l_EQ_temp_linewidth_dim
          {-\@mathmargin}
      \fi:
    }
  \else:
    \MH_if_boolean:nT {multline}{
      \EQ_multline_modify_linewidth_notag:
    }
  \fi:}
\def\EQ_modify_linewidth_vi:{\EQ_modify_linewidth_v:}
\def\find_displaywidth_v:{%
  \MH_setlength:dn
    \l_EQ_displaywidth_dim{\l_EQ_temp_linewidth_dim}}
\def\find_displaywidth_vi:{\find_displaywidth_v:}
\def\EQ_place_display_v:{%
  \if_case:w \l_EQ_alignmentmode_int
  \or: % #1:
    \if_dim:w \g_EQ_widesttag_dim>\z@
      \MH_if_boolean:nTF {multline}{
        \rlap{\kern\multlinegap  \kern-.5\linewidth
          \EQ_use_mathdisplay:
          \kern-\mintagsep  \kern\multlinetaggap}%
      }{
        \rlap{\kern-.5\linewidth\EQ_use_mathdisplay:}%
      }
    \else:
      \EQ_use_mathdisplay:
    \fi:
  \or: % #2:
    \if_dim:w \g_EQ_widesttag_dim>\z@
      \MH_if_boolean:nTF {multline}{
        \llap{\kern-\mintagsep \kern\multlinetaggap
          \EQ_use_mathdisplay:
          \kern\multlinegap  \kern-.5\linewidth}%
      }{
        \llap{\EQ_use_mathdisplay:\kern-.5\linewidth}%
      }
    \else:
      \EQ_use_mathdisplay:
    \fi:
  \or: % #3:
    \MH_if_boolean:nF {multline}{ \kern-\multlinetaggap }
    \EQ_use_mathdisplay:
  \or: % #4:
    \if_dim:w \g_EQ_widesttag_dim>\z@
    \else:
      \MH_if_boolean:nF {multline}{ \kern-\@mathmargin }
    \fi:
    \EQ_use_mathdisplay:
  \fi:
  \MH_set_boolean_F:n {multline}}
\def\EQ_place_display_vi:{\EQ_place_display_v:}
%    \end{macrocode}
%
%  \subsection{The \env{alignat} Family}
%  This is just as easy as \env{gather}. Actually only one line is
%  different from \env{gather}!
%    \begin{macrocode}
\EQ_new_equationtype_arg:nnn {EmphEqEnv}{alignat}{7}
\EQ_new_equationtype_arg:nnn {EmphEqEnv}{alignat*}{8}
\def\EQ_modify_linewidth_vii:{}
\def\EQ_modify_linewidth_viii:{\EQ_modify_linewidth_vii:}
\def\find_displaywidth_vii:{%
  \MH_setlength:dn \l_EQ_displaywidth_dim{\totwidth@}%
  \if_case:w \l_EQ_alignmentmode_int
  \or: % #1:
  \or: % #2:
  \or: % #3:
    \MH_addtolength:dn \l_EQ_displaywidth_dim{-\eqnshift@}%
  \or: % #4:
    \MH_addtolength:dn \l_EQ_displaywidth_dim{-\@mathmargin}%
  \fi:}
\def\find_displaywidth_viii:{\find_displaywidth_vii:}
\def\EQ_place_display_vii:{\EQ_use_mathdisplay:}
\def\EQ_place_display_viii:{\EQ_place_display_vii:}
%    \end{macrocode}
%  \section{The User Interface}
%  The different keys.
%    \begin{macrocode}
\define@key{\EQ_options_name:}{box}{\def\EQ_outerbox:{#1}}
\MH_keyval_alias:nnn {\EQ_options_name:}{outerbox}{box}
%    \end{macrocode}
%  \begin{macro}{\EQ_marginbox:nn}
%  The margin box. Uwe Siart wanted it.
%    \begin{macrocode}
\def\EQ_marginbox:nn #1#2{%
  \settowidth{\@tempdima}{#2}%
  \makebox[\z@]{\hspace{\@tempdima}#1{#2}}}
\define@key{\EQ_options_name:}{marginbox}{%
  \def\EQ_outerbox:{\EQ_marginbox:nn {#1}}}
%    \end{macrocode}
%  \end{macro}
%  Almost self-explanatory, isn't?
%    \begin{macrocode}
\define@key{\EQ_options_name:}{innerbox}{\def\EQ_innerbox:{#1}}
\define@key{\EQ_options_name:}{left}{\def\EQ_left_side:{#1}}
\define@key{\EQ_options_name:}{right}{\def\EQ_right_side:{#1}}
%    \end{macrocode}
%  \begin{macro}{\empheqset}
%  Just a user friendly way to set keys. Uwe Siart again.
%    \begin{macrocode}
\newcommand*\empheqset[1]{\setkeys{\EQ_options_name:}{#1}}
%    \end{macrocode}
%  \end{macro}
%  The old keys from the original version are no longer supported.
%  Instead they give an error message (RM):
%    \begin{macrocode}
\def\EQ_upgrade_error:{%
  \PackageError{empheq}
    {You~ are~ using~ the~ old~ syntax!\MessageBreak
    The~ `empheq'~ environment~ takes~ a~ mandatory~
    argument~ now.\MessageBreak
    You~ need~ to~ exit~ and~ change~ your~ source~ file.}
    {It~ won't~ work,~ trust me.~ Press~ `X'~ now.}}
\define@key{\EQ_options_name:}{boxtype}{\EQ_upgrade_error:}
\define@key{\EQ_options_name:}{Left}{\EQ_upgrade_error:}
\define@key{\EQ_options_name:}{Right}{\EQ_upgrade_error:}
%    \end{macrocode}
% \begin{macro}{\empheqeqrefbase}
%   This macro is used to locally redefine \cs{eqref}, it will just
%   point at \cs{textup}, the user can then redefine it if nessary.
%    \begin{macrocode}
\let\empheqeqrefbase\textup
%    \end{macrocode}
%   
% \end{macro}
%  \begin{environment}{empheq}
%  The environment.
%    \begin{macrocode}
\newenvironment{empheq}[2][]{%
%    \end{macrocode}
% \begin{macro}{\EQ_saved_maketag}
% \changes{v2.13}{2013/02/11}{Fixing \cs{eqref}}
% We start with a small caveat: During processing \cs{maketag@@@} is
% redefined, enabling us to separate math and eqn numbers. But the
% user may have used \cs{eqref} in a comment in the math, and
% \cs{eqref} use \cs{maketag@@@}\dots{} First save the original \cs{maketag@@@}:
%    \begin{macrocode}
  \MH_let:NwN\EQ_saved_maketag: \maketag@@@%
%    \end{macrocode}
% Next redefine \cs{eqref} to locally bring back the original:
%    \begin{macrocode}
  \renewcommand\eqref[1]{\empheqeqrefbase{%
       \let\maketag@@@\EQ_saved_maketag:%
       \tagform@{\ref{##1}}}}
  \setkeys{EmphEqEnv}{#2}\setkeys{\EQ_options_name:}{#1}%
  \EmphEqMainEnv}{\endEmphEqMainEnv}
%    \end{macrocode}
%  \end{macro}
% \end{environment}
%  \begin{macro}{\DeclareLeftDelimiter}
%  \begin{macro}{\EQ_declare_left_delimiter:nnnn}
%  \begin{macro}{\DeclareRightDelimiter}
%  \begin{macro}{\EQ_declare_right_delimiter:nnnn}
%  The delimiter interface.
%  \changes{v0.7}{2003/05/11}{Introduced delimiter interface}
%  \changes{v1.05d}{2004/02/26}{Changed the definitions.}
%  We now use the \cs{EQ_tag_box} to calculate the height but
%  adjust the vertical spacing. All this to prevent running
%  \cs{EQ_typeset_display:} twice.
%    \begin{macrocode}
\newcommand*\DeclareLeftDelimiter[2][]{%
  \EQ_declare_left_delimiter:nnnn {#1}{#2}{}{}%
  \EQ_declare_left_delimiter:nnnn {#1}{#2}{big}{\EQ_innerbox:}}
\def\EQ_declare_left_delimiter:nnnn #1#2#3#4{%
  \@namedef{empheq#3\expandafter\@gobble\string#2}{%
    \left#2%
    \mkern-\thinmuskip
    \vphantom{#4{\parbox{.5\maxdimen}{%
      \EQ_predisplay_adjustment:
      \EQ_prevent_vertical_space:
      \usebox{\EQ_tag_box}%
      \EQ_postdisplay_adjustment:}}}%
    \right.%
    \kern-\nulldelimiterspace #1}}
\newcommand*\DeclareRightDelimiter[2][]{%
  \EQ_declare_right_delimiter:nnnn {#1}{#2}{}{}%
  \EQ_declare_right_delimiter:nnnn {#1}{#2}{big}{\EQ_innerbox:}}
\def\EQ_declare_right_delimiter:nnnn #1#2#3#4{%
  \@namedef{empheq#3\expandafter\@gobble\string#2}{%
    \kern-\nulldelimiterspace
    #1%
    \left.%
    \vphantom{#4{\parbox{.5\maxdimen}{%
      \EQ_predisplay_adjustment:
      \EQ_prevent_vertical_space:
      \usebox{\EQ_tag_box}%
      \EQ_postdisplay_adjustment:}}}%
    \mkern-\thinmuskip
    \right#2}}
%    \end{macrocode}
%  \end{macro}
%  \end{macro}
%  \end{macro}
%  \end{macro}
%  The actual declarations.
%    \begin{macrocode}
\DeclareLeftDelimiter{\lbrace}
\DeclareRightDelimiter{\rbrace}
\DeclareLeftDelimiter{\lbrack}
\DeclareRightDelimiter{\rbrack}
\DeclareLeftDelimiter{\langle}
\DeclareRightDelimiter{\rangle}
\DeclareLeftDelimiter{\lvert}
\DeclareRightDelimiter{\rvert}
\DeclareLeftDelimiter{\lVert}
\DeclareRightDelimiter{\rVert}
\DeclareLeftDelimiter{\lfloor}
\DeclareRightDelimiter{\rfloor}
\DeclareLeftDelimiter{\lceil}
\DeclareRightDelimiter{\rceil}
\DeclareLeftDelimiter{\lparen}
\DeclareRightDelimiter{\rparen}
%    \end{macrocode}
%
%  \section{Support for Other Packages}
%
%  Support for various other packages is found in this section. The
%  \pkg{ntheorem} package now has a dedicated section (\S
%  \vref{sec:impl:ntheorem}), as it now a module.
%
%
%
%    \begin{macrocode}
\def\EQ_typeset_tags_hook:{}
\def\EQ_typeset_display_hook:{}
%    \end{macrocode}
%
%  \subsection{\pkg{showkeys}}
%  \changes{v2.10}{2004/07/26}{Added support for the \pkg{showkeys}
%  package}
%  We just disable \pkg{showkeys} when typesetting the labels.
%    \begin{macrocode}
\g@addto@macro\EQ_typeset_display_hook:{
  \def\SK@@label#1>#2\SK@{}
}
%    \end{macrocode}
%  If we overload the \AmS{} equation types we should still be able
%  to use \pkg{showkeys}.
%    \begin{macrocode}
\@namedef{SK@AmSequation}  {\SK@equationtrue}
\@namedef{SK@AmSequation*} {\SK@equationtrue}
\@namedef{SK@AmSalign}     {\SK@equationtrue}
\@namedef{SK@AmSalign*}    {\SK@equationtrue}
\@namedef{SK@AmSalignat}   {\SK@equationtrue}
\@namedef{SK@AmSalignat*}  {\SK@equationtrue}
\@namedef{SK@AmSgather}    {\SK@equationtrue}
\@namedef{SK@AmSgather*}   {\SK@equationtrue}
\@namedef{SK@AmSmultline}  {\SK@equationtrue}
\@namedef{SK@AmSmultline*} {\SK@equationtrue}
\@namedef{SK@AmSflalign}   {\SK@equationtrue}
\@namedef{SK@AmSflalign*}  {\SK@equationtrue}
%    \end{macrocode}
%
%  \subsection{\pkg{hyperref}}
%
%
%  A little trick to avoid the (in)famous warning
%  \begin{verbatim}
%  ! pdfTeX warning (ext4): destination with the same identifier
%  (name{equation.1a }) has been already used, duplicate ignored
%  <to be read again>
%                     \endgroup \set@typeset@protect
%  \end{verbatim}
%  This doesn't require \pkg{hyperref}.
%    \begin{macrocode}
\AtBeginDocument{
  \@ifundefined{Hy@linkcounter}{\newcount\Hy@linkcounter}{}
}
\g@addto@macro\EQ_typeset_display_hook:{
  \MH_let:NwN \hyper@refstepcounter\@gobble
}
%    \end{macrocode}
% 
% \subsection{\pkg{nccmath}}
%
% Must disable this macro, otherwise the spacing is wrong.
%    \begin{macrocode}
 \g@addto@macro\EQ_typeset_display_hook:{
   \let\NCC@ignorepar\relax
 }
 \g@addto@macro\EQ_typeset_tags_hook:{
   \let\NCC@ignorepar\relax
 }
%    \end{macrocode}
%
%  \subsection{Basic \pkg{ntheorem} support}
%    \begin{macrocode}
\g@addto@macro\EQ_typeset_tags_hook:{
  \MH_let:NwN\TagsPlusEndmarks\relax
  \MH_let:NwN\RestoreTags\relax
}
\g@addto@macro\EQ_typeset_display_hook:{
  \MH_let:NwN\TagsPlusEndmarks\relax
  \MH_let:NwN\RestoreTags\relax
}
%    \end{macrocode}
%  \subsection{Basic \LaTeXe{} support}
%  Gobble the \cs{ltx@label} and disable most error messages.
%  \changes{v2.00}{2004/04/14}{Make a special label-gobbling command
%  to support \pkg{ntheorem} even more}
%    \begin{macrocode}
\g@addto@macro\EQ_typeset_display_hook:{
  \MH_let:NwN\ltx@label\MT_gobblelabel:w
  \MH_let:NwN \GenericError \@gobblefour
}
%    \end{macrode}
%  \subsection{Support for \pkg{fancybox}}
%  First check if \pkg{fancybox} is loaded and then see if
%  \pkg{color} is loaded. If the latter is loaded we define
%  \texttt{shadowcolor} else we let \cs{textcolor} gobble the first
%  argument (the color).
%    \begin{macrocode}
\AtBeginDocument{%
  \@ifpackageloaded{fancybox}{%
    \@ifpackageloaded{color}{\definecolor{shadowcolor}{rgb}{0,0,0}}%
    {\newcommand*\textcolor[2]{\normalcolor{#2}}}%
%    \end{macrocode}
%  \begin{macro}{\shadowbox}
%  Then define the \cs{shadowbox}. Use a conditional to let us know if
%  it was starred or not.
%    \begin{macrocode}
    \MH_new_boolean:n {shadowbox_star}
    \renewcommand*\shadowbox{%
      \@ifstar{\MH_set_boolean_T:n {shadowbox_star}
               \VerbBox\EQ_shadowbox:n}
              {\MH_set_boolean_F:n {shadowbox_star}
               \VerbBox\EQ_shadowbox:n}}
%    \end{macrocode}
%  \end{macro}
%  \begin{macro}{\EQ_shadowbox:n}
%  The internal version of \cs{shadowbox}.
%    \begin{macrocode}
    \def\EQ_shadowbox:n #1{%
      \sbox{\z@}{\fbox{#1}}%
%    \end{macrocode}
%  The difference between the two versions is that \cs{shadowbox}
%  aligns the bottom of the shadow at the baseline whereas
%  \cs{shadowbox*} aligns the argument on the baseline, so for the
%  starred version the trick is to raise the box by the depth of the
%  argument plus the size of the shadow.
%    \begin{macrocode}
      \MH_if_boolean:nTF {shadowbox_star}{
        \MH_setlength:dn \@tempdima{\z@}%
      }{
        \MH_setlength:dn \@tempdima{\dp\z@+\shadowsize}%
      }
      \raisebox{\@tempdima}{%
        \makebox[\z@][l]{%
          \kern\shadowsize \textcolor{shadowcolor}{%
            \rule[-\dp\z@-\shadowsize]{\wd\z@}{\shadowsize}}%
        }%
        \usebox\z@\textcolor{shadowcolor}{%
          \rule[-\dp\z@-\shadowsize]{\shadowsize}{\ht\z@+\dp\z@}}}}%
  }{}%
}
%    \end{macrocode}
%  \end{macro}
%
%
%  \section{The \opt{overload} option}
%  \begin{macro}{\EQ_overload_ams_begin:n}
%  A helper for saving snapshots:
%    \begin{macrocode}
\def\EQ_overload_ams_begin:n #1{%
  \MH_let:cc {AmS#1}{#1}%
  \MH_let:cc {AmS#1*}{#1*}
}
%    \end{macrocode}
%  \end{macro}
%  \begin{macro}{\EQ_overload_ams_end:n}
%  All but one of the aligned variations must be set to the original
%    \begin{macrocode}
\def\EQ_overload_ams_end:n #1{%
  \MH_let:cc {endAmS#1}{endAmSalign}
}
%    \end{macrocode}
%  \end{macro}
%  \begin{environment}{AmSequation}
%  \begin{environment}{AmSequation*}
%  \changes{v2.10}{2004/07/26}{Moved redefinitions to
%  \cs{AtBeginDocument} to give \pkg{hyperref} a chance}
%  Before we start redefining environments, we'll need a basic
%  \env{equation} environment that is not redefined.
%    \begin{macrocode}
\AtBeginDocument{
  \EQ_overload_ams_begin:n {equation}
  \MH_let:cc {endAmSequation}{endequation}
  \MH_let:cc {endAmSequation*}{endequation*}
%    \end{macrocode}
%  \end{environment}
%  \end{environment}
%  Lets start overloading.
%    \begin{macrocode}
  \MH_if_boolean:nT {overload_ams}{
%    \end{macrocode}
%  \begin{macro}{\EQ_renew_ams_environment:n}
%  \changes{v2.10}{2004/07/26}{Disallow spaces before the optional
%  argument.}
%  Then we can renew the environments. Syntax is
%  \cs{begin}\marg{\AmS{} env}\oarg{markup}. Exception to this rule
%  is \env{alignat} (see further down).
%    \begin{macrocode}
    \def\EQ_renew_ams_environment:n #1{%
      \MaybeMHPrecedingSpacesOff
      \renewenvironment{#1}[1][]{%
%    \end{macrocode}
% \changes{v2.13}{2013/02/11}{Adding \cs{eqref} fix}
% Remember to add \cs{eqref} fix to the overloaded envs as well.
%    \begin{macrocode}
        \MH_let:NwN\EQ_saved_maketag: \maketag@@@%
        \renewcommand\eqref[1]{\empheqeqrefbase{%
            \let\maketag@@@\EQ_saved_maketag:%
            \tagform@{\ref{####1}}}}
        \setkeys{\EQ_options_name:}{##1}%
        \setkeys{EmphEqEnv}{AmS#1}%
        \EmphEqMainEnv
      }{\endEmphEqMainEnv}
      \MHPrecedingSpacesOn
    }
%    \end{macrocode}
%  \end{macro}
%  Add equation type: \emph{Must} have same number as original type!
%    \begin{macrocode}
    \EQ_new_equationtype:nnn     {EmphEqEnv}{AmSalign}   {1}
    \EQ_new_equationtype:nnn     {EmphEqEnv}{AmSalign*}  {2}
    \EQ_new_equationtype:nnn     {EmphEqEnv}{AmSgather}  {3}
    \EQ_new_equationtype:nnn     {EmphEqEnv}{AmSgather*} {4}
    \EQ_new_equationtype:nnn     {EmphEqEnv}{AmSflalign} {5}
    \EQ_new_equationtype:nnn     {EmphEqEnv}{AmSflalign*}{6}
    \EQ_new_equationtype_arg:nnn {EmphEqEnv}{AmSalignat} {7}
    \EQ_new_equationtype_arg:nnn {EmphEqEnv}{AmSalignat*}{8}
%    \end{macrocode}
%  Then overload by saving the snapshots of the old definitions. We
%  do it via \cs{AtBeginDocument}, because a package like
%  \pkg{ntheorem} redefines \cs{endgather} etc.
%
%  First the \env{gather} and \env{equation} families.
%    \begin{macrocode}
    \EQ_overload_ams_begin:n {gather}
    \MH_let:cc {endAmSgather}{endgather}
    \MH_let:cc {endAmSgather*}{endAmSgather}
    \EQ_renew_ams_environment:n {gather}
    \EQ_renew_ams_environment:n {gather*}
    \EQ_overload_ams_begin:n {equation}
    \MH_keyval_alias_with_addon:nnnn {EmphEqEnv} {AmSequation}
      {AmSgather}{\EQ_disable_linebreak:}
    \MH_keyval_alias_with_addon:nnnn {EmphEqEnv}{AmSequation*}
      {AmSgather*}{\EQ_disable_linebreak:}
    \EQ_renew_ams_environment:n {equation}
    \EQ_renew_ams_environment:n {equation*}
%    \end{macrocode}
%  The \env{align} family.
%    \begin{macrocode}
    \EQ_overload_ams_begin:n {align}
    \MH_let:cc {endAmSalign}{endalign}
    \EQ_overload_ams_end:n {align*}
    \EQ_renew_ams_environment:n {align}
    \EQ_renew_ams_environment:n {align*}
%    \end{macrocode}
%  The \env{flalign} family.
%    \begin{macrocode}
    \EQ_overload_ams_begin:n {flalign}
    \EQ_overload_ams_end:n {flalign}
    \EQ_overload_ams_end:n {flalign*}
    \EQ_renew_ams_environment:n {flalign}
    \EQ_renew_ams_environment:n {flalign*}
%    \end{macrocode}
%  The \env{alignat} family.
%    \begin{macrocode}
    \EQ_overload_ams_begin:n {alignat}
    \EQ_overload_ams_end:n {alignat}
    \EQ_overload_ams_end:n {alignat*}
%    \end{macrocode}
%  \begin{macro}{\EQ_alignat_optional:w}
%  \changes{v2.10}{2004/07/26}{Disallow spaces before the optional
%  argument.}
%  \begin{environment}{alignat}
%  \begin{environment}{alignat*}
%  This one is a little different because we want a syntax like
%  \cs{begin\{alignat\}}\linebreak[0]\marg{cols}\oarg{markup}, so we
%  have to make a separate command for testing the trailing optional
%  argument.
%    \begin{macrocode}
    \MaybeMHPrecedingSpacesOff
    \newcommand*\EQ_alignat_optional:w [1][]{
      \setkeys{\EQ_options_name:}{#1}
      \EmphEqMainEnv}
    \MHPrecedingSpacesOn
%    \end{macrocode}
% \changes{v2.13}{2013/02/11}{Adding \cs{eqref} fix}
% Remember to add \cs{eqref} fix to the overloaded envs as well.
%    \begin{macrocode}
    \renewenvironment{alignat}[1]{
      \MH_let:NwN\EQ_saved_maketag: \maketag@@@%
      \renewcommand\eqref[1]{\empheqeqrefbase{%
          \let\maketag@@@\EQ_saved_maketag:%
          \tagform@{\ref{##1}}}}
      \setkeys{EmphEqEnv}{AmSalignat=#1}
      \EQ_alignat_optional:w}
      {\endEmphEqMainEnv}
    \renewenvironment{alignat*}[1]{%
      \MH_let:NwN\EQ_saved_maketag: \maketag@@@%
      \renewcommand\eqref[1]{\empheqeqrefbase{%
          \let\maketag@@@\EQ_saved_maketag:%
          \tagform@{\ref{##1}}}}
      \setkeys{EmphEqEnv}{AmSalignat*=#1}%
      \EQ_alignat_optional:w}%
      {\endEmphEqMainEnv}
%    \end{macrocode}
%  \end{environment}
%  \end{environment}
%  \end{macro}
%  \begin{macro}{\[}
%  \begin{macro}{\]}
%  Then restore \cs{[} and \cs{]}.
%    \begin{macrocode}
    \DeclareRobustCommand{\[}{\begin{AmSequation*}}
    \DeclareRobustCommand{\]}{\end{AmSequation*}}
%    \end{macrocode}
%  \end{macro}
%  \end{macro}
%  Finally a string of aliases so that we can blend \opt{overload}
%  syntax with regular syntax.
%    \begin{macrocode}
    \MH_keyval_alias:nnn {EmphEqEnv}{gather}  {AmSgather}
    \MH_keyval_alias:nnn {EmphEqEnv}{gather*} {AmSgather*}
    \MH_keyval_alias:nnn {EmphEqEnv}{align}   {AmSalign}
    \MH_keyval_alias:nnn {EmphEqEnv}{align*}  {AmSalign*}
    \MH_keyval_alias:nnn {EmphEqEnv}{flalign} {AmSflalign}
    \MH_keyval_alias:nnn {EmphEqEnv}{flalign*}{AmSflalign*}
    \MH_keyval_alias:nnn {EmphEqEnv}{alignat} {AmSalignat}
    \MH_keyval_alias:nnn {EmphEqEnv}{alignat*}{AmSalignat*}
    \MH_if_boolean:nT {good_mult}{
      \EQ_overload_ams_begin:n {multline}
      \MH_let:cc {endAmSmultline}{endmultline}
      \MH_let:cc {endAmSmultline*}{endAmSmultline}
      \MH_keyval_alias_with_addon:nnnn {EmphEqEnv}{AmSmultline}
        {AmSgather}{\MH_set_boolean_T:n {outer_mult}}
      \MH_keyval_alias_with_addon:nnnn {EmphEqEnv}{AmSmultline*}
        {AmSgather*}{\MH_set_boolean_T:n {outer_mult}}
      \EQ_renew_ams_environment:n {multline}
      \EQ_renew_ams_environment:n {multline*}
    }
%    \end{macrocode}
%  End of \opt{overload}. We go into \opt{overload2}\ldots This
%  is dangerous, but will probably work. The macro \cs{[} is now
%  fragile except if you're running \eTeX{} as engine for \LaTeX.
%    \begin{macrocode}
    \MH_if_boolean:nT {overload_amsII}{
      \MH_protected:\def\[#1\]{\begin{equation*} #1\end{equation*}}
%    \end{macrocode}
%  Then we add the \env{multline} family.
%    \begin{macrocode}
      \MH_if_boolean:nF {good_mult}{
        \EQ_overload_ams_begin:n {multline}
        \MH_keyval_alias_with_addon:nnnn {EmphEqEnv}{AmSmultline}
          {AmSflalign}{\MH_set_boolean_T:n {multline}}
        \MH_keyval_alias_with_addon:nnnn {EmphEqEnv}{AmSmultline*}
          {AmSflalign*}{\MH_set_boolean_T:n {multline}}
        \EQ_renew_ams_environment:n {multline}
        \EQ_renew_ams_environment:n {multline*}
      }
    }
  }
}
%    \end{macrocode}
%  End of \cs{AtBeginDocument} additions.
%
%  If we need to load \pkg{ntheorem} support this is where it
%  happens. Otherwise we just stop here.
%    \begin{macrocode}
\if\g@EQ@ntheorem@support@bool
\else 
  \expandafter\endinput
\fi
%    \end{macrocode}
%
%  \section{Patching Things up for \pkg{ntheorem}}
%  \label{sec:impl:ntheorem}
%
%  Although \pkg{ntheorem} provides a lot of useful features it
%  does however ruin things in the \pkg{amsmath} environments,
%  because of the alternative syntax of its \cs{label} command.
%  \begin{macro}{\default_raisetag}
%  \begin{macro}{\raisetag}
%  \begin{macro}{\raise@tag}
%  \changes{v1.06}{2004/03/13}{Fixed \pkg{amsmath} bug \# 3624}
%  This corrects a bug\footnote{See
%  \url{http://www.latex-project.org/cgi-bin/ltxbugs2html?pr=amslatex/3624}}
%  in \pkg{amsmath} that could cause tags to cross page boundaries
%  without warning. The \cs{nobreak} is the addition.
%    \begin{macrocode}
\def\default_raisetag:{\nobreak}
\def\raisetag#1{
  \skip@#1\relax
  \xdef\raise@tag{\default_raisetag:
                  \vskip\iftagsleft@\else:-\fi:\the\skip@\relax}}
\MH_let:NwN\raise@tag\default_raisetag:
%    \end{macrocode}
%  \end{macro}
%  \end{macro}
%  \end{macro}
%  \begin{macro}{\displ@y@}
%  Patch it so that it doesn't set \cs{raise@tag} to \cs{@empty}
%  globally.
%    \begin{macrocode}
\def\displ@y@{\@display@init{%
  \global\column@\z@ \global\dspbrk@lvl\m@ne
  \global\tag@false \global\MH_let:NwN\raise@tag\default_raisetag:
}}
%    \end{macrocode}
%  \end{macro}
%  \begin{macro}{\MT_gobblelabel:w}
%  \changes{v2.00}{2004/04/14}{Added \cs{MT_gobblelabel:w}}
%  \begin{macro}{\EQ_gobbleoptional:w}
%  \changes{v2.00}{2004/04/14}{Added \cs{EQ_gobbleoptional:w}}
%  Macros for gobbling the `unorthodox' \cs{label} syntax of
%  \pkg{ntheorem}. Also overrides the normal \cs{label}-gobbling.
%    \begin{macrocode}
\def\MT_gobblelabel:w #1{\@ifnextchar[{\EQ_gobbleoptional:w}{}}
\def\EQ_gobbleoptional:w [#1]{}
%    \end{macrocode}
%  \end{macro}
%  \end{macro}
%  \subsection{Fixing \env{gather}}
%  The measuring phase of \env{gather} is patched by letting
%  \cs{label} be a copy of the internal \cs{MT_gobblelabel:w}
%  instead of \cs{@gobble}.
%    \begin{macrocode}
\def\gmeasure@#1{%
  \begingroup
    \measuring@true
    \totwidth@\z@
    \global\MH_let:NwN\tag@lengths\@empty
    \savecounters@
    \setbox\@ne\vbox{%
      \everycr{\noalign{\global\tag@false
%    \end{macrocode}
%  The patch for \cs{raisetag}.
%    \begin{macrocode}
        \global\MH_let:NwN\raise@tag\default_raisetag:
        \global\column@\z@}}%
%    \end{macrocode}
%  The patch for \cs{label}.
%    \begin{macrocode}
      \MH_let:NwN\label\MT_gobblelabel:w
      \halign{%
        \setboxz@h{$\m@th\displaystyle{##}$}%
        \if_dim:w \wdz@>\totwidth@
          \global\totwidth@\wdz@
        \fi:
        &\setboxz@h{\strut@{##}}%
        \savetaglength@
        \crcr
        #1%
        \math@cr@@@
      }%
    }%
    \restorecounters@
    \if@fleqn
        \global\advance\totwidth@\@mathmargin
    \fi:
    \iftagsleft@
        \if_dim:w \totwidth@>\displaywidth
            \global\MH_let:NwN\gdisplaywidth@\totwidth@
        \else:
            \global\MH_let:NwN\gdisplaywidth@\displaywidth
        \fi:
    \fi:
  \endgroup
}
%    \end{macrocode}
%
%  \subsection{Fixing \env{align}, \env{alignat}, and \env{flalign}}
%  Just copy the definition of \cs{measure@} from \pkg{amsmath} and
%  change one (1!) line like we did for \env{gather}.
%    \begin{macrocode}
\def\measure@#1{%
  \begingroup
    \measuring@true
    \global\eqnshift@\z@
    \global\alignsep@\z@
    \global\MH_let:NwN\tag@lengths\@empty
    \global\MH_let:NwN\field@lengths\@empty
    \savecounters@
    \global\setbox0\vbox{%
      \MH_let:NwN\math@cr@@@\math@cr@@@align@measure
      \everycr{\noalign{\global\tag@false
%    \end{macrocode}
%  The patch for \cs{raisetag}.
%    \begin{macrocode}
        \global\MH_let:NwN\raise@tag\default_raisetag:
        \global\column@\z@}}%
%    \end{macrocode}
%  The patch.
%    \begin{macrocode}
      \MH_let:NwN\label\MT_gobblelabel:w
      \global\row@\z@
      \tabskip\z@
      \halign{\span\align@preamble\crcr
        #1%
        \math@cr@@@
        \global\column@\z@
        \add@amps\maxfields@\cr
      }%
    }%
    \restorecounters@
    \ifodd\maxfields@
      \global\advance\maxfields@\@ne
    \fi:
    \if_num:w \xatlevel@=\tw@
      \if_num:w \maxfields@<\thr@@
        \MH_let:NwN\xatlevel@\z@
      \fi:
    \fi:
    \setbox\z@\vbox{%
      \unvbox\z@ \unpenalty \global\setbox\@ne\lastbox
    }%
    \global\totwidth@\wd\@ne
    \if@fleqn \global\advance\totwidth@\@mathmargin \fi:
    \global\MH_let:NwN\maxcolumn@widths\@empty
    \begingroup
      \MH_let:NwN\or:\relax
      \loop
        \global\setbox\@ne\hbox{%
          \unhbox\@ne \unskip \global\setbox\thr@@\lastbox
        }%
      \ifhbox\thr@@
        \xdef\maxcolumn@widths{ \or: \the\wd\thr@@ \maxcolumn@widths}%
      \repeat
    \endgroup
    \dimen@\displaywidth
    \advance\dimen@-\totwidth@
    \if_case:w \xatlevel@
      \global\alignsep@\z@
      \MH_let:NwN\minalignsep\z@
      \@tempcntb\z@
      \if@fleqn
        \@tempcnta\@ne
        \global\eqnshift@\@mathmargin
      \else:
        \@tempcnta\tw@
        \global\eqnshift@\dimen@
        \global\divide\eqnshift@\@tempcnta
      \fi:
    \or:
      \@tempcntb\maxfields@
      \divide\@tempcntb\tw@
      \@tempcnta\@tempcntb
      \advance\@tempcntb\m@ne
      \if@fleqn
        \global\eqnshift@\@mathmargin
        \global\alignsep@\dimen@
        \global\divide\alignsep@\@tempcnta
      \else:
        \global\advance\@tempcnta\@ne
        \global\eqnshift@\dimen@
        \global\divide\eqnshift@\@tempcnta
        \global\alignsep@\eqnshift@
      \fi:
    \or:
      \@tempcntb\maxfields@
      \divide\@tempcntb\tw@
      \global\advance\@tempcntb\m@ne
      \global\@tempcnta\@tempcntb
      \global\eqnshift@\z@
      \global\alignsep@\dimen@
      \if@fleqn
        \global\advance\alignsep@\@mathmargin\relax
      \fi:
      \global\divide\alignsep@\@tempcntb
    \fi:
    \if_dim:w \alignsep@<\minalignsep\relax
      \global\alignsep@\minalignsep\relax
      \if_dim:w \eqnshift@>\z@
        \if@fleqn\else:
          \global\eqnshift@\displaywidth
          \global\advance\eqnshift@-\totwidth@
          \global\advance\eqnshift@-\@tempcntb\alignsep@
          \global\divide\eqnshift@\tw@
        \fi:
      \fi:
    \fi:
    \if_dim:w \eqnshift@<\z@
      \global\eqnshift@\z@
    \fi:
    \calc@shift@align
    \global\tagshift@\totwidth@
    \global\advance\tagshift@\@tempcntb\alignsep@
    \if@fleqn
      \if_num:w \xatlevel@=\tw@
        \global\advance\tagshift@-\@mathmargin\relax
      \fi:
    \else:
      \global\advance\tagshift@\eqnshift@
    \fi:
    \iftagsleft@ \else:
      \global\advance\tagshift@-\displaywidth
    \fi:
    \dimen@\minalignsep\relax
    \global\advance\totwidth@\@tempcntb\dimen@
    \if_dim:w \totwidth@>\displaywidth
      \global\MH_let:NwN\displaywidth@\totwidth@
    \else:
      \global\MH_let:NwN\displaywidth@\displaywidth
    \fi:
  \endgroup
}
%    \end{macrocode}
%
%  \subsection{Fixing \env{multline}}
%  \begin{macro}{\EQ_multline_labelhack:n}
%  \begin{macro}{\EQ_multline_labelhack_opt:nw}
%  As \pkg{ntheorem} defines \cs{label} to have the syntax
%  \cs{label}\marg{label}\oarg{name}, we need to make sure that it
%  has that inside \env{multline} as well. Especially since the
%  measuring phase of \env{multline} \emph{defines} \cs{label} as
%  well.
%    \begin{macrocode}
\def\EQ_multline_labelhack:n #1{%
  \begingroup\measuring@false\label@in@display{#1}\endgroup}
\def\EQ_multline_labelhack_opt:nw #1[#2]{%
  \begingroup\measuring@false\label@in@display{#1}[#2]\endgroup}
%    \end{macrocode}
%  \end{macro}
%  \end{macro}
%  \begin{macro}{\multline@}
%  Patching \env{multline} takes more work because it does its
%  processing differently, so we need to apply hooks twice: One in
%  the measuring phase \cs{mmeasure@} where \cs{label} is redefined
%  and one in the production phase \cs{multline@} where the ordinary
%  \cs{label} is gobbled (we use \cs{MT_gobblelabel:w} as
%  before). First the typesetting phase.
%    \begin{macrocode}
\def\multline@#1{%
  \Let@
  \@display@init{\global\advance\row@\@ne \global\dspbrk@lvl\m@ne}%
  \chardef\dspbrk@context\z@
  \restore@math@cr
  \MH_let:NwN\tag\tag@in@align
%    \end{macrocode}
%  The patch for \cs{raisetag}.
%    \begin{macrocode}
  \global\tag@false \global\MH_let:NwN\raise@tag\default_raisetag:
  \mmeasure@{#1}%
  \MH_let:NwN\tag\gobble@tag
%    \end{macrocode}
%  The patch.
%    \begin{macrocode}
  \MH_let:NwN\label\MT_gobblelabel:w
  \tabskip \if@fleqn \@mathmargin \else: \z@skip \fi:
  \totwidth@\displaywidth
  \if@fleqn
    \advance\totwidth@-\@mathmargin
  \fi:
  \halign\bgroup
    \hbox to\totwidth@{%
      \if@fleqn
        \hskip \@centering \relax
      \else:
        \hfil
      \fi:
      \strut@
      $\m@th\displaystyle{}##\endmultline@math%$
      \hfil
    }%
    \crcr
    \if@fleqn
      \hskip-\@mathmargin
      \def\multline@indent{\hskip\@mathmargin}% put it back
    \else:
      \hfilneg
      \def\multline@indent{\hskip\multlinegap}%
    \fi:
    \iftagsleft@
      \iftag@
        \begingroup
          \ifshifttag@
            \rlap{\vbox{%
              \normalbaselines
              \hbox{%
                \strut@
                \make@display@tag
              }%
              \vbox to\lineht@{}%
              \raise@tag
            }}%
              \multline@indent
          \else:
            \setbox\z@\hbox{\make@display@tag}%
            \dimen@\@mathmargin \advance\dimen@-\wd\z@
            \if_dim:w \dimen@<\multlinetaggap
              \dimen@\multlinetaggap
            \fi:
            \box\z@ \hskip\dimen@\relax
          \fi:
        \endgroup
      \else:
        \multline@indent
      \fi:
    \else:
      \multline@indent
    \fi:
  #1%
}
%    \end{macrocode}
%  \end{macro}
%  \begin{macro}{\mmeasure@}
%  Then redefine the label correctly.
%    \begin{macrocode}
\def\mmeasure@#1{%
  \begingroup
    \measuring@true
%    \end{macrocode}
%  The patch.
%    \begin{macrocode}
    \def\label##1{%
      \@ifnextchar[
        {\EQ_multline_labelhack_opt:nw {##1}}
        {\EQ_multline_labelhack:n {##1}}
    }
    \def\math@cr@@@{\cr}%
    \MH_let:NwN\shoveleft\@iden \MH_let:NwN\shoveright\@iden
    \savecounters@
    \global\row@\z@
    \setbox\@ne\vbox{%
      \global\MH_let:NwN\df@tag\@empty
      \halign{%
        \setboxz@h{\@lign$\m@th\displaystyle{}##$}%
        \iftagsleft@
          \if_num:w \row@=\@ne
            \global\totwidth@\wdz@
            \global\lineht@\ht\z@
          \fi:
        \else:
          \global\totwidth@\wdz@
          \global\lineht@\dp\z@
        \fi:
        \crcr
        #1%
        \crcr
      }%
    }%
    \if_meaning:NN
      \df@tag\@empty
    \else:
      \global\tag@true
    \fi:
    \if@eqnsw\global\tag@true\fi:
    \iftag@
      \setboxz@h{%
        \if@eqnsw
          \stepcounter{equation}%
          \tagform@\theequation
        \else:
          \df@tag
        \fi:
      }%
      \global\tagwidth@\wdz@
      \dimen@\totwidth@
      \advance\dimen@\tagwidth@
      \advance\dimen@\multlinetaggap
      \iftagsleft@\else:
        \if@fleqn
          \advance\dimen@\@mathmargin
        \fi:
      \fi:
      \if_dim:w \dimen@>\displaywidth
        \global\shifttag@true
      \else:
        \global\shifttag@false
      \fi:
    \fi:
    \restorecounters@
  \endgroup
}
%    \end{macrocode}
%  \end{macro}
%
%  \subsection{Setting End Marks Correctly}
%
%  \begin{macro}{\mintagvsep}
%  \changes{v1.05}{2004/02/16}{Introduced \cs{mintagvsep} to
%  support the \pkg{ntheorem} package}
%  \begin{macro}{\EQ_only_endmark:}
%  \changes{v1.05}{2004/02/16}{Introduced \cs{EQ_only_endmark:}
%  to support the \pkg{ntheorem} package}
%  \changes{v2.11}{2004/08/03}{Make sure the tag is printed even with the
%  \pkg{mathtools} key \key{showonlyrefs} activated}
%  \begin{macro}{\EQ_tag_plus_endmark:}
%  \changes{v1.05}{2004/02/16}{Introduced
%  \cs{EQ_tag_plus_endmark:} to support the \pkg{ntheorem} package}
%  \pkg{ntheorem} provides an extension to the theorem environment
%  and also supports automatic end of theorem marks. To do so it must
%  redefine parts of the \pkg{amsmath} code and it eventually ends up
%  in problems when the tags and the marker are all typeset at the
%  right margin.
%
%  There are basically two different possibilities:
%  \begin{center}\hfill
%  \unitlength2.5pt \thicklines
%  \begin{picture}(60,45)(0,25)
%    \put(0,50){\framebox(25,20){Math Height}}
%    \put(0,25){\framebox(25,25){Math Depth}}
%    \put(40,50){\framebox(20,10){Tag height}}
%    \put(40,45){\framebox(20,5){Tag depth}}
%    \put(40,25){\framebox(20,10){Marker}}
%    \put(50,35){\vector(0,1){10}}
%    \put(50,45){\vector(0,-1){10}}
%    \put(50,39){\ \ $\delta>0$}
%    \multiput(26,25)(2,0){7}{\line(1,0){1}}
%  \end{picture} \hfill\vrule\hfill
%  \begin{picture}(60,45)(0,25)
%    \put(0,50){\framebox(25,20){Math Height}}
%    \put(0,40){\framebox(25,10){Math Depth}}
%    \put(40,50){\framebox(20,10){Tag height}}
%    \put(40,45){\framebox(20,5){Tag depth}}
%    \put(40,28){\framebox(20,10){Marker}}
%    \put(50,38){\vector(0,1){7}}
%    \put(50,45){\vector(0,-1){7}}
%    \put(50,40){\ \ $\delta=0$}
%  \end{picture}\hfill\mbox{}
%  \end{center}
%  Let's look closer at this. $h_M$ is the height of the end of
%  theorem marker, which we'll try to place so that its baseline
%  aligns with the bottom of the math display. $d_D$ is the depth of
%  the the last line of the math display including the depth of the
%  surrounding boxes while $d_T$ is the depth of the tag. Also we let
%  $m$ denote the minimum vertical distance between tag and marker
%  and $\delta$ is some positive length that may be added to $m$
%  depending on the other values. To ensure a ``proper'' placement of
%  the marker we simply test the inequality
%  \begin{gather}
%     h_M + d_T + m > d_D \label{eq:tag_placement}
%  \end{gather}
%  If the test is true we're done and just have to set the marker so
%  that the top of it is placed at a distance of $m$ from the bottom
%  of the tag. If it's not we can safely put the marker at the bottom
%  of the display.
%    \begin{macrocode}
\newlength\mintagvsep
\mintagvsep5pt
\def\EQ_only_endmark:{%
  \MH_set_boolean_T:n {show_manual_tags}
  \global\tag@true
  \iftagsleft@
    \gdef\df@tag{%
      \hbox to \displaywidth{%
        \hss
%    \end{macrocode}
%  If the tags are set to the left we just have to lower the end mark
%  by the total depth of the display including its surrounding box.
%  Also we have to take care of \cs{normallineskip}.
%    \begin{macrocode}
        \raisebox{-\normallineskip
                  -\depthof{\EQ_outerbox:{}}
                  -\heightof{\EQ_outerbox:{}}
                  -\dp\EQ_mathdisplay_box
                  -\ht\EQ_mathdisplay_box}
                 [0pt][0pt]{\PotEndMark{\maketag@@@}}}}
  \else:
    \gdef\df@tag{%
      \raisebox{-\dp\EQ_mathdisplay_box
                -\depthof{\EQ_outerbox:{}}}
               {\PotEndMark{\maketag@@@}}}%
  \fi:}
%    \end{macrocode}
%  \changes{v2.00}{2004/04/14}{Fixed another bug in placement of the
%  marker}
%  \changes{v2.10}{2004/07/26}{Fixed horizontal spacing in \env{equation}}
%  In v2.00 I fixed another bug I had overlooked. This is getting
%  embarrassing. The outer box must be taken into account as well.
%    \begin{macrocode}
\def\EQ_tag_plus_endmark:{%
  \iftagsleft@
    \gdef\maketag@@@##1{%
%    \end{macrocode}
%  The following fixes a bug when there is a wide tag in an
%  \env{equation} environment and was reported by Bernard
%  Alfonsi.\footnote{\email{alfonsi@math.u-psud.fr}} The problem
%  arises when \pkg{ntheorem} tries to typeset a tag and an end mark,
%  because the tag is set at the left margin and the end mark at the
%  right margin and \pkg{ntheorem} over-simplifies this and forgets
%  to take the displayed math into account! The right solution is to
%  wrap the tag in a box of the correct size, namely |\widthof{##1}|
%  and then push the end mark to the right margin. Then the displayed
%  math is set correcly.
%    \begin{macrocode}
      \hfuzz\displaywidth
      \makebox[\widthof{##1}][l]{%
        \hbox to \displaywidth{\m@th\normalfont##1\hss
          \raisebox{-\dp\EQ_mathdisplay_box
                    -\depthof{\EQ_outerbox:{}}}
                   {\PotEndMark{\hss}}}}}%
  \else:
    \gdef\maketag@@@##1{%
      \hbox{\m@th\normalfont##1\llap{\hss\PotEndMark{%
        \settoheight{\@tempdima}{\@nameuse{\InTheoType Symbol}}%
        \MH_addtolength:dn \@tempdima
          {\g_EQ_bottomtag_depth_dim+\mintagvsep}
        \setlength{\@tempdimb}
          {\g_EQ_bottomrow_depth_dim+\depthof{\EQ_outerbox:{}}
            +\depthof{\EQ_innerbox:{}}}
        \if_dim:w \@tempdima>\@tempdimb
          \MH_setlength:dn \@tempdima{-\dp\EQ_mathdisplay_box
            -\@tempdima+\g_EQ_bottomrow_depth_dim}
          \addtolength{\@tempdima}{\depthof{\EQ_innerbox:{}}}%
          \def\EQ_next:{\raisebox{\@tempdima}}%
        \else:
%    \end{macrocode}
%  \changes{v1.06}{2004/03/13}{Fixed bug in placement of the marker
%  and re-arranged some code}
%  In v1.06 I fixed a stupid bug regarding placement of the marker.
%  If I had done a proper test I would have discovered it
%  immediately!
%    \begin{macrocode}
          \def\EQ_next:{
            \raisebox{-\dp\EQ_mathdisplay_box
                      -\depthof{\EQ_outerbox:{}}}}%
        \fi:\EQ_next:}}}}%
  \fi:}
%    \end{macrocode}
%  \end{macro}
%  \end{macro}
%  \end{macro}
%
%  \begin{macro}{\MT_nonumber:}
%    \begin{macrocode}
\MH_let:NwN \MT_nonumber: \nonumber
%    \end{macrocode}
%  \end{macro}
%  \begin{macro}{\nonumber}
%  Redefine \cs{nonumber} to reset depth of bottom tag.
%    \begin{macrocode}
\def\g_EQ_latest_nonzerodepth_row_fint{\z@}
\renewcommand*\nonumber{
  \@tempcnta=\row@
  \if_num:w \l_EQ_equationtype_int=\thr@@
    \advance\@tempcnta\m@ne
  \else:
    \if_num:w \l_EQ_equationtype_int=4\relax
      \advance\@tempcnta\m@ne
    \fi:
  \fi:
  \if_num:w \g_EQ_latest_nonzerodepth_row_fint=\@tempcnta
  \else:
    \global\g_EQ_bottomtag_depth_dim=-\baselineskip
  \fi:
  \MT_nonumber:
}
%    \end{macrocode}
%  \end{macro}
%
%  End of \pkg{ntheorem} patches and package.
%    \begin{macrocode}
%</package>
%    \end{macrocode}
% Finale
\endinput
