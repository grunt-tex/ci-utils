% \iffalse meta-comment
%
% File: bitset.dtx
% Version: 2011/01/30 v1.1
% Info: Handle bit-vector datatype
%
% Copyright (C) 2007, 2011 by
%    Heiko Oberdiek <heiko.oberdiek at googlemail.com>
%
% This work may be distributed and/or modified under the
% conditions of the LaTeX Project Public License, either
% version 1.3c of this license or (at your option) any later
% version. This version of this license is in
%    http://www.latex-project.org/lppl/lppl-1-3c.txt
% and the latest version of this license is in
%    http://www.latex-project.org/lppl.txt
% and version 1.3 or later is part of all distributions of
% LaTeX version 2005/12/01 or later.
%
% This work has the LPPL maintenance status "maintained".
%
% This Current Maintainer of this work is Heiko Oberdiek.
%
% The Base Interpreter refers to any `TeX-Format',
% because some files are installed in TDS:tex/generic//.
%
% This work consists of the main source file bitset.dtx
% and the derived files
%    bitset.sty, bitset.pdf, bitset.ins, bitset.drv, bitset-test1.tex,
%    bitset-test2.tex, bitset-test3.tex.
%
% Distribution:
%    CTAN:macros/latex/contrib/oberdiek/bitset.dtx
%    CTAN:macros/latex/contrib/oberdiek/bitset.pdf
%
% Unpacking:
%    (a) If bitset.ins is present:
%           tex bitset.ins
%    (b) Without bitset.ins:
%           tex bitset.dtx
%    (c) If you insist on using LaTeX
%           latex \let\install=y% \iffalse meta-comment
%
% File: bitset.dtx
% Version: 2011/01/30 v1.1
% Info: Handle bit-vector datatype
%
% Copyright (C) 2007, 2011 by
%    Heiko Oberdiek <heiko.oberdiek at googlemail.com>
%
% This work may be distributed and/or modified under the
% conditions of the LaTeX Project Public License, either
% version 1.3c of this license or (at your option) any later
% version. This version of this license is in
%    http://www.latex-project.org/lppl/lppl-1-3c.txt
% and the latest version of this license is in
%    http://www.latex-project.org/lppl.txt
% and version 1.3 or later is part of all distributions of
% LaTeX version 2005/12/01 or later.
%
% This work has the LPPL maintenance status "maintained".
%
% This Current Maintainer of this work is Heiko Oberdiek.
%
% The Base Interpreter refers to any `TeX-Format',
% because some files are installed in TDS:tex/generic//.
%
% This work consists of the main source file bitset.dtx
% and the derived files
%    bitset.sty, bitset.pdf, bitset.ins, bitset.drv, bitset-test1.tex,
%    bitset-test2.tex, bitset-test3.tex.
%
% Distribution:
%    CTAN:macros/latex/contrib/oberdiek/bitset.dtx
%    CTAN:macros/latex/contrib/oberdiek/bitset.pdf
%
% Unpacking:
%    (a) If bitset.ins is present:
%           tex bitset.ins
%    (b) Without bitset.ins:
%           tex bitset.dtx
%    (c) If you insist on using LaTeX
%           latex \let\install=y% \iffalse meta-comment
%
% File: bitset.dtx
% Version: 2011/01/30 v1.1
% Info: Handle bit-vector datatype
%
% Copyright (C) 2007, 2011 by
%    Heiko Oberdiek <heiko.oberdiek at googlemail.com>
%
% This work may be distributed and/or modified under the
% conditions of the LaTeX Project Public License, either
% version 1.3c of this license or (at your option) any later
% version. This version of this license is in
%    http://www.latex-project.org/lppl/lppl-1-3c.txt
% and the latest version of this license is in
%    http://www.latex-project.org/lppl.txt
% and version 1.3 or later is part of all distributions of
% LaTeX version 2005/12/01 or later.
%
% This work has the LPPL maintenance status "maintained".
%
% This Current Maintainer of this work is Heiko Oberdiek.
%
% The Base Interpreter refers to any `TeX-Format',
% because some files are installed in TDS:tex/generic//.
%
% This work consists of the main source file bitset.dtx
% and the derived files
%    bitset.sty, bitset.pdf, bitset.ins, bitset.drv, bitset-test1.tex,
%    bitset-test2.tex, bitset-test3.tex.
%
% Distribution:
%    CTAN:macros/latex/contrib/oberdiek/bitset.dtx
%    CTAN:macros/latex/contrib/oberdiek/bitset.pdf
%
% Unpacking:
%    (a) If bitset.ins is present:
%           tex bitset.ins
%    (b) Without bitset.ins:
%           tex bitset.dtx
%    (c) If you insist on using LaTeX
%           latex \let\install=y% \iffalse meta-comment
%
% File: bitset.dtx
% Version: 2011/01/30 v1.1
% Info: Handle bit-vector datatype
%
% Copyright (C) 2007, 2011 by
%    Heiko Oberdiek <heiko.oberdiek at googlemail.com>
%
% This work may be distributed and/or modified under the
% conditions of the LaTeX Project Public License, either
% version 1.3c of this license or (at your option) any later
% version. This version of this license is in
%    http://www.latex-project.org/lppl/lppl-1-3c.txt
% and the latest version of this license is in
%    http://www.latex-project.org/lppl.txt
% and version 1.3 or later is part of all distributions of
% LaTeX version 2005/12/01 or later.
%
% This work has the LPPL maintenance status "maintained".
%
% This Current Maintainer of this work is Heiko Oberdiek.
%
% The Base Interpreter refers to any `TeX-Format',
% because some files are installed in TDS:tex/generic//.
%
% This work consists of the main source file bitset.dtx
% and the derived files
%    bitset.sty, bitset.pdf, bitset.ins, bitset.drv, bitset-test1.tex,
%    bitset-test2.tex, bitset-test3.tex.
%
% Distribution:
%    CTAN:macros/latex/contrib/oberdiek/bitset.dtx
%    CTAN:macros/latex/contrib/oberdiek/bitset.pdf
%
% Unpacking:
%    (a) If bitset.ins is present:
%           tex bitset.ins
%    (b) Without bitset.ins:
%           tex bitset.dtx
%    (c) If you insist on using LaTeX
%           latex \let\install=y\input{bitset.dtx}
%        (quote the arguments according to the demands of your shell)
%
% Documentation:
%    (a) If bitset.drv is present:
%           latex bitset.drv
%    (b) Without bitset.drv:
%           latex bitset.dtx; ...
%    The class ltxdoc loads the configuration file ltxdoc.cfg
%    if available. Here you can specify further options, e.g.
%    use A4 as paper format:
%       \PassOptionsToClass{a4paper}{article}
%
%    Programm calls to get the documentation (example):
%       pdflatex bitset.dtx
%       makeindex -s gind.ist bitset.idx
%       pdflatex bitset.dtx
%       makeindex -s gind.ist bitset.idx
%       pdflatex bitset.dtx
%
% Installation:
%    TDS:tex/generic/oberdiek/bitset.sty
%    TDS:doc/latex/oberdiek/bitset.pdf
%    TDS:doc/latex/oberdiek/test/bitset-test1.tex
%    TDS:doc/latex/oberdiek/test/bitset-test2.tex
%    TDS:doc/latex/oberdiek/test/bitset-test3.tex
%    TDS:source/latex/oberdiek/bitset.dtx
%
%<*ignore>
\begingroup
  \catcode123=1 %
  \catcode125=2 %
  \def\x{LaTeX2e}%
\expandafter\endgroup
\ifcase 0\ifx\install y1\fi\expandafter
         \ifx\csname processbatchFile\endcsname\relax\else1\fi
         \ifx\fmtname\x\else 1\fi\relax
\else\csname fi\endcsname
%</ignore>
%<*install>
\input docstrip.tex
\Msg{************************************************************************}
\Msg{* Installation}
\Msg{* Package: bitset 2011/01/30 v1.1 Handle bit-vector datatype (HO)}
\Msg{************************************************************************}

\keepsilent
\askforoverwritefalse

\let\MetaPrefix\relax
\preamble

This is a generated file.

Project: bitset
Version: 2011/01/30 v1.1

Copyright (C) 2007, 2011 by
   Heiko Oberdiek <heiko.oberdiek at googlemail.com>

This work may be distributed and/or modified under the
conditions of the LaTeX Project Public License, either
version 1.3c of this license or (at your option) any later
version. This version of this license is in
   http://www.latex-project.org/lppl/lppl-1-3c.txt
and the latest version of this license is in
   http://www.latex-project.org/lppl.txt
and version 1.3 or later is part of all distributions of
LaTeX version 2005/12/01 or later.

This work has the LPPL maintenance status "maintained".

This Current Maintainer of this work is Heiko Oberdiek.

The Base Interpreter refers to any `TeX-Format',
because some files are installed in TDS:tex/generic//.

This work consists of the main source file bitset.dtx
and the derived files
   bitset.sty, bitset.pdf, bitset.ins, bitset.drv, bitset-test1.tex,
   bitset-test2.tex, bitset-test3.tex.

\endpreamble
\let\MetaPrefix\DoubleperCent

\generate{%
  \file{bitset.ins}{\from{bitset.dtx}{install}}%
  \file{bitset.drv}{\from{bitset.dtx}{driver}}%
  \usedir{tex/generic/oberdiek}%
  \file{bitset.sty}{\from{bitset.dtx}{package}}%
  \usedir{doc/latex/oberdiek/test}%
  \file{bitset-test1.tex}{\from{bitset.dtx}{test1}}%
  \file{bitset-test2.tex}{\from{bitset.dtx}{test2,etex}}%
  \file{bitset-test3.tex}{\from{bitset.dtx}{test2,noetex}}%
  \nopreamble
  \nopostamble
  \usedir{source/latex/oberdiek/catalogue}%
  \file{bitset.xml}{\from{bitset.dtx}{catalogue}}%
}

\catcode32=13\relax% active space
\let =\space%
\Msg{************************************************************************}
\Msg{*}
\Msg{* To finish the installation you have to move the following}
\Msg{* file into a directory searched by TeX:}
\Msg{*}
\Msg{*     bitset.sty}
\Msg{*}
\Msg{* To produce the documentation run the file `bitset.drv'}
\Msg{* through LaTeX.}
\Msg{*}
\Msg{* Happy TeXing!}
\Msg{*}
\Msg{************************************************************************}

\endbatchfile
%</install>
%<*ignore>
\fi
%</ignore>
%<*driver>
\NeedsTeXFormat{LaTeX2e}
\ProvidesFile{bitset.drv}%
  [2011/01/30 v1.1 Handle bit-vector datatype (HO)]%
\documentclass{ltxdoc}
\usepackage{holtxdoc}[2011/11/22]
\usepackage{amstext}
\usepackage{ifluatex}
\ifluatex
\else
  \usepackage[T1]{fontenc}%
  \usepackage{lmodern}%
\fi
\begin{document}
  \DocInput{bitset.dtx}%
\end{document}
%</driver>
% \fi
%
% \CheckSum{3510}
%
% \CharacterTable
%  {Upper-case    \A\B\C\D\E\F\G\H\I\J\K\L\M\N\O\P\Q\R\S\T\U\V\W\X\Y\Z
%   Lower-case    \a\b\c\d\e\f\g\h\i\j\k\l\m\n\o\p\q\r\s\t\u\v\w\x\y\z
%   Digits        \0\1\2\3\4\5\6\7\8\9
%   Exclamation   \!     Double quote  \"     Hash (number) \#
%   Dollar        \$     Percent       \%     Ampersand     \&
%   Acute accent  \'     Left paren    \(     Right paren   \)
%   Asterisk      \*     Plus          \+     Comma         \,
%   Minus         \-     Point         \.     Solidus       \/
%   Colon         \:     Semicolon     \;     Less than     \<
%   Equals        \=     Greater than  \>     Question mark \?
%   Commercial at \@     Left bracket  \[     Backslash     \\
%   Right bracket \]     Circumflex    \^     Underscore    \_
%   Grave accent  \`     Left brace    \{     Vertical bar  \|
%   Right brace   \}     Tilde         \~}
%
% \GetFileInfo{bitset.drv}
%
% \title{The \xpackage{bitset} package}
% \date{2011/01/30 v1.1}
% \author{Heiko Oberdiek\\\xemail{heiko.oberdiek at googlemail.com}}
%
% \maketitle
%
% \begin{abstract}
% This package defines and implements the data type bit set,
% a vector of bits. The size of the vector may grow dynamically.
% Individual bits can be manipulated.
% \end{abstract}
%
% \tableofcontents
%
% \clearpage
% \section{Documentation}
%
% \subsection{Introduction}
%
% Annotations in the PDF format know entries whose values are
% integers. This numbers are interpreted as set of flags
% specifying properties. For example, annotation dictionaries
% can have a key |/F|. The bits of its integer value
% are interpreted the following way:
% \begin{quote}
% \begin{tabular}{@{}l|l@{}}
%   Bit position & Property name\\
%   \hline
%   1 & Invisible\\
%   2 & Hidden\\
%   3 & Print\\
%   4 & NoZoom\\
%   5 & NoRotate\\
%   6 & NoView\\
%   7 & ReadOnly\\
%   \dots & \dots
% \end{tabular}
% \end{quote}
% Now, let's see how these values are set in package \xpackage{hyperref}
% before it uses this package (before v6.77a):
%\begin{quote}
%\begin{verbatim}
%\ifFld@hidden /F 6\else /F 4\fi
%\end{verbatim}
%\end{quote}
% Where are the other flags? The following example for key |/Ff|
% in a widget annotation
% supports at least three properties:
%\begin{quote}
%\begin{verbatim}
%\ifFld@multiline
%  \ifFld@readonly /Ff 4097\else /Ff 4096\fi
%\else
%  \ifFld@password
%    \ifFld@readonly /Ff 8193\else /Ff 8192\fi
%  \else
%    \ifFld@readonly /Ff 1\fi
%  \fi
%\fi
%\end{verbatim}
%\end{quote}
% But you see the point. It would be a nightmare to continue
% this way in supporting the missing flag settings. This kind
% of integers may have up to 32 bits.
%
% Therefore I wanted a data structure for setting and
% clearing individual bits. Also it should provide
% an export as decimal number. The snipsets above are
% executed in expansion contexts without \TeX's stomach
% commands. It would be convenient to have an expandable
% conversion from the data structure to the integer that
% gets written to the PDF file.
%
% This package \xpackage{bitset} implements such a data
% structure. The interface is quite close to Java's
% class |BitSet| in order not to learn to many interfaces
% for the same kind of data structure.
%
% \subsection{Glossary}
%
% \begin{description}
% \item[Bit set:]
%   A bit set is a vector of bits or flags.
%   The vector size is unlimited and
%   grows dynamically. An undefined bit set is treated as bit
%   set where all bits are cleared.
%
%   Bit sets are addressed by name. A name should consists of
%   letters or digits. Technically it must survive \cs{csname},
%   see \LaTeX's environment names for other names with such
%   a constraint. Package \xpackage{babel}'s shorthands are
%   not supported due to technical reasons. Shorthand support
%   would break expandable operations.
% \item[Size:]
%   A size of a bit set is the number of bits in use.
%   It's the number of the highest index, incremented by one.
%   Sizes are in the range |0| up to |2147483647|, the highest
%   number supported by \TeX.
% \item[Index:]
%   Bit positions in a bit set are addressed by an index number.
%   The bit vector is zero based. The first and least significant
%   bit is addressed by index |0| and the highest possible bit
%   by |2147483646|.
% \item[Bit:]
%   A bit is enoded as |0| for cleared/disabled or |1| for set/enabled.
% \end{description}
%
% \subsection{Design principles}
%
% \begin{description}
% \item[Name conventions:]
%   To avoid conflicts with existing macro names, the operations
%   are prefixed by the package name.
% \item[Zero based indexes:] The first bit is addressed by zero.
%   (Convention of array indexing in C, Java, \dots)
% \item[Unlimited size:]
%   There is no restriction on the size of a bit set other than
%   usual memory limitations. \cs{bitsetSetDec} and \cs{bitsetGetDec}
%   transparently switch to package \xpackage{bigintcalc} if the numbers
%   get too large for \TeX's number limit.
% \item[Expandibility:]
%   Any operation that does not change the bit set is expandable.
%   And all operations that extract or calculate some result
%   do this in exact two expansion steps. For example,
%   a macro \cs{Macro} wants a bit set as decimal number.
%   But the argument must be a plain number without macros.
%   Thus you could prefix \cs{bitsetGetDec} with \cs{number}.
%   However this won't work for bit sets with 31 or more bits
%   because of \TeX's number limit of $2^{31}-1$.
%   then just hit the operator with two \cs{expandafter}:
%   \begin{quote}
%   |\expandafter\expandafter\expandafter|\\
%   |\Macro\bitsetGetDec{foo}|
%   \end{quote}
%   |\bitsetGetDec| is hit first by the third \cs{expandafter}
%   and then by the second one.
% \item[Format independence:] This package is written as \LaTeX\ package,
%   but it does not depend on \LaTeX. It will also work for other formats
%   such as \plainTeX.
% \item[Independence from \TeX\ engines:]
%   Vanilla \TeX\ is all you need. Calculations are delegated to
%   packages \xpackage{intcalc} and \xpackage{bigintcalc}.
%   They don't need any special features, but they will switch
%   to a little more efficient implementation if features such
%   as \cs{numexpr} are available.
% \item[Numeric arguments:] Anything that is accepted by
%   \cs{number}. If \eTeX\ is detected, also expressions
%   for \cs{numexpr} are supported. The only exception so far
%   is the number for \cs{bitsetSetDec}. The number might be too
%   large for \cs{number} or \cs{numexpr}.
% \item[Error messages:]
%   In expandable contexts, only a limited set of \TeX\ primitive
%   commands work as expected. So called stomach commands behave
%   like \cs{relax} and don't get expanded or executed.
%   Unhappily also the error commands belong to this category.
%   The expandable operations will throw an unknown control
%   sequence instead to get \TeX's and user's attention.
%   The name of these control sequences starts with
%   \cs{BitSetError:} with the type of error after the colon.
% \end{description}
%
% \subsection{Operator overview}
%
% \begin{minipage}{\linewidth}
% \begin{description}
% \def\It#1#2{^^A
%   \item[^^A
%     {#1} ^^A
%     \textmd{(\hyperref[{sec:#2}]{section \ref*{sec:#2}})}^^A
%   ]^^A
%   \mbox{}\\^^A
% }
% \def\op#1{^^A
%   \cs{bitset}\texttt{\bfseries#1}^^A
% }
% \def\M#1{\meta{#1}}
% \def\Q{\hspace*{\fill}}
% \It{Miscellaneous}{misc}
%   \op{Reset}\Q \M{BitSet}\\
%   \op{Let}\Q \M{BitSet A} \M{BitSet B}
% \It{Import}{import}
%   \op{SetBin}, \op{SetOct}, \op{SetHex}\Q \M{BitSet} \M{Value}\\
%   \op{SetDec}\Q \M{BitSet} \M{Value}
% \It{Export\textsuperscript{a}}{export}
%   \op{GetBin}, \op{GetOct}, \op{GetHex}\Q \M{BitSet} \M{MinSize}\\
%   \op{GetDec}\Q \M{BitSet}
% \It{Logical operators}{log}
%   \op{And}, \op{AndNot}\Q \M{BitSet A} \M{BitSet B}\\
%   \op{Or}, \op{Xor}\Q \M{BitSet A} \M{BitSet B}
% \It{Shifting}{shift}
%   \op{ShiftLeft}, \op{ShiftRight}\Q \M{BitSet} \M{ShiftAmount}\\
% \It{Bit manipulation}{bitman}
%   \op{Clear}, \op{Set}, \op{Flip}\Q \M{BitSet} \M{Index}\\
%   \op{SetValue}\Q \M{BitSet} \M{Index} \M{Value}\\
%   \op{ClearRange}, \op{SetRange}, \op{FlipRange}\\
%     \Q \M{BitSet} \M{IndexFrom} \M{IndexTo}\\
%   \op{SetValueRange}\Q \M{BitSet} \M{IndexFrom} \M{IndexTo}
% \It{Bit retrieval\textsuperscript{a}}{bitret}
%   \op{Get}\Q \M{BitSet} \M{Index}\\
%   \op{NextClearBit}, \op{NextSetBit}\Q \M{BitSet} \M{Index}\\
%   \op{GetSetBitList}\Q \M{BitSet}
% \It{Bit set properties}{prop}
%   \op{Size}, \op{Cardinality}\Q \M{BitSet}
% \It{Queries\textsuperscript{b}}{query}
%   \op{IsDefined}, \op{IsEmpty}\Q \M{BitSet} \M{Then} \M{Else}\\
%   \op{Equals}, \op{Intersects}\Q \M{BitSet A} \M{BitSet B} \M{Then} \M{Else}\\
%   \op{Query}\Q \M{BitSet} \M{Index} \M{Then} \M{Else}
% \end{description}
% \footnotetext[1]{Macros are expandable, full expansion by two steps.}
% \footnotetext[2]{Macros are expandable.}
% \end{minipage}
%
% \subsection{Package loading}
%
% The package can be used as normal \LaTeX\ package:
%\begin{quote}
%|\usepackage{bitset}|
%\end{quote}
% Also \plainTeX\ is supported:
%\begin{quote}
%|\input bitset.sty\relax|
%\end{quote}
%
% \subsection{Operators}
%
% The following macros work on and with bit sets.
% A bit set \meta{BitSet} is represented by a name.
% The should consist of letters and digits. Technically
% it must survive \cs{csname}. It is the same constraint
% that must be satisfied by label or environment names
% in \LaTeX.
%
% However active characters that are shorthands
% of package \xpackage{babel} are not supported.
% Support for shorthands works by an assignment.
% But many operators
% such as \cs{bitsetGetDec} must be usable in expandable
% contexts. There assignments will not be executed in the
% best case or they will cause errors.
%
% The bits in a bit set are addressed by non-negative
% integers starting from zero. Thus negative index
% numbers cause an error message. Because index numbers
% are \TeX\ numbers. The largest index is 2147483647.
% But in practice memory limits and patience limits
% will be very likely reached much before.
%
% \subsubsection{Miscellaneous}\label{sec:misc}
%
% There isn't a separate operation for bit set creation.
% For simplicity an undefined bit set is treated as bit set
% with all bits cleared.
%
% \begin{declcs}{bitsetReset} \M{BitSet}
% \end{declcs}
% Macro \cs{bitsetReset} clears all bits. The result is an
% empty bit set. It may also be used as replacement for
% an operation ``new'', because an undefined bit set is defined
% afterwards.
%
% \begin{declcs}{bitsetLet} \M{BitSet A} \M{BitSet B}
% \end{declcs}
% Macro \cs{bitsetLet} performs a simple assignment similar
% to \TeX's \cs{let}. After the operation \meta{BitSet A} has
% the same value as \meta{BitSet B}. If \meta{BitSet B} is
% undefined, then \meta{BitSet A} will be the empty bit set.
%
% Note: If \meta{BitSet A} exists, it will be overwritten.
%
% \subsubsection{Import}\label{sec:import}
%
% \begin{declcs}{bitsetSetBin} \M{BitSet} \M{BinaryNumber}\\
%   \cs{bitsetSetOct} \M{BitSet} \M{OctalNumber}\\
%   \cs{bitsetSetHex} \M{BitSet} \M{HexadecimalNumber}
% \end{declcs}
% The numbers are interpreted as bit vectors and the flags in
% the bit \meta{BitSet} set are set accordingly.
% These numeric arguments are the only arguments where
% spaces are allowed. Then the numbers are easier to read.
%
% \begin{declcs}{bitsetSetDec} \M{BitSet} \M{DecimalNumber}
% \end{declcs}
% Macro \cs{bitsetSetDec} uses \meta{DecimalNumber} to
% set the bit set \meta{BitSet}.
% The numeric argument must expand
% to a plain number consisting of decimal digits without
% command tokens or spaces. Internally this argument is
% expanded only. It cannot be passed to \cs{number} or
% \cs{numexpr}, because the number may be too large for them.
% However \cs{number} or |\the\numexpr| may be used
% explicitely. This also helps for unexpandable number
% command tokens or registers (\cs{z@}, \cs{@ne}, \cs{count@}, \dots).
% Also \LaTeX'\ \cs{value} needs prefixing:
% \begin{quote}
% |\bitsetSetDec{foo}{\number\value{bar}}|
% \end{quote}
%
% \subsubsection{Export}\label{sec:export}
%
% \begin{declcs}{bitsetGetBin} \M{BitSet} \M{MinSize}\\
%   \cs{bitsetGetOct} \M{BitSet} \M{MinSize}\\
%   \cs{bitsetGetHex} \M{BitSet} \M{MinSize}
% \end{declcs}
% These macros returns the bit set as binary, octal or hexadecimal
% number. If the bit size is smaller than \meta{MinSize} the gap is
% filled with leading zeros.
% Example:
% \begin{quote}
% |\bitsetReset{abc}|\\
% |\bitsetSet{abc}{2}|\\
% |\bitsetGetBin{abc}{8}| $\rightarrow$ |00000100|\\
% |\bitsetSet{abc}{5}\bitsetSet{abc}{7}|\\
% |\bitsetGetHex{abc}{16}| $\rightarrow$ |00A2|
% \end{quote}
% Macro \cs{bitsetGetHex} uses the uppercase letters |A| to |F|.
% The catcode of the letters is one of 11 (letter) or 12 (other).
%
% \begin{declcs}{bitsetGetDec} \M{BitSet}
% \end{declcs}
% Macro \cs{bitsetGetDec} returns the bit set \meta{BitSet} as
% decimal number. The returned number can be larger than
% \TeX's number limit of $2^{31}-1$.
%
% \subsubsection{Logical operators}\label{sec:log}
%
% \begin{declcs}{bitsetAnd} \M{BitSet A} \M{BitSet B}
% \end{declcs}
% $A_{\text{new}} \mathrel{\mathop{:}}= A_{\text{old}}
% \mathbin{\text{and}} B \qquad (\forall\, \text{bits})$
%
% \begin{declcs}{bitsetAndNot} \M{BitSet A} \M{BitSet B}
% \end{declcs}
% $A_{\text{new}} \mathrel{\mathop{:}}= A_{\text{old}}
% \mathbin{\text{and}} (\text{not}\,B) \qquad (\forall\, \text{bits})$
%
% \begin{declcs}{bitsetOr} \M{BitSet A} \M{BitSet B}
% \end{declcs}
% $A_{\text{new}} \mathrel{\mathop{:}}= A_{\text{old}}
% \mathbin{\text{or}} B \qquad (\forall\, \text{bits})$
%
% \begin{declcs}{bitsetXor} \M{BitSet A} \M{BitSet B}
% \end{declcs}
% $A_{\text{new}} \mathrel{\mathop{:}}= A_{\text{old}}
% \mathbin{\text{xor}} B \qquad (\forall\, \text{bits})$
%
% \subsubsection{Shifting}\label{sec:shift}
%
% \begin{declcs}{bitsetShiftLeft} \M{BitSet} \M{ShiftAmount}\\
%   \cs{bitsetShiftRight} \M{BitSet} \M{ShiftAmount}
% \end{declcs}
% A left shift by one is a multiplication by two, thus left shifting
% moves the flags to higher positions. The new created low positions
% are filled by zeros.
%
% A right shift is the opposite, dividing by two, movint the bits
% to lower positions. The number will become smaller, the lowest
% bits are lost.
%
% If the \meta{ShiftAmount} is negative, it reverts the meaning of
% the shift operation. A left shift becomes a right shift.
% A \meta{ShiftAmount} of zero is ignored.
%
% \subsubsection{Bit manipulation}\label{sec:bitman}
%
% \begin{declcs}{bitsetClear} \M{BitSet} \M{Index}\\
%   \cs{bitsetSet} \M{BitSet} \M{Index}\\
%   \cs{bitsetFlip} \M{BitSet} \M{Index}
% \end{declcs}
% This macros manipulate a single bit in \meta{BitSet} addressed
% by \cs{Index}. Macro \cs{bitsetClear} disables the bit,
% \cs{bitsetSet} enables it and \cs{bitsetFlip} reverts the current
% setting of the bit.
%
% \begin{declcs}{bitsetSetValue} \M{BitSet} \M{Index} \M{Bit}
% \end{declcs}
% Macro \cs{bitsetSetValue} puts bit \meta{Bit} at position \meta{Index}
% in bit set \meta{BitSet}. \meta{Bit} must be a valid \TeX\ number
% equals to zero (disabled/cleared) or one (enabled/set).
%
% \subsubsection{Bit retrieval}\label{sec:bitret}
%
% \begin{declcs}{bitsetGet} \M{BitSet} \M{Index}
% \end{declcs}
% Macro \cs{bitsetGet} extracts the status of the bit at
% position \meta{Index} in bit set \meta{BitSet}.
% Digit |1| is returned if the bit is set/enabled. If the
% bit is cleared/disabled and in cases of an undefined bitset or
% an index number out of range the return value is |0|.
%
% \begin{declcs}{bitsetNextClearBit} \M{BitSet} \M{Index}
% \end{declcs}
% Starting at position \meta{Index} (inclusive) the bits
% are inspected. The first position without a set bit is returned.
% Possible results are decimal numbers: \meta{Index}, \meta{Index} + 1,
% \dots, ($\infty$)
%
% \begin{declcs}{bitsetNextSetBit} \M{BitSet} \M{Index}
% \end{declcs}
% Starting at position \meta{Index} (inclusive) the bits are
% inspected and the index position of the first found set bit
% is returned. If there isn't such a bit, then the result is
% |-1|. In summary possible results are decimal numbers:
% |-1|, \meta{Index}, \meta{Index} + 1, \dots, ($\infty$)
%
% \begin{declcs}{bitsetGetSetBitList} \M{BitSet}
% \end{declcs}
% Macro \cs{bitsetGetSetBitList} is an application for
% \cs{bitsetNextSetBit}. The set bits are iterated and
% returned as comma separated list of index positions
% in increasing order. The list is empty in case of
% an empty bit set.
%
% \subsubsection{Bit set properties}\label{sec:prop}
%
% \begin{declcs}{bitsetSize} \M{BitSet}
% \end{declcs}
% Macro \cs{bitsetSize} returns number of bits in use.
% It is the same as the index number of the highest set/enabled bit
% incremented by one.
%
% \begin{declcs}{bitsetCardinality} \M{BitSet}
% \end{declcs}
% Macro \cs{bitsetCardinality} counts the number of set/enabled bits.
%
% \subsubsection{Queries}\label{sec:query}
%
% Also the query procedures are expandable. They ask for a piece
% of information about a bit set and execute code depending
% on the answer.
%
% \begin{declcs}{bitsetIsDefined} \M{BitSet} \M{Then} \M{Else}
% \end{declcs}
% If the bit set with the name \meta{BitSet} exists the code
% given in \meta{Then} is executed, otherwise \meta{Else} is used.
%
% \begin{declcs}{bitsetIsEmpty} \M{BitSet} \M{Then} \M{Else}
% \end{declcs}
% If the bit set \meta{BitSet} exists and at least one bit
% is set/enabled, the code in \meta{Then} is executed,
% \meta{Else} otherwise.
%
% \begin{declcs}{bitsetEquals} \M{BitSet A} \M{BitSet B} \M{Then} \M{Else}
% \end{declcs}
% Both bit sets are equal if and only if either both are undefined
% or both are defined and represents the same bit values at the
% same positions. Thus this definition is reflexive, symmeric,
% and transitive, enough for an equivalent relation.
%
% \begin{declcs}{bitsetIntersects} \M{BitSet A} \M{BitSet B} \M{Then} \M{Else}
% \end{declcs}
% If and only if \meta{BitSet A} and \meta{BitSet B} have
% at least one bit at the same position that is set, then
% code part \meta{Then} is executed.
%
% \begin{declcs}{bitsetQuery} \M{BitSet} \M{Index} \M{Then} \M{Else}
% \end{declcs}
% It's just a wrapper for \cs{bitsetGet}. If the bit at position
% \meta{Index} is enabled, code \meta{Then} is called.
%
% \StopEventually{
% }
%
% \section{Implementation}
%
% The internal format of a bit set is quite simple, a sequence
% of digits |0| and |1|. The least significant bit is left.
% A bit set without any flag set is encoded by |0|. Also
% undefined bit sets are treated that way. After the
% highest bit that is set there are no further zeroes.
% A regular expression of valid bit sets values:
% \begin{quote}
%  |0|\string||[01]*1|
% \end{quote}
%
%    \begin{macrocode}
%<*package>
%    \end{macrocode}
%
% \subsection{Reload check and package identification}
%    Reload check, especially if the package is not used with \LaTeX.
%    \begin{macrocode}
\begingroup\catcode61\catcode48\catcode32=10\relax%
  \catcode13=5 % ^^M
  \endlinechar=13 %
  \catcode35=6 % #
  \catcode39=12 % '
  \catcode44=12 % ,
  \catcode45=12 % -
  \catcode46=12 % .
  \catcode58=12 % :
  \catcode64=11 % @
  \catcode123=1 % {
  \catcode125=2 % }
  \expandafter\let\expandafter\x\csname ver@bitset.sty\endcsname
  \ifx\x\relax % plain-TeX, first loading
  \else
    \def\empty{}%
    \ifx\x\empty % LaTeX, first loading,
      % variable is initialized, but \ProvidesPackage not yet seen
    \else
      \expandafter\ifx\csname PackageInfo\endcsname\relax
        \def\x#1#2{%
          \immediate\write-1{Package #1 Info: #2.}%
        }%
      \else
        \def\x#1#2{\PackageInfo{#1}{#2, stopped}}%
      \fi
      \x{bitset}{The package is already loaded}%
      \aftergroup\endinput
    \fi
  \fi
\endgroup%
%    \end{macrocode}
%    Package identification:
%    \begin{macrocode}
\begingroup\catcode61\catcode48\catcode32=10\relax%
  \catcode13=5 % ^^M
  \endlinechar=13 %
  \catcode35=6 % #
  \catcode39=12 % '
  \catcode40=12 % (
  \catcode41=12 % )
  \catcode44=12 % ,
  \catcode45=12 % -
  \catcode46=12 % .
  \catcode47=12 % /
  \catcode58=12 % :
  \catcode64=11 % @
  \catcode91=12 % [
  \catcode93=12 % ]
  \catcode123=1 % {
  \catcode125=2 % }
  \expandafter\ifx\csname ProvidesPackage\endcsname\relax
    \def\x#1#2#3[#4]{\endgroup
      \immediate\write-1{Package: #3 #4}%
      \xdef#1{#4}%
    }%
  \else
    \def\x#1#2[#3]{\endgroup
      #2[{#3}]%
      \ifx#1\@undefined
        \xdef#1{#3}%
      \fi
      \ifx#1\relax
        \xdef#1{#3}%
      \fi
    }%
  \fi
\expandafter\x\csname ver@bitset.sty\endcsname
\ProvidesPackage{bitset}%
  [2011/01/30 v1.1 Handle bit-vector datatype (HO)]%
%    \end{macrocode}
%
% \subsection{Catcodes}
%
%    \begin{macrocode}
\begingroup\catcode61\catcode48\catcode32=10\relax%
  \catcode13=5 % ^^M
  \endlinechar=13 %
  \catcode123=1 % {
  \catcode125=2 % }
  \catcode64=11 % @
  \def\x{\endgroup
    \expandafter\edef\csname BitSet@AtEnd\endcsname{%
      \endlinechar=\the\endlinechar\relax
      \catcode13=\the\catcode13\relax
      \catcode32=\the\catcode32\relax
      \catcode35=\the\catcode35\relax
      \catcode61=\the\catcode61\relax
      \catcode64=\the\catcode64\relax
      \catcode123=\the\catcode123\relax
      \catcode125=\the\catcode125\relax
    }%
  }%
\x\catcode61\catcode48\catcode32=10\relax%
\catcode13=5 % ^^M
\endlinechar=13 %
\catcode35=6 % #
\catcode64=11 % @
\catcode123=1 % {
\catcode125=2 % }
\def\TMP@EnsureCode#1#2{%
  \edef\BitSet@AtEnd{%
    \BitSet@AtEnd
    \catcode#1=\the\catcode#1\relax
  }%
  \catcode#1=#2\relax
}
\TMP@EnsureCode{33}{12}% !
\TMP@EnsureCode{39}{12}% '
\TMP@EnsureCode{40}{12}% (
\TMP@EnsureCode{41}{12}% )
\TMP@EnsureCode{42}{12}% *
\TMP@EnsureCode{43}{12}% +
\TMP@EnsureCode{44}{12}% ,
\TMP@EnsureCode{45}{12}% -
\TMP@EnsureCode{46}{12}% .
\TMP@EnsureCode{47}{12}% /
\TMP@EnsureCode{58}{11}% : (letter!)
\TMP@EnsureCode{60}{12}% <
\TMP@EnsureCode{62}{12}% >
\TMP@EnsureCode{63}{14}% ? (comment!)
\TMP@EnsureCode{91}{12}% [
\TMP@EnsureCode{93}{12}% ]
\TMP@EnsureCode{96}{12}% `
\edef\BitSet@AtEnd{\BitSet@AtEnd\noexpand\endinput}
\begingroup\expandafter\expandafter\expandafter\endgroup
\expandafter\ifx\csname BitSet@TestMode\endcsname\relax
\else
  \catcode63=9 % ? (ignore)
\fi
? \let\BitSet@@TestMode\BitSet@TestMode
%    \end{macrocode}
%
% \subsection{Package loading}
%
%    \begin{macrocode}
\begingroup\expandafter\expandafter\expandafter\endgroup
\expandafter\ifx\csname RequirePackage\endcsname\relax
  \def\TMP@RequirePackage#1[#2]{%
    \begingroup\expandafter\expandafter\expandafter\endgroup
    \expandafter\ifx\csname ver@#1.sty\endcsname\relax
      \input #1.sty\relax
    \fi
  }%
  \TMP@RequirePackage{infwarerr}[2007/09/09]%
  \TMP@RequirePackage{intcalc}[2007/09/27]%
  \TMP@RequirePackage{bigintcalc}[2007/09/27]%
\else
  \RequirePackage{infwarerr}[2007/09/09]%
  \RequirePackage{intcalc}[2007/09/27]%
  \RequirePackage{bigintcalc}[2007/09/27]%
\fi
%    \end{macrocode}
%
% \subsection{Help macros}
%
% \subsubsection{Number constant}
%
%    \begin{macro}{\BitSet@MaxSize}
%    \begin{macrocode}
\def\BitSet@MaxSize{2147483647}%
%    \end{macrocode}
%    \end{macro}
%
% \subsubsection{General basic macros}
%
%    \begin{macro}{\BitSet@Empty}
%    \begin{macrocode}
\def\BitSet@Empty{}
%    \end{macrocode}
%    \end{macro}
%    \begin{macro}{\BitSet@FirstOfOne}
%    \begin{macrocode}
\def\BitSet@FirstOfOne#1{#1}
%    \end{macrocode}
%    \end{macro}
%    \begin{macro}{\BitSet@Gobble}
%    \begin{macrocode}
\def\BitSet@Gobble#1{}
%    \end{macrocode}
%    \end{macro}
%    \begin{macro}{\BitSet@FirstOfTwo}
%    \begin{macrocode}
\def\BitSet@FirstOfTwo#1#2{#1}
%    \end{macrocode}
%    \end{macro}
%    \begin{macro}{\BitSet@SecondOfTwo}
%    \begin{macrocode}
\def\BitSet@SecondOfTwo#1#2{#2}
%    \end{macrocode}
%    \end{macro}
%
%    \begin{macro}{\BitSet@Space}
%    \begin{macrocode}
\def\BitSet@Space{ }
%    \end{macrocode}
%    \end{macro}
%    \begin{macro}{\BitSet@ZapSpace}
%    \begin{macrocode}
\def\BitSet@ZapSpace#1 #2{%
  #1%
  \ifx\BitSet@Empty#2%
  \else
    \expandafter\BitSet@ZapSpace
  \fi
  #2%
}
%    \end{macrocode}
%    \end{macro}
%
% \subsubsection{Tail recursion}
%
%    \begin{macro}{\BitSet@Fi}
%    \begin{macrocode}
\let\BitSet@Fi\fi
%    \end{macrocode}
%    \end{macro}
%    \begin{macro}{\BitSet@AfterFi}
%    \begin{macrocode}
\def\BitSet@AfterFi#1#2\BitSet@Fi{\fi#1}
%    \end{macrocode}
%    \end{macro}
%    \begin{macro}{\BitSet@AfterFiFi}
%    \begin{macrocode}
\def\BitSet@AfterFiFi#1#2\BitSet@Fi{\fi\fi#1}%
%    \end{macrocode}
%    \end{macro}
%    \begin{macro}{\BitSet@AfterFiFiFi}
%    \begin{macrocode}
\def\BitSet@AfterFiFiFi#1#2\BitSet@Fi{\fi\fi\fi#1}%
%    \end{macrocode}
%    \end{macro}
%
% \subsubsection{Check macros}
%
%    \begin{macro}{\BitSet@IfUndefined}
%    \begin{macrocode}
\def\BitSet@IfUndefined#1{%
  \expandafter\ifx\csname BS@#1\endcsname\relax
    \expandafter\BitSet@FirstOfTwo
  \else
    \expandafter\BitSet@SecondOfTwo
  \fi
}
%    \end{macrocode}
%    \end{macro}
%
%    \begin{macro}{\BitSet@CheckIndex}
%    |#1|: continuation code\\
%    |#2|: BitSet\\
%    |#3|: Index
%    \begin{macrocode}
\def\BitSet@CheckIndex#1#2#3{%
  \BitSet@IfUndefined{#2}{\bitsetReset{#2}}{}%
  \expandafter\expandafter\expandafter\BitSet@@CheckIndex
  \intcalcNum{#3}!%
  {#2}{#1}%
}
%    \end{macrocode}
%    \end{macro}
%    \begin{macro}{\BitSet@@CheckIndex}
%    |#1|: plain Index\\
%    |#2|: BitSet\\
%    |#3|: continuation code
%    \begin{macrocode}
\def\BitSet@@CheckIndex#1!#2#3{%
  \ifnum#1<0 %
    \BitSet@AfterFi{%
      \@PackageError{bitset}{%
        Invalid negative index (#1)%
      }\@ehc
    }%
  \else
    \BitSet@AfterFi{%
      #3{#2}{#1}%
    }%
  \BitSet@Fi
}
%    \end{macrocode}
%    \end{macro}
%
% \subsection{Miscellaneous}\label{src:misc}
%
%    \begin{macro}{\bitsetReset}
%    \begin{macrocode}
\def\bitsetReset#1{%
  \expandafter\def\csname BS@#1\endcsname{0}%
}
%    \end{macrocode}
%    \end{macro}
%
%    \begin{macro}{\bitsetLet}
%    \begin{macrocode}
\def\bitsetLet#1#2{%
  \BitSet@IfUndefined{#2}{%
    \bitsetReset{#1}%
  }{%
    \expandafter\let\csname BS@#1\expandafter\endcsname
                    \csname BS@#2\endcsname
  }%
}
%    \end{macrocode}
%    \end{macro}
%
% \subsection{Import}\label{src:import}
%
% \subsubsection{From binary number}
%
%    \begin{macro}{\bitsetSetBin}
%    \begin{macrocode}
\def\bitsetSetBin#1#2{%
  \edef\BitSet@Temp{#2}%
  \edef\BitSet@Temp{%
    \expandafter\expandafter\expandafter\BitSet@ZapSpace
    \expandafter\BitSet@Temp\BitSet@Space\BitSet@Empty
  }%
  \edef\BitSet@Temp{%
    \expandafter\BitSet@KillZeros\BitSet@Temp\BitSet@Empty
  }%
  \ifx\BitSet@Temp\BitSet@Empty
    \expandafter\let\csname BS@#1\endcsname\BitSet@Zero
  \else
    \expandafter\edef\csname BS@#1\endcsname{%
      \expandafter\BitSet@Reverse\BitSet@Temp!%
    }%
  \fi
}
%    \end{macrocode}
%    \end{macro}
%    \begin{macro}{\BitSet@KillZeros}
%    \begin{macrocode}
\def\BitSet@KillZeros#1{%
  \ifx#10%
    \expandafter\BitSet@KillZeros
  \else
    #1%
  \fi
}
%    \end{macrocode}
%    \end{macro}
%    \begin{macro}{\BitSet@Reverse}
%    \begin{macrocode}
\def\BitSet@Reverse#1#2!{%
  \ifx\\#2\\%
    #1%
  \else
    \BitSet@AfterFi{%
      \BitSet@Reverse#2!#1%
    }%
  \BitSet@Fi
}
%    \end{macrocode}
%    \end{macro}
%
% \subsubsection{From octal/hex number}
%
%    \begin{macro}{\bitsetSetOct}
%    \begin{macrocode}
\def\bitsetSetOct{%
  \BitSet@SetOctHex\BitSet@FromFirstOct
}
%    \end{macrocode}
%    \end{macro}
%    \begin{macro}{\bitsetSetHex}
%    \begin{macrocode}
\def\bitsetSetHex{%
  \BitSet@SetOctHex\BitSet@FromFirstHex
}
%    \end{macrocode}
%    \end{macro}
%    \begin{macro}{\BitSet@SetOctHex}
%    \begin{macrocode}
\def\BitSet@SetOctHex#1#2#3{%
  \edef\BitSet@Temp{#3}%
  \edef\BitSet@Temp{%
    \expandafter\expandafter\expandafter\BitSet@ZapSpace
    \expandafter\BitSet@Temp\BitSet@Space\BitSet@Empty
  }%
  \edef\BitSet@Temp{%
    \expandafter\BitSet@KillZeros\BitSet@Temp\BitSet@Empty
  }%
  \ifx\BitSet@Temp\BitSet@Empty
    \expandafter\let\csname BS@#2\endcsname\BitSet@Zero
  \else
    \edef\BitSet@Temp{%
      \expandafter#1\BitSet@Temp!%
    }%
    \ifx\BitSet@Temp\BitSet@Empty
      \expandafter\let\csname BS@#2\endcsname\BitSet@Zero
    \else
      \expandafter\edef\csname BS@#2\endcsname{%
        \expandafter\BitSet@Reverse\BitSet@Temp!%
      }%
    \fi
  \fi
}
%    \end{macrocode}
%    \end{macro}
%    \begin{macro}{\BitSet@FromFirstOct}
%    \begin{macrocode}
\def\BitSet@FromFirstOct#1{%
  \ifx#1!%
  \else
    \ifcase#1 \BitSet@AfterFiFi\BitSet@FromFirstOct
    \or 1%
    \or 10%
    \or 11%
    \or 100%
    \or 101%
    \or 110%
    \or 111%
    \else \BitSetError:WrongOctalDigit%
    \fi
    \expandafter\BitSet@FromOct
  \BitSet@Fi
}
%    \end{macrocode}
%    \end{macro}
%    \begin{macro}{\BitSet@FromOct}
%    \begin{macrocode}
\def\BitSet@FromOct#1{%
  \ifx#1!%
  \else
    \ifcase#1 000%
    \or 001%
    \or 010%
    \or 011%
    \or 100%
    \or 101%
    \or 110%
    \or 111%
    \else \BitSetError:WrongOctalDigit%
    \fi
    \expandafter\BitSet@FromOct
  \fi
}
%    \end{macrocode}
%    \end{macro}
%
%    \begin{macro}{\BitSet@FromFirstHex}
%    \begin{macrocode}
\def\BitSet@FromFirstHex#1{%
  \ifx#1!%
  \else
    \ifx#10%
      \BitSet@AfterFiFi\BitSet@FromFirstHex
    \fi
    \expandafter\ifx\csname BitSet@Hex#1\endcsname\relax
      \BitSetError:InvalidHexDigit%
    \else
      \expandafter\expandafter\expandafter\BitSet@KillZeros
      \csname BitSet@Hex#1\endcsname
    \fi
    \expandafter\BitSet@FromHex
  \BitSet@Fi
}
%    \end{macrocode}
%    \end{macro}
%    \begin{macro}{\BitSet@FromHex}
%    \begin{macrocode}
\def\BitSet@FromHex#1{%
  \ifx#1!%
  \else
    \expandafter\ifx\csname BitSet@Hex#1\endcsname\relax
      \BitSetError:InvalidHexDigit%
    \else
      \csname BitSet@Hex#1\endcsname
    \fi
    \expandafter\BitSet@FromHex
  \fi
}
%    \end{macrocode}
%    \end{macro}
%    \begin{macro}{\BitSet@Hex[0..F]}
%    \begin{macrocode}
\def\BitSet@Temp#1{%
  \expandafter\def\csname BitSet@Hex#1\endcsname
}
\BitSet@Temp 0{0000}%
\BitSet@Temp 1{0001}%
\BitSet@Temp 2{0010}%
\BitSet@Temp 3{0011}%
\BitSet@Temp 4{0100}%
\BitSet@Temp 5{0101}%
\BitSet@Temp 6{0110}%
\BitSet@Temp 7{0111}%
\BitSet@Temp 8{1000}%
\BitSet@Temp 9{1001}%
\BitSet@Temp A{1010}%
\BitSet@Temp B{1011}%
\BitSet@Temp C{1100}%
\BitSet@Temp D{1101}%
\BitSet@Temp E{1110}%
\BitSet@Temp F{1111}%
\BitSet@Temp a{1010}%
\BitSet@Temp b{1011}%
\BitSet@Temp c{1100}%
\BitSet@Temp d{1101}%
\BitSet@Temp e{1110}%
\BitSet@Temp f{1111}%
%    \end{macrocode}
%    \end{macro}
%
% \subsubsection{From decimal number}
%
%    \begin{macro}{\bitsetSetDec}
%    \begin{macrocode}
\def\bitsetSetDec#1#2{%
  \edef\BitSet@Temp{#2}%
  \edef\BitSet@Temp{%
    \expandafter\expandafter\expandafter\BitSet@ZapSpace
    \expandafter\BitSet@Temp\BitSet@Space\BitSet@Empty
  }%
  \edef\BitSet@Temp{%
    \expandafter\BitSet@KillZeros\BitSet@Temp\BitSet@Empty
  }%
  \ifx\BitSet@Temp\BitSet@Empty
    \expandafter\let\csname BS@#1\endcsname\BitSet@Zero
  \else
    \ifcase\bigintcalcSgn{\BitSet@Temp} %
      \expandafter\let\csname BS@#1\endcsname\BitSet@Zero
    \or
      \ifnum\bigintcalcCmp\BitSet@Temp\BitSet@MaxSize>0 %
        \expandafter\edef\csname BS@#1\endcsname{%
          \expandafter\BitSet@SetDecBig\BitSet@Temp!%
        }%
      \else
        \expandafter\edef\csname BS@#1\endcsname{%
          \expandafter\BitSet@SetDec\BitSet@Temp!%
        }%
      \fi
    \else
      \@PackageError{bitset}{%
        Bit sets cannot be negative%
      }\@ehc
    \fi
  \fi
}
%    \end{macrocode}
%    \end{macro}
%    \begin{macro}{\BitSet@SetDecBig}
%    \begin{macrocode}
\def\BitSet@SetDecBig#1#2#3#4#5#6#7#8#9!{%
  \ifx\\#9\\%
    \BitSet@SetDec#1#2#3#4#5#6#7#8!%
  \else
    \ifcase\BigIntCalcOdd#1#2#4#5#6#7#8#9! %
      0%
    \or
      1%
?   \else\BitSetError:ThisCannotHappen%
    \fi
    \BitSet@AfterFi{%
      \expandafter\expandafter\expandafter\BitSet@SetDecBig
      \BigIntCalcShr#1#2#3#4#5#6#7#8#9!!%
    }%
  \BitSet@Fi
}
%    \end{macrocode}
%    \end{macro}
%    \begin{macro}{\BitSet@SetDec}
%    \begin{macrocode}
\def\BitSet@SetDec#1!{%
  \ifcase#1 %
  \or 1%
  \else
    \ifodd#1 %
      1%
    \else
      0%
    \fi
    \BitSet@AfterFi{%
      \expandafter\expandafter\expandafter\BitSet@SetDec
      \IntCalcShr#1!!%
    }%
  \BitSet@Fi
}
%    \end{macrocode}
%    \end{macro}
%
% \subsection{Export}\label{src:export}
%
% \subsubsection{To binary number}
%
%    \begin{macro}{\bitsetGetBin}
%    \begin{macrocode}
\def\bitsetGetBin#1#2{%
  \romannumeral0%
  \expandafter\expandafter\expandafter\BitSet@@GetBin
  \intcalcNum{#2}!{#1}%
}
%    \end{macrocode}
%    \end{macro}
%    \begin{macro}{\BitSet@@GetBin}
%    \begin{macrocode}
\def\BitSet@@GetBin#1!#2{%
  \BitSet@IfUndefined{#2}{%
    \ifnum#1>1 %
      \BitSet@AfterFi{%
        \expandafter\expandafter\expandafter\BitSet@Fill
        \IntCalcDec#1!!0%
      }%
    \else
      \BitSet@AfterFi{ 0}%
    \BitSet@Fi
  }{%
    \expandafter\expandafter\expandafter\BitSet@NumBinRev
    \expandafter\expandafter\expandafter1%
    \expandafter\expandafter\expandafter!%
    \csname BS@#2\endcsname!!#1!%
  }%
}
%    \end{macrocode}
%    \end{macro}
%    \begin{macro}{\BitSet@Fill}
%    |#1|: number of leading digits |0|\\
%    |#2|: result
%    \begin{macrocode}
\def\BitSet@Fill#1!{%
  \ifnum#1>0 %
    \BitSet@AfterFi{%
      \expandafter\expandafter\expandafter\BitSet@Fill
      \IntCalcDec#1!!0%
    }%
  \else
    \BitSet@AfterFi{ }%
  \BitSet@Fi
}
%    \end{macrocode}
%    \end{macro}
%
%    \begin{macro}{\BitSet@NumBinRev}
%    |#1|: bit counter (including |#2|)\\
%    |#2#3|: reverted number\\
%    |#4|: result\\
%    |#5|: min size
%    \begin{macrocode}
\def\BitSet@NumBinRev#1!#2#3!{%
  \ifx\\#3\\%
    \BitSet@AfterFi{%
      \BitSet@NumBinFill#1!#2%
    }%
  \else
    \BitSet@AfterFi{%
      \expandafter\expandafter\expandafter\BitSet@NumBinRev
      \IntCalcInc#1!!#3!#2%
    }%
  \BitSet@Fi
}
%    \end{macrocode}
%    \end{macro}
%    \begin{macro}{\BitSet@NumBinFill}
%    \begin{macrocode}
\def\BitSet@NumBinFill#1!#2!#3!{%
  \ifnum#3>#1 %
    \BitSet@AfterFi{%
      \expandafter\expandafter\expandafter\BitSet@Fill
      \IntCalcSub#3!#1!!#2%
    }%
  \else
    \BitSet@AfterFi{ #2}%
  \BitSet@Fi
}
%    \end{macrocode}
%    \end{macro}
%
% \subsubsection{To octal/hexadecimal number}
%
%    \begin{macro}{\bitsetGetOct}
%    \begin{macrocode}
\def\bitsetGetOct#1#2{%
  \romannumeral0%
  \bitsetIsEmpty{#1}{%
    \expandafter\expandafter\expandafter\BitSet@@GetOctHex
    \intcalcNum{#2}!3!230%
  }{%
    \expandafter\expandafter\expandafter\BitSet@@GetOct
    \expandafter\expandafter\expandafter1%
    \expandafter\expandafter\expandafter!%
    \expandafter\expandafter\expandafter!%
    \csname BS@#1\endcsname00%
    \BitSet@Empty\BitSet@Empty\BitSet@Empty!{#2}%
  }%
}
%    \end{macrocode}
%    \end{macro}
%    \begin{macro}{\bitsetGetHex}
%    \begin{macrocode}
\def\bitsetGetHex#1#2{%
  \romannumeral0%
  \bitsetIsEmpty{#1}{%
    \expandafter\expandafter\expandafter\BitSet@@GetOctHex
    \intcalcNum{#2}!4!340%
  }{%
    \expandafter\expandafter\expandafter\BitSet@@GetHex
    \expandafter\expandafter\expandafter1%
    \expandafter\expandafter\expandafter!%
    \expandafter\expandafter\expandafter!%
    \csname BS@#1\endcsname000%
    \BitSet@Empty\BitSet@Empty\BitSet@Empty\BitSet@Empty!{#2}%
  }%
}
%    \end{macrocode}
%    \end{macro}
%    \begin{macro}{\BitSet@@GetOct}
%    |#1|: number of digits\\
%    |#2|: result\\
%    |#3#4#5|: bits
%    \begin{macrocode}
\def\BitSet@@GetOct#1!#2!#3#4#5{%
  \ifx#5\BitSet@Empty
    \BitSet@AfterFi{%
      \expandafter\expandafter\expandafter\BitSet@GetOctHex
      \IntCalcDec#1!!#2!23%
    }%
  \else
    \BitSet@AfterFi{%
      \expandafter\expandafter\expandafter\BitSet@@GetOct
      \number\IntCalcInc#1!\expandafter\expandafter\expandafter!%
      \csname BitSet@Oct#5#4#3\endcsname#2!%
    }%
  \BitSet@Fi
}
%    \end{macrocode}
%    \end{macro}
%    \begin{macro}{\BitSet@Oct[000..111]}
%    \begin{macrocode}
\def\BitSet@Temp#1#2#3#4{%
  \expandafter\def\csname BitSet@Oct#1#2#3\endcsname{#4}%
}
\BitSet@Temp0000%
\BitSet@Temp0011%
\BitSet@Temp0102%
\BitSet@Temp0113%
\BitSet@Temp1004%
\BitSet@Temp1015%
\BitSet@Temp1106%
\BitSet@Temp1117%
%    \end{macrocode}
%    \end{macro}
%    \begin{macro}{\BitSet@@GetHex}
%    |#1|: number of digits\\
%    |#2|: result\\
%    |#3#4#5#6|: bits
%    \begin{macrocode}
\def\BitSet@@GetHex#1!#2!#3#4#5#6{%
  \ifx#6\BitSet@Empty
    \BitSet@AfterFi{%
      \expandafter\expandafter\expandafter\BitSet@GetOctHex
      \IntCalcDec#1!!#2!34%
    }%
  \else
    \BitSet@AfterFi{%
      \expandafter\expandafter\expandafter\BitSet@@GetHex
      \number\IntCalcInc#1!\expandafter\expandafter\expandafter!%
      \csname BitSet@Hex#6#5#4#3\endcsname#2!%
    }%
  \BitSet@Fi
}
%    \end{macrocode}
%    \end{macro}
%    \begin{macro}{\BitSet@Hex[0000..1111]}
%    \begin{macrocode}
\def\BitSet@Temp#1#2#3#4#5{%
  \expandafter\def\csname BitSet@Hex#1#2#3#4\endcsname{#5}%
}
\BitSet@Temp00000%
\BitSet@Temp00011%
\BitSet@Temp00102%
\BitSet@Temp00113%
\BitSet@Temp01004%
\BitSet@Temp01015%
\BitSet@Temp01106%
\BitSet@Temp01117%
\BitSet@Temp10008%
\BitSet@Temp10019%
\BitSet@Temp1010A%
\BitSet@Temp1011B%
\BitSet@Temp1100C%
\BitSet@Temp1101D%
\BitSet@Temp1110E%
\BitSet@Temp1111F%
%    \end{macrocode}
%    \end{macro}
%    \begin{macro}{\BitSet@GetOctHex}
%    Leading zeros $ (|#4| - |#1| * 3 + 2)/3 $ if $ |#4| > |#1| * 3$\\
%    |#1|: digit size\\
%    |#2|: result\\
%    |#3|: bits per digit - 1\\
%    |#4|: bits per digit
%    |#5|: garbage\\
%    |#6|: min size
%    \begin{macrocode}
\def\BitSet@GetOctHex#1!#2!#3#4#5!#6{%
  \expandafter\BitSet@@GetOctHex
  \number\intcalcNum{#6}\expandafter\expandafter\expandafter!%
  \IntCalcMul#1!#4!!#3#4#2%
}
%    \end{macrocode}
%    \end{macro}
%    \begin{macro}{\BitSet@@GetOctHex}
%    |#1|: plain min size\\
%    |#2|: digits * (bits per digit)\\
%    |#3|: bits per digit - 1\\
%    |#4|: bits per digit
%    \begin{macrocode}
\def\BitSet@@GetOctHex#1!#2!#3#4{%
  \ifnum#1>#2 %
    \BitSet@AfterFi{%
      \expandafter\expandafter\expandafter\expandafter
      \expandafter\expandafter\expandafter\BitSet@Fill
      \expandafter\IntCalcDiv\number
      \expandafter\expandafter\expandafter\IntCalcAdd
      \IntCalcSub#1!#2!!#3!!#4!!%
    }%
  \else
    \BitSet@AfterFi{ }%
  \BitSet@Fi
}
%    \end{macrocode}
%    \end{macro}
%
% \subsubsection{To decimal number}
%
%    \begin{macro}{\bitsetGetDec}
%    \begin{macrocode}
\def\bitsetGetDec#1{%
  \romannumeral0%
  \BitSet@IfUndefined{#1}{ 0}{%
    \expandafter\expandafter\expandafter\BitSet@GetDec
    \csname BS@#1\endcsname!%
  }%
}
%    \end{macrocode}
%    \end{macro}
%    \begin{macro}{\BitSet@GetDec}
%    \begin{macrocode}
\def\BitSet@GetDec#1#2!{%
  \ifx\\#2\\%
    \BitSet@AfterFi{ #1}%
  \else
    \BitSet@AfterFi{%
      \BitSet@@GetDec2!#1!#2!%
    }%
  \BitSet@Fi
}
%    \end{macrocode}
%    \end{macro}
%    \begin{macro}{\BitSet@@GetDec}
%    |#1|: power of two\\
%    |#2|: result\\
%    |#3#4|: number
%    \begin{macrocode}
\def\BitSet@@GetDec#1!#2!#3#4!{%
  \ifx\\#4\\%
    \ifx#31%
      \BitSet@AfterFiFi{%
        \expandafter\expandafter\expandafter\BitSet@Space
        \IntCalcAdd#1!#2!%
      }%
    \else
      \BitSet@AfterFiFi{ #2}%
    \fi
  \else
    \ifx#31%
      \BitSet@AfterFiFi{%
        \csname BitSet@N#1%
        \expandafter\expandafter\expandafter\endcsname
        \IntCalcAdd#1!#2!!#4!%
      }%
    \else
      \BitSet@AfterFiFi{%
        \csname BitSet@N#1\endcsname#2!#4!%
      }%
    \fi
  \BitSet@Fi
}
%    \end{macrocode}
%    \end{macro}
%    \begin{macro}{\BitSet@N[1,2,4,...]}
%    \begin{macrocode}
\def\BitSet@Temp#1#2{%
  \expandafter\def\csname BitSet@N#1\endcsname{%
    \BitSet@@GetDec#2!%
  }%
}
\BitSet@Temp{1}{2}
\BitSet@Temp{2}{4}
\BitSet@Temp{4}{8}
\BitSet@Temp{8}{16}
\BitSet@Temp{16}{32}
\BitSet@Temp{32}{64}
\BitSet@Temp{64}{128}
\BitSet@Temp{128}{256}
\BitSet@Temp{256}{512}
\BitSet@Temp{512}{1024}
\BitSet@Temp{1024}{2048}
\BitSet@Temp{2048}{4096}
\BitSet@Temp{4096}{8192}
\BitSet@Temp{8192}{16384}
\BitSet@Temp{16384}{32768}
\BitSet@Temp{32768}{65536}
\BitSet@Temp{65536}{131072}
\BitSet@Temp{131072}{262144}
\BitSet@Temp{262144}{524288}
\BitSet@Temp{524288}{1048576}
\BitSet@Temp{1048576}{2097152}
\BitSet@Temp{2097152}{4194304}
\BitSet@Temp{4194304}{8388608}
\BitSet@Temp{8388608}{16777216}
\BitSet@Temp{16777216}{33554432}
\BitSet@Temp{33554432}{67108864}
\BitSet@Temp{67108864}{134217728}
\BitSet@Temp{134217728}{268435456}
\BitSet@Temp{268435456}{536870912}
\BitSet@Temp{536870912}{1073741824}
%    \end{macrocode}
%    \end{macro}
%    \begin{macro}{\BitSet@N1073741824}
%    \begin{macrocode}
\expandafter\def\csname BitSet@N1073741824\endcsname{%
  \BitSet@GetDecBig2147483648!%
}%
%    \end{macrocode}
%    \end{macro}
%
%    \begin{macro}{\BitSet@GetDecBig}
%    |#1|: current power of two\\
%    |#2|: result\\
%    |#3#4|: number
%    \begin{macrocode}
\def\BitSet@GetDecBig#1!#2!#3#4!{%
  \ifx\\#4\\%
    \ifx#31%
      \BitSet@AfterFiFi{%
        \expandafter\expandafter\expandafter\BitSet@Space
        \BigIntCalcAdd#1!#2!%
      }%
    \else
      \BitSet@AfterFiFi{ #2}%
    \fi
  \else
    \ifx#31%
      \BitSet@AfterFiFi{%
        \expandafter\expandafter\expandafter\BitSet@@GetDecBig
        \BigIntCalcAdd#1!#2!!#1!#4!%
      }%
    \else
      \BitSet@AfterFiFi{%
        \expandafter\expandafter\expandafter\BitSet@GetDecBig
        \BigIntCalcShl#1!!#2!#4!%
      }%
    \fi
  \BitSet@Fi
}
%    \end{macrocode}
%    \end{macro}
%    \begin{macro}{\BitSet@@GetDecBig}
%    |#1|: result\\
%    |#2|: power of two\\
%    |#3#4|: number
%    \begin{macrocode}
\def\BitSet@@GetDecBig#1!#2!{%
  \expandafter\expandafter\expandafter\BitSet@GetDecBig
  \BigIntCalcShl#2!!#1!%
}
%    \end{macrocode}
%    \end{macro}
%
% \subsection{Logical operators}\label{src:log}
%
% \subsubsection{\cs{bitsetAnd}}
%
%    \begin{macro}{\bitsetAnd}
%    Decision table for \cs{bitsetAnd}:
%    \begin{quote}
%    \begin{tabular}{l|l|l|l|}
%                         & undef(B)   & empty(B)   & cardinality(B)$>$0\\
%      \hline
%      undef(A)           & A := empty & A := empty & A := empty\\
%      \hline
%      empty(A)           &            &            &\\
%      \hline
%      cardinality(A)$>$0 & A := empty & A := empty & A \&= B\\
%      \hline
%    \end{tabular}
%    \end{quote}
%    \begin{macrocode}
\def\bitsetAnd#1#2{%
  \bitsetIsEmpty{#1}{%
    \bitsetReset{#1}%
  }{%
    \bitsetIsEmpty{#2}{%
      \bitsetReset{#1}%
    }{%
      \expandafter\edef\csname BS@#1\endcsname{%
        \expandafter\expandafter\expandafter\BitSet@And
        \csname BS@#1\expandafter\expandafter\expandafter\endcsname
        \expandafter\expandafter\expandafter!%
        \csname BS@#2\endcsname!!%
      }%
      \expandafter\ifx\csname BS@#1\endcsname\BitSet@Empty
        \bitsetReset{#1}%
      \fi
    }%
  }%
}
%    \end{macrocode}
%    \end{macro}
%    \begin{macro}{\BitSet@And}
%    \begin{macrocode}
\def\BitSet@And#1#2!#3#4!#5!{%
  \ifx\\#2\\%
    \ifnum#1#3=11 #51\fi
  \else
    \ifx\\#4\\%
      \ifnum#1#3=11 #51\fi
    \else
      \ifnum#1#3=11 %
        #51%
        \BitSet@AfterFiFiFi{%
          \BitSet@And#2!#4!!%
        }%
      \else
        \BitSet@AfterFiFiFi{%
          \BitSet@And#2!#4!#50!%
        }%
      \fi
    \fi
  \BitSet@Fi
}
%    \end{macrocode}
%    \end{macro}
%
% \subsubsection{\cs{bitsetAndNot}}
%
%    \begin{macro}{\bitsetAndNot}
%    Decision table for \cs{bitsetAndNot}:
%    \begin{quote}
%    \begin{tabular}{l|l|l|l|}
%                         & undef(B)   & empty(B)   & cardinality(B)$>$0\\
%      \hline
%      undef(A)           & A := empty & A := empty & A := empty\\
%      \hline
%      empty(A)           &            &            &\\
%      \hline
%      cardinality(A)$>$0 &            &            & A \&= !B\\
%      \hline
%    \end{tabular}
%    \end{quote}
%    \begin{macrocode}
\def\bitsetAndNot#1#2{%
  \bitsetIsEmpty{#1}{%
    \bitsetReset{#1}%
  }{%
    \bitsetIsEmpty{#2}{%
    }{%
      \expandafter\edef\csname BS@#1\endcsname{%
        \expandafter\expandafter\expandafter\BitSet@AndNot
        \csname BS@#1\expandafter\expandafter\expandafter\endcsname
        \expandafter\expandafter\expandafter!%
        \csname BS@#2\endcsname!!%
      }%
      \expandafter\ifx\csname BS@#1\endcsname\BitSet@Empty
        \bitsetReset{#1}%
      \fi
    }%
  }%
}
%    \end{macrocode}
%    \end{macro}
%    \begin{macro}{\BitSet@AndNot}
%    \begin{macrocode}
\def\BitSet@AndNot#1#2!#3#4!#5!{%
  \ifx\\#2\\%
    \ifnum#1#3=10 #51\fi
  \else
    \ifx\\#4\\%
      #5%
      \ifnum#1#3=10 1\else 0\fi
      #2%
    \else
      \ifnum#1#3=10 %
        #51%
        \BitSet@AfterFiFiFi{%
          \BitSet@AndNot#2!#4!!%
        }%
      \else
        \BitSet@AfterFiFiFi{%
          \BitSet@AndNot#2!#4!#50!%
        }%
      \fi
    \fi
  \BitSet@Fi
}
%    \end{macrocode}
%    \end{macro}
%
% \subsubsection{\cs{bitsetOr}}
%
%    \begin{macro}{\bitsetOr}
%    Decision table for \cs{bitsetOr}:
%    \begin{quote}
%    \begin{tabular}{l|l|l|l|}
%                         & undef(B)   & empty(B)   & cardinality(B)$>$0\\
%      \hline
%      undef(A)           & A := empty & A := empty & A := B\\
%      \hline
%      empty(A)           &            &            & A := B\\
%      \hline
%      cardinality(A)$>$0 &            &            & A \string|= B\\
%      \hline
%    \end{tabular}
%    \end{quote}
%    \begin{macrocode}
\def\bitsetOr#1#2{%
  \bitsetIsEmpty{#2}{%
    \BitSet@IfUndefined{#1}{\bitsetReset{#1}}{}%
  }{%
    \bitsetIsEmpty{#1}{%
      \expandafter\let\csname BS@#1\expandafter\endcsname
                      \csname BS@#2\endcsname
    }{%
      \expandafter\edef\csname BS@#1\endcsname{%
        \expandafter\expandafter\expandafter\BitSet@Or
        \csname BS@#1\expandafter\expandafter\expandafter\endcsname
        \expandafter\expandafter\expandafter!%
        \csname BS@#2\endcsname!%
      }%
    }%
  }%
}
%    \end{macrocode}
%    \end{macro}
%    \begin{macro}{\BitSet@Or}
%    \begin{macrocode}
\def\BitSet@Or#1#2!#3#4!{%
  \ifnum#1#3>0 1\else 0\fi
  \ifx\\#2\\%
    #4%
  \else
    \ifx\\#4\\%
      #2%
    \else
      \BitSet@AfterFiFi{%
        \BitSet@Or#2!#4!%
      }%
    \fi
  \BitSet@Fi
}
%    \end{macrocode}
%    \end{macro}
%
% \subsubsection{\cs{bitsetXor}}
%
%    \begin{macro}{\bitsetXor}
%    Decision table for \cs{bitsetXor}:
%    \begin{quote}
%    \begin{tabular}{l|l|l|l|}
%                         & undef(B)   & empty(B)   & cardinality(B)$>$0\\
%      \hline
%      undef(A)           & A := empty & A := empty & A := B\\
%      \hline
%      empty(A)           &            &            & A := B\\
%      \hline
%      cardinality(A)$>$0 &            &            & A \string^= B\\
%      \hline
%    \end{tabular}
%    \end{quote}
%    \begin{macrocode}
\def\bitsetXor#1#2{%
  \bitsetIsEmpty{#2}{%
    \BitSet@IfUndefined{#1}{\bitsetReset{#1}}{}%
  }{%
    \bitsetIsEmpty{#1}{%
      \expandafter\let\csname BS@#1\expandafter\endcsname
                      \csname BS@#2\endcsname
    }{%
      \expandafter\edef\csname BS@#1\endcsname{%
        \expandafter\expandafter\expandafter\BitSet@Xor
        \csname BS@#1\expandafter\expandafter\expandafter\endcsname
        \expandafter\expandafter\expandafter!%
        \csname BS@#2\endcsname!!%
      }%
      \expandafter\ifx\csname BS@#1\endcsname\BitSet@Empty
        \bitsetReset{#1}%
      \fi
    }%
  }%
}
%    \end{macrocode}
%    \end{macro}
%    \begin{macro}{\BitSet@Xor}
%    \begin{macrocode}
\def\BitSet@Xor#1#2!#3#4!#5!{%
  \ifx\\#2\\%
    \ifx#1#3%
      \ifx\\#4\\%
      \else
        #50#4%
      \fi
    \else
      #51#4%
    \fi
  \else
    \ifx\\#4\\%
      #5%
      \ifx#1#30\else 1\fi
      #2%
    \else
      \ifx#1#3%
        \BitSet@AfterFiFiFi{%
          \BitSet@Xor#2!#4!#50!%
        }%
      \else
        #51%
        \BitSet@AfterFiFiFi{%
          \BitSet@Xor#2!#4!!%
        }%
      \fi
    \fi
  \BitSet@Fi
}
%    \end{macrocode}
%    \end{macro}
%
% \subsubsection{Shifting}\label{src:shift}
%
% \subsubsection{\cs{bitsetShiftLeft}}
%
%    \begin{macro}{\bitsetShiftLeft}
%    \begin{macrocode}
\def\bitsetShiftLeft#1#2{%
  \BitSet@IfUndefined{#1}{%
    \bitsetReset{#1}%
  }{%
    \bitsetIsEmpty{#1}{%
    }{%
      \expandafter\expandafter\expandafter\BitSet@ShiftLeft
      \intcalcNum{#2}!{#1}%
    }%
  }%
}
%    \end{macrocode}
%    \end{macro}
%    \begin{macro}{\BitSet@ShiftLeft}
%    \begin{macrocode}
\def\BitSet@ShiftLeft#1!#2{%
  \ifcase\intcalcSgn{#1} %
  \or
    \begingroup
      \uccode`m=`0 %
    \uppercase\expandafter{\expandafter\endgroup
      \expandafter\edef\csname BS@#2\expandafter\endcsname
      \expandafter{%
        \romannumeral#1000\expandafter\BitSet@Space
        \csname BS@#2\endcsname
      }%
    }%
  \else
    \expandafter\BitSet@ShiftRight\BitSet@Gobble#1!{#2}%
  \fi
}
%    \end{macrocode}
%    \end{macro}
%
% \subsubsection{\cs{bitsetShiftRight}}
%
%    \begin{macro}{\bitsetShiftRight}
%    \begin{macrocode}
\def\bitsetShiftRight#1#2{%
  \BitSet@IfUndefined{#1}{%
    \bitsetReset{#1}%
  }{%
    \bitsetIsEmpty{#1}{%
    }{%
      \expandafter\expandafter\expandafter\BitSet@ShiftRight
      \intcalcNum{#2}!{#1}%
    }%
  }%
}
%    \end{macrocode}
%    \end{macro}
%    \begin{macro}{\BitSet@ShiftRight}
%    \begin{macrocode}
\def\BitSet@ShiftRight#1!#2{%
  \ifcase\intcalcSgn{#1} %
  \or
    \expandafter\edef\csname BS@#2\endcsname{%
      \expandafter\expandafter\expandafter\BitSet@Kill
      \csname BS@#2\expandafter\endcsname\expandafter\BitSet@Empty
      \expandafter=%
      \expandafter{\expandafter}\expandafter{\expandafter}%
      \romannumeral#1000!%
    }%
  \else
    \expandafter\BitSet@ShiftLeft\BitSet@Gobble#1!{#2}%
  \fi
}
%    \end{macrocode}
%    \end{macro}
%    \begin{macro}{\BitSet@Kill}
%    \begin{macrocode}
\def\BitSet@Kill#1#2=#3#4#5{%
  #3#4%
  \ifx#5!%
    \ifx#1\BitSet@Empty
      0%
    \else
      #1#2%
    \fi
  \else
    \ifx#1\BitSet@Empty
      0%
      \BitSet@AfterFiFi\BitSet@Cleanup
    \else
      \BitSet@Kill#2=%
    \fi
  \BitSet@Fi
}
%    \end{macrocode}
%    \end{macro}
%
% \subsection{Bit manipulation}\label{src:bitman}
%
%    \begin{macro}{\bitsetClear}
%    \begin{macrocode}
\def\bitsetClear{%
  \BitSet@CheckIndex\BitSet@Clear
}
%    \end{macrocode}
%    \end{macro}
%    \begin{macro}{\bitsetSet}
%    \begin{macrocode}
\def\bitsetSet{%
  \BitSet@CheckIndex\BitSet@Set
}
%    \end{macrocode}
%    \end{macro}
%    \begin{macro}{\bitsetFlip}
%    \begin{macrocode}
\def\bitsetFlip{%
  \BitSet@CheckIndex\BitSet@Flip
}
%    \end{macrocode}
%    \end{macro}
%    \begin{macro}{\bitsetSetValue}
%    \begin{macrocode}
\def\bitsetSetValue#1#2#3{%
  \expandafter\expandafter\expandafter\BitSet@SetValue
  \intcalcNum{#3}!{#1}{#2}%
}
%    \end{macrocode}
%    \end{macro}
%    \begin{macro}{\BitSet@SetValue}
%    |#1|: plain value\\
%    |#2|: BitSet\\
%    |#3|: Index
%    \begin{macrocode}
\def\BitSet@SetValue#1!{%
  \BitSet@CheckIndex{%
    \ifcase#1 %
      \expandafter\BitSet@Clear
    \or
      \expandafter\BitSet@Set
    \else
      \BitSet@ErrorInvalidBitValue{#1}%
      \expandafter\expandafter\expandafter\BitSet@Gobble
      \expandafter\BitSet@Gobble
    \fi
  }%
}
%    \end{macrocode}
%    \end{macro}
%    \begin{macro}{\BitSet@ErrorInvalidBitValue}
%    |#1|: Wrong bit value
%    \begin{macrocode}
\def\BitSet@ErrorInvalidBitValue#1{%
  \@PackageError{bitset}{%
    Invalid bit value (#1) not in range 0..1%
  }\@ehc
}
%    \end{macrocode}
%    \end{macro}
%
% \subsubsection{Clear operation}
%
%    \begin{macro}{\BitSet@Clear}
%    |#1|: BitSet\\
%    |#2|: plain and checked index
%    \begin{macrocode}
\def\BitSet@Clear#1#2{%
  \edef\BitSet@Temp{%
    \expandafter\expandafter\expandafter\BitSet@@Clear
    \csname BS@#1\expandafter\endcsname
    \expandafter\BitSet@Empty\expandafter=\expandafter!%
    \romannumeral#2000!%
  }%
  \expandafter\let\csname BS@#1\expandafter\endcsname
  \ifx\BitSet@Temp\BitSet@Empty
    \BitSet@Zero
  \else
    \BitSet@Temp
  \fi
}
%    \end{macrocode}
%    \end{macro}
%    \begin{macro}{\BitSet@@Clear}
%    \begin{macrocode}
\def\BitSet@@Clear#1#2=#3!#4{%
  \ifx#4!%
    \ifx#1\BitSet@Empty
    \else
      \ifx\BitSet@Empty#2%
      \else
        #30#2%
      \fi
    \fi
  \else
    \ifx#1\BitSet@Empty
      \BitSet@AfterFiFi\BitSet@Cleanup
    \else
      \ifx#10%
        \BitSet@AfterFiFiFi{%
          \BitSet@@Clear#2=#30!%
        }%
      \else
        #31%
        \BitSet@AfterFiFiFi{%
          \BitSet@@Clear#2=!%
        }%
      \fi
    \fi
  \BitSet@Fi
}
%    \end{macrocode}
%    \end{macro}
%
% \subsubsection{Set operation}
%
%    \begin{macro}{\BitSet@Set}
%    |#1|: BitSet\\
%    |#2|: plain and checked Index
%    \begin{macrocode}
\def\BitSet@Set#1#2{%
  \expandafter\edef\csname BS@#1\endcsname{%
    \expandafter\expandafter\expandafter\BitSet@@Set
    \csname BS@#1\expandafter\endcsname
    \expandafter\BitSet@Empty\expandafter=%
    \expandafter{\expandafter}\expandafter{\expandafter}%
    \romannumeral#2000!%
  }%
}
%    \end{macrocode}
%    \end{macro}
%    \begin{macro}{\BitSet@@Set}
%    \begin{macrocode}
\def\BitSet@@Set#1#2=#3#4#5{%
  #3#4%
  \ifx#5!%
    1#2%
  \else
    \ifx#1\BitSet@Empty
      0%
      \BitSet@AfterFiFi\BitSet@@@Set
    \else
      #1%
      \BitSet@@Set#2=%
    \fi
  \BitSet@Fi
}
%    \end{macrocode}
%    \end{macro}
%    \begin{macro}{\BitSet@@@Set}
%    \begin{macrocode}
\def\BitSet@@@Set#1{%
  \ifx#1!%
    1%
  \else
    0%
    \expandafter\BitSet@@@Set
  \fi
}
%    \end{macrocode}
%    \end{macro}
%
% \subsubsection{Flip operation}
%
%    \begin{macro}{\BitSet@Flip}
%    |#1|: BitSet\\
%    |#2|: plain and checked Index
%    \begin{macrocode}
\def\BitSet@Flip#1#2{%
  \edef\BitSet@Temp{%
    \expandafter\expandafter\expandafter\BitSet@@Flip
    \csname BS@#1\expandafter\endcsname
    \expandafter\BitSet@Empty\expandafter=\expandafter!%
    \romannumeral#2000!%
  }%
  \expandafter\let\csname BS@#1\expandafter\endcsname
  \ifx\BitSet@Temp\BitSet@Empty
    \BitSet@Zero
  \else
    \BitSet@Temp
  \fi
}
%    \end{macrocode}
%    \end{macro}
%    \begin{macro}{\BitSet@@Flip}
%    \begin{macrocode}
\def\BitSet@@Flip#1#2=#3!#4{%
  \ifx#4!%
    \ifx#11%
      \ifx\BitSet@Empty#2%
      \else
        #30#2%
      \fi
    \else
      #31#2%
    \fi
  \else
    \ifx#1\BitSet@Empty
      #30%
      \BitSet@AfterFiFi\BitSet@@@Set
    \else
      \ifx#10%
        \BitSet@AfterFiFiFi{%
          \BitSet@@Flip#2=#30!%
        }%
      \else
        #31%
        \BitSet@AfterFiFiFi{%
          \BitSet@@Flip#2=!%
        }%
      \fi
    \fi
  \BitSet@Fi
}
%    \end{macrocode}
%    \end{macro}
%
% \subsubsection{Range operators}
%
%    \begin{macro}{\bitsetClearRange}
%    \begin{macrocode}
\def\bitsetClearRange{%
  \BitSet@Range\BitSet@Clear
}
%    \end{macrocode}
%    \end{macro}
%    \begin{macro}{\bitsetSetRange}
%    \begin{macrocode}
\def\bitsetSetRange{%
  \BitSet@Range\BitSet@Set
}
%    \end{macrocode}
%    \end{macro}
%    \begin{macro}{\bitsetFlipRange}
%    \begin{macrocode}
\def\bitsetFlipRange{%
  \BitSet@Range\BitSet@Flip
}
%    \end{macrocode}
%    \end{macro}
%    \begin{macro}{\bitsetSetValueRange}
%    \begin{macrocode}
\def\bitsetSetValueRange#1#2#3#4{%
  \expandafter\expandafter\expandafter\BitSet@SetValueRange
  \intcalcNum{#4}!{#1}{#2}{#3}%
}
%    \end{macrocode}
%    \end{macro}
%    \begin{macro}{\BitSet@SetValueRange}
%    \begin{macrocode}
\def\BitSet@SetValueRange#1!#2#3#4{%
  \ifcase#1 %
    \BitSet@Range\BitSet@Clear{#2}{#3}{#4}%
  \or
    \BitSet@Range\BitSet@Set{#2}{#3}{#4}%
  \else
    \BitSet@ErrorInvalidBitValue{#1}%
  \fi
}
%    \end{macrocode}
%    \end{macro}
%
%    \begin{macro}{\BitSet@Range}
%    |#1|: clear/set/flip macro\\
%    |#2|: BitSet\\
%    |#3|: Index from\\
%    |#4|: Index to
%    \begin{macrocode}
\def\BitSet@Range#1#2#3#4{%
  \edef\BitSet@Temp{%
    \noexpand\BitSet@@Range\noexpand#1{#2}%
    \intcalcNum{#3}!\intcalcNum{#4}!%
  }%
  \BitSet@Temp
}
%    \end{macrocode}
%    \end{macro}
%    \begin{macro}{\BitSet@@Range}
%    |#1|: clear/set/flip macro\\
%    |#2|: BitSet\\
%    |#3|: Index from\\
%    |#4|: Index to
%    \begin{macrocode}
\def\BitSet@@Range#1#2#3!#4!{%
  \ifnum#3<0 %
    \BitSet@NegativeIndex#1{#2}#3!#4!0!#4!%
  \else
    \ifnum#4<0 %
      \BitSet@NegativeIndex#1{#2}#3!#4!#3!0!%
    \else
      \ifcase\intcalcCmp{#3}{#4} %
      \or
        \@PackageError{bitset}{%
          Wrong index numbers in range [#3..#4]\MessageBreak% hash-ok
          for clear/set/flip on bit set `#2'.\MessageBreak
          The lower index exceeds the upper index.\MessageBreak
          Canceling the operation as error recovery%
        }\@ehc
      \else
        \BitSet@@@Range#3!#4!#1{#2}%
      \fi
    \fi
  \fi
}
%    \end{macrocode}
%    \end{macro}
%    \begin{macro}{\BitSet@NegativeIndex}
%    \begin{macrocode}
\def\BitSet@NegativeIndex#1#2#3!#4!#5!#6!{%
  \@PackageError{bitset}{%
    Negative index in range [#3..#4]\MessageBreak % hash-ok
    for \string\bitset
    \ifx#1\BitSet@Clear
      Clear%
    \else
      \ifx#1\BitSet@Set
        Set%
      \else
        Flip%
      \fi
    \fi
    Range on bit set `#2'.\MessageBreak
    Using [#5..#6] as error recovery% hash-ok
  }\@ehc
  \BitSet@@Range#1{#2}#5!#6!%
}
%    \end{macrocode}
%    \end{macro}
%    \begin{macro}{\BitSet@@Range}
%    \begin{macrocode}
\def\BitSet@@@Range#1!#2!#3#4{%
  \ifnum#1<#2 %
    #3{#4}{#1}%
    \BitSet@AfterFi{%
      \expandafter\expandafter\expandafter\BitSet@@@Range
      \IntCalcInc#1!!#2!#3{#4}%
    }%
  \BitSet@Fi
}
%    \end{macrocode}
%    \end{macro}
%
% \subsection{Bit retrieval}\label{src:bitret}
%
% \subsubsection{\cs{bitsetGet}}
%
%    \begin{macro}{\bitsetGet}
%    \begin{macrocode}
\def\bitsetGet#1#2{%
  \number
  \expandafter\expandafter\expandafter\BitSet@Get
  \intcalcNum{#2}!{#1}%
}
%    \end{macrocode}
%    \end{macro}
%    \begin{macro}{\BitSet@Get}
%    |#1|: plain index\\
%    |#2|: BitSet
%    \begin{macrocode}
\def\BitSet@Get#1!#2{%
  \ifnum#1<0 %
    \BitSet@AfterFi{%
      0 \BitSetError:NegativeIndex%
    }%
  \else
    \BitSet@IfUndefined{#2}{0}{%
      \expandafter\expandafter\expandafter\BitSet@@Get
      \csname BS@#2\expandafter\endcsname
      \expandafter!\expandafter=%
      \expandafter{\expandafter}\expandafter{\expandafter}%
      \romannumeral\intcalcNum{#1}000!%
    }%
    \expandafter\BitSet@Space
  \BitSet@Fi
}
%    \end{macrocode}
%    \end{macro}
%    \begin{macro}{\BitSet@@Get}
%    \begin{macrocode}
\def\BitSet@@Get#1#2=#3#4#5{%
  #3#4%
  \ifx#5!%
    \ifx#1!%
      0%
    \else
      #1%
    \fi
  \else
    \ifx#1!%
      0%
      \BitSet@AfterFiFi\BitSet@Cleanup
    \else
      \BitSet@@Get#2=%
    \fi
  \BitSet@Fi
}
%    \end{macrocode}
%    \end{macro}
%
% \subsubsection{\cs{bitsetNextClearBit}, \cs{bitsetNextSetBit}}
%
%    \begin{macro}{\bitsetNextClearBit}
%    \begin{macrocode}
\def\bitsetNextClearBit#1#2{%
  \number
  \expandafter\expandafter\expandafter\BitSet@NextClearBit
  \intcalcNum{#2}!{#1} %
}
%    \end{macrocode}
%    \end{macro}
%    \begin{macro}{\BitSet@NextClearBit}
%    |#1|: Index\\
%    |#2|: BitSet
%    \begin{macrocode}
\def\BitSet@NextClearBit#1!#2{%
  \ifnum#1<0 %
    \BitSet@NextClearBit0!{#2}%
    \BitSet@AfterFi{%
      \expandafter\BitSet@Space
      \expandafter\BitSetError:NegativeIndex\romannumeral0%
    }%
  \else
    \bitsetIsEmpty{#2}{#1}{%
      \expandafter\BitSet@Skip
      \number#1\expandafter\expandafter\expandafter!%
      \csname BS@#2\endcsname!!!!!!!!!=%
      {\BitSet@@NextClearBit#1!}%
    }%
  \BitSet@Fi
}
%    \end{macrocode}
%    \end{macro}
%    \begin{macro}{\BitSet@@NextClearBit}
%    |#1|: index for next bit in |#2|\\
%    |#2|: next bit
%    \begin{macrocode}
\def\BitSet@@NextClearBit#1!#2{%
  \ifx#2!%
    #1%
  \else
    \ifx#20%
      #1%
      \BitSet@AfterFiFi\BitSet@Cleanup
    \else
      \BitSet@AfterFiFi{%
        \expandafter\expandafter\expandafter\BitSet@@NextClearBit
        \IntCalcInc#1!!%
      }%
    \fi
  \BitSet@Fi
}
%    \end{macrocode}
%    \end{macro}
%
%    \begin{macro}{\bitsetNextSetBit}
%    \begin{macrocode}
\def\bitsetNextSetBit#1#2{%
  \number
  \expandafter\expandafter\expandafter\BitSet@NextSetBit
  \intcalcNum{#2}!{#1} %
}
%    \end{macrocode}
%    \end{macro}
%    \begin{macro}{\BitSet@NextSetBit}
%    |#1|: Index\\
%    |#2|: BitSet
%    \begin{macrocode}
\def\BitSet@NextSetBit#1!#2{%
  \ifnum#1<0 %
    \BitSet@NextSetBit0!{#2}%
    \BitSet@AfterFi{%
      \expandafter\BitSet@Space
      \expandafter\BitSetError:NegativeIndex\romannumeral0%
    }%
  \else
    \bitsetIsEmpty{#2}{-1}{%
      \expandafter\BitSet@Skip
      \number#1\expandafter\expandafter\expandafter!%
      \csname BS@#2\endcsname!!!!!!!!!=%
      {\BitSet@@NextSetBit#1!}%
    }%
  \BitSet@Fi
}
%    \end{macrocode}
%    \end{macro}
%    \begin{macro}{\BitSet@@NextSetBit}
%    |#1|: index for next bit in |#2|\\
%    |#2|: next bit
%    \begin{macrocode}
\def\BitSet@@NextSetBit#1!#2{%
  \ifx#2!%
    -1%
  \else
    \ifx#21%
      #1%
      \BitSet@AfterFiFi\BitSet@Cleanup
    \else
      \BitSet@AfterFiFi{%
        \expandafter\expandafter\expandafter\BitSet@@NextSetBit
        \IntCalcInc#1!!%
      }%
    \fi
  \BitSet@Fi
}
%    \end{macrocode}
%    \end{macro}
%    \begin{macro}{\BitSet@Cleanup}
%    \begin{macrocode}
\def\BitSet@Cleanup#1!{}
%    \end{macrocode}
%    \end{macro}
%
%    \begin{macro}{\BitSet@Skip}
%    |#1|: number of bits to skip\\
%    |#2|: bits\\
%    |#3|: continuation code
%    \begin{macrocode}
\def\BitSet@Skip#1!#2{%
  \ifx#2!%
    \BitSet@AfterFi{%
      \BitSet@SkipContinue%
    }%
  \else
    \ifcase#1 %
      \BitSet@AfterFiFi{%
        \BitSet@SkipContinue#2%
      }%
    \or
      \BitSet@AfterFiFi\BitSet@SkipContinue
    \or
      \BitSet@AfterFiFi{%
        \expandafter\BitSet@SkipContinue\BitSet@Gobble
      }%
    \else
      \ifnum#1>8 %
        \BitSet@AfterFiFiFi{%
          \expandafter\BitSet@Skip
          \number\IntCalcSub#1!8!\expandafter!%
          \BitSet@GobbleSeven
        }%
      \else
        \BitSet@AfterFiFiFi{%
          \expandafter\expandafter\expandafter\BitSet@Skip
          \IntCalcDec#1!!%
        }%
      \fi
    \fi
  \BitSet@Fi
}
%    \end{macrocode}
%    \end{macro}
%    \begin{macro}{\BitSet@SkipContinue}
%    |#1|: remaining bits\\
%    |#2|: continuation code
%    \begin{macrocode}
\def\BitSet@SkipContinue#1!#2=#3{%
  #3#1!%
}
%    \end{macrocode}
%    \end{macro}
%    \begin{macro}{\BitSet@GobbleSeven}
%    \begin{macrocode}
\def\BitSet@GobbleSeven#1#2#3#4#5#6#7{}
%    \end{macrocode}
%    \end{macro}
%
% \subsubsection{\cs{bitsetGetSetBitList}}
%
%    \begin{macro}{\bitsetGetSetBitList}
%    It's just a wrapper for \cs{bitsetNextSetBit}.
%    \begin{macrocode}
\def\bitsetGetSetBitList#1{%
  \romannumeral0%
  \bitsetIsEmpty{#1}{ }{%
    \expandafter\BitSet@GetSetBitList
    \number\BitSet@NextSetBit0!{#1}!{#1}{}!%
  }%
}
%    \end{macrocode}
%    \end{macro}
%    \begin{macro}{\BitSet@GetSetBitList}
%    |#1|: found index\\
%    |#2|: BitSet\\
%    |#3|: comma
%    |#4|: result
%    \begin{macrocode}
\def\BitSet@GetSetBitList#1!#2#3#4!{%
  \ifnum#1<0 %
    \BitSet@AfterFi{ #4}%
  \else
    \BitSet@AfterFi{%
      \expandafter\BitSet@GetSetBitList\number
      \expandafter\expandafter\expandafter\BitSet@NextSetBit
      \IntCalcInc#1!!{#2}!{#2},#4#3#1!%
    }%
  \BitSet@Fi
}
%    \end{macrocode}
%    \end{macro}
%
% \subsection{Bit set properties}\label{src:prop}
%
%    \begin{macro}{\bitsetSize}
%    \begin{macrocode}
\def\bitsetSize#1{%
  \number
  \BitSet@IfUndefined{#1}{0 }{%
    \expandafter\expandafter\expandafter\BitSet@Size
    \expandafter\expandafter\expandafter1%
    \expandafter\expandafter\expandafter!%
    \csname BS@#1\endcsname!0!%
  }%
}
%    \end{macrocode}
%    \end{macro}
%    \begin{macro}{\BitSet@Size}
%    |#1|: counter\\
%    |#2#3|: bits\\
%    |#4|: result
%    \begin{macrocode}
\def\BitSet@Size#1!#2#3!#4!{%
  \ifx#21%
    \ifx\\#3\\%
      \BitSet@AfterFiFi{#1 }%
    \else
      \BitSet@AfterFiFi{%
        \expandafter\expandafter\expandafter\BitSet@Size
        \IntCalcInc#1!!#3!#1!%
      }%
    \fi
  \else
    \ifx\\#3\\%
      \BitSet@AfterFiFi{#4 }%
    \else
      \BitSet@AfterFiFi{%
        \expandafter\expandafter\expandafter\BitSet@Size
        \IntCalcInc#1!!#3!#4!%
      }%
    \fi
  \fi
  \BitSet@Fi
}
%    \end{macrocode}
%    \end{macro}
%
%    \begin{macro}{\bitsetCardinality}
%    \begin{macrocode}
\def\bitsetCardinality#1{%
  \number
  \BitSet@IfUndefined{#1}{0 }{%
    \expandafter\expandafter\expandafter\BitSet@Cardinality
    \expandafter\expandafter\expandafter0%
    \expandafter\expandafter\expandafter!%
    \csname BS@#1\endcsname!%
  }%
}
%    \end{macrocode}
%    \end{macro}
%    \begin{macro}{\BitSet@Cardinality}
%    |#1|: result\\
%    |#2#3|: bits
%    \begin{macrocode}
\def\BitSet@Cardinality#1!#2#3!{%
  \ifx#21%
    \ifx\\#3\\%
      \BitSet@AfterFiFi{\IntCalcInc#1! }%
    \else
      \BitSet@AfterFiFi{%
        \expandafter\expandafter\expandafter\BitSet@Cardinality
        \IntCalcInc#1!!#3!%
      }%
    \fi
  \else
    \ifx\\#3\\%
      \BitSet@AfterFiFi{#1 }%
    \else
      \BitSet@AfterFiFi{%
        \BitSet@Cardinality#1!#3!%
      }%
    \fi
  \fi
  \BitSet@Fi
}
%    \end{macrocode}
%    \end{macro}
%
% \subsection{Queries}\label{src:query}
%
%    \begin{macro}{\bitsetIsDefined}
%    \begin{macrocode}
\def\bitsetIsDefined#1{%
  \BitSet@IfUndefined{#1}%
  \BitSet@SecondOfTwo
  \BitSet@FirstOfTwo
}
%    \end{macrocode}
%    \end{macro}
%    \begin{macro}{\bitsetIsEmpty}
%    \begin{macrocode}
\def\bitsetIsEmpty#1{%
  \BitSet@IfUndefined{#1}\BitSet@FirstOfTwo{%
    \expandafter\ifx\csname BS@#1\endcsname\BitSet@Zero
      \expandafter\BitSet@FirstOfTwo
    \else
      \expandafter\BitSet@SecondOfTwo
    \fi
  }%
}
%    \end{macrocode}
%    \end{macro}
%    \begin{macro}{\BitSet@Zero}
%    \begin{macrocode}
\def\BitSet@Zero{0}
%    \end{macrocode}
%    \end{macro}
%
%    \begin{macro}{\bitsetQuery}
%    \begin{macrocode}
\def\bitsetQuery#1#2{%
  \ifnum\bitsetGet{#1}{#2}=1 %
    \expandafter\BitSet@FirstOfTwo
  \else
    \expandafter\BitSet@SecondOfTwo
  \fi
}
%    \end{macrocode}
%    \end{macro}
%
%    \begin{macro}{\bitsetEquals}
%    \begin{macrocode}
\def\bitsetEquals#1#2{%
  \BitSet@IfUndefined{#1}{%
    \BitSet@IfUndefined{#2}\BitSet@FirstOfTwo\BitSet@SecondOfTwo
  }{%
    \BitSet@IfUndefined{#2}\BitSet@SecondOfTwo{%
      \expandafter\ifx\csname BS@#1\expandafter\endcsname
                      \csname BS@#2\endcsname
        \expandafter\BitSet@FirstOfTwo
      \else
        \expandafter\BitSet@SecondOfTwo
      \fi
    }%
  }%
}
%    \end{macrocode}
%    \end{macro}
%
%    \begin{macro}{\bitsetIntersects}
%    \begin{macrocode}
\def\bitsetIntersects#1#2{%
  \bitsetIsEmpty{#1}\BitSet@SecondOfTwo{%
    \bitsetIsEmpty{#2}\BitSet@SecondOfTwo{%
      \expandafter\expandafter\expandafter\BitSet@Intersects
      \csname BS@#1\expandafter\expandafter\expandafter\endcsname
      \expandafter\expandafter\expandafter!%
      \csname BS@#2\endcsname!%
    }%
  }%
}
%    \end{macrocode}
%    \end{macro}
%    \begin{macro}{\BitSet@Intersects}
%    \begin{macrocode}
\def\BitSet@Intersects#1#2!#3#4!{%
  \ifnum#1#3=11 %
    \BitSet@AfterFi\BitSet@FirstOfTwo
  \else
    \ifx\\#2\\%
      \BitSet@AfterFiFi\BitSet@SecondOfTwo
    \else
      \ifx\\#4\\%
        \BitSet@AfterFiFiFi\BitSet@SecondOfTwo
      \else
        \BitSet@AfterFiFiFi{%
          \BitSet@Intersects#2!#4!%
        }%
      \fi
    \fi
  \BitSet@Fi
}
%    \end{macrocode}
%    \end{macro}
%
%    \begin{macrocode}
\BitSet@AtEnd%
%</package>
%    \end{macrocode}
%
% \section{Test}
%
% \subsection{Catcode checks for loading}
%
%    \begin{macrocode}
%<*test1>
%    \end{macrocode}
%    \begin{macrocode}
\catcode`\{=1 %
\catcode`\}=2 %
\catcode`\#=6 %
\catcode`\@=11 %
\expandafter\ifx\csname count@\endcsname\relax
  \countdef\count@=255 %
\fi
\expandafter\ifx\csname @gobble\endcsname\relax
  \long\def\@gobble#1{}%
\fi
\expandafter\ifx\csname @firstofone\endcsname\relax
  \long\def\@firstofone#1{#1}%
\fi
\expandafter\ifx\csname loop\endcsname\relax
  \expandafter\@firstofone
\else
  \expandafter\@gobble
\fi
{%
  \def\loop#1\repeat{%
    \def\body{#1}%
    \iterate
  }%
  \def\iterate{%
    \body
      \let\next\iterate
    \else
      \let\next\relax
    \fi
    \next
  }%
  \let\repeat=\fi
}%
\def\RestoreCatcodes{}
\count@=0 %
\loop
  \edef\RestoreCatcodes{%
    \RestoreCatcodes
    \catcode\the\count@=\the\catcode\count@\relax
  }%
\ifnum\count@<255 %
  \advance\count@ 1 %
\repeat

\def\RangeCatcodeInvalid#1#2{%
  \count@=#1\relax
  \loop
    \catcode\count@=15 %
  \ifnum\count@<#2\relax
    \advance\count@ 1 %
  \repeat
}
\def\RangeCatcodeCheck#1#2#3{%
  \count@=#1\relax
  \loop
    \ifnum#3=\catcode\count@
    \else
      \errmessage{%
        Character \the\count@\space
        with wrong catcode \the\catcode\count@\space
        instead of \number#3%
      }%
    \fi
  \ifnum\count@<#2\relax
    \advance\count@ 1 %
  \repeat
}
\def\space{ }
\expandafter\ifx\csname LoadCommand\endcsname\relax
  \def\LoadCommand{\input bitset.sty\relax}%
\fi
\def\Test{%
  \RangeCatcodeInvalid{0}{47}%
  \RangeCatcodeInvalid{58}{64}%
  \RangeCatcodeInvalid{91}{96}%
  \RangeCatcodeInvalid{123}{255}%
  \catcode`\@=12 %
  \catcode`\\=0 %
  \catcode`\%=14 %
  \LoadCommand
  \RangeCatcodeCheck{0}{36}{15}%
  \RangeCatcodeCheck{37}{37}{14}%
  \RangeCatcodeCheck{38}{47}{15}%
  \RangeCatcodeCheck{48}{57}{12}%
  \RangeCatcodeCheck{58}{63}{15}%
  \RangeCatcodeCheck{64}{64}{12}%
  \RangeCatcodeCheck{65}{90}{11}%
  \RangeCatcodeCheck{91}{91}{15}%
  \RangeCatcodeCheck{92}{92}{0}%
  \RangeCatcodeCheck{93}{96}{15}%
  \RangeCatcodeCheck{97}{122}{11}%
  \RangeCatcodeCheck{123}{255}{15}%
  \RestoreCatcodes
}
\Test
\csname @@end\endcsname
\end
%    \end{macrocode}
%    \begin{macrocode}
%</test1>
%    \end{macrocode}
%
% \subsection{Macro tests}
%
% \subsubsection{Preamble}
%
%    \begin{macrocode}
%<*test2>
\NeedsTeXFormat{LaTeX2e}
\nofiles
\documentclass{article}
\makeatletter
%<*noetex>
\let\SavedNumexpr\numexpr
\let\SavedIfcsname\ifcsname
\let\SavedCurrentgrouplevel\currentgrouplevel
\def\ETeXDisable{%
  \let\ifcsname\@undefined
  \let\numexpr\@undefined
  \let\currentgrouplevel\@undefined
}
\ETeXDisable
%</noetex>
\makeatletter
\chardef\BitSet@TestMode=1 %
\makeatother
\usepackage{bitset}[2011/01/30]
%<*noetex>
\def\ETeXEnable{%
  \let\numexpr\SavedNumexpr
  \let\ifcsname\SavedIfcsname
  \let\currentgrouplevel\SavedCurrentgrouplevel
}
\ETeXEnable
%</noetex>
\usepackage{qstest}
\IncludeTests{*}
\LogTests{log}{*}{*}
\makeatletter
%    \end{macrocode}
%
% \subsubsection{Time}
%
%    \begin{macrocode}
\begingroup\expandafter\expandafter\expandafter\endgroup
\expandafter\ifx\csname pdfresettimer\endcsname\relax
\else
  \newcount\SummaryTime
  \newcount\TestTime
  \SummaryTime=\z@
  \newcommand*{\PrintTime}[2]{%
    \typeout{%
      [Time #1: \strip@pt\dimexpr\number#2sp\relax\space s]%
    }%
  }%
  \newcommand*{\StartTime}[1]{%
    \renewcommand*{\TimeDescription}{#1}%
    \pdfresettimer
  }%
  \newcommand*{\TimeDescription}{}%
  \newcommand*{\StopTime}{%
    \TestTime=\pdfelapsedtime
    \global\advance\SummaryTime\TestTime
    \PrintTime\TimeDescription\TestTime
  }%
  \let\saved@qstest\qstest
  \let\saved@endqstest\endqstest
  \def\qstest#1#2{%
    \saved@qstest{#1}{#2}%
    \StartTime{#1}%
  }%
  \def\endqstest{%
    \StopTime
    \saved@endqstest
  }%
  \AtEndDocument{%
    \PrintTime{summary}\SummaryTime
  }%
\fi
%    \end{macrocode}
%
% \subsubsection{Detection of unwanted space}
%
%    \begin{macrocode}
\let\orig@qstest\qstest
\let\orig@endqstest\endqstest
\def\qstest#1#2{%
  \orig@qstest{#1}{#2}%
  \setbox0\hbox\bgroup\begingroup\ignorespaces
}
\def\endqstest{%
  \endgroup\egroup
  \Expect*{\the\wd0}{0.0pt}%
  \orig@endqstest
}
%    \end{macrocode}
%
% \subsubsection{Test macros}
%
%    \begin{macrocode}
\newcounter{Test}

\def\TestError#1#2{%
  \begingroup
    \setcounter{Test}{0}%
    \sbox0{%
      \def\@PackageError##1##2##3{%
        \stepcounter{Test}%
        \begingroup
          \let\MessageBreak\relax
%<*noetex>
          \ETeXEnable
%</noetex>
          \Expect{##1}{bitset}%
          \Expect*{##2}*{#1}%
        \endgroup
      }%
%<*noetex>
      \ETeXDisable
%</noetex>
      #2%
    }%
    \Expect*{\theTest}{1}%
    \Expect*{\the\wd0}{0.0pt}%
  \endgroup
}

\def\TestErrorNegativeIndex#1#2{%
  \TestError{Invalid negative index (#1)}{#2}%
}

\def\TestGetterUndefined#1{%
  \CheckUndef{dummy}%
  \expandafter\expandafter\expandafter\Expect
  \expandafter\expandafter\expandafter{#1{dummy}}{0}%
}

\def\ExpectBitSet#1#2{%
  \expandafter\expandafter\expandafter\Expect
  \expandafter\expandafter\expandafter
  {\csname BS@#1\endcsname}*{#2}%
}
\def\Check#1#2{%
  \ExpectBitSet{#1}{#2}%
}
\def\CheckUndef#1{%
  \begingroup
    \Expect*{%
      \expandafter
      \ifx\csname BS@#1\endcsname\relax true\else false\fi
    }{true}%
  \endgroup
}
\def\RevCheck#1#2{%
  \ExpectBitSet{#1}{\Reverse#2!!}%
}
\def\Set#1#2{%
  \expandafter\def\csname BS@#1\endcsname{#2}%
}
\def\RevSet#1#2{%
  \expandafter\edef\csname BS@#1\endcsname{%
    \Reverse#2!!%
  }%
}
\def\Reverse#1#2!#3!{%
  \ifx\\#2\\%
    #1#3%
    \expandafter\@gobble
  \else
    \expandafter\@firstofone
  \fi
  {\Reverse#2!#1#3!}%
}
%    \end{macrocode}
%
% \subsubsection{Test sets}
%
%    \begin{macrocode}
\begin{qstest}{Let}{Let}
  \CheckUndef{abc}%
  \CheckUndef{xyz}%
  \bitsetLet{xyz}{abc}%
  \CheckUndef{abc}%
  \Check{xyz}{0}%
  \Set{abc}{1}%
  \Check{abc}{1}%
  \Check{xyz}{0}%
  \bitsetLet{xyz}{abc}%
  \Check{abc}{1}%
  \Check{xyz}{1}%
  \Set{xyz}{11}%
  \Check{abc}{1}%
  \Check{xyz}{11}%
\end{qstest}

\begin{qstest}{Reset}{Reset}
  \bitsetReset{xyz}%
  \Check{xyz}{0}%
  \bitsetReset{abc}%
  \Check{abc}{0}%
  \Set{abc}{10101}%
  \bitsetReset{abc}%
  \Check{abc}{0}%
\end{qstest}

\begin{qstest}{Get/Query}{Get/Query}
  \expandafter\expandafter\expandafter\Expect
  \expandafter\expandafter\expandafter{%
    \bitsetGet{dummy}{0}%
  }{0}%
  \begingroup
    \expandafter\def\csname BitSetError:NegativeIndex\endcsname{}%
    \Set{abc}{1}%
    \Expect*{\bitsetQuery{abc}{-1}{true}{false}}{false}%
  \endgroup
  \def\Test#1#2#3{%
    \Set{abc}{#1}%
    \expandafter\expandafter\expandafter\Expect
    \expandafter\expandafter\expandafter{\bitsetGet{abc}{#2}}{#3}%
    \Expect*{\bitsetQuery{abc}{#2}{true}{false}}%
           *{\ifcase#3 false\or true\else error\fi}%
  }%
  \Test{1}{100}{0}%
  \Test{0}{0}{0}%
  \Test{1}{0}{1}%
  \Test{11}{1}{1}%
  \Test{111}{1}{1}%
  \Test{101}{1}{0}%
  \Test{101}{2}{1}%
  \Test{10100110011}{10}{1}%
\end{qstest}

\begin{qstest}{Size}{Size}
  \TestGetterUndefined\bitsetSize
  \def\Test#1#2{%
    \Set{abc}{#1}%
    \expandafter\expandafter\expandafter\Expect
    \expandafter\expandafter\expandafter{\bitsetSize{abc}}{#2}%
  }%
  \Test{0}{0}%
  \Test{1}{1}%
  \Test{00}{0}%
  \Test{0000000}{0}%
  \Test{10}{1}%
  \Test{01}{2}%
  \Test{11}{2}%
  \Test{010}{2}%
  \Test{011}{3}%
  \Test{100110011}{9}%
  \Test{0000011111000001111100000}{20}%
  \Test{000000000000000000000000011111111111111111111}{45}%
\end{qstest}

\begin{qstest}{Cardinality}{Cardinality}
  \TestGetterUndefined\bitsetCardinality
  \def\Test#1#2{%
    \Set{abc}{#1}%
    \expandafter\expandafter\expandafter\Expect
    \expandafter\expandafter\expandafter{%
      \bitsetCardinality{abc}%
    }{#2}%
  }%
  \Test{0}{0}%
  \Test{1}{1}%
  \Test{00}{0}%
  \Test{0000000}{0}%
  \Test{10}{1}%
  \Test{01}{1}%
  \Test{11}{2}%
  \Test{010}{1}%
  \Test{011}{2}%
  \Test{100110011}{5}%
  \Test{0000011111000001111100000}{10}%
  \Test{000000000000000000000000011111111111111111111}{20}%
\end{qstest}

\begin{qstest}{NextClearBit/NextSetBit}{NextClearBit/NextSetBit}
  \def\Test#1#2{%
    \expandafter\expandafter\expandafter\Expect
    \expandafter\expandafter\expandafter{%
      \TestOp{abc}{#1}%
    }{#2}%
  }%
  \def\Clear{\let\TestOp\bitsetNextClearBit}%
  \def\Set{\let\TestOp\bitsetNextSetBit}%
  \begingroup
    \catcode`\:=11 %
    \bitsetSetBin{abc}{1}%
    \Clear
    \Test{-1}{1\BitSetError:NegativeIndex}%
    \Set
    \Test{-1}{0\BitSetError:NegativeIndex}%
  \endgroup
  \let\BS@abc\@undefined
  \Clear
  \Test{0}{0}%
  \Test{1}{1}%
  \Test{2}{2}%
  \Test{100}{100}%
  \Set
  \Test{0}{-1}%
  \Test{1}{-1}%
  \Test{100}{-1}%
  \bitsetReset{abc}%
  \Clear
  \Test{0}{0}%
  \Test{1}{1}%
  \Test{2}{2}%
  \Test{100}{100}%
  \Set
  \Test{0}{-1}%
  \Test{1}{-1}%
  \Test{100}{-1}%
  \bitsetSetBin{abc}{1}%
  \Clear
  \Test{0}{1}%
  \Test{1}{1}%
  \Test{2}{2}%
  \Test{100}{100}%
  \Set
  \Test{0}{0}%
  \Test{1}{-1}%
  \Test{100}{-1}%
  \bitsetSetBin{abc}{111000111000111000111}%
  \Clear
  \Test{0}{3}%
  \Test{1}{3}%
  \Test{2}{3}%
  \Test{3}{3}%
  \Test{4}{4}%
  \Test{5}{5}%
  \Test{6}{9}%
  \Test{7}{9}%
  \Test{8}{9}%
  \Test{9}{9}%
  \Test{10}{10}%
  \Test{11}{11}%
  \Test{12}{15}%
  \Test{13}{15}%
  \Test{14}{15}%
  \Test{15}{15}%
  \Test{16}{16}%
  \Test{17}{17}%
  \Test{18}{21}%
  \Test{19}{21}%
  \Test{20}{21}%
  \Test{21}{21}%
  \Test{22}{22}%
  \Test{100}{100}%
  \Set
  \Test{0}{0}%
  \Test{1}{1}%
  \Test{2}{2}%
  \Test{3}{6}%
  \Test{4}{6}%
  \Test{5}{6}%
  \Test{6}{6}%
  \Test{7}{7}%
  \Test{8}{8}%
  \Test{9}{12}%
  \Test{10}{12}%
  \Test{11}{12}%
  \Test{12}{12}%
  \Test{13}{13}%
  \Test{14}{14}%
  \Test{15}{18}%
  \Test{16}{18}%
  \Test{17}{18}%
  \Test{18}{18}%
  \Test{19}{19}%
  \Test{20}{20}%
  \Test{21}{-1}%
  \Test{22}{-1}%
  \Test{100}{-1}%
  \bitsetSetBin{abc}{1111111}%
  \Clear
  \Test{6}{7}%
  \Test{7}{7}%
  \Test{8}{8}%
  \Test{100}{100}%
  \Set
  \Test{6}{6}%
  \Test{7}{-1}%
  \Test{8}{-1}%
  \Test{100}{-1}%
  \bitsetSetBin{abc}{11111111}%
  \Clear
  \Test{7}{8}%
  \Test{8}{8}%
  \Test{9}{9}%
  \Test{100}{100}%
  \Set
  \Test{7}{7}%
  \Test{8}{-1}%
  \Test{9}{-1}%
  \Test{100}{-1}%
  \bitsetSetBin{abc}{111111111}%
  \Clear
  \Test{8}{9}%
  \Test{9}{9}%
  \Test{10}{10}%
  \Test{100}{100}%
  \Set
  \Test{8}{8}%
  \Test{9}{-1}%
  \Test{10}{-1}%
  \Test{100}{-1}%
  \bitsetSetBin{abc}{1111111111}%
  \Clear
  \Test{9}{10}%
  \Test{10}{10}%
  \Test{11}{11}%
  \Test{100}{100}%
  \Set
  \Test{9}{9}%
  \Test{10}{-1}%
  \Test{11}{-1}%
  \Test{100}{-1}%
\end{qstest}

\begin{qstest}{GetSetBitList}{GetSetBitList}
  \let\BS@abc\@undefined
  \expandafter\expandafter\expandafter\Expect
  \expandafter\expandafter\expandafter{%
    \bitsetGetSetBitList{abc}%
  }{}%
  \def\Test#1#2{%
    \bitsetSetBin{abc}{#1}%
    \expandafter\expandafter\expandafter\Expect
    \expandafter\expandafter\expandafter{%
      \bitsetGetSetBitList{abc}%
    }{#2}%
  }%
  \Test{0}{}%
  \Test{1}{0}%
  \Test{10}{1}%
  \Test{11}{0,1}%
  \Test{10110100}{2,4,5,7}%
  \Test{101101001010011}{0,1,4,6,9,11,12,14}%
\end{qstest}

\begin{qstest}{GetDec}{GetDec}
  \TestGetterUndefined\bitsetGetDec
  \def\Test#1#2{%
    \RevSet{abc}{#1}%
%<*noetex>
    \begingroup\expandafter\expandafter\expandafter\endgroup
%</noetex>
    \expandafter\expandafter\expandafter\Expect
    \expandafter\expandafter\expandafter{%
      \bitsetGetDec{abc}%
    }{#2}%
  }%
  \Test{0}{0}%
  \Test{1}{1}%
  \Test{10}{2}%
  \Test{11}{3}%
  \Test{100}{4}%
  \Test{101}{5}%
  \Test{110}{6}%
  \Test{111}{7}%
  \Test{1000}{8}%
  \Test{000111}{7}%
  \Test{1111111111111111%
        111111111111111}{2147483647}%
  \Test{0001111111111111111%
        111111111111111}{2147483647}%
  \Test{1000000000000000%
        0000000000000000}{2147483648}%
  \Test{1000000000000000%
        00000000000000000}{4294967296}%
  \Test{0001000000000000000%
        00000000000000000}{4294967296}%
  \Test{1100000000000000%
        00000000000000011}{6442450947}%
\end{qstest}

\begin{qstest}{Clear}{Clear}
  \def\Test#1#2#3{%
    \RevSet{abc}{#1}%
    \bitsetClear{abc}{#2}%
    \Expect*{\BS@abc}*{\Reverse#3!!}%
  }%
  \bitsetClear{abc}{2}%
  \RevCheck{abc}{0}%
  \TestErrorNegativeIndex{-1}{\bitsetClear{abc}{-1}}%
  \RevCheck{abc}{0}%
  \Test{0}{0}{0}%
  \Test{1}{0}{0}%
  \Test{111}{1}{101}%
  \Test{111}{30}{111}%
  \Test{0000111}{5}{0000111}% 111 would also be ok
  \Test{10000111}{5}{10000111}%
  \Test{1001001}{3}{1000001}%
\end{qstest}

\begin{qstest}{Set}{Set}
  \def\Test#1#2#3{%
    \RevSet{abc}{#1}%
    \bitsetSet{abc}{#2}%
    \Expect*{\BS@abc}*{\Reverse#3!!}%
  }%
  \bitsetSet{abc}{2}%
  \RevCheck{abc}{100}%
  \TestErrorNegativeIndex{-1}{\bitsetSet{abc}{-1}}%
  \RevCheck{abc}{100}%
  \Test{0}{0}{1}%
  \Test{1}{0}{1}%
  \Test{100}{1}{110}%
  \Test{111}{1}{111}%
  \Test{11}{1}{11}%
  \Test{11}{2}{111}%
  \Test{11}{3}{1011}%
  \Test{111}{10}{10000000111}%
  \Test{0000111}{5}{0100111}% 100111 would also be ok
  \Test{10000111}{5}{10100111}%
  \Test{1000001}{3}{1001001}%
  \Test{1001001}{3}{1001001}%
\end{qstest}

\begin{qstest}{Flip}{Flip}
  \def\Test#1#2#3{%
    \RevSet{abc}{#1}%
    \bitsetFlip{abc}{#2}%
    \Expect*{\BS@abc}*{\Reverse#3!!}%
  }%
  \bitsetFlip{abc}{2}%
  \RevCheck{abc}{100}%
  \TestErrorNegativeIndex{-1}{\bitsetFlip{abc}{-1}}%
  \RevCheck{abc}{100}%
  \Test{0}{0}{1}%
  \Test{1}{0}{0}%
  \Test{0}{2}{100}%
  \Test{100}{1}{110}%
  \Test{111}{1}{101}%
  \Test{11}{1}{1}%
  \Test{11}{2}{111}%
  \Test{11}{3}{1011}%
  \Test{111}{10}{10000000111}%
  \Test{0000111}{5}{0100111}% 100111 would also be ok
  \Test{10000111}{5}{10100111}%
  \Test{1000001}{3}{1001001}%
  \Test{1001001}{3}{1000001}%
  \Test{11111}{2}{11011}%
\end{qstest}

\begin{qstest}{SetValue}{SetValue}
  \def\Test#1#2{%
    \TestError{Invalid bit value (#2) not in range 0..1}{%
      \bitsetSetValue{abc}{#1}{#2}%
    }%
  }%
  \Test{0}{-1}%
  \Test{0}{2}%
  \Test{0}{10}%
  \def\Test#1#2#3{%
    \let\BS@abc\@undefined
    \bitsetSetValue{abc}{#1}{#2}%
    \bitsetSetBin{result}{#3}%
    \Expect*{\BS@abc}*{\BS@result}%
  }%
  \Test{0}{0}{0}%
  \Test{0}{1}{1}%
  \Test{1}{0}{0}%
  \Test{1}{1}{10}%
  \def\Test#1#2#3#4{%
    \bitsetSetBin{abc}{#1}%
    \bitsetSetBin{result}{#4}%
    \bitsetSetValue{abc}{#2}{#3}%
    \Expect*{\BS@abc}*{\BS@result}%
  }%
  \Test{0}{0}{0}{0}%
  \Test{0}{0}{0}{0}%
  \Test{0}{0}{1}{1}%
  \Test{0}{1}{0}{0}%
  \Test{0}{1}{1}{10}%
  \Test{1010}{2}{1}{1110}%
  \Test{1010}{4}{1}{11010}%
  \Test{1010}{6}{1}{1001010}%
  \Test{1010}{1}{0}{1000}%
  \Test{1010}{2}{0}{1010}%
  \Test{1010}{3}{0}{10}%
  \Test{1010}{4}{0}{1010}%
  \Test{1010}{6}{0}{1010}%
  \Test{1010}{2}{\csname iffalse\endcsname 0\else 1\fi}{1110}%
  \Test{1010}{1}{\csname iffalse\endcsname 1\else 0\fi}{1000}%
\end{qstest}

\begin{qstest}{IsDefined}{IsDefined}
  \let\BS@abc\@undefined
  \Expect*{\bitsetIsDefined{abc}{true}{false}}{false}%
  \bitsetReset{abc}%
  \Expect*{\bitsetIsDefined{abc}{true}{false}}{true}%
\end{qstest}

\begin{qstest}{IsEmpty}{IsEmpty}
  \let\BS@abc\@undefined
  \Expect*{\bitsetIsEmpty{abc}{true}{false}}{true}%
  \bitsetReset{abc}%
  \Expect*{\bitsetIsEmpty{abc}{true}{false}}{true}%
  \bitsetSet{abc}{1}%
  \Expect*{\bitsetIsEmpty{abc}{true}{false}}{false}%
\end{qstest}

\begin{qstest}{Equals}{Equals}
  \def\Test#1#2#3{%
    \Expect*{\bitsetEquals{#1}{#2}{true}{false}}{#3}%
  }%
  \let\BS@abc\@undefined
  \Test{abc}{abc}{true}%
  \Test{abc}{foo}{true}%
  \Test{foo}{abc}{true}%
  \bitsetReset{abc}%
  \Test{abc}{abc}{true}%
  \Test{abc}{foo}{false}%
  \Test{foo}{abc}{false}%
  \bitsetReset{foo}%
  \Test{abc}{foo}{true}%
  \Test{foo}{abc}{true}%
  \bitsetSet{abc}{4}%
  \Test{abc}{foo}{false}%
  \Test{foo}{abc}{false}%
  \bitsetFlip{foo}{4}%
  \Test{abc}{foo}{true}%
  \Test{foo}{abc}{true}%
\end{qstest}

\begin{qstest}{Intersects}{Intersects}
  \def\Test#1{%
    \Expect*{\bitsetIntersects{abc}{foo}{true}{false}}{#1}%
  }%
  \let\BS@abc\@undefined
  \let\BS@foo\@undefined
  \Test{false}%
  \Set{abc}{0}%
  \Test{false}%
  \Set{foo}{0}%
  \Test{false}%
  \let\BS@abc\@undefined
  \Test{false}%
  \Set{foo}{1}%
  \Test{false}%
  \Set{abc}{0}%
  \Test{false}%
  \Set{abc}{1}%
  \Test{true}%
  \let\BS@foo\@undefined
  \Test{false}%
  \Set{foo}{0}%
  \Test{false}%
  \def\Test#1#2#3{%
    \bitsetSetBin{abc}{#1}%
    \bitsetSetBin{foo}{#2}%
    \Expect*{\bitsetIntersects{abc}{foo}{true}{false}}{#3}%
  }%
  \Test{1010}{0101}{false}%
  \Test{0}{10}{false}%
  \Test{1}{11}{true}%
  \Test{11}{1}{true}%
  \Test{10}{1}{false}%
\end{qstest}

\begin{qstest}{And/AndNot/Or/Xor}{And/AndNot/Or/Xor}
  \def\@Test#1#2#3#4#5{%
    \begingroup
      #5%
      \begingroup
        \let\BS@foo\@undefined
        \csname bitset#1\endcsname{abc}{foo}%
        \CheckUndef{foo}%
        \Check{abc}{#2}%
      \endgroup
      \begingroup
        \bitsetReset{foo}%
        \csname bitset#1\endcsname{abc}{foo}%
        \Check{foo}{0}%
        \Check{abc}{#3}%
      \endgroup
      \begingroup
        \def\BS@foo{0101}%
        \csname bitset#1\endcsname{abc}{foo}%
        \Check{foo}{0101}%
        \Check{abc}{#4}%
      \endgroup
    \endgroup
  }%
  \def\Test#1{%
    \def\Op{#1}%
    \Test@
  }%
  \def\Test@#1#2#3#4#5#6#7#8#9{%
    \@Test\Op{#1}{#2}{#3}{%
      \let\BS@abc\@undefined
    }%
    \@Test\Op{#4}{#5}{#6}{%
      \bitsetReset{abc}%
    }%
    \@Test\Op{#7}{#8}{#9}{%
      \def\BS@abc{1001}%
    }%
  }%
  \Test{And}%
       {0}{0}{0}%
       {0}{0}{0}%
       {0}{0}{0001}%
  \Test{AndNot}%
       {0}{0}{0}%
       {0}{0}{0}%
       {1001}{1001}{1}%
  \Test{Or}%
       {0}{0}{0101}%
       {0}{0}{0101}%
       {1001}{1001}{1101}%
  \Test{Xor}%
       {0}{0}{0101}%
       {0}{0}{0101}%
       {1001}{1001}{11}%
  \def\Test#1#2#3{%
    \bitsetSetBin{abc}{#1}%
    \bitsetSetBin{foo}{#2}%
    \csname bitset\Op\endcsname{abc}{foo}%
    \RevCheck{foo}{#2}%
    \RevCheck{abc}{#3}%
  }%
  \def\Op{And}%
  \Test{1}{111}{1}%
  \Test{111}{1}{1}%
  \Test{10}{111}{10}%
  \Test{111}{10}{10}%
  \Test{111}{1000}{0}%
  \Test{1000}{111}{0}%
  \def\Op{AndNot}%
  \Test{1010}{11}{1000}%
  \Test{100}{100}{0}%
  \Test{111}{1111}{0}%
  \Test{100}{111}{0}%
  \def\Op{Or}%
  \Test{0}{0}{0}%
  \Test{1}{0}{1}%
  \Test{0}{1}{1}%
  \Test{1}{1}{1}%
  \Test{1000}{10}{1010}%
  \Test{10}{1000}{1010}%
  \def\Op{Xor}%
  \Test{0}{0}{0}%
  \Test{1}{0}{1}%
  \Test{0}{1}{1}%
  \Test{1}{1}{0}%
  \Test{1000}{10}{1010}%
  \Test{10}{1000}{1010}%
  \Test   {110011001100}%
       {111000111000111}%
       {111110100001011}%
  \Test{111000111000111}%
          {110011001100}%
       {111110100001011}%
\end{qstest}

\begin{qstest}{GetUndef}{GetUndef, GetBin, GetOct, GetHex}
  \def\TestUndef#1#2{%
    \let\BS@abc\@undefined
    \expandafter\expandafter\expandafter\Expect
    \expandafter\expandafter\expandafter{%
      \x{abc}{#1}%
    }{#2}%
  }%
  \let\x\bitsetGetBin
  \TestUndef{-1}{0}%
  \TestUndef{0}{0}%
  \TestUndef{1}{0}%
  \TestUndef{2}{00}%
  \TestUndef{8}{00000000}%
  \let\x\bitsetGetOct
  \TestUndef{-1}{0}%
  \TestUndef{0}{0}%
  \TestUndef{1}{0}%
  \TestUndef{2}{0}%
  \TestUndef{3}{0}%
  \TestUndef{4}{00}%
  \TestUndef{5}{00}%
  \TestUndef{6}{00}%
  \TestUndef{7}{000}%
  \TestUndef{8}{000}%
  \TestUndef{9}{000}%
  \TestUndef{10}{0000}%
  \let\x\bitsetGetHex
  \TestUndef{-1}{0}%
  \TestUndef{0}{0}%
  \TestUndef{1}{0}%
  \TestUndef{2}{0}%
  \TestUndef{3}{0}%
  \TestUndef{4}{0}%
  \TestUndef{5}{00}%
  \TestUndef{6}{00}%
  \TestUndef{7}{00}%
  \TestUndef{8}{00}%
  \TestUndef{9}{000}%
  \TestUndef{10}{000}%
  \TestUndef{12}{000}%
  \TestUndef{13}{0000}%
  \TestUndef{16}{0000}%
  \TestUndef{17}{00000}%
\end{qstest}

\begin{qstest}{SetBin}{SetBin}
  \def\Test#1#2{%
    \let\BS@abc\@undefined
    \bitsetSetBin{abc}{#1}%
    \expandafter\Expect\expandafter{\BS@abc}{#2}%
  }%
  \Test{}{0}%
  \Test{0}{0}%
  \Test{1}{1}%
  \Test{10}{01}%
  \Test{11}{11}%
  \Test{010}{01}%
  \Test{011}{11}%
  \Test{0010}{01}%
  \Test{1010}{0101}%
\end{qstest}

\begin{qstest}{SetOct}{SetOct}
  \def\Test#1#2{%
    \bitsetSetOct{abc}{#1}%
    \expandafter\Expect\expandafter{\BS@abc}{#2}%
  }%
  \Test{}{0}%
  \Test{0}{0}%
  \Test{000}{0}%
  \Test{1}{1}%
  \Test{001}{1}%
  \Test{010}{0001}%
  \Test{020}{00001}%
  \Test{42}{010001}%
  \Test{377}{11111111}%
  \Test{0377}{11111111}%
  \Test{76543210}{000100010110001101011111}%
  \Test{ 0 7 0 7 1 }{100111000111}%
\end{qstest}

\begin{qstest}{SetHex}{SetHex}
  \def\Test#1#2{%
    \bitsetSetHex{abc}{#1}%
    \expandafter\Expect\expandafter{\BS@abc}{#2}%
  }%
  \Test{}{0}%
  \Test{0}{0}%
  \Test{000}{0}%
  \Test{1}{1}%
  \Test{001}{1}%
  \Test{010}{00001}%
  \Test{020}{000001}%
  \Test{42}{0100001}%
  \Test{3F}{111111}%
  \Test{03F}{111111}%
  \Test{43210}{0000100001001100001}%
  \Test{98765}{10100110111000011001}%
  \Test{FEDCBA}{010111010011101101111111}%
  \Test{ 0 F 0 F 1 }{1000111100001111}%
\end{qstest}

\begin{qstest}{SetDec}{SetDec}
  \def\Test#1#2{%
    \bitsetSetDec{abc}{#1}%
    \expandafter\Expect\expandafter{\BS@abc}{#2}%
  }%
  \Test{}{0}%
  \Test{0}{0}%
  \Test{000}{0}%
  \Test{1}{1}%
  \Test{7}{111}%
  \Test{8}{0001}%
  \Test{001}{1}%
  \Test{010}{0101}%
  \Test{020}{00101}%
  \Test{53}{101011}%
  \Test{255}{11111111}%
  \Test{256}{000000001}%
  \Test{999999999}{111111111001001101011001110111}%
  \Test{1000000000}{000000000101001101011001110111}%
  \Test{4210987654}{01100001010010010111111101011111}%
  \Test{2147483647}{1111111111111111111111111111111}%
  \Test{2147483648}{00000000000000000000000000000001}%
\end{qstest}

\begin{qstest}{GetBin}{GetBin}
  \def\TestUndef#1#2{%
    \let\BS@abc\@undefined
    \expandafter\expandafter\expandafter\Expect
    \expandafter\expandafter\expandafter{%
      \bitsetGetBin{abc}{#1}%
    }{#2}%
  }%
  \TestUndef{-1}{0}%
  \TestUndef{0}{0}%
  \TestUndef{1}{0}%
  \TestUndef{2}{00}%
  \TestUndef{8}{00000000}%
  \def\Test#1#2{%
    \bitsetSetBin{abc}{#2}%
    \expandafter\expandafter\expandafter\Expect
    \expandafter\expandafter\expandafter{%
      \bitsetGetBin{abc}{#1}%
    }{#2}%
  }%
  \Test{-1}{0}%
  \Test{0}{0}%
  \Test{1}{0}%
  \Test{1}{1}%
  \Test{2}{01}%
  \Test{2}{10}%
  \Test{3}{010}%
  \Test{2}{00}%
  \Test{2}{01}%
  \Test{8}{00101100}%
  \Test{2}{10101}%
  \Test{-100}{11011}%
\end{qstest}

\begin{qstest}{GetOct}{GetOct}
  \def\Test#1#2#3{%
    \edef\x{\zap@space#1 \@empty}%
    \edef\x{\noexpand\bitsetSetBin{abc}{\x}}%
    \x
    \expandafter\expandafter\expandafter\Expect
    \expandafter\expandafter\expandafter{%
      \bitsetGetOct{abc}{#2}%
    }{#3}%
  }%
  \Test{111 110 101 100 011 010 001 000}{0}{76543210}%
  \Test{000 111}{0}{7}%
  \Test{101 000}{-1}{50}%
  \Test{111}{-1}{7}%
  \Test{111}{0}{7}%
  \Test{111}{1}{7}%
  \Test{111}{3}{7}%
  \Test{111}{4}{07}%
  \Test{111}{6}{07}%
  \Test{111}{7}{007}%
  \Test{111 010}{6}{72}%
  \Test{111 010}{7}{072}%
  \Test{011 111}{0}{37}%
  \Test{011 111}{6}{37}%
  \Test{011 111}{7}{037}%
  \Test{001 111}{0}{17}%
  \Test{001 111}{6}{17}%
  \Test{001 111}{7}{017}%
\end{qstest}

\begin{qstest}{GetHex}{GetHex}
  \def\Test#1#2#3{%
    \bitsetSetBin{abc}{#1}%
    \expandafter\expandafter\expandafter\Expect
    \expandafter\expandafter\expandafter{%
      \bitsetGetHex{abc}{#2}%
    }{#3}%
  }%
  \Test{1111 1110 1101 1100 1011 1010 1001 1000}{0}{FEDCBA98}%
  \Test{0111 0110 0101 0100 0011 0010 0001 0000}{0}{76543210}%
  \Test{0000 1111}{0}{F}%
  \Test{0101 0000}{-1}{50}%
  \Test{1111}{-1}{F}%
  \Test{1111}{0}{F}%
  \Test{1111}{1}{F}%
  \Test{1111}{4}{F}%
  \Test{1111}{5}{0F}%
  \Test{1111}{8}{0F}%
  \Test{1111}{9}{00F}%
  \Test{1111 0010}{8}{F2}%
  \Test{1111 0010}{9}{0F2}%
  \Test{0111 1111}{0}{7F}%
  \Test{0111 1111}{8}{7F}%
  \Test{0111 1111}{9}{07F}%
  \Test{0011 1111}{0}{3F}%
  \Test{0011 1111}{8}{3F}%
  \Test{0011 1111}{9}{03F}%
  \Test{0001 1111}{0}{1F}%
  \Test{0001 1111}{8}{1F}%
  \Test{0001 1111}{9}{01F}%
\end{qstest}

\begin{qstest}{Range}{Range}
  \TestError{%
    Wrong index numbers in range [9..8]\MessageBreak% hash-ok
    for clear/set/flip on bit set `abc'.\MessageBreak
    The lower index exceeds the upper index.\MessageBreak
    Canceling the operation as error recovery%
  }{%
    \bitsetSetRange{abc}{9}{8}%
  }%
  \def\TestErrorNegInd#1#2#3#4#5#6{%
    \TestError{%
      Negative index in range [#2..#3]\MessageBreak % hash-ok
      for \string\bitset #1Range on bit set `abc'.\MessageBreak
      Using [#4..#5] as error recovery% hash-ok
    }{%
      \csname bitset#1Range\endcsname{abc}{#2}{#3}%
      \global\let\BS@global\BS@abc
    }%
    \Check{global}{#6}%
  }%
  \Set{abc}{111}%
  \TestErrorNegInd{Clear}{-1}{0}{0}{0}{111}%
  \TestErrorNegInd{Clear}{0}{-1}{0}{0}{111}%
  \TestErrorNegInd{Clear}{-2}{2}{0}{2}{001}%
  \bitsetReset{abc}%
  \TestErrorNegInd{Set}{-1}{0}{0}{0}{0}%
  \TestErrorNegInd{Set}{0}{-1}{0}{0}{0}%
  \TestErrorNegInd{Set}{-2}{2}{0}{2}{11}%
  \Set{abc}{101}%
  \TestErrorNegInd{Flip}{-1}{0}{0}{0}{101}%
  \TestErrorNegInd{Flip}{0}{-1}{0}{0}{101}%
  \TestErrorNegInd{Flip}{-2}{2}{0}{2}{011}%
  \def\Test#1#2#3#4{%
    \bitsetSetBin{abc}{#1}%
    \csname bitset\TestOp Range\endcsname{abc}{#2}{#3}%
    \Expect*{\bitsetGetBin{abc}{0}}{#4}%
  }%
  \def\TestOp{Clear}%
  \Test{0}{0}{1}{0}%
  \Test{1111}{1}{2}{1101}%
  \Test{1111}{1}{3}{1001}%
  \Test{1111111100000000}{12}{14}{1100111100000000}%
  \def\TestOp{Set}%
  \Test{0}{0}{1}{1}%
  \Test{1000}{1}{2}{1010}%
  \Test{0}{1}{2}{10}%
  \Test{1}{12}{15}{111000000000001}%
  \Test{1111}{1}{3}{1111}%
  \Test{1000000000000000}{12}{14}{1011000000000000}%
  \def\TestOp{Flip}%
  \Test{0}{0}{1}{1}%
  \Test{1}{0}{1}{0}%
  \Test{10101010}{1}{5}{10110100}%
  \def\Test#1#2#3#4#5{%
    \bitsetSetBin{abc}{#1}%
    \bitsetSetValueRange{abc}{#2}{#3}{#4}%
    \Expect*{\bitsetGetBin{abc}{0}}{#5}%
  }%
  \Test{0}{0}{1}{0}{0}%
  \Test{0}{0}{1}{1}{1}%
  \Test{1010}{1}{3}{0}{1000}%
  \Test{1010}{1}{3}{1}{1110}%
\end{qstest}

\begin{qstest}{ShiftLeft/ShiftRight}{ShiftLeft/ShiftRight}
  \def\@Test#1#2{%
    \let\BS@abc\@undefined
    \csname bitsetShift#1\endcsname{abc}{#2}%
    \Expect*{\BS@abc}{0}%
  }%
  \def\Test#1{%
    \@Test{Left}{#1}%
    \@Test{Right}{#1}%
  }%
  \Test{-16}%
  \Test{-1}%
  \Test{0}%
  \Test{1}%
  \Test{16}%
  \def\Test#1#2#3{%
    \bitsetSetBin{abc}{#1}%
    \bitsetSetBin{result}{#3}%
    \csname bitsetShift\Op\endcsname{abc}{#2}%
    \Expect*{\bitsetGetBin{abc}{0}}*{\bitsetGetBin{result}{0}}%
  }%
  \def\Op{Left}%
  \Test{0}{0}{0}%
  \Test{0}{1}{0}%
  \Test{0}{-1}{0}%
  \Test{1}{0}{1}%
  \Test{1}{1}{10}%
  \Test{1}{-1}{0}%
  \Test{10}{1}{100}%
  \Test{10}{-1}{1}%
  \Test{1}{32}{100000000000000000000000000000000}%
  \Test{1}{-100}{0}%
  \def\Op{Right}%
  \Test{0}{0}{0}%
  \Test{0}{1}{0}%
  \Test{0}{-1}{0}%
  \Test{1}{0}{1}%
  \Test{1}{1}{0}%
  \Test{1}{-1}{10}%
  \Test{10}{1}{1}%
  \Test{10}{-1}{100}%
  \Test{1}{-32}{100000000000000000000000000000000}%
  \Test{1}{100}{0}%
  \Test{110110110110110}{10}{11011}%
  \Test{110110110110110}{100}{0}%
  \Test{1}{100000}{0}%
\end{qstest}

\begin{qstest}{Profile: Set}{Profile: Set}
  \bitsetSet{abc}{4095}%
  \global\let\BS@global\BS@abc
\end{qstest}

\begin{qstest}{Profile: Get}{Profile: Get}
  \edef\x{\bitsetGet{global}{4095}}%
\end{qstest}

\begin{document}
\end{document}
%</test2>
%    \end{macrocode}
%
% \section{Installation}
%
% \subsection{Download}
%
% \paragraph{Package.} This package is available on
% CTAN\footnote{\url{ftp://ftp.ctan.org/tex-archive/}}:
% \begin{description}
% \item[\CTAN{macros/latex/contrib/oberdiek/bitset.dtx}] The source file.
% \item[\CTAN{macros/latex/contrib/oberdiek/bitset.pdf}] Documentation.
% \end{description}
%
%
% \paragraph{Bundle.} All the packages of the bundle `oberdiek'
% are also available in a TDS compliant ZIP archive. There
% the packages are already unpacked and the documentation files
% are generated. The files and directories obey the TDS standard.
% \begin{description}
% \item[\CTAN{install/macros/latex/contrib/oberdiek.tds.zip}]
% \end{description}
% \emph{TDS} refers to the standard ``A Directory Structure
% for \TeX\ Files'' (\CTAN{tds/tds.pdf}). Directories
% with \xfile{texmf} in their name are usually organized this way.
%
% \subsection{Bundle installation}
%
% \paragraph{Unpacking.} Unpack the \xfile{oberdiek.tds.zip} in the
% TDS tree (also known as \xfile{texmf} tree) of your choice.
% Example (linux):
% \begin{quote}
%   |unzip oberdiek.tds.zip -d ~/texmf|
% \end{quote}
%
% \paragraph{Script installation.}
% Check the directory \xfile{TDS:scripts/oberdiek/} for
% scripts that need further installation steps.
% Package \xpackage{attachfile2} comes with the Perl script
% \xfile{pdfatfi.pl} that should be installed in such a way
% that it can be called as \texttt{pdfatfi}.
% Example (linux):
% \begin{quote}
%   |chmod +x scripts/oberdiek/pdfatfi.pl|\\
%   |cp scripts/oberdiek/pdfatfi.pl /usr/local/bin/|
% \end{quote}
%
% \subsection{Package installation}
%
% \paragraph{Unpacking.} The \xfile{.dtx} file is a self-extracting
% \docstrip\ archive. The files are extracted by running the
% \xfile{.dtx} through \plainTeX:
% \begin{quote}
%   \verb|tex bitset.dtx|
% \end{quote}
%
% \paragraph{TDS.} Now the different files must be moved into
% the different directories in your installation TDS tree
% (also known as \xfile{texmf} tree):
% \begin{quote}
% \def\t{^^A
% \begin{tabular}{@{}>{\ttfamily}l@{ $\rightarrow$ }>{\ttfamily}l@{}}
%   bitset.sty & tex/generic/oberdiek/bitset.sty\\
%   bitset.pdf & doc/latex/oberdiek/bitset.pdf\\
%   test/bitset-test1.tex & doc/latex/oberdiek/test/bitset-test1.tex\\
%   test/bitset-test2.tex & doc/latex/oberdiek/test/bitset-test2.tex\\
%   test/bitset-test3.tex & doc/latex/oberdiek/test/bitset-test3.tex\\
%   bitset.dtx & source/latex/oberdiek/bitset.dtx\\
% \end{tabular}^^A
% }^^A
% \sbox0{\t}^^A
% \ifdim\wd0>\linewidth
%   \begingroup
%     \advance\linewidth by\leftmargin
%     \advance\linewidth by\rightmargin
%   \edef\x{\endgroup
%     \def\noexpand\lw{\the\linewidth}^^A
%   }\x
%   \def\lwbox{^^A
%     \leavevmode
%     \hbox to \linewidth{^^A
%       \kern-\leftmargin\relax
%       \hss
%       \usebox0
%       \hss
%       \kern-\rightmargin\relax
%     }^^A
%   }^^A
%   \ifdim\wd0>\lw
%     \sbox0{\small\t}^^A
%     \ifdim\wd0>\linewidth
%       \ifdim\wd0>\lw
%         \sbox0{\footnotesize\t}^^A
%         \ifdim\wd0>\linewidth
%           \ifdim\wd0>\lw
%             \sbox0{\scriptsize\t}^^A
%             \ifdim\wd0>\linewidth
%               \ifdim\wd0>\lw
%                 \sbox0{\tiny\t}^^A
%                 \ifdim\wd0>\linewidth
%                   \lwbox
%                 \else
%                   \usebox0
%                 \fi
%               \else
%                 \lwbox
%               \fi
%             \else
%               \usebox0
%             \fi
%           \else
%             \lwbox
%           \fi
%         \else
%           \usebox0
%         \fi
%       \else
%         \lwbox
%       \fi
%     \else
%       \usebox0
%     \fi
%   \else
%     \lwbox
%   \fi
% \else
%   \usebox0
% \fi
% \end{quote}
% If you have a \xfile{docstrip.cfg} that configures and enables \docstrip's
% TDS installing feature, then some files can already be in the right
% place, see the documentation of \docstrip.
%
% \subsection{Refresh file name databases}
%
% If your \TeX~distribution
% (\teTeX, \mikTeX, \dots) relies on file name databases, you must refresh
% these. For example, \teTeX\ users run \verb|texhash| or
% \verb|mktexlsr|.
%
% \subsection{Some details for the interested}
%
% \paragraph{Attached source.}
%
% The PDF documentation on CTAN also includes the
% \xfile{.dtx} source file. It can be extracted by
% AcrobatReader 6 or higher. Another option is \textsf{pdftk},
% e.g. unpack the file into the current directory:
% \begin{quote}
%   \verb|pdftk bitset.pdf unpack_files output .|
% \end{quote}
%
% \paragraph{Unpacking with \LaTeX.}
% The \xfile{.dtx} chooses its action depending on the format:
% \begin{description}
% \item[\plainTeX:] Run \docstrip\ and extract the files.
% \item[\LaTeX:] Generate the documentation.
% \end{description}
% If you insist on using \LaTeX\ for \docstrip\ (really,
% \docstrip\ does not need \LaTeX), then inform the autodetect routine
% about your intention:
% \begin{quote}
%   \verb|latex \let\install=y\input{bitset.dtx}|
% \end{quote}
% Do not forget to quote the argument according to the demands
% of your shell.
%
% \paragraph{Generating the documentation.}
% You can use both the \xfile{.dtx} or the \xfile{.drv} to generate
% the documentation. The process can be configured by the
% configuration file \xfile{ltxdoc.cfg}. For instance, put this
% line into this file, if you want to have A4 as paper format:
% \begin{quote}
%   \verb|\PassOptionsToClass{a4paper}{article}|
% \end{quote}
% An example follows how to generate the
% documentation with pdf\LaTeX:
% \begin{quote}
%\begin{verbatim}
%pdflatex bitset.dtx
%makeindex -s gind.ist bitset.idx
%pdflatex bitset.dtx
%makeindex -s gind.ist bitset.idx
%pdflatex bitset.dtx
%\end{verbatim}
% \end{quote}
%
% \section{Catalogue}
%
% The following XML file can be used as source for the
% \href{http://mirror.ctan.org/help/Catalogue/catalogue.html}{\TeX\ Catalogue}.
% The elements \texttt{caption} and \texttt{description} are imported
% from the original XML file from the Catalogue.
% The name of the XML file in the Catalogue is \xfile{bitset.xml}.
%    \begin{macrocode}
%<*catalogue>
<?xml version='1.0' encoding='us-ascii'?>
<!DOCTYPE entry SYSTEM 'catalogue.dtd'>
<entry datestamp='$Date$' modifier='$Author$' id='bitset'>
  <name>bitset</name>
  <caption>Handle bit-vector datatype.</caption>
  <authorref id='auth:oberdiek'/>
  <copyright owner='Heiko Oberdiek' year='2007,2011'/>
  <license type='lppl1.3'/>
  <version number='1.1'/>
  <description>
    This package defines and implements the data type bit set,
    a vector of bits. The size of the vector may grow dynamically.
    Individual bits can be manipulated.
    <p/>
    The package is part of the <xref refid='oberdiek'>oberdiek</xref> bundle.
  </description>
  <documentation details='Package documentation'
      href='ctan:/macros/latex/contrib/oberdiek/bitset.pdf'/>
  <ctan file='true' path='/macros/latex/contrib/oberdiek/bitset.dtx'/>
  <miktex location='oberdiek'/>
  <texlive location='oberdiek'/>
  <install path='/macros/latex/contrib/oberdiek/oberdiek.tds.zip'/>
</entry>
%</catalogue>
%    \end{macrocode}
%
% \begin{History}
%   \begin{Version}{2007/09/28 v1.0}
%   \item
%     First version.
%   \end{Version}
%   \begin{Version}{2011/01/30 v1.1}
%   \item
%     Already loaded package files are not input in \hologo{plainTeX}.
%   \end{Version}
% \end{History}
%
% \PrintIndex
%
% \Finale
\endinput

%        (quote the arguments according to the demands of your shell)
%
% Documentation:
%    (a) If bitset.drv is present:
%           latex bitset.drv
%    (b) Without bitset.drv:
%           latex bitset.dtx; ...
%    The class ltxdoc loads the configuration file ltxdoc.cfg
%    if available. Here you can specify further options, e.g.
%    use A4 as paper format:
%       \PassOptionsToClass{a4paper}{article}
%
%    Programm calls to get the documentation (example):
%       pdflatex bitset.dtx
%       makeindex -s gind.ist bitset.idx
%       pdflatex bitset.dtx
%       makeindex -s gind.ist bitset.idx
%       pdflatex bitset.dtx
%
% Installation:
%    TDS:tex/generic/oberdiek/bitset.sty
%    TDS:doc/latex/oberdiek/bitset.pdf
%    TDS:doc/latex/oberdiek/test/bitset-test1.tex
%    TDS:doc/latex/oberdiek/test/bitset-test2.tex
%    TDS:doc/latex/oberdiek/test/bitset-test3.tex
%    TDS:source/latex/oberdiek/bitset.dtx
%
%<*ignore>
\begingroup
  \catcode123=1 %
  \catcode125=2 %
  \def\x{LaTeX2e}%
\expandafter\endgroup
\ifcase 0\ifx\install y1\fi\expandafter
         \ifx\csname processbatchFile\endcsname\relax\else1\fi
         \ifx\fmtname\x\else 1\fi\relax
\else\csname fi\endcsname
%</ignore>
%<*install>
\input docstrip.tex
\Msg{************************************************************************}
\Msg{* Installation}
\Msg{* Package: bitset 2011/01/30 v1.1 Handle bit-vector datatype (HO)}
\Msg{************************************************************************}

\keepsilent
\askforoverwritefalse

\let\MetaPrefix\relax
\preamble

This is a generated file.

Project: bitset
Version: 2011/01/30 v1.1

Copyright (C) 2007, 2011 by
   Heiko Oberdiek <heiko.oberdiek at googlemail.com>

This work may be distributed and/or modified under the
conditions of the LaTeX Project Public License, either
version 1.3c of this license or (at your option) any later
version. This version of this license is in
   http://www.latex-project.org/lppl/lppl-1-3c.txt
and the latest version of this license is in
   http://www.latex-project.org/lppl.txt
and version 1.3 or later is part of all distributions of
LaTeX version 2005/12/01 or later.

This work has the LPPL maintenance status "maintained".

This Current Maintainer of this work is Heiko Oberdiek.

The Base Interpreter refers to any `TeX-Format',
because some files are installed in TDS:tex/generic//.

This work consists of the main source file bitset.dtx
and the derived files
   bitset.sty, bitset.pdf, bitset.ins, bitset.drv, bitset-test1.tex,
   bitset-test2.tex, bitset-test3.tex.

\endpreamble
\let\MetaPrefix\DoubleperCent

\generate{%
  \file{bitset.ins}{\from{bitset.dtx}{install}}%
  \file{bitset.drv}{\from{bitset.dtx}{driver}}%
  \usedir{tex/generic/oberdiek}%
  \file{bitset.sty}{\from{bitset.dtx}{package}}%
  \usedir{doc/latex/oberdiek/test}%
  \file{bitset-test1.tex}{\from{bitset.dtx}{test1}}%
  \file{bitset-test2.tex}{\from{bitset.dtx}{test2,etex}}%
  \file{bitset-test3.tex}{\from{bitset.dtx}{test2,noetex}}%
  \nopreamble
  \nopostamble
  \usedir{source/latex/oberdiek/catalogue}%
  \file{bitset.xml}{\from{bitset.dtx}{catalogue}}%
}

\catcode32=13\relax% active space
\let =\space%
\Msg{************************************************************************}
\Msg{*}
\Msg{* To finish the installation you have to move the following}
\Msg{* file into a directory searched by TeX:}
\Msg{*}
\Msg{*     bitset.sty}
\Msg{*}
\Msg{* To produce the documentation run the file `bitset.drv'}
\Msg{* through LaTeX.}
\Msg{*}
\Msg{* Happy TeXing!}
\Msg{*}
\Msg{************************************************************************}

\endbatchfile
%</install>
%<*ignore>
\fi
%</ignore>
%<*driver>
\NeedsTeXFormat{LaTeX2e}
\ProvidesFile{bitset.drv}%
  [2011/01/30 v1.1 Handle bit-vector datatype (HO)]%
\documentclass{ltxdoc}
\usepackage{holtxdoc}[2011/11/22]
\usepackage{amstext}
\usepackage{ifluatex}
\ifluatex
\else
  \usepackage[T1]{fontenc}%
  \usepackage{lmodern}%
\fi
\begin{document}
  \DocInput{bitset.dtx}%
\end{document}
%</driver>
% \fi
%
% \CheckSum{3510}
%
% \CharacterTable
%  {Upper-case    \A\B\C\D\E\F\G\H\I\J\K\L\M\N\O\P\Q\R\S\T\U\V\W\X\Y\Z
%   Lower-case    \a\b\c\d\e\f\g\h\i\j\k\l\m\n\o\p\q\r\s\t\u\v\w\x\y\z
%   Digits        \0\1\2\3\4\5\6\7\8\9
%   Exclamation   \!     Double quote  \"     Hash (number) \#
%   Dollar        \$     Percent       \%     Ampersand     \&
%   Acute accent  \'     Left paren    \(     Right paren   \)
%   Asterisk      \*     Plus          \+     Comma         \,
%   Minus         \-     Point         \.     Solidus       \/
%   Colon         \:     Semicolon     \;     Less than     \<
%   Equals        \=     Greater than  \>     Question mark \?
%   Commercial at \@     Left bracket  \[     Backslash     \\
%   Right bracket \]     Circumflex    \^     Underscore    \_
%   Grave accent  \`     Left brace    \{     Vertical bar  \|
%   Right brace   \}     Tilde         \~}
%
% \GetFileInfo{bitset.drv}
%
% \title{The \xpackage{bitset} package}
% \date{2011/01/30 v1.1}
% \author{Heiko Oberdiek\\\xemail{heiko.oberdiek at googlemail.com}}
%
% \maketitle
%
% \begin{abstract}
% This package defines and implements the data type bit set,
% a vector of bits. The size of the vector may grow dynamically.
% Individual bits can be manipulated.
% \end{abstract}
%
% \tableofcontents
%
% \clearpage
% \section{Documentation}
%
% \subsection{Introduction}
%
% Annotations in the PDF format know entries whose values are
% integers. This numbers are interpreted as set of flags
% specifying properties. For example, annotation dictionaries
% can have a key |/F|. The bits of its integer value
% are interpreted the following way:
% \begin{quote}
% \begin{tabular}{@{}l|l@{}}
%   Bit position & Property name\\
%   \hline
%   1 & Invisible\\
%   2 & Hidden\\
%   3 & Print\\
%   4 & NoZoom\\
%   5 & NoRotate\\
%   6 & NoView\\
%   7 & ReadOnly\\
%   \dots & \dots
% \end{tabular}
% \end{quote}
% Now, let's see how these values are set in package \xpackage{hyperref}
% before it uses this package (before v6.77a):
%\begin{quote}
%\begin{verbatim}
%\ifFld@hidden /F 6\else /F 4\fi
%\end{verbatim}
%\end{quote}
% Where are the other flags? The following example for key |/Ff|
% in a widget annotation
% supports at least three properties:
%\begin{quote}
%\begin{verbatim}
%\ifFld@multiline
%  \ifFld@readonly /Ff 4097\else /Ff 4096\fi
%\else
%  \ifFld@password
%    \ifFld@readonly /Ff 8193\else /Ff 8192\fi
%  \else
%    \ifFld@readonly /Ff 1\fi
%  \fi
%\fi
%\end{verbatim}
%\end{quote}
% But you see the point. It would be a nightmare to continue
% this way in supporting the missing flag settings. This kind
% of integers may have up to 32 bits.
%
% Therefore I wanted a data structure for setting and
% clearing individual bits. Also it should provide
% an export as decimal number. The snipsets above are
% executed in expansion contexts without \TeX's stomach
% commands. It would be convenient to have an expandable
% conversion from the data structure to the integer that
% gets written to the PDF file.
%
% This package \xpackage{bitset} implements such a data
% structure. The interface is quite close to Java's
% class |BitSet| in order not to learn to many interfaces
% for the same kind of data structure.
%
% \subsection{Glossary}
%
% \begin{description}
% \item[Bit set:]
%   A bit set is a vector of bits or flags.
%   The vector size is unlimited and
%   grows dynamically. An undefined bit set is treated as bit
%   set where all bits are cleared.
%
%   Bit sets are addressed by name. A name should consists of
%   letters or digits. Technically it must survive \cs{csname},
%   see \LaTeX's environment names for other names with such
%   a constraint. Package \xpackage{babel}'s shorthands are
%   not supported due to technical reasons. Shorthand support
%   would break expandable operations.
% \item[Size:]
%   A size of a bit set is the number of bits in use.
%   It's the number of the highest index, incremented by one.
%   Sizes are in the range |0| up to |2147483647|, the highest
%   number supported by \TeX.
% \item[Index:]
%   Bit positions in a bit set are addressed by an index number.
%   The bit vector is zero based. The first and least significant
%   bit is addressed by index |0| and the highest possible bit
%   by |2147483646|.
% \item[Bit:]
%   A bit is enoded as |0| for cleared/disabled or |1| for set/enabled.
% \end{description}
%
% \subsection{Design principles}
%
% \begin{description}
% \item[Name conventions:]
%   To avoid conflicts with existing macro names, the operations
%   are prefixed by the package name.
% \item[Zero based indexes:] The first bit is addressed by zero.
%   (Convention of array indexing in C, Java, \dots)
% \item[Unlimited size:]
%   There is no restriction on the size of a bit set other than
%   usual memory limitations. \cs{bitsetSetDec} and \cs{bitsetGetDec}
%   transparently switch to package \xpackage{bigintcalc} if the numbers
%   get too large for \TeX's number limit.
% \item[Expandibility:]
%   Any operation that does not change the bit set is expandable.
%   And all operations that extract or calculate some result
%   do this in exact two expansion steps. For example,
%   a macro \cs{Macro} wants a bit set as decimal number.
%   But the argument must be a plain number without macros.
%   Thus you could prefix \cs{bitsetGetDec} with \cs{number}.
%   However this won't work for bit sets with 31 or more bits
%   because of \TeX's number limit of $2^{31}-1$.
%   then just hit the operator with two \cs{expandafter}:
%   \begin{quote}
%   |\expandafter\expandafter\expandafter|\\
%   |\Macro\bitsetGetDec{foo}|
%   \end{quote}
%   |\bitsetGetDec| is hit first by the third \cs{expandafter}
%   and then by the second one.
% \item[Format independence:] This package is written as \LaTeX\ package,
%   but it does not depend on \LaTeX. It will also work for other formats
%   such as \plainTeX.
% \item[Independence from \TeX\ engines:]
%   Vanilla \TeX\ is all you need. Calculations are delegated to
%   packages \xpackage{intcalc} and \xpackage{bigintcalc}.
%   They don't need any special features, but they will switch
%   to a little more efficient implementation if features such
%   as \cs{numexpr} are available.
% \item[Numeric arguments:] Anything that is accepted by
%   \cs{number}. If \eTeX\ is detected, also expressions
%   for \cs{numexpr} are supported. The only exception so far
%   is the number for \cs{bitsetSetDec}. The number might be too
%   large for \cs{number} or \cs{numexpr}.
% \item[Error messages:]
%   In expandable contexts, only a limited set of \TeX\ primitive
%   commands work as expected. So called stomach commands behave
%   like \cs{relax} and don't get expanded or executed.
%   Unhappily also the error commands belong to this category.
%   The expandable operations will throw an unknown control
%   sequence instead to get \TeX's and user's attention.
%   The name of these control sequences starts with
%   \cs{BitSetError:} with the type of error after the colon.
% \end{description}
%
% \subsection{Operator overview}
%
% \begin{minipage}{\linewidth}
% \begin{description}
% \def\It#1#2{^^A
%   \item[^^A
%     {#1} ^^A
%     \textmd{(\hyperref[{sec:#2}]{section \ref*{sec:#2}})}^^A
%   ]^^A
%   \mbox{}\\^^A
% }
% \def\op#1{^^A
%   \cs{bitset}\texttt{\bfseries#1}^^A
% }
% \def\M#1{\meta{#1}}
% \def\Q{\hspace*{\fill}}
% \It{Miscellaneous}{misc}
%   \op{Reset}\Q \M{BitSet}\\
%   \op{Let}\Q \M{BitSet A} \M{BitSet B}
% \It{Import}{import}
%   \op{SetBin}, \op{SetOct}, \op{SetHex}\Q \M{BitSet} \M{Value}\\
%   \op{SetDec}\Q \M{BitSet} \M{Value}
% \It{Export\textsuperscript{a}}{export}
%   \op{GetBin}, \op{GetOct}, \op{GetHex}\Q \M{BitSet} \M{MinSize}\\
%   \op{GetDec}\Q \M{BitSet}
% \It{Logical operators}{log}
%   \op{And}, \op{AndNot}\Q \M{BitSet A} \M{BitSet B}\\
%   \op{Or}, \op{Xor}\Q \M{BitSet A} \M{BitSet B}
% \It{Shifting}{shift}
%   \op{ShiftLeft}, \op{ShiftRight}\Q \M{BitSet} \M{ShiftAmount}\\
% \It{Bit manipulation}{bitman}
%   \op{Clear}, \op{Set}, \op{Flip}\Q \M{BitSet} \M{Index}\\
%   \op{SetValue}\Q \M{BitSet} \M{Index} \M{Value}\\
%   \op{ClearRange}, \op{SetRange}, \op{FlipRange}\\
%     \Q \M{BitSet} \M{IndexFrom} \M{IndexTo}\\
%   \op{SetValueRange}\Q \M{BitSet} \M{IndexFrom} \M{IndexTo}
% \It{Bit retrieval\textsuperscript{a}}{bitret}
%   \op{Get}\Q \M{BitSet} \M{Index}\\
%   \op{NextClearBit}, \op{NextSetBit}\Q \M{BitSet} \M{Index}\\
%   \op{GetSetBitList}\Q \M{BitSet}
% \It{Bit set properties}{prop}
%   \op{Size}, \op{Cardinality}\Q \M{BitSet}
% \It{Queries\textsuperscript{b}}{query}
%   \op{IsDefined}, \op{IsEmpty}\Q \M{BitSet} \M{Then} \M{Else}\\
%   \op{Equals}, \op{Intersects}\Q \M{BitSet A} \M{BitSet B} \M{Then} \M{Else}\\
%   \op{Query}\Q \M{BitSet} \M{Index} \M{Then} \M{Else}
% \end{description}
% \footnotetext[1]{Macros are expandable, full expansion by two steps.}
% \footnotetext[2]{Macros are expandable.}
% \end{minipage}
%
% \subsection{Package loading}
%
% The package can be used as normal \LaTeX\ package:
%\begin{quote}
%|\usepackage{bitset}|
%\end{quote}
% Also \plainTeX\ is supported:
%\begin{quote}
%|\input bitset.sty\relax|
%\end{quote}
%
% \subsection{Operators}
%
% The following macros work on and with bit sets.
% A bit set \meta{BitSet} is represented by a name.
% The should consist of letters and digits. Technically
% it must survive \cs{csname}. It is the same constraint
% that must be satisfied by label or environment names
% in \LaTeX.
%
% However active characters that are shorthands
% of package \xpackage{babel} are not supported.
% Support for shorthands works by an assignment.
% But many operators
% such as \cs{bitsetGetDec} must be usable in expandable
% contexts. There assignments will not be executed in the
% best case or they will cause errors.
%
% The bits in a bit set are addressed by non-negative
% integers starting from zero. Thus negative index
% numbers cause an error message. Because index numbers
% are \TeX\ numbers. The largest index is 2147483647.
% But in practice memory limits and patience limits
% will be very likely reached much before.
%
% \subsubsection{Miscellaneous}\label{sec:misc}
%
% There isn't a separate operation for bit set creation.
% For simplicity an undefined bit set is treated as bit set
% with all bits cleared.
%
% \begin{declcs}{bitsetReset} \M{BitSet}
% \end{declcs}
% Macro \cs{bitsetReset} clears all bits. The result is an
% empty bit set. It may also be used as replacement for
% an operation ``new'', because an undefined bit set is defined
% afterwards.
%
% \begin{declcs}{bitsetLet} \M{BitSet A} \M{BitSet B}
% \end{declcs}
% Macro \cs{bitsetLet} performs a simple assignment similar
% to \TeX's \cs{let}. After the operation \meta{BitSet A} has
% the same value as \meta{BitSet B}. If \meta{BitSet B} is
% undefined, then \meta{BitSet A} will be the empty bit set.
%
% Note: If \meta{BitSet A} exists, it will be overwritten.
%
% \subsubsection{Import}\label{sec:import}
%
% \begin{declcs}{bitsetSetBin} \M{BitSet} \M{BinaryNumber}\\
%   \cs{bitsetSetOct} \M{BitSet} \M{OctalNumber}\\
%   \cs{bitsetSetHex} \M{BitSet} \M{HexadecimalNumber}
% \end{declcs}
% The numbers are interpreted as bit vectors and the flags in
% the bit \meta{BitSet} set are set accordingly.
% These numeric arguments are the only arguments where
% spaces are allowed. Then the numbers are easier to read.
%
% \begin{declcs}{bitsetSetDec} \M{BitSet} \M{DecimalNumber}
% \end{declcs}
% Macro \cs{bitsetSetDec} uses \meta{DecimalNumber} to
% set the bit set \meta{BitSet}.
% The numeric argument must expand
% to a plain number consisting of decimal digits without
% command tokens or spaces. Internally this argument is
% expanded only. It cannot be passed to \cs{number} or
% \cs{numexpr}, because the number may be too large for them.
% However \cs{number} or |\the\numexpr| may be used
% explicitely. This also helps for unexpandable number
% command tokens or registers (\cs{z@}, \cs{@ne}, \cs{count@}, \dots).
% Also \LaTeX'\ \cs{value} needs prefixing:
% \begin{quote}
% |\bitsetSetDec{foo}{\number\value{bar}}|
% \end{quote}
%
% \subsubsection{Export}\label{sec:export}
%
% \begin{declcs}{bitsetGetBin} \M{BitSet} \M{MinSize}\\
%   \cs{bitsetGetOct} \M{BitSet} \M{MinSize}\\
%   \cs{bitsetGetHex} \M{BitSet} \M{MinSize}
% \end{declcs}
% These macros returns the bit set as binary, octal or hexadecimal
% number. If the bit size is smaller than \meta{MinSize} the gap is
% filled with leading zeros.
% Example:
% \begin{quote}
% |\bitsetReset{abc}|\\
% |\bitsetSet{abc}{2}|\\
% |\bitsetGetBin{abc}{8}| $\rightarrow$ |00000100|\\
% |\bitsetSet{abc}{5}\bitsetSet{abc}{7}|\\
% |\bitsetGetHex{abc}{16}| $\rightarrow$ |00A2|
% \end{quote}
% Macro \cs{bitsetGetHex} uses the uppercase letters |A| to |F|.
% The catcode of the letters is one of 11 (letter) or 12 (other).
%
% \begin{declcs}{bitsetGetDec} \M{BitSet}
% \end{declcs}
% Macro \cs{bitsetGetDec} returns the bit set \meta{BitSet} as
% decimal number. The returned number can be larger than
% \TeX's number limit of $2^{31}-1$.
%
% \subsubsection{Logical operators}\label{sec:log}
%
% \begin{declcs}{bitsetAnd} \M{BitSet A} \M{BitSet B}
% \end{declcs}
% $A_{\text{new}} \mathrel{\mathop{:}}= A_{\text{old}}
% \mathbin{\text{and}} B \qquad (\forall\, \text{bits})$
%
% \begin{declcs}{bitsetAndNot} \M{BitSet A} \M{BitSet B}
% \end{declcs}
% $A_{\text{new}} \mathrel{\mathop{:}}= A_{\text{old}}
% \mathbin{\text{and}} (\text{not}\,B) \qquad (\forall\, \text{bits})$
%
% \begin{declcs}{bitsetOr} \M{BitSet A} \M{BitSet B}
% \end{declcs}
% $A_{\text{new}} \mathrel{\mathop{:}}= A_{\text{old}}
% \mathbin{\text{or}} B \qquad (\forall\, \text{bits})$
%
% \begin{declcs}{bitsetXor} \M{BitSet A} \M{BitSet B}
% \end{declcs}
% $A_{\text{new}} \mathrel{\mathop{:}}= A_{\text{old}}
% \mathbin{\text{xor}} B \qquad (\forall\, \text{bits})$
%
% \subsubsection{Shifting}\label{sec:shift}
%
% \begin{declcs}{bitsetShiftLeft} \M{BitSet} \M{ShiftAmount}\\
%   \cs{bitsetShiftRight} \M{BitSet} \M{ShiftAmount}
% \end{declcs}
% A left shift by one is a multiplication by two, thus left shifting
% moves the flags to higher positions. The new created low positions
% are filled by zeros.
%
% A right shift is the opposite, dividing by two, movint the bits
% to lower positions. The number will become smaller, the lowest
% bits are lost.
%
% If the \meta{ShiftAmount} is negative, it reverts the meaning of
% the shift operation. A left shift becomes a right shift.
% A \meta{ShiftAmount} of zero is ignored.
%
% \subsubsection{Bit manipulation}\label{sec:bitman}
%
% \begin{declcs}{bitsetClear} \M{BitSet} \M{Index}\\
%   \cs{bitsetSet} \M{BitSet} \M{Index}\\
%   \cs{bitsetFlip} \M{BitSet} \M{Index}
% \end{declcs}
% This macros manipulate a single bit in \meta{BitSet} addressed
% by \cs{Index}. Macro \cs{bitsetClear} disables the bit,
% \cs{bitsetSet} enables it and \cs{bitsetFlip} reverts the current
% setting of the bit.
%
% \begin{declcs}{bitsetSetValue} \M{BitSet} \M{Index} \M{Bit}
% \end{declcs}
% Macro \cs{bitsetSetValue} puts bit \meta{Bit} at position \meta{Index}
% in bit set \meta{BitSet}. \meta{Bit} must be a valid \TeX\ number
% equals to zero (disabled/cleared) or one (enabled/set).
%
% \subsubsection{Bit retrieval}\label{sec:bitret}
%
% \begin{declcs}{bitsetGet} \M{BitSet} \M{Index}
% \end{declcs}
% Macro \cs{bitsetGet} extracts the status of the bit at
% position \meta{Index} in bit set \meta{BitSet}.
% Digit |1| is returned if the bit is set/enabled. If the
% bit is cleared/disabled and in cases of an undefined bitset or
% an index number out of range the return value is |0|.
%
% \begin{declcs}{bitsetNextClearBit} \M{BitSet} \M{Index}
% \end{declcs}
% Starting at position \meta{Index} (inclusive) the bits
% are inspected. The first position without a set bit is returned.
% Possible results are decimal numbers: \meta{Index}, \meta{Index} + 1,
% \dots, ($\infty$)
%
% \begin{declcs}{bitsetNextSetBit} \M{BitSet} \M{Index}
% \end{declcs}
% Starting at position \meta{Index} (inclusive) the bits are
% inspected and the index position of the first found set bit
% is returned. If there isn't such a bit, then the result is
% |-1|. In summary possible results are decimal numbers:
% |-1|, \meta{Index}, \meta{Index} + 1, \dots, ($\infty$)
%
% \begin{declcs}{bitsetGetSetBitList} \M{BitSet}
% \end{declcs}
% Macro \cs{bitsetGetSetBitList} is an application for
% \cs{bitsetNextSetBit}. The set bits are iterated and
% returned as comma separated list of index positions
% in increasing order. The list is empty in case of
% an empty bit set.
%
% \subsubsection{Bit set properties}\label{sec:prop}
%
% \begin{declcs}{bitsetSize} \M{BitSet}
% \end{declcs}
% Macro \cs{bitsetSize} returns number of bits in use.
% It is the same as the index number of the highest set/enabled bit
% incremented by one.
%
% \begin{declcs}{bitsetCardinality} \M{BitSet}
% \end{declcs}
% Macro \cs{bitsetCardinality} counts the number of set/enabled bits.
%
% \subsubsection{Queries}\label{sec:query}
%
% Also the query procedures are expandable. They ask for a piece
% of information about a bit set and execute code depending
% on the answer.
%
% \begin{declcs}{bitsetIsDefined} \M{BitSet} \M{Then} \M{Else}
% \end{declcs}
% If the bit set with the name \meta{BitSet} exists the code
% given in \meta{Then} is executed, otherwise \meta{Else} is used.
%
% \begin{declcs}{bitsetIsEmpty} \M{BitSet} \M{Then} \M{Else}
% \end{declcs}
% If the bit set \meta{BitSet} exists and at least one bit
% is set/enabled, the code in \meta{Then} is executed,
% \meta{Else} otherwise.
%
% \begin{declcs}{bitsetEquals} \M{BitSet A} \M{BitSet B} \M{Then} \M{Else}
% \end{declcs}
% Both bit sets are equal if and only if either both are undefined
% or both are defined and represents the same bit values at the
% same positions. Thus this definition is reflexive, symmeric,
% and transitive, enough for an equivalent relation.
%
% \begin{declcs}{bitsetIntersects} \M{BitSet A} \M{BitSet B} \M{Then} \M{Else}
% \end{declcs}
% If and only if \meta{BitSet A} and \meta{BitSet B} have
% at least one bit at the same position that is set, then
% code part \meta{Then} is executed.
%
% \begin{declcs}{bitsetQuery} \M{BitSet} \M{Index} \M{Then} \M{Else}
% \end{declcs}
% It's just a wrapper for \cs{bitsetGet}. If the bit at position
% \meta{Index} is enabled, code \meta{Then} is called.
%
% \StopEventually{
% }
%
% \section{Implementation}
%
% The internal format of a bit set is quite simple, a sequence
% of digits |0| and |1|. The least significant bit is left.
% A bit set without any flag set is encoded by |0|. Also
% undefined bit sets are treated that way. After the
% highest bit that is set there are no further zeroes.
% A regular expression of valid bit sets values:
% \begin{quote}
%  |0|\string||[01]*1|
% \end{quote}
%
%    \begin{macrocode}
%<*package>
%    \end{macrocode}
%
% \subsection{Reload check and package identification}
%    Reload check, especially if the package is not used with \LaTeX.
%    \begin{macrocode}
\begingroup\catcode61\catcode48\catcode32=10\relax%
  \catcode13=5 % ^^M
  \endlinechar=13 %
  \catcode35=6 % #
  \catcode39=12 % '
  \catcode44=12 % ,
  \catcode45=12 % -
  \catcode46=12 % .
  \catcode58=12 % :
  \catcode64=11 % @
  \catcode123=1 % {
  \catcode125=2 % }
  \expandafter\let\expandafter\x\csname ver@bitset.sty\endcsname
  \ifx\x\relax % plain-TeX, first loading
  \else
    \def\empty{}%
    \ifx\x\empty % LaTeX, first loading,
      % variable is initialized, but \ProvidesPackage not yet seen
    \else
      \expandafter\ifx\csname PackageInfo\endcsname\relax
        \def\x#1#2{%
          \immediate\write-1{Package #1 Info: #2.}%
        }%
      \else
        \def\x#1#2{\PackageInfo{#1}{#2, stopped}}%
      \fi
      \x{bitset}{The package is already loaded}%
      \aftergroup\endinput
    \fi
  \fi
\endgroup%
%    \end{macrocode}
%    Package identification:
%    \begin{macrocode}
\begingroup\catcode61\catcode48\catcode32=10\relax%
  \catcode13=5 % ^^M
  \endlinechar=13 %
  \catcode35=6 % #
  \catcode39=12 % '
  \catcode40=12 % (
  \catcode41=12 % )
  \catcode44=12 % ,
  \catcode45=12 % -
  \catcode46=12 % .
  \catcode47=12 % /
  \catcode58=12 % :
  \catcode64=11 % @
  \catcode91=12 % [
  \catcode93=12 % ]
  \catcode123=1 % {
  \catcode125=2 % }
  \expandafter\ifx\csname ProvidesPackage\endcsname\relax
    \def\x#1#2#3[#4]{\endgroup
      \immediate\write-1{Package: #3 #4}%
      \xdef#1{#4}%
    }%
  \else
    \def\x#1#2[#3]{\endgroup
      #2[{#3}]%
      \ifx#1\@undefined
        \xdef#1{#3}%
      \fi
      \ifx#1\relax
        \xdef#1{#3}%
      \fi
    }%
  \fi
\expandafter\x\csname ver@bitset.sty\endcsname
\ProvidesPackage{bitset}%
  [2011/01/30 v1.1 Handle bit-vector datatype (HO)]%
%    \end{macrocode}
%
% \subsection{Catcodes}
%
%    \begin{macrocode}
\begingroup\catcode61\catcode48\catcode32=10\relax%
  \catcode13=5 % ^^M
  \endlinechar=13 %
  \catcode123=1 % {
  \catcode125=2 % }
  \catcode64=11 % @
  \def\x{\endgroup
    \expandafter\edef\csname BitSet@AtEnd\endcsname{%
      \endlinechar=\the\endlinechar\relax
      \catcode13=\the\catcode13\relax
      \catcode32=\the\catcode32\relax
      \catcode35=\the\catcode35\relax
      \catcode61=\the\catcode61\relax
      \catcode64=\the\catcode64\relax
      \catcode123=\the\catcode123\relax
      \catcode125=\the\catcode125\relax
    }%
  }%
\x\catcode61\catcode48\catcode32=10\relax%
\catcode13=5 % ^^M
\endlinechar=13 %
\catcode35=6 % #
\catcode64=11 % @
\catcode123=1 % {
\catcode125=2 % }
\def\TMP@EnsureCode#1#2{%
  \edef\BitSet@AtEnd{%
    \BitSet@AtEnd
    \catcode#1=\the\catcode#1\relax
  }%
  \catcode#1=#2\relax
}
\TMP@EnsureCode{33}{12}% !
\TMP@EnsureCode{39}{12}% '
\TMP@EnsureCode{40}{12}% (
\TMP@EnsureCode{41}{12}% )
\TMP@EnsureCode{42}{12}% *
\TMP@EnsureCode{43}{12}% +
\TMP@EnsureCode{44}{12}% ,
\TMP@EnsureCode{45}{12}% -
\TMP@EnsureCode{46}{12}% .
\TMP@EnsureCode{47}{12}% /
\TMP@EnsureCode{58}{11}% : (letter!)
\TMP@EnsureCode{60}{12}% <
\TMP@EnsureCode{62}{12}% >
\TMP@EnsureCode{63}{14}% ? (comment!)
\TMP@EnsureCode{91}{12}% [
\TMP@EnsureCode{93}{12}% ]
\TMP@EnsureCode{96}{12}% `
\edef\BitSet@AtEnd{\BitSet@AtEnd\noexpand\endinput}
\begingroup\expandafter\expandafter\expandafter\endgroup
\expandafter\ifx\csname BitSet@TestMode\endcsname\relax
\else
  \catcode63=9 % ? (ignore)
\fi
? \let\BitSet@@TestMode\BitSet@TestMode
%    \end{macrocode}
%
% \subsection{Package loading}
%
%    \begin{macrocode}
\begingroup\expandafter\expandafter\expandafter\endgroup
\expandafter\ifx\csname RequirePackage\endcsname\relax
  \def\TMP@RequirePackage#1[#2]{%
    \begingroup\expandafter\expandafter\expandafter\endgroup
    \expandafter\ifx\csname ver@#1.sty\endcsname\relax
      \input #1.sty\relax
    \fi
  }%
  \TMP@RequirePackage{infwarerr}[2007/09/09]%
  \TMP@RequirePackage{intcalc}[2007/09/27]%
  \TMP@RequirePackage{bigintcalc}[2007/09/27]%
\else
  \RequirePackage{infwarerr}[2007/09/09]%
  \RequirePackage{intcalc}[2007/09/27]%
  \RequirePackage{bigintcalc}[2007/09/27]%
\fi
%    \end{macrocode}
%
% \subsection{Help macros}
%
% \subsubsection{Number constant}
%
%    \begin{macro}{\BitSet@MaxSize}
%    \begin{macrocode}
\def\BitSet@MaxSize{2147483647}%
%    \end{macrocode}
%    \end{macro}
%
% \subsubsection{General basic macros}
%
%    \begin{macro}{\BitSet@Empty}
%    \begin{macrocode}
\def\BitSet@Empty{}
%    \end{macrocode}
%    \end{macro}
%    \begin{macro}{\BitSet@FirstOfOne}
%    \begin{macrocode}
\def\BitSet@FirstOfOne#1{#1}
%    \end{macrocode}
%    \end{macro}
%    \begin{macro}{\BitSet@Gobble}
%    \begin{macrocode}
\def\BitSet@Gobble#1{}
%    \end{macrocode}
%    \end{macro}
%    \begin{macro}{\BitSet@FirstOfTwo}
%    \begin{macrocode}
\def\BitSet@FirstOfTwo#1#2{#1}
%    \end{macrocode}
%    \end{macro}
%    \begin{macro}{\BitSet@SecondOfTwo}
%    \begin{macrocode}
\def\BitSet@SecondOfTwo#1#2{#2}
%    \end{macrocode}
%    \end{macro}
%
%    \begin{macro}{\BitSet@Space}
%    \begin{macrocode}
\def\BitSet@Space{ }
%    \end{macrocode}
%    \end{macro}
%    \begin{macro}{\BitSet@ZapSpace}
%    \begin{macrocode}
\def\BitSet@ZapSpace#1 #2{%
  #1%
  \ifx\BitSet@Empty#2%
  \else
    \expandafter\BitSet@ZapSpace
  \fi
  #2%
}
%    \end{macrocode}
%    \end{macro}
%
% \subsubsection{Tail recursion}
%
%    \begin{macro}{\BitSet@Fi}
%    \begin{macrocode}
\let\BitSet@Fi\fi
%    \end{macrocode}
%    \end{macro}
%    \begin{macro}{\BitSet@AfterFi}
%    \begin{macrocode}
\def\BitSet@AfterFi#1#2\BitSet@Fi{\fi#1}
%    \end{macrocode}
%    \end{macro}
%    \begin{macro}{\BitSet@AfterFiFi}
%    \begin{macrocode}
\def\BitSet@AfterFiFi#1#2\BitSet@Fi{\fi\fi#1}%
%    \end{macrocode}
%    \end{macro}
%    \begin{macro}{\BitSet@AfterFiFiFi}
%    \begin{macrocode}
\def\BitSet@AfterFiFiFi#1#2\BitSet@Fi{\fi\fi\fi#1}%
%    \end{macrocode}
%    \end{macro}
%
% \subsubsection{Check macros}
%
%    \begin{macro}{\BitSet@IfUndefined}
%    \begin{macrocode}
\def\BitSet@IfUndefined#1{%
  \expandafter\ifx\csname BS@#1\endcsname\relax
    \expandafter\BitSet@FirstOfTwo
  \else
    \expandafter\BitSet@SecondOfTwo
  \fi
}
%    \end{macrocode}
%    \end{macro}
%
%    \begin{macro}{\BitSet@CheckIndex}
%    |#1|: continuation code\\
%    |#2|: BitSet\\
%    |#3|: Index
%    \begin{macrocode}
\def\BitSet@CheckIndex#1#2#3{%
  \BitSet@IfUndefined{#2}{\bitsetReset{#2}}{}%
  \expandafter\expandafter\expandafter\BitSet@@CheckIndex
  \intcalcNum{#3}!%
  {#2}{#1}%
}
%    \end{macrocode}
%    \end{macro}
%    \begin{macro}{\BitSet@@CheckIndex}
%    |#1|: plain Index\\
%    |#2|: BitSet\\
%    |#3|: continuation code
%    \begin{macrocode}
\def\BitSet@@CheckIndex#1!#2#3{%
  \ifnum#1<0 %
    \BitSet@AfterFi{%
      \@PackageError{bitset}{%
        Invalid negative index (#1)%
      }\@ehc
    }%
  \else
    \BitSet@AfterFi{%
      #3{#2}{#1}%
    }%
  \BitSet@Fi
}
%    \end{macrocode}
%    \end{macro}
%
% \subsection{Miscellaneous}\label{src:misc}
%
%    \begin{macro}{\bitsetReset}
%    \begin{macrocode}
\def\bitsetReset#1{%
  \expandafter\def\csname BS@#1\endcsname{0}%
}
%    \end{macrocode}
%    \end{macro}
%
%    \begin{macro}{\bitsetLet}
%    \begin{macrocode}
\def\bitsetLet#1#2{%
  \BitSet@IfUndefined{#2}{%
    \bitsetReset{#1}%
  }{%
    \expandafter\let\csname BS@#1\expandafter\endcsname
                    \csname BS@#2\endcsname
  }%
}
%    \end{macrocode}
%    \end{macro}
%
% \subsection{Import}\label{src:import}
%
% \subsubsection{From binary number}
%
%    \begin{macro}{\bitsetSetBin}
%    \begin{macrocode}
\def\bitsetSetBin#1#2{%
  \edef\BitSet@Temp{#2}%
  \edef\BitSet@Temp{%
    \expandafter\expandafter\expandafter\BitSet@ZapSpace
    \expandafter\BitSet@Temp\BitSet@Space\BitSet@Empty
  }%
  \edef\BitSet@Temp{%
    \expandafter\BitSet@KillZeros\BitSet@Temp\BitSet@Empty
  }%
  \ifx\BitSet@Temp\BitSet@Empty
    \expandafter\let\csname BS@#1\endcsname\BitSet@Zero
  \else
    \expandafter\edef\csname BS@#1\endcsname{%
      \expandafter\BitSet@Reverse\BitSet@Temp!%
    }%
  \fi
}
%    \end{macrocode}
%    \end{macro}
%    \begin{macro}{\BitSet@KillZeros}
%    \begin{macrocode}
\def\BitSet@KillZeros#1{%
  \ifx#10%
    \expandafter\BitSet@KillZeros
  \else
    #1%
  \fi
}
%    \end{macrocode}
%    \end{macro}
%    \begin{macro}{\BitSet@Reverse}
%    \begin{macrocode}
\def\BitSet@Reverse#1#2!{%
  \ifx\\#2\\%
    #1%
  \else
    \BitSet@AfterFi{%
      \BitSet@Reverse#2!#1%
    }%
  \BitSet@Fi
}
%    \end{macrocode}
%    \end{macro}
%
% \subsubsection{From octal/hex number}
%
%    \begin{macro}{\bitsetSetOct}
%    \begin{macrocode}
\def\bitsetSetOct{%
  \BitSet@SetOctHex\BitSet@FromFirstOct
}
%    \end{macrocode}
%    \end{macro}
%    \begin{macro}{\bitsetSetHex}
%    \begin{macrocode}
\def\bitsetSetHex{%
  \BitSet@SetOctHex\BitSet@FromFirstHex
}
%    \end{macrocode}
%    \end{macro}
%    \begin{macro}{\BitSet@SetOctHex}
%    \begin{macrocode}
\def\BitSet@SetOctHex#1#2#3{%
  \edef\BitSet@Temp{#3}%
  \edef\BitSet@Temp{%
    \expandafter\expandafter\expandafter\BitSet@ZapSpace
    \expandafter\BitSet@Temp\BitSet@Space\BitSet@Empty
  }%
  \edef\BitSet@Temp{%
    \expandafter\BitSet@KillZeros\BitSet@Temp\BitSet@Empty
  }%
  \ifx\BitSet@Temp\BitSet@Empty
    \expandafter\let\csname BS@#2\endcsname\BitSet@Zero
  \else
    \edef\BitSet@Temp{%
      \expandafter#1\BitSet@Temp!%
    }%
    \ifx\BitSet@Temp\BitSet@Empty
      \expandafter\let\csname BS@#2\endcsname\BitSet@Zero
    \else
      \expandafter\edef\csname BS@#2\endcsname{%
        \expandafter\BitSet@Reverse\BitSet@Temp!%
      }%
    \fi
  \fi
}
%    \end{macrocode}
%    \end{macro}
%    \begin{macro}{\BitSet@FromFirstOct}
%    \begin{macrocode}
\def\BitSet@FromFirstOct#1{%
  \ifx#1!%
  \else
    \ifcase#1 \BitSet@AfterFiFi\BitSet@FromFirstOct
    \or 1%
    \or 10%
    \or 11%
    \or 100%
    \or 101%
    \or 110%
    \or 111%
    \else \BitSetError:WrongOctalDigit%
    \fi
    \expandafter\BitSet@FromOct
  \BitSet@Fi
}
%    \end{macrocode}
%    \end{macro}
%    \begin{macro}{\BitSet@FromOct}
%    \begin{macrocode}
\def\BitSet@FromOct#1{%
  \ifx#1!%
  \else
    \ifcase#1 000%
    \or 001%
    \or 010%
    \or 011%
    \or 100%
    \or 101%
    \or 110%
    \or 111%
    \else \BitSetError:WrongOctalDigit%
    \fi
    \expandafter\BitSet@FromOct
  \fi
}
%    \end{macrocode}
%    \end{macro}
%
%    \begin{macro}{\BitSet@FromFirstHex}
%    \begin{macrocode}
\def\BitSet@FromFirstHex#1{%
  \ifx#1!%
  \else
    \ifx#10%
      \BitSet@AfterFiFi\BitSet@FromFirstHex
    \fi
    \expandafter\ifx\csname BitSet@Hex#1\endcsname\relax
      \BitSetError:InvalidHexDigit%
    \else
      \expandafter\expandafter\expandafter\BitSet@KillZeros
      \csname BitSet@Hex#1\endcsname
    \fi
    \expandafter\BitSet@FromHex
  \BitSet@Fi
}
%    \end{macrocode}
%    \end{macro}
%    \begin{macro}{\BitSet@FromHex}
%    \begin{macrocode}
\def\BitSet@FromHex#1{%
  \ifx#1!%
  \else
    \expandafter\ifx\csname BitSet@Hex#1\endcsname\relax
      \BitSetError:InvalidHexDigit%
    \else
      \csname BitSet@Hex#1\endcsname
    \fi
    \expandafter\BitSet@FromHex
  \fi
}
%    \end{macrocode}
%    \end{macro}
%    \begin{macro}{\BitSet@Hex[0..F]}
%    \begin{macrocode}
\def\BitSet@Temp#1{%
  \expandafter\def\csname BitSet@Hex#1\endcsname
}
\BitSet@Temp 0{0000}%
\BitSet@Temp 1{0001}%
\BitSet@Temp 2{0010}%
\BitSet@Temp 3{0011}%
\BitSet@Temp 4{0100}%
\BitSet@Temp 5{0101}%
\BitSet@Temp 6{0110}%
\BitSet@Temp 7{0111}%
\BitSet@Temp 8{1000}%
\BitSet@Temp 9{1001}%
\BitSet@Temp A{1010}%
\BitSet@Temp B{1011}%
\BitSet@Temp C{1100}%
\BitSet@Temp D{1101}%
\BitSet@Temp E{1110}%
\BitSet@Temp F{1111}%
\BitSet@Temp a{1010}%
\BitSet@Temp b{1011}%
\BitSet@Temp c{1100}%
\BitSet@Temp d{1101}%
\BitSet@Temp e{1110}%
\BitSet@Temp f{1111}%
%    \end{macrocode}
%    \end{macro}
%
% \subsubsection{From decimal number}
%
%    \begin{macro}{\bitsetSetDec}
%    \begin{macrocode}
\def\bitsetSetDec#1#2{%
  \edef\BitSet@Temp{#2}%
  \edef\BitSet@Temp{%
    \expandafter\expandafter\expandafter\BitSet@ZapSpace
    \expandafter\BitSet@Temp\BitSet@Space\BitSet@Empty
  }%
  \edef\BitSet@Temp{%
    \expandafter\BitSet@KillZeros\BitSet@Temp\BitSet@Empty
  }%
  \ifx\BitSet@Temp\BitSet@Empty
    \expandafter\let\csname BS@#1\endcsname\BitSet@Zero
  \else
    \ifcase\bigintcalcSgn{\BitSet@Temp} %
      \expandafter\let\csname BS@#1\endcsname\BitSet@Zero
    \or
      \ifnum\bigintcalcCmp\BitSet@Temp\BitSet@MaxSize>0 %
        \expandafter\edef\csname BS@#1\endcsname{%
          \expandafter\BitSet@SetDecBig\BitSet@Temp!%
        }%
      \else
        \expandafter\edef\csname BS@#1\endcsname{%
          \expandafter\BitSet@SetDec\BitSet@Temp!%
        }%
      \fi
    \else
      \@PackageError{bitset}{%
        Bit sets cannot be negative%
      }\@ehc
    \fi
  \fi
}
%    \end{macrocode}
%    \end{macro}
%    \begin{macro}{\BitSet@SetDecBig}
%    \begin{macrocode}
\def\BitSet@SetDecBig#1#2#3#4#5#6#7#8#9!{%
  \ifx\\#9\\%
    \BitSet@SetDec#1#2#3#4#5#6#7#8!%
  \else
    \ifcase\BigIntCalcOdd#1#2#4#5#6#7#8#9! %
      0%
    \or
      1%
?   \else\BitSetError:ThisCannotHappen%
    \fi
    \BitSet@AfterFi{%
      \expandafter\expandafter\expandafter\BitSet@SetDecBig
      \BigIntCalcShr#1#2#3#4#5#6#7#8#9!!%
    }%
  \BitSet@Fi
}
%    \end{macrocode}
%    \end{macro}
%    \begin{macro}{\BitSet@SetDec}
%    \begin{macrocode}
\def\BitSet@SetDec#1!{%
  \ifcase#1 %
  \or 1%
  \else
    \ifodd#1 %
      1%
    \else
      0%
    \fi
    \BitSet@AfterFi{%
      \expandafter\expandafter\expandafter\BitSet@SetDec
      \IntCalcShr#1!!%
    }%
  \BitSet@Fi
}
%    \end{macrocode}
%    \end{macro}
%
% \subsection{Export}\label{src:export}
%
% \subsubsection{To binary number}
%
%    \begin{macro}{\bitsetGetBin}
%    \begin{macrocode}
\def\bitsetGetBin#1#2{%
  \romannumeral0%
  \expandafter\expandafter\expandafter\BitSet@@GetBin
  \intcalcNum{#2}!{#1}%
}
%    \end{macrocode}
%    \end{macro}
%    \begin{macro}{\BitSet@@GetBin}
%    \begin{macrocode}
\def\BitSet@@GetBin#1!#2{%
  \BitSet@IfUndefined{#2}{%
    \ifnum#1>1 %
      \BitSet@AfterFi{%
        \expandafter\expandafter\expandafter\BitSet@Fill
        \IntCalcDec#1!!0%
      }%
    \else
      \BitSet@AfterFi{ 0}%
    \BitSet@Fi
  }{%
    \expandafter\expandafter\expandafter\BitSet@NumBinRev
    \expandafter\expandafter\expandafter1%
    \expandafter\expandafter\expandafter!%
    \csname BS@#2\endcsname!!#1!%
  }%
}
%    \end{macrocode}
%    \end{macro}
%    \begin{macro}{\BitSet@Fill}
%    |#1|: number of leading digits |0|\\
%    |#2|: result
%    \begin{macrocode}
\def\BitSet@Fill#1!{%
  \ifnum#1>0 %
    \BitSet@AfterFi{%
      \expandafter\expandafter\expandafter\BitSet@Fill
      \IntCalcDec#1!!0%
    }%
  \else
    \BitSet@AfterFi{ }%
  \BitSet@Fi
}
%    \end{macrocode}
%    \end{macro}
%
%    \begin{macro}{\BitSet@NumBinRev}
%    |#1|: bit counter (including |#2|)\\
%    |#2#3|: reverted number\\
%    |#4|: result\\
%    |#5|: min size
%    \begin{macrocode}
\def\BitSet@NumBinRev#1!#2#3!{%
  \ifx\\#3\\%
    \BitSet@AfterFi{%
      \BitSet@NumBinFill#1!#2%
    }%
  \else
    \BitSet@AfterFi{%
      \expandafter\expandafter\expandafter\BitSet@NumBinRev
      \IntCalcInc#1!!#3!#2%
    }%
  \BitSet@Fi
}
%    \end{macrocode}
%    \end{macro}
%    \begin{macro}{\BitSet@NumBinFill}
%    \begin{macrocode}
\def\BitSet@NumBinFill#1!#2!#3!{%
  \ifnum#3>#1 %
    \BitSet@AfterFi{%
      \expandafter\expandafter\expandafter\BitSet@Fill
      \IntCalcSub#3!#1!!#2%
    }%
  \else
    \BitSet@AfterFi{ #2}%
  \BitSet@Fi
}
%    \end{macrocode}
%    \end{macro}
%
% \subsubsection{To octal/hexadecimal number}
%
%    \begin{macro}{\bitsetGetOct}
%    \begin{macrocode}
\def\bitsetGetOct#1#2{%
  \romannumeral0%
  \bitsetIsEmpty{#1}{%
    \expandafter\expandafter\expandafter\BitSet@@GetOctHex
    \intcalcNum{#2}!3!230%
  }{%
    \expandafter\expandafter\expandafter\BitSet@@GetOct
    \expandafter\expandafter\expandafter1%
    \expandafter\expandafter\expandafter!%
    \expandafter\expandafter\expandafter!%
    \csname BS@#1\endcsname00%
    \BitSet@Empty\BitSet@Empty\BitSet@Empty!{#2}%
  }%
}
%    \end{macrocode}
%    \end{macro}
%    \begin{macro}{\bitsetGetHex}
%    \begin{macrocode}
\def\bitsetGetHex#1#2{%
  \romannumeral0%
  \bitsetIsEmpty{#1}{%
    \expandafter\expandafter\expandafter\BitSet@@GetOctHex
    \intcalcNum{#2}!4!340%
  }{%
    \expandafter\expandafter\expandafter\BitSet@@GetHex
    \expandafter\expandafter\expandafter1%
    \expandafter\expandafter\expandafter!%
    \expandafter\expandafter\expandafter!%
    \csname BS@#1\endcsname000%
    \BitSet@Empty\BitSet@Empty\BitSet@Empty\BitSet@Empty!{#2}%
  }%
}
%    \end{macrocode}
%    \end{macro}
%    \begin{macro}{\BitSet@@GetOct}
%    |#1|: number of digits\\
%    |#2|: result\\
%    |#3#4#5|: bits
%    \begin{macrocode}
\def\BitSet@@GetOct#1!#2!#3#4#5{%
  \ifx#5\BitSet@Empty
    \BitSet@AfterFi{%
      \expandafter\expandafter\expandafter\BitSet@GetOctHex
      \IntCalcDec#1!!#2!23%
    }%
  \else
    \BitSet@AfterFi{%
      \expandafter\expandafter\expandafter\BitSet@@GetOct
      \number\IntCalcInc#1!\expandafter\expandafter\expandafter!%
      \csname BitSet@Oct#5#4#3\endcsname#2!%
    }%
  \BitSet@Fi
}
%    \end{macrocode}
%    \end{macro}
%    \begin{macro}{\BitSet@Oct[000..111]}
%    \begin{macrocode}
\def\BitSet@Temp#1#2#3#4{%
  \expandafter\def\csname BitSet@Oct#1#2#3\endcsname{#4}%
}
\BitSet@Temp0000%
\BitSet@Temp0011%
\BitSet@Temp0102%
\BitSet@Temp0113%
\BitSet@Temp1004%
\BitSet@Temp1015%
\BitSet@Temp1106%
\BitSet@Temp1117%
%    \end{macrocode}
%    \end{macro}
%    \begin{macro}{\BitSet@@GetHex}
%    |#1|: number of digits\\
%    |#2|: result\\
%    |#3#4#5#6|: bits
%    \begin{macrocode}
\def\BitSet@@GetHex#1!#2!#3#4#5#6{%
  \ifx#6\BitSet@Empty
    \BitSet@AfterFi{%
      \expandafter\expandafter\expandafter\BitSet@GetOctHex
      \IntCalcDec#1!!#2!34%
    }%
  \else
    \BitSet@AfterFi{%
      \expandafter\expandafter\expandafter\BitSet@@GetHex
      \number\IntCalcInc#1!\expandafter\expandafter\expandafter!%
      \csname BitSet@Hex#6#5#4#3\endcsname#2!%
    }%
  \BitSet@Fi
}
%    \end{macrocode}
%    \end{macro}
%    \begin{macro}{\BitSet@Hex[0000..1111]}
%    \begin{macrocode}
\def\BitSet@Temp#1#2#3#4#5{%
  \expandafter\def\csname BitSet@Hex#1#2#3#4\endcsname{#5}%
}
\BitSet@Temp00000%
\BitSet@Temp00011%
\BitSet@Temp00102%
\BitSet@Temp00113%
\BitSet@Temp01004%
\BitSet@Temp01015%
\BitSet@Temp01106%
\BitSet@Temp01117%
\BitSet@Temp10008%
\BitSet@Temp10019%
\BitSet@Temp1010A%
\BitSet@Temp1011B%
\BitSet@Temp1100C%
\BitSet@Temp1101D%
\BitSet@Temp1110E%
\BitSet@Temp1111F%
%    \end{macrocode}
%    \end{macro}
%    \begin{macro}{\BitSet@GetOctHex}
%    Leading zeros $ (|#4| - |#1| * 3 + 2)/3 $ if $ |#4| > |#1| * 3$\\
%    |#1|: digit size\\
%    |#2|: result\\
%    |#3|: bits per digit - 1\\
%    |#4|: bits per digit
%    |#5|: garbage\\
%    |#6|: min size
%    \begin{macrocode}
\def\BitSet@GetOctHex#1!#2!#3#4#5!#6{%
  \expandafter\BitSet@@GetOctHex
  \number\intcalcNum{#6}\expandafter\expandafter\expandafter!%
  \IntCalcMul#1!#4!!#3#4#2%
}
%    \end{macrocode}
%    \end{macro}
%    \begin{macro}{\BitSet@@GetOctHex}
%    |#1|: plain min size\\
%    |#2|: digits * (bits per digit)\\
%    |#3|: bits per digit - 1\\
%    |#4|: bits per digit
%    \begin{macrocode}
\def\BitSet@@GetOctHex#1!#2!#3#4{%
  \ifnum#1>#2 %
    \BitSet@AfterFi{%
      \expandafter\expandafter\expandafter\expandafter
      \expandafter\expandafter\expandafter\BitSet@Fill
      \expandafter\IntCalcDiv\number
      \expandafter\expandafter\expandafter\IntCalcAdd
      \IntCalcSub#1!#2!!#3!!#4!!%
    }%
  \else
    \BitSet@AfterFi{ }%
  \BitSet@Fi
}
%    \end{macrocode}
%    \end{macro}
%
% \subsubsection{To decimal number}
%
%    \begin{macro}{\bitsetGetDec}
%    \begin{macrocode}
\def\bitsetGetDec#1{%
  \romannumeral0%
  \BitSet@IfUndefined{#1}{ 0}{%
    \expandafter\expandafter\expandafter\BitSet@GetDec
    \csname BS@#1\endcsname!%
  }%
}
%    \end{macrocode}
%    \end{macro}
%    \begin{macro}{\BitSet@GetDec}
%    \begin{macrocode}
\def\BitSet@GetDec#1#2!{%
  \ifx\\#2\\%
    \BitSet@AfterFi{ #1}%
  \else
    \BitSet@AfterFi{%
      \BitSet@@GetDec2!#1!#2!%
    }%
  \BitSet@Fi
}
%    \end{macrocode}
%    \end{macro}
%    \begin{macro}{\BitSet@@GetDec}
%    |#1|: power of two\\
%    |#2|: result\\
%    |#3#4|: number
%    \begin{macrocode}
\def\BitSet@@GetDec#1!#2!#3#4!{%
  \ifx\\#4\\%
    \ifx#31%
      \BitSet@AfterFiFi{%
        \expandafter\expandafter\expandafter\BitSet@Space
        \IntCalcAdd#1!#2!%
      }%
    \else
      \BitSet@AfterFiFi{ #2}%
    \fi
  \else
    \ifx#31%
      \BitSet@AfterFiFi{%
        \csname BitSet@N#1%
        \expandafter\expandafter\expandafter\endcsname
        \IntCalcAdd#1!#2!!#4!%
      }%
    \else
      \BitSet@AfterFiFi{%
        \csname BitSet@N#1\endcsname#2!#4!%
      }%
    \fi
  \BitSet@Fi
}
%    \end{macrocode}
%    \end{macro}
%    \begin{macro}{\BitSet@N[1,2,4,...]}
%    \begin{macrocode}
\def\BitSet@Temp#1#2{%
  \expandafter\def\csname BitSet@N#1\endcsname{%
    \BitSet@@GetDec#2!%
  }%
}
\BitSet@Temp{1}{2}
\BitSet@Temp{2}{4}
\BitSet@Temp{4}{8}
\BitSet@Temp{8}{16}
\BitSet@Temp{16}{32}
\BitSet@Temp{32}{64}
\BitSet@Temp{64}{128}
\BitSet@Temp{128}{256}
\BitSet@Temp{256}{512}
\BitSet@Temp{512}{1024}
\BitSet@Temp{1024}{2048}
\BitSet@Temp{2048}{4096}
\BitSet@Temp{4096}{8192}
\BitSet@Temp{8192}{16384}
\BitSet@Temp{16384}{32768}
\BitSet@Temp{32768}{65536}
\BitSet@Temp{65536}{131072}
\BitSet@Temp{131072}{262144}
\BitSet@Temp{262144}{524288}
\BitSet@Temp{524288}{1048576}
\BitSet@Temp{1048576}{2097152}
\BitSet@Temp{2097152}{4194304}
\BitSet@Temp{4194304}{8388608}
\BitSet@Temp{8388608}{16777216}
\BitSet@Temp{16777216}{33554432}
\BitSet@Temp{33554432}{67108864}
\BitSet@Temp{67108864}{134217728}
\BitSet@Temp{134217728}{268435456}
\BitSet@Temp{268435456}{536870912}
\BitSet@Temp{536870912}{1073741824}
%    \end{macrocode}
%    \end{macro}
%    \begin{macro}{\BitSet@N1073741824}
%    \begin{macrocode}
\expandafter\def\csname BitSet@N1073741824\endcsname{%
  \BitSet@GetDecBig2147483648!%
}%
%    \end{macrocode}
%    \end{macro}
%
%    \begin{macro}{\BitSet@GetDecBig}
%    |#1|: current power of two\\
%    |#2|: result\\
%    |#3#4|: number
%    \begin{macrocode}
\def\BitSet@GetDecBig#1!#2!#3#4!{%
  \ifx\\#4\\%
    \ifx#31%
      \BitSet@AfterFiFi{%
        \expandafter\expandafter\expandafter\BitSet@Space
        \BigIntCalcAdd#1!#2!%
      }%
    \else
      \BitSet@AfterFiFi{ #2}%
    \fi
  \else
    \ifx#31%
      \BitSet@AfterFiFi{%
        \expandafter\expandafter\expandafter\BitSet@@GetDecBig
        \BigIntCalcAdd#1!#2!!#1!#4!%
      }%
    \else
      \BitSet@AfterFiFi{%
        \expandafter\expandafter\expandafter\BitSet@GetDecBig
        \BigIntCalcShl#1!!#2!#4!%
      }%
    \fi
  \BitSet@Fi
}
%    \end{macrocode}
%    \end{macro}
%    \begin{macro}{\BitSet@@GetDecBig}
%    |#1|: result\\
%    |#2|: power of two\\
%    |#3#4|: number
%    \begin{macrocode}
\def\BitSet@@GetDecBig#1!#2!{%
  \expandafter\expandafter\expandafter\BitSet@GetDecBig
  \BigIntCalcShl#2!!#1!%
}
%    \end{macrocode}
%    \end{macro}
%
% \subsection{Logical operators}\label{src:log}
%
% \subsubsection{\cs{bitsetAnd}}
%
%    \begin{macro}{\bitsetAnd}
%    Decision table for \cs{bitsetAnd}:
%    \begin{quote}
%    \begin{tabular}{l|l|l|l|}
%                         & undef(B)   & empty(B)   & cardinality(B)$>$0\\
%      \hline
%      undef(A)           & A := empty & A := empty & A := empty\\
%      \hline
%      empty(A)           &            &            &\\
%      \hline
%      cardinality(A)$>$0 & A := empty & A := empty & A \&= B\\
%      \hline
%    \end{tabular}
%    \end{quote}
%    \begin{macrocode}
\def\bitsetAnd#1#2{%
  \bitsetIsEmpty{#1}{%
    \bitsetReset{#1}%
  }{%
    \bitsetIsEmpty{#2}{%
      \bitsetReset{#1}%
    }{%
      \expandafter\edef\csname BS@#1\endcsname{%
        \expandafter\expandafter\expandafter\BitSet@And
        \csname BS@#1\expandafter\expandafter\expandafter\endcsname
        \expandafter\expandafter\expandafter!%
        \csname BS@#2\endcsname!!%
      }%
      \expandafter\ifx\csname BS@#1\endcsname\BitSet@Empty
        \bitsetReset{#1}%
      \fi
    }%
  }%
}
%    \end{macrocode}
%    \end{macro}
%    \begin{macro}{\BitSet@And}
%    \begin{macrocode}
\def\BitSet@And#1#2!#3#4!#5!{%
  \ifx\\#2\\%
    \ifnum#1#3=11 #51\fi
  \else
    \ifx\\#4\\%
      \ifnum#1#3=11 #51\fi
    \else
      \ifnum#1#3=11 %
        #51%
        \BitSet@AfterFiFiFi{%
          \BitSet@And#2!#4!!%
        }%
      \else
        \BitSet@AfterFiFiFi{%
          \BitSet@And#2!#4!#50!%
        }%
      \fi
    \fi
  \BitSet@Fi
}
%    \end{macrocode}
%    \end{macro}
%
% \subsubsection{\cs{bitsetAndNot}}
%
%    \begin{macro}{\bitsetAndNot}
%    Decision table for \cs{bitsetAndNot}:
%    \begin{quote}
%    \begin{tabular}{l|l|l|l|}
%                         & undef(B)   & empty(B)   & cardinality(B)$>$0\\
%      \hline
%      undef(A)           & A := empty & A := empty & A := empty\\
%      \hline
%      empty(A)           &            &            &\\
%      \hline
%      cardinality(A)$>$0 &            &            & A \&= !B\\
%      \hline
%    \end{tabular}
%    \end{quote}
%    \begin{macrocode}
\def\bitsetAndNot#1#2{%
  \bitsetIsEmpty{#1}{%
    \bitsetReset{#1}%
  }{%
    \bitsetIsEmpty{#2}{%
    }{%
      \expandafter\edef\csname BS@#1\endcsname{%
        \expandafter\expandafter\expandafter\BitSet@AndNot
        \csname BS@#1\expandafter\expandafter\expandafter\endcsname
        \expandafter\expandafter\expandafter!%
        \csname BS@#2\endcsname!!%
      }%
      \expandafter\ifx\csname BS@#1\endcsname\BitSet@Empty
        \bitsetReset{#1}%
      \fi
    }%
  }%
}
%    \end{macrocode}
%    \end{macro}
%    \begin{macro}{\BitSet@AndNot}
%    \begin{macrocode}
\def\BitSet@AndNot#1#2!#3#4!#5!{%
  \ifx\\#2\\%
    \ifnum#1#3=10 #51\fi
  \else
    \ifx\\#4\\%
      #5%
      \ifnum#1#3=10 1\else 0\fi
      #2%
    \else
      \ifnum#1#3=10 %
        #51%
        \BitSet@AfterFiFiFi{%
          \BitSet@AndNot#2!#4!!%
        }%
      \else
        \BitSet@AfterFiFiFi{%
          \BitSet@AndNot#2!#4!#50!%
        }%
      \fi
    \fi
  \BitSet@Fi
}
%    \end{macrocode}
%    \end{macro}
%
% \subsubsection{\cs{bitsetOr}}
%
%    \begin{macro}{\bitsetOr}
%    Decision table for \cs{bitsetOr}:
%    \begin{quote}
%    \begin{tabular}{l|l|l|l|}
%                         & undef(B)   & empty(B)   & cardinality(B)$>$0\\
%      \hline
%      undef(A)           & A := empty & A := empty & A := B\\
%      \hline
%      empty(A)           &            &            & A := B\\
%      \hline
%      cardinality(A)$>$0 &            &            & A \string|= B\\
%      \hline
%    \end{tabular}
%    \end{quote}
%    \begin{macrocode}
\def\bitsetOr#1#2{%
  \bitsetIsEmpty{#2}{%
    \BitSet@IfUndefined{#1}{\bitsetReset{#1}}{}%
  }{%
    \bitsetIsEmpty{#1}{%
      \expandafter\let\csname BS@#1\expandafter\endcsname
                      \csname BS@#2\endcsname
    }{%
      \expandafter\edef\csname BS@#1\endcsname{%
        \expandafter\expandafter\expandafter\BitSet@Or
        \csname BS@#1\expandafter\expandafter\expandafter\endcsname
        \expandafter\expandafter\expandafter!%
        \csname BS@#2\endcsname!%
      }%
    }%
  }%
}
%    \end{macrocode}
%    \end{macro}
%    \begin{macro}{\BitSet@Or}
%    \begin{macrocode}
\def\BitSet@Or#1#2!#3#4!{%
  \ifnum#1#3>0 1\else 0\fi
  \ifx\\#2\\%
    #4%
  \else
    \ifx\\#4\\%
      #2%
    \else
      \BitSet@AfterFiFi{%
        \BitSet@Or#2!#4!%
      }%
    \fi
  \BitSet@Fi
}
%    \end{macrocode}
%    \end{macro}
%
% \subsubsection{\cs{bitsetXor}}
%
%    \begin{macro}{\bitsetXor}
%    Decision table for \cs{bitsetXor}:
%    \begin{quote}
%    \begin{tabular}{l|l|l|l|}
%                         & undef(B)   & empty(B)   & cardinality(B)$>$0\\
%      \hline
%      undef(A)           & A := empty & A := empty & A := B\\
%      \hline
%      empty(A)           &            &            & A := B\\
%      \hline
%      cardinality(A)$>$0 &            &            & A \string^= B\\
%      \hline
%    \end{tabular}
%    \end{quote}
%    \begin{macrocode}
\def\bitsetXor#1#2{%
  \bitsetIsEmpty{#2}{%
    \BitSet@IfUndefined{#1}{\bitsetReset{#1}}{}%
  }{%
    \bitsetIsEmpty{#1}{%
      \expandafter\let\csname BS@#1\expandafter\endcsname
                      \csname BS@#2\endcsname
    }{%
      \expandafter\edef\csname BS@#1\endcsname{%
        \expandafter\expandafter\expandafter\BitSet@Xor
        \csname BS@#1\expandafter\expandafter\expandafter\endcsname
        \expandafter\expandafter\expandafter!%
        \csname BS@#2\endcsname!!%
      }%
      \expandafter\ifx\csname BS@#1\endcsname\BitSet@Empty
        \bitsetReset{#1}%
      \fi
    }%
  }%
}
%    \end{macrocode}
%    \end{macro}
%    \begin{macro}{\BitSet@Xor}
%    \begin{macrocode}
\def\BitSet@Xor#1#2!#3#4!#5!{%
  \ifx\\#2\\%
    \ifx#1#3%
      \ifx\\#4\\%
      \else
        #50#4%
      \fi
    \else
      #51#4%
    \fi
  \else
    \ifx\\#4\\%
      #5%
      \ifx#1#30\else 1\fi
      #2%
    \else
      \ifx#1#3%
        \BitSet@AfterFiFiFi{%
          \BitSet@Xor#2!#4!#50!%
        }%
      \else
        #51%
        \BitSet@AfterFiFiFi{%
          \BitSet@Xor#2!#4!!%
        }%
      \fi
    \fi
  \BitSet@Fi
}
%    \end{macrocode}
%    \end{macro}
%
% \subsubsection{Shifting}\label{src:shift}
%
% \subsubsection{\cs{bitsetShiftLeft}}
%
%    \begin{macro}{\bitsetShiftLeft}
%    \begin{macrocode}
\def\bitsetShiftLeft#1#2{%
  \BitSet@IfUndefined{#1}{%
    \bitsetReset{#1}%
  }{%
    \bitsetIsEmpty{#1}{%
    }{%
      \expandafter\expandafter\expandafter\BitSet@ShiftLeft
      \intcalcNum{#2}!{#1}%
    }%
  }%
}
%    \end{macrocode}
%    \end{macro}
%    \begin{macro}{\BitSet@ShiftLeft}
%    \begin{macrocode}
\def\BitSet@ShiftLeft#1!#2{%
  \ifcase\intcalcSgn{#1} %
  \or
    \begingroup
      \uccode`m=`0 %
    \uppercase\expandafter{\expandafter\endgroup
      \expandafter\edef\csname BS@#2\expandafter\endcsname
      \expandafter{%
        \romannumeral#1000\expandafter\BitSet@Space
        \csname BS@#2\endcsname
      }%
    }%
  \else
    \expandafter\BitSet@ShiftRight\BitSet@Gobble#1!{#2}%
  \fi
}
%    \end{macrocode}
%    \end{macro}
%
% \subsubsection{\cs{bitsetShiftRight}}
%
%    \begin{macro}{\bitsetShiftRight}
%    \begin{macrocode}
\def\bitsetShiftRight#1#2{%
  \BitSet@IfUndefined{#1}{%
    \bitsetReset{#1}%
  }{%
    \bitsetIsEmpty{#1}{%
    }{%
      \expandafter\expandafter\expandafter\BitSet@ShiftRight
      \intcalcNum{#2}!{#1}%
    }%
  }%
}
%    \end{macrocode}
%    \end{macro}
%    \begin{macro}{\BitSet@ShiftRight}
%    \begin{macrocode}
\def\BitSet@ShiftRight#1!#2{%
  \ifcase\intcalcSgn{#1} %
  \or
    \expandafter\edef\csname BS@#2\endcsname{%
      \expandafter\expandafter\expandafter\BitSet@Kill
      \csname BS@#2\expandafter\endcsname\expandafter\BitSet@Empty
      \expandafter=%
      \expandafter{\expandafter}\expandafter{\expandafter}%
      \romannumeral#1000!%
    }%
  \else
    \expandafter\BitSet@ShiftLeft\BitSet@Gobble#1!{#2}%
  \fi
}
%    \end{macrocode}
%    \end{macro}
%    \begin{macro}{\BitSet@Kill}
%    \begin{macrocode}
\def\BitSet@Kill#1#2=#3#4#5{%
  #3#4%
  \ifx#5!%
    \ifx#1\BitSet@Empty
      0%
    \else
      #1#2%
    \fi
  \else
    \ifx#1\BitSet@Empty
      0%
      \BitSet@AfterFiFi\BitSet@Cleanup
    \else
      \BitSet@Kill#2=%
    \fi
  \BitSet@Fi
}
%    \end{macrocode}
%    \end{macro}
%
% \subsection{Bit manipulation}\label{src:bitman}
%
%    \begin{macro}{\bitsetClear}
%    \begin{macrocode}
\def\bitsetClear{%
  \BitSet@CheckIndex\BitSet@Clear
}
%    \end{macrocode}
%    \end{macro}
%    \begin{macro}{\bitsetSet}
%    \begin{macrocode}
\def\bitsetSet{%
  \BitSet@CheckIndex\BitSet@Set
}
%    \end{macrocode}
%    \end{macro}
%    \begin{macro}{\bitsetFlip}
%    \begin{macrocode}
\def\bitsetFlip{%
  \BitSet@CheckIndex\BitSet@Flip
}
%    \end{macrocode}
%    \end{macro}
%    \begin{macro}{\bitsetSetValue}
%    \begin{macrocode}
\def\bitsetSetValue#1#2#3{%
  \expandafter\expandafter\expandafter\BitSet@SetValue
  \intcalcNum{#3}!{#1}{#2}%
}
%    \end{macrocode}
%    \end{macro}
%    \begin{macro}{\BitSet@SetValue}
%    |#1|: plain value\\
%    |#2|: BitSet\\
%    |#3|: Index
%    \begin{macrocode}
\def\BitSet@SetValue#1!{%
  \BitSet@CheckIndex{%
    \ifcase#1 %
      \expandafter\BitSet@Clear
    \or
      \expandafter\BitSet@Set
    \else
      \BitSet@ErrorInvalidBitValue{#1}%
      \expandafter\expandafter\expandafter\BitSet@Gobble
      \expandafter\BitSet@Gobble
    \fi
  }%
}
%    \end{macrocode}
%    \end{macro}
%    \begin{macro}{\BitSet@ErrorInvalidBitValue}
%    |#1|: Wrong bit value
%    \begin{macrocode}
\def\BitSet@ErrorInvalidBitValue#1{%
  \@PackageError{bitset}{%
    Invalid bit value (#1) not in range 0..1%
  }\@ehc
}
%    \end{macrocode}
%    \end{macro}
%
% \subsubsection{Clear operation}
%
%    \begin{macro}{\BitSet@Clear}
%    |#1|: BitSet\\
%    |#2|: plain and checked index
%    \begin{macrocode}
\def\BitSet@Clear#1#2{%
  \edef\BitSet@Temp{%
    \expandafter\expandafter\expandafter\BitSet@@Clear
    \csname BS@#1\expandafter\endcsname
    \expandafter\BitSet@Empty\expandafter=\expandafter!%
    \romannumeral#2000!%
  }%
  \expandafter\let\csname BS@#1\expandafter\endcsname
  \ifx\BitSet@Temp\BitSet@Empty
    \BitSet@Zero
  \else
    \BitSet@Temp
  \fi
}
%    \end{macrocode}
%    \end{macro}
%    \begin{macro}{\BitSet@@Clear}
%    \begin{macrocode}
\def\BitSet@@Clear#1#2=#3!#4{%
  \ifx#4!%
    \ifx#1\BitSet@Empty
    \else
      \ifx\BitSet@Empty#2%
      \else
        #30#2%
      \fi
    \fi
  \else
    \ifx#1\BitSet@Empty
      \BitSet@AfterFiFi\BitSet@Cleanup
    \else
      \ifx#10%
        \BitSet@AfterFiFiFi{%
          \BitSet@@Clear#2=#30!%
        }%
      \else
        #31%
        \BitSet@AfterFiFiFi{%
          \BitSet@@Clear#2=!%
        }%
      \fi
    \fi
  \BitSet@Fi
}
%    \end{macrocode}
%    \end{macro}
%
% \subsubsection{Set operation}
%
%    \begin{macro}{\BitSet@Set}
%    |#1|: BitSet\\
%    |#2|: plain and checked Index
%    \begin{macrocode}
\def\BitSet@Set#1#2{%
  \expandafter\edef\csname BS@#1\endcsname{%
    \expandafter\expandafter\expandafter\BitSet@@Set
    \csname BS@#1\expandafter\endcsname
    \expandafter\BitSet@Empty\expandafter=%
    \expandafter{\expandafter}\expandafter{\expandafter}%
    \romannumeral#2000!%
  }%
}
%    \end{macrocode}
%    \end{macro}
%    \begin{macro}{\BitSet@@Set}
%    \begin{macrocode}
\def\BitSet@@Set#1#2=#3#4#5{%
  #3#4%
  \ifx#5!%
    1#2%
  \else
    \ifx#1\BitSet@Empty
      0%
      \BitSet@AfterFiFi\BitSet@@@Set
    \else
      #1%
      \BitSet@@Set#2=%
    \fi
  \BitSet@Fi
}
%    \end{macrocode}
%    \end{macro}
%    \begin{macro}{\BitSet@@@Set}
%    \begin{macrocode}
\def\BitSet@@@Set#1{%
  \ifx#1!%
    1%
  \else
    0%
    \expandafter\BitSet@@@Set
  \fi
}
%    \end{macrocode}
%    \end{macro}
%
% \subsubsection{Flip operation}
%
%    \begin{macro}{\BitSet@Flip}
%    |#1|: BitSet\\
%    |#2|: plain and checked Index
%    \begin{macrocode}
\def\BitSet@Flip#1#2{%
  \edef\BitSet@Temp{%
    \expandafter\expandafter\expandafter\BitSet@@Flip
    \csname BS@#1\expandafter\endcsname
    \expandafter\BitSet@Empty\expandafter=\expandafter!%
    \romannumeral#2000!%
  }%
  \expandafter\let\csname BS@#1\expandafter\endcsname
  \ifx\BitSet@Temp\BitSet@Empty
    \BitSet@Zero
  \else
    \BitSet@Temp
  \fi
}
%    \end{macrocode}
%    \end{macro}
%    \begin{macro}{\BitSet@@Flip}
%    \begin{macrocode}
\def\BitSet@@Flip#1#2=#3!#4{%
  \ifx#4!%
    \ifx#11%
      \ifx\BitSet@Empty#2%
      \else
        #30#2%
      \fi
    \else
      #31#2%
    \fi
  \else
    \ifx#1\BitSet@Empty
      #30%
      \BitSet@AfterFiFi\BitSet@@@Set
    \else
      \ifx#10%
        \BitSet@AfterFiFiFi{%
          \BitSet@@Flip#2=#30!%
        }%
      \else
        #31%
        \BitSet@AfterFiFiFi{%
          \BitSet@@Flip#2=!%
        }%
      \fi
    \fi
  \BitSet@Fi
}
%    \end{macrocode}
%    \end{macro}
%
% \subsubsection{Range operators}
%
%    \begin{macro}{\bitsetClearRange}
%    \begin{macrocode}
\def\bitsetClearRange{%
  \BitSet@Range\BitSet@Clear
}
%    \end{macrocode}
%    \end{macro}
%    \begin{macro}{\bitsetSetRange}
%    \begin{macrocode}
\def\bitsetSetRange{%
  \BitSet@Range\BitSet@Set
}
%    \end{macrocode}
%    \end{macro}
%    \begin{macro}{\bitsetFlipRange}
%    \begin{macrocode}
\def\bitsetFlipRange{%
  \BitSet@Range\BitSet@Flip
}
%    \end{macrocode}
%    \end{macro}
%    \begin{macro}{\bitsetSetValueRange}
%    \begin{macrocode}
\def\bitsetSetValueRange#1#2#3#4{%
  \expandafter\expandafter\expandafter\BitSet@SetValueRange
  \intcalcNum{#4}!{#1}{#2}{#3}%
}
%    \end{macrocode}
%    \end{macro}
%    \begin{macro}{\BitSet@SetValueRange}
%    \begin{macrocode}
\def\BitSet@SetValueRange#1!#2#3#4{%
  \ifcase#1 %
    \BitSet@Range\BitSet@Clear{#2}{#3}{#4}%
  \or
    \BitSet@Range\BitSet@Set{#2}{#3}{#4}%
  \else
    \BitSet@ErrorInvalidBitValue{#1}%
  \fi
}
%    \end{macrocode}
%    \end{macro}
%
%    \begin{macro}{\BitSet@Range}
%    |#1|: clear/set/flip macro\\
%    |#2|: BitSet\\
%    |#3|: Index from\\
%    |#4|: Index to
%    \begin{macrocode}
\def\BitSet@Range#1#2#3#4{%
  \edef\BitSet@Temp{%
    \noexpand\BitSet@@Range\noexpand#1{#2}%
    \intcalcNum{#3}!\intcalcNum{#4}!%
  }%
  \BitSet@Temp
}
%    \end{macrocode}
%    \end{macro}
%    \begin{macro}{\BitSet@@Range}
%    |#1|: clear/set/flip macro\\
%    |#2|: BitSet\\
%    |#3|: Index from\\
%    |#4|: Index to
%    \begin{macrocode}
\def\BitSet@@Range#1#2#3!#4!{%
  \ifnum#3<0 %
    \BitSet@NegativeIndex#1{#2}#3!#4!0!#4!%
  \else
    \ifnum#4<0 %
      \BitSet@NegativeIndex#1{#2}#3!#4!#3!0!%
    \else
      \ifcase\intcalcCmp{#3}{#4} %
      \or
        \@PackageError{bitset}{%
          Wrong index numbers in range [#3..#4]\MessageBreak% hash-ok
          for clear/set/flip on bit set `#2'.\MessageBreak
          The lower index exceeds the upper index.\MessageBreak
          Canceling the operation as error recovery%
        }\@ehc
      \else
        \BitSet@@@Range#3!#4!#1{#2}%
      \fi
    \fi
  \fi
}
%    \end{macrocode}
%    \end{macro}
%    \begin{macro}{\BitSet@NegativeIndex}
%    \begin{macrocode}
\def\BitSet@NegativeIndex#1#2#3!#4!#5!#6!{%
  \@PackageError{bitset}{%
    Negative index in range [#3..#4]\MessageBreak % hash-ok
    for \string\bitset
    \ifx#1\BitSet@Clear
      Clear%
    \else
      \ifx#1\BitSet@Set
        Set%
      \else
        Flip%
      \fi
    \fi
    Range on bit set `#2'.\MessageBreak
    Using [#5..#6] as error recovery% hash-ok
  }\@ehc
  \BitSet@@Range#1{#2}#5!#6!%
}
%    \end{macrocode}
%    \end{macro}
%    \begin{macro}{\BitSet@@Range}
%    \begin{macrocode}
\def\BitSet@@@Range#1!#2!#3#4{%
  \ifnum#1<#2 %
    #3{#4}{#1}%
    \BitSet@AfterFi{%
      \expandafter\expandafter\expandafter\BitSet@@@Range
      \IntCalcInc#1!!#2!#3{#4}%
    }%
  \BitSet@Fi
}
%    \end{macrocode}
%    \end{macro}
%
% \subsection{Bit retrieval}\label{src:bitret}
%
% \subsubsection{\cs{bitsetGet}}
%
%    \begin{macro}{\bitsetGet}
%    \begin{macrocode}
\def\bitsetGet#1#2{%
  \number
  \expandafter\expandafter\expandafter\BitSet@Get
  \intcalcNum{#2}!{#1}%
}
%    \end{macrocode}
%    \end{macro}
%    \begin{macro}{\BitSet@Get}
%    |#1|: plain index\\
%    |#2|: BitSet
%    \begin{macrocode}
\def\BitSet@Get#1!#2{%
  \ifnum#1<0 %
    \BitSet@AfterFi{%
      0 \BitSetError:NegativeIndex%
    }%
  \else
    \BitSet@IfUndefined{#2}{0}{%
      \expandafter\expandafter\expandafter\BitSet@@Get
      \csname BS@#2\expandafter\endcsname
      \expandafter!\expandafter=%
      \expandafter{\expandafter}\expandafter{\expandafter}%
      \romannumeral\intcalcNum{#1}000!%
    }%
    \expandafter\BitSet@Space
  \BitSet@Fi
}
%    \end{macrocode}
%    \end{macro}
%    \begin{macro}{\BitSet@@Get}
%    \begin{macrocode}
\def\BitSet@@Get#1#2=#3#4#5{%
  #3#4%
  \ifx#5!%
    \ifx#1!%
      0%
    \else
      #1%
    \fi
  \else
    \ifx#1!%
      0%
      \BitSet@AfterFiFi\BitSet@Cleanup
    \else
      \BitSet@@Get#2=%
    \fi
  \BitSet@Fi
}
%    \end{macrocode}
%    \end{macro}
%
% \subsubsection{\cs{bitsetNextClearBit}, \cs{bitsetNextSetBit}}
%
%    \begin{macro}{\bitsetNextClearBit}
%    \begin{macrocode}
\def\bitsetNextClearBit#1#2{%
  \number
  \expandafter\expandafter\expandafter\BitSet@NextClearBit
  \intcalcNum{#2}!{#1} %
}
%    \end{macrocode}
%    \end{macro}
%    \begin{macro}{\BitSet@NextClearBit}
%    |#1|: Index\\
%    |#2|: BitSet
%    \begin{macrocode}
\def\BitSet@NextClearBit#1!#2{%
  \ifnum#1<0 %
    \BitSet@NextClearBit0!{#2}%
    \BitSet@AfterFi{%
      \expandafter\BitSet@Space
      \expandafter\BitSetError:NegativeIndex\romannumeral0%
    }%
  \else
    \bitsetIsEmpty{#2}{#1}{%
      \expandafter\BitSet@Skip
      \number#1\expandafter\expandafter\expandafter!%
      \csname BS@#2\endcsname!!!!!!!!!=%
      {\BitSet@@NextClearBit#1!}%
    }%
  \BitSet@Fi
}
%    \end{macrocode}
%    \end{macro}
%    \begin{macro}{\BitSet@@NextClearBit}
%    |#1|: index for next bit in |#2|\\
%    |#2|: next bit
%    \begin{macrocode}
\def\BitSet@@NextClearBit#1!#2{%
  \ifx#2!%
    #1%
  \else
    \ifx#20%
      #1%
      \BitSet@AfterFiFi\BitSet@Cleanup
    \else
      \BitSet@AfterFiFi{%
        \expandafter\expandafter\expandafter\BitSet@@NextClearBit
        \IntCalcInc#1!!%
      }%
    \fi
  \BitSet@Fi
}
%    \end{macrocode}
%    \end{macro}
%
%    \begin{macro}{\bitsetNextSetBit}
%    \begin{macrocode}
\def\bitsetNextSetBit#1#2{%
  \number
  \expandafter\expandafter\expandafter\BitSet@NextSetBit
  \intcalcNum{#2}!{#1} %
}
%    \end{macrocode}
%    \end{macro}
%    \begin{macro}{\BitSet@NextSetBit}
%    |#1|: Index\\
%    |#2|: BitSet
%    \begin{macrocode}
\def\BitSet@NextSetBit#1!#2{%
  \ifnum#1<0 %
    \BitSet@NextSetBit0!{#2}%
    \BitSet@AfterFi{%
      \expandafter\BitSet@Space
      \expandafter\BitSetError:NegativeIndex\romannumeral0%
    }%
  \else
    \bitsetIsEmpty{#2}{-1}{%
      \expandafter\BitSet@Skip
      \number#1\expandafter\expandafter\expandafter!%
      \csname BS@#2\endcsname!!!!!!!!!=%
      {\BitSet@@NextSetBit#1!}%
    }%
  \BitSet@Fi
}
%    \end{macrocode}
%    \end{macro}
%    \begin{macro}{\BitSet@@NextSetBit}
%    |#1|: index for next bit in |#2|\\
%    |#2|: next bit
%    \begin{macrocode}
\def\BitSet@@NextSetBit#1!#2{%
  \ifx#2!%
    -1%
  \else
    \ifx#21%
      #1%
      \BitSet@AfterFiFi\BitSet@Cleanup
    \else
      \BitSet@AfterFiFi{%
        \expandafter\expandafter\expandafter\BitSet@@NextSetBit
        \IntCalcInc#1!!%
      }%
    \fi
  \BitSet@Fi
}
%    \end{macrocode}
%    \end{macro}
%    \begin{macro}{\BitSet@Cleanup}
%    \begin{macrocode}
\def\BitSet@Cleanup#1!{}
%    \end{macrocode}
%    \end{macro}
%
%    \begin{macro}{\BitSet@Skip}
%    |#1|: number of bits to skip\\
%    |#2|: bits\\
%    |#3|: continuation code
%    \begin{macrocode}
\def\BitSet@Skip#1!#2{%
  \ifx#2!%
    \BitSet@AfterFi{%
      \BitSet@SkipContinue%
    }%
  \else
    \ifcase#1 %
      \BitSet@AfterFiFi{%
        \BitSet@SkipContinue#2%
      }%
    \or
      \BitSet@AfterFiFi\BitSet@SkipContinue
    \or
      \BitSet@AfterFiFi{%
        \expandafter\BitSet@SkipContinue\BitSet@Gobble
      }%
    \else
      \ifnum#1>8 %
        \BitSet@AfterFiFiFi{%
          \expandafter\BitSet@Skip
          \number\IntCalcSub#1!8!\expandafter!%
          \BitSet@GobbleSeven
        }%
      \else
        \BitSet@AfterFiFiFi{%
          \expandafter\expandafter\expandafter\BitSet@Skip
          \IntCalcDec#1!!%
        }%
      \fi
    \fi
  \BitSet@Fi
}
%    \end{macrocode}
%    \end{macro}
%    \begin{macro}{\BitSet@SkipContinue}
%    |#1|: remaining bits\\
%    |#2|: continuation code
%    \begin{macrocode}
\def\BitSet@SkipContinue#1!#2=#3{%
  #3#1!%
}
%    \end{macrocode}
%    \end{macro}
%    \begin{macro}{\BitSet@GobbleSeven}
%    \begin{macrocode}
\def\BitSet@GobbleSeven#1#2#3#4#5#6#7{}
%    \end{macrocode}
%    \end{macro}
%
% \subsubsection{\cs{bitsetGetSetBitList}}
%
%    \begin{macro}{\bitsetGetSetBitList}
%    It's just a wrapper for \cs{bitsetNextSetBit}.
%    \begin{macrocode}
\def\bitsetGetSetBitList#1{%
  \romannumeral0%
  \bitsetIsEmpty{#1}{ }{%
    \expandafter\BitSet@GetSetBitList
    \number\BitSet@NextSetBit0!{#1}!{#1}{}!%
  }%
}
%    \end{macrocode}
%    \end{macro}
%    \begin{macro}{\BitSet@GetSetBitList}
%    |#1|: found index\\
%    |#2|: BitSet\\
%    |#3|: comma
%    |#4|: result
%    \begin{macrocode}
\def\BitSet@GetSetBitList#1!#2#3#4!{%
  \ifnum#1<0 %
    \BitSet@AfterFi{ #4}%
  \else
    \BitSet@AfterFi{%
      \expandafter\BitSet@GetSetBitList\number
      \expandafter\expandafter\expandafter\BitSet@NextSetBit
      \IntCalcInc#1!!{#2}!{#2},#4#3#1!%
    }%
  \BitSet@Fi
}
%    \end{macrocode}
%    \end{macro}
%
% \subsection{Bit set properties}\label{src:prop}
%
%    \begin{macro}{\bitsetSize}
%    \begin{macrocode}
\def\bitsetSize#1{%
  \number
  \BitSet@IfUndefined{#1}{0 }{%
    \expandafter\expandafter\expandafter\BitSet@Size
    \expandafter\expandafter\expandafter1%
    \expandafter\expandafter\expandafter!%
    \csname BS@#1\endcsname!0!%
  }%
}
%    \end{macrocode}
%    \end{macro}
%    \begin{macro}{\BitSet@Size}
%    |#1|: counter\\
%    |#2#3|: bits\\
%    |#4|: result
%    \begin{macrocode}
\def\BitSet@Size#1!#2#3!#4!{%
  \ifx#21%
    \ifx\\#3\\%
      \BitSet@AfterFiFi{#1 }%
    \else
      \BitSet@AfterFiFi{%
        \expandafter\expandafter\expandafter\BitSet@Size
        \IntCalcInc#1!!#3!#1!%
      }%
    \fi
  \else
    \ifx\\#3\\%
      \BitSet@AfterFiFi{#4 }%
    \else
      \BitSet@AfterFiFi{%
        \expandafter\expandafter\expandafter\BitSet@Size
        \IntCalcInc#1!!#3!#4!%
      }%
    \fi
  \fi
  \BitSet@Fi
}
%    \end{macrocode}
%    \end{macro}
%
%    \begin{macro}{\bitsetCardinality}
%    \begin{macrocode}
\def\bitsetCardinality#1{%
  \number
  \BitSet@IfUndefined{#1}{0 }{%
    \expandafter\expandafter\expandafter\BitSet@Cardinality
    \expandafter\expandafter\expandafter0%
    \expandafter\expandafter\expandafter!%
    \csname BS@#1\endcsname!%
  }%
}
%    \end{macrocode}
%    \end{macro}
%    \begin{macro}{\BitSet@Cardinality}
%    |#1|: result\\
%    |#2#3|: bits
%    \begin{macrocode}
\def\BitSet@Cardinality#1!#2#3!{%
  \ifx#21%
    \ifx\\#3\\%
      \BitSet@AfterFiFi{\IntCalcInc#1! }%
    \else
      \BitSet@AfterFiFi{%
        \expandafter\expandafter\expandafter\BitSet@Cardinality
        \IntCalcInc#1!!#3!%
      }%
    \fi
  \else
    \ifx\\#3\\%
      \BitSet@AfterFiFi{#1 }%
    \else
      \BitSet@AfterFiFi{%
        \BitSet@Cardinality#1!#3!%
      }%
    \fi
  \fi
  \BitSet@Fi
}
%    \end{macrocode}
%    \end{macro}
%
% \subsection{Queries}\label{src:query}
%
%    \begin{macro}{\bitsetIsDefined}
%    \begin{macrocode}
\def\bitsetIsDefined#1{%
  \BitSet@IfUndefined{#1}%
  \BitSet@SecondOfTwo
  \BitSet@FirstOfTwo
}
%    \end{macrocode}
%    \end{macro}
%    \begin{macro}{\bitsetIsEmpty}
%    \begin{macrocode}
\def\bitsetIsEmpty#1{%
  \BitSet@IfUndefined{#1}\BitSet@FirstOfTwo{%
    \expandafter\ifx\csname BS@#1\endcsname\BitSet@Zero
      \expandafter\BitSet@FirstOfTwo
    \else
      \expandafter\BitSet@SecondOfTwo
    \fi
  }%
}
%    \end{macrocode}
%    \end{macro}
%    \begin{macro}{\BitSet@Zero}
%    \begin{macrocode}
\def\BitSet@Zero{0}
%    \end{macrocode}
%    \end{macro}
%
%    \begin{macro}{\bitsetQuery}
%    \begin{macrocode}
\def\bitsetQuery#1#2{%
  \ifnum\bitsetGet{#1}{#2}=1 %
    \expandafter\BitSet@FirstOfTwo
  \else
    \expandafter\BitSet@SecondOfTwo
  \fi
}
%    \end{macrocode}
%    \end{macro}
%
%    \begin{macro}{\bitsetEquals}
%    \begin{macrocode}
\def\bitsetEquals#1#2{%
  \BitSet@IfUndefined{#1}{%
    \BitSet@IfUndefined{#2}\BitSet@FirstOfTwo\BitSet@SecondOfTwo
  }{%
    \BitSet@IfUndefined{#2}\BitSet@SecondOfTwo{%
      \expandafter\ifx\csname BS@#1\expandafter\endcsname
                      \csname BS@#2\endcsname
        \expandafter\BitSet@FirstOfTwo
      \else
        \expandafter\BitSet@SecondOfTwo
      \fi
    }%
  }%
}
%    \end{macrocode}
%    \end{macro}
%
%    \begin{macro}{\bitsetIntersects}
%    \begin{macrocode}
\def\bitsetIntersects#1#2{%
  \bitsetIsEmpty{#1}\BitSet@SecondOfTwo{%
    \bitsetIsEmpty{#2}\BitSet@SecondOfTwo{%
      \expandafter\expandafter\expandafter\BitSet@Intersects
      \csname BS@#1\expandafter\expandafter\expandafter\endcsname
      \expandafter\expandafter\expandafter!%
      \csname BS@#2\endcsname!%
    }%
  }%
}
%    \end{macrocode}
%    \end{macro}
%    \begin{macro}{\BitSet@Intersects}
%    \begin{macrocode}
\def\BitSet@Intersects#1#2!#3#4!{%
  \ifnum#1#3=11 %
    \BitSet@AfterFi\BitSet@FirstOfTwo
  \else
    \ifx\\#2\\%
      \BitSet@AfterFiFi\BitSet@SecondOfTwo
    \else
      \ifx\\#4\\%
        \BitSet@AfterFiFiFi\BitSet@SecondOfTwo
      \else
        \BitSet@AfterFiFiFi{%
          \BitSet@Intersects#2!#4!%
        }%
      \fi
    \fi
  \BitSet@Fi
}
%    \end{macrocode}
%    \end{macro}
%
%    \begin{macrocode}
\BitSet@AtEnd%
%</package>
%    \end{macrocode}
%
% \section{Test}
%
% \subsection{Catcode checks for loading}
%
%    \begin{macrocode}
%<*test1>
%    \end{macrocode}
%    \begin{macrocode}
\catcode`\{=1 %
\catcode`\}=2 %
\catcode`\#=6 %
\catcode`\@=11 %
\expandafter\ifx\csname count@\endcsname\relax
  \countdef\count@=255 %
\fi
\expandafter\ifx\csname @gobble\endcsname\relax
  \long\def\@gobble#1{}%
\fi
\expandafter\ifx\csname @firstofone\endcsname\relax
  \long\def\@firstofone#1{#1}%
\fi
\expandafter\ifx\csname loop\endcsname\relax
  \expandafter\@firstofone
\else
  \expandafter\@gobble
\fi
{%
  \def\loop#1\repeat{%
    \def\body{#1}%
    \iterate
  }%
  \def\iterate{%
    \body
      \let\next\iterate
    \else
      \let\next\relax
    \fi
    \next
  }%
  \let\repeat=\fi
}%
\def\RestoreCatcodes{}
\count@=0 %
\loop
  \edef\RestoreCatcodes{%
    \RestoreCatcodes
    \catcode\the\count@=\the\catcode\count@\relax
  }%
\ifnum\count@<255 %
  \advance\count@ 1 %
\repeat

\def\RangeCatcodeInvalid#1#2{%
  \count@=#1\relax
  \loop
    \catcode\count@=15 %
  \ifnum\count@<#2\relax
    \advance\count@ 1 %
  \repeat
}
\def\RangeCatcodeCheck#1#2#3{%
  \count@=#1\relax
  \loop
    \ifnum#3=\catcode\count@
    \else
      \errmessage{%
        Character \the\count@\space
        with wrong catcode \the\catcode\count@\space
        instead of \number#3%
      }%
    \fi
  \ifnum\count@<#2\relax
    \advance\count@ 1 %
  \repeat
}
\def\space{ }
\expandafter\ifx\csname LoadCommand\endcsname\relax
  \def\LoadCommand{\input bitset.sty\relax}%
\fi
\def\Test{%
  \RangeCatcodeInvalid{0}{47}%
  \RangeCatcodeInvalid{58}{64}%
  \RangeCatcodeInvalid{91}{96}%
  \RangeCatcodeInvalid{123}{255}%
  \catcode`\@=12 %
  \catcode`\\=0 %
  \catcode`\%=14 %
  \LoadCommand
  \RangeCatcodeCheck{0}{36}{15}%
  \RangeCatcodeCheck{37}{37}{14}%
  \RangeCatcodeCheck{38}{47}{15}%
  \RangeCatcodeCheck{48}{57}{12}%
  \RangeCatcodeCheck{58}{63}{15}%
  \RangeCatcodeCheck{64}{64}{12}%
  \RangeCatcodeCheck{65}{90}{11}%
  \RangeCatcodeCheck{91}{91}{15}%
  \RangeCatcodeCheck{92}{92}{0}%
  \RangeCatcodeCheck{93}{96}{15}%
  \RangeCatcodeCheck{97}{122}{11}%
  \RangeCatcodeCheck{123}{255}{15}%
  \RestoreCatcodes
}
\Test
\csname @@end\endcsname
\end
%    \end{macrocode}
%    \begin{macrocode}
%</test1>
%    \end{macrocode}
%
% \subsection{Macro tests}
%
% \subsubsection{Preamble}
%
%    \begin{macrocode}
%<*test2>
\NeedsTeXFormat{LaTeX2e}
\nofiles
\documentclass{article}
\makeatletter
%<*noetex>
\let\SavedNumexpr\numexpr
\let\SavedIfcsname\ifcsname
\let\SavedCurrentgrouplevel\currentgrouplevel
\def\ETeXDisable{%
  \let\ifcsname\@undefined
  \let\numexpr\@undefined
  \let\currentgrouplevel\@undefined
}
\ETeXDisable
%</noetex>
\makeatletter
\chardef\BitSet@TestMode=1 %
\makeatother
\usepackage{bitset}[2011/01/30]
%<*noetex>
\def\ETeXEnable{%
  \let\numexpr\SavedNumexpr
  \let\ifcsname\SavedIfcsname
  \let\currentgrouplevel\SavedCurrentgrouplevel
}
\ETeXEnable
%</noetex>
\usepackage{qstest}
\IncludeTests{*}
\LogTests{log}{*}{*}
\makeatletter
%    \end{macrocode}
%
% \subsubsection{Time}
%
%    \begin{macrocode}
\begingroup\expandafter\expandafter\expandafter\endgroup
\expandafter\ifx\csname pdfresettimer\endcsname\relax
\else
  \newcount\SummaryTime
  \newcount\TestTime
  \SummaryTime=\z@
  \newcommand*{\PrintTime}[2]{%
    \typeout{%
      [Time #1: \strip@pt\dimexpr\number#2sp\relax\space s]%
    }%
  }%
  \newcommand*{\StartTime}[1]{%
    \renewcommand*{\TimeDescription}{#1}%
    \pdfresettimer
  }%
  \newcommand*{\TimeDescription}{}%
  \newcommand*{\StopTime}{%
    \TestTime=\pdfelapsedtime
    \global\advance\SummaryTime\TestTime
    \PrintTime\TimeDescription\TestTime
  }%
  \let\saved@qstest\qstest
  \let\saved@endqstest\endqstest
  \def\qstest#1#2{%
    \saved@qstest{#1}{#2}%
    \StartTime{#1}%
  }%
  \def\endqstest{%
    \StopTime
    \saved@endqstest
  }%
  \AtEndDocument{%
    \PrintTime{summary}\SummaryTime
  }%
\fi
%    \end{macrocode}
%
% \subsubsection{Detection of unwanted space}
%
%    \begin{macrocode}
\let\orig@qstest\qstest
\let\orig@endqstest\endqstest
\def\qstest#1#2{%
  \orig@qstest{#1}{#2}%
  \setbox0\hbox\bgroup\begingroup\ignorespaces
}
\def\endqstest{%
  \endgroup\egroup
  \Expect*{\the\wd0}{0.0pt}%
  \orig@endqstest
}
%    \end{macrocode}
%
% \subsubsection{Test macros}
%
%    \begin{macrocode}
\newcounter{Test}

\def\TestError#1#2{%
  \begingroup
    \setcounter{Test}{0}%
    \sbox0{%
      \def\@PackageError##1##2##3{%
        \stepcounter{Test}%
        \begingroup
          \let\MessageBreak\relax
%<*noetex>
          \ETeXEnable
%</noetex>
          \Expect{##1}{bitset}%
          \Expect*{##2}*{#1}%
        \endgroup
      }%
%<*noetex>
      \ETeXDisable
%</noetex>
      #2%
    }%
    \Expect*{\theTest}{1}%
    \Expect*{\the\wd0}{0.0pt}%
  \endgroup
}

\def\TestErrorNegativeIndex#1#2{%
  \TestError{Invalid negative index (#1)}{#2}%
}

\def\TestGetterUndefined#1{%
  \CheckUndef{dummy}%
  \expandafter\expandafter\expandafter\Expect
  \expandafter\expandafter\expandafter{#1{dummy}}{0}%
}

\def\ExpectBitSet#1#2{%
  \expandafter\expandafter\expandafter\Expect
  \expandafter\expandafter\expandafter
  {\csname BS@#1\endcsname}*{#2}%
}
\def\Check#1#2{%
  \ExpectBitSet{#1}{#2}%
}
\def\CheckUndef#1{%
  \begingroup
    \Expect*{%
      \expandafter
      \ifx\csname BS@#1\endcsname\relax true\else false\fi
    }{true}%
  \endgroup
}
\def\RevCheck#1#2{%
  \ExpectBitSet{#1}{\Reverse#2!!}%
}
\def\Set#1#2{%
  \expandafter\def\csname BS@#1\endcsname{#2}%
}
\def\RevSet#1#2{%
  \expandafter\edef\csname BS@#1\endcsname{%
    \Reverse#2!!%
  }%
}
\def\Reverse#1#2!#3!{%
  \ifx\\#2\\%
    #1#3%
    \expandafter\@gobble
  \else
    \expandafter\@firstofone
  \fi
  {\Reverse#2!#1#3!}%
}
%    \end{macrocode}
%
% \subsubsection{Test sets}
%
%    \begin{macrocode}
\begin{qstest}{Let}{Let}
  \CheckUndef{abc}%
  \CheckUndef{xyz}%
  \bitsetLet{xyz}{abc}%
  \CheckUndef{abc}%
  \Check{xyz}{0}%
  \Set{abc}{1}%
  \Check{abc}{1}%
  \Check{xyz}{0}%
  \bitsetLet{xyz}{abc}%
  \Check{abc}{1}%
  \Check{xyz}{1}%
  \Set{xyz}{11}%
  \Check{abc}{1}%
  \Check{xyz}{11}%
\end{qstest}

\begin{qstest}{Reset}{Reset}
  \bitsetReset{xyz}%
  \Check{xyz}{0}%
  \bitsetReset{abc}%
  \Check{abc}{0}%
  \Set{abc}{10101}%
  \bitsetReset{abc}%
  \Check{abc}{0}%
\end{qstest}

\begin{qstest}{Get/Query}{Get/Query}
  \expandafter\expandafter\expandafter\Expect
  \expandafter\expandafter\expandafter{%
    \bitsetGet{dummy}{0}%
  }{0}%
  \begingroup
    \expandafter\def\csname BitSetError:NegativeIndex\endcsname{}%
    \Set{abc}{1}%
    \Expect*{\bitsetQuery{abc}{-1}{true}{false}}{false}%
  \endgroup
  \def\Test#1#2#3{%
    \Set{abc}{#1}%
    \expandafter\expandafter\expandafter\Expect
    \expandafter\expandafter\expandafter{\bitsetGet{abc}{#2}}{#3}%
    \Expect*{\bitsetQuery{abc}{#2}{true}{false}}%
           *{\ifcase#3 false\or true\else error\fi}%
  }%
  \Test{1}{100}{0}%
  \Test{0}{0}{0}%
  \Test{1}{0}{1}%
  \Test{11}{1}{1}%
  \Test{111}{1}{1}%
  \Test{101}{1}{0}%
  \Test{101}{2}{1}%
  \Test{10100110011}{10}{1}%
\end{qstest}

\begin{qstest}{Size}{Size}
  \TestGetterUndefined\bitsetSize
  \def\Test#1#2{%
    \Set{abc}{#1}%
    \expandafter\expandafter\expandafter\Expect
    \expandafter\expandafter\expandafter{\bitsetSize{abc}}{#2}%
  }%
  \Test{0}{0}%
  \Test{1}{1}%
  \Test{00}{0}%
  \Test{0000000}{0}%
  \Test{10}{1}%
  \Test{01}{2}%
  \Test{11}{2}%
  \Test{010}{2}%
  \Test{011}{3}%
  \Test{100110011}{9}%
  \Test{0000011111000001111100000}{20}%
  \Test{000000000000000000000000011111111111111111111}{45}%
\end{qstest}

\begin{qstest}{Cardinality}{Cardinality}
  \TestGetterUndefined\bitsetCardinality
  \def\Test#1#2{%
    \Set{abc}{#1}%
    \expandafter\expandafter\expandafter\Expect
    \expandafter\expandafter\expandafter{%
      \bitsetCardinality{abc}%
    }{#2}%
  }%
  \Test{0}{0}%
  \Test{1}{1}%
  \Test{00}{0}%
  \Test{0000000}{0}%
  \Test{10}{1}%
  \Test{01}{1}%
  \Test{11}{2}%
  \Test{010}{1}%
  \Test{011}{2}%
  \Test{100110011}{5}%
  \Test{0000011111000001111100000}{10}%
  \Test{000000000000000000000000011111111111111111111}{20}%
\end{qstest}

\begin{qstest}{NextClearBit/NextSetBit}{NextClearBit/NextSetBit}
  \def\Test#1#2{%
    \expandafter\expandafter\expandafter\Expect
    \expandafter\expandafter\expandafter{%
      \TestOp{abc}{#1}%
    }{#2}%
  }%
  \def\Clear{\let\TestOp\bitsetNextClearBit}%
  \def\Set{\let\TestOp\bitsetNextSetBit}%
  \begingroup
    \catcode`\:=11 %
    \bitsetSetBin{abc}{1}%
    \Clear
    \Test{-1}{1\BitSetError:NegativeIndex}%
    \Set
    \Test{-1}{0\BitSetError:NegativeIndex}%
  \endgroup
  \let\BS@abc\@undefined
  \Clear
  \Test{0}{0}%
  \Test{1}{1}%
  \Test{2}{2}%
  \Test{100}{100}%
  \Set
  \Test{0}{-1}%
  \Test{1}{-1}%
  \Test{100}{-1}%
  \bitsetReset{abc}%
  \Clear
  \Test{0}{0}%
  \Test{1}{1}%
  \Test{2}{2}%
  \Test{100}{100}%
  \Set
  \Test{0}{-1}%
  \Test{1}{-1}%
  \Test{100}{-1}%
  \bitsetSetBin{abc}{1}%
  \Clear
  \Test{0}{1}%
  \Test{1}{1}%
  \Test{2}{2}%
  \Test{100}{100}%
  \Set
  \Test{0}{0}%
  \Test{1}{-1}%
  \Test{100}{-1}%
  \bitsetSetBin{abc}{111000111000111000111}%
  \Clear
  \Test{0}{3}%
  \Test{1}{3}%
  \Test{2}{3}%
  \Test{3}{3}%
  \Test{4}{4}%
  \Test{5}{5}%
  \Test{6}{9}%
  \Test{7}{9}%
  \Test{8}{9}%
  \Test{9}{9}%
  \Test{10}{10}%
  \Test{11}{11}%
  \Test{12}{15}%
  \Test{13}{15}%
  \Test{14}{15}%
  \Test{15}{15}%
  \Test{16}{16}%
  \Test{17}{17}%
  \Test{18}{21}%
  \Test{19}{21}%
  \Test{20}{21}%
  \Test{21}{21}%
  \Test{22}{22}%
  \Test{100}{100}%
  \Set
  \Test{0}{0}%
  \Test{1}{1}%
  \Test{2}{2}%
  \Test{3}{6}%
  \Test{4}{6}%
  \Test{5}{6}%
  \Test{6}{6}%
  \Test{7}{7}%
  \Test{8}{8}%
  \Test{9}{12}%
  \Test{10}{12}%
  \Test{11}{12}%
  \Test{12}{12}%
  \Test{13}{13}%
  \Test{14}{14}%
  \Test{15}{18}%
  \Test{16}{18}%
  \Test{17}{18}%
  \Test{18}{18}%
  \Test{19}{19}%
  \Test{20}{20}%
  \Test{21}{-1}%
  \Test{22}{-1}%
  \Test{100}{-1}%
  \bitsetSetBin{abc}{1111111}%
  \Clear
  \Test{6}{7}%
  \Test{7}{7}%
  \Test{8}{8}%
  \Test{100}{100}%
  \Set
  \Test{6}{6}%
  \Test{7}{-1}%
  \Test{8}{-1}%
  \Test{100}{-1}%
  \bitsetSetBin{abc}{11111111}%
  \Clear
  \Test{7}{8}%
  \Test{8}{8}%
  \Test{9}{9}%
  \Test{100}{100}%
  \Set
  \Test{7}{7}%
  \Test{8}{-1}%
  \Test{9}{-1}%
  \Test{100}{-1}%
  \bitsetSetBin{abc}{111111111}%
  \Clear
  \Test{8}{9}%
  \Test{9}{9}%
  \Test{10}{10}%
  \Test{100}{100}%
  \Set
  \Test{8}{8}%
  \Test{9}{-1}%
  \Test{10}{-1}%
  \Test{100}{-1}%
  \bitsetSetBin{abc}{1111111111}%
  \Clear
  \Test{9}{10}%
  \Test{10}{10}%
  \Test{11}{11}%
  \Test{100}{100}%
  \Set
  \Test{9}{9}%
  \Test{10}{-1}%
  \Test{11}{-1}%
  \Test{100}{-1}%
\end{qstest}

\begin{qstest}{GetSetBitList}{GetSetBitList}
  \let\BS@abc\@undefined
  \expandafter\expandafter\expandafter\Expect
  \expandafter\expandafter\expandafter{%
    \bitsetGetSetBitList{abc}%
  }{}%
  \def\Test#1#2{%
    \bitsetSetBin{abc}{#1}%
    \expandafter\expandafter\expandafter\Expect
    \expandafter\expandafter\expandafter{%
      \bitsetGetSetBitList{abc}%
    }{#2}%
  }%
  \Test{0}{}%
  \Test{1}{0}%
  \Test{10}{1}%
  \Test{11}{0,1}%
  \Test{10110100}{2,4,5,7}%
  \Test{101101001010011}{0,1,4,6,9,11,12,14}%
\end{qstest}

\begin{qstest}{GetDec}{GetDec}
  \TestGetterUndefined\bitsetGetDec
  \def\Test#1#2{%
    \RevSet{abc}{#1}%
%<*noetex>
    \begingroup\expandafter\expandafter\expandafter\endgroup
%</noetex>
    \expandafter\expandafter\expandafter\Expect
    \expandafter\expandafter\expandafter{%
      \bitsetGetDec{abc}%
    }{#2}%
  }%
  \Test{0}{0}%
  \Test{1}{1}%
  \Test{10}{2}%
  \Test{11}{3}%
  \Test{100}{4}%
  \Test{101}{5}%
  \Test{110}{6}%
  \Test{111}{7}%
  \Test{1000}{8}%
  \Test{000111}{7}%
  \Test{1111111111111111%
        111111111111111}{2147483647}%
  \Test{0001111111111111111%
        111111111111111}{2147483647}%
  \Test{1000000000000000%
        0000000000000000}{2147483648}%
  \Test{1000000000000000%
        00000000000000000}{4294967296}%
  \Test{0001000000000000000%
        00000000000000000}{4294967296}%
  \Test{1100000000000000%
        00000000000000011}{6442450947}%
\end{qstest}

\begin{qstest}{Clear}{Clear}
  \def\Test#1#2#3{%
    \RevSet{abc}{#1}%
    \bitsetClear{abc}{#2}%
    \Expect*{\BS@abc}*{\Reverse#3!!}%
  }%
  \bitsetClear{abc}{2}%
  \RevCheck{abc}{0}%
  \TestErrorNegativeIndex{-1}{\bitsetClear{abc}{-1}}%
  \RevCheck{abc}{0}%
  \Test{0}{0}{0}%
  \Test{1}{0}{0}%
  \Test{111}{1}{101}%
  \Test{111}{30}{111}%
  \Test{0000111}{5}{0000111}% 111 would also be ok
  \Test{10000111}{5}{10000111}%
  \Test{1001001}{3}{1000001}%
\end{qstest}

\begin{qstest}{Set}{Set}
  \def\Test#1#2#3{%
    \RevSet{abc}{#1}%
    \bitsetSet{abc}{#2}%
    \Expect*{\BS@abc}*{\Reverse#3!!}%
  }%
  \bitsetSet{abc}{2}%
  \RevCheck{abc}{100}%
  \TestErrorNegativeIndex{-1}{\bitsetSet{abc}{-1}}%
  \RevCheck{abc}{100}%
  \Test{0}{0}{1}%
  \Test{1}{0}{1}%
  \Test{100}{1}{110}%
  \Test{111}{1}{111}%
  \Test{11}{1}{11}%
  \Test{11}{2}{111}%
  \Test{11}{3}{1011}%
  \Test{111}{10}{10000000111}%
  \Test{0000111}{5}{0100111}% 100111 would also be ok
  \Test{10000111}{5}{10100111}%
  \Test{1000001}{3}{1001001}%
  \Test{1001001}{3}{1001001}%
\end{qstest}

\begin{qstest}{Flip}{Flip}
  \def\Test#1#2#3{%
    \RevSet{abc}{#1}%
    \bitsetFlip{abc}{#2}%
    \Expect*{\BS@abc}*{\Reverse#3!!}%
  }%
  \bitsetFlip{abc}{2}%
  \RevCheck{abc}{100}%
  \TestErrorNegativeIndex{-1}{\bitsetFlip{abc}{-1}}%
  \RevCheck{abc}{100}%
  \Test{0}{0}{1}%
  \Test{1}{0}{0}%
  \Test{0}{2}{100}%
  \Test{100}{1}{110}%
  \Test{111}{1}{101}%
  \Test{11}{1}{1}%
  \Test{11}{2}{111}%
  \Test{11}{3}{1011}%
  \Test{111}{10}{10000000111}%
  \Test{0000111}{5}{0100111}% 100111 would also be ok
  \Test{10000111}{5}{10100111}%
  \Test{1000001}{3}{1001001}%
  \Test{1001001}{3}{1000001}%
  \Test{11111}{2}{11011}%
\end{qstest}

\begin{qstest}{SetValue}{SetValue}
  \def\Test#1#2{%
    \TestError{Invalid bit value (#2) not in range 0..1}{%
      \bitsetSetValue{abc}{#1}{#2}%
    }%
  }%
  \Test{0}{-1}%
  \Test{0}{2}%
  \Test{0}{10}%
  \def\Test#1#2#3{%
    \let\BS@abc\@undefined
    \bitsetSetValue{abc}{#1}{#2}%
    \bitsetSetBin{result}{#3}%
    \Expect*{\BS@abc}*{\BS@result}%
  }%
  \Test{0}{0}{0}%
  \Test{0}{1}{1}%
  \Test{1}{0}{0}%
  \Test{1}{1}{10}%
  \def\Test#1#2#3#4{%
    \bitsetSetBin{abc}{#1}%
    \bitsetSetBin{result}{#4}%
    \bitsetSetValue{abc}{#2}{#3}%
    \Expect*{\BS@abc}*{\BS@result}%
  }%
  \Test{0}{0}{0}{0}%
  \Test{0}{0}{0}{0}%
  \Test{0}{0}{1}{1}%
  \Test{0}{1}{0}{0}%
  \Test{0}{1}{1}{10}%
  \Test{1010}{2}{1}{1110}%
  \Test{1010}{4}{1}{11010}%
  \Test{1010}{6}{1}{1001010}%
  \Test{1010}{1}{0}{1000}%
  \Test{1010}{2}{0}{1010}%
  \Test{1010}{3}{0}{10}%
  \Test{1010}{4}{0}{1010}%
  \Test{1010}{6}{0}{1010}%
  \Test{1010}{2}{\csname iffalse\endcsname 0\else 1\fi}{1110}%
  \Test{1010}{1}{\csname iffalse\endcsname 1\else 0\fi}{1000}%
\end{qstest}

\begin{qstest}{IsDefined}{IsDefined}
  \let\BS@abc\@undefined
  \Expect*{\bitsetIsDefined{abc}{true}{false}}{false}%
  \bitsetReset{abc}%
  \Expect*{\bitsetIsDefined{abc}{true}{false}}{true}%
\end{qstest}

\begin{qstest}{IsEmpty}{IsEmpty}
  \let\BS@abc\@undefined
  \Expect*{\bitsetIsEmpty{abc}{true}{false}}{true}%
  \bitsetReset{abc}%
  \Expect*{\bitsetIsEmpty{abc}{true}{false}}{true}%
  \bitsetSet{abc}{1}%
  \Expect*{\bitsetIsEmpty{abc}{true}{false}}{false}%
\end{qstest}

\begin{qstest}{Equals}{Equals}
  \def\Test#1#2#3{%
    \Expect*{\bitsetEquals{#1}{#2}{true}{false}}{#3}%
  }%
  \let\BS@abc\@undefined
  \Test{abc}{abc}{true}%
  \Test{abc}{foo}{true}%
  \Test{foo}{abc}{true}%
  \bitsetReset{abc}%
  \Test{abc}{abc}{true}%
  \Test{abc}{foo}{false}%
  \Test{foo}{abc}{false}%
  \bitsetReset{foo}%
  \Test{abc}{foo}{true}%
  \Test{foo}{abc}{true}%
  \bitsetSet{abc}{4}%
  \Test{abc}{foo}{false}%
  \Test{foo}{abc}{false}%
  \bitsetFlip{foo}{4}%
  \Test{abc}{foo}{true}%
  \Test{foo}{abc}{true}%
\end{qstest}

\begin{qstest}{Intersects}{Intersects}
  \def\Test#1{%
    \Expect*{\bitsetIntersects{abc}{foo}{true}{false}}{#1}%
  }%
  \let\BS@abc\@undefined
  \let\BS@foo\@undefined
  \Test{false}%
  \Set{abc}{0}%
  \Test{false}%
  \Set{foo}{0}%
  \Test{false}%
  \let\BS@abc\@undefined
  \Test{false}%
  \Set{foo}{1}%
  \Test{false}%
  \Set{abc}{0}%
  \Test{false}%
  \Set{abc}{1}%
  \Test{true}%
  \let\BS@foo\@undefined
  \Test{false}%
  \Set{foo}{0}%
  \Test{false}%
  \def\Test#1#2#3{%
    \bitsetSetBin{abc}{#1}%
    \bitsetSetBin{foo}{#2}%
    \Expect*{\bitsetIntersects{abc}{foo}{true}{false}}{#3}%
  }%
  \Test{1010}{0101}{false}%
  \Test{0}{10}{false}%
  \Test{1}{11}{true}%
  \Test{11}{1}{true}%
  \Test{10}{1}{false}%
\end{qstest}

\begin{qstest}{And/AndNot/Or/Xor}{And/AndNot/Or/Xor}
  \def\@Test#1#2#3#4#5{%
    \begingroup
      #5%
      \begingroup
        \let\BS@foo\@undefined
        \csname bitset#1\endcsname{abc}{foo}%
        \CheckUndef{foo}%
        \Check{abc}{#2}%
      \endgroup
      \begingroup
        \bitsetReset{foo}%
        \csname bitset#1\endcsname{abc}{foo}%
        \Check{foo}{0}%
        \Check{abc}{#3}%
      \endgroup
      \begingroup
        \def\BS@foo{0101}%
        \csname bitset#1\endcsname{abc}{foo}%
        \Check{foo}{0101}%
        \Check{abc}{#4}%
      \endgroup
    \endgroup
  }%
  \def\Test#1{%
    \def\Op{#1}%
    \Test@
  }%
  \def\Test@#1#2#3#4#5#6#7#8#9{%
    \@Test\Op{#1}{#2}{#3}{%
      \let\BS@abc\@undefined
    }%
    \@Test\Op{#4}{#5}{#6}{%
      \bitsetReset{abc}%
    }%
    \@Test\Op{#7}{#8}{#9}{%
      \def\BS@abc{1001}%
    }%
  }%
  \Test{And}%
       {0}{0}{0}%
       {0}{0}{0}%
       {0}{0}{0001}%
  \Test{AndNot}%
       {0}{0}{0}%
       {0}{0}{0}%
       {1001}{1001}{1}%
  \Test{Or}%
       {0}{0}{0101}%
       {0}{0}{0101}%
       {1001}{1001}{1101}%
  \Test{Xor}%
       {0}{0}{0101}%
       {0}{0}{0101}%
       {1001}{1001}{11}%
  \def\Test#1#2#3{%
    \bitsetSetBin{abc}{#1}%
    \bitsetSetBin{foo}{#2}%
    \csname bitset\Op\endcsname{abc}{foo}%
    \RevCheck{foo}{#2}%
    \RevCheck{abc}{#3}%
  }%
  \def\Op{And}%
  \Test{1}{111}{1}%
  \Test{111}{1}{1}%
  \Test{10}{111}{10}%
  \Test{111}{10}{10}%
  \Test{111}{1000}{0}%
  \Test{1000}{111}{0}%
  \def\Op{AndNot}%
  \Test{1010}{11}{1000}%
  \Test{100}{100}{0}%
  \Test{111}{1111}{0}%
  \Test{100}{111}{0}%
  \def\Op{Or}%
  \Test{0}{0}{0}%
  \Test{1}{0}{1}%
  \Test{0}{1}{1}%
  \Test{1}{1}{1}%
  \Test{1000}{10}{1010}%
  \Test{10}{1000}{1010}%
  \def\Op{Xor}%
  \Test{0}{0}{0}%
  \Test{1}{0}{1}%
  \Test{0}{1}{1}%
  \Test{1}{1}{0}%
  \Test{1000}{10}{1010}%
  \Test{10}{1000}{1010}%
  \Test   {110011001100}%
       {111000111000111}%
       {111110100001011}%
  \Test{111000111000111}%
          {110011001100}%
       {111110100001011}%
\end{qstest}

\begin{qstest}{GetUndef}{GetUndef, GetBin, GetOct, GetHex}
  \def\TestUndef#1#2{%
    \let\BS@abc\@undefined
    \expandafter\expandafter\expandafter\Expect
    \expandafter\expandafter\expandafter{%
      \x{abc}{#1}%
    }{#2}%
  }%
  \let\x\bitsetGetBin
  \TestUndef{-1}{0}%
  \TestUndef{0}{0}%
  \TestUndef{1}{0}%
  \TestUndef{2}{00}%
  \TestUndef{8}{00000000}%
  \let\x\bitsetGetOct
  \TestUndef{-1}{0}%
  \TestUndef{0}{0}%
  \TestUndef{1}{0}%
  \TestUndef{2}{0}%
  \TestUndef{3}{0}%
  \TestUndef{4}{00}%
  \TestUndef{5}{00}%
  \TestUndef{6}{00}%
  \TestUndef{7}{000}%
  \TestUndef{8}{000}%
  \TestUndef{9}{000}%
  \TestUndef{10}{0000}%
  \let\x\bitsetGetHex
  \TestUndef{-1}{0}%
  \TestUndef{0}{0}%
  \TestUndef{1}{0}%
  \TestUndef{2}{0}%
  \TestUndef{3}{0}%
  \TestUndef{4}{0}%
  \TestUndef{5}{00}%
  \TestUndef{6}{00}%
  \TestUndef{7}{00}%
  \TestUndef{8}{00}%
  \TestUndef{9}{000}%
  \TestUndef{10}{000}%
  \TestUndef{12}{000}%
  \TestUndef{13}{0000}%
  \TestUndef{16}{0000}%
  \TestUndef{17}{00000}%
\end{qstest}

\begin{qstest}{SetBin}{SetBin}
  \def\Test#1#2{%
    \let\BS@abc\@undefined
    \bitsetSetBin{abc}{#1}%
    \expandafter\Expect\expandafter{\BS@abc}{#2}%
  }%
  \Test{}{0}%
  \Test{0}{0}%
  \Test{1}{1}%
  \Test{10}{01}%
  \Test{11}{11}%
  \Test{010}{01}%
  \Test{011}{11}%
  \Test{0010}{01}%
  \Test{1010}{0101}%
\end{qstest}

\begin{qstest}{SetOct}{SetOct}
  \def\Test#1#2{%
    \bitsetSetOct{abc}{#1}%
    \expandafter\Expect\expandafter{\BS@abc}{#2}%
  }%
  \Test{}{0}%
  \Test{0}{0}%
  \Test{000}{0}%
  \Test{1}{1}%
  \Test{001}{1}%
  \Test{010}{0001}%
  \Test{020}{00001}%
  \Test{42}{010001}%
  \Test{377}{11111111}%
  \Test{0377}{11111111}%
  \Test{76543210}{000100010110001101011111}%
  \Test{ 0 7 0 7 1 }{100111000111}%
\end{qstest}

\begin{qstest}{SetHex}{SetHex}
  \def\Test#1#2{%
    \bitsetSetHex{abc}{#1}%
    \expandafter\Expect\expandafter{\BS@abc}{#2}%
  }%
  \Test{}{0}%
  \Test{0}{0}%
  \Test{000}{0}%
  \Test{1}{1}%
  \Test{001}{1}%
  \Test{010}{00001}%
  \Test{020}{000001}%
  \Test{42}{0100001}%
  \Test{3F}{111111}%
  \Test{03F}{111111}%
  \Test{43210}{0000100001001100001}%
  \Test{98765}{10100110111000011001}%
  \Test{FEDCBA}{010111010011101101111111}%
  \Test{ 0 F 0 F 1 }{1000111100001111}%
\end{qstest}

\begin{qstest}{SetDec}{SetDec}
  \def\Test#1#2{%
    \bitsetSetDec{abc}{#1}%
    \expandafter\Expect\expandafter{\BS@abc}{#2}%
  }%
  \Test{}{0}%
  \Test{0}{0}%
  \Test{000}{0}%
  \Test{1}{1}%
  \Test{7}{111}%
  \Test{8}{0001}%
  \Test{001}{1}%
  \Test{010}{0101}%
  \Test{020}{00101}%
  \Test{53}{101011}%
  \Test{255}{11111111}%
  \Test{256}{000000001}%
  \Test{999999999}{111111111001001101011001110111}%
  \Test{1000000000}{000000000101001101011001110111}%
  \Test{4210987654}{01100001010010010111111101011111}%
  \Test{2147483647}{1111111111111111111111111111111}%
  \Test{2147483648}{00000000000000000000000000000001}%
\end{qstest}

\begin{qstest}{GetBin}{GetBin}
  \def\TestUndef#1#2{%
    \let\BS@abc\@undefined
    \expandafter\expandafter\expandafter\Expect
    \expandafter\expandafter\expandafter{%
      \bitsetGetBin{abc}{#1}%
    }{#2}%
  }%
  \TestUndef{-1}{0}%
  \TestUndef{0}{0}%
  \TestUndef{1}{0}%
  \TestUndef{2}{00}%
  \TestUndef{8}{00000000}%
  \def\Test#1#2{%
    \bitsetSetBin{abc}{#2}%
    \expandafter\expandafter\expandafter\Expect
    \expandafter\expandafter\expandafter{%
      \bitsetGetBin{abc}{#1}%
    }{#2}%
  }%
  \Test{-1}{0}%
  \Test{0}{0}%
  \Test{1}{0}%
  \Test{1}{1}%
  \Test{2}{01}%
  \Test{2}{10}%
  \Test{3}{010}%
  \Test{2}{00}%
  \Test{2}{01}%
  \Test{8}{00101100}%
  \Test{2}{10101}%
  \Test{-100}{11011}%
\end{qstest}

\begin{qstest}{GetOct}{GetOct}
  \def\Test#1#2#3{%
    \edef\x{\zap@space#1 \@empty}%
    \edef\x{\noexpand\bitsetSetBin{abc}{\x}}%
    \x
    \expandafter\expandafter\expandafter\Expect
    \expandafter\expandafter\expandafter{%
      \bitsetGetOct{abc}{#2}%
    }{#3}%
  }%
  \Test{111 110 101 100 011 010 001 000}{0}{76543210}%
  \Test{000 111}{0}{7}%
  \Test{101 000}{-1}{50}%
  \Test{111}{-1}{7}%
  \Test{111}{0}{7}%
  \Test{111}{1}{7}%
  \Test{111}{3}{7}%
  \Test{111}{4}{07}%
  \Test{111}{6}{07}%
  \Test{111}{7}{007}%
  \Test{111 010}{6}{72}%
  \Test{111 010}{7}{072}%
  \Test{011 111}{0}{37}%
  \Test{011 111}{6}{37}%
  \Test{011 111}{7}{037}%
  \Test{001 111}{0}{17}%
  \Test{001 111}{6}{17}%
  \Test{001 111}{7}{017}%
\end{qstest}

\begin{qstest}{GetHex}{GetHex}
  \def\Test#1#2#3{%
    \bitsetSetBin{abc}{#1}%
    \expandafter\expandafter\expandafter\Expect
    \expandafter\expandafter\expandafter{%
      \bitsetGetHex{abc}{#2}%
    }{#3}%
  }%
  \Test{1111 1110 1101 1100 1011 1010 1001 1000}{0}{FEDCBA98}%
  \Test{0111 0110 0101 0100 0011 0010 0001 0000}{0}{76543210}%
  \Test{0000 1111}{0}{F}%
  \Test{0101 0000}{-1}{50}%
  \Test{1111}{-1}{F}%
  \Test{1111}{0}{F}%
  \Test{1111}{1}{F}%
  \Test{1111}{4}{F}%
  \Test{1111}{5}{0F}%
  \Test{1111}{8}{0F}%
  \Test{1111}{9}{00F}%
  \Test{1111 0010}{8}{F2}%
  \Test{1111 0010}{9}{0F2}%
  \Test{0111 1111}{0}{7F}%
  \Test{0111 1111}{8}{7F}%
  \Test{0111 1111}{9}{07F}%
  \Test{0011 1111}{0}{3F}%
  \Test{0011 1111}{8}{3F}%
  \Test{0011 1111}{9}{03F}%
  \Test{0001 1111}{0}{1F}%
  \Test{0001 1111}{8}{1F}%
  \Test{0001 1111}{9}{01F}%
\end{qstest}

\begin{qstest}{Range}{Range}
  \TestError{%
    Wrong index numbers in range [9..8]\MessageBreak% hash-ok
    for clear/set/flip on bit set `abc'.\MessageBreak
    The lower index exceeds the upper index.\MessageBreak
    Canceling the operation as error recovery%
  }{%
    \bitsetSetRange{abc}{9}{8}%
  }%
  \def\TestErrorNegInd#1#2#3#4#5#6{%
    \TestError{%
      Negative index in range [#2..#3]\MessageBreak % hash-ok
      for \string\bitset #1Range on bit set `abc'.\MessageBreak
      Using [#4..#5] as error recovery% hash-ok
    }{%
      \csname bitset#1Range\endcsname{abc}{#2}{#3}%
      \global\let\BS@global\BS@abc
    }%
    \Check{global}{#6}%
  }%
  \Set{abc}{111}%
  \TestErrorNegInd{Clear}{-1}{0}{0}{0}{111}%
  \TestErrorNegInd{Clear}{0}{-1}{0}{0}{111}%
  \TestErrorNegInd{Clear}{-2}{2}{0}{2}{001}%
  \bitsetReset{abc}%
  \TestErrorNegInd{Set}{-1}{0}{0}{0}{0}%
  \TestErrorNegInd{Set}{0}{-1}{0}{0}{0}%
  \TestErrorNegInd{Set}{-2}{2}{0}{2}{11}%
  \Set{abc}{101}%
  \TestErrorNegInd{Flip}{-1}{0}{0}{0}{101}%
  \TestErrorNegInd{Flip}{0}{-1}{0}{0}{101}%
  \TestErrorNegInd{Flip}{-2}{2}{0}{2}{011}%
  \def\Test#1#2#3#4{%
    \bitsetSetBin{abc}{#1}%
    \csname bitset\TestOp Range\endcsname{abc}{#2}{#3}%
    \Expect*{\bitsetGetBin{abc}{0}}{#4}%
  }%
  \def\TestOp{Clear}%
  \Test{0}{0}{1}{0}%
  \Test{1111}{1}{2}{1101}%
  \Test{1111}{1}{3}{1001}%
  \Test{1111111100000000}{12}{14}{1100111100000000}%
  \def\TestOp{Set}%
  \Test{0}{0}{1}{1}%
  \Test{1000}{1}{2}{1010}%
  \Test{0}{1}{2}{10}%
  \Test{1}{12}{15}{111000000000001}%
  \Test{1111}{1}{3}{1111}%
  \Test{1000000000000000}{12}{14}{1011000000000000}%
  \def\TestOp{Flip}%
  \Test{0}{0}{1}{1}%
  \Test{1}{0}{1}{0}%
  \Test{10101010}{1}{5}{10110100}%
  \def\Test#1#2#3#4#5{%
    \bitsetSetBin{abc}{#1}%
    \bitsetSetValueRange{abc}{#2}{#3}{#4}%
    \Expect*{\bitsetGetBin{abc}{0}}{#5}%
  }%
  \Test{0}{0}{1}{0}{0}%
  \Test{0}{0}{1}{1}{1}%
  \Test{1010}{1}{3}{0}{1000}%
  \Test{1010}{1}{3}{1}{1110}%
\end{qstest}

\begin{qstest}{ShiftLeft/ShiftRight}{ShiftLeft/ShiftRight}
  \def\@Test#1#2{%
    \let\BS@abc\@undefined
    \csname bitsetShift#1\endcsname{abc}{#2}%
    \Expect*{\BS@abc}{0}%
  }%
  \def\Test#1{%
    \@Test{Left}{#1}%
    \@Test{Right}{#1}%
  }%
  \Test{-16}%
  \Test{-1}%
  \Test{0}%
  \Test{1}%
  \Test{16}%
  \def\Test#1#2#3{%
    \bitsetSetBin{abc}{#1}%
    \bitsetSetBin{result}{#3}%
    \csname bitsetShift\Op\endcsname{abc}{#2}%
    \Expect*{\bitsetGetBin{abc}{0}}*{\bitsetGetBin{result}{0}}%
  }%
  \def\Op{Left}%
  \Test{0}{0}{0}%
  \Test{0}{1}{0}%
  \Test{0}{-1}{0}%
  \Test{1}{0}{1}%
  \Test{1}{1}{10}%
  \Test{1}{-1}{0}%
  \Test{10}{1}{100}%
  \Test{10}{-1}{1}%
  \Test{1}{32}{100000000000000000000000000000000}%
  \Test{1}{-100}{0}%
  \def\Op{Right}%
  \Test{0}{0}{0}%
  \Test{0}{1}{0}%
  \Test{0}{-1}{0}%
  \Test{1}{0}{1}%
  \Test{1}{1}{0}%
  \Test{1}{-1}{10}%
  \Test{10}{1}{1}%
  \Test{10}{-1}{100}%
  \Test{1}{-32}{100000000000000000000000000000000}%
  \Test{1}{100}{0}%
  \Test{110110110110110}{10}{11011}%
  \Test{110110110110110}{100}{0}%
  \Test{1}{100000}{0}%
\end{qstest}

\begin{qstest}{Profile: Set}{Profile: Set}
  \bitsetSet{abc}{4095}%
  \global\let\BS@global\BS@abc
\end{qstest}

\begin{qstest}{Profile: Get}{Profile: Get}
  \edef\x{\bitsetGet{global}{4095}}%
\end{qstest}

\begin{document}
\end{document}
%</test2>
%    \end{macrocode}
%
% \section{Installation}
%
% \subsection{Download}
%
% \paragraph{Package.} This package is available on
% CTAN\footnote{\url{ftp://ftp.ctan.org/tex-archive/}}:
% \begin{description}
% \item[\CTAN{macros/latex/contrib/oberdiek/bitset.dtx}] The source file.
% \item[\CTAN{macros/latex/contrib/oberdiek/bitset.pdf}] Documentation.
% \end{description}
%
%
% \paragraph{Bundle.} All the packages of the bundle `oberdiek'
% are also available in a TDS compliant ZIP archive. There
% the packages are already unpacked and the documentation files
% are generated. The files and directories obey the TDS standard.
% \begin{description}
% \item[\CTAN{install/macros/latex/contrib/oberdiek.tds.zip}]
% \end{description}
% \emph{TDS} refers to the standard ``A Directory Structure
% for \TeX\ Files'' (\CTAN{tds/tds.pdf}). Directories
% with \xfile{texmf} in their name are usually organized this way.
%
% \subsection{Bundle installation}
%
% \paragraph{Unpacking.} Unpack the \xfile{oberdiek.tds.zip} in the
% TDS tree (also known as \xfile{texmf} tree) of your choice.
% Example (linux):
% \begin{quote}
%   |unzip oberdiek.tds.zip -d ~/texmf|
% \end{quote}
%
% \paragraph{Script installation.}
% Check the directory \xfile{TDS:scripts/oberdiek/} for
% scripts that need further installation steps.
% Package \xpackage{attachfile2} comes with the Perl script
% \xfile{pdfatfi.pl} that should be installed in such a way
% that it can be called as \texttt{pdfatfi}.
% Example (linux):
% \begin{quote}
%   |chmod +x scripts/oberdiek/pdfatfi.pl|\\
%   |cp scripts/oberdiek/pdfatfi.pl /usr/local/bin/|
% \end{quote}
%
% \subsection{Package installation}
%
% \paragraph{Unpacking.} The \xfile{.dtx} file is a self-extracting
% \docstrip\ archive. The files are extracted by running the
% \xfile{.dtx} through \plainTeX:
% \begin{quote}
%   \verb|tex bitset.dtx|
% \end{quote}
%
% \paragraph{TDS.} Now the different files must be moved into
% the different directories in your installation TDS tree
% (also known as \xfile{texmf} tree):
% \begin{quote}
% \def\t{^^A
% \begin{tabular}{@{}>{\ttfamily}l@{ $\rightarrow$ }>{\ttfamily}l@{}}
%   bitset.sty & tex/generic/oberdiek/bitset.sty\\
%   bitset.pdf & doc/latex/oberdiek/bitset.pdf\\
%   test/bitset-test1.tex & doc/latex/oberdiek/test/bitset-test1.tex\\
%   test/bitset-test2.tex & doc/latex/oberdiek/test/bitset-test2.tex\\
%   test/bitset-test3.tex & doc/latex/oberdiek/test/bitset-test3.tex\\
%   bitset.dtx & source/latex/oberdiek/bitset.dtx\\
% \end{tabular}^^A
% }^^A
% \sbox0{\t}^^A
% \ifdim\wd0>\linewidth
%   \begingroup
%     \advance\linewidth by\leftmargin
%     \advance\linewidth by\rightmargin
%   \edef\x{\endgroup
%     \def\noexpand\lw{\the\linewidth}^^A
%   }\x
%   \def\lwbox{^^A
%     \leavevmode
%     \hbox to \linewidth{^^A
%       \kern-\leftmargin\relax
%       \hss
%       \usebox0
%       \hss
%       \kern-\rightmargin\relax
%     }^^A
%   }^^A
%   \ifdim\wd0>\lw
%     \sbox0{\small\t}^^A
%     \ifdim\wd0>\linewidth
%       \ifdim\wd0>\lw
%         \sbox0{\footnotesize\t}^^A
%         \ifdim\wd0>\linewidth
%           \ifdim\wd0>\lw
%             \sbox0{\scriptsize\t}^^A
%             \ifdim\wd0>\linewidth
%               \ifdim\wd0>\lw
%                 \sbox0{\tiny\t}^^A
%                 \ifdim\wd0>\linewidth
%                   \lwbox
%                 \else
%                   \usebox0
%                 \fi
%               \else
%                 \lwbox
%               \fi
%             \else
%               \usebox0
%             \fi
%           \else
%             \lwbox
%           \fi
%         \else
%           \usebox0
%         \fi
%       \else
%         \lwbox
%       \fi
%     \else
%       \usebox0
%     \fi
%   \else
%     \lwbox
%   \fi
% \else
%   \usebox0
% \fi
% \end{quote}
% If you have a \xfile{docstrip.cfg} that configures and enables \docstrip's
% TDS installing feature, then some files can already be in the right
% place, see the documentation of \docstrip.
%
% \subsection{Refresh file name databases}
%
% If your \TeX~distribution
% (\teTeX, \mikTeX, \dots) relies on file name databases, you must refresh
% these. For example, \teTeX\ users run \verb|texhash| or
% \verb|mktexlsr|.
%
% \subsection{Some details for the interested}
%
% \paragraph{Attached source.}
%
% The PDF documentation on CTAN also includes the
% \xfile{.dtx} source file. It can be extracted by
% AcrobatReader 6 or higher. Another option is \textsf{pdftk},
% e.g. unpack the file into the current directory:
% \begin{quote}
%   \verb|pdftk bitset.pdf unpack_files output .|
% \end{quote}
%
% \paragraph{Unpacking with \LaTeX.}
% The \xfile{.dtx} chooses its action depending on the format:
% \begin{description}
% \item[\plainTeX:] Run \docstrip\ and extract the files.
% \item[\LaTeX:] Generate the documentation.
% \end{description}
% If you insist on using \LaTeX\ for \docstrip\ (really,
% \docstrip\ does not need \LaTeX), then inform the autodetect routine
% about your intention:
% \begin{quote}
%   \verb|latex \let\install=y% \iffalse meta-comment
%
% File: bitset.dtx
% Version: 2011/01/30 v1.1
% Info: Handle bit-vector datatype
%
% Copyright (C) 2007, 2011 by
%    Heiko Oberdiek <heiko.oberdiek at googlemail.com>
%
% This work may be distributed and/or modified under the
% conditions of the LaTeX Project Public License, either
% version 1.3c of this license or (at your option) any later
% version. This version of this license is in
%    http://www.latex-project.org/lppl/lppl-1-3c.txt
% and the latest version of this license is in
%    http://www.latex-project.org/lppl.txt
% and version 1.3 or later is part of all distributions of
% LaTeX version 2005/12/01 or later.
%
% This work has the LPPL maintenance status "maintained".
%
% This Current Maintainer of this work is Heiko Oberdiek.
%
% The Base Interpreter refers to any `TeX-Format',
% because some files are installed in TDS:tex/generic//.
%
% This work consists of the main source file bitset.dtx
% and the derived files
%    bitset.sty, bitset.pdf, bitset.ins, bitset.drv, bitset-test1.tex,
%    bitset-test2.tex, bitset-test3.tex.
%
% Distribution:
%    CTAN:macros/latex/contrib/oberdiek/bitset.dtx
%    CTAN:macros/latex/contrib/oberdiek/bitset.pdf
%
% Unpacking:
%    (a) If bitset.ins is present:
%           tex bitset.ins
%    (b) Without bitset.ins:
%           tex bitset.dtx
%    (c) If you insist on using LaTeX
%           latex \let\install=y\input{bitset.dtx}
%        (quote the arguments according to the demands of your shell)
%
% Documentation:
%    (a) If bitset.drv is present:
%           latex bitset.drv
%    (b) Without bitset.drv:
%           latex bitset.dtx; ...
%    The class ltxdoc loads the configuration file ltxdoc.cfg
%    if available. Here you can specify further options, e.g.
%    use A4 as paper format:
%       \PassOptionsToClass{a4paper}{article}
%
%    Programm calls to get the documentation (example):
%       pdflatex bitset.dtx
%       makeindex -s gind.ist bitset.idx
%       pdflatex bitset.dtx
%       makeindex -s gind.ist bitset.idx
%       pdflatex bitset.dtx
%
% Installation:
%    TDS:tex/generic/oberdiek/bitset.sty
%    TDS:doc/latex/oberdiek/bitset.pdf
%    TDS:doc/latex/oberdiek/test/bitset-test1.tex
%    TDS:doc/latex/oberdiek/test/bitset-test2.tex
%    TDS:doc/latex/oberdiek/test/bitset-test3.tex
%    TDS:source/latex/oberdiek/bitset.dtx
%
%<*ignore>
\begingroup
  \catcode123=1 %
  \catcode125=2 %
  \def\x{LaTeX2e}%
\expandafter\endgroup
\ifcase 0\ifx\install y1\fi\expandafter
         \ifx\csname processbatchFile\endcsname\relax\else1\fi
         \ifx\fmtname\x\else 1\fi\relax
\else\csname fi\endcsname
%</ignore>
%<*install>
\input docstrip.tex
\Msg{************************************************************************}
\Msg{* Installation}
\Msg{* Package: bitset 2011/01/30 v1.1 Handle bit-vector datatype (HO)}
\Msg{************************************************************************}

\keepsilent
\askforoverwritefalse

\let\MetaPrefix\relax
\preamble

This is a generated file.

Project: bitset
Version: 2011/01/30 v1.1

Copyright (C) 2007, 2011 by
   Heiko Oberdiek <heiko.oberdiek at googlemail.com>

This work may be distributed and/or modified under the
conditions of the LaTeX Project Public License, either
version 1.3c of this license or (at your option) any later
version. This version of this license is in
   http://www.latex-project.org/lppl/lppl-1-3c.txt
and the latest version of this license is in
   http://www.latex-project.org/lppl.txt
and version 1.3 or later is part of all distributions of
LaTeX version 2005/12/01 or later.

This work has the LPPL maintenance status "maintained".

This Current Maintainer of this work is Heiko Oberdiek.

The Base Interpreter refers to any `TeX-Format',
because some files are installed in TDS:tex/generic//.

This work consists of the main source file bitset.dtx
and the derived files
   bitset.sty, bitset.pdf, bitset.ins, bitset.drv, bitset-test1.tex,
   bitset-test2.tex, bitset-test3.tex.

\endpreamble
\let\MetaPrefix\DoubleperCent

\generate{%
  \file{bitset.ins}{\from{bitset.dtx}{install}}%
  \file{bitset.drv}{\from{bitset.dtx}{driver}}%
  \usedir{tex/generic/oberdiek}%
  \file{bitset.sty}{\from{bitset.dtx}{package}}%
  \usedir{doc/latex/oberdiek/test}%
  \file{bitset-test1.tex}{\from{bitset.dtx}{test1}}%
  \file{bitset-test2.tex}{\from{bitset.dtx}{test2,etex}}%
  \file{bitset-test3.tex}{\from{bitset.dtx}{test2,noetex}}%
  \nopreamble
  \nopostamble
  \usedir{source/latex/oberdiek/catalogue}%
  \file{bitset.xml}{\from{bitset.dtx}{catalogue}}%
}

\catcode32=13\relax% active space
\let =\space%
\Msg{************************************************************************}
\Msg{*}
\Msg{* To finish the installation you have to move the following}
\Msg{* file into a directory searched by TeX:}
\Msg{*}
\Msg{*     bitset.sty}
\Msg{*}
\Msg{* To produce the documentation run the file `bitset.drv'}
\Msg{* through LaTeX.}
\Msg{*}
\Msg{* Happy TeXing!}
\Msg{*}
\Msg{************************************************************************}

\endbatchfile
%</install>
%<*ignore>
\fi
%</ignore>
%<*driver>
\NeedsTeXFormat{LaTeX2e}
\ProvidesFile{bitset.drv}%
  [2011/01/30 v1.1 Handle bit-vector datatype (HO)]%
\documentclass{ltxdoc}
\usepackage{holtxdoc}[2011/11/22]
\usepackage{amstext}
\usepackage{ifluatex}
\ifluatex
\else
  \usepackage[T1]{fontenc}%
  \usepackage{lmodern}%
\fi
\begin{document}
  \DocInput{bitset.dtx}%
\end{document}
%</driver>
% \fi
%
% \CheckSum{3510}
%
% \CharacterTable
%  {Upper-case    \A\B\C\D\E\F\G\H\I\J\K\L\M\N\O\P\Q\R\S\T\U\V\W\X\Y\Z
%   Lower-case    \a\b\c\d\e\f\g\h\i\j\k\l\m\n\o\p\q\r\s\t\u\v\w\x\y\z
%   Digits        \0\1\2\3\4\5\6\7\8\9
%   Exclamation   \!     Double quote  \"     Hash (number) \#
%   Dollar        \$     Percent       \%     Ampersand     \&
%   Acute accent  \'     Left paren    \(     Right paren   \)
%   Asterisk      \*     Plus          \+     Comma         \,
%   Minus         \-     Point         \.     Solidus       \/
%   Colon         \:     Semicolon     \;     Less than     \<
%   Equals        \=     Greater than  \>     Question mark \?
%   Commercial at \@     Left bracket  \[     Backslash     \\
%   Right bracket \]     Circumflex    \^     Underscore    \_
%   Grave accent  \`     Left brace    \{     Vertical bar  \|
%   Right brace   \}     Tilde         \~}
%
% \GetFileInfo{bitset.drv}
%
% \title{The \xpackage{bitset} package}
% \date{2011/01/30 v1.1}
% \author{Heiko Oberdiek\\\xemail{heiko.oberdiek at googlemail.com}}
%
% \maketitle
%
% \begin{abstract}
% This package defines and implements the data type bit set,
% a vector of bits. The size of the vector may grow dynamically.
% Individual bits can be manipulated.
% \end{abstract}
%
% \tableofcontents
%
% \clearpage
% \section{Documentation}
%
% \subsection{Introduction}
%
% Annotations in the PDF format know entries whose values are
% integers. This numbers are interpreted as set of flags
% specifying properties. For example, annotation dictionaries
% can have a key |/F|. The bits of its integer value
% are interpreted the following way:
% \begin{quote}
% \begin{tabular}{@{}l|l@{}}
%   Bit position & Property name\\
%   \hline
%   1 & Invisible\\
%   2 & Hidden\\
%   3 & Print\\
%   4 & NoZoom\\
%   5 & NoRotate\\
%   6 & NoView\\
%   7 & ReadOnly\\
%   \dots & \dots
% \end{tabular}
% \end{quote}
% Now, let's see how these values are set in package \xpackage{hyperref}
% before it uses this package (before v6.77a):
%\begin{quote}
%\begin{verbatim}
%\ifFld@hidden /F 6\else /F 4\fi
%\end{verbatim}
%\end{quote}
% Where are the other flags? The following example for key |/Ff|
% in a widget annotation
% supports at least three properties:
%\begin{quote}
%\begin{verbatim}
%\ifFld@multiline
%  \ifFld@readonly /Ff 4097\else /Ff 4096\fi
%\else
%  \ifFld@password
%    \ifFld@readonly /Ff 8193\else /Ff 8192\fi
%  \else
%    \ifFld@readonly /Ff 1\fi
%  \fi
%\fi
%\end{verbatim}
%\end{quote}
% But you see the point. It would be a nightmare to continue
% this way in supporting the missing flag settings. This kind
% of integers may have up to 32 bits.
%
% Therefore I wanted a data structure for setting and
% clearing individual bits. Also it should provide
% an export as decimal number. The snipsets above are
% executed in expansion contexts without \TeX's stomach
% commands. It would be convenient to have an expandable
% conversion from the data structure to the integer that
% gets written to the PDF file.
%
% This package \xpackage{bitset} implements such a data
% structure. The interface is quite close to Java's
% class |BitSet| in order not to learn to many interfaces
% for the same kind of data structure.
%
% \subsection{Glossary}
%
% \begin{description}
% \item[Bit set:]
%   A bit set is a vector of bits or flags.
%   The vector size is unlimited and
%   grows dynamically. An undefined bit set is treated as bit
%   set where all bits are cleared.
%
%   Bit sets are addressed by name. A name should consists of
%   letters or digits. Technically it must survive \cs{csname},
%   see \LaTeX's environment names for other names with such
%   a constraint. Package \xpackage{babel}'s shorthands are
%   not supported due to technical reasons. Shorthand support
%   would break expandable operations.
% \item[Size:]
%   A size of a bit set is the number of bits in use.
%   It's the number of the highest index, incremented by one.
%   Sizes are in the range |0| up to |2147483647|, the highest
%   number supported by \TeX.
% \item[Index:]
%   Bit positions in a bit set are addressed by an index number.
%   The bit vector is zero based. The first and least significant
%   bit is addressed by index |0| and the highest possible bit
%   by |2147483646|.
% \item[Bit:]
%   A bit is enoded as |0| for cleared/disabled or |1| for set/enabled.
% \end{description}
%
% \subsection{Design principles}
%
% \begin{description}
% \item[Name conventions:]
%   To avoid conflicts with existing macro names, the operations
%   are prefixed by the package name.
% \item[Zero based indexes:] The first bit is addressed by zero.
%   (Convention of array indexing in C, Java, \dots)
% \item[Unlimited size:]
%   There is no restriction on the size of a bit set other than
%   usual memory limitations. \cs{bitsetSetDec} and \cs{bitsetGetDec}
%   transparently switch to package \xpackage{bigintcalc} if the numbers
%   get too large for \TeX's number limit.
% \item[Expandibility:]
%   Any operation that does not change the bit set is expandable.
%   And all operations that extract or calculate some result
%   do this in exact two expansion steps. For example,
%   a macro \cs{Macro} wants a bit set as decimal number.
%   But the argument must be a plain number without macros.
%   Thus you could prefix \cs{bitsetGetDec} with \cs{number}.
%   However this won't work for bit sets with 31 or more bits
%   because of \TeX's number limit of $2^{31}-1$.
%   then just hit the operator with two \cs{expandafter}:
%   \begin{quote}
%   |\expandafter\expandafter\expandafter|\\
%   |\Macro\bitsetGetDec{foo}|
%   \end{quote}
%   |\bitsetGetDec| is hit first by the third \cs{expandafter}
%   and then by the second one.
% \item[Format independence:] This package is written as \LaTeX\ package,
%   but it does not depend on \LaTeX. It will also work for other formats
%   such as \plainTeX.
% \item[Independence from \TeX\ engines:]
%   Vanilla \TeX\ is all you need. Calculations are delegated to
%   packages \xpackage{intcalc} and \xpackage{bigintcalc}.
%   They don't need any special features, but they will switch
%   to a little more efficient implementation if features such
%   as \cs{numexpr} are available.
% \item[Numeric arguments:] Anything that is accepted by
%   \cs{number}. If \eTeX\ is detected, also expressions
%   for \cs{numexpr} are supported. The only exception so far
%   is the number for \cs{bitsetSetDec}. The number might be too
%   large for \cs{number} or \cs{numexpr}.
% \item[Error messages:]
%   In expandable contexts, only a limited set of \TeX\ primitive
%   commands work as expected. So called stomach commands behave
%   like \cs{relax} and don't get expanded or executed.
%   Unhappily also the error commands belong to this category.
%   The expandable operations will throw an unknown control
%   sequence instead to get \TeX's and user's attention.
%   The name of these control sequences starts with
%   \cs{BitSetError:} with the type of error after the colon.
% \end{description}
%
% \subsection{Operator overview}
%
% \begin{minipage}{\linewidth}
% \begin{description}
% \def\It#1#2{^^A
%   \item[^^A
%     {#1} ^^A
%     \textmd{(\hyperref[{sec:#2}]{section \ref*{sec:#2}})}^^A
%   ]^^A
%   \mbox{}\\^^A
% }
% \def\op#1{^^A
%   \cs{bitset}\texttt{\bfseries#1}^^A
% }
% \def\M#1{\meta{#1}}
% \def\Q{\hspace*{\fill}}
% \It{Miscellaneous}{misc}
%   \op{Reset}\Q \M{BitSet}\\
%   \op{Let}\Q \M{BitSet A} \M{BitSet B}
% \It{Import}{import}
%   \op{SetBin}, \op{SetOct}, \op{SetHex}\Q \M{BitSet} \M{Value}\\
%   \op{SetDec}\Q \M{BitSet} \M{Value}
% \It{Export\textsuperscript{a}}{export}
%   \op{GetBin}, \op{GetOct}, \op{GetHex}\Q \M{BitSet} \M{MinSize}\\
%   \op{GetDec}\Q \M{BitSet}
% \It{Logical operators}{log}
%   \op{And}, \op{AndNot}\Q \M{BitSet A} \M{BitSet B}\\
%   \op{Or}, \op{Xor}\Q \M{BitSet A} \M{BitSet B}
% \It{Shifting}{shift}
%   \op{ShiftLeft}, \op{ShiftRight}\Q \M{BitSet} \M{ShiftAmount}\\
% \It{Bit manipulation}{bitman}
%   \op{Clear}, \op{Set}, \op{Flip}\Q \M{BitSet} \M{Index}\\
%   \op{SetValue}\Q \M{BitSet} \M{Index} \M{Value}\\
%   \op{ClearRange}, \op{SetRange}, \op{FlipRange}\\
%     \Q \M{BitSet} \M{IndexFrom} \M{IndexTo}\\
%   \op{SetValueRange}\Q \M{BitSet} \M{IndexFrom} \M{IndexTo}
% \It{Bit retrieval\textsuperscript{a}}{bitret}
%   \op{Get}\Q \M{BitSet} \M{Index}\\
%   \op{NextClearBit}, \op{NextSetBit}\Q \M{BitSet} \M{Index}\\
%   \op{GetSetBitList}\Q \M{BitSet}
% \It{Bit set properties}{prop}
%   \op{Size}, \op{Cardinality}\Q \M{BitSet}
% \It{Queries\textsuperscript{b}}{query}
%   \op{IsDefined}, \op{IsEmpty}\Q \M{BitSet} \M{Then} \M{Else}\\
%   \op{Equals}, \op{Intersects}\Q \M{BitSet A} \M{BitSet B} \M{Then} \M{Else}\\
%   \op{Query}\Q \M{BitSet} \M{Index} \M{Then} \M{Else}
% \end{description}
% \footnotetext[1]{Macros are expandable, full expansion by two steps.}
% \footnotetext[2]{Macros are expandable.}
% \end{minipage}
%
% \subsection{Package loading}
%
% The package can be used as normal \LaTeX\ package:
%\begin{quote}
%|\usepackage{bitset}|
%\end{quote}
% Also \plainTeX\ is supported:
%\begin{quote}
%|\input bitset.sty\relax|
%\end{quote}
%
% \subsection{Operators}
%
% The following macros work on and with bit sets.
% A bit set \meta{BitSet} is represented by a name.
% The should consist of letters and digits. Technically
% it must survive \cs{csname}. It is the same constraint
% that must be satisfied by label or environment names
% in \LaTeX.
%
% However active characters that are shorthands
% of package \xpackage{babel} are not supported.
% Support for shorthands works by an assignment.
% But many operators
% such as \cs{bitsetGetDec} must be usable in expandable
% contexts. There assignments will not be executed in the
% best case or they will cause errors.
%
% The bits in a bit set are addressed by non-negative
% integers starting from zero. Thus negative index
% numbers cause an error message. Because index numbers
% are \TeX\ numbers. The largest index is 2147483647.
% But in practice memory limits and patience limits
% will be very likely reached much before.
%
% \subsubsection{Miscellaneous}\label{sec:misc}
%
% There isn't a separate operation for bit set creation.
% For simplicity an undefined bit set is treated as bit set
% with all bits cleared.
%
% \begin{declcs}{bitsetReset} \M{BitSet}
% \end{declcs}
% Macro \cs{bitsetReset} clears all bits. The result is an
% empty bit set. It may also be used as replacement for
% an operation ``new'', because an undefined bit set is defined
% afterwards.
%
% \begin{declcs}{bitsetLet} \M{BitSet A} \M{BitSet B}
% \end{declcs}
% Macro \cs{bitsetLet} performs a simple assignment similar
% to \TeX's \cs{let}. After the operation \meta{BitSet A} has
% the same value as \meta{BitSet B}. If \meta{BitSet B} is
% undefined, then \meta{BitSet A} will be the empty bit set.
%
% Note: If \meta{BitSet A} exists, it will be overwritten.
%
% \subsubsection{Import}\label{sec:import}
%
% \begin{declcs}{bitsetSetBin} \M{BitSet} \M{BinaryNumber}\\
%   \cs{bitsetSetOct} \M{BitSet} \M{OctalNumber}\\
%   \cs{bitsetSetHex} \M{BitSet} \M{HexadecimalNumber}
% \end{declcs}
% The numbers are interpreted as bit vectors and the flags in
% the bit \meta{BitSet} set are set accordingly.
% These numeric arguments are the only arguments where
% spaces are allowed. Then the numbers are easier to read.
%
% \begin{declcs}{bitsetSetDec} \M{BitSet} \M{DecimalNumber}
% \end{declcs}
% Macro \cs{bitsetSetDec} uses \meta{DecimalNumber} to
% set the bit set \meta{BitSet}.
% The numeric argument must expand
% to a plain number consisting of decimal digits without
% command tokens or spaces. Internally this argument is
% expanded only. It cannot be passed to \cs{number} or
% \cs{numexpr}, because the number may be too large for them.
% However \cs{number} or |\the\numexpr| may be used
% explicitely. This also helps for unexpandable number
% command tokens or registers (\cs{z@}, \cs{@ne}, \cs{count@}, \dots).
% Also \LaTeX'\ \cs{value} needs prefixing:
% \begin{quote}
% |\bitsetSetDec{foo}{\number\value{bar}}|
% \end{quote}
%
% \subsubsection{Export}\label{sec:export}
%
% \begin{declcs}{bitsetGetBin} \M{BitSet} \M{MinSize}\\
%   \cs{bitsetGetOct} \M{BitSet} \M{MinSize}\\
%   \cs{bitsetGetHex} \M{BitSet} \M{MinSize}
% \end{declcs}
% These macros returns the bit set as binary, octal or hexadecimal
% number. If the bit size is smaller than \meta{MinSize} the gap is
% filled with leading zeros.
% Example:
% \begin{quote}
% |\bitsetReset{abc}|\\
% |\bitsetSet{abc}{2}|\\
% |\bitsetGetBin{abc}{8}| $\rightarrow$ |00000100|\\
% |\bitsetSet{abc}{5}\bitsetSet{abc}{7}|\\
% |\bitsetGetHex{abc}{16}| $\rightarrow$ |00A2|
% \end{quote}
% Macro \cs{bitsetGetHex} uses the uppercase letters |A| to |F|.
% The catcode of the letters is one of 11 (letter) or 12 (other).
%
% \begin{declcs}{bitsetGetDec} \M{BitSet}
% \end{declcs}
% Macro \cs{bitsetGetDec} returns the bit set \meta{BitSet} as
% decimal number. The returned number can be larger than
% \TeX's number limit of $2^{31}-1$.
%
% \subsubsection{Logical operators}\label{sec:log}
%
% \begin{declcs}{bitsetAnd} \M{BitSet A} \M{BitSet B}
% \end{declcs}
% $A_{\text{new}} \mathrel{\mathop{:}}= A_{\text{old}}
% \mathbin{\text{and}} B \qquad (\forall\, \text{bits})$
%
% \begin{declcs}{bitsetAndNot} \M{BitSet A} \M{BitSet B}
% \end{declcs}
% $A_{\text{new}} \mathrel{\mathop{:}}= A_{\text{old}}
% \mathbin{\text{and}} (\text{not}\,B) \qquad (\forall\, \text{bits})$
%
% \begin{declcs}{bitsetOr} \M{BitSet A} \M{BitSet B}
% \end{declcs}
% $A_{\text{new}} \mathrel{\mathop{:}}= A_{\text{old}}
% \mathbin{\text{or}} B \qquad (\forall\, \text{bits})$
%
% \begin{declcs}{bitsetXor} \M{BitSet A} \M{BitSet B}
% \end{declcs}
% $A_{\text{new}} \mathrel{\mathop{:}}= A_{\text{old}}
% \mathbin{\text{xor}} B \qquad (\forall\, \text{bits})$
%
% \subsubsection{Shifting}\label{sec:shift}
%
% \begin{declcs}{bitsetShiftLeft} \M{BitSet} \M{ShiftAmount}\\
%   \cs{bitsetShiftRight} \M{BitSet} \M{ShiftAmount}
% \end{declcs}
% A left shift by one is a multiplication by two, thus left shifting
% moves the flags to higher positions. The new created low positions
% are filled by zeros.
%
% A right shift is the opposite, dividing by two, movint the bits
% to lower positions. The number will become smaller, the lowest
% bits are lost.
%
% If the \meta{ShiftAmount} is negative, it reverts the meaning of
% the shift operation. A left shift becomes a right shift.
% A \meta{ShiftAmount} of zero is ignored.
%
% \subsubsection{Bit manipulation}\label{sec:bitman}
%
% \begin{declcs}{bitsetClear} \M{BitSet} \M{Index}\\
%   \cs{bitsetSet} \M{BitSet} \M{Index}\\
%   \cs{bitsetFlip} \M{BitSet} \M{Index}
% \end{declcs}
% This macros manipulate a single bit in \meta{BitSet} addressed
% by \cs{Index}. Macro \cs{bitsetClear} disables the bit,
% \cs{bitsetSet} enables it and \cs{bitsetFlip} reverts the current
% setting of the bit.
%
% \begin{declcs}{bitsetSetValue} \M{BitSet} \M{Index} \M{Bit}
% \end{declcs}
% Macro \cs{bitsetSetValue} puts bit \meta{Bit} at position \meta{Index}
% in bit set \meta{BitSet}. \meta{Bit} must be a valid \TeX\ number
% equals to zero (disabled/cleared) or one (enabled/set).
%
% \subsubsection{Bit retrieval}\label{sec:bitret}
%
% \begin{declcs}{bitsetGet} \M{BitSet} \M{Index}
% \end{declcs}
% Macro \cs{bitsetGet} extracts the status of the bit at
% position \meta{Index} in bit set \meta{BitSet}.
% Digit |1| is returned if the bit is set/enabled. If the
% bit is cleared/disabled and in cases of an undefined bitset or
% an index number out of range the return value is |0|.
%
% \begin{declcs}{bitsetNextClearBit} \M{BitSet} \M{Index}
% \end{declcs}
% Starting at position \meta{Index} (inclusive) the bits
% are inspected. The first position without a set bit is returned.
% Possible results are decimal numbers: \meta{Index}, \meta{Index} + 1,
% \dots, ($\infty$)
%
% \begin{declcs}{bitsetNextSetBit} \M{BitSet} \M{Index}
% \end{declcs}
% Starting at position \meta{Index} (inclusive) the bits are
% inspected and the index position of the first found set bit
% is returned. If there isn't such a bit, then the result is
% |-1|. In summary possible results are decimal numbers:
% |-1|, \meta{Index}, \meta{Index} + 1, \dots, ($\infty$)
%
% \begin{declcs}{bitsetGetSetBitList} \M{BitSet}
% \end{declcs}
% Macro \cs{bitsetGetSetBitList} is an application for
% \cs{bitsetNextSetBit}. The set bits are iterated and
% returned as comma separated list of index positions
% in increasing order. The list is empty in case of
% an empty bit set.
%
% \subsubsection{Bit set properties}\label{sec:prop}
%
% \begin{declcs}{bitsetSize} \M{BitSet}
% \end{declcs}
% Macro \cs{bitsetSize} returns number of bits in use.
% It is the same as the index number of the highest set/enabled bit
% incremented by one.
%
% \begin{declcs}{bitsetCardinality} \M{BitSet}
% \end{declcs}
% Macro \cs{bitsetCardinality} counts the number of set/enabled bits.
%
% \subsubsection{Queries}\label{sec:query}
%
% Also the query procedures are expandable. They ask for a piece
% of information about a bit set and execute code depending
% on the answer.
%
% \begin{declcs}{bitsetIsDefined} \M{BitSet} \M{Then} \M{Else}
% \end{declcs}
% If the bit set with the name \meta{BitSet} exists the code
% given in \meta{Then} is executed, otherwise \meta{Else} is used.
%
% \begin{declcs}{bitsetIsEmpty} \M{BitSet} \M{Then} \M{Else}
% \end{declcs}
% If the bit set \meta{BitSet} exists and at least one bit
% is set/enabled, the code in \meta{Then} is executed,
% \meta{Else} otherwise.
%
% \begin{declcs}{bitsetEquals} \M{BitSet A} \M{BitSet B} \M{Then} \M{Else}
% \end{declcs}
% Both bit sets are equal if and only if either both are undefined
% or both are defined and represents the same bit values at the
% same positions. Thus this definition is reflexive, symmeric,
% and transitive, enough for an equivalent relation.
%
% \begin{declcs}{bitsetIntersects} \M{BitSet A} \M{BitSet B} \M{Then} \M{Else}
% \end{declcs}
% If and only if \meta{BitSet A} and \meta{BitSet B} have
% at least one bit at the same position that is set, then
% code part \meta{Then} is executed.
%
% \begin{declcs}{bitsetQuery} \M{BitSet} \M{Index} \M{Then} \M{Else}
% \end{declcs}
% It's just a wrapper for \cs{bitsetGet}. If the bit at position
% \meta{Index} is enabled, code \meta{Then} is called.
%
% \StopEventually{
% }
%
% \section{Implementation}
%
% The internal format of a bit set is quite simple, a sequence
% of digits |0| and |1|. The least significant bit is left.
% A bit set without any flag set is encoded by |0|. Also
% undefined bit sets are treated that way. After the
% highest bit that is set there are no further zeroes.
% A regular expression of valid bit sets values:
% \begin{quote}
%  |0|\string||[01]*1|
% \end{quote}
%
%    \begin{macrocode}
%<*package>
%    \end{macrocode}
%
% \subsection{Reload check and package identification}
%    Reload check, especially if the package is not used with \LaTeX.
%    \begin{macrocode}
\begingroup\catcode61\catcode48\catcode32=10\relax%
  \catcode13=5 % ^^M
  \endlinechar=13 %
  \catcode35=6 % #
  \catcode39=12 % '
  \catcode44=12 % ,
  \catcode45=12 % -
  \catcode46=12 % .
  \catcode58=12 % :
  \catcode64=11 % @
  \catcode123=1 % {
  \catcode125=2 % }
  \expandafter\let\expandafter\x\csname ver@bitset.sty\endcsname
  \ifx\x\relax % plain-TeX, first loading
  \else
    \def\empty{}%
    \ifx\x\empty % LaTeX, first loading,
      % variable is initialized, but \ProvidesPackage not yet seen
    \else
      \expandafter\ifx\csname PackageInfo\endcsname\relax
        \def\x#1#2{%
          \immediate\write-1{Package #1 Info: #2.}%
        }%
      \else
        \def\x#1#2{\PackageInfo{#1}{#2, stopped}}%
      \fi
      \x{bitset}{The package is already loaded}%
      \aftergroup\endinput
    \fi
  \fi
\endgroup%
%    \end{macrocode}
%    Package identification:
%    \begin{macrocode}
\begingroup\catcode61\catcode48\catcode32=10\relax%
  \catcode13=5 % ^^M
  \endlinechar=13 %
  \catcode35=6 % #
  \catcode39=12 % '
  \catcode40=12 % (
  \catcode41=12 % )
  \catcode44=12 % ,
  \catcode45=12 % -
  \catcode46=12 % .
  \catcode47=12 % /
  \catcode58=12 % :
  \catcode64=11 % @
  \catcode91=12 % [
  \catcode93=12 % ]
  \catcode123=1 % {
  \catcode125=2 % }
  \expandafter\ifx\csname ProvidesPackage\endcsname\relax
    \def\x#1#2#3[#4]{\endgroup
      \immediate\write-1{Package: #3 #4}%
      \xdef#1{#4}%
    }%
  \else
    \def\x#1#2[#3]{\endgroup
      #2[{#3}]%
      \ifx#1\@undefined
        \xdef#1{#3}%
      \fi
      \ifx#1\relax
        \xdef#1{#3}%
      \fi
    }%
  \fi
\expandafter\x\csname ver@bitset.sty\endcsname
\ProvidesPackage{bitset}%
  [2011/01/30 v1.1 Handle bit-vector datatype (HO)]%
%    \end{macrocode}
%
% \subsection{Catcodes}
%
%    \begin{macrocode}
\begingroup\catcode61\catcode48\catcode32=10\relax%
  \catcode13=5 % ^^M
  \endlinechar=13 %
  \catcode123=1 % {
  \catcode125=2 % }
  \catcode64=11 % @
  \def\x{\endgroup
    \expandafter\edef\csname BitSet@AtEnd\endcsname{%
      \endlinechar=\the\endlinechar\relax
      \catcode13=\the\catcode13\relax
      \catcode32=\the\catcode32\relax
      \catcode35=\the\catcode35\relax
      \catcode61=\the\catcode61\relax
      \catcode64=\the\catcode64\relax
      \catcode123=\the\catcode123\relax
      \catcode125=\the\catcode125\relax
    }%
  }%
\x\catcode61\catcode48\catcode32=10\relax%
\catcode13=5 % ^^M
\endlinechar=13 %
\catcode35=6 % #
\catcode64=11 % @
\catcode123=1 % {
\catcode125=2 % }
\def\TMP@EnsureCode#1#2{%
  \edef\BitSet@AtEnd{%
    \BitSet@AtEnd
    \catcode#1=\the\catcode#1\relax
  }%
  \catcode#1=#2\relax
}
\TMP@EnsureCode{33}{12}% !
\TMP@EnsureCode{39}{12}% '
\TMP@EnsureCode{40}{12}% (
\TMP@EnsureCode{41}{12}% )
\TMP@EnsureCode{42}{12}% *
\TMP@EnsureCode{43}{12}% +
\TMP@EnsureCode{44}{12}% ,
\TMP@EnsureCode{45}{12}% -
\TMP@EnsureCode{46}{12}% .
\TMP@EnsureCode{47}{12}% /
\TMP@EnsureCode{58}{11}% : (letter!)
\TMP@EnsureCode{60}{12}% <
\TMP@EnsureCode{62}{12}% >
\TMP@EnsureCode{63}{14}% ? (comment!)
\TMP@EnsureCode{91}{12}% [
\TMP@EnsureCode{93}{12}% ]
\TMP@EnsureCode{96}{12}% `
\edef\BitSet@AtEnd{\BitSet@AtEnd\noexpand\endinput}
\begingroup\expandafter\expandafter\expandafter\endgroup
\expandafter\ifx\csname BitSet@TestMode\endcsname\relax
\else
  \catcode63=9 % ? (ignore)
\fi
? \let\BitSet@@TestMode\BitSet@TestMode
%    \end{macrocode}
%
% \subsection{Package loading}
%
%    \begin{macrocode}
\begingroup\expandafter\expandafter\expandafter\endgroup
\expandafter\ifx\csname RequirePackage\endcsname\relax
  \def\TMP@RequirePackage#1[#2]{%
    \begingroup\expandafter\expandafter\expandafter\endgroup
    \expandafter\ifx\csname ver@#1.sty\endcsname\relax
      \input #1.sty\relax
    \fi
  }%
  \TMP@RequirePackage{infwarerr}[2007/09/09]%
  \TMP@RequirePackage{intcalc}[2007/09/27]%
  \TMP@RequirePackage{bigintcalc}[2007/09/27]%
\else
  \RequirePackage{infwarerr}[2007/09/09]%
  \RequirePackage{intcalc}[2007/09/27]%
  \RequirePackage{bigintcalc}[2007/09/27]%
\fi
%    \end{macrocode}
%
% \subsection{Help macros}
%
% \subsubsection{Number constant}
%
%    \begin{macro}{\BitSet@MaxSize}
%    \begin{macrocode}
\def\BitSet@MaxSize{2147483647}%
%    \end{macrocode}
%    \end{macro}
%
% \subsubsection{General basic macros}
%
%    \begin{macro}{\BitSet@Empty}
%    \begin{macrocode}
\def\BitSet@Empty{}
%    \end{macrocode}
%    \end{macro}
%    \begin{macro}{\BitSet@FirstOfOne}
%    \begin{macrocode}
\def\BitSet@FirstOfOne#1{#1}
%    \end{macrocode}
%    \end{macro}
%    \begin{macro}{\BitSet@Gobble}
%    \begin{macrocode}
\def\BitSet@Gobble#1{}
%    \end{macrocode}
%    \end{macro}
%    \begin{macro}{\BitSet@FirstOfTwo}
%    \begin{macrocode}
\def\BitSet@FirstOfTwo#1#2{#1}
%    \end{macrocode}
%    \end{macro}
%    \begin{macro}{\BitSet@SecondOfTwo}
%    \begin{macrocode}
\def\BitSet@SecondOfTwo#1#2{#2}
%    \end{macrocode}
%    \end{macro}
%
%    \begin{macro}{\BitSet@Space}
%    \begin{macrocode}
\def\BitSet@Space{ }
%    \end{macrocode}
%    \end{macro}
%    \begin{macro}{\BitSet@ZapSpace}
%    \begin{macrocode}
\def\BitSet@ZapSpace#1 #2{%
  #1%
  \ifx\BitSet@Empty#2%
  \else
    \expandafter\BitSet@ZapSpace
  \fi
  #2%
}
%    \end{macrocode}
%    \end{macro}
%
% \subsubsection{Tail recursion}
%
%    \begin{macro}{\BitSet@Fi}
%    \begin{macrocode}
\let\BitSet@Fi\fi
%    \end{macrocode}
%    \end{macro}
%    \begin{macro}{\BitSet@AfterFi}
%    \begin{macrocode}
\def\BitSet@AfterFi#1#2\BitSet@Fi{\fi#1}
%    \end{macrocode}
%    \end{macro}
%    \begin{macro}{\BitSet@AfterFiFi}
%    \begin{macrocode}
\def\BitSet@AfterFiFi#1#2\BitSet@Fi{\fi\fi#1}%
%    \end{macrocode}
%    \end{macro}
%    \begin{macro}{\BitSet@AfterFiFiFi}
%    \begin{macrocode}
\def\BitSet@AfterFiFiFi#1#2\BitSet@Fi{\fi\fi\fi#1}%
%    \end{macrocode}
%    \end{macro}
%
% \subsubsection{Check macros}
%
%    \begin{macro}{\BitSet@IfUndefined}
%    \begin{macrocode}
\def\BitSet@IfUndefined#1{%
  \expandafter\ifx\csname BS@#1\endcsname\relax
    \expandafter\BitSet@FirstOfTwo
  \else
    \expandafter\BitSet@SecondOfTwo
  \fi
}
%    \end{macrocode}
%    \end{macro}
%
%    \begin{macro}{\BitSet@CheckIndex}
%    |#1|: continuation code\\
%    |#2|: BitSet\\
%    |#3|: Index
%    \begin{macrocode}
\def\BitSet@CheckIndex#1#2#3{%
  \BitSet@IfUndefined{#2}{\bitsetReset{#2}}{}%
  \expandafter\expandafter\expandafter\BitSet@@CheckIndex
  \intcalcNum{#3}!%
  {#2}{#1}%
}
%    \end{macrocode}
%    \end{macro}
%    \begin{macro}{\BitSet@@CheckIndex}
%    |#1|: plain Index\\
%    |#2|: BitSet\\
%    |#3|: continuation code
%    \begin{macrocode}
\def\BitSet@@CheckIndex#1!#2#3{%
  \ifnum#1<0 %
    \BitSet@AfterFi{%
      \@PackageError{bitset}{%
        Invalid negative index (#1)%
      }\@ehc
    }%
  \else
    \BitSet@AfterFi{%
      #3{#2}{#1}%
    }%
  \BitSet@Fi
}
%    \end{macrocode}
%    \end{macro}
%
% \subsection{Miscellaneous}\label{src:misc}
%
%    \begin{macro}{\bitsetReset}
%    \begin{macrocode}
\def\bitsetReset#1{%
  \expandafter\def\csname BS@#1\endcsname{0}%
}
%    \end{macrocode}
%    \end{macro}
%
%    \begin{macro}{\bitsetLet}
%    \begin{macrocode}
\def\bitsetLet#1#2{%
  \BitSet@IfUndefined{#2}{%
    \bitsetReset{#1}%
  }{%
    \expandafter\let\csname BS@#1\expandafter\endcsname
                    \csname BS@#2\endcsname
  }%
}
%    \end{macrocode}
%    \end{macro}
%
% \subsection{Import}\label{src:import}
%
% \subsubsection{From binary number}
%
%    \begin{macro}{\bitsetSetBin}
%    \begin{macrocode}
\def\bitsetSetBin#1#2{%
  \edef\BitSet@Temp{#2}%
  \edef\BitSet@Temp{%
    \expandafter\expandafter\expandafter\BitSet@ZapSpace
    \expandafter\BitSet@Temp\BitSet@Space\BitSet@Empty
  }%
  \edef\BitSet@Temp{%
    \expandafter\BitSet@KillZeros\BitSet@Temp\BitSet@Empty
  }%
  \ifx\BitSet@Temp\BitSet@Empty
    \expandafter\let\csname BS@#1\endcsname\BitSet@Zero
  \else
    \expandafter\edef\csname BS@#1\endcsname{%
      \expandafter\BitSet@Reverse\BitSet@Temp!%
    }%
  \fi
}
%    \end{macrocode}
%    \end{macro}
%    \begin{macro}{\BitSet@KillZeros}
%    \begin{macrocode}
\def\BitSet@KillZeros#1{%
  \ifx#10%
    \expandafter\BitSet@KillZeros
  \else
    #1%
  \fi
}
%    \end{macrocode}
%    \end{macro}
%    \begin{macro}{\BitSet@Reverse}
%    \begin{macrocode}
\def\BitSet@Reverse#1#2!{%
  \ifx\\#2\\%
    #1%
  \else
    \BitSet@AfterFi{%
      \BitSet@Reverse#2!#1%
    }%
  \BitSet@Fi
}
%    \end{macrocode}
%    \end{macro}
%
% \subsubsection{From octal/hex number}
%
%    \begin{macro}{\bitsetSetOct}
%    \begin{macrocode}
\def\bitsetSetOct{%
  \BitSet@SetOctHex\BitSet@FromFirstOct
}
%    \end{macrocode}
%    \end{macro}
%    \begin{macro}{\bitsetSetHex}
%    \begin{macrocode}
\def\bitsetSetHex{%
  \BitSet@SetOctHex\BitSet@FromFirstHex
}
%    \end{macrocode}
%    \end{macro}
%    \begin{macro}{\BitSet@SetOctHex}
%    \begin{macrocode}
\def\BitSet@SetOctHex#1#2#3{%
  \edef\BitSet@Temp{#3}%
  \edef\BitSet@Temp{%
    \expandafter\expandafter\expandafter\BitSet@ZapSpace
    \expandafter\BitSet@Temp\BitSet@Space\BitSet@Empty
  }%
  \edef\BitSet@Temp{%
    \expandafter\BitSet@KillZeros\BitSet@Temp\BitSet@Empty
  }%
  \ifx\BitSet@Temp\BitSet@Empty
    \expandafter\let\csname BS@#2\endcsname\BitSet@Zero
  \else
    \edef\BitSet@Temp{%
      \expandafter#1\BitSet@Temp!%
    }%
    \ifx\BitSet@Temp\BitSet@Empty
      \expandafter\let\csname BS@#2\endcsname\BitSet@Zero
    \else
      \expandafter\edef\csname BS@#2\endcsname{%
        \expandafter\BitSet@Reverse\BitSet@Temp!%
      }%
    \fi
  \fi
}
%    \end{macrocode}
%    \end{macro}
%    \begin{macro}{\BitSet@FromFirstOct}
%    \begin{macrocode}
\def\BitSet@FromFirstOct#1{%
  \ifx#1!%
  \else
    \ifcase#1 \BitSet@AfterFiFi\BitSet@FromFirstOct
    \or 1%
    \or 10%
    \or 11%
    \or 100%
    \or 101%
    \or 110%
    \or 111%
    \else \BitSetError:WrongOctalDigit%
    \fi
    \expandafter\BitSet@FromOct
  \BitSet@Fi
}
%    \end{macrocode}
%    \end{macro}
%    \begin{macro}{\BitSet@FromOct}
%    \begin{macrocode}
\def\BitSet@FromOct#1{%
  \ifx#1!%
  \else
    \ifcase#1 000%
    \or 001%
    \or 010%
    \or 011%
    \or 100%
    \or 101%
    \or 110%
    \or 111%
    \else \BitSetError:WrongOctalDigit%
    \fi
    \expandafter\BitSet@FromOct
  \fi
}
%    \end{macrocode}
%    \end{macro}
%
%    \begin{macro}{\BitSet@FromFirstHex}
%    \begin{macrocode}
\def\BitSet@FromFirstHex#1{%
  \ifx#1!%
  \else
    \ifx#10%
      \BitSet@AfterFiFi\BitSet@FromFirstHex
    \fi
    \expandafter\ifx\csname BitSet@Hex#1\endcsname\relax
      \BitSetError:InvalidHexDigit%
    \else
      \expandafter\expandafter\expandafter\BitSet@KillZeros
      \csname BitSet@Hex#1\endcsname
    \fi
    \expandafter\BitSet@FromHex
  \BitSet@Fi
}
%    \end{macrocode}
%    \end{macro}
%    \begin{macro}{\BitSet@FromHex}
%    \begin{macrocode}
\def\BitSet@FromHex#1{%
  \ifx#1!%
  \else
    \expandafter\ifx\csname BitSet@Hex#1\endcsname\relax
      \BitSetError:InvalidHexDigit%
    \else
      \csname BitSet@Hex#1\endcsname
    \fi
    \expandafter\BitSet@FromHex
  \fi
}
%    \end{macrocode}
%    \end{macro}
%    \begin{macro}{\BitSet@Hex[0..F]}
%    \begin{macrocode}
\def\BitSet@Temp#1{%
  \expandafter\def\csname BitSet@Hex#1\endcsname
}
\BitSet@Temp 0{0000}%
\BitSet@Temp 1{0001}%
\BitSet@Temp 2{0010}%
\BitSet@Temp 3{0011}%
\BitSet@Temp 4{0100}%
\BitSet@Temp 5{0101}%
\BitSet@Temp 6{0110}%
\BitSet@Temp 7{0111}%
\BitSet@Temp 8{1000}%
\BitSet@Temp 9{1001}%
\BitSet@Temp A{1010}%
\BitSet@Temp B{1011}%
\BitSet@Temp C{1100}%
\BitSet@Temp D{1101}%
\BitSet@Temp E{1110}%
\BitSet@Temp F{1111}%
\BitSet@Temp a{1010}%
\BitSet@Temp b{1011}%
\BitSet@Temp c{1100}%
\BitSet@Temp d{1101}%
\BitSet@Temp e{1110}%
\BitSet@Temp f{1111}%
%    \end{macrocode}
%    \end{macro}
%
% \subsubsection{From decimal number}
%
%    \begin{macro}{\bitsetSetDec}
%    \begin{macrocode}
\def\bitsetSetDec#1#2{%
  \edef\BitSet@Temp{#2}%
  \edef\BitSet@Temp{%
    \expandafter\expandafter\expandafter\BitSet@ZapSpace
    \expandafter\BitSet@Temp\BitSet@Space\BitSet@Empty
  }%
  \edef\BitSet@Temp{%
    \expandafter\BitSet@KillZeros\BitSet@Temp\BitSet@Empty
  }%
  \ifx\BitSet@Temp\BitSet@Empty
    \expandafter\let\csname BS@#1\endcsname\BitSet@Zero
  \else
    \ifcase\bigintcalcSgn{\BitSet@Temp} %
      \expandafter\let\csname BS@#1\endcsname\BitSet@Zero
    \or
      \ifnum\bigintcalcCmp\BitSet@Temp\BitSet@MaxSize>0 %
        \expandafter\edef\csname BS@#1\endcsname{%
          \expandafter\BitSet@SetDecBig\BitSet@Temp!%
        }%
      \else
        \expandafter\edef\csname BS@#1\endcsname{%
          \expandafter\BitSet@SetDec\BitSet@Temp!%
        }%
      \fi
    \else
      \@PackageError{bitset}{%
        Bit sets cannot be negative%
      }\@ehc
    \fi
  \fi
}
%    \end{macrocode}
%    \end{macro}
%    \begin{macro}{\BitSet@SetDecBig}
%    \begin{macrocode}
\def\BitSet@SetDecBig#1#2#3#4#5#6#7#8#9!{%
  \ifx\\#9\\%
    \BitSet@SetDec#1#2#3#4#5#6#7#8!%
  \else
    \ifcase\BigIntCalcOdd#1#2#4#5#6#7#8#9! %
      0%
    \or
      1%
?   \else\BitSetError:ThisCannotHappen%
    \fi
    \BitSet@AfterFi{%
      \expandafter\expandafter\expandafter\BitSet@SetDecBig
      \BigIntCalcShr#1#2#3#4#5#6#7#8#9!!%
    }%
  \BitSet@Fi
}
%    \end{macrocode}
%    \end{macro}
%    \begin{macro}{\BitSet@SetDec}
%    \begin{macrocode}
\def\BitSet@SetDec#1!{%
  \ifcase#1 %
  \or 1%
  \else
    \ifodd#1 %
      1%
    \else
      0%
    \fi
    \BitSet@AfterFi{%
      \expandafter\expandafter\expandafter\BitSet@SetDec
      \IntCalcShr#1!!%
    }%
  \BitSet@Fi
}
%    \end{macrocode}
%    \end{macro}
%
% \subsection{Export}\label{src:export}
%
% \subsubsection{To binary number}
%
%    \begin{macro}{\bitsetGetBin}
%    \begin{macrocode}
\def\bitsetGetBin#1#2{%
  \romannumeral0%
  \expandafter\expandafter\expandafter\BitSet@@GetBin
  \intcalcNum{#2}!{#1}%
}
%    \end{macrocode}
%    \end{macro}
%    \begin{macro}{\BitSet@@GetBin}
%    \begin{macrocode}
\def\BitSet@@GetBin#1!#2{%
  \BitSet@IfUndefined{#2}{%
    \ifnum#1>1 %
      \BitSet@AfterFi{%
        \expandafter\expandafter\expandafter\BitSet@Fill
        \IntCalcDec#1!!0%
      }%
    \else
      \BitSet@AfterFi{ 0}%
    \BitSet@Fi
  }{%
    \expandafter\expandafter\expandafter\BitSet@NumBinRev
    \expandafter\expandafter\expandafter1%
    \expandafter\expandafter\expandafter!%
    \csname BS@#2\endcsname!!#1!%
  }%
}
%    \end{macrocode}
%    \end{macro}
%    \begin{macro}{\BitSet@Fill}
%    |#1|: number of leading digits |0|\\
%    |#2|: result
%    \begin{macrocode}
\def\BitSet@Fill#1!{%
  \ifnum#1>0 %
    \BitSet@AfterFi{%
      \expandafter\expandafter\expandafter\BitSet@Fill
      \IntCalcDec#1!!0%
    }%
  \else
    \BitSet@AfterFi{ }%
  \BitSet@Fi
}
%    \end{macrocode}
%    \end{macro}
%
%    \begin{macro}{\BitSet@NumBinRev}
%    |#1|: bit counter (including |#2|)\\
%    |#2#3|: reverted number\\
%    |#4|: result\\
%    |#5|: min size
%    \begin{macrocode}
\def\BitSet@NumBinRev#1!#2#3!{%
  \ifx\\#3\\%
    \BitSet@AfterFi{%
      \BitSet@NumBinFill#1!#2%
    }%
  \else
    \BitSet@AfterFi{%
      \expandafter\expandafter\expandafter\BitSet@NumBinRev
      \IntCalcInc#1!!#3!#2%
    }%
  \BitSet@Fi
}
%    \end{macrocode}
%    \end{macro}
%    \begin{macro}{\BitSet@NumBinFill}
%    \begin{macrocode}
\def\BitSet@NumBinFill#1!#2!#3!{%
  \ifnum#3>#1 %
    \BitSet@AfterFi{%
      \expandafter\expandafter\expandafter\BitSet@Fill
      \IntCalcSub#3!#1!!#2%
    }%
  \else
    \BitSet@AfterFi{ #2}%
  \BitSet@Fi
}
%    \end{macrocode}
%    \end{macro}
%
% \subsubsection{To octal/hexadecimal number}
%
%    \begin{macro}{\bitsetGetOct}
%    \begin{macrocode}
\def\bitsetGetOct#1#2{%
  \romannumeral0%
  \bitsetIsEmpty{#1}{%
    \expandafter\expandafter\expandafter\BitSet@@GetOctHex
    \intcalcNum{#2}!3!230%
  }{%
    \expandafter\expandafter\expandafter\BitSet@@GetOct
    \expandafter\expandafter\expandafter1%
    \expandafter\expandafter\expandafter!%
    \expandafter\expandafter\expandafter!%
    \csname BS@#1\endcsname00%
    \BitSet@Empty\BitSet@Empty\BitSet@Empty!{#2}%
  }%
}
%    \end{macrocode}
%    \end{macro}
%    \begin{macro}{\bitsetGetHex}
%    \begin{macrocode}
\def\bitsetGetHex#1#2{%
  \romannumeral0%
  \bitsetIsEmpty{#1}{%
    \expandafter\expandafter\expandafter\BitSet@@GetOctHex
    \intcalcNum{#2}!4!340%
  }{%
    \expandafter\expandafter\expandafter\BitSet@@GetHex
    \expandafter\expandafter\expandafter1%
    \expandafter\expandafter\expandafter!%
    \expandafter\expandafter\expandafter!%
    \csname BS@#1\endcsname000%
    \BitSet@Empty\BitSet@Empty\BitSet@Empty\BitSet@Empty!{#2}%
  }%
}
%    \end{macrocode}
%    \end{macro}
%    \begin{macro}{\BitSet@@GetOct}
%    |#1|: number of digits\\
%    |#2|: result\\
%    |#3#4#5|: bits
%    \begin{macrocode}
\def\BitSet@@GetOct#1!#2!#3#4#5{%
  \ifx#5\BitSet@Empty
    \BitSet@AfterFi{%
      \expandafter\expandafter\expandafter\BitSet@GetOctHex
      \IntCalcDec#1!!#2!23%
    }%
  \else
    \BitSet@AfterFi{%
      \expandafter\expandafter\expandafter\BitSet@@GetOct
      \number\IntCalcInc#1!\expandafter\expandafter\expandafter!%
      \csname BitSet@Oct#5#4#3\endcsname#2!%
    }%
  \BitSet@Fi
}
%    \end{macrocode}
%    \end{macro}
%    \begin{macro}{\BitSet@Oct[000..111]}
%    \begin{macrocode}
\def\BitSet@Temp#1#2#3#4{%
  \expandafter\def\csname BitSet@Oct#1#2#3\endcsname{#4}%
}
\BitSet@Temp0000%
\BitSet@Temp0011%
\BitSet@Temp0102%
\BitSet@Temp0113%
\BitSet@Temp1004%
\BitSet@Temp1015%
\BitSet@Temp1106%
\BitSet@Temp1117%
%    \end{macrocode}
%    \end{macro}
%    \begin{macro}{\BitSet@@GetHex}
%    |#1|: number of digits\\
%    |#2|: result\\
%    |#3#4#5#6|: bits
%    \begin{macrocode}
\def\BitSet@@GetHex#1!#2!#3#4#5#6{%
  \ifx#6\BitSet@Empty
    \BitSet@AfterFi{%
      \expandafter\expandafter\expandafter\BitSet@GetOctHex
      \IntCalcDec#1!!#2!34%
    }%
  \else
    \BitSet@AfterFi{%
      \expandafter\expandafter\expandafter\BitSet@@GetHex
      \number\IntCalcInc#1!\expandafter\expandafter\expandafter!%
      \csname BitSet@Hex#6#5#4#3\endcsname#2!%
    }%
  \BitSet@Fi
}
%    \end{macrocode}
%    \end{macro}
%    \begin{macro}{\BitSet@Hex[0000..1111]}
%    \begin{macrocode}
\def\BitSet@Temp#1#2#3#4#5{%
  \expandafter\def\csname BitSet@Hex#1#2#3#4\endcsname{#5}%
}
\BitSet@Temp00000%
\BitSet@Temp00011%
\BitSet@Temp00102%
\BitSet@Temp00113%
\BitSet@Temp01004%
\BitSet@Temp01015%
\BitSet@Temp01106%
\BitSet@Temp01117%
\BitSet@Temp10008%
\BitSet@Temp10019%
\BitSet@Temp1010A%
\BitSet@Temp1011B%
\BitSet@Temp1100C%
\BitSet@Temp1101D%
\BitSet@Temp1110E%
\BitSet@Temp1111F%
%    \end{macrocode}
%    \end{macro}
%    \begin{macro}{\BitSet@GetOctHex}
%    Leading zeros $ (|#4| - |#1| * 3 + 2)/3 $ if $ |#4| > |#1| * 3$\\
%    |#1|: digit size\\
%    |#2|: result\\
%    |#3|: bits per digit - 1\\
%    |#4|: bits per digit
%    |#5|: garbage\\
%    |#6|: min size
%    \begin{macrocode}
\def\BitSet@GetOctHex#1!#2!#3#4#5!#6{%
  \expandafter\BitSet@@GetOctHex
  \number\intcalcNum{#6}\expandafter\expandafter\expandafter!%
  \IntCalcMul#1!#4!!#3#4#2%
}
%    \end{macrocode}
%    \end{macro}
%    \begin{macro}{\BitSet@@GetOctHex}
%    |#1|: plain min size\\
%    |#2|: digits * (bits per digit)\\
%    |#3|: bits per digit - 1\\
%    |#4|: bits per digit
%    \begin{macrocode}
\def\BitSet@@GetOctHex#1!#2!#3#4{%
  \ifnum#1>#2 %
    \BitSet@AfterFi{%
      \expandafter\expandafter\expandafter\expandafter
      \expandafter\expandafter\expandafter\BitSet@Fill
      \expandafter\IntCalcDiv\number
      \expandafter\expandafter\expandafter\IntCalcAdd
      \IntCalcSub#1!#2!!#3!!#4!!%
    }%
  \else
    \BitSet@AfterFi{ }%
  \BitSet@Fi
}
%    \end{macrocode}
%    \end{macro}
%
% \subsubsection{To decimal number}
%
%    \begin{macro}{\bitsetGetDec}
%    \begin{macrocode}
\def\bitsetGetDec#1{%
  \romannumeral0%
  \BitSet@IfUndefined{#1}{ 0}{%
    \expandafter\expandafter\expandafter\BitSet@GetDec
    \csname BS@#1\endcsname!%
  }%
}
%    \end{macrocode}
%    \end{macro}
%    \begin{macro}{\BitSet@GetDec}
%    \begin{macrocode}
\def\BitSet@GetDec#1#2!{%
  \ifx\\#2\\%
    \BitSet@AfterFi{ #1}%
  \else
    \BitSet@AfterFi{%
      \BitSet@@GetDec2!#1!#2!%
    }%
  \BitSet@Fi
}
%    \end{macrocode}
%    \end{macro}
%    \begin{macro}{\BitSet@@GetDec}
%    |#1|: power of two\\
%    |#2|: result\\
%    |#3#4|: number
%    \begin{macrocode}
\def\BitSet@@GetDec#1!#2!#3#4!{%
  \ifx\\#4\\%
    \ifx#31%
      \BitSet@AfterFiFi{%
        \expandafter\expandafter\expandafter\BitSet@Space
        \IntCalcAdd#1!#2!%
      }%
    \else
      \BitSet@AfterFiFi{ #2}%
    \fi
  \else
    \ifx#31%
      \BitSet@AfterFiFi{%
        \csname BitSet@N#1%
        \expandafter\expandafter\expandafter\endcsname
        \IntCalcAdd#1!#2!!#4!%
      }%
    \else
      \BitSet@AfterFiFi{%
        \csname BitSet@N#1\endcsname#2!#4!%
      }%
    \fi
  \BitSet@Fi
}
%    \end{macrocode}
%    \end{macro}
%    \begin{macro}{\BitSet@N[1,2,4,...]}
%    \begin{macrocode}
\def\BitSet@Temp#1#2{%
  \expandafter\def\csname BitSet@N#1\endcsname{%
    \BitSet@@GetDec#2!%
  }%
}
\BitSet@Temp{1}{2}
\BitSet@Temp{2}{4}
\BitSet@Temp{4}{8}
\BitSet@Temp{8}{16}
\BitSet@Temp{16}{32}
\BitSet@Temp{32}{64}
\BitSet@Temp{64}{128}
\BitSet@Temp{128}{256}
\BitSet@Temp{256}{512}
\BitSet@Temp{512}{1024}
\BitSet@Temp{1024}{2048}
\BitSet@Temp{2048}{4096}
\BitSet@Temp{4096}{8192}
\BitSet@Temp{8192}{16384}
\BitSet@Temp{16384}{32768}
\BitSet@Temp{32768}{65536}
\BitSet@Temp{65536}{131072}
\BitSet@Temp{131072}{262144}
\BitSet@Temp{262144}{524288}
\BitSet@Temp{524288}{1048576}
\BitSet@Temp{1048576}{2097152}
\BitSet@Temp{2097152}{4194304}
\BitSet@Temp{4194304}{8388608}
\BitSet@Temp{8388608}{16777216}
\BitSet@Temp{16777216}{33554432}
\BitSet@Temp{33554432}{67108864}
\BitSet@Temp{67108864}{134217728}
\BitSet@Temp{134217728}{268435456}
\BitSet@Temp{268435456}{536870912}
\BitSet@Temp{536870912}{1073741824}
%    \end{macrocode}
%    \end{macro}
%    \begin{macro}{\BitSet@N1073741824}
%    \begin{macrocode}
\expandafter\def\csname BitSet@N1073741824\endcsname{%
  \BitSet@GetDecBig2147483648!%
}%
%    \end{macrocode}
%    \end{macro}
%
%    \begin{macro}{\BitSet@GetDecBig}
%    |#1|: current power of two\\
%    |#2|: result\\
%    |#3#4|: number
%    \begin{macrocode}
\def\BitSet@GetDecBig#1!#2!#3#4!{%
  \ifx\\#4\\%
    \ifx#31%
      \BitSet@AfterFiFi{%
        \expandafter\expandafter\expandafter\BitSet@Space
        \BigIntCalcAdd#1!#2!%
      }%
    \else
      \BitSet@AfterFiFi{ #2}%
    \fi
  \else
    \ifx#31%
      \BitSet@AfterFiFi{%
        \expandafter\expandafter\expandafter\BitSet@@GetDecBig
        \BigIntCalcAdd#1!#2!!#1!#4!%
      }%
    \else
      \BitSet@AfterFiFi{%
        \expandafter\expandafter\expandafter\BitSet@GetDecBig
        \BigIntCalcShl#1!!#2!#4!%
      }%
    \fi
  \BitSet@Fi
}
%    \end{macrocode}
%    \end{macro}
%    \begin{macro}{\BitSet@@GetDecBig}
%    |#1|: result\\
%    |#2|: power of two\\
%    |#3#4|: number
%    \begin{macrocode}
\def\BitSet@@GetDecBig#1!#2!{%
  \expandafter\expandafter\expandafter\BitSet@GetDecBig
  \BigIntCalcShl#2!!#1!%
}
%    \end{macrocode}
%    \end{macro}
%
% \subsection{Logical operators}\label{src:log}
%
% \subsubsection{\cs{bitsetAnd}}
%
%    \begin{macro}{\bitsetAnd}
%    Decision table for \cs{bitsetAnd}:
%    \begin{quote}
%    \begin{tabular}{l|l|l|l|}
%                         & undef(B)   & empty(B)   & cardinality(B)$>$0\\
%      \hline
%      undef(A)           & A := empty & A := empty & A := empty\\
%      \hline
%      empty(A)           &            &            &\\
%      \hline
%      cardinality(A)$>$0 & A := empty & A := empty & A \&= B\\
%      \hline
%    \end{tabular}
%    \end{quote}
%    \begin{macrocode}
\def\bitsetAnd#1#2{%
  \bitsetIsEmpty{#1}{%
    \bitsetReset{#1}%
  }{%
    \bitsetIsEmpty{#2}{%
      \bitsetReset{#1}%
    }{%
      \expandafter\edef\csname BS@#1\endcsname{%
        \expandafter\expandafter\expandafter\BitSet@And
        \csname BS@#1\expandafter\expandafter\expandafter\endcsname
        \expandafter\expandafter\expandafter!%
        \csname BS@#2\endcsname!!%
      }%
      \expandafter\ifx\csname BS@#1\endcsname\BitSet@Empty
        \bitsetReset{#1}%
      \fi
    }%
  }%
}
%    \end{macrocode}
%    \end{macro}
%    \begin{macro}{\BitSet@And}
%    \begin{macrocode}
\def\BitSet@And#1#2!#3#4!#5!{%
  \ifx\\#2\\%
    \ifnum#1#3=11 #51\fi
  \else
    \ifx\\#4\\%
      \ifnum#1#3=11 #51\fi
    \else
      \ifnum#1#3=11 %
        #51%
        \BitSet@AfterFiFiFi{%
          \BitSet@And#2!#4!!%
        }%
      \else
        \BitSet@AfterFiFiFi{%
          \BitSet@And#2!#4!#50!%
        }%
      \fi
    \fi
  \BitSet@Fi
}
%    \end{macrocode}
%    \end{macro}
%
% \subsubsection{\cs{bitsetAndNot}}
%
%    \begin{macro}{\bitsetAndNot}
%    Decision table for \cs{bitsetAndNot}:
%    \begin{quote}
%    \begin{tabular}{l|l|l|l|}
%                         & undef(B)   & empty(B)   & cardinality(B)$>$0\\
%      \hline
%      undef(A)           & A := empty & A := empty & A := empty\\
%      \hline
%      empty(A)           &            &            &\\
%      \hline
%      cardinality(A)$>$0 &            &            & A \&= !B\\
%      \hline
%    \end{tabular}
%    \end{quote}
%    \begin{macrocode}
\def\bitsetAndNot#1#2{%
  \bitsetIsEmpty{#1}{%
    \bitsetReset{#1}%
  }{%
    \bitsetIsEmpty{#2}{%
    }{%
      \expandafter\edef\csname BS@#1\endcsname{%
        \expandafter\expandafter\expandafter\BitSet@AndNot
        \csname BS@#1\expandafter\expandafter\expandafter\endcsname
        \expandafter\expandafter\expandafter!%
        \csname BS@#2\endcsname!!%
      }%
      \expandafter\ifx\csname BS@#1\endcsname\BitSet@Empty
        \bitsetReset{#1}%
      \fi
    }%
  }%
}
%    \end{macrocode}
%    \end{macro}
%    \begin{macro}{\BitSet@AndNot}
%    \begin{macrocode}
\def\BitSet@AndNot#1#2!#3#4!#5!{%
  \ifx\\#2\\%
    \ifnum#1#3=10 #51\fi
  \else
    \ifx\\#4\\%
      #5%
      \ifnum#1#3=10 1\else 0\fi
      #2%
    \else
      \ifnum#1#3=10 %
        #51%
        \BitSet@AfterFiFiFi{%
          \BitSet@AndNot#2!#4!!%
        }%
      \else
        \BitSet@AfterFiFiFi{%
          \BitSet@AndNot#2!#4!#50!%
        }%
      \fi
    \fi
  \BitSet@Fi
}
%    \end{macrocode}
%    \end{macro}
%
% \subsubsection{\cs{bitsetOr}}
%
%    \begin{macro}{\bitsetOr}
%    Decision table for \cs{bitsetOr}:
%    \begin{quote}
%    \begin{tabular}{l|l|l|l|}
%                         & undef(B)   & empty(B)   & cardinality(B)$>$0\\
%      \hline
%      undef(A)           & A := empty & A := empty & A := B\\
%      \hline
%      empty(A)           &            &            & A := B\\
%      \hline
%      cardinality(A)$>$0 &            &            & A \string|= B\\
%      \hline
%    \end{tabular}
%    \end{quote}
%    \begin{macrocode}
\def\bitsetOr#1#2{%
  \bitsetIsEmpty{#2}{%
    \BitSet@IfUndefined{#1}{\bitsetReset{#1}}{}%
  }{%
    \bitsetIsEmpty{#1}{%
      \expandafter\let\csname BS@#1\expandafter\endcsname
                      \csname BS@#2\endcsname
    }{%
      \expandafter\edef\csname BS@#1\endcsname{%
        \expandafter\expandafter\expandafter\BitSet@Or
        \csname BS@#1\expandafter\expandafter\expandafter\endcsname
        \expandafter\expandafter\expandafter!%
        \csname BS@#2\endcsname!%
      }%
    }%
  }%
}
%    \end{macrocode}
%    \end{macro}
%    \begin{macro}{\BitSet@Or}
%    \begin{macrocode}
\def\BitSet@Or#1#2!#3#4!{%
  \ifnum#1#3>0 1\else 0\fi
  \ifx\\#2\\%
    #4%
  \else
    \ifx\\#4\\%
      #2%
    \else
      \BitSet@AfterFiFi{%
        \BitSet@Or#2!#4!%
      }%
    \fi
  \BitSet@Fi
}
%    \end{macrocode}
%    \end{macro}
%
% \subsubsection{\cs{bitsetXor}}
%
%    \begin{macro}{\bitsetXor}
%    Decision table for \cs{bitsetXor}:
%    \begin{quote}
%    \begin{tabular}{l|l|l|l|}
%                         & undef(B)   & empty(B)   & cardinality(B)$>$0\\
%      \hline
%      undef(A)           & A := empty & A := empty & A := B\\
%      \hline
%      empty(A)           &            &            & A := B\\
%      \hline
%      cardinality(A)$>$0 &            &            & A \string^= B\\
%      \hline
%    \end{tabular}
%    \end{quote}
%    \begin{macrocode}
\def\bitsetXor#1#2{%
  \bitsetIsEmpty{#2}{%
    \BitSet@IfUndefined{#1}{\bitsetReset{#1}}{}%
  }{%
    \bitsetIsEmpty{#1}{%
      \expandafter\let\csname BS@#1\expandafter\endcsname
                      \csname BS@#2\endcsname
    }{%
      \expandafter\edef\csname BS@#1\endcsname{%
        \expandafter\expandafter\expandafter\BitSet@Xor
        \csname BS@#1\expandafter\expandafter\expandafter\endcsname
        \expandafter\expandafter\expandafter!%
        \csname BS@#2\endcsname!!%
      }%
      \expandafter\ifx\csname BS@#1\endcsname\BitSet@Empty
        \bitsetReset{#1}%
      \fi
    }%
  }%
}
%    \end{macrocode}
%    \end{macro}
%    \begin{macro}{\BitSet@Xor}
%    \begin{macrocode}
\def\BitSet@Xor#1#2!#3#4!#5!{%
  \ifx\\#2\\%
    \ifx#1#3%
      \ifx\\#4\\%
      \else
        #50#4%
      \fi
    \else
      #51#4%
    \fi
  \else
    \ifx\\#4\\%
      #5%
      \ifx#1#30\else 1\fi
      #2%
    \else
      \ifx#1#3%
        \BitSet@AfterFiFiFi{%
          \BitSet@Xor#2!#4!#50!%
        }%
      \else
        #51%
        \BitSet@AfterFiFiFi{%
          \BitSet@Xor#2!#4!!%
        }%
      \fi
    \fi
  \BitSet@Fi
}
%    \end{macrocode}
%    \end{macro}
%
% \subsubsection{Shifting}\label{src:shift}
%
% \subsubsection{\cs{bitsetShiftLeft}}
%
%    \begin{macro}{\bitsetShiftLeft}
%    \begin{macrocode}
\def\bitsetShiftLeft#1#2{%
  \BitSet@IfUndefined{#1}{%
    \bitsetReset{#1}%
  }{%
    \bitsetIsEmpty{#1}{%
    }{%
      \expandafter\expandafter\expandafter\BitSet@ShiftLeft
      \intcalcNum{#2}!{#1}%
    }%
  }%
}
%    \end{macrocode}
%    \end{macro}
%    \begin{macro}{\BitSet@ShiftLeft}
%    \begin{macrocode}
\def\BitSet@ShiftLeft#1!#2{%
  \ifcase\intcalcSgn{#1} %
  \or
    \begingroup
      \uccode`m=`0 %
    \uppercase\expandafter{\expandafter\endgroup
      \expandafter\edef\csname BS@#2\expandafter\endcsname
      \expandafter{%
        \romannumeral#1000\expandafter\BitSet@Space
        \csname BS@#2\endcsname
      }%
    }%
  \else
    \expandafter\BitSet@ShiftRight\BitSet@Gobble#1!{#2}%
  \fi
}
%    \end{macrocode}
%    \end{macro}
%
% \subsubsection{\cs{bitsetShiftRight}}
%
%    \begin{macro}{\bitsetShiftRight}
%    \begin{macrocode}
\def\bitsetShiftRight#1#2{%
  \BitSet@IfUndefined{#1}{%
    \bitsetReset{#1}%
  }{%
    \bitsetIsEmpty{#1}{%
    }{%
      \expandafter\expandafter\expandafter\BitSet@ShiftRight
      \intcalcNum{#2}!{#1}%
    }%
  }%
}
%    \end{macrocode}
%    \end{macro}
%    \begin{macro}{\BitSet@ShiftRight}
%    \begin{macrocode}
\def\BitSet@ShiftRight#1!#2{%
  \ifcase\intcalcSgn{#1} %
  \or
    \expandafter\edef\csname BS@#2\endcsname{%
      \expandafter\expandafter\expandafter\BitSet@Kill
      \csname BS@#2\expandafter\endcsname\expandafter\BitSet@Empty
      \expandafter=%
      \expandafter{\expandafter}\expandafter{\expandafter}%
      \romannumeral#1000!%
    }%
  \else
    \expandafter\BitSet@ShiftLeft\BitSet@Gobble#1!{#2}%
  \fi
}
%    \end{macrocode}
%    \end{macro}
%    \begin{macro}{\BitSet@Kill}
%    \begin{macrocode}
\def\BitSet@Kill#1#2=#3#4#5{%
  #3#4%
  \ifx#5!%
    \ifx#1\BitSet@Empty
      0%
    \else
      #1#2%
    \fi
  \else
    \ifx#1\BitSet@Empty
      0%
      \BitSet@AfterFiFi\BitSet@Cleanup
    \else
      \BitSet@Kill#2=%
    \fi
  \BitSet@Fi
}
%    \end{macrocode}
%    \end{macro}
%
% \subsection{Bit manipulation}\label{src:bitman}
%
%    \begin{macro}{\bitsetClear}
%    \begin{macrocode}
\def\bitsetClear{%
  \BitSet@CheckIndex\BitSet@Clear
}
%    \end{macrocode}
%    \end{macro}
%    \begin{macro}{\bitsetSet}
%    \begin{macrocode}
\def\bitsetSet{%
  \BitSet@CheckIndex\BitSet@Set
}
%    \end{macrocode}
%    \end{macro}
%    \begin{macro}{\bitsetFlip}
%    \begin{macrocode}
\def\bitsetFlip{%
  \BitSet@CheckIndex\BitSet@Flip
}
%    \end{macrocode}
%    \end{macro}
%    \begin{macro}{\bitsetSetValue}
%    \begin{macrocode}
\def\bitsetSetValue#1#2#3{%
  \expandafter\expandafter\expandafter\BitSet@SetValue
  \intcalcNum{#3}!{#1}{#2}%
}
%    \end{macrocode}
%    \end{macro}
%    \begin{macro}{\BitSet@SetValue}
%    |#1|: plain value\\
%    |#2|: BitSet\\
%    |#3|: Index
%    \begin{macrocode}
\def\BitSet@SetValue#1!{%
  \BitSet@CheckIndex{%
    \ifcase#1 %
      \expandafter\BitSet@Clear
    \or
      \expandafter\BitSet@Set
    \else
      \BitSet@ErrorInvalidBitValue{#1}%
      \expandafter\expandafter\expandafter\BitSet@Gobble
      \expandafter\BitSet@Gobble
    \fi
  }%
}
%    \end{macrocode}
%    \end{macro}
%    \begin{macro}{\BitSet@ErrorInvalidBitValue}
%    |#1|: Wrong bit value
%    \begin{macrocode}
\def\BitSet@ErrorInvalidBitValue#1{%
  \@PackageError{bitset}{%
    Invalid bit value (#1) not in range 0..1%
  }\@ehc
}
%    \end{macrocode}
%    \end{macro}
%
% \subsubsection{Clear operation}
%
%    \begin{macro}{\BitSet@Clear}
%    |#1|: BitSet\\
%    |#2|: plain and checked index
%    \begin{macrocode}
\def\BitSet@Clear#1#2{%
  \edef\BitSet@Temp{%
    \expandafter\expandafter\expandafter\BitSet@@Clear
    \csname BS@#1\expandafter\endcsname
    \expandafter\BitSet@Empty\expandafter=\expandafter!%
    \romannumeral#2000!%
  }%
  \expandafter\let\csname BS@#1\expandafter\endcsname
  \ifx\BitSet@Temp\BitSet@Empty
    \BitSet@Zero
  \else
    \BitSet@Temp
  \fi
}
%    \end{macrocode}
%    \end{macro}
%    \begin{macro}{\BitSet@@Clear}
%    \begin{macrocode}
\def\BitSet@@Clear#1#2=#3!#4{%
  \ifx#4!%
    \ifx#1\BitSet@Empty
    \else
      \ifx\BitSet@Empty#2%
      \else
        #30#2%
      \fi
    \fi
  \else
    \ifx#1\BitSet@Empty
      \BitSet@AfterFiFi\BitSet@Cleanup
    \else
      \ifx#10%
        \BitSet@AfterFiFiFi{%
          \BitSet@@Clear#2=#30!%
        }%
      \else
        #31%
        \BitSet@AfterFiFiFi{%
          \BitSet@@Clear#2=!%
        }%
      \fi
    \fi
  \BitSet@Fi
}
%    \end{macrocode}
%    \end{macro}
%
% \subsubsection{Set operation}
%
%    \begin{macro}{\BitSet@Set}
%    |#1|: BitSet\\
%    |#2|: plain and checked Index
%    \begin{macrocode}
\def\BitSet@Set#1#2{%
  \expandafter\edef\csname BS@#1\endcsname{%
    \expandafter\expandafter\expandafter\BitSet@@Set
    \csname BS@#1\expandafter\endcsname
    \expandafter\BitSet@Empty\expandafter=%
    \expandafter{\expandafter}\expandafter{\expandafter}%
    \romannumeral#2000!%
  }%
}
%    \end{macrocode}
%    \end{macro}
%    \begin{macro}{\BitSet@@Set}
%    \begin{macrocode}
\def\BitSet@@Set#1#2=#3#4#5{%
  #3#4%
  \ifx#5!%
    1#2%
  \else
    \ifx#1\BitSet@Empty
      0%
      \BitSet@AfterFiFi\BitSet@@@Set
    \else
      #1%
      \BitSet@@Set#2=%
    \fi
  \BitSet@Fi
}
%    \end{macrocode}
%    \end{macro}
%    \begin{macro}{\BitSet@@@Set}
%    \begin{macrocode}
\def\BitSet@@@Set#1{%
  \ifx#1!%
    1%
  \else
    0%
    \expandafter\BitSet@@@Set
  \fi
}
%    \end{macrocode}
%    \end{macro}
%
% \subsubsection{Flip operation}
%
%    \begin{macro}{\BitSet@Flip}
%    |#1|: BitSet\\
%    |#2|: plain and checked Index
%    \begin{macrocode}
\def\BitSet@Flip#1#2{%
  \edef\BitSet@Temp{%
    \expandafter\expandafter\expandafter\BitSet@@Flip
    \csname BS@#1\expandafter\endcsname
    \expandafter\BitSet@Empty\expandafter=\expandafter!%
    \romannumeral#2000!%
  }%
  \expandafter\let\csname BS@#1\expandafter\endcsname
  \ifx\BitSet@Temp\BitSet@Empty
    \BitSet@Zero
  \else
    \BitSet@Temp
  \fi
}
%    \end{macrocode}
%    \end{macro}
%    \begin{macro}{\BitSet@@Flip}
%    \begin{macrocode}
\def\BitSet@@Flip#1#2=#3!#4{%
  \ifx#4!%
    \ifx#11%
      \ifx\BitSet@Empty#2%
      \else
        #30#2%
      \fi
    \else
      #31#2%
    \fi
  \else
    \ifx#1\BitSet@Empty
      #30%
      \BitSet@AfterFiFi\BitSet@@@Set
    \else
      \ifx#10%
        \BitSet@AfterFiFiFi{%
          \BitSet@@Flip#2=#30!%
        }%
      \else
        #31%
        \BitSet@AfterFiFiFi{%
          \BitSet@@Flip#2=!%
        }%
      \fi
    \fi
  \BitSet@Fi
}
%    \end{macrocode}
%    \end{macro}
%
% \subsubsection{Range operators}
%
%    \begin{macro}{\bitsetClearRange}
%    \begin{macrocode}
\def\bitsetClearRange{%
  \BitSet@Range\BitSet@Clear
}
%    \end{macrocode}
%    \end{macro}
%    \begin{macro}{\bitsetSetRange}
%    \begin{macrocode}
\def\bitsetSetRange{%
  \BitSet@Range\BitSet@Set
}
%    \end{macrocode}
%    \end{macro}
%    \begin{macro}{\bitsetFlipRange}
%    \begin{macrocode}
\def\bitsetFlipRange{%
  \BitSet@Range\BitSet@Flip
}
%    \end{macrocode}
%    \end{macro}
%    \begin{macro}{\bitsetSetValueRange}
%    \begin{macrocode}
\def\bitsetSetValueRange#1#2#3#4{%
  \expandafter\expandafter\expandafter\BitSet@SetValueRange
  \intcalcNum{#4}!{#1}{#2}{#3}%
}
%    \end{macrocode}
%    \end{macro}
%    \begin{macro}{\BitSet@SetValueRange}
%    \begin{macrocode}
\def\BitSet@SetValueRange#1!#2#3#4{%
  \ifcase#1 %
    \BitSet@Range\BitSet@Clear{#2}{#3}{#4}%
  \or
    \BitSet@Range\BitSet@Set{#2}{#3}{#4}%
  \else
    \BitSet@ErrorInvalidBitValue{#1}%
  \fi
}
%    \end{macrocode}
%    \end{macro}
%
%    \begin{macro}{\BitSet@Range}
%    |#1|: clear/set/flip macro\\
%    |#2|: BitSet\\
%    |#3|: Index from\\
%    |#4|: Index to
%    \begin{macrocode}
\def\BitSet@Range#1#2#3#4{%
  \edef\BitSet@Temp{%
    \noexpand\BitSet@@Range\noexpand#1{#2}%
    \intcalcNum{#3}!\intcalcNum{#4}!%
  }%
  \BitSet@Temp
}
%    \end{macrocode}
%    \end{macro}
%    \begin{macro}{\BitSet@@Range}
%    |#1|: clear/set/flip macro\\
%    |#2|: BitSet\\
%    |#3|: Index from\\
%    |#4|: Index to
%    \begin{macrocode}
\def\BitSet@@Range#1#2#3!#4!{%
  \ifnum#3<0 %
    \BitSet@NegativeIndex#1{#2}#3!#4!0!#4!%
  \else
    \ifnum#4<0 %
      \BitSet@NegativeIndex#1{#2}#3!#4!#3!0!%
    \else
      \ifcase\intcalcCmp{#3}{#4} %
      \or
        \@PackageError{bitset}{%
          Wrong index numbers in range [#3..#4]\MessageBreak% hash-ok
          for clear/set/flip on bit set `#2'.\MessageBreak
          The lower index exceeds the upper index.\MessageBreak
          Canceling the operation as error recovery%
        }\@ehc
      \else
        \BitSet@@@Range#3!#4!#1{#2}%
      \fi
    \fi
  \fi
}
%    \end{macrocode}
%    \end{macro}
%    \begin{macro}{\BitSet@NegativeIndex}
%    \begin{macrocode}
\def\BitSet@NegativeIndex#1#2#3!#4!#5!#6!{%
  \@PackageError{bitset}{%
    Negative index in range [#3..#4]\MessageBreak % hash-ok
    for \string\bitset
    \ifx#1\BitSet@Clear
      Clear%
    \else
      \ifx#1\BitSet@Set
        Set%
      \else
        Flip%
      \fi
    \fi
    Range on bit set `#2'.\MessageBreak
    Using [#5..#6] as error recovery% hash-ok
  }\@ehc
  \BitSet@@Range#1{#2}#5!#6!%
}
%    \end{macrocode}
%    \end{macro}
%    \begin{macro}{\BitSet@@Range}
%    \begin{macrocode}
\def\BitSet@@@Range#1!#2!#3#4{%
  \ifnum#1<#2 %
    #3{#4}{#1}%
    \BitSet@AfterFi{%
      \expandafter\expandafter\expandafter\BitSet@@@Range
      \IntCalcInc#1!!#2!#3{#4}%
    }%
  \BitSet@Fi
}
%    \end{macrocode}
%    \end{macro}
%
% \subsection{Bit retrieval}\label{src:bitret}
%
% \subsubsection{\cs{bitsetGet}}
%
%    \begin{macro}{\bitsetGet}
%    \begin{macrocode}
\def\bitsetGet#1#2{%
  \number
  \expandafter\expandafter\expandafter\BitSet@Get
  \intcalcNum{#2}!{#1}%
}
%    \end{macrocode}
%    \end{macro}
%    \begin{macro}{\BitSet@Get}
%    |#1|: plain index\\
%    |#2|: BitSet
%    \begin{macrocode}
\def\BitSet@Get#1!#2{%
  \ifnum#1<0 %
    \BitSet@AfterFi{%
      0 \BitSetError:NegativeIndex%
    }%
  \else
    \BitSet@IfUndefined{#2}{0}{%
      \expandafter\expandafter\expandafter\BitSet@@Get
      \csname BS@#2\expandafter\endcsname
      \expandafter!\expandafter=%
      \expandafter{\expandafter}\expandafter{\expandafter}%
      \romannumeral\intcalcNum{#1}000!%
    }%
    \expandafter\BitSet@Space
  \BitSet@Fi
}
%    \end{macrocode}
%    \end{macro}
%    \begin{macro}{\BitSet@@Get}
%    \begin{macrocode}
\def\BitSet@@Get#1#2=#3#4#5{%
  #3#4%
  \ifx#5!%
    \ifx#1!%
      0%
    \else
      #1%
    \fi
  \else
    \ifx#1!%
      0%
      \BitSet@AfterFiFi\BitSet@Cleanup
    \else
      \BitSet@@Get#2=%
    \fi
  \BitSet@Fi
}
%    \end{macrocode}
%    \end{macro}
%
% \subsubsection{\cs{bitsetNextClearBit}, \cs{bitsetNextSetBit}}
%
%    \begin{macro}{\bitsetNextClearBit}
%    \begin{macrocode}
\def\bitsetNextClearBit#1#2{%
  \number
  \expandafter\expandafter\expandafter\BitSet@NextClearBit
  \intcalcNum{#2}!{#1} %
}
%    \end{macrocode}
%    \end{macro}
%    \begin{macro}{\BitSet@NextClearBit}
%    |#1|: Index\\
%    |#2|: BitSet
%    \begin{macrocode}
\def\BitSet@NextClearBit#1!#2{%
  \ifnum#1<0 %
    \BitSet@NextClearBit0!{#2}%
    \BitSet@AfterFi{%
      \expandafter\BitSet@Space
      \expandafter\BitSetError:NegativeIndex\romannumeral0%
    }%
  \else
    \bitsetIsEmpty{#2}{#1}{%
      \expandafter\BitSet@Skip
      \number#1\expandafter\expandafter\expandafter!%
      \csname BS@#2\endcsname!!!!!!!!!=%
      {\BitSet@@NextClearBit#1!}%
    }%
  \BitSet@Fi
}
%    \end{macrocode}
%    \end{macro}
%    \begin{macro}{\BitSet@@NextClearBit}
%    |#1|: index for next bit in |#2|\\
%    |#2|: next bit
%    \begin{macrocode}
\def\BitSet@@NextClearBit#1!#2{%
  \ifx#2!%
    #1%
  \else
    \ifx#20%
      #1%
      \BitSet@AfterFiFi\BitSet@Cleanup
    \else
      \BitSet@AfterFiFi{%
        \expandafter\expandafter\expandafter\BitSet@@NextClearBit
        \IntCalcInc#1!!%
      }%
    \fi
  \BitSet@Fi
}
%    \end{macrocode}
%    \end{macro}
%
%    \begin{macro}{\bitsetNextSetBit}
%    \begin{macrocode}
\def\bitsetNextSetBit#1#2{%
  \number
  \expandafter\expandafter\expandafter\BitSet@NextSetBit
  \intcalcNum{#2}!{#1} %
}
%    \end{macrocode}
%    \end{macro}
%    \begin{macro}{\BitSet@NextSetBit}
%    |#1|: Index\\
%    |#2|: BitSet
%    \begin{macrocode}
\def\BitSet@NextSetBit#1!#2{%
  \ifnum#1<0 %
    \BitSet@NextSetBit0!{#2}%
    \BitSet@AfterFi{%
      \expandafter\BitSet@Space
      \expandafter\BitSetError:NegativeIndex\romannumeral0%
    }%
  \else
    \bitsetIsEmpty{#2}{-1}{%
      \expandafter\BitSet@Skip
      \number#1\expandafter\expandafter\expandafter!%
      \csname BS@#2\endcsname!!!!!!!!!=%
      {\BitSet@@NextSetBit#1!}%
    }%
  \BitSet@Fi
}
%    \end{macrocode}
%    \end{macro}
%    \begin{macro}{\BitSet@@NextSetBit}
%    |#1|: index for next bit in |#2|\\
%    |#2|: next bit
%    \begin{macrocode}
\def\BitSet@@NextSetBit#1!#2{%
  \ifx#2!%
    -1%
  \else
    \ifx#21%
      #1%
      \BitSet@AfterFiFi\BitSet@Cleanup
    \else
      \BitSet@AfterFiFi{%
        \expandafter\expandafter\expandafter\BitSet@@NextSetBit
        \IntCalcInc#1!!%
      }%
    \fi
  \BitSet@Fi
}
%    \end{macrocode}
%    \end{macro}
%    \begin{macro}{\BitSet@Cleanup}
%    \begin{macrocode}
\def\BitSet@Cleanup#1!{}
%    \end{macrocode}
%    \end{macro}
%
%    \begin{macro}{\BitSet@Skip}
%    |#1|: number of bits to skip\\
%    |#2|: bits\\
%    |#3|: continuation code
%    \begin{macrocode}
\def\BitSet@Skip#1!#2{%
  \ifx#2!%
    \BitSet@AfterFi{%
      \BitSet@SkipContinue%
    }%
  \else
    \ifcase#1 %
      \BitSet@AfterFiFi{%
        \BitSet@SkipContinue#2%
      }%
    \or
      \BitSet@AfterFiFi\BitSet@SkipContinue
    \or
      \BitSet@AfterFiFi{%
        \expandafter\BitSet@SkipContinue\BitSet@Gobble
      }%
    \else
      \ifnum#1>8 %
        \BitSet@AfterFiFiFi{%
          \expandafter\BitSet@Skip
          \number\IntCalcSub#1!8!\expandafter!%
          \BitSet@GobbleSeven
        }%
      \else
        \BitSet@AfterFiFiFi{%
          \expandafter\expandafter\expandafter\BitSet@Skip
          \IntCalcDec#1!!%
        }%
      \fi
    \fi
  \BitSet@Fi
}
%    \end{macrocode}
%    \end{macro}
%    \begin{macro}{\BitSet@SkipContinue}
%    |#1|: remaining bits\\
%    |#2|: continuation code
%    \begin{macrocode}
\def\BitSet@SkipContinue#1!#2=#3{%
  #3#1!%
}
%    \end{macrocode}
%    \end{macro}
%    \begin{macro}{\BitSet@GobbleSeven}
%    \begin{macrocode}
\def\BitSet@GobbleSeven#1#2#3#4#5#6#7{}
%    \end{macrocode}
%    \end{macro}
%
% \subsubsection{\cs{bitsetGetSetBitList}}
%
%    \begin{macro}{\bitsetGetSetBitList}
%    It's just a wrapper for \cs{bitsetNextSetBit}.
%    \begin{macrocode}
\def\bitsetGetSetBitList#1{%
  \romannumeral0%
  \bitsetIsEmpty{#1}{ }{%
    \expandafter\BitSet@GetSetBitList
    \number\BitSet@NextSetBit0!{#1}!{#1}{}!%
  }%
}
%    \end{macrocode}
%    \end{macro}
%    \begin{macro}{\BitSet@GetSetBitList}
%    |#1|: found index\\
%    |#2|: BitSet\\
%    |#3|: comma
%    |#4|: result
%    \begin{macrocode}
\def\BitSet@GetSetBitList#1!#2#3#4!{%
  \ifnum#1<0 %
    \BitSet@AfterFi{ #4}%
  \else
    \BitSet@AfterFi{%
      \expandafter\BitSet@GetSetBitList\number
      \expandafter\expandafter\expandafter\BitSet@NextSetBit
      \IntCalcInc#1!!{#2}!{#2},#4#3#1!%
    }%
  \BitSet@Fi
}
%    \end{macrocode}
%    \end{macro}
%
% \subsection{Bit set properties}\label{src:prop}
%
%    \begin{macro}{\bitsetSize}
%    \begin{macrocode}
\def\bitsetSize#1{%
  \number
  \BitSet@IfUndefined{#1}{0 }{%
    \expandafter\expandafter\expandafter\BitSet@Size
    \expandafter\expandafter\expandafter1%
    \expandafter\expandafter\expandafter!%
    \csname BS@#1\endcsname!0!%
  }%
}
%    \end{macrocode}
%    \end{macro}
%    \begin{macro}{\BitSet@Size}
%    |#1|: counter\\
%    |#2#3|: bits\\
%    |#4|: result
%    \begin{macrocode}
\def\BitSet@Size#1!#2#3!#4!{%
  \ifx#21%
    \ifx\\#3\\%
      \BitSet@AfterFiFi{#1 }%
    \else
      \BitSet@AfterFiFi{%
        \expandafter\expandafter\expandafter\BitSet@Size
        \IntCalcInc#1!!#3!#1!%
      }%
    \fi
  \else
    \ifx\\#3\\%
      \BitSet@AfterFiFi{#4 }%
    \else
      \BitSet@AfterFiFi{%
        \expandafter\expandafter\expandafter\BitSet@Size
        \IntCalcInc#1!!#3!#4!%
      }%
    \fi
  \fi
  \BitSet@Fi
}
%    \end{macrocode}
%    \end{macro}
%
%    \begin{macro}{\bitsetCardinality}
%    \begin{macrocode}
\def\bitsetCardinality#1{%
  \number
  \BitSet@IfUndefined{#1}{0 }{%
    \expandafter\expandafter\expandafter\BitSet@Cardinality
    \expandafter\expandafter\expandafter0%
    \expandafter\expandafter\expandafter!%
    \csname BS@#1\endcsname!%
  }%
}
%    \end{macrocode}
%    \end{macro}
%    \begin{macro}{\BitSet@Cardinality}
%    |#1|: result\\
%    |#2#3|: bits
%    \begin{macrocode}
\def\BitSet@Cardinality#1!#2#3!{%
  \ifx#21%
    \ifx\\#3\\%
      \BitSet@AfterFiFi{\IntCalcInc#1! }%
    \else
      \BitSet@AfterFiFi{%
        \expandafter\expandafter\expandafter\BitSet@Cardinality
        \IntCalcInc#1!!#3!%
      }%
    \fi
  \else
    \ifx\\#3\\%
      \BitSet@AfterFiFi{#1 }%
    \else
      \BitSet@AfterFiFi{%
        \BitSet@Cardinality#1!#3!%
      }%
    \fi
  \fi
  \BitSet@Fi
}
%    \end{macrocode}
%    \end{macro}
%
% \subsection{Queries}\label{src:query}
%
%    \begin{macro}{\bitsetIsDefined}
%    \begin{macrocode}
\def\bitsetIsDefined#1{%
  \BitSet@IfUndefined{#1}%
  \BitSet@SecondOfTwo
  \BitSet@FirstOfTwo
}
%    \end{macrocode}
%    \end{macro}
%    \begin{macro}{\bitsetIsEmpty}
%    \begin{macrocode}
\def\bitsetIsEmpty#1{%
  \BitSet@IfUndefined{#1}\BitSet@FirstOfTwo{%
    \expandafter\ifx\csname BS@#1\endcsname\BitSet@Zero
      \expandafter\BitSet@FirstOfTwo
    \else
      \expandafter\BitSet@SecondOfTwo
    \fi
  }%
}
%    \end{macrocode}
%    \end{macro}
%    \begin{macro}{\BitSet@Zero}
%    \begin{macrocode}
\def\BitSet@Zero{0}
%    \end{macrocode}
%    \end{macro}
%
%    \begin{macro}{\bitsetQuery}
%    \begin{macrocode}
\def\bitsetQuery#1#2{%
  \ifnum\bitsetGet{#1}{#2}=1 %
    \expandafter\BitSet@FirstOfTwo
  \else
    \expandafter\BitSet@SecondOfTwo
  \fi
}
%    \end{macrocode}
%    \end{macro}
%
%    \begin{macro}{\bitsetEquals}
%    \begin{macrocode}
\def\bitsetEquals#1#2{%
  \BitSet@IfUndefined{#1}{%
    \BitSet@IfUndefined{#2}\BitSet@FirstOfTwo\BitSet@SecondOfTwo
  }{%
    \BitSet@IfUndefined{#2}\BitSet@SecondOfTwo{%
      \expandafter\ifx\csname BS@#1\expandafter\endcsname
                      \csname BS@#2\endcsname
        \expandafter\BitSet@FirstOfTwo
      \else
        \expandafter\BitSet@SecondOfTwo
      \fi
    }%
  }%
}
%    \end{macrocode}
%    \end{macro}
%
%    \begin{macro}{\bitsetIntersects}
%    \begin{macrocode}
\def\bitsetIntersects#1#2{%
  \bitsetIsEmpty{#1}\BitSet@SecondOfTwo{%
    \bitsetIsEmpty{#2}\BitSet@SecondOfTwo{%
      \expandafter\expandafter\expandafter\BitSet@Intersects
      \csname BS@#1\expandafter\expandafter\expandafter\endcsname
      \expandafter\expandafter\expandafter!%
      \csname BS@#2\endcsname!%
    }%
  }%
}
%    \end{macrocode}
%    \end{macro}
%    \begin{macro}{\BitSet@Intersects}
%    \begin{macrocode}
\def\BitSet@Intersects#1#2!#3#4!{%
  \ifnum#1#3=11 %
    \BitSet@AfterFi\BitSet@FirstOfTwo
  \else
    \ifx\\#2\\%
      \BitSet@AfterFiFi\BitSet@SecondOfTwo
    \else
      \ifx\\#4\\%
        \BitSet@AfterFiFiFi\BitSet@SecondOfTwo
      \else
        \BitSet@AfterFiFiFi{%
          \BitSet@Intersects#2!#4!%
        }%
      \fi
    \fi
  \BitSet@Fi
}
%    \end{macrocode}
%    \end{macro}
%
%    \begin{macrocode}
\BitSet@AtEnd%
%</package>
%    \end{macrocode}
%
% \section{Test}
%
% \subsection{Catcode checks for loading}
%
%    \begin{macrocode}
%<*test1>
%    \end{macrocode}
%    \begin{macrocode}
\catcode`\{=1 %
\catcode`\}=2 %
\catcode`\#=6 %
\catcode`\@=11 %
\expandafter\ifx\csname count@\endcsname\relax
  \countdef\count@=255 %
\fi
\expandafter\ifx\csname @gobble\endcsname\relax
  \long\def\@gobble#1{}%
\fi
\expandafter\ifx\csname @firstofone\endcsname\relax
  \long\def\@firstofone#1{#1}%
\fi
\expandafter\ifx\csname loop\endcsname\relax
  \expandafter\@firstofone
\else
  \expandafter\@gobble
\fi
{%
  \def\loop#1\repeat{%
    \def\body{#1}%
    \iterate
  }%
  \def\iterate{%
    \body
      \let\next\iterate
    \else
      \let\next\relax
    \fi
    \next
  }%
  \let\repeat=\fi
}%
\def\RestoreCatcodes{}
\count@=0 %
\loop
  \edef\RestoreCatcodes{%
    \RestoreCatcodes
    \catcode\the\count@=\the\catcode\count@\relax
  }%
\ifnum\count@<255 %
  \advance\count@ 1 %
\repeat

\def\RangeCatcodeInvalid#1#2{%
  \count@=#1\relax
  \loop
    \catcode\count@=15 %
  \ifnum\count@<#2\relax
    \advance\count@ 1 %
  \repeat
}
\def\RangeCatcodeCheck#1#2#3{%
  \count@=#1\relax
  \loop
    \ifnum#3=\catcode\count@
    \else
      \errmessage{%
        Character \the\count@\space
        with wrong catcode \the\catcode\count@\space
        instead of \number#3%
      }%
    \fi
  \ifnum\count@<#2\relax
    \advance\count@ 1 %
  \repeat
}
\def\space{ }
\expandafter\ifx\csname LoadCommand\endcsname\relax
  \def\LoadCommand{\input bitset.sty\relax}%
\fi
\def\Test{%
  \RangeCatcodeInvalid{0}{47}%
  \RangeCatcodeInvalid{58}{64}%
  \RangeCatcodeInvalid{91}{96}%
  \RangeCatcodeInvalid{123}{255}%
  \catcode`\@=12 %
  \catcode`\\=0 %
  \catcode`\%=14 %
  \LoadCommand
  \RangeCatcodeCheck{0}{36}{15}%
  \RangeCatcodeCheck{37}{37}{14}%
  \RangeCatcodeCheck{38}{47}{15}%
  \RangeCatcodeCheck{48}{57}{12}%
  \RangeCatcodeCheck{58}{63}{15}%
  \RangeCatcodeCheck{64}{64}{12}%
  \RangeCatcodeCheck{65}{90}{11}%
  \RangeCatcodeCheck{91}{91}{15}%
  \RangeCatcodeCheck{92}{92}{0}%
  \RangeCatcodeCheck{93}{96}{15}%
  \RangeCatcodeCheck{97}{122}{11}%
  \RangeCatcodeCheck{123}{255}{15}%
  \RestoreCatcodes
}
\Test
\csname @@end\endcsname
\end
%    \end{macrocode}
%    \begin{macrocode}
%</test1>
%    \end{macrocode}
%
% \subsection{Macro tests}
%
% \subsubsection{Preamble}
%
%    \begin{macrocode}
%<*test2>
\NeedsTeXFormat{LaTeX2e}
\nofiles
\documentclass{article}
\makeatletter
%<*noetex>
\let\SavedNumexpr\numexpr
\let\SavedIfcsname\ifcsname
\let\SavedCurrentgrouplevel\currentgrouplevel
\def\ETeXDisable{%
  \let\ifcsname\@undefined
  \let\numexpr\@undefined
  \let\currentgrouplevel\@undefined
}
\ETeXDisable
%</noetex>
\makeatletter
\chardef\BitSet@TestMode=1 %
\makeatother
\usepackage{bitset}[2011/01/30]
%<*noetex>
\def\ETeXEnable{%
  \let\numexpr\SavedNumexpr
  \let\ifcsname\SavedIfcsname
  \let\currentgrouplevel\SavedCurrentgrouplevel
}
\ETeXEnable
%</noetex>
\usepackage{qstest}
\IncludeTests{*}
\LogTests{log}{*}{*}
\makeatletter
%    \end{macrocode}
%
% \subsubsection{Time}
%
%    \begin{macrocode}
\begingroup\expandafter\expandafter\expandafter\endgroup
\expandafter\ifx\csname pdfresettimer\endcsname\relax
\else
  \newcount\SummaryTime
  \newcount\TestTime
  \SummaryTime=\z@
  \newcommand*{\PrintTime}[2]{%
    \typeout{%
      [Time #1: \strip@pt\dimexpr\number#2sp\relax\space s]%
    }%
  }%
  \newcommand*{\StartTime}[1]{%
    \renewcommand*{\TimeDescription}{#1}%
    \pdfresettimer
  }%
  \newcommand*{\TimeDescription}{}%
  \newcommand*{\StopTime}{%
    \TestTime=\pdfelapsedtime
    \global\advance\SummaryTime\TestTime
    \PrintTime\TimeDescription\TestTime
  }%
  \let\saved@qstest\qstest
  \let\saved@endqstest\endqstest
  \def\qstest#1#2{%
    \saved@qstest{#1}{#2}%
    \StartTime{#1}%
  }%
  \def\endqstest{%
    \StopTime
    \saved@endqstest
  }%
  \AtEndDocument{%
    \PrintTime{summary}\SummaryTime
  }%
\fi
%    \end{macrocode}
%
% \subsubsection{Detection of unwanted space}
%
%    \begin{macrocode}
\let\orig@qstest\qstest
\let\orig@endqstest\endqstest
\def\qstest#1#2{%
  \orig@qstest{#1}{#2}%
  \setbox0\hbox\bgroup\begingroup\ignorespaces
}
\def\endqstest{%
  \endgroup\egroup
  \Expect*{\the\wd0}{0.0pt}%
  \orig@endqstest
}
%    \end{macrocode}
%
% \subsubsection{Test macros}
%
%    \begin{macrocode}
\newcounter{Test}

\def\TestError#1#2{%
  \begingroup
    \setcounter{Test}{0}%
    \sbox0{%
      \def\@PackageError##1##2##3{%
        \stepcounter{Test}%
        \begingroup
          \let\MessageBreak\relax
%<*noetex>
          \ETeXEnable
%</noetex>
          \Expect{##1}{bitset}%
          \Expect*{##2}*{#1}%
        \endgroup
      }%
%<*noetex>
      \ETeXDisable
%</noetex>
      #2%
    }%
    \Expect*{\theTest}{1}%
    \Expect*{\the\wd0}{0.0pt}%
  \endgroup
}

\def\TestErrorNegativeIndex#1#2{%
  \TestError{Invalid negative index (#1)}{#2}%
}

\def\TestGetterUndefined#1{%
  \CheckUndef{dummy}%
  \expandafter\expandafter\expandafter\Expect
  \expandafter\expandafter\expandafter{#1{dummy}}{0}%
}

\def\ExpectBitSet#1#2{%
  \expandafter\expandafter\expandafter\Expect
  \expandafter\expandafter\expandafter
  {\csname BS@#1\endcsname}*{#2}%
}
\def\Check#1#2{%
  \ExpectBitSet{#1}{#2}%
}
\def\CheckUndef#1{%
  \begingroup
    \Expect*{%
      \expandafter
      \ifx\csname BS@#1\endcsname\relax true\else false\fi
    }{true}%
  \endgroup
}
\def\RevCheck#1#2{%
  \ExpectBitSet{#1}{\Reverse#2!!}%
}
\def\Set#1#2{%
  \expandafter\def\csname BS@#1\endcsname{#2}%
}
\def\RevSet#1#2{%
  \expandafter\edef\csname BS@#1\endcsname{%
    \Reverse#2!!%
  }%
}
\def\Reverse#1#2!#3!{%
  \ifx\\#2\\%
    #1#3%
    \expandafter\@gobble
  \else
    \expandafter\@firstofone
  \fi
  {\Reverse#2!#1#3!}%
}
%    \end{macrocode}
%
% \subsubsection{Test sets}
%
%    \begin{macrocode}
\begin{qstest}{Let}{Let}
  \CheckUndef{abc}%
  \CheckUndef{xyz}%
  \bitsetLet{xyz}{abc}%
  \CheckUndef{abc}%
  \Check{xyz}{0}%
  \Set{abc}{1}%
  \Check{abc}{1}%
  \Check{xyz}{0}%
  \bitsetLet{xyz}{abc}%
  \Check{abc}{1}%
  \Check{xyz}{1}%
  \Set{xyz}{11}%
  \Check{abc}{1}%
  \Check{xyz}{11}%
\end{qstest}

\begin{qstest}{Reset}{Reset}
  \bitsetReset{xyz}%
  \Check{xyz}{0}%
  \bitsetReset{abc}%
  \Check{abc}{0}%
  \Set{abc}{10101}%
  \bitsetReset{abc}%
  \Check{abc}{0}%
\end{qstest}

\begin{qstest}{Get/Query}{Get/Query}
  \expandafter\expandafter\expandafter\Expect
  \expandafter\expandafter\expandafter{%
    \bitsetGet{dummy}{0}%
  }{0}%
  \begingroup
    \expandafter\def\csname BitSetError:NegativeIndex\endcsname{}%
    \Set{abc}{1}%
    \Expect*{\bitsetQuery{abc}{-1}{true}{false}}{false}%
  \endgroup
  \def\Test#1#2#3{%
    \Set{abc}{#1}%
    \expandafter\expandafter\expandafter\Expect
    \expandafter\expandafter\expandafter{\bitsetGet{abc}{#2}}{#3}%
    \Expect*{\bitsetQuery{abc}{#2}{true}{false}}%
           *{\ifcase#3 false\or true\else error\fi}%
  }%
  \Test{1}{100}{0}%
  \Test{0}{0}{0}%
  \Test{1}{0}{1}%
  \Test{11}{1}{1}%
  \Test{111}{1}{1}%
  \Test{101}{1}{0}%
  \Test{101}{2}{1}%
  \Test{10100110011}{10}{1}%
\end{qstest}

\begin{qstest}{Size}{Size}
  \TestGetterUndefined\bitsetSize
  \def\Test#1#2{%
    \Set{abc}{#1}%
    \expandafter\expandafter\expandafter\Expect
    \expandafter\expandafter\expandafter{\bitsetSize{abc}}{#2}%
  }%
  \Test{0}{0}%
  \Test{1}{1}%
  \Test{00}{0}%
  \Test{0000000}{0}%
  \Test{10}{1}%
  \Test{01}{2}%
  \Test{11}{2}%
  \Test{010}{2}%
  \Test{011}{3}%
  \Test{100110011}{9}%
  \Test{0000011111000001111100000}{20}%
  \Test{000000000000000000000000011111111111111111111}{45}%
\end{qstest}

\begin{qstest}{Cardinality}{Cardinality}
  \TestGetterUndefined\bitsetCardinality
  \def\Test#1#2{%
    \Set{abc}{#1}%
    \expandafter\expandafter\expandafter\Expect
    \expandafter\expandafter\expandafter{%
      \bitsetCardinality{abc}%
    }{#2}%
  }%
  \Test{0}{0}%
  \Test{1}{1}%
  \Test{00}{0}%
  \Test{0000000}{0}%
  \Test{10}{1}%
  \Test{01}{1}%
  \Test{11}{2}%
  \Test{010}{1}%
  \Test{011}{2}%
  \Test{100110011}{5}%
  \Test{0000011111000001111100000}{10}%
  \Test{000000000000000000000000011111111111111111111}{20}%
\end{qstest}

\begin{qstest}{NextClearBit/NextSetBit}{NextClearBit/NextSetBit}
  \def\Test#1#2{%
    \expandafter\expandafter\expandafter\Expect
    \expandafter\expandafter\expandafter{%
      \TestOp{abc}{#1}%
    }{#2}%
  }%
  \def\Clear{\let\TestOp\bitsetNextClearBit}%
  \def\Set{\let\TestOp\bitsetNextSetBit}%
  \begingroup
    \catcode`\:=11 %
    \bitsetSetBin{abc}{1}%
    \Clear
    \Test{-1}{1\BitSetError:NegativeIndex}%
    \Set
    \Test{-1}{0\BitSetError:NegativeIndex}%
  \endgroup
  \let\BS@abc\@undefined
  \Clear
  \Test{0}{0}%
  \Test{1}{1}%
  \Test{2}{2}%
  \Test{100}{100}%
  \Set
  \Test{0}{-1}%
  \Test{1}{-1}%
  \Test{100}{-1}%
  \bitsetReset{abc}%
  \Clear
  \Test{0}{0}%
  \Test{1}{1}%
  \Test{2}{2}%
  \Test{100}{100}%
  \Set
  \Test{0}{-1}%
  \Test{1}{-1}%
  \Test{100}{-1}%
  \bitsetSetBin{abc}{1}%
  \Clear
  \Test{0}{1}%
  \Test{1}{1}%
  \Test{2}{2}%
  \Test{100}{100}%
  \Set
  \Test{0}{0}%
  \Test{1}{-1}%
  \Test{100}{-1}%
  \bitsetSetBin{abc}{111000111000111000111}%
  \Clear
  \Test{0}{3}%
  \Test{1}{3}%
  \Test{2}{3}%
  \Test{3}{3}%
  \Test{4}{4}%
  \Test{5}{5}%
  \Test{6}{9}%
  \Test{7}{9}%
  \Test{8}{9}%
  \Test{9}{9}%
  \Test{10}{10}%
  \Test{11}{11}%
  \Test{12}{15}%
  \Test{13}{15}%
  \Test{14}{15}%
  \Test{15}{15}%
  \Test{16}{16}%
  \Test{17}{17}%
  \Test{18}{21}%
  \Test{19}{21}%
  \Test{20}{21}%
  \Test{21}{21}%
  \Test{22}{22}%
  \Test{100}{100}%
  \Set
  \Test{0}{0}%
  \Test{1}{1}%
  \Test{2}{2}%
  \Test{3}{6}%
  \Test{4}{6}%
  \Test{5}{6}%
  \Test{6}{6}%
  \Test{7}{7}%
  \Test{8}{8}%
  \Test{9}{12}%
  \Test{10}{12}%
  \Test{11}{12}%
  \Test{12}{12}%
  \Test{13}{13}%
  \Test{14}{14}%
  \Test{15}{18}%
  \Test{16}{18}%
  \Test{17}{18}%
  \Test{18}{18}%
  \Test{19}{19}%
  \Test{20}{20}%
  \Test{21}{-1}%
  \Test{22}{-1}%
  \Test{100}{-1}%
  \bitsetSetBin{abc}{1111111}%
  \Clear
  \Test{6}{7}%
  \Test{7}{7}%
  \Test{8}{8}%
  \Test{100}{100}%
  \Set
  \Test{6}{6}%
  \Test{7}{-1}%
  \Test{8}{-1}%
  \Test{100}{-1}%
  \bitsetSetBin{abc}{11111111}%
  \Clear
  \Test{7}{8}%
  \Test{8}{8}%
  \Test{9}{9}%
  \Test{100}{100}%
  \Set
  \Test{7}{7}%
  \Test{8}{-1}%
  \Test{9}{-1}%
  \Test{100}{-1}%
  \bitsetSetBin{abc}{111111111}%
  \Clear
  \Test{8}{9}%
  \Test{9}{9}%
  \Test{10}{10}%
  \Test{100}{100}%
  \Set
  \Test{8}{8}%
  \Test{9}{-1}%
  \Test{10}{-1}%
  \Test{100}{-1}%
  \bitsetSetBin{abc}{1111111111}%
  \Clear
  \Test{9}{10}%
  \Test{10}{10}%
  \Test{11}{11}%
  \Test{100}{100}%
  \Set
  \Test{9}{9}%
  \Test{10}{-1}%
  \Test{11}{-1}%
  \Test{100}{-1}%
\end{qstest}

\begin{qstest}{GetSetBitList}{GetSetBitList}
  \let\BS@abc\@undefined
  \expandafter\expandafter\expandafter\Expect
  \expandafter\expandafter\expandafter{%
    \bitsetGetSetBitList{abc}%
  }{}%
  \def\Test#1#2{%
    \bitsetSetBin{abc}{#1}%
    \expandafter\expandafter\expandafter\Expect
    \expandafter\expandafter\expandafter{%
      \bitsetGetSetBitList{abc}%
    }{#2}%
  }%
  \Test{0}{}%
  \Test{1}{0}%
  \Test{10}{1}%
  \Test{11}{0,1}%
  \Test{10110100}{2,4,5,7}%
  \Test{101101001010011}{0,1,4,6,9,11,12,14}%
\end{qstest}

\begin{qstest}{GetDec}{GetDec}
  \TestGetterUndefined\bitsetGetDec
  \def\Test#1#2{%
    \RevSet{abc}{#1}%
%<*noetex>
    \begingroup\expandafter\expandafter\expandafter\endgroup
%</noetex>
    \expandafter\expandafter\expandafter\Expect
    \expandafter\expandafter\expandafter{%
      \bitsetGetDec{abc}%
    }{#2}%
  }%
  \Test{0}{0}%
  \Test{1}{1}%
  \Test{10}{2}%
  \Test{11}{3}%
  \Test{100}{4}%
  \Test{101}{5}%
  \Test{110}{6}%
  \Test{111}{7}%
  \Test{1000}{8}%
  \Test{000111}{7}%
  \Test{1111111111111111%
        111111111111111}{2147483647}%
  \Test{0001111111111111111%
        111111111111111}{2147483647}%
  \Test{1000000000000000%
        0000000000000000}{2147483648}%
  \Test{1000000000000000%
        00000000000000000}{4294967296}%
  \Test{0001000000000000000%
        00000000000000000}{4294967296}%
  \Test{1100000000000000%
        00000000000000011}{6442450947}%
\end{qstest}

\begin{qstest}{Clear}{Clear}
  \def\Test#1#2#3{%
    \RevSet{abc}{#1}%
    \bitsetClear{abc}{#2}%
    \Expect*{\BS@abc}*{\Reverse#3!!}%
  }%
  \bitsetClear{abc}{2}%
  \RevCheck{abc}{0}%
  \TestErrorNegativeIndex{-1}{\bitsetClear{abc}{-1}}%
  \RevCheck{abc}{0}%
  \Test{0}{0}{0}%
  \Test{1}{0}{0}%
  \Test{111}{1}{101}%
  \Test{111}{30}{111}%
  \Test{0000111}{5}{0000111}% 111 would also be ok
  \Test{10000111}{5}{10000111}%
  \Test{1001001}{3}{1000001}%
\end{qstest}

\begin{qstest}{Set}{Set}
  \def\Test#1#2#3{%
    \RevSet{abc}{#1}%
    \bitsetSet{abc}{#2}%
    \Expect*{\BS@abc}*{\Reverse#3!!}%
  }%
  \bitsetSet{abc}{2}%
  \RevCheck{abc}{100}%
  \TestErrorNegativeIndex{-1}{\bitsetSet{abc}{-1}}%
  \RevCheck{abc}{100}%
  \Test{0}{0}{1}%
  \Test{1}{0}{1}%
  \Test{100}{1}{110}%
  \Test{111}{1}{111}%
  \Test{11}{1}{11}%
  \Test{11}{2}{111}%
  \Test{11}{3}{1011}%
  \Test{111}{10}{10000000111}%
  \Test{0000111}{5}{0100111}% 100111 would also be ok
  \Test{10000111}{5}{10100111}%
  \Test{1000001}{3}{1001001}%
  \Test{1001001}{3}{1001001}%
\end{qstest}

\begin{qstest}{Flip}{Flip}
  \def\Test#1#2#3{%
    \RevSet{abc}{#1}%
    \bitsetFlip{abc}{#2}%
    \Expect*{\BS@abc}*{\Reverse#3!!}%
  }%
  \bitsetFlip{abc}{2}%
  \RevCheck{abc}{100}%
  \TestErrorNegativeIndex{-1}{\bitsetFlip{abc}{-1}}%
  \RevCheck{abc}{100}%
  \Test{0}{0}{1}%
  \Test{1}{0}{0}%
  \Test{0}{2}{100}%
  \Test{100}{1}{110}%
  \Test{111}{1}{101}%
  \Test{11}{1}{1}%
  \Test{11}{2}{111}%
  \Test{11}{3}{1011}%
  \Test{111}{10}{10000000111}%
  \Test{0000111}{5}{0100111}% 100111 would also be ok
  \Test{10000111}{5}{10100111}%
  \Test{1000001}{3}{1001001}%
  \Test{1001001}{3}{1000001}%
  \Test{11111}{2}{11011}%
\end{qstest}

\begin{qstest}{SetValue}{SetValue}
  \def\Test#1#2{%
    \TestError{Invalid bit value (#2) not in range 0..1}{%
      \bitsetSetValue{abc}{#1}{#2}%
    }%
  }%
  \Test{0}{-1}%
  \Test{0}{2}%
  \Test{0}{10}%
  \def\Test#1#2#3{%
    \let\BS@abc\@undefined
    \bitsetSetValue{abc}{#1}{#2}%
    \bitsetSetBin{result}{#3}%
    \Expect*{\BS@abc}*{\BS@result}%
  }%
  \Test{0}{0}{0}%
  \Test{0}{1}{1}%
  \Test{1}{0}{0}%
  \Test{1}{1}{10}%
  \def\Test#1#2#3#4{%
    \bitsetSetBin{abc}{#1}%
    \bitsetSetBin{result}{#4}%
    \bitsetSetValue{abc}{#2}{#3}%
    \Expect*{\BS@abc}*{\BS@result}%
  }%
  \Test{0}{0}{0}{0}%
  \Test{0}{0}{0}{0}%
  \Test{0}{0}{1}{1}%
  \Test{0}{1}{0}{0}%
  \Test{0}{1}{1}{10}%
  \Test{1010}{2}{1}{1110}%
  \Test{1010}{4}{1}{11010}%
  \Test{1010}{6}{1}{1001010}%
  \Test{1010}{1}{0}{1000}%
  \Test{1010}{2}{0}{1010}%
  \Test{1010}{3}{0}{10}%
  \Test{1010}{4}{0}{1010}%
  \Test{1010}{6}{0}{1010}%
  \Test{1010}{2}{\csname iffalse\endcsname 0\else 1\fi}{1110}%
  \Test{1010}{1}{\csname iffalse\endcsname 1\else 0\fi}{1000}%
\end{qstest}

\begin{qstest}{IsDefined}{IsDefined}
  \let\BS@abc\@undefined
  \Expect*{\bitsetIsDefined{abc}{true}{false}}{false}%
  \bitsetReset{abc}%
  \Expect*{\bitsetIsDefined{abc}{true}{false}}{true}%
\end{qstest}

\begin{qstest}{IsEmpty}{IsEmpty}
  \let\BS@abc\@undefined
  \Expect*{\bitsetIsEmpty{abc}{true}{false}}{true}%
  \bitsetReset{abc}%
  \Expect*{\bitsetIsEmpty{abc}{true}{false}}{true}%
  \bitsetSet{abc}{1}%
  \Expect*{\bitsetIsEmpty{abc}{true}{false}}{false}%
\end{qstest}

\begin{qstest}{Equals}{Equals}
  \def\Test#1#2#3{%
    \Expect*{\bitsetEquals{#1}{#2}{true}{false}}{#3}%
  }%
  \let\BS@abc\@undefined
  \Test{abc}{abc}{true}%
  \Test{abc}{foo}{true}%
  \Test{foo}{abc}{true}%
  \bitsetReset{abc}%
  \Test{abc}{abc}{true}%
  \Test{abc}{foo}{false}%
  \Test{foo}{abc}{false}%
  \bitsetReset{foo}%
  \Test{abc}{foo}{true}%
  \Test{foo}{abc}{true}%
  \bitsetSet{abc}{4}%
  \Test{abc}{foo}{false}%
  \Test{foo}{abc}{false}%
  \bitsetFlip{foo}{4}%
  \Test{abc}{foo}{true}%
  \Test{foo}{abc}{true}%
\end{qstest}

\begin{qstest}{Intersects}{Intersects}
  \def\Test#1{%
    \Expect*{\bitsetIntersects{abc}{foo}{true}{false}}{#1}%
  }%
  \let\BS@abc\@undefined
  \let\BS@foo\@undefined
  \Test{false}%
  \Set{abc}{0}%
  \Test{false}%
  \Set{foo}{0}%
  \Test{false}%
  \let\BS@abc\@undefined
  \Test{false}%
  \Set{foo}{1}%
  \Test{false}%
  \Set{abc}{0}%
  \Test{false}%
  \Set{abc}{1}%
  \Test{true}%
  \let\BS@foo\@undefined
  \Test{false}%
  \Set{foo}{0}%
  \Test{false}%
  \def\Test#1#2#3{%
    \bitsetSetBin{abc}{#1}%
    \bitsetSetBin{foo}{#2}%
    \Expect*{\bitsetIntersects{abc}{foo}{true}{false}}{#3}%
  }%
  \Test{1010}{0101}{false}%
  \Test{0}{10}{false}%
  \Test{1}{11}{true}%
  \Test{11}{1}{true}%
  \Test{10}{1}{false}%
\end{qstest}

\begin{qstest}{And/AndNot/Or/Xor}{And/AndNot/Or/Xor}
  \def\@Test#1#2#3#4#5{%
    \begingroup
      #5%
      \begingroup
        \let\BS@foo\@undefined
        \csname bitset#1\endcsname{abc}{foo}%
        \CheckUndef{foo}%
        \Check{abc}{#2}%
      \endgroup
      \begingroup
        \bitsetReset{foo}%
        \csname bitset#1\endcsname{abc}{foo}%
        \Check{foo}{0}%
        \Check{abc}{#3}%
      \endgroup
      \begingroup
        \def\BS@foo{0101}%
        \csname bitset#1\endcsname{abc}{foo}%
        \Check{foo}{0101}%
        \Check{abc}{#4}%
      \endgroup
    \endgroup
  }%
  \def\Test#1{%
    \def\Op{#1}%
    \Test@
  }%
  \def\Test@#1#2#3#4#5#6#7#8#9{%
    \@Test\Op{#1}{#2}{#3}{%
      \let\BS@abc\@undefined
    }%
    \@Test\Op{#4}{#5}{#6}{%
      \bitsetReset{abc}%
    }%
    \@Test\Op{#7}{#8}{#9}{%
      \def\BS@abc{1001}%
    }%
  }%
  \Test{And}%
       {0}{0}{0}%
       {0}{0}{0}%
       {0}{0}{0001}%
  \Test{AndNot}%
       {0}{0}{0}%
       {0}{0}{0}%
       {1001}{1001}{1}%
  \Test{Or}%
       {0}{0}{0101}%
       {0}{0}{0101}%
       {1001}{1001}{1101}%
  \Test{Xor}%
       {0}{0}{0101}%
       {0}{0}{0101}%
       {1001}{1001}{11}%
  \def\Test#1#2#3{%
    \bitsetSetBin{abc}{#1}%
    \bitsetSetBin{foo}{#2}%
    \csname bitset\Op\endcsname{abc}{foo}%
    \RevCheck{foo}{#2}%
    \RevCheck{abc}{#3}%
  }%
  \def\Op{And}%
  \Test{1}{111}{1}%
  \Test{111}{1}{1}%
  \Test{10}{111}{10}%
  \Test{111}{10}{10}%
  \Test{111}{1000}{0}%
  \Test{1000}{111}{0}%
  \def\Op{AndNot}%
  \Test{1010}{11}{1000}%
  \Test{100}{100}{0}%
  \Test{111}{1111}{0}%
  \Test{100}{111}{0}%
  \def\Op{Or}%
  \Test{0}{0}{0}%
  \Test{1}{0}{1}%
  \Test{0}{1}{1}%
  \Test{1}{1}{1}%
  \Test{1000}{10}{1010}%
  \Test{10}{1000}{1010}%
  \def\Op{Xor}%
  \Test{0}{0}{0}%
  \Test{1}{0}{1}%
  \Test{0}{1}{1}%
  \Test{1}{1}{0}%
  \Test{1000}{10}{1010}%
  \Test{10}{1000}{1010}%
  \Test   {110011001100}%
       {111000111000111}%
       {111110100001011}%
  \Test{111000111000111}%
          {110011001100}%
       {111110100001011}%
\end{qstest}

\begin{qstest}{GetUndef}{GetUndef, GetBin, GetOct, GetHex}
  \def\TestUndef#1#2{%
    \let\BS@abc\@undefined
    \expandafter\expandafter\expandafter\Expect
    \expandafter\expandafter\expandafter{%
      \x{abc}{#1}%
    }{#2}%
  }%
  \let\x\bitsetGetBin
  \TestUndef{-1}{0}%
  \TestUndef{0}{0}%
  \TestUndef{1}{0}%
  \TestUndef{2}{00}%
  \TestUndef{8}{00000000}%
  \let\x\bitsetGetOct
  \TestUndef{-1}{0}%
  \TestUndef{0}{0}%
  \TestUndef{1}{0}%
  \TestUndef{2}{0}%
  \TestUndef{3}{0}%
  \TestUndef{4}{00}%
  \TestUndef{5}{00}%
  \TestUndef{6}{00}%
  \TestUndef{7}{000}%
  \TestUndef{8}{000}%
  \TestUndef{9}{000}%
  \TestUndef{10}{0000}%
  \let\x\bitsetGetHex
  \TestUndef{-1}{0}%
  \TestUndef{0}{0}%
  \TestUndef{1}{0}%
  \TestUndef{2}{0}%
  \TestUndef{3}{0}%
  \TestUndef{4}{0}%
  \TestUndef{5}{00}%
  \TestUndef{6}{00}%
  \TestUndef{7}{00}%
  \TestUndef{8}{00}%
  \TestUndef{9}{000}%
  \TestUndef{10}{000}%
  \TestUndef{12}{000}%
  \TestUndef{13}{0000}%
  \TestUndef{16}{0000}%
  \TestUndef{17}{00000}%
\end{qstest}

\begin{qstest}{SetBin}{SetBin}
  \def\Test#1#2{%
    \let\BS@abc\@undefined
    \bitsetSetBin{abc}{#1}%
    \expandafter\Expect\expandafter{\BS@abc}{#2}%
  }%
  \Test{}{0}%
  \Test{0}{0}%
  \Test{1}{1}%
  \Test{10}{01}%
  \Test{11}{11}%
  \Test{010}{01}%
  \Test{011}{11}%
  \Test{0010}{01}%
  \Test{1010}{0101}%
\end{qstest}

\begin{qstest}{SetOct}{SetOct}
  \def\Test#1#2{%
    \bitsetSetOct{abc}{#1}%
    \expandafter\Expect\expandafter{\BS@abc}{#2}%
  }%
  \Test{}{0}%
  \Test{0}{0}%
  \Test{000}{0}%
  \Test{1}{1}%
  \Test{001}{1}%
  \Test{010}{0001}%
  \Test{020}{00001}%
  \Test{42}{010001}%
  \Test{377}{11111111}%
  \Test{0377}{11111111}%
  \Test{76543210}{000100010110001101011111}%
  \Test{ 0 7 0 7 1 }{100111000111}%
\end{qstest}

\begin{qstest}{SetHex}{SetHex}
  \def\Test#1#2{%
    \bitsetSetHex{abc}{#1}%
    \expandafter\Expect\expandafter{\BS@abc}{#2}%
  }%
  \Test{}{0}%
  \Test{0}{0}%
  \Test{000}{0}%
  \Test{1}{1}%
  \Test{001}{1}%
  \Test{010}{00001}%
  \Test{020}{000001}%
  \Test{42}{0100001}%
  \Test{3F}{111111}%
  \Test{03F}{111111}%
  \Test{43210}{0000100001001100001}%
  \Test{98765}{10100110111000011001}%
  \Test{FEDCBA}{010111010011101101111111}%
  \Test{ 0 F 0 F 1 }{1000111100001111}%
\end{qstest}

\begin{qstest}{SetDec}{SetDec}
  \def\Test#1#2{%
    \bitsetSetDec{abc}{#1}%
    \expandafter\Expect\expandafter{\BS@abc}{#2}%
  }%
  \Test{}{0}%
  \Test{0}{0}%
  \Test{000}{0}%
  \Test{1}{1}%
  \Test{7}{111}%
  \Test{8}{0001}%
  \Test{001}{1}%
  \Test{010}{0101}%
  \Test{020}{00101}%
  \Test{53}{101011}%
  \Test{255}{11111111}%
  \Test{256}{000000001}%
  \Test{999999999}{111111111001001101011001110111}%
  \Test{1000000000}{000000000101001101011001110111}%
  \Test{4210987654}{01100001010010010111111101011111}%
  \Test{2147483647}{1111111111111111111111111111111}%
  \Test{2147483648}{00000000000000000000000000000001}%
\end{qstest}

\begin{qstest}{GetBin}{GetBin}
  \def\TestUndef#1#2{%
    \let\BS@abc\@undefined
    \expandafter\expandafter\expandafter\Expect
    \expandafter\expandafter\expandafter{%
      \bitsetGetBin{abc}{#1}%
    }{#2}%
  }%
  \TestUndef{-1}{0}%
  \TestUndef{0}{0}%
  \TestUndef{1}{0}%
  \TestUndef{2}{00}%
  \TestUndef{8}{00000000}%
  \def\Test#1#2{%
    \bitsetSetBin{abc}{#2}%
    \expandafter\expandafter\expandafter\Expect
    \expandafter\expandafter\expandafter{%
      \bitsetGetBin{abc}{#1}%
    }{#2}%
  }%
  \Test{-1}{0}%
  \Test{0}{0}%
  \Test{1}{0}%
  \Test{1}{1}%
  \Test{2}{01}%
  \Test{2}{10}%
  \Test{3}{010}%
  \Test{2}{00}%
  \Test{2}{01}%
  \Test{8}{00101100}%
  \Test{2}{10101}%
  \Test{-100}{11011}%
\end{qstest}

\begin{qstest}{GetOct}{GetOct}
  \def\Test#1#2#3{%
    \edef\x{\zap@space#1 \@empty}%
    \edef\x{\noexpand\bitsetSetBin{abc}{\x}}%
    \x
    \expandafter\expandafter\expandafter\Expect
    \expandafter\expandafter\expandafter{%
      \bitsetGetOct{abc}{#2}%
    }{#3}%
  }%
  \Test{111 110 101 100 011 010 001 000}{0}{76543210}%
  \Test{000 111}{0}{7}%
  \Test{101 000}{-1}{50}%
  \Test{111}{-1}{7}%
  \Test{111}{0}{7}%
  \Test{111}{1}{7}%
  \Test{111}{3}{7}%
  \Test{111}{4}{07}%
  \Test{111}{6}{07}%
  \Test{111}{7}{007}%
  \Test{111 010}{6}{72}%
  \Test{111 010}{7}{072}%
  \Test{011 111}{0}{37}%
  \Test{011 111}{6}{37}%
  \Test{011 111}{7}{037}%
  \Test{001 111}{0}{17}%
  \Test{001 111}{6}{17}%
  \Test{001 111}{7}{017}%
\end{qstest}

\begin{qstest}{GetHex}{GetHex}
  \def\Test#1#2#3{%
    \bitsetSetBin{abc}{#1}%
    \expandafter\expandafter\expandafter\Expect
    \expandafter\expandafter\expandafter{%
      \bitsetGetHex{abc}{#2}%
    }{#3}%
  }%
  \Test{1111 1110 1101 1100 1011 1010 1001 1000}{0}{FEDCBA98}%
  \Test{0111 0110 0101 0100 0011 0010 0001 0000}{0}{76543210}%
  \Test{0000 1111}{0}{F}%
  \Test{0101 0000}{-1}{50}%
  \Test{1111}{-1}{F}%
  \Test{1111}{0}{F}%
  \Test{1111}{1}{F}%
  \Test{1111}{4}{F}%
  \Test{1111}{5}{0F}%
  \Test{1111}{8}{0F}%
  \Test{1111}{9}{00F}%
  \Test{1111 0010}{8}{F2}%
  \Test{1111 0010}{9}{0F2}%
  \Test{0111 1111}{0}{7F}%
  \Test{0111 1111}{8}{7F}%
  \Test{0111 1111}{9}{07F}%
  \Test{0011 1111}{0}{3F}%
  \Test{0011 1111}{8}{3F}%
  \Test{0011 1111}{9}{03F}%
  \Test{0001 1111}{0}{1F}%
  \Test{0001 1111}{8}{1F}%
  \Test{0001 1111}{9}{01F}%
\end{qstest}

\begin{qstest}{Range}{Range}
  \TestError{%
    Wrong index numbers in range [9..8]\MessageBreak% hash-ok
    for clear/set/flip on bit set `abc'.\MessageBreak
    The lower index exceeds the upper index.\MessageBreak
    Canceling the operation as error recovery%
  }{%
    \bitsetSetRange{abc}{9}{8}%
  }%
  \def\TestErrorNegInd#1#2#3#4#5#6{%
    \TestError{%
      Negative index in range [#2..#3]\MessageBreak % hash-ok
      for \string\bitset #1Range on bit set `abc'.\MessageBreak
      Using [#4..#5] as error recovery% hash-ok
    }{%
      \csname bitset#1Range\endcsname{abc}{#2}{#3}%
      \global\let\BS@global\BS@abc
    }%
    \Check{global}{#6}%
  }%
  \Set{abc}{111}%
  \TestErrorNegInd{Clear}{-1}{0}{0}{0}{111}%
  \TestErrorNegInd{Clear}{0}{-1}{0}{0}{111}%
  \TestErrorNegInd{Clear}{-2}{2}{0}{2}{001}%
  \bitsetReset{abc}%
  \TestErrorNegInd{Set}{-1}{0}{0}{0}{0}%
  \TestErrorNegInd{Set}{0}{-1}{0}{0}{0}%
  \TestErrorNegInd{Set}{-2}{2}{0}{2}{11}%
  \Set{abc}{101}%
  \TestErrorNegInd{Flip}{-1}{0}{0}{0}{101}%
  \TestErrorNegInd{Flip}{0}{-1}{0}{0}{101}%
  \TestErrorNegInd{Flip}{-2}{2}{0}{2}{011}%
  \def\Test#1#2#3#4{%
    \bitsetSetBin{abc}{#1}%
    \csname bitset\TestOp Range\endcsname{abc}{#2}{#3}%
    \Expect*{\bitsetGetBin{abc}{0}}{#4}%
  }%
  \def\TestOp{Clear}%
  \Test{0}{0}{1}{0}%
  \Test{1111}{1}{2}{1101}%
  \Test{1111}{1}{3}{1001}%
  \Test{1111111100000000}{12}{14}{1100111100000000}%
  \def\TestOp{Set}%
  \Test{0}{0}{1}{1}%
  \Test{1000}{1}{2}{1010}%
  \Test{0}{1}{2}{10}%
  \Test{1}{12}{15}{111000000000001}%
  \Test{1111}{1}{3}{1111}%
  \Test{1000000000000000}{12}{14}{1011000000000000}%
  \def\TestOp{Flip}%
  \Test{0}{0}{1}{1}%
  \Test{1}{0}{1}{0}%
  \Test{10101010}{1}{5}{10110100}%
  \def\Test#1#2#3#4#5{%
    \bitsetSetBin{abc}{#1}%
    \bitsetSetValueRange{abc}{#2}{#3}{#4}%
    \Expect*{\bitsetGetBin{abc}{0}}{#5}%
  }%
  \Test{0}{0}{1}{0}{0}%
  \Test{0}{0}{1}{1}{1}%
  \Test{1010}{1}{3}{0}{1000}%
  \Test{1010}{1}{3}{1}{1110}%
\end{qstest}

\begin{qstest}{ShiftLeft/ShiftRight}{ShiftLeft/ShiftRight}
  \def\@Test#1#2{%
    \let\BS@abc\@undefined
    \csname bitsetShift#1\endcsname{abc}{#2}%
    \Expect*{\BS@abc}{0}%
  }%
  \def\Test#1{%
    \@Test{Left}{#1}%
    \@Test{Right}{#1}%
  }%
  \Test{-16}%
  \Test{-1}%
  \Test{0}%
  \Test{1}%
  \Test{16}%
  \def\Test#1#2#3{%
    \bitsetSetBin{abc}{#1}%
    \bitsetSetBin{result}{#3}%
    \csname bitsetShift\Op\endcsname{abc}{#2}%
    \Expect*{\bitsetGetBin{abc}{0}}*{\bitsetGetBin{result}{0}}%
  }%
  \def\Op{Left}%
  \Test{0}{0}{0}%
  \Test{0}{1}{0}%
  \Test{0}{-1}{0}%
  \Test{1}{0}{1}%
  \Test{1}{1}{10}%
  \Test{1}{-1}{0}%
  \Test{10}{1}{100}%
  \Test{10}{-1}{1}%
  \Test{1}{32}{100000000000000000000000000000000}%
  \Test{1}{-100}{0}%
  \def\Op{Right}%
  \Test{0}{0}{0}%
  \Test{0}{1}{0}%
  \Test{0}{-1}{0}%
  \Test{1}{0}{1}%
  \Test{1}{1}{0}%
  \Test{1}{-1}{10}%
  \Test{10}{1}{1}%
  \Test{10}{-1}{100}%
  \Test{1}{-32}{100000000000000000000000000000000}%
  \Test{1}{100}{0}%
  \Test{110110110110110}{10}{11011}%
  \Test{110110110110110}{100}{0}%
  \Test{1}{100000}{0}%
\end{qstest}

\begin{qstest}{Profile: Set}{Profile: Set}
  \bitsetSet{abc}{4095}%
  \global\let\BS@global\BS@abc
\end{qstest}

\begin{qstest}{Profile: Get}{Profile: Get}
  \edef\x{\bitsetGet{global}{4095}}%
\end{qstest}

\begin{document}
\end{document}
%</test2>
%    \end{macrocode}
%
% \section{Installation}
%
% \subsection{Download}
%
% \paragraph{Package.} This package is available on
% CTAN\footnote{\url{ftp://ftp.ctan.org/tex-archive/}}:
% \begin{description}
% \item[\CTAN{macros/latex/contrib/oberdiek/bitset.dtx}] The source file.
% \item[\CTAN{macros/latex/contrib/oberdiek/bitset.pdf}] Documentation.
% \end{description}
%
%
% \paragraph{Bundle.} All the packages of the bundle `oberdiek'
% are also available in a TDS compliant ZIP archive. There
% the packages are already unpacked and the documentation files
% are generated. The files and directories obey the TDS standard.
% \begin{description}
% \item[\CTAN{install/macros/latex/contrib/oberdiek.tds.zip}]
% \end{description}
% \emph{TDS} refers to the standard ``A Directory Structure
% for \TeX\ Files'' (\CTAN{tds/tds.pdf}). Directories
% with \xfile{texmf} in their name are usually organized this way.
%
% \subsection{Bundle installation}
%
% \paragraph{Unpacking.} Unpack the \xfile{oberdiek.tds.zip} in the
% TDS tree (also known as \xfile{texmf} tree) of your choice.
% Example (linux):
% \begin{quote}
%   |unzip oberdiek.tds.zip -d ~/texmf|
% \end{quote}
%
% \paragraph{Script installation.}
% Check the directory \xfile{TDS:scripts/oberdiek/} for
% scripts that need further installation steps.
% Package \xpackage{attachfile2} comes with the Perl script
% \xfile{pdfatfi.pl} that should be installed in such a way
% that it can be called as \texttt{pdfatfi}.
% Example (linux):
% \begin{quote}
%   |chmod +x scripts/oberdiek/pdfatfi.pl|\\
%   |cp scripts/oberdiek/pdfatfi.pl /usr/local/bin/|
% \end{quote}
%
% \subsection{Package installation}
%
% \paragraph{Unpacking.} The \xfile{.dtx} file is a self-extracting
% \docstrip\ archive. The files are extracted by running the
% \xfile{.dtx} through \plainTeX:
% \begin{quote}
%   \verb|tex bitset.dtx|
% \end{quote}
%
% \paragraph{TDS.} Now the different files must be moved into
% the different directories in your installation TDS tree
% (also known as \xfile{texmf} tree):
% \begin{quote}
% \def\t{^^A
% \begin{tabular}{@{}>{\ttfamily}l@{ $\rightarrow$ }>{\ttfamily}l@{}}
%   bitset.sty & tex/generic/oberdiek/bitset.sty\\
%   bitset.pdf & doc/latex/oberdiek/bitset.pdf\\
%   test/bitset-test1.tex & doc/latex/oberdiek/test/bitset-test1.tex\\
%   test/bitset-test2.tex & doc/latex/oberdiek/test/bitset-test2.tex\\
%   test/bitset-test3.tex & doc/latex/oberdiek/test/bitset-test3.tex\\
%   bitset.dtx & source/latex/oberdiek/bitset.dtx\\
% \end{tabular}^^A
% }^^A
% \sbox0{\t}^^A
% \ifdim\wd0>\linewidth
%   \begingroup
%     \advance\linewidth by\leftmargin
%     \advance\linewidth by\rightmargin
%   \edef\x{\endgroup
%     \def\noexpand\lw{\the\linewidth}^^A
%   }\x
%   \def\lwbox{^^A
%     \leavevmode
%     \hbox to \linewidth{^^A
%       \kern-\leftmargin\relax
%       \hss
%       \usebox0
%       \hss
%       \kern-\rightmargin\relax
%     }^^A
%   }^^A
%   \ifdim\wd0>\lw
%     \sbox0{\small\t}^^A
%     \ifdim\wd0>\linewidth
%       \ifdim\wd0>\lw
%         \sbox0{\footnotesize\t}^^A
%         \ifdim\wd0>\linewidth
%           \ifdim\wd0>\lw
%             \sbox0{\scriptsize\t}^^A
%             \ifdim\wd0>\linewidth
%               \ifdim\wd0>\lw
%                 \sbox0{\tiny\t}^^A
%                 \ifdim\wd0>\linewidth
%                   \lwbox
%                 \else
%                   \usebox0
%                 \fi
%               \else
%                 \lwbox
%               \fi
%             \else
%               \usebox0
%             \fi
%           \else
%             \lwbox
%           \fi
%         \else
%           \usebox0
%         \fi
%       \else
%         \lwbox
%       \fi
%     \else
%       \usebox0
%     \fi
%   \else
%     \lwbox
%   \fi
% \else
%   \usebox0
% \fi
% \end{quote}
% If you have a \xfile{docstrip.cfg} that configures and enables \docstrip's
% TDS installing feature, then some files can already be in the right
% place, see the documentation of \docstrip.
%
% \subsection{Refresh file name databases}
%
% If your \TeX~distribution
% (\teTeX, \mikTeX, \dots) relies on file name databases, you must refresh
% these. For example, \teTeX\ users run \verb|texhash| or
% \verb|mktexlsr|.
%
% \subsection{Some details for the interested}
%
% \paragraph{Attached source.}
%
% The PDF documentation on CTAN also includes the
% \xfile{.dtx} source file. It can be extracted by
% AcrobatReader 6 or higher. Another option is \textsf{pdftk},
% e.g. unpack the file into the current directory:
% \begin{quote}
%   \verb|pdftk bitset.pdf unpack_files output .|
% \end{quote}
%
% \paragraph{Unpacking with \LaTeX.}
% The \xfile{.dtx} chooses its action depending on the format:
% \begin{description}
% \item[\plainTeX:] Run \docstrip\ and extract the files.
% \item[\LaTeX:] Generate the documentation.
% \end{description}
% If you insist on using \LaTeX\ for \docstrip\ (really,
% \docstrip\ does not need \LaTeX), then inform the autodetect routine
% about your intention:
% \begin{quote}
%   \verb|latex \let\install=y\input{bitset.dtx}|
% \end{quote}
% Do not forget to quote the argument according to the demands
% of your shell.
%
% \paragraph{Generating the documentation.}
% You can use both the \xfile{.dtx} or the \xfile{.drv} to generate
% the documentation. The process can be configured by the
% configuration file \xfile{ltxdoc.cfg}. For instance, put this
% line into this file, if you want to have A4 as paper format:
% \begin{quote}
%   \verb|\PassOptionsToClass{a4paper}{article}|
% \end{quote}
% An example follows how to generate the
% documentation with pdf\LaTeX:
% \begin{quote}
%\begin{verbatim}
%pdflatex bitset.dtx
%makeindex -s gind.ist bitset.idx
%pdflatex bitset.dtx
%makeindex -s gind.ist bitset.idx
%pdflatex bitset.dtx
%\end{verbatim}
% \end{quote}
%
% \section{Catalogue}
%
% The following XML file can be used as source for the
% \href{http://mirror.ctan.org/help/Catalogue/catalogue.html}{\TeX\ Catalogue}.
% The elements \texttt{caption} and \texttt{description} are imported
% from the original XML file from the Catalogue.
% The name of the XML file in the Catalogue is \xfile{bitset.xml}.
%    \begin{macrocode}
%<*catalogue>
<?xml version='1.0' encoding='us-ascii'?>
<!DOCTYPE entry SYSTEM 'catalogue.dtd'>
<entry datestamp='$Date$' modifier='$Author$' id='bitset'>
  <name>bitset</name>
  <caption>Handle bit-vector datatype.</caption>
  <authorref id='auth:oberdiek'/>
  <copyright owner='Heiko Oberdiek' year='2007,2011'/>
  <license type='lppl1.3'/>
  <version number='1.1'/>
  <description>
    This package defines and implements the data type bit set,
    a vector of bits. The size of the vector may grow dynamically.
    Individual bits can be manipulated.
    <p/>
    The package is part of the <xref refid='oberdiek'>oberdiek</xref> bundle.
  </description>
  <documentation details='Package documentation'
      href='ctan:/macros/latex/contrib/oberdiek/bitset.pdf'/>
  <ctan file='true' path='/macros/latex/contrib/oberdiek/bitset.dtx'/>
  <miktex location='oberdiek'/>
  <texlive location='oberdiek'/>
  <install path='/macros/latex/contrib/oberdiek/oberdiek.tds.zip'/>
</entry>
%</catalogue>
%    \end{macrocode}
%
% \begin{History}
%   \begin{Version}{2007/09/28 v1.0}
%   \item
%     First version.
%   \end{Version}
%   \begin{Version}{2011/01/30 v1.1}
%   \item
%     Already loaded package files are not input in \hologo{plainTeX}.
%   \end{Version}
% \end{History}
%
% \PrintIndex
%
% \Finale
\endinput
|
% \end{quote}
% Do not forget to quote the argument according to the demands
% of your shell.
%
% \paragraph{Generating the documentation.}
% You can use both the \xfile{.dtx} or the \xfile{.drv} to generate
% the documentation. The process can be configured by the
% configuration file \xfile{ltxdoc.cfg}. For instance, put this
% line into this file, if you want to have A4 as paper format:
% \begin{quote}
%   \verb|\PassOptionsToClass{a4paper}{article}|
% \end{quote}
% An example follows how to generate the
% documentation with pdf\LaTeX:
% \begin{quote}
%\begin{verbatim}
%pdflatex bitset.dtx
%makeindex -s gind.ist bitset.idx
%pdflatex bitset.dtx
%makeindex -s gind.ist bitset.idx
%pdflatex bitset.dtx
%\end{verbatim}
% \end{quote}
%
% \section{Catalogue}
%
% The following XML file can be used as source for the
% \href{http://mirror.ctan.org/help/Catalogue/catalogue.html}{\TeX\ Catalogue}.
% The elements \texttt{caption} and \texttt{description} are imported
% from the original XML file from the Catalogue.
% The name of the XML file in the Catalogue is \xfile{bitset.xml}.
%    \begin{macrocode}
%<*catalogue>
<?xml version='1.0' encoding='us-ascii'?>
<!DOCTYPE entry SYSTEM 'catalogue.dtd'>
<entry datestamp='$Date$' modifier='$Author$' id='bitset'>
  <name>bitset</name>
  <caption>Handle bit-vector datatype.</caption>
  <authorref id='auth:oberdiek'/>
  <copyright owner='Heiko Oberdiek' year='2007,2011'/>
  <license type='lppl1.3'/>
  <version number='1.1'/>
  <description>
    This package defines and implements the data type bit set,
    a vector of bits. The size of the vector may grow dynamically.
    Individual bits can be manipulated.
    <p/>
    The package is part of the <xref refid='oberdiek'>oberdiek</xref> bundle.
  </description>
  <documentation details='Package documentation'
      href='ctan:/macros/latex/contrib/oberdiek/bitset.pdf'/>
  <ctan file='true' path='/macros/latex/contrib/oberdiek/bitset.dtx'/>
  <miktex location='oberdiek'/>
  <texlive location='oberdiek'/>
  <install path='/macros/latex/contrib/oberdiek/oberdiek.tds.zip'/>
</entry>
%</catalogue>
%    \end{macrocode}
%
% \begin{History}
%   \begin{Version}{2007/09/28 v1.0}
%   \item
%     First version.
%   \end{Version}
%   \begin{Version}{2011/01/30 v1.1}
%   \item
%     Already loaded package files are not input in \hologo{plainTeX}.
%   \end{Version}
% \end{History}
%
% \PrintIndex
%
% \Finale
\endinput

%        (quote the arguments according to the demands of your shell)
%
% Documentation:
%    (a) If bitset.drv is present:
%           latex bitset.drv
%    (b) Without bitset.drv:
%           latex bitset.dtx; ...
%    The class ltxdoc loads the configuration file ltxdoc.cfg
%    if available. Here you can specify further options, e.g.
%    use A4 as paper format:
%       \PassOptionsToClass{a4paper}{article}
%
%    Programm calls to get the documentation (example):
%       pdflatex bitset.dtx
%       makeindex -s gind.ist bitset.idx
%       pdflatex bitset.dtx
%       makeindex -s gind.ist bitset.idx
%       pdflatex bitset.dtx
%
% Installation:
%    TDS:tex/generic/oberdiek/bitset.sty
%    TDS:doc/latex/oberdiek/bitset.pdf
%    TDS:doc/latex/oberdiek/test/bitset-test1.tex
%    TDS:doc/latex/oberdiek/test/bitset-test2.tex
%    TDS:doc/latex/oberdiek/test/bitset-test3.tex
%    TDS:source/latex/oberdiek/bitset.dtx
%
%<*ignore>
\begingroup
  \catcode123=1 %
  \catcode125=2 %
  \def\x{LaTeX2e}%
\expandafter\endgroup
\ifcase 0\ifx\install y1\fi\expandafter
         \ifx\csname processbatchFile\endcsname\relax\else1\fi
         \ifx\fmtname\x\else 1\fi\relax
\else\csname fi\endcsname
%</ignore>
%<*install>
\input docstrip.tex
\Msg{************************************************************************}
\Msg{* Installation}
\Msg{* Package: bitset 2011/01/30 v1.1 Handle bit-vector datatype (HO)}
\Msg{************************************************************************}

\keepsilent
\askforoverwritefalse

\let\MetaPrefix\relax
\preamble

This is a generated file.

Project: bitset
Version: 2011/01/30 v1.1

Copyright (C) 2007, 2011 by
   Heiko Oberdiek <heiko.oberdiek at googlemail.com>

This work may be distributed and/or modified under the
conditions of the LaTeX Project Public License, either
version 1.3c of this license or (at your option) any later
version. This version of this license is in
   http://www.latex-project.org/lppl/lppl-1-3c.txt
and the latest version of this license is in
   http://www.latex-project.org/lppl.txt
and version 1.3 or later is part of all distributions of
LaTeX version 2005/12/01 or later.

This work has the LPPL maintenance status "maintained".

This Current Maintainer of this work is Heiko Oberdiek.

The Base Interpreter refers to any `TeX-Format',
because some files are installed in TDS:tex/generic//.

This work consists of the main source file bitset.dtx
and the derived files
   bitset.sty, bitset.pdf, bitset.ins, bitset.drv, bitset-test1.tex,
   bitset-test2.tex, bitset-test3.tex.

\endpreamble
\let\MetaPrefix\DoubleperCent

\generate{%
  \file{bitset.ins}{\from{bitset.dtx}{install}}%
  \file{bitset.drv}{\from{bitset.dtx}{driver}}%
  \usedir{tex/generic/oberdiek}%
  \file{bitset.sty}{\from{bitset.dtx}{package}}%
  \usedir{doc/latex/oberdiek/test}%
  \file{bitset-test1.tex}{\from{bitset.dtx}{test1}}%
  \file{bitset-test2.tex}{\from{bitset.dtx}{test2,etex}}%
  \file{bitset-test3.tex}{\from{bitset.dtx}{test2,noetex}}%
  \nopreamble
  \nopostamble
  \usedir{source/latex/oberdiek/catalogue}%
  \file{bitset.xml}{\from{bitset.dtx}{catalogue}}%
}

\catcode32=13\relax% active space
\let =\space%
\Msg{************************************************************************}
\Msg{*}
\Msg{* To finish the installation you have to move the following}
\Msg{* file into a directory searched by TeX:}
\Msg{*}
\Msg{*     bitset.sty}
\Msg{*}
\Msg{* To produce the documentation run the file `bitset.drv'}
\Msg{* through LaTeX.}
\Msg{*}
\Msg{* Happy TeXing!}
\Msg{*}
\Msg{************************************************************************}

\endbatchfile
%</install>
%<*ignore>
\fi
%</ignore>
%<*driver>
\NeedsTeXFormat{LaTeX2e}
\ProvidesFile{bitset.drv}%
  [2011/01/30 v1.1 Handle bit-vector datatype (HO)]%
\documentclass{ltxdoc}
\usepackage{holtxdoc}[2011/11/22]
\usepackage{amstext}
\usepackage{ifluatex}
\ifluatex
\else
  \usepackage[T1]{fontenc}%
  \usepackage{lmodern}%
\fi
\begin{document}
  \DocInput{bitset.dtx}%
\end{document}
%</driver>
% \fi
%
% \CheckSum{3510}
%
% \CharacterTable
%  {Upper-case    \A\B\C\D\E\F\G\H\I\J\K\L\M\N\O\P\Q\R\S\T\U\V\W\X\Y\Z
%   Lower-case    \a\b\c\d\e\f\g\h\i\j\k\l\m\n\o\p\q\r\s\t\u\v\w\x\y\z
%   Digits        \0\1\2\3\4\5\6\7\8\9
%   Exclamation   \!     Double quote  \"     Hash (number) \#
%   Dollar        \$     Percent       \%     Ampersand     \&
%   Acute accent  \'     Left paren    \(     Right paren   \)
%   Asterisk      \*     Plus          \+     Comma         \,
%   Minus         \-     Point         \.     Solidus       \/
%   Colon         \:     Semicolon     \;     Less than     \<
%   Equals        \=     Greater than  \>     Question mark \?
%   Commercial at \@     Left bracket  \[     Backslash     \\
%   Right bracket \]     Circumflex    \^     Underscore    \_
%   Grave accent  \`     Left brace    \{     Vertical bar  \|
%   Right brace   \}     Tilde         \~}
%
% \GetFileInfo{bitset.drv}
%
% \title{The \xpackage{bitset} package}
% \date{2011/01/30 v1.1}
% \author{Heiko Oberdiek\\\xemail{heiko.oberdiek at googlemail.com}}
%
% \maketitle
%
% \begin{abstract}
% This package defines and implements the data type bit set,
% a vector of bits. The size of the vector may grow dynamically.
% Individual bits can be manipulated.
% \end{abstract}
%
% \tableofcontents
%
% \clearpage
% \section{Documentation}
%
% \subsection{Introduction}
%
% Annotations in the PDF format know entries whose values are
% integers. This numbers are interpreted as set of flags
% specifying properties. For example, annotation dictionaries
% can have a key |/F|. The bits of its integer value
% are interpreted the following way:
% \begin{quote}
% \begin{tabular}{@{}l|l@{}}
%   Bit position & Property name\\
%   \hline
%   1 & Invisible\\
%   2 & Hidden\\
%   3 & Print\\
%   4 & NoZoom\\
%   5 & NoRotate\\
%   6 & NoView\\
%   7 & ReadOnly\\
%   \dots & \dots
% \end{tabular}
% \end{quote}
% Now, let's see how these values are set in package \xpackage{hyperref}
% before it uses this package (before v6.77a):
%\begin{quote}
%\begin{verbatim}
%\ifFld@hidden /F 6\else /F 4\fi
%\end{verbatim}
%\end{quote}
% Where are the other flags? The following example for key |/Ff|
% in a widget annotation
% supports at least three properties:
%\begin{quote}
%\begin{verbatim}
%\ifFld@multiline
%  \ifFld@readonly /Ff 4097\else /Ff 4096\fi
%\else
%  \ifFld@password
%    \ifFld@readonly /Ff 8193\else /Ff 8192\fi
%  \else
%    \ifFld@readonly /Ff 1\fi
%  \fi
%\fi
%\end{verbatim}
%\end{quote}
% But you see the point. It would be a nightmare to continue
% this way in supporting the missing flag settings. This kind
% of integers may have up to 32 bits.
%
% Therefore I wanted a data structure for setting and
% clearing individual bits. Also it should provide
% an export as decimal number. The snipsets above are
% executed in expansion contexts without \TeX's stomach
% commands. It would be convenient to have an expandable
% conversion from the data structure to the integer that
% gets written to the PDF file.
%
% This package \xpackage{bitset} implements such a data
% structure. The interface is quite close to Java's
% class |BitSet| in order not to learn to many interfaces
% for the same kind of data structure.
%
% \subsection{Glossary}
%
% \begin{description}
% \item[Bit set:]
%   A bit set is a vector of bits or flags.
%   The vector size is unlimited and
%   grows dynamically. An undefined bit set is treated as bit
%   set where all bits are cleared.
%
%   Bit sets are addressed by name. A name should consists of
%   letters or digits. Technically it must survive \cs{csname},
%   see \LaTeX's environment names for other names with such
%   a constraint. Package \xpackage{babel}'s shorthands are
%   not supported due to technical reasons. Shorthand support
%   would break expandable operations.
% \item[Size:]
%   A size of a bit set is the number of bits in use.
%   It's the number of the highest index, incremented by one.
%   Sizes are in the range |0| up to |2147483647|, the highest
%   number supported by \TeX.
% \item[Index:]
%   Bit positions in a bit set are addressed by an index number.
%   The bit vector is zero based. The first and least significant
%   bit is addressed by index |0| and the highest possible bit
%   by |2147483646|.
% \item[Bit:]
%   A bit is enoded as |0| for cleared/disabled or |1| for set/enabled.
% \end{description}
%
% \subsection{Design principles}
%
% \begin{description}
% \item[Name conventions:]
%   To avoid conflicts with existing macro names, the operations
%   are prefixed by the package name.
% \item[Zero based indexes:] The first bit is addressed by zero.
%   (Convention of array indexing in C, Java, \dots)
% \item[Unlimited size:]
%   There is no restriction on the size of a bit set other than
%   usual memory limitations. \cs{bitsetSetDec} and \cs{bitsetGetDec}
%   transparently switch to package \xpackage{bigintcalc} if the numbers
%   get too large for \TeX's number limit.
% \item[Expandibility:]
%   Any operation that does not change the bit set is expandable.
%   And all operations that extract or calculate some result
%   do this in exact two expansion steps. For example,
%   a macro \cs{Macro} wants a bit set as decimal number.
%   But the argument must be a plain number without macros.
%   Thus you could prefix \cs{bitsetGetDec} with \cs{number}.
%   However this won't work for bit sets with 31 or more bits
%   because of \TeX's number limit of $2^{31}-1$.
%   then just hit the operator with two \cs{expandafter}:
%   \begin{quote}
%   |\expandafter\expandafter\expandafter|\\
%   |\Macro\bitsetGetDec{foo}|
%   \end{quote}
%   |\bitsetGetDec| is hit first by the third \cs{expandafter}
%   and then by the second one.
% \item[Format independence:] This package is written as \LaTeX\ package,
%   but it does not depend on \LaTeX. It will also work for other formats
%   such as \plainTeX.
% \item[Independence from \TeX\ engines:]
%   Vanilla \TeX\ is all you need. Calculations are delegated to
%   packages \xpackage{intcalc} and \xpackage{bigintcalc}.
%   They don't need any special features, but they will switch
%   to a little more efficient implementation if features such
%   as \cs{numexpr} are available.
% \item[Numeric arguments:] Anything that is accepted by
%   \cs{number}. If \eTeX\ is detected, also expressions
%   for \cs{numexpr} are supported. The only exception so far
%   is the number for \cs{bitsetSetDec}. The number might be too
%   large for \cs{number} or \cs{numexpr}.
% \item[Error messages:]
%   In expandable contexts, only a limited set of \TeX\ primitive
%   commands work as expected. So called stomach commands behave
%   like \cs{relax} and don't get expanded or executed.
%   Unhappily also the error commands belong to this category.
%   The expandable operations will throw an unknown control
%   sequence instead to get \TeX's and user's attention.
%   The name of these control sequences starts with
%   \cs{BitSetError:} with the type of error after the colon.
% \end{description}
%
% \subsection{Operator overview}
%
% \begin{minipage}{\linewidth}
% \begin{description}
% \def\It#1#2{^^A
%   \item[^^A
%     {#1} ^^A
%     \textmd{(\hyperref[{sec:#2}]{section \ref*{sec:#2}})}^^A
%   ]^^A
%   \mbox{}\\^^A
% }
% \def\op#1{^^A
%   \cs{bitset}\texttt{\bfseries#1}^^A
% }
% \def\M#1{\meta{#1}}
% \def\Q{\hspace*{\fill}}
% \It{Miscellaneous}{misc}
%   \op{Reset}\Q \M{BitSet}\\
%   \op{Let}\Q \M{BitSet A} \M{BitSet B}
% \It{Import}{import}
%   \op{SetBin}, \op{SetOct}, \op{SetHex}\Q \M{BitSet} \M{Value}\\
%   \op{SetDec}\Q \M{BitSet} \M{Value}
% \It{Export\textsuperscript{a}}{export}
%   \op{GetBin}, \op{GetOct}, \op{GetHex}\Q \M{BitSet} \M{MinSize}\\
%   \op{GetDec}\Q \M{BitSet}
% \It{Logical operators}{log}
%   \op{And}, \op{AndNot}\Q \M{BitSet A} \M{BitSet B}\\
%   \op{Or}, \op{Xor}\Q \M{BitSet A} \M{BitSet B}
% \It{Shifting}{shift}
%   \op{ShiftLeft}, \op{ShiftRight}\Q \M{BitSet} \M{ShiftAmount}\\
% \It{Bit manipulation}{bitman}
%   \op{Clear}, \op{Set}, \op{Flip}\Q \M{BitSet} \M{Index}\\
%   \op{SetValue}\Q \M{BitSet} \M{Index} \M{Value}\\
%   \op{ClearRange}, \op{SetRange}, \op{FlipRange}\\
%     \Q \M{BitSet} \M{IndexFrom} \M{IndexTo}\\
%   \op{SetValueRange}\Q \M{BitSet} \M{IndexFrom} \M{IndexTo}
% \It{Bit retrieval\textsuperscript{a}}{bitret}
%   \op{Get}\Q \M{BitSet} \M{Index}\\
%   \op{NextClearBit}, \op{NextSetBit}\Q \M{BitSet} \M{Index}\\
%   \op{GetSetBitList}\Q \M{BitSet}
% \It{Bit set properties}{prop}
%   \op{Size}, \op{Cardinality}\Q \M{BitSet}
% \It{Queries\textsuperscript{b}}{query}
%   \op{IsDefined}, \op{IsEmpty}\Q \M{BitSet} \M{Then} \M{Else}\\
%   \op{Equals}, \op{Intersects}\Q \M{BitSet A} \M{BitSet B} \M{Then} \M{Else}\\
%   \op{Query}\Q \M{BitSet} \M{Index} \M{Then} \M{Else}
% \end{description}
% \footnotetext[1]{Macros are expandable, full expansion by two steps.}
% \footnotetext[2]{Macros are expandable.}
% \end{minipage}
%
% \subsection{Package loading}
%
% The package can be used as normal \LaTeX\ package:
%\begin{quote}
%|\usepackage{bitset}|
%\end{quote}
% Also \plainTeX\ is supported:
%\begin{quote}
%|\input bitset.sty\relax|
%\end{quote}
%
% \subsection{Operators}
%
% The following macros work on and with bit sets.
% A bit set \meta{BitSet} is represented by a name.
% The should consist of letters and digits. Technically
% it must survive \cs{csname}. It is the same constraint
% that must be satisfied by label or environment names
% in \LaTeX.
%
% However active characters that are shorthands
% of package \xpackage{babel} are not supported.
% Support for shorthands works by an assignment.
% But many operators
% such as \cs{bitsetGetDec} must be usable in expandable
% contexts. There assignments will not be executed in the
% best case or they will cause errors.
%
% The bits in a bit set are addressed by non-negative
% integers starting from zero. Thus negative index
% numbers cause an error message. Because index numbers
% are \TeX\ numbers. The largest index is 2147483647.
% But in practice memory limits and patience limits
% will be very likely reached much before.
%
% \subsubsection{Miscellaneous}\label{sec:misc}
%
% There isn't a separate operation for bit set creation.
% For simplicity an undefined bit set is treated as bit set
% with all bits cleared.
%
% \begin{declcs}{bitsetReset} \M{BitSet}
% \end{declcs}
% Macro \cs{bitsetReset} clears all bits. The result is an
% empty bit set. It may also be used as replacement for
% an operation ``new'', because an undefined bit set is defined
% afterwards.
%
% \begin{declcs}{bitsetLet} \M{BitSet A} \M{BitSet B}
% \end{declcs}
% Macro \cs{bitsetLet} performs a simple assignment similar
% to \TeX's \cs{let}. After the operation \meta{BitSet A} has
% the same value as \meta{BitSet B}. If \meta{BitSet B} is
% undefined, then \meta{BitSet A} will be the empty bit set.
%
% Note: If \meta{BitSet A} exists, it will be overwritten.
%
% \subsubsection{Import}\label{sec:import}
%
% \begin{declcs}{bitsetSetBin} \M{BitSet} \M{BinaryNumber}\\
%   \cs{bitsetSetOct} \M{BitSet} \M{OctalNumber}\\
%   \cs{bitsetSetHex} \M{BitSet} \M{HexadecimalNumber}
% \end{declcs}
% The numbers are interpreted as bit vectors and the flags in
% the bit \meta{BitSet} set are set accordingly.
% These numeric arguments are the only arguments where
% spaces are allowed. Then the numbers are easier to read.
%
% \begin{declcs}{bitsetSetDec} \M{BitSet} \M{DecimalNumber}
% \end{declcs}
% Macro \cs{bitsetSetDec} uses \meta{DecimalNumber} to
% set the bit set \meta{BitSet}.
% The numeric argument must expand
% to a plain number consisting of decimal digits without
% command tokens or spaces. Internally this argument is
% expanded only. It cannot be passed to \cs{number} or
% \cs{numexpr}, because the number may be too large for them.
% However \cs{number} or |\the\numexpr| may be used
% explicitely. This also helps for unexpandable number
% command tokens or registers (\cs{z@}, \cs{@ne}, \cs{count@}, \dots).
% Also \LaTeX'\ \cs{value} needs prefixing:
% \begin{quote}
% |\bitsetSetDec{foo}{\number\value{bar}}|
% \end{quote}
%
% \subsubsection{Export}\label{sec:export}
%
% \begin{declcs}{bitsetGetBin} \M{BitSet} \M{MinSize}\\
%   \cs{bitsetGetOct} \M{BitSet} \M{MinSize}\\
%   \cs{bitsetGetHex} \M{BitSet} \M{MinSize}
% \end{declcs}
% These macros returns the bit set as binary, octal or hexadecimal
% number. If the bit size is smaller than \meta{MinSize} the gap is
% filled with leading zeros.
% Example:
% \begin{quote}
% |\bitsetReset{abc}|\\
% |\bitsetSet{abc}{2}|\\
% |\bitsetGetBin{abc}{8}| $\rightarrow$ |00000100|\\
% |\bitsetSet{abc}{5}\bitsetSet{abc}{7}|\\
% |\bitsetGetHex{abc}{16}| $\rightarrow$ |00A2|
% \end{quote}
% Macro \cs{bitsetGetHex} uses the uppercase letters |A| to |F|.
% The catcode of the letters is one of 11 (letter) or 12 (other).
%
% \begin{declcs}{bitsetGetDec} \M{BitSet}
% \end{declcs}
% Macro \cs{bitsetGetDec} returns the bit set \meta{BitSet} as
% decimal number. The returned number can be larger than
% \TeX's number limit of $2^{31}-1$.
%
% \subsubsection{Logical operators}\label{sec:log}
%
% \begin{declcs}{bitsetAnd} \M{BitSet A} \M{BitSet B}
% \end{declcs}
% $A_{\text{new}} \mathrel{\mathop{:}}= A_{\text{old}}
% \mathbin{\text{and}} B \qquad (\forall\, \text{bits})$
%
% \begin{declcs}{bitsetAndNot} \M{BitSet A} \M{BitSet B}
% \end{declcs}
% $A_{\text{new}} \mathrel{\mathop{:}}= A_{\text{old}}
% \mathbin{\text{and}} (\text{not}\,B) \qquad (\forall\, \text{bits})$
%
% \begin{declcs}{bitsetOr} \M{BitSet A} \M{BitSet B}
% \end{declcs}
% $A_{\text{new}} \mathrel{\mathop{:}}= A_{\text{old}}
% \mathbin{\text{or}} B \qquad (\forall\, \text{bits})$
%
% \begin{declcs}{bitsetXor} \M{BitSet A} \M{BitSet B}
% \end{declcs}
% $A_{\text{new}} \mathrel{\mathop{:}}= A_{\text{old}}
% \mathbin{\text{xor}} B \qquad (\forall\, \text{bits})$
%
% \subsubsection{Shifting}\label{sec:shift}
%
% \begin{declcs}{bitsetShiftLeft} \M{BitSet} \M{ShiftAmount}\\
%   \cs{bitsetShiftRight} \M{BitSet} \M{ShiftAmount}
% \end{declcs}
% A left shift by one is a multiplication by two, thus left shifting
% moves the flags to higher positions. The new created low positions
% are filled by zeros.
%
% A right shift is the opposite, dividing by two, movint the bits
% to lower positions. The number will become smaller, the lowest
% bits are lost.
%
% If the \meta{ShiftAmount} is negative, it reverts the meaning of
% the shift operation. A left shift becomes a right shift.
% A \meta{ShiftAmount} of zero is ignored.
%
% \subsubsection{Bit manipulation}\label{sec:bitman}
%
% \begin{declcs}{bitsetClear} \M{BitSet} \M{Index}\\
%   \cs{bitsetSet} \M{BitSet} \M{Index}\\
%   \cs{bitsetFlip} \M{BitSet} \M{Index}
% \end{declcs}
% This macros manipulate a single bit in \meta{BitSet} addressed
% by \cs{Index}. Macro \cs{bitsetClear} disables the bit,
% \cs{bitsetSet} enables it and \cs{bitsetFlip} reverts the current
% setting of the bit.
%
% \begin{declcs}{bitsetSetValue} \M{BitSet} \M{Index} \M{Bit}
% \end{declcs}
% Macro \cs{bitsetSetValue} puts bit \meta{Bit} at position \meta{Index}
% in bit set \meta{BitSet}. \meta{Bit} must be a valid \TeX\ number
% equals to zero (disabled/cleared) or one (enabled/set).
%
% \subsubsection{Bit retrieval}\label{sec:bitret}
%
% \begin{declcs}{bitsetGet} \M{BitSet} \M{Index}
% \end{declcs}
% Macro \cs{bitsetGet} extracts the status of the bit at
% position \meta{Index} in bit set \meta{BitSet}.
% Digit |1| is returned if the bit is set/enabled. If the
% bit is cleared/disabled and in cases of an undefined bitset or
% an index number out of range the return value is |0|.
%
% \begin{declcs}{bitsetNextClearBit} \M{BitSet} \M{Index}
% \end{declcs}
% Starting at position \meta{Index} (inclusive) the bits
% are inspected. The first position without a set bit is returned.
% Possible results are decimal numbers: \meta{Index}, \meta{Index} + 1,
% \dots, ($\infty$)
%
% \begin{declcs}{bitsetNextSetBit} \M{BitSet} \M{Index}
% \end{declcs}
% Starting at position \meta{Index} (inclusive) the bits are
% inspected and the index position of the first found set bit
% is returned. If there isn't such a bit, then the result is
% |-1|. In summary possible results are decimal numbers:
% |-1|, \meta{Index}, \meta{Index} + 1, \dots, ($\infty$)
%
% \begin{declcs}{bitsetGetSetBitList} \M{BitSet}
% \end{declcs}
% Macro \cs{bitsetGetSetBitList} is an application for
% \cs{bitsetNextSetBit}. The set bits are iterated and
% returned as comma separated list of index positions
% in increasing order. The list is empty in case of
% an empty bit set.
%
% \subsubsection{Bit set properties}\label{sec:prop}
%
% \begin{declcs}{bitsetSize} \M{BitSet}
% \end{declcs}
% Macro \cs{bitsetSize} returns number of bits in use.
% It is the same as the index number of the highest set/enabled bit
% incremented by one.
%
% \begin{declcs}{bitsetCardinality} \M{BitSet}
% \end{declcs}
% Macro \cs{bitsetCardinality} counts the number of set/enabled bits.
%
% \subsubsection{Queries}\label{sec:query}
%
% Also the query procedures are expandable. They ask for a piece
% of information about a bit set and execute code depending
% on the answer.
%
% \begin{declcs}{bitsetIsDefined} \M{BitSet} \M{Then} \M{Else}
% \end{declcs}
% If the bit set with the name \meta{BitSet} exists the code
% given in \meta{Then} is executed, otherwise \meta{Else} is used.
%
% \begin{declcs}{bitsetIsEmpty} \M{BitSet} \M{Then} \M{Else}
% \end{declcs}
% If the bit set \meta{BitSet} exists and at least one bit
% is set/enabled, the code in \meta{Then} is executed,
% \meta{Else} otherwise.
%
% \begin{declcs}{bitsetEquals} \M{BitSet A} \M{BitSet B} \M{Then} \M{Else}
% \end{declcs}
% Both bit sets are equal if and only if either both are undefined
% or both are defined and represents the same bit values at the
% same positions. Thus this definition is reflexive, symmeric,
% and transitive, enough for an equivalent relation.
%
% \begin{declcs}{bitsetIntersects} \M{BitSet A} \M{BitSet B} \M{Then} \M{Else}
% \end{declcs}
% If and only if \meta{BitSet A} and \meta{BitSet B} have
% at least one bit at the same position that is set, then
% code part \meta{Then} is executed.
%
% \begin{declcs}{bitsetQuery} \M{BitSet} \M{Index} \M{Then} \M{Else}
% \end{declcs}
% It's just a wrapper for \cs{bitsetGet}. If the bit at position
% \meta{Index} is enabled, code \meta{Then} is called.
%
% \StopEventually{
% }
%
% \section{Implementation}
%
% The internal format of a bit set is quite simple, a sequence
% of digits |0| and |1|. The least significant bit is left.
% A bit set without any flag set is encoded by |0|. Also
% undefined bit sets are treated that way. After the
% highest bit that is set there are no further zeroes.
% A regular expression of valid bit sets values:
% \begin{quote}
%  |0|\string||[01]*1|
% \end{quote}
%
%    \begin{macrocode}
%<*package>
%    \end{macrocode}
%
% \subsection{Reload check and package identification}
%    Reload check, especially if the package is not used with \LaTeX.
%    \begin{macrocode}
\begingroup\catcode61\catcode48\catcode32=10\relax%
  \catcode13=5 % ^^M
  \endlinechar=13 %
  \catcode35=6 % #
  \catcode39=12 % '
  \catcode44=12 % ,
  \catcode45=12 % -
  \catcode46=12 % .
  \catcode58=12 % :
  \catcode64=11 % @
  \catcode123=1 % {
  \catcode125=2 % }
  \expandafter\let\expandafter\x\csname ver@bitset.sty\endcsname
  \ifx\x\relax % plain-TeX, first loading
  \else
    \def\empty{}%
    \ifx\x\empty % LaTeX, first loading,
      % variable is initialized, but \ProvidesPackage not yet seen
    \else
      \expandafter\ifx\csname PackageInfo\endcsname\relax
        \def\x#1#2{%
          \immediate\write-1{Package #1 Info: #2.}%
        }%
      \else
        \def\x#1#2{\PackageInfo{#1}{#2, stopped}}%
      \fi
      \x{bitset}{The package is already loaded}%
      \aftergroup\endinput
    \fi
  \fi
\endgroup%
%    \end{macrocode}
%    Package identification:
%    \begin{macrocode}
\begingroup\catcode61\catcode48\catcode32=10\relax%
  \catcode13=5 % ^^M
  \endlinechar=13 %
  \catcode35=6 % #
  \catcode39=12 % '
  \catcode40=12 % (
  \catcode41=12 % )
  \catcode44=12 % ,
  \catcode45=12 % -
  \catcode46=12 % .
  \catcode47=12 % /
  \catcode58=12 % :
  \catcode64=11 % @
  \catcode91=12 % [
  \catcode93=12 % ]
  \catcode123=1 % {
  \catcode125=2 % }
  \expandafter\ifx\csname ProvidesPackage\endcsname\relax
    \def\x#1#2#3[#4]{\endgroup
      \immediate\write-1{Package: #3 #4}%
      \xdef#1{#4}%
    }%
  \else
    \def\x#1#2[#3]{\endgroup
      #2[{#3}]%
      \ifx#1\@undefined
        \xdef#1{#3}%
      \fi
      \ifx#1\relax
        \xdef#1{#3}%
      \fi
    }%
  \fi
\expandafter\x\csname ver@bitset.sty\endcsname
\ProvidesPackage{bitset}%
  [2011/01/30 v1.1 Handle bit-vector datatype (HO)]%
%    \end{macrocode}
%
% \subsection{Catcodes}
%
%    \begin{macrocode}
\begingroup\catcode61\catcode48\catcode32=10\relax%
  \catcode13=5 % ^^M
  \endlinechar=13 %
  \catcode123=1 % {
  \catcode125=2 % }
  \catcode64=11 % @
  \def\x{\endgroup
    \expandafter\edef\csname BitSet@AtEnd\endcsname{%
      \endlinechar=\the\endlinechar\relax
      \catcode13=\the\catcode13\relax
      \catcode32=\the\catcode32\relax
      \catcode35=\the\catcode35\relax
      \catcode61=\the\catcode61\relax
      \catcode64=\the\catcode64\relax
      \catcode123=\the\catcode123\relax
      \catcode125=\the\catcode125\relax
    }%
  }%
\x\catcode61\catcode48\catcode32=10\relax%
\catcode13=5 % ^^M
\endlinechar=13 %
\catcode35=6 % #
\catcode64=11 % @
\catcode123=1 % {
\catcode125=2 % }
\def\TMP@EnsureCode#1#2{%
  \edef\BitSet@AtEnd{%
    \BitSet@AtEnd
    \catcode#1=\the\catcode#1\relax
  }%
  \catcode#1=#2\relax
}
\TMP@EnsureCode{33}{12}% !
\TMP@EnsureCode{39}{12}% '
\TMP@EnsureCode{40}{12}% (
\TMP@EnsureCode{41}{12}% )
\TMP@EnsureCode{42}{12}% *
\TMP@EnsureCode{43}{12}% +
\TMP@EnsureCode{44}{12}% ,
\TMP@EnsureCode{45}{12}% -
\TMP@EnsureCode{46}{12}% .
\TMP@EnsureCode{47}{12}% /
\TMP@EnsureCode{58}{11}% : (letter!)
\TMP@EnsureCode{60}{12}% <
\TMP@EnsureCode{62}{12}% >
\TMP@EnsureCode{63}{14}% ? (comment!)
\TMP@EnsureCode{91}{12}% [
\TMP@EnsureCode{93}{12}% ]
\TMP@EnsureCode{96}{12}% `
\edef\BitSet@AtEnd{\BitSet@AtEnd\noexpand\endinput}
\begingroup\expandafter\expandafter\expandafter\endgroup
\expandafter\ifx\csname BitSet@TestMode\endcsname\relax
\else
  \catcode63=9 % ? (ignore)
\fi
? \let\BitSet@@TestMode\BitSet@TestMode
%    \end{macrocode}
%
% \subsection{Package loading}
%
%    \begin{macrocode}
\begingroup\expandafter\expandafter\expandafter\endgroup
\expandafter\ifx\csname RequirePackage\endcsname\relax
  \def\TMP@RequirePackage#1[#2]{%
    \begingroup\expandafter\expandafter\expandafter\endgroup
    \expandafter\ifx\csname ver@#1.sty\endcsname\relax
      \input #1.sty\relax
    \fi
  }%
  \TMP@RequirePackage{infwarerr}[2007/09/09]%
  \TMP@RequirePackage{intcalc}[2007/09/27]%
  \TMP@RequirePackage{bigintcalc}[2007/09/27]%
\else
  \RequirePackage{infwarerr}[2007/09/09]%
  \RequirePackage{intcalc}[2007/09/27]%
  \RequirePackage{bigintcalc}[2007/09/27]%
\fi
%    \end{macrocode}
%
% \subsection{Help macros}
%
% \subsubsection{Number constant}
%
%    \begin{macro}{\BitSet@MaxSize}
%    \begin{macrocode}
\def\BitSet@MaxSize{2147483647}%
%    \end{macrocode}
%    \end{macro}
%
% \subsubsection{General basic macros}
%
%    \begin{macro}{\BitSet@Empty}
%    \begin{macrocode}
\def\BitSet@Empty{}
%    \end{macrocode}
%    \end{macro}
%    \begin{macro}{\BitSet@FirstOfOne}
%    \begin{macrocode}
\def\BitSet@FirstOfOne#1{#1}
%    \end{macrocode}
%    \end{macro}
%    \begin{macro}{\BitSet@Gobble}
%    \begin{macrocode}
\def\BitSet@Gobble#1{}
%    \end{macrocode}
%    \end{macro}
%    \begin{macro}{\BitSet@FirstOfTwo}
%    \begin{macrocode}
\def\BitSet@FirstOfTwo#1#2{#1}
%    \end{macrocode}
%    \end{macro}
%    \begin{macro}{\BitSet@SecondOfTwo}
%    \begin{macrocode}
\def\BitSet@SecondOfTwo#1#2{#2}
%    \end{macrocode}
%    \end{macro}
%
%    \begin{macro}{\BitSet@Space}
%    \begin{macrocode}
\def\BitSet@Space{ }
%    \end{macrocode}
%    \end{macro}
%    \begin{macro}{\BitSet@ZapSpace}
%    \begin{macrocode}
\def\BitSet@ZapSpace#1 #2{%
  #1%
  \ifx\BitSet@Empty#2%
  \else
    \expandafter\BitSet@ZapSpace
  \fi
  #2%
}
%    \end{macrocode}
%    \end{macro}
%
% \subsubsection{Tail recursion}
%
%    \begin{macro}{\BitSet@Fi}
%    \begin{macrocode}
\let\BitSet@Fi\fi
%    \end{macrocode}
%    \end{macro}
%    \begin{macro}{\BitSet@AfterFi}
%    \begin{macrocode}
\def\BitSet@AfterFi#1#2\BitSet@Fi{\fi#1}
%    \end{macrocode}
%    \end{macro}
%    \begin{macro}{\BitSet@AfterFiFi}
%    \begin{macrocode}
\def\BitSet@AfterFiFi#1#2\BitSet@Fi{\fi\fi#1}%
%    \end{macrocode}
%    \end{macro}
%    \begin{macro}{\BitSet@AfterFiFiFi}
%    \begin{macrocode}
\def\BitSet@AfterFiFiFi#1#2\BitSet@Fi{\fi\fi\fi#1}%
%    \end{macrocode}
%    \end{macro}
%
% \subsubsection{Check macros}
%
%    \begin{macro}{\BitSet@IfUndefined}
%    \begin{macrocode}
\def\BitSet@IfUndefined#1{%
  \expandafter\ifx\csname BS@#1\endcsname\relax
    \expandafter\BitSet@FirstOfTwo
  \else
    \expandafter\BitSet@SecondOfTwo
  \fi
}
%    \end{macrocode}
%    \end{macro}
%
%    \begin{macro}{\BitSet@CheckIndex}
%    |#1|: continuation code\\
%    |#2|: BitSet\\
%    |#3|: Index
%    \begin{macrocode}
\def\BitSet@CheckIndex#1#2#3{%
  \BitSet@IfUndefined{#2}{\bitsetReset{#2}}{}%
  \expandafter\expandafter\expandafter\BitSet@@CheckIndex
  \intcalcNum{#3}!%
  {#2}{#1}%
}
%    \end{macrocode}
%    \end{macro}
%    \begin{macro}{\BitSet@@CheckIndex}
%    |#1|: plain Index\\
%    |#2|: BitSet\\
%    |#3|: continuation code
%    \begin{macrocode}
\def\BitSet@@CheckIndex#1!#2#3{%
  \ifnum#1<0 %
    \BitSet@AfterFi{%
      \@PackageError{bitset}{%
        Invalid negative index (#1)%
      }\@ehc
    }%
  \else
    \BitSet@AfterFi{%
      #3{#2}{#1}%
    }%
  \BitSet@Fi
}
%    \end{macrocode}
%    \end{macro}
%
% \subsection{Miscellaneous}\label{src:misc}
%
%    \begin{macro}{\bitsetReset}
%    \begin{macrocode}
\def\bitsetReset#1{%
  \expandafter\def\csname BS@#1\endcsname{0}%
}
%    \end{macrocode}
%    \end{macro}
%
%    \begin{macro}{\bitsetLet}
%    \begin{macrocode}
\def\bitsetLet#1#2{%
  \BitSet@IfUndefined{#2}{%
    \bitsetReset{#1}%
  }{%
    \expandafter\let\csname BS@#1\expandafter\endcsname
                    \csname BS@#2\endcsname
  }%
}
%    \end{macrocode}
%    \end{macro}
%
% \subsection{Import}\label{src:import}
%
% \subsubsection{From binary number}
%
%    \begin{macro}{\bitsetSetBin}
%    \begin{macrocode}
\def\bitsetSetBin#1#2{%
  \edef\BitSet@Temp{#2}%
  \edef\BitSet@Temp{%
    \expandafter\expandafter\expandafter\BitSet@ZapSpace
    \expandafter\BitSet@Temp\BitSet@Space\BitSet@Empty
  }%
  \edef\BitSet@Temp{%
    \expandafter\BitSet@KillZeros\BitSet@Temp\BitSet@Empty
  }%
  \ifx\BitSet@Temp\BitSet@Empty
    \expandafter\let\csname BS@#1\endcsname\BitSet@Zero
  \else
    \expandafter\edef\csname BS@#1\endcsname{%
      \expandafter\BitSet@Reverse\BitSet@Temp!%
    }%
  \fi
}
%    \end{macrocode}
%    \end{macro}
%    \begin{macro}{\BitSet@KillZeros}
%    \begin{macrocode}
\def\BitSet@KillZeros#1{%
  \ifx#10%
    \expandafter\BitSet@KillZeros
  \else
    #1%
  \fi
}
%    \end{macrocode}
%    \end{macro}
%    \begin{macro}{\BitSet@Reverse}
%    \begin{macrocode}
\def\BitSet@Reverse#1#2!{%
  \ifx\\#2\\%
    #1%
  \else
    \BitSet@AfterFi{%
      \BitSet@Reverse#2!#1%
    }%
  \BitSet@Fi
}
%    \end{macrocode}
%    \end{macro}
%
% \subsubsection{From octal/hex number}
%
%    \begin{macro}{\bitsetSetOct}
%    \begin{macrocode}
\def\bitsetSetOct{%
  \BitSet@SetOctHex\BitSet@FromFirstOct
}
%    \end{macrocode}
%    \end{macro}
%    \begin{macro}{\bitsetSetHex}
%    \begin{macrocode}
\def\bitsetSetHex{%
  \BitSet@SetOctHex\BitSet@FromFirstHex
}
%    \end{macrocode}
%    \end{macro}
%    \begin{macro}{\BitSet@SetOctHex}
%    \begin{macrocode}
\def\BitSet@SetOctHex#1#2#3{%
  \edef\BitSet@Temp{#3}%
  \edef\BitSet@Temp{%
    \expandafter\expandafter\expandafter\BitSet@ZapSpace
    \expandafter\BitSet@Temp\BitSet@Space\BitSet@Empty
  }%
  \edef\BitSet@Temp{%
    \expandafter\BitSet@KillZeros\BitSet@Temp\BitSet@Empty
  }%
  \ifx\BitSet@Temp\BitSet@Empty
    \expandafter\let\csname BS@#2\endcsname\BitSet@Zero
  \else
    \edef\BitSet@Temp{%
      \expandafter#1\BitSet@Temp!%
    }%
    \ifx\BitSet@Temp\BitSet@Empty
      \expandafter\let\csname BS@#2\endcsname\BitSet@Zero
    \else
      \expandafter\edef\csname BS@#2\endcsname{%
        \expandafter\BitSet@Reverse\BitSet@Temp!%
      }%
    \fi
  \fi
}
%    \end{macrocode}
%    \end{macro}
%    \begin{macro}{\BitSet@FromFirstOct}
%    \begin{macrocode}
\def\BitSet@FromFirstOct#1{%
  \ifx#1!%
  \else
    \ifcase#1 \BitSet@AfterFiFi\BitSet@FromFirstOct
    \or 1%
    \or 10%
    \or 11%
    \or 100%
    \or 101%
    \or 110%
    \or 111%
    \else \BitSetError:WrongOctalDigit%
    \fi
    \expandafter\BitSet@FromOct
  \BitSet@Fi
}
%    \end{macrocode}
%    \end{macro}
%    \begin{macro}{\BitSet@FromOct}
%    \begin{macrocode}
\def\BitSet@FromOct#1{%
  \ifx#1!%
  \else
    \ifcase#1 000%
    \or 001%
    \or 010%
    \or 011%
    \or 100%
    \or 101%
    \or 110%
    \or 111%
    \else \BitSetError:WrongOctalDigit%
    \fi
    \expandafter\BitSet@FromOct
  \fi
}
%    \end{macrocode}
%    \end{macro}
%
%    \begin{macro}{\BitSet@FromFirstHex}
%    \begin{macrocode}
\def\BitSet@FromFirstHex#1{%
  \ifx#1!%
  \else
    \ifx#10%
      \BitSet@AfterFiFi\BitSet@FromFirstHex
    \fi
    \expandafter\ifx\csname BitSet@Hex#1\endcsname\relax
      \BitSetError:InvalidHexDigit%
    \else
      \expandafter\expandafter\expandafter\BitSet@KillZeros
      \csname BitSet@Hex#1\endcsname
    \fi
    \expandafter\BitSet@FromHex
  \BitSet@Fi
}
%    \end{macrocode}
%    \end{macro}
%    \begin{macro}{\BitSet@FromHex}
%    \begin{macrocode}
\def\BitSet@FromHex#1{%
  \ifx#1!%
  \else
    \expandafter\ifx\csname BitSet@Hex#1\endcsname\relax
      \BitSetError:InvalidHexDigit%
    \else
      \csname BitSet@Hex#1\endcsname
    \fi
    \expandafter\BitSet@FromHex
  \fi
}
%    \end{macrocode}
%    \end{macro}
%    \begin{macro}{\BitSet@Hex[0..F]}
%    \begin{macrocode}
\def\BitSet@Temp#1{%
  \expandafter\def\csname BitSet@Hex#1\endcsname
}
\BitSet@Temp 0{0000}%
\BitSet@Temp 1{0001}%
\BitSet@Temp 2{0010}%
\BitSet@Temp 3{0011}%
\BitSet@Temp 4{0100}%
\BitSet@Temp 5{0101}%
\BitSet@Temp 6{0110}%
\BitSet@Temp 7{0111}%
\BitSet@Temp 8{1000}%
\BitSet@Temp 9{1001}%
\BitSet@Temp A{1010}%
\BitSet@Temp B{1011}%
\BitSet@Temp C{1100}%
\BitSet@Temp D{1101}%
\BitSet@Temp E{1110}%
\BitSet@Temp F{1111}%
\BitSet@Temp a{1010}%
\BitSet@Temp b{1011}%
\BitSet@Temp c{1100}%
\BitSet@Temp d{1101}%
\BitSet@Temp e{1110}%
\BitSet@Temp f{1111}%
%    \end{macrocode}
%    \end{macro}
%
% \subsubsection{From decimal number}
%
%    \begin{macro}{\bitsetSetDec}
%    \begin{macrocode}
\def\bitsetSetDec#1#2{%
  \edef\BitSet@Temp{#2}%
  \edef\BitSet@Temp{%
    \expandafter\expandafter\expandafter\BitSet@ZapSpace
    \expandafter\BitSet@Temp\BitSet@Space\BitSet@Empty
  }%
  \edef\BitSet@Temp{%
    \expandafter\BitSet@KillZeros\BitSet@Temp\BitSet@Empty
  }%
  \ifx\BitSet@Temp\BitSet@Empty
    \expandafter\let\csname BS@#1\endcsname\BitSet@Zero
  \else
    \ifcase\bigintcalcSgn{\BitSet@Temp} %
      \expandafter\let\csname BS@#1\endcsname\BitSet@Zero
    \or
      \ifnum\bigintcalcCmp\BitSet@Temp\BitSet@MaxSize>0 %
        \expandafter\edef\csname BS@#1\endcsname{%
          \expandafter\BitSet@SetDecBig\BitSet@Temp!%
        }%
      \else
        \expandafter\edef\csname BS@#1\endcsname{%
          \expandafter\BitSet@SetDec\BitSet@Temp!%
        }%
      \fi
    \else
      \@PackageError{bitset}{%
        Bit sets cannot be negative%
      }\@ehc
    \fi
  \fi
}
%    \end{macrocode}
%    \end{macro}
%    \begin{macro}{\BitSet@SetDecBig}
%    \begin{macrocode}
\def\BitSet@SetDecBig#1#2#3#4#5#6#7#8#9!{%
  \ifx\\#9\\%
    \BitSet@SetDec#1#2#3#4#5#6#7#8!%
  \else
    \ifcase\BigIntCalcOdd#1#2#4#5#6#7#8#9! %
      0%
    \or
      1%
?   \else\BitSetError:ThisCannotHappen%
    \fi
    \BitSet@AfterFi{%
      \expandafter\expandafter\expandafter\BitSet@SetDecBig
      \BigIntCalcShr#1#2#3#4#5#6#7#8#9!!%
    }%
  \BitSet@Fi
}
%    \end{macrocode}
%    \end{macro}
%    \begin{macro}{\BitSet@SetDec}
%    \begin{macrocode}
\def\BitSet@SetDec#1!{%
  \ifcase#1 %
  \or 1%
  \else
    \ifodd#1 %
      1%
    \else
      0%
    \fi
    \BitSet@AfterFi{%
      \expandafter\expandafter\expandafter\BitSet@SetDec
      \IntCalcShr#1!!%
    }%
  \BitSet@Fi
}
%    \end{macrocode}
%    \end{macro}
%
% \subsection{Export}\label{src:export}
%
% \subsubsection{To binary number}
%
%    \begin{macro}{\bitsetGetBin}
%    \begin{macrocode}
\def\bitsetGetBin#1#2{%
  \romannumeral0%
  \expandafter\expandafter\expandafter\BitSet@@GetBin
  \intcalcNum{#2}!{#1}%
}
%    \end{macrocode}
%    \end{macro}
%    \begin{macro}{\BitSet@@GetBin}
%    \begin{macrocode}
\def\BitSet@@GetBin#1!#2{%
  \BitSet@IfUndefined{#2}{%
    \ifnum#1>1 %
      \BitSet@AfterFi{%
        \expandafter\expandafter\expandafter\BitSet@Fill
        \IntCalcDec#1!!0%
      }%
    \else
      \BitSet@AfterFi{ 0}%
    \BitSet@Fi
  }{%
    \expandafter\expandafter\expandafter\BitSet@NumBinRev
    \expandafter\expandafter\expandafter1%
    \expandafter\expandafter\expandafter!%
    \csname BS@#2\endcsname!!#1!%
  }%
}
%    \end{macrocode}
%    \end{macro}
%    \begin{macro}{\BitSet@Fill}
%    |#1|: number of leading digits |0|\\
%    |#2|: result
%    \begin{macrocode}
\def\BitSet@Fill#1!{%
  \ifnum#1>0 %
    \BitSet@AfterFi{%
      \expandafter\expandafter\expandafter\BitSet@Fill
      \IntCalcDec#1!!0%
    }%
  \else
    \BitSet@AfterFi{ }%
  \BitSet@Fi
}
%    \end{macrocode}
%    \end{macro}
%
%    \begin{macro}{\BitSet@NumBinRev}
%    |#1|: bit counter (including |#2|)\\
%    |#2#3|: reverted number\\
%    |#4|: result\\
%    |#5|: min size
%    \begin{macrocode}
\def\BitSet@NumBinRev#1!#2#3!{%
  \ifx\\#3\\%
    \BitSet@AfterFi{%
      \BitSet@NumBinFill#1!#2%
    }%
  \else
    \BitSet@AfterFi{%
      \expandafter\expandafter\expandafter\BitSet@NumBinRev
      \IntCalcInc#1!!#3!#2%
    }%
  \BitSet@Fi
}
%    \end{macrocode}
%    \end{macro}
%    \begin{macro}{\BitSet@NumBinFill}
%    \begin{macrocode}
\def\BitSet@NumBinFill#1!#2!#3!{%
  \ifnum#3>#1 %
    \BitSet@AfterFi{%
      \expandafter\expandafter\expandafter\BitSet@Fill
      \IntCalcSub#3!#1!!#2%
    }%
  \else
    \BitSet@AfterFi{ #2}%
  \BitSet@Fi
}
%    \end{macrocode}
%    \end{macro}
%
% \subsubsection{To octal/hexadecimal number}
%
%    \begin{macro}{\bitsetGetOct}
%    \begin{macrocode}
\def\bitsetGetOct#1#2{%
  \romannumeral0%
  \bitsetIsEmpty{#1}{%
    \expandafter\expandafter\expandafter\BitSet@@GetOctHex
    \intcalcNum{#2}!3!230%
  }{%
    \expandafter\expandafter\expandafter\BitSet@@GetOct
    \expandafter\expandafter\expandafter1%
    \expandafter\expandafter\expandafter!%
    \expandafter\expandafter\expandafter!%
    \csname BS@#1\endcsname00%
    \BitSet@Empty\BitSet@Empty\BitSet@Empty!{#2}%
  }%
}
%    \end{macrocode}
%    \end{macro}
%    \begin{macro}{\bitsetGetHex}
%    \begin{macrocode}
\def\bitsetGetHex#1#2{%
  \romannumeral0%
  \bitsetIsEmpty{#1}{%
    \expandafter\expandafter\expandafter\BitSet@@GetOctHex
    \intcalcNum{#2}!4!340%
  }{%
    \expandafter\expandafter\expandafter\BitSet@@GetHex
    \expandafter\expandafter\expandafter1%
    \expandafter\expandafter\expandafter!%
    \expandafter\expandafter\expandafter!%
    \csname BS@#1\endcsname000%
    \BitSet@Empty\BitSet@Empty\BitSet@Empty\BitSet@Empty!{#2}%
  }%
}
%    \end{macrocode}
%    \end{macro}
%    \begin{macro}{\BitSet@@GetOct}
%    |#1|: number of digits\\
%    |#2|: result\\
%    |#3#4#5|: bits
%    \begin{macrocode}
\def\BitSet@@GetOct#1!#2!#3#4#5{%
  \ifx#5\BitSet@Empty
    \BitSet@AfterFi{%
      \expandafter\expandafter\expandafter\BitSet@GetOctHex
      \IntCalcDec#1!!#2!23%
    }%
  \else
    \BitSet@AfterFi{%
      \expandafter\expandafter\expandafter\BitSet@@GetOct
      \number\IntCalcInc#1!\expandafter\expandafter\expandafter!%
      \csname BitSet@Oct#5#4#3\endcsname#2!%
    }%
  \BitSet@Fi
}
%    \end{macrocode}
%    \end{macro}
%    \begin{macro}{\BitSet@Oct[000..111]}
%    \begin{macrocode}
\def\BitSet@Temp#1#2#3#4{%
  \expandafter\def\csname BitSet@Oct#1#2#3\endcsname{#4}%
}
\BitSet@Temp0000%
\BitSet@Temp0011%
\BitSet@Temp0102%
\BitSet@Temp0113%
\BitSet@Temp1004%
\BitSet@Temp1015%
\BitSet@Temp1106%
\BitSet@Temp1117%
%    \end{macrocode}
%    \end{macro}
%    \begin{macro}{\BitSet@@GetHex}
%    |#1|: number of digits\\
%    |#2|: result\\
%    |#3#4#5#6|: bits
%    \begin{macrocode}
\def\BitSet@@GetHex#1!#2!#3#4#5#6{%
  \ifx#6\BitSet@Empty
    \BitSet@AfterFi{%
      \expandafter\expandafter\expandafter\BitSet@GetOctHex
      \IntCalcDec#1!!#2!34%
    }%
  \else
    \BitSet@AfterFi{%
      \expandafter\expandafter\expandafter\BitSet@@GetHex
      \number\IntCalcInc#1!\expandafter\expandafter\expandafter!%
      \csname BitSet@Hex#6#5#4#3\endcsname#2!%
    }%
  \BitSet@Fi
}
%    \end{macrocode}
%    \end{macro}
%    \begin{macro}{\BitSet@Hex[0000..1111]}
%    \begin{macrocode}
\def\BitSet@Temp#1#2#3#4#5{%
  \expandafter\def\csname BitSet@Hex#1#2#3#4\endcsname{#5}%
}
\BitSet@Temp00000%
\BitSet@Temp00011%
\BitSet@Temp00102%
\BitSet@Temp00113%
\BitSet@Temp01004%
\BitSet@Temp01015%
\BitSet@Temp01106%
\BitSet@Temp01117%
\BitSet@Temp10008%
\BitSet@Temp10019%
\BitSet@Temp1010A%
\BitSet@Temp1011B%
\BitSet@Temp1100C%
\BitSet@Temp1101D%
\BitSet@Temp1110E%
\BitSet@Temp1111F%
%    \end{macrocode}
%    \end{macro}
%    \begin{macro}{\BitSet@GetOctHex}
%    Leading zeros $ (|#4| - |#1| * 3 + 2)/3 $ if $ |#4| > |#1| * 3$\\
%    |#1|: digit size\\
%    |#2|: result\\
%    |#3|: bits per digit - 1\\
%    |#4|: bits per digit
%    |#5|: garbage\\
%    |#6|: min size
%    \begin{macrocode}
\def\BitSet@GetOctHex#1!#2!#3#4#5!#6{%
  \expandafter\BitSet@@GetOctHex
  \number\intcalcNum{#6}\expandafter\expandafter\expandafter!%
  \IntCalcMul#1!#4!!#3#4#2%
}
%    \end{macrocode}
%    \end{macro}
%    \begin{macro}{\BitSet@@GetOctHex}
%    |#1|: plain min size\\
%    |#2|: digits * (bits per digit)\\
%    |#3|: bits per digit - 1\\
%    |#4|: bits per digit
%    \begin{macrocode}
\def\BitSet@@GetOctHex#1!#2!#3#4{%
  \ifnum#1>#2 %
    \BitSet@AfterFi{%
      \expandafter\expandafter\expandafter\expandafter
      \expandafter\expandafter\expandafter\BitSet@Fill
      \expandafter\IntCalcDiv\number
      \expandafter\expandafter\expandafter\IntCalcAdd
      \IntCalcSub#1!#2!!#3!!#4!!%
    }%
  \else
    \BitSet@AfterFi{ }%
  \BitSet@Fi
}
%    \end{macrocode}
%    \end{macro}
%
% \subsubsection{To decimal number}
%
%    \begin{macro}{\bitsetGetDec}
%    \begin{macrocode}
\def\bitsetGetDec#1{%
  \romannumeral0%
  \BitSet@IfUndefined{#1}{ 0}{%
    \expandafter\expandafter\expandafter\BitSet@GetDec
    \csname BS@#1\endcsname!%
  }%
}
%    \end{macrocode}
%    \end{macro}
%    \begin{macro}{\BitSet@GetDec}
%    \begin{macrocode}
\def\BitSet@GetDec#1#2!{%
  \ifx\\#2\\%
    \BitSet@AfterFi{ #1}%
  \else
    \BitSet@AfterFi{%
      \BitSet@@GetDec2!#1!#2!%
    }%
  \BitSet@Fi
}
%    \end{macrocode}
%    \end{macro}
%    \begin{macro}{\BitSet@@GetDec}
%    |#1|: power of two\\
%    |#2|: result\\
%    |#3#4|: number
%    \begin{macrocode}
\def\BitSet@@GetDec#1!#2!#3#4!{%
  \ifx\\#4\\%
    \ifx#31%
      \BitSet@AfterFiFi{%
        \expandafter\expandafter\expandafter\BitSet@Space
        \IntCalcAdd#1!#2!%
      }%
    \else
      \BitSet@AfterFiFi{ #2}%
    \fi
  \else
    \ifx#31%
      \BitSet@AfterFiFi{%
        \csname BitSet@N#1%
        \expandafter\expandafter\expandafter\endcsname
        \IntCalcAdd#1!#2!!#4!%
      }%
    \else
      \BitSet@AfterFiFi{%
        \csname BitSet@N#1\endcsname#2!#4!%
      }%
    \fi
  \BitSet@Fi
}
%    \end{macrocode}
%    \end{macro}
%    \begin{macro}{\BitSet@N[1,2,4,...]}
%    \begin{macrocode}
\def\BitSet@Temp#1#2{%
  \expandafter\def\csname BitSet@N#1\endcsname{%
    \BitSet@@GetDec#2!%
  }%
}
\BitSet@Temp{1}{2}
\BitSet@Temp{2}{4}
\BitSet@Temp{4}{8}
\BitSet@Temp{8}{16}
\BitSet@Temp{16}{32}
\BitSet@Temp{32}{64}
\BitSet@Temp{64}{128}
\BitSet@Temp{128}{256}
\BitSet@Temp{256}{512}
\BitSet@Temp{512}{1024}
\BitSet@Temp{1024}{2048}
\BitSet@Temp{2048}{4096}
\BitSet@Temp{4096}{8192}
\BitSet@Temp{8192}{16384}
\BitSet@Temp{16384}{32768}
\BitSet@Temp{32768}{65536}
\BitSet@Temp{65536}{131072}
\BitSet@Temp{131072}{262144}
\BitSet@Temp{262144}{524288}
\BitSet@Temp{524288}{1048576}
\BitSet@Temp{1048576}{2097152}
\BitSet@Temp{2097152}{4194304}
\BitSet@Temp{4194304}{8388608}
\BitSet@Temp{8388608}{16777216}
\BitSet@Temp{16777216}{33554432}
\BitSet@Temp{33554432}{67108864}
\BitSet@Temp{67108864}{134217728}
\BitSet@Temp{134217728}{268435456}
\BitSet@Temp{268435456}{536870912}
\BitSet@Temp{536870912}{1073741824}
%    \end{macrocode}
%    \end{macro}
%    \begin{macro}{\BitSet@N1073741824}
%    \begin{macrocode}
\expandafter\def\csname BitSet@N1073741824\endcsname{%
  \BitSet@GetDecBig2147483648!%
}%
%    \end{macrocode}
%    \end{macro}
%
%    \begin{macro}{\BitSet@GetDecBig}
%    |#1|: current power of two\\
%    |#2|: result\\
%    |#3#4|: number
%    \begin{macrocode}
\def\BitSet@GetDecBig#1!#2!#3#4!{%
  \ifx\\#4\\%
    \ifx#31%
      \BitSet@AfterFiFi{%
        \expandafter\expandafter\expandafter\BitSet@Space
        \BigIntCalcAdd#1!#2!%
      }%
    \else
      \BitSet@AfterFiFi{ #2}%
    \fi
  \else
    \ifx#31%
      \BitSet@AfterFiFi{%
        \expandafter\expandafter\expandafter\BitSet@@GetDecBig
        \BigIntCalcAdd#1!#2!!#1!#4!%
      }%
    \else
      \BitSet@AfterFiFi{%
        \expandafter\expandafter\expandafter\BitSet@GetDecBig
        \BigIntCalcShl#1!!#2!#4!%
      }%
    \fi
  \BitSet@Fi
}
%    \end{macrocode}
%    \end{macro}
%    \begin{macro}{\BitSet@@GetDecBig}
%    |#1|: result\\
%    |#2|: power of two\\
%    |#3#4|: number
%    \begin{macrocode}
\def\BitSet@@GetDecBig#1!#2!{%
  \expandafter\expandafter\expandafter\BitSet@GetDecBig
  \BigIntCalcShl#2!!#1!%
}
%    \end{macrocode}
%    \end{macro}
%
% \subsection{Logical operators}\label{src:log}
%
% \subsubsection{\cs{bitsetAnd}}
%
%    \begin{macro}{\bitsetAnd}
%    Decision table for \cs{bitsetAnd}:
%    \begin{quote}
%    \begin{tabular}{l|l|l|l|}
%                         & undef(B)   & empty(B)   & cardinality(B)$>$0\\
%      \hline
%      undef(A)           & A := empty & A := empty & A := empty\\
%      \hline
%      empty(A)           &            &            &\\
%      \hline
%      cardinality(A)$>$0 & A := empty & A := empty & A \&= B\\
%      \hline
%    \end{tabular}
%    \end{quote}
%    \begin{macrocode}
\def\bitsetAnd#1#2{%
  \bitsetIsEmpty{#1}{%
    \bitsetReset{#1}%
  }{%
    \bitsetIsEmpty{#2}{%
      \bitsetReset{#1}%
    }{%
      \expandafter\edef\csname BS@#1\endcsname{%
        \expandafter\expandafter\expandafter\BitSet@And
        \csname BS@#1\expandafter\expandafter\expandafter\endcsname
        \expandafter\expandafter\expandafter!%
        \csname BS@#2\endcsname!!%
      }%
      \expandafter\ifx\csname BS@#1\endcsname\BitSet@Empty
        \bitsetReset{#1}%
      \fi
    }%
  }%
}
%    \end{macrocode}
%    \end{macro}
%    \begin{macro}{\BitSet@And}
%    \begin{macrocode}
\def\BitSet@And#1#2!#3#4!#5!{%
  \ifx\\#2\\%
    \ifnum#1#3=11 #51\fi
  \else
    \ifx\\#4\\%
      \ifnum#1#3=11 #51\fi
    \else
      \ifnum#1#3=11 %
        #51%
        \BitSet@AfterFiFiFi{%
          \BitSet@And#2!#4!!%
        }%
      \else
        \BitSet@AfterFiFiFi{%
          \BitSet@And#2!#4!#50!%
        }%
      \fi
    \fi
  \BitSet@Fi
}
%    \end{macrocode}
%    \end{macro}
%
% \subsubsection{\cs{bitsetAndNot}}
%
%    \begin{macro}{\bitsetAndNot}
%    Decision table for \cs{bitsetAndNot}:
%    \begin{quote}
%    \begin{tabular}{l|l|l|l|}
%                         & undef(B)   & empty(B)   & cardinality(B)$>$0\\
%      \hline
%      undef(A)           & A := empty & A := empty & A := empty\\
%      \hline
%      empty(A)           &            &            &\\
%      \hline
%      cardinality(A)$>$0 &            &            & A \&= !B\\
%      \hline
%    \end{tabular}
%    \end{quote}
%    \begin{macrocode}
\def\bitsetAndNot#1#2{%
  \bitsetIsEmpty{#1}{%
    \bitsetReset{#1}%
  }{%
    \bitsetIsEmpty{#2}{%
    }{%
      \expandafter\edef\csname BS@#1\endcsname{%
        \expandafter\expandafter\expandafter\BitSet@AndNot
        \csname BS@#1\expandafter\expandafter\expandafter\endcsname
        \expandafter\expandafter\expandafter!%
        \csname BS@#2\endcsname!!%
      }%
      \expandafter\ifx\csname BS@#1\endcsname\BitSet@Empty
        \bitsetReset{#1}%
      \fi
    }%
  }%
}
%    \end{macrocode}
%    \end{macro}
%    \begin{macro}{\BitSet@AndNot}
%    \begin{macrocode}
\def\BitSet@AndNot#1#2!#3#4!#5!{%
  \ifx\\#2\\%
    \ifnum#1#3=10 #51\fi
  \else
    \ifx\\#4\\%
      #5%
      \ifnum#1#3=10 1\else 0\fi
      #2%
    \else
      \ifnum#1#3=10 %
        #51%
        \BitSet@AfterFiFiFi{%
          \BitSet@AndNot#2!#4!!%
        }%
      \else
        \BitSet@AfterFiFiFi{%
          \BitSet@AndNot#2!#4!#50!%
        }%
      \fi
    \fi
  \BitSet@Fi
}
%    \end{macrocode}
%    \end{macro}
%
% \subsubsection{\cs{bitsetOr}}
%
%    \begin{macro}{\bitsetOr}
%    Decision table for \cs{bitsetOr}:
%    \begin{quote}
%    \begin{tabular}{l|l|l|l|}
%                         & undef(B)   & empty(B)   & cardinality(B)$>$0\\
%      \hline
%      undef(A)           & A := empty & A := empty & A := B\\
%      \hline
%      empty(A)           &            &            & A := B\\
%      \hline
%      cardinality(A)$>$0 &            &            & A \string|= B\\
%      \hline
%    \end{tabular}
%    \end{quote}
%    \begin{macrocode}
\def\bitsetOr#1#2{%
  \bitsetIsEmpty{#2}{%
    \BitSet@IfUndefined{#1}{\bitsetReset{#1}}{}%
  }{%
    \bitsetIsEmpty{#1}{%
      \expandafter\let\csname BS@#1\expandafter\endcsname
                      \csname BS@#2\endcsname
    }{%
      \expandafter\edef\csname BS@#1\endcsname{%
        \expandafter\expandafter\expandafter\BitSet@Or
        \csname BS@#1\expandafter\expandafter\expandafter\endcsname
        \expandafter\expandafter\expandafter!%
        \csname BS@#2\endcsname!%
      }%
    }%
  }%
}
%    \end{macrocode}
%    \end{macro}
%    \begin{macro}{\BitSet@Or}
%    \begin{macrocode}
\def\BitSet@Or#1#2!#3#4!{%
  \ifnum#1#3>0 1\else 0\fi
  \ifx\\#2\\%
    #4%
  \else
    \ifx\\#4\\%
      #2%
    \else
      \BitSet@AfterFiFi{%
        \BitSet@Or#2!#4!%
      }%
    \fi
  \BitSet@Fi
}
%    \end{macrocode}
%    \end{macro}
%
% \subsubsection{\cs{bitsetXor}}
%
%    \begin{macro}{\bitsetXor}
%    Decision table for \cs{bitsetXor}:
%    \begin{quote}
%    \begin{tabular}{l|l|l|l|}
%                         & undef(B)   & empty(B)   & cardinality(B)$>$0\\
%      \hline
%      undef(A)           & A := empty & A := empty & A := B\\
%      \hline
%      empty(A)           &            &            & A := B\\
%      \hline
%      cardinality(A)$>$0 &            &            & A \string^= B\\
%      \hline
%    \end{tabular}
%    \end{quote}
%    \begin{macrocode}
\def\bitsetXor#1#2{%
  \bitsetIsEmpty{#2}{%
    \BitSet@IfUndefined{#1}{\bitsetReset{#1}}{}%
  }{%
    \bitsetIsEmpty{#1}{%
      \expandafter\let\csname BS@#1\expandafter\endcsname
                      \csname BS@#2\endcsname
    }{%
      \expandafter\edef\csname BS@#1\endcsname{%
        \expandafter\expandafter\expandafter\BitSet@Xor
        \csname BS@#1\expandafter\expandafter\expandafter\endcsname
        \expandafter\expandafter\expandafter!%
        \csname BS@#2\endcsname!!%
      }%
      \expandafter\ifx\csname BS@#1\endcsname\BitSet@Empty
        \bitsetReset{#1}%
      \fi
    }%
  }%
}
%    \end{macrocode}
%    \end{macro}
%    \begin{macro}{\BitSet@Xor}
%    \begin{macrocode}
\def\BitSet@Xor#1#2!#3#4!#5!{%
  \ifx\\#2\\%
    \ifx#1#3%
      \ifx\\#4\\%
      \else
        #50#4%
      \fi
    \else
      #51#4%
    \fi
  \else
    \ifx\\#4\\%
      #5%
      \ifx#1#30\else 1\fi
      #2%
    \else
      \ifx#1#3%
        \BitSet@AfterFiFiFi{%
          \BitSet@Xor#2!#4!#50!%
        }%
      \else
        #51%
        \BitSet@AfterFiFiFi{%
          \BitSet@Xor#2!#4!!%
        }%
      \fi
    \fi
  \BitSet@Fi
}
%    \end{macrocode}
%    \end{macro}
%
% \subsubsection{Shifting}\label{src:shift}
%
% \subsubsection{\cs{bitsetShiftLeft}}
%
%    \begin{macro}{\bitsetShiftLeft}
%    \begin{macrocode}
\def\bitsetShiftLeft#1#2{%
  \BitSet@IfUndefined{#1}{%
    \bitsetReset{#1}%
  }{%
    \bitsetIsEmpty{#1}{%
    }{%
      \expandafter\expandafter\expandafter\BitSet@ShiftLeft
      \intcalcNum{#2}!{#1}%
    }%
  }%
}
%    \end{macrocode}
%    \end{macro}
%    \begin{macro}{\BitSet@ShiftLeft}
%    \begin{macrocode}
\def\BitSet@ShiftLeft#1!#2{%
  \ifcase\intcalcSgn{#1} %
  \or
    \begingroup
      \uccode`m=`0 %
    \uppercase\expandafter{\expandafter\endgroup
      \expandafter\edef\csname BS@#2\expandafter\endcsname
      \expandafter{%
        \romannumeral#1000\expandafter\BitSet@Space
        \csname BS@#2\endcsname
      }%
    }%
  \else
    \expandafter\BitSet@ShiftRight\BitSet@Gobble#1!{#2}%
  \fi
}
%    \end{macrocode}
%    \end{macro}
%
% \subsubsection{\cs{bitsetShiftRight}}
%
%    \begin{macro}{\bitsetShiftRight}
%    \begin{macrocode}
\def\bitsetShiftRight#1#2{%
  \BitSet@IfUndefined{#1}{%
    \bitsetReset{#1}%
  }{%
    \bitsetIsEmpty{#1}{%
    }{%
      \expandafter\expandafter\expandafter\BitSet@ShiftRight
      \intcalcNum{#2}!{#1}%
    }%
  }%
}
%    \end{macrocode}
%    \end{macro}
%    \begin{macro}{\BitSet@ShiftRight}
%    \begin{macrocode}
\def\BitSet@ShiftRight#1!#2{%
  \ifcase\intcalcSgn{#1} %
  \or
    \expandafter\edef\csname BS@#2\endcsname{%
      \expandafter\expandafter\expandafter\BitSet@Kill
      \csname BS@#2\expandafter\endcsname\expandafter\BitSet@Empty
      \expandafter=%
      \expandafter{\expandafter}\expandafter{\expandafter}%
      \romannumeral#1000!%
    }%
  \else
    \expandafter\BitSet@ShiftLeft\BitSet@Gobble#1!{#2}%
  \fi
}
%    \end{macrocode}
%    \end{macro}
%    \begin{macro}{\BitSet@Kill}
%    \begin{macrocode}
\def\BitSet@Kill#1#2=#3#4#5{%
  #3#4%
  \ifx#5!%
    \ifx#1\BitSet@Empty
      0%
    \else
      #1#2%
    \fi
  \else
    \ifx#1\BitSet@Empty
      0%
      \BitSet@AfterFiFi\BitSet@Cleanup
    \else
      \BitSet@Kill#2=%
    \fi
  \BitSet@Fi
}
%    \end{macrocode}
%    \end{macro}
%
% \subsection{Bit manipulation}\label{src:bitman}
%
%    \begin{macro}{\bitsetClear}
%    \begin{macrocode}
\def\bitsetClear{%
  \BitSet@CheckIndex\BitSet@Clear
}
%    \end{macrocode}
%    \end{macro}
%    \begin{macro}{\bitsetSet}
%    \begin{macrocode}
\def\bitsetSet{%
  \BitSet@CheckIndex\BitSet@Set
}
%    \end{macrocode}
%    \end{macro}
%    \begin{macro}{\bitsetFlip}
%    \begin{macrocode}
\def\bitsetFlip{%
  \BitSet@CheckIndex\BitSet@Flip
}
%    \end{macrocode}
%    \end{macro}
%    \begin{macro}{\bitsetSetValue}
%    \begin{macrocode}
\def\bitsetSetValue#1#2#3{%
  \expandafter\expandafter\expandafter\BitSet@SetValue
  \intcalcNum{#3}!{#1}{#2}%
}
%    \end{macrocode}
%    \end{macro}
%    \begin{macro}{\BitSet@SetValue}
%    |#1|: plain value\\
%    |#2|: BitSet\\
%    |#3|: Index
%    \begin{macrocode}
\def\BitSet@SetValue#1!{%
  \BitSet@CheckIndex{%
    \ifcase#1 %
      \expandafter\BitSet@Clear
    \or
      \expandafter\BitSet@Set
    \else
      \BitSet@ErrorInvalidBitValue{#1}%
      \expandafter\expandafter\expandafter\BitSet@Gobble
      \expandafter\BitSet@Gobble
    \fi
  }%
}
%    \end{macrocode}
%    \end{macro}
%    \begin{macro}{\BitSet@ErrorInvalidBitValue}
%    |#1|: Wrong bit value
%    \begin{macrocode}
\def\BitSet@ErrorInvalidBitValue#1{%
  \@PackageError{bitset}{%
    Invalid bit value (#1) not in range 0..1%
  }\@ehc
}
%    \end{macrocode}
%    \end{macro}
%
% \subsubsection{Clear operation}
%
%    \begin{macro}{\BitSet@Clear}
%    |#1|: BitSet\\
%    |#2|: plain and checked index
%    \begin{macrocode}
\def\BitSet@Clear#1#2{%
  \edef\BitSet@Temp{%
    \expandafter\expandafter\expandafter\BitSet@@Clear
    \csname BS@#1\expandafter\endcsname
    \expandafter\BitSet@Empty\expandafter=\expandafter!%
    \romannumeral#2000!%
  }%
  \expandafter\let\csname BS@#1\expandafter\endcsname
  \ifx\BitSet@Temp\BitSet@Empty
    \BitSet@Zero
  \else
    \BitSet@Temp
  \fi
}
%    \end{macrocode}
%    \end{macro}
%    \begin{macro}{\BitSet@@Clear}
%    \begin{macrocode}
\def\BitSet@@Clear#1#2=#3!#4{%
  \ifx#4!%
    \ifx#1\BitSet@Empty
    \else
      \ifx\BitSet@Empty#2%
      \else
        #30#2%
      \fi
    \fi
  \else
    \ifx#1\BitSet@Empty
      \BitSet@AfterFiFi\BitSet@Cleanup
    \else
      \ifx#10%
        \BitSet@AfterFiFiFi{%
          \BitSet@@Clear#2=#30!%
        }%
      \else
        #31%
        \BitSet@AfterFiFiFi{%
          \BitSet@@Clear#2=!%
        }%
      \fi
    \fi
  \BitSet@Fi
}
%    \end{macrocode}
%    \end{macro}
%
% \subsubsection{Set operation}
%
%    \begin{macro}{\BitSet@Set}
%    |#1|: BitSet\\
%    |#2|: plain and checked Index
%    \begin{macrocode}
\def\BitSet@Set#1#2{%
  \expandafter\edef\csname BS@#1\endcsname{%
    \expandafter\expandafter\expandafter\BitSet@@Set
    \csname BS@#1\expandafter\endcsname
    \expandafter\BitSet@Empty\expandafter=%
    \expandafter{\expandafter}\expandafter{\expandafter}%
    \romannumeral#2000!%
  }%
}
%    \end{macrocode}
%    \end{macro}
%    \begin{macro}{\BitSet@@Set}
%    \begin{macrocode}
\def\BitSet@@Set#1#2=#3#4#5{%
  #3#4%
  \ifx#5!%
    1#2%
  \else
    \ifx#1\BitSet@Empty
      0%
      \BitSet@AfterFiFi\BitSet@@@Set
    \else
      #1%
      \BitSet@@Set#2=%
    \fi
  \BitSet@Fi
}
%    \end{macrocode}
%    \end{macro}
%    \begin{macro}{\BitSet@@@Set}
%    \begin{macrocode}
\def\BitSet@@@Set#1{%
  \ifx#1!%
    1%
  \else
    0%
    \expandafter\BitSet@@@Set
  \fi
}
%    \end{macrocode}
%    \end{macro}
%
% \subsubsection{Flip operation}
%
%    \begin{macro}{\BitSet@Flip}
%    |#1|: BitSet\\
%    |#2|: plain and checked Index
%    \begin{macrocode}
\def\BitSet@Flip#1#2{%
  \edef\BitSet@Temp{%
    \expandafter\expandafter\expandafter\BitSet@@Flip
    \csname BS@#1\expandafter\endcsname
    \expandafter\BitSet@Empty\expandafter=\expandafter!%
    \romannumeral#2000!%
  }%
  \expandafter\let\csname BS@#1\expandafter\endcsname
  \ifx\BitSet@Temp\BitSet@Empty
    \BitSet@Zero
  \else
    \BitSet@Temp
  \fi
}
%    \end{macrocode}
%    \end{macro}
%    \begin{macro}{\BitSet@@Flip}
%    \begin{macrocode}
\def\BitSet@@Flip#1#2=#3!#4{%
  \ifx#4!%
    \ifx#11%
      \ifx\BitSet@Empty#2%
      \else
        #30#2%
      \fi
    \else
      #31#2%
    \fi
  \else
    \ifx#1\BitSet@Empty
      #30%
      \BitSet@AfterFiFi\BitSet@@@Set
    \else
      \ifx#10%
        \BitSet@AfterFiFiFi{%
          \BitSet@@Flip#2=#30!%
        }%
      \else
        #31%
        \BitSet@AfterFiFiFi{%
          \BitSet@@Flip#2=!%
        }%
      \fi
    \fi
  \BitSet@Fi
}
%    \end{macrocode}
%    \end{macro}
%
% \subsubsection{Range operators}
%
%    \begin{macro}{\bitsetClearRange}
%    \begin{macrocode}
\def\bitsetClearRange{%
  \BitSet@Range\BitSet@Clear
}
%    \end{macrocode}
%    \end{macro}
%    \begin{macro}{\bitsetSetRange}
%    \begin{macrocode}
\def\bitsetSetRange{%
  \BitSet@Range\BitSet@Set
}
%    \end{macrocode}
%    \end{macro}
%    \begin{macro}{\bitsetFlipRange}
%    \begin{macrocode}
\def\bitsetFlipRange{%
  \BitSet@Range\BitSet@Flip
}
%    \end{macrocode}
%    \end{macro}
%    \begin{macro}{\bitsetSetValueRange}
%    \begin{macrocode}
\def\bitsetSetValueRange#1#2#3#4{%
  \expandafter\expandafter\expandafter\BitSet@SetValueRange
  \intcalcNum{#4}!{#1}{#2}{#3}%
}
%    \end{macrocode}
%    \end{macro}
%    \begin{macro}{\BitSet@SetValueRange}
%    \begin{macrocode}
\def\BitSet@SetValueRange#1!#2#3#4{%
  \ifcase#1 %
    \BitSet@Range\BitSet@Clear{#2}{#3}{#4}%
  \or
    \BitSet@Range\BitSet@Set{#2}{#3}{#4}%
  \else
    \BitSet@ErrorInvalidBitValue{#1}%
  \fi
}
%    \end{macrocode}
%    \end{macro}
%
%    \begin{macro}{\BitSet@Range}
%    |#1|: clear/set/flip macro\\
%    |#2|: BitSet\\
%    |#3|: Index from\\
%    |#4|: Index to
%    \begin{macrocode}
\def\BitSet@Range#1#2#3#4{%
  \edef\BitSet@Temp{%
    \noexpand\BitSet@@Range\noexpand#1{#2}%
    \intcalcNum{#3}!\intcalcNum{#4}!%
  }%
  \BitSet@Temp
}
%    \end{macrocode}
%    \end{macro}
%    \begin{macro}{\BitSet@@Range}
%    |#1|: clear/set/flip macro\\
%    |#2|: BitSet\\
%    |#3|: Index from\\
%    |#4|: Index to
%    \begin{macrocode}
\def\BitSet@@Range#1#2#3!#4!{%
  \ifnum#3<0 %
    \BitSet@NegativeIndex#1{#2}#3!#4!0!#4!%
  \else
    \ifnum#4<0 %
      \BitSet@NegativeIndex#1{#2}#3!#4!#3!0!%
    \else
      \ifcase\intcalcCmp{#3}{#4} %
      \or
        \@PackageError{bitset}{%
          Wrong index numbers in range [#3..#4]\MessageBreak% hash-ok
          for clear/set/flip on bit set `#2'.\MessageBreak
          The lower index exceeds the upper index.\MessageBreak
          Canceling the operation as error recovery%
        }\@ehc
      \else
        \BitSet@@@Range#3!#4!#1{#2}%
      \fi
    \fi
  \fi
}
%    \end{macrocode}
%    \end{macro}
%    \begin{macro}{\BitSet@NegativeIndex}
%    \begin{macrocode}
\def\BitSet@NegativeIndex#1#2#3!#4!#5!#6!{%
  \@PackageError{bitset}{%
    Negative index in range [#3..#4]\MessageBreak % hash-ok
    for \string\bitset
    \ifx#1\BitSet@Clear
      Clear%
    \else
      \ifx#1\BitSet@Set
        Set%
      \else
        Flip%
      \fi
    \fi
    Range on bit set `#2'.\MessageBreak
    Using [#5..#6] as error recovery% hash-ok
  }\@ehc
  \BitSet@@Range#1{#2}#5!#6!%
}
%    \end{macrocode}
%    \end{macro}
%    \begin{macro}{\BitSet@@Range}
%    \begin{macrocode}
\def\BitSet@@@Range#1!#2!#3#4{%
  \ifnum#1<#2 %
    #3{#4}{#1}%
    \BitSet@AfterFi{%
      \expandafter\expandafter\expandafter\BitSet@@@Range
      \IntCalcInc#1!!#2!#3{#4}%
    }%
  \BitSet@Fi
}
%    \end{macrocode}
%    \end{macro}
%
% \subsection{Bit retrieval}\label{src:bitret}
%
% \subsubsection{\cs{bitsetGet}}
%
%    \begin{macro}{\bitsetGet}
%    \begin{macrocode}
\def\bitsetGet#1#2{%
  \number
  \expandafter\expandafter\expandafter\BitSet@Get
  \intcalcNum{#2}!{#1}%
}
%    \end{macrocode}
%    \end{macro}
%    \begin{macro}{\BitSet@Get}
%    |#1|: plain index\\
%    |#2|: BitSet
%    \begin{macrocode}
\def\BitSet@Get#1!#2{%
  \ifnum#1<0 %
    \BitSet@AfterFi{%
      0 \BitSetError:NegativeIndex%
    }%
  \else
    \BitSet@IfUndefined{#2}{0}{%
      \expandafter\expandafter\expandafter\BitSet@@Get
      \csname BS@#2\expandafter\endcsname
      \expandafter!\expandafter=%
      \expandafter{\expandafter}\expandafter{\expandafter}%
      \romannumeral\intcalcNum{#1}000!%
    }%
    \expandafter\BitSet@Space
  \BitSet@Fi
}
%    \end{macrocode}
%    \end{macro}
%    \begin{macro}{\BitSet@@Get}
%    \begin{macrocode}
\def\BitSet@@Get#1#2=#3#4#5{%
  #3#4%
  \ifx#5!%
    \ifx#1!%
      0%
    \else
      #1%
    \fi
  \else
    \ifx#1!%
      0%
      \BitSet@AfterFiFi\BitSet@Cleanup
    \else
      \BitSet@@Get#2=%
    \fi
  \BitSet@Fi
}
%    \end{macrocode}
%    \end{macro}
%
% \subsubsection{\cs{bitsetNextClearBit}, \cs{bitsetNextSetBit}}
%
%    \begin{macro}{\bitsetNextClearBit}
%    \begin{macrocode}
\def\bitsetNextClearBit#1#2{%
  \number
  \expandafter\expandafter\expandafter\BitSet@NextClearBit
  \intcalcNum{#2}!{#1} %
}
%    \end{macrocode}
%    \end{macro}
%    \begin{macro}{\BitSet@NextClearBit}
%    |#1|: Index\\
%    |#2|: BitSet
%    \begin{macrocode}
\def\BitSet@NextClearBit#1!#2{%
  \ifnum#1<0 %
    \BitSet@NextClearBit0!{#2}%
    \BitSet@AfterFi{%
      \expandafter\BitSet@Space
      \expandafter\BitSetError:NegativeIndex\romannumeral0%
    }%
  \else
    \bitsetIsEmpty{#2}{#1}{%
      \expandafter\BitSet@Skip
      \number#1\expandafter\expandafter\expandafter!%
      \csname BS@#2\endcsname!!!!!!!!!=%
      {\BitSet@@NextClearBit#1!}%
    }%
  \BitSet@Fi
}
%    \end{macrocode}
%    \end{macro}
%    \begin{macro}{\BitSet@@NextClearBit}
%    |#1|: index for next bit in |#2|\\
%    |#2|: next bit
%    \begin{macrocode}
\def\BitSet@@NextClearBit#1!#2{%
  \ifx#2!%
    #1%
  \else
    \ifx#20%
      #1%
      \BitSet@AfterFiFi\BitSet@Cleanup
    \else
      \BitSet@AfterFiFi{%
        \expandafter\expandafter\expandafter\BitSet@@NextClearBit
        \IntCalcInc#1!!%
      }%
    \fi
  \BitSet@Fi
}
%    \end{macrocode}
%    \end{macro}
%
%    \begin{macro}{\bitsetNextSetBit}
%    \begin{macrocode}
\def\bitsetNextSetBit#1#2{%
  \number
  \expandafter\expandafter\expandafter\BitSet@NextSetBit
  \intcalcNum{#2}!{#1} %
}
%    \end{macrocode}
%    \end{macro}
%    \begin{macro}{\BitSet@NextSetBit}
%    |#1|: Index\\
%    |#2|: BitSet
%    \begin{macrocode}
\def\BitSet@NextSetBit#1!#2{%
  \ifnum#1<0 %
    \BitSet@NextSetBit0!{#2}%
    \BitSet@AfterFi{%
      \expandafter\BitSet@Space
      \expandafter\BitSetError:NegativeIndex\romannumeral0%
    }%
  \else
    \bitsetIsEmpty{#2}{-1}{%
      \expandafter\BitSet@Skip
      \number#1\expandafter\expandafter\expandafter!%
      \csname BS@#2\endcsname!!!!!!!!!=%
      {\BitSet@@NextSetBit#1!}%
    }%
  \BitSet@Fi
}
%    \end{macrocode}
%    \end{macro}
%    \begin{macro}{\BitSet@@NextSetBit}
%    |#1|: index for next bit in |#2|\\
%    |#2|: next bit
%    \begin{macrocode}
\def\BitSet@@NextSetBit#1!#2{%
  \ifx#2!%
    -1%
  \else
    \ifx#21%
      #1%
      \BitSet@AfterFiFi\BitSet@Cleanup
    \else
      \BitSet@AfterFiFi{%
        \expandafter\expandafter\expandafter\BitSet@@NextSetBit
        \IntCalcInc#1!!%
      }%
    \fi
  \BitSet@Fi
}
%    \end{macrocode}
%    \end{macro}
%    \begin{macro}{\BitSet@Cleanup}
%    \begin{macrocode}
\def\BitSet@Cleanup#1!{}
%    \end{macrocode}
%    \end{macro}
%
%    \begin{macro}{\BitSet@Skip}
%    |#1|: number of bits to skip\\
%    |#2|: bits\\
%    |#3|: continuation code
%    \begin{macrocode}
\def\BitSet@Skip#1!#2{%
  \ifx#2!%
    \BitSet@AfterFi{%
      \BitSet@SkipContinue%
    }%
  \else
    \ifcase#1 %
      \BitSet@AfterFiFi{%
        \BitSet@SkipContinue#2%
      }%
    \or
      \BitSet@AfterFiFi\BitSet@SkipContinue
    \or
      \BitSet@AfterFiFi{%
        \expandafter\BitSet@SkipContinue\BitSet@Gobble
      }%
    \else
      \ifnum#1>8 %
        \BitSet@AfterFiFiFi{%
          \expandafter\BitSet@Skip
          \number\IntCalcSub#1!8!\expandafter!%
          \BitSet@GobbleSeven
        }%
      \else
        \BitSet@AfterFiFiFi{%
          \expandafter\expandafter\expandafter\BitSet@Skip
          \IntCalcDec#1!!%
        }%
      \fi
    \fi
  \BitSet@Fi
}
%    \end{macrocode}
%    \end{macro}
%    \begin{macro}{\BitSet@SkipContinue}
%    |#1|: remaining bits\\
%    |#2|: continuation code
%    \begin{macrocode}
\def\BitSet@SkipContinue#1!#2=#3{%
  #3#1!%
}
%    \end{macrocode}
%    \end{macro}
%    \begin{macro}{\BitSet@GobbleSeven}
%    \begin{macrocode}
\def\BitSet@GobbleSeven#1#2#3#4#5#6#7{}
%    \end{macrocode}
%    \end{macro}
%
% \subsubsection{\cs{bitsetGetSetBitList}}
%
%    \begin{macro}{\bitsetGetSetBitList}
%    It's just a wrapper for \cs{bitsetNextSetBit}.
%    \begin{macrocode}
\def\bitsetGetSetBitList#1{%
  \romannumeral0%
  \bitsetIsEmpty{#1}{ }{%
    \expandafter\BitSet@GetSetBitList
    \number\BitSet@NextSetBit0!{#1}!{#1}{}!%
  }%
}
%    \end{macrocode}
%    \end{macro}
%    \begin{macro}{\BitSet@GetSetBitList}
%    |#1|: found index\\
%    |#2|: BitSet\\
%    |#3|: comma
%    |#4|: result
%    \begin{macrocode}
\def\BitSet@GetSetBitList#1!#2#3#4!{%
  \ifnum#1<0 %
    \BitSet@AfterFi{ #4}%
  \else
    \BitSet@AfterFi{%
      \expandafter\BitSet@GetSetBitList\number
      \expandafter\expandafter\expandafter\BitSet@NextSetBit
      \IntCalcInc#1!!{#2}!{#2},#4#3#1!%
    }%
  \BitSet@Fi
}
%    \end{macrocode}
%    \end{macro}
%
% \subsection{Bit set properties}\label{src:prop}
%
%    \begin{macro}{\bitsetSize}
%    \begin{macrocode}
\def\bitsetSize#1{%
  \number
  \BitSet@IfUndefined{#1}{0 }{%
    \expandafter\expandafter\expandafter\BitSet@Size
    \expandafter\expandafter\expandafter1%
    \expandafter\expandafter\expandafter!%
    \csname BS@#1\endcsname!0!%
  }%
}
%    \end{macrocode}
%    \end{macro}
%    \begin{macro}{\BitSet@Size}
%    |#1|: counter\\
%    |#2#3|: bits\\
%    |#4|: result
%    \begin{macrocode}
\def\BitSet@Size#1!#2#3!#4!{%
  \ifx#21%
    \ifx\\#3\\%
      \BitSet@AfterFiFi{#1 }%
    \else
      \BitSet@AfterFiFi{%
        \expandafter\expandafter\expandafter\BitSet@Size
        \IntCalcInc#1!!#3!#1!%
      }%
    \fi
  \else
    \ifx\\#3\\%
      \BitSet@AfterFiFi{#4 }%
    \else
      \BitSet@AfterFiFi{%
        \expandafter\expandafter\expandafter\BitSet@Size
        \IntCalcInc#1!!#3!#4!%
      }%
    \fi
  \fi
  \BitSet@Fi
}
%    \end{macrocode}
%    \end{macro}
%
%    \begin{macro}{\bitsetCardinality}
%    \begin{macrocode}
\def\bitsetCardinality#1{%
  \number
  \BitSet@IfUndefined{#1}{0 }{%
    \expandafter\expandafter\expandafter\BitSet@Cardinality
    \expandafter\expandafter\expandafter0%
    \expandafter\expandafter\expandafter!%
    \csname BS@#1\endcsname!%
  }%
}
%    \end{macrocode}
%    \end{macro}
%    \begin{macro}{\BitSet@Cardinality}
%    |#1|: result\\
%    |#2#3|: bits
%    \begin{macrocode}
\def\BitSet@Cardinality#1!#2#3!{%
  \ifx#21%
    \ifx\\#3\\%
      \BitSet@AfterFiFi{\IntCalcInc#1! }%
    \else
      \BitSet@AfterFiFi{%
        \expandafter\expandafter\expandafter\BitSet@Cardinality
        \IntCalcInc#1!!#3!%
      }%
    \fi
  \else
    \ifx\\#3\\%
      \BitSet@AfterFiFi{#1 }%
    \else
      \BitSet@AfterFiFi{%
        \BitSet@Cardinality#1!#3!%
      }%
    \fi
  \fi
  \BitSet@Fi
}
%    \end{macrocode}
%    \end{macro}
%
% \subsection{Queries}\label{src:query}
%
%    \begin{macro}{\bitsetIsDefined}
%    \begin{macrocode}
\def\bitsetIsDefined#1{%
  \BitSet@IfUndefined{#1}%
  \BitSet@SecondOfTwo
  \BitSet@FirstOfTwo
}
%    \end{macrocode}
%    \end{macro}
%    \begin{macro}{\bitsetIsEmpty}
%    \begin{macrocode}
\def\bitsetIsEmpty#1{%
  \BitSet@IfUndefined{#1}\BitSet@FirstOfTwo{%
    \expandafter\ifx\csname BS@#1\endcsname\BitSet@Zero
      \expandafter\BitSet@FirstOfTwo
    \else
      \expandafter\BitSet@SecondOfTwo
    \fi
  }%
}
%    \end{macrocode}
%    \end{macro}
%    \begin{macro}{\BitSet@Zero}
%    \begin{macrocode}
\def\BitSet@Zero{0}
%    \end{macrocode}
%    \end{macro}
%
%    \begin{macro}{\bitsetQuery}
%    \begin{macrocode}
\def\bitsetQuery#1#2{%
  \ifnum\bitsetGet{#1}{#2}=1 %
    \expandafter\BitSet@FirstOfTwo
  \else
    \expandafter\BitSet@SecondOfTwo
  \fi
}
%    \end{macrocode}
%    \end{macro}
%
%    \begin{macro}{\bitsetEquals}
%    \begin{macrocode}
\def\bitsetEquals#1#2{%
  \BitSet@IfUndefined{#1}{%
    \BitSet@IfUndefined{#2}\BitSet@FirstOfTwo\BitSet@SecondOfTwo
  }{%
    \BitSet@IfUndefined{#2}\BitSet@SecondOfTwo{%
      \expandafter\ifx\csname BS@#1\expandafter\endcsname
                      \csname BS@#2\endcsname
        \expandafter\BitSet@FirstOfTwo
      \else
        \expandafter\BitSet@SecondOfTwo
      \fi
    }%
  }%
}
%    \end{macrocode}
%    \end{macro}
%
%    \begin{macro}{\bitsetIntersects}
%    \begin{macrocode}
\def\bitsetIntersects#1#2{%
  \bitsetIsEmpty{#1}\BitSet@SecondOfTwo{%
    \bitsetIsEmpty{#2}\BitSet@SecondOfTwo{%
      \expandafter\expandafter\expandafter\BitSet@Intersects
      \csname BS@#1\expandafter\expandafter\expandafter\endcsname
      \expandafter\expandafter\expandafter!%
      \csname BS@#2\endcsname!%
    }%
  }%
}
%    \end{macrocode}
%    \end{macro}
%    \begin{macro}{\BitSet@Intersects}
%    \begin{macrocode}
\def\BitSet@Intersects#1#2!#3#4!{%
  \ifnum#1#3=11 %
    \BitSet@AfterFi\BitSet@FirstOfTwo
  \else
    \ifx\\#2\\%
      \BitSet@AfterFiFi\BitSet@SecondOfTwo
    \else
      \ifx\\#4\\%
        \BitSet@AfterFiFiFi\BitSet@SecondOfTwo
      \else
        \BitSet@AfterFiFiFi{%
          \BitSet@Intersects#2!#4!%
        }%
      \fi
    \fi
  \BitSet@Fi
}
%    \end{macrocode}
%    \end{macro}
%
%    \begin{macrocode}
\BitSet@AtEnd%
%</package>
%    \end{macrocode}
%
% \section{Test}
%
% \subsection{Catcode checks for loading}
%
%    \begin{macrocode}
%<*test1>
%    \end{macrocode}
%    \begin{macrocode}
\catcode`\{=1 %
\catcode`\}=2 %
\catcode`\#=6 %
\catcode`\@=11 %
\expandafter\ifx\csname count@\endcsname\relax
  \countdef\count@=255 %
\fi
\expandafter\ifx\csname @gobble\endcsname\relax
  \long\def\@gobble#1{}%
\fi
\expandafter\ifx\csname @firstofone\endcsname\relax
  \long\def\@firstofone#1{#1}%
\fi
\expandafter\ifx\csname loop\endcsname\relax
  \expandafter\@firstofone
\else
  \expandafter\@gobble
\fi
{%
  \def\loop#1\repeat{%
    \def\body{#1}%
    \iterate
  }%
  \def\iterate{%
    \body
      \let\next\iterate
    \else
      \let\next\relax
    \fi
    \next
  }%
  \let\repeat=\fi
}%
\def\RestoreCatcodes{}
\count@=0 %
\loop
  \edef\RestoreCatcodes{%
    \RestoreCatcodes
    \catcode\the\count@=\the\catcode\count@\relax
  }%
\ifnum\count@<255 %
  \advance\count@ 1 %
\repeat

\def\RangeCatcodeInvalid#1#2{%
  \count@=#1\relax
  \loop
    \catcode\count@=15 %
  \ifnum\count@<#2\relax
    \advance\count@ 1 %
  \repeat
}
\def\RangeCatcodeCheck#1#2#3{%
  \count@=#1\relax
  \loop
    \ifnum#3=\catcode\count@
    \else
      \errmessage{%
        Character \the\count@\space
        with wrong catcode \the\catcode\count@\space
        instead of \number#3%
      }%
    \fi
  \ifnum\count@<#2\relax
    \advance\count@ 1 %
  \repeat
}
\def\space{ }
\expandafter\ifx\csname LoadCommand\endcsname\relax
  \def\LoadCommand{\input bitset.sty\relax}%
\fi
\def\Test{%
  \RangeCatcodeInvalid{0}{47}%
  \RangeCatcodeInvalid{58}{64}%
  \RangeCatcodeInvalid{91}{96}%
  \RangeCatcodeInvalid{123}{255}%
  \catcode`\@=12 %
  \catcode`\\=0 %
  \catcode`\%=14 %
  \LoadCommand
  \RangeCatcodeCheck{0}{36}{15}%
  \RangeCatcodeCheck{37}{37}{14}%
  \RangeCatcodeCheck{38}{47}{15}%
  \RangeCatcodeCheck{48}{57}{12}%
  \RangeCatcodeCheck{58}{63}{15}%
  \RangeCatcodeCheck{64}{64}{12}%
  \RangeCatcodeCheck{65}{90}{11}%
  \RangeCatcodeCheck{91}{91}{15}%
  \RangeCatcodeCheck{92}{92}{0}%
  \RangeCatcodeCheck{93}{96}{15}%
  \RangeCatcodeCheck{97}{122}{11}%
  \RangeCatcodeCheck{123}{255}{15}%
  \RestoreCatcodes
}
\Test
\csname @@end\endcsname
\end
%    \end{macrocode}
%    \begin{macrocode}
%</test1>
%    \end{macrocode}
%
% \subsection{Macro tests}
%
% \subsubsection{Preamble}
%
%    \begin{macrocode}
%<*test2>
\NeedsTeXFormat{LaTeX2e}
\nofiles
\documentclass{article}
\makeatletter
%<*noetex>
\let\SavedNumexpr\numexpr
\let\SavedIfcsname\ifcsname
\let\SavedCurrentgrouplevel\currentgrouplevel
\def\ETeXDisable{%
  \let\ifcsname\@undefined
  \let\numexpr\@undefined
  \let\currentgrouplevel\@undefined
}
\ETeXDisable
%</noetex>
\makeatletter
\chardef\BitSet@TestMode=1 %
\makeatother
\usepackage{bitset}[2011/01/30]
%<*noetex>
\def\ETeXEnable{%
  \let\numexpr\SavedNumexpr
  \let\ifcsname\SavedIfcsname
  \let\currentgrouplevel\SavedCurrentgrouplevel
}
\ETeXEnable
%</noetex>
\usepackage{qstest}
\IncludeTests{*}
\LogTests{log}{*}{*}
\makeatletter
%    \end{macrocode}
%
% \subsubsection{Time}
%
%    \begin{macrocode}
\begingroup\expandafter\expandafter\expandafter\endgroup
\expandafter\ifx\csname pdfresettimer\endcsname\relax
\else
  \newcount\SummaryTime
  \newcount\TestTime
  \SummaryTime=\z@
  \newcommand*{\PrintTime}[2]{%
    \typeout{%
      [Time #1: \strip@pt\dimexpr\number#2sp\relax\space s]%
    }%
  }%
  \newcommand*{\StartTime}[1]{%
    \renewcommand*{\TimeDescription}{#1}%
    \pdfresettimer
  }%
  \newcommand*{\TimeDescription}{}%
  \newcommand*{\StopTime}{%
    \TestTime=\pdfelapsedtime
    \global\advance\SummaryTime\TestTime
    \PrintTime\TimeDescription\TestTime
  }%
  \let\saved@qstest\qstest
  \let\saved@endqstest\endqstest
  \def\qstest#1#2{%
    \saved@qstest{#1}{#2}%
    \StartTime{#1}%
  }%
  \def\endqstest{%
    \StopTime
    \saved@endqstest
  }%
  \AtEndDocument{%
    \PrintTime{summary}\SummaryTime
  }%
\fi
%    \end{macrocode}
%
% \subsubsection{Detection of unwanted space}
%
%    \begin{macrocode}
\let\orig@qstest\qstest
\let\orig@endqstest\endqstest
\def\qstest#1#2{%
  \orig@qstest{#1}{#2}%
  \setbox0\hbox\bgroup\begingroup\ignorespaces
}
\def\endqstest{%
  \endgroup\egroup
  \Expect*{\the\wd0}{0.0pt}%
  \orig@endqstest
}
%    \end{macrocode}
%
% \subsubsection{Test macros}
%
%    \begin{macrocode}
\newcounter{Test}

\def\TestError#1#2{%
  \begingroup
    \setcounter{Test}{0}%
    \sbox0{%
      \def\@PackageError##1##2##3{%
        \stepcounter{Test}%
        \begingroup
          \let\MessageBreak\relax
%<*noetex>
          \ETeXEnable
%</noetex>
          \Expect{##1}{bitset}%
          \Expect*{##2}*{#1}%
        \endgroup
      }%
%<*noetex>
      \ETeXDisable
%</noetex>
      #2%
    }%
    \Expect*{\theTest}{1}%
    \Expect*{\the\wd0}{0.0pt}%
  \endgroup
}

\def\TestErrorNegativeIndex#1#2{%
  \TestError{Invalid negative index (#1)}{#2}%
}

\def\TestGetterUndefined#1{%
  \CheckUndef{dummy}%
  \expandafter\expandafter\expandafter\Expect
  \expandafter\expandafter\expandafter{#1{dummy}}{0}%
}

\def\ExpectBitSet#1#2{%
  \expandafter\expandafter\expandafter\Expect
  \expandafter\expandafter\expandafter
  {\csname BS@#1\endcsname}*{#2}%
}
\def\Check#1#2{%
  \ExpectBitSet{#1}{#2}%
}
\def\CheckUndef#1{%
  \begingroup
    \Expect*{%
      \expandafter
      \ifx\csname BS@#1\endcsname\relax true\else false\fi
    }{true}%
  \endgroup
}
\def\RevCheck#1#2{%
  \ExpectBitSet{#1}{\Reverse#2!!}%
}
\def\Set#1#2{%
  \expandafter\def\csname BS@#1\endcsname{#2}%
}
\def\RevSet#1#2{%
  \expandafter\edef\csname BS@#1\endcsname{%
    \Reverse#2!!%
  }%
}
\def\Reverse#1#2!#3!{%
  \ifx\\#2\\%
    #1#3%
    \expandafter\@gobble
  \else
    \expandafter\@firstofone
  \fi
  {\Reverse#2!#1#3!}%
}
%    \end{macrocode}
%
% \subsubsection{Test sets}
%
%    \begin{macrocode}
\begin{qstest}{Let}{Let}
  \CheckUndef{abc}%
  \CheckUndef{xyz}%
  \bitsetLet{xyz}{abc}%
  \CheckUndef{abc}%
  \Check{xyz}{0}%
  \Set{abc}{1}%
  \Check{abc}{1}%
  \Check{xyz}{0}%
  \bitsetLet{xyz}{abc}%
  \Check{abc}{1}%
  \Check{xyz}{1}%
  \Set{xyz}{11}%
  \Check{abc}{1}%
  \Check{xyz}{11}%
\end{qstest}

\begin{qstest}{Reset}{Reset}
  \bitsetReset{xyz}%
  \Check{xyz}{0}%
  \bitsetReset{abc}%
  \Check{abc}{0}%
  \Set{abc}{10101}%
  \bitsetReset{abc}%
  \Check{abc}{0}%
\end{qstest}

\begin{qstest}{Get/Query}{Get/Query}
  \expandafter\expandafter\expandafter\Expect
  \expandafter\expandafter\expandafter{%
    \bitsetGet{dummy}{0}%
  }{0}%
  \begingroup
    \expandafter\def\csname BitSetError:NegativeIndex\endcsname{}%
    \Set{abc}{1}%
    \Expect*{\bitsetQuery{abc}{-1}{true}{false}}{false}%
  \endgroup
  \def\Test#1#2#3{%
    \Set{abc}{#1}%
    \expandafter\expandafter\expandafter\Expect
    \expandafter\expandafter\expandafter{\bitsetGet{abc}{#2}}{#3}%
    \Expect*{\bitsetQuery{abc}{#2}{true}{false}}%
           *{\ifcase#3 false\or true\else error\fi}%
  }%
  \Test{1}{100}{0}%
  \Test{0}{0}{0}%
  \Test{1}{0}{1}%
  \Test{11}{1}{1}%
  \Test{111}{1}{1}%
  \Test{101}{1}{0}%
  \Test{101}{2}{1}%
  \Test{10100110011}{10}{1}%
\end{qstest}

\begin{qstest}{Size}{Size}
  \TestGetterUndefined\bitsetSize
  \def\Test#1#2{%
    \Set{abc}{#1}%
    \expandafter\expandafter\expandafter\Expect
    \expandafter\expandafter\expandafter{\bitsetSize{abc}}{#2}%
  }%
  \Test{0}{0}%
  \Test{1}{1}%
  \Test{00}{0}%
  \Test{0000000}{0}%
  \Test{10}{1}%
  \Test{01}{2}%
  \Test{11}{2}%
  \Test{010}{2}%
  \Test{011}{3}%
  \Test{100110011}{9}%
  \Test{0000011111000001111100000}{20}%
  \Test{000000000000000000000000011111111111111111111}{45}%
\end{qstest}

\begin{qstest}{Cardinality}{Cardinality}
  \TestGetterUndefined\bitsetCardinality
  \def\Test#1#2{%
    \Set{abc}{#1}%
    \expandafter\expandafter\expandafter\Expect
    \expandafter\expandafter\expandafter{%
      \bitsetCardinality{abc}%
    }{#2}%
  }%
  \Test{0}{0}%
  \Test{1}{1}%
  \Test{00}{0}%
  \Test{0000000}{0}%
  \Test{10}{1}%
  \Test{01}{1}%
  \Test{11}{2}%
  \Test{010}{1}%
  \Test{011}{2}%
  \Test{100110011}{5}%
  \Test{0000011111000001111100000}{10}%
  \Test{000000000000000000000000011111111111111111111}{20}%
\end{qstest}

\begin{qstest}{NextClearBit/NextSetBit}{NextClearBit/NextSetBit}
  \def\Test#1#2{%
    \expandafter\expandafter\expandafter\Expect
    \expandafter\expandafter\expandafter{%
      \TestOp{abc}{#1}%
    }{#2}%
  }%
  \def\Clear{\let\TestOp\bitsetNextClearBit}%
  \def\Set{\let\TestOp\bitsetNextSetBit}%
  \begingroup
    \catcode`\:=11 %
    \bitsetSetBin{abc}{1}%
    \Clear
    \Test{-1}{1\BitSetError:NegativeIndex}%
    \Set
    \Test{-1}{0\BitSetError:NegativeIndex}%
  \endgroup
  \let\BS@abc\@undefined
  \Clear
  \Test{0}{0}%
  \Test{1}{1}%
  \Test{2}{2}%
  \Test{100}{100}%
  \Set
  \Test{0}{-1}%
  \Test{1}{-1}%
  \Test{100}{-1}%
  \bitsetReset{abc}%
  \Clear
  \Test{0}{0}%
  \Test{1}{1}%
  \Test{2}{2}%
  \Test{100}{100}%
  \Set
  \Test{0}{-1}%
  \Test{1}{-1}%
  \Test{100}{-1}%
  \bitsetSetBin{abc}{1}%
  \Clear
  \Test{0}{1}%
  \Test{1}{1}%
  \Test{2}{2}%
  \Test{100}{100}%
  \Set
  \Test{0}{0}%
  \Test{1}{-1}%
  \Test{100}{-1}%
  \bitsetSetBin{abc}{111000111000111000111}%
  \Clear
  \Test{0}{3}%
  \Test{1}{3}%
  \Test{2}{3}%
  \Test{3}{3}%
  \Test{4}{4}%
  \Test{5}{5}%
  \Test{6}{9}%
  \Test{7}{9}%
  \Test{8}{9}%
  \Test{9}{9}%
  \Test{10}{10}%
  \Test{11}{11}%
  \Test{12}{15}%
  \Test{13}{15}%
  \Test{14}{15}%
  \Test{15}{15}%
  \Test{16}{16}%
  \Test{17}{17}%
  \Test{18}{21}%
  \Test{19}{21}%
  \Test{20}{21}%
  \Test{21}{21}%
  \Test{22}{22}%
  \Test{100}{100}%
  \Set
  \Test{0}{0}%
  \Test{1}{1}%
  \Test{2}{2}%
  \Test{3}{6}%
  \Test{4}{6}%
  \Test{5}{6}%
  \Test{6}{6}%
  \Test{7}{7}%
  \Test{8}{8}%
  \Test{9}{12}%
  \Test{10}{12}%
  \Test{11}{12}%
  \Test{12}{12}%
  \Test{13}{13}%
  \Test{14}{14}%
  \Test{15}{18}%
  \Test{16}{18}%
  \Test{17}{18}%
  \Test{18}{18}%
  \Test{19}{19}%
  \Test{20}{20}%
  \Test{21}{-1}%
  \Test{22}{-1}%
  \Test{100}{-1}%
  \bitsetSetBin{abc}{1111111}%
  \Clear
  \Test{6}{7}%
  \Test{7}{7}%
  \Test{8}{8}%
  \Test{100}{100}%
  \Set
  \Test{6}{6}%
  \Test{7}{-1}%
  \Test{8}{-1}%
  \Test{100}{-1}%
  \bitsetSetBin{abc}{11111111}%
  \Clear
  \Test{7}{8}%
  \Test{8}{8}%
  \Test{9}{9}%
  \Test{100}{100}%
  \Set
  \Test{7}{7}%
  \Test{8}{-1}%
  \Test{9}{-1}%
  \Test{100}{-1}%
  \bitsetSetBin{abc}{111111111}%
  \Clear
  \Test{8}{9}%
  \Test{9}{9}%
  \Test{10}{10}%
  \Test{100}{100}%
  \Set
  \Test{8}{8}%
  \Test{9}{-1}%
  \Test{10}{-1}%
  \Test{100}{-1}%
  \bitsetSetBin{abc}{1111111111}%
  \Clear
  \Test{9}{10}%
  \Test{10}{10}%
  \Test{11}{11}%
  \Test{100}{100}%
  \Set
  \Test{9}{9}%
  \Test{10}{-1}%
  \Test{11}{-1}%
  \Test{100}{-1}%
\end{qstest}

\begin{qstest}{GetSetBitList}{GetSetBitList}
  \let\BS@abc\@undefined
  \expandafter\expandafter\expandafter\Expect
  \expandafter\expandafter\expandafter{%
    \bitsetGetSetBitList{abc}%
  }{}%
  \def\Test#1#2{%
    \bitsetSetBin{abc}{#1}%
    \expandafter\expandafter\expandafter\Expect
    \expandafter\expandafter\expandafter{%
      \bitsetGetSetBitList{abc}%
    }{#2}%
  }%
  \Test{0}{}%
  \Test{1}{0}%
  \Test{10}{1}%
  \Test{11}{0,1}%
  \Test{10110100}{2,4,5,7}%
  \Test{101101001010011}{0,1,4,6,9,11,12,14}%
\end{qstest}

\begin{qstest}{GetDec}{GetDec}
  \TestGetterUndefined\bitsetGetDec
  \def\Test#1#2{%
    \RevSet{abc}{#1}%
%<*noetex>
    \begingroup\expandafter\expandafter\expandafter\endgroup
%</noetex>
    \expandafter\expandafter\expandafter\Expect
    \expandafter\expandafter\expandafter{%
      \bitsetGetDec{abc}%
    }{#2}%
  }%
  \Test{0}{0}%
  \Test{1}{1}%
  \Test{10}{2}%
  \Test{11}{3}%
  \Test{100}{4}%
  \Test{101}{5}%
  \Test{110}{6}%
  \Test{111}{7}%
  \Test{1000}{8}%
  \Test{000111}{7}%
  \Test{1111111111111111%
        111111111111111}{2147483647}%
  \Test{0001111111111111111%
        111111111111111}{2147483647}%
  \Test{1000000000000000%
        0000000000000000}{2147483648}%
  \Test{1000000000000000%
        00000000000000000}{4294967296}%
  \Test{0001000000000000000%
        00000000000000000}{4294967296}%
  \Test{1100000000000000%
        00000000000000011}{6442450947}%
\end{qstest}

\begin{qstest}{Clear}{Clear}
  \def\Test#1#2#3{%
    \RevSet{abc}{#1}%
    \bitsetClear{abc}{#2}%
    \Expect*{\BS@abc}*{\Reverse#3!!}%
  }%
  \bitsetClear{abc}{2}%
  \RevCheck{abc}{0}%
  \TestErrorNegativeIndex{-1}{\bitsetClear{abc}{-1}}%
  \RevCheck{abc}{0}%
  \Test{0}{0}{0}%
  \Test{1}{0}{0}%
  \Test{111}{1}{101}%
  \Test{111}{30}{111}%
  \Test{0000111}{5}{0000111}% 111 would also be ok
  \Test{10000111}{5}{10000111}%
  \Test{1001001}{3}{1000001}%
\end{qstest}

\begin{qstest}{Set}{Set}
  \def\Test#1#2#3{%
    \RevSet{abc}{#1}%
    \bitsetSet{abc}{#2}%
    \Expect*{\BS@abc}*{\Reverse#3!!}%
  }%
  \bitsetSet{abc}{2}%
  \RevCheck{abc}{100}%
  \TestErrorNegativeIndex{-1}{\bitsetSet{abc}{-1}}%
  \RevCheck{abc}{100}%
  \Test{0}{0}{1}%
  \Test{1}{0}{1}%
  \Test{100}{1}{110}%
  \Test{111}{1}{111}%
  \Test{11}{1}{11}%
  \Test{11}{2}{111}%
  \Test{11}{3}{1011}%
  \Test{111}{10}{10000000111}%
  \Test{0000111}{5}{0100111}% 100111 would also be ok
  \Test{10000111}{5}{10100111}%
  \Test{1000001}{3}{1001001}%
  \Test{1001001}{3}{1001001}%
\end{qstest}

\begin{qstest}{Flip}{Flip}
  \def\Test#1#2#3{%
    \RevSet{abc}{#1}%
    \bitsetFlip{abc}{#2}%
    \Expect*{\BS@abc}*{\Reverse#3!!}%
  }%
  \bitsetFlip{abc}{2}%
  \RevCheck{abc}{100}%
  \TestErrorNegativeIndex{-1}{\bitsetFlip{abc}{-1}}%
  \RevCheck{abc}{100}%
  \Test{0}{0}{1}%
  \Test{1}{0}{0}%
  \Test{0}{2}{100}%
  \Test{100}{1}{110}%
  \Test{111}{1}{101}%
  \Test{11}{1}{1}%
  \Test{11}{2}{111}%
  \Test{11}{3}{1011}%
  \Test{111}{10}{10000000111}%
  \Test{0000111}{5}{0100111}% 100111 would also be ok
  \Test{10000111}{5}{10100111}%
  \Test{1000001}{3}{1001001}%
  \Test{1001001}{3}{1000001}%
  \Test{11111}{2}{11011}%
\end{qstest}

\begin{qstest}{SetValue}{SetValue}
  \def\Test#1#2{%
    \TestError{Invalid bit value (#2) not in range 0..1}{%
      \bitsetSetValue{abc}{#1}{#2}%
    }%
  }%
  \Test{0}{-1}%
  \Test{0}{2}%
  \Test{0}{10}%
  \def\Test#1#2#3{%
    \let\BS@abc\@undefined
    \bitsetSetValue{abc}{#1}{#2}%
    \bitsetSetBin{result}{#3}%
    \Expect*{\BS@abc}*{\BS@result}%
  }%
  \Test{0}{0}{0}%
  \Test{0}{1}{1}%
  \Test{1}{0}{0}%
  \Test{1}{1}{10}%
  \def\Test#1#2#3#4{%
    \bitsetSetBin{abc}{#1}%
    \bitsetSetBin{result}{#4}%
    \bitsetSetValue{abc}{#2}{#3}%
    \Expect*{\BS@abc}*{\BS@result}%
  }%
  \Test{0}{0}{0}{0}%
  \Test{0}{0}{0}{0}%
  \Test{0}{0}{1}{1}%
  \Test{0}{1}{0}{0}%
  \Test{0}{1}{1}{10}%
  \Test{1010}{2}{1}{1110}%
  \Test{1010}{4}{1}{11010}%
  \Test{1010}{6}{1}{1001010}%
  \Test{1010}{1}{0}{1000}%
  \Test{1010}{2}{0}{1010}%
  \Test{1010}{3}{0}{10}%
  \Test{1010}{4}{0}{1010}%
  \Test{1010}{6}{0}{1010}%
  \Test{1010}{2}{\csname iffalse\endcsname 0\else 1\fi}{1110}%
  \Test{1010}{1}{\csname iffalse\endcsname 1\else 0\fi}{1000}%
\end{qstest}

\begin{qstest}{IsDefined}{IsDefined}
  \let\BS@abc\@undefined
  \Expect*{\bitsetIsDefined{abc}{true}{false}}{false}%
  \bitsetReset{abc}%
  \Expect*{\bitsetIsDefined{abc}{true}{false}}{true}%
\end{qstest}

\begin{qstest}{IsEmpty}{IsEmpty}
  \let\BS@abc\@undefined
  \Expect*{\bitsetIsEmpty{abc}{true}{false}}{true}%
  \bitsetReset{abc}%
  \Expect*{\bitsetIsEmpty{abc}{true}{false}}{true}%
  \bitsetSet{abc}{1}%
  \Expect*{\bitsetIsEmpty{abc}{true}{false}}{false}%
\end{qstest}

\begin{qstest}{Equals}{Equals}
  \def\Test#1#2#3{%
    \Expect*{\bitsetEquals{#1}{#2}{true}{false}}{#3}%
  }%
  \let\BS@abc\@undefined
  \Test{abc}{abc}{true}%
  \Test{abc}{foo}{true}%
  \Test{foo}{abc}{true}%
  \bitsetReset{abc}%
  \Test{abc}{abc}{true}%
  \Test{abc}{foo}{false}%
  \Test{foo}{abc}{false}%
  \bitsetReset{foo}%
  \Test{abc}{foo}{true}%
  \Test{foo}{abc}{true}%
  \bitsetSet{abc}{4}%
  \Test{abc}{foo}{false}%
  \Test{foo}{abc}{false}%
  \bitsetFlip{foo}{4}%
  \Test{abc}{foo}{true}%
  \Test{foo}{abc}{true}%
\end{qstest}

\begin{qstest}{Intersects}{Intersects}
  \def\Test#1{%
    \Expect*{\bitsetIntersects{abc}{foo}{true}{false}}{#1}%
  }%
  \let\BS@abc\@undefined
  \let\BS@foo\@undefined
  \Test{false}%
  \Set{abc}{0}%
  \Test{false}%
  \Set{foo}{0}%
  \Test{false}%
  \let\BS@abc\@undefined
  \Test{false}%
  \Set{foo}{1}%
  \Test{false}%
  \Set{abc}{0}%
  \Test{false}%
  \Set{abc}{1}%
  \Test{true}%
  \let\BS@foo\@undefined
  \Test{false}%
  \Set{foo}{0}%
  \Test{false}%
  \def\Test#1#2#3{%
    \bitsetSetBin{abc}{#1}%
    \bitsetSetBin{foo}{#2}%
    \Expect*{\bitsetIntersects{abc}{foo}{true}{false}}{#3}%
  }%
  \Test{1010}{0101}{false}%
  \Test{0}{10}{false}%
  \Test{1}{11}{true}%
  \Test{11}{1}{true}%
  \Test{10}{1}{false}%
\end{qstest}

\begin{qstest}{And/AndNot/Or/Xor}{And/AndNot/Or/Xor}
  \def\@Test#1#2#3#4#5{%
    \begingroup
      #5%
      \begingroup
        \let\BS@foo\@undefined
        \csname bitset#1\endcsname{abc}{foo}%
        \CheckUndef{foo}%
        \Check{abc}{#2}%
      \endgroup
      \begingroup
        \bitsetReset{foo}%
        \csname bitset#1\endcsname{abc}{foo}%
        \Check{foo}{0}%
        \Check{abc}{#3}%
      \endgroup
      \begingroup
        \def\BS@foo{0101}%
        \csname bitset#1\endcsname{abc}{foo}%
        \Check{foo}{0101}%
        \Check{abc}{#4}%
      \endgroup
    \endgroup
  }%
  \def\Test#1{%
    \def\Op{#1}%
    \Test@
  }%
  \def\Test@#1#2#3#4#5#6#7#8#9{%
    \@Test\Op{#1}{#2}{#3}{%
      \let\BS@abc\@undefined
    }%
    \@Test\Op{#4}{#5}{#6}{%
      \bitsetReset{abc}%
    }%
    \@Test\Op{#7}{#8}{#9}{%
      \def\BS@abc{1001}%
    }%
  }%
  \Test{And}%
       {0}{0}{0}%
       {0}{0}{0}%
       {0}{0}{0001}%
  \Test{AndNot}%
       {0}{0}{0}%
       {0}{0}{0}%
       {1001}{1001}{1}%
  \Test{Or}%
       {0}{0}{0101}%
       {0}{0}{0101}%
       {1001}{1001}{1101}%
  \Test{Xor}%
       {0}{0}{0101}%
       {0}{0}{0101}%
       {1001}{1001}{11}%
  \def\Test#1#2#3{%
    \bitsetSetBin{abc}{#1}%
    \bitsetSetBin{foo}{#2}%
    \csname bitset\Op\endcsname{abc}{foo}%
    \RevCheck{foo}{#2}%
    \RevCheck{abc}{#3}%
  }%
  \def\Op{And}%
  \Test{1}{111}{1}%
  \Test{111}{1}{1}%
  \Test{10}{111}{10}%
  \Test{111}{10}{10}%
  \Test{111}{1000}{0}%
  \Test{1000}{111}{0}%
  \def\Op{AndNot}%
  \Test{1010}{11}{1000}%
  \Test{100}{100}{0}%
  \Test{111}{1111}{0}%
  \Test{100}{111}{0}%
  \def\Op{Or}%
  \Test{0}{0}{0}%
  \Test{1}{0}{1}%
  \Test{0}{1}{1}%
  \Test{1}{1}{1}%
  \Test{1000}{10}{1010}%
  \Test{10}{1000}{1010}%
  \def\Op{Xor}%
  \Test{0}{0}{0}%
  \Test{1}{0}{1}%
  \Test{0}{1}{1}%
  \Test{1}{1}{0}%
  \Test{1000}{10}{1010}%
  \Test{10}{1000}{1010}%
  \Test   {110011001100}%
       {111000111000111}%
       {111110100001011}%
  \Test{111000111000111}%
          {110011001100}%
       {111110100001011}%
\end{qstest}

\begin{qstest}{GetUndef}{GetUndef, GetBin, GetOct, GetHex}
  \def\TestUndef#1#2{%
    \let\BS@abc\@undefined
    \expandafter\expandafter\expandafter\Expect
    \expandafter\expandafter\expandafter{%
      \x{abc}{#1}%
    }{#2}%
  }%
  \let\x\bitsetGetBin
  \TestUndef{-1}{0}%
  \TestUndef{0}{0}%
  \TestUndef{1}{0}%
  \TestUndef{2}{00}%
  \TestUndef{8}{00000000}%
  \let\x\bitsetGetOct
  \TestUndef{-1}{0}%
  \TestUndef{0}{0}%
  \TestUndef{1}{0}%
  \TestUndef{2}{0}%
  \TestUndef{3}{0}%
  \TestUndef{4}{00}%
  \TestUndef{5}{00}%
  \TestUndef{6}{00}%
  \TestUndef{7}{000}%
  \TestUndef{8}{000}%
  \TestUndef{9}{000}%
  \TestUndef{10}{0000}%
  \let\x\bitsetGetHex
  \TestUndef{-1}{0}%
  \TestUndef{0}{0}%
  \TestUndef{1}{0}%
  \TestUndef{2}{0}%
  \TestUndef{3}{0}%
  \TestUndef{4}{0}%
  \TestUndef{5}{00}%
  \TestUndef{6}{00}%
  \TestUndef{7}{00}%
  \TestUndef{8}{00}%
  \TestUndef{9}{000}%
  \TestUndef{10}{000}%
  \TestUndef{12}{000}%
  \TestUndef{13}{0000}%
  \TestUndef{16}{0000}%
  \TestUndef{17}{00000}%
\end{qstest}

\begin{qstest}{SetBin}{SetBin}
  \def\Test#1#2{%
    \let\BS@abc\@undefined
    \bitsetSetBin{abc}{#1}%
    \expandafter\Expect\expandafter{\BS@abc}{#2}%
  }%
  \Test{}{0}%
  \Test{0}{0}%
  \Test{1}{1}%
  \Test{10}{01}%
  \Test{11}{11}%
  \Test{010}{01}%
  \Test{011}{11}%
  \Test{0010}{01}%
  \Test{1010}{0101}%
\end{qstest}

\begin{qstest}{SetOct}{SetOct}
  \def\Test#1#2{%
    \bitsetSetOct{abc}{#1}%
    \expandafter\Expect\expandafter{\BS@abc}{#2}%
  }%
  \Test{}{0}%
  \Test{0}{0}%
  \Test{000}{0}%
  \Test{1}{1}%
  \Test{001}{1}%
  \Test{010}{0001}%
  \Test{020}{00001}%
  \Test{42}{010001}%
  \Test{377}{11111111}%
  \Test{0377}{11111111}%
  \Test{76543210}{000100010110001101011111}%
  \Test{ 0 7 0 7 1 }{100111000111}%
\end{qstest}

\begin{qstest}{SetHex}{SetHex}
  \def\Test#1#2{%
    \bitsetSetHex{abc}{#1}%
    \expandafter\Expect\expandafter{\BS@abc}{#2}%
  }%
  \Test{}{0}%
  \Test{0}{0}%
  \Test{000}{0}%
  \Test{1}{1}%
  \Test{001}{1}%
  \Test{010}{00001}%
  \Test{020}{000001}%
  \Test{42}{0100001}%
  \Test{3F}{111111}%
  \Test{03F}{111111}%
  \Test{43210}{0000100001001100001}%
  \Test{98765}{10100110111000011001}%
  \Test{FEDCBA}{010111010011101101111111}%
  \Test{ 0 F 0 F 1 }{1000111100001111}%
\end{qstest}

\begin{qstest}{SetDec}{SetDec}
  \def\Test#1#2{%
    \bitsetSetDec{abc}{#1}%
    \expandafter\Expect\expandafter{\BS@abc}{#2}%
  }%
  \Test{}{0}%
  \Test{0}{0}%
  \Test{000}{0}%
  \Test{1}{1}%
  \Test{7}{111}%
  \Test{8}{0001}%
  \Test{001}{1}%
  \Test{010}{0101}%
  \Test{020}{00101}%
  \Test{53}{101011}%
  \Test{255}{11111111}%
  \Test{256}{000000001}%
  \Test{999999999}{111111111001001101011001110111}%
  \Test{1000000000}{000000000101001101011001110111}%
  \Test{4210987654}{01100001010010010111111101011111}%
  \Test{2147483647}{1111111111111111111111111111111}%
  \Test{2147483648}{00000000000000000000000000000001}%
\end{qstest}

\begin{qstest}{GetBin}{GetBin}
  \def\TestUndef#1#2{%
    \let\BS@abc\@undefined
    \expandafter\expandafter\expandafter\Expect
    \expandafter\expandafter\expandafter{%
      \bitsetGetBin{abc}{#1}%
    }{#2}%
  }%
  \TestUndef{-1}{0}%
  \TestUndef{0}{0}%
  \TestUndef{1}{0}%
  \TestUndef{2}{00}%
  \TestUndef{8}{00000000}%
  \def\Test#1#2{%
    \bitsetSetBin{abc}{#2}%
    \expandafter\expandafter\expandafter\Expect
    \expandafter\expandafter\expandafter{%
      \bitsetGetBin{abc}{#1}%
    }{#2}%
  }%
  \Test{-1}{0}%
  \Test{0}{0}%
  \Test{1}{0}%
  \Test{1}{1}%
  \Test{2}{01}%
  \Test{2}{10}%
  \Test{3}{010}%
  \Test{2}{00}%
  \Test{2}{01}%
  \Test{8}{00101100}%
  \Test{2}{10101}%
  \Test{-100}{11011}%
\end{qstest}

\begin{qstest}{GetOct}{GetOct}
  \def\Test#1#2#3{%
    \edef\x{\zap@space#1 \@empty}%
    \edef\x{\noexpand\bitsetSetBin{abc}{\x}}%
    \x
    \expandafter\expandafter\expandafter\Expect
    \expandafter\expandafter\expandafter{%
      \bitsetGetOct{abc}{#2}%
    }{#3}%
  }%
  \Test{111 110 101 100 011 010 001 000}{0}{76543210}%
  \Test{000 111}{0}{7}%
  \Test{101 000}{-1}{50}%
  \Test{111}{-1}{7}%
  \Test{111}{0}{7}%
  \Test{111}{1}{7}%
  \Test{111}{3}{7}%
  \Test{111}{4}{07}%
  \Test{111}{6}{07}%
  \Test{111}{7}{007}%
  \Test{111 010}{6}{72}%
  \Test{111 010}{7}{072}%
  \Test{011 111}{0}{37}%
  \Test{011 111}{6}{37}%
  \Test{011 111}{7}{037}%
  \Test{001 111}{0}{17}%
  \Test{001 111}{6}{17}%
  \Test{001 111}{7}{017}%
\end{qstest}

\begin{qstest}{GetHex}{GetHex}
  \def\Test#1#2#3{%
    \bitsetSetBin{abc}{#1}%
    \expandafter\expandafter\expandafter\Expect
    \expandafter\expandafter\expandafter{%
      \bitsetGetHex{abc}{#2}%
    }{#3}%
  }%
  \Test{1111 1110 1101 1100 1011 1010 1001 1000}{0}{FEDCBA98}%
  \Test{0111 0110 0101 0100 0011 0010 0001 0000}{0}{76543210}%
  \Test{0000 1111}{0}{F}%
  \Test{0101 0000}{-1}{50}%
  \Test{1111}{-1}{F}%
  \Test{1111}{0}{F}%
  \Test{1111}{1}{F}%
  \Test{1111}{4}{F}%
  \Test{1111}{5}{0F}%
  \Test{1111}{8}{0F}%
  \Test{1111}{9}{00F}%
  \Test{1111 0010}{8}{F2}%
  \Test{1111 0010}{9}{0F2}%
  \Test{0111 1111}{0}{7F}%
  \Test{0111 1111}{8}{7F}%
  \Test{0111 1111}{9}{07F}%
  \Test{0011 1111}{0}{3F}%
  \Test{0011 1111}{8}{3F}%
  \Test{0011 1111}{9}{03F}%
  \Test{0001 1111}{0}{1F}%
  \Test{0001 1111}{8}{1F}%
  \Test{0001 1111}{9}{01F}%
\end{qstest}

\begin{qstest}{Range}{Range}
  \TestError{%
    Wrong index numbers in range [9..8]\MessageBreak% hash-ok
    for clear/set/flip on bit set `abc'.\MessageBreak
    The lower index exceeds the upper index.\MessageBreak
    Canceling the operation as error recovery%
  }{%
    \bitsetSetRange{abc}{9}{8}%
  }%
  \def\TestErrorNegInd#1#2#3#4#5#6{%
    \TestError{%
      Negative index in range [#2..#3]\MessageBreak % hash-ok
      for \string\bitset #1Range on bit set `abc'.\MessageBreak
      Using [#4..#5] as error recovery% hash-ok
    }{%
      \csname bitset#1Range\endcsname{abc}{#2}{#3}%
      \global\let\BS@global\BS@abc
    }%
    \Check{global}{#6}%
  }%
  \Set{abc}{111}%
  \TestErrorNegInd{Clear}{-1}{0}{0}{0}{111}%
  \TestErrorNegInd{Clear}{0}{-1}{0}{0}{111}%
  \TestErrorNegInd{Clear}{-2}{2}{0}{2}{001}%
  \bitsetReset{abc}%
  \TestErrorNegInd{Set}{-1}{0}{0}{0}{0}%
  \TestErrorNegInd{Set}{0}{-1}{0}{0}{0}%
  \TestErrorNegInd{Set}{-2}{2}{0}{2}{11}%
  \Set{abc}{101}%
  \TestErrorNegInd{Flip}{-1}{0}{0}{0}{101}%
  \TestErrorNegInd{Flip}{0}{-1}{0}{0}{101}%
  \TestErrorNegInd{Flip}{-2}{2}{0}{2}{011}%
  \def\Test#1#2#3#4{%
    \bitsetSetBin{abc}{#1}%
    \csname bitset\TestOp Range\endcsname{abc}{#2}{#3}%
    \Expect*{\bitsetGetBin{abc}{0}}{#4}%
  }%
  \def\TestOp{Clear}%
  \Test{0}{0}{1}{0}%
  \Test{1111}{1}{2}{1101}%
  \Test{1111}{1}{3}{1001}%
  \Test{1111111100000000}{12}{14}{1100111100000000}%
  \def\TestOp{Set}%
  \Test{0}{0}{1}{1}%
  \Test{1000}{1}{2}{1010}%
  \Test{0}{1}{2}{10}%
  \Test{1}{12}{15}{111000000000001}%
  \Test{1111}{1}{3}{1111}%
  \Test{1000000000000000}{12}{14}{1011000000000000}%
  \def\TestOp{Flip}%
  \Test{0}{0}{1}{1}%
  \Test{1}{0}{1}{0}%
  \Test{10101010}{1}{5}{10110100}%
  \def\Test#1#2#3#4#5{%
    \bitsetSetBin{abc}{#1}%
    \bitsetSetValueRange{abc}{#2}{#3}{#4}%
    \Expect*{\bitsetGetBin{abc}{0}}{#5}%
  }%
  \Test{0}{0}{1}{0}{0}%
  \Test{0}{0}{1}{1}{1}%
  \Test{1010}{1}{3}{0}{1000}%
  \Test{1010}{1}{3}{1}{1110}%
\end{qstest}

\begin{qstest}{ShiftLeft/ShiftRight}{ShiftLeft/ShiftRight}
  \def\@Test#1#2{%
    \let\BS@abc\@undefined
    \csname bitsetShift#1\endcsname{abc}{#2}%
    \Expect*{\BS@abc}{0}%
  }%
  \def\Test#1{%
    \@Test{Left}{#1}%
    \@Test{Right}{#1}%
  }%
  \Test{-16}%
  \Test{-1}%
  \Test{0}%
  \Test{1}%
  \Test{16}%
  \def\Test#1#2#3{%
    \bitsetSetBin{abc}{#1}%
    \bitsetSetBin{result}{#3}%
    \csname bitsetShift\Op\endcsname{abc}{#2}%
    \Expect*{\bitsetGetBin{abc}{0}}*{\bitsetGetBin{result}{0}}%
  }%
  \def\Op{Left}%
  \Test{0}{0}{0}%
  \Test{0}{1}{0}%
  \Test{0}{-1}{0}%
  \Test{1}{0}{1}%
  \Test{1}{1}{10}%
  \Test{1}{-1}{0}%
  \Test{10}{1}{100}%
  \Test{10}{-1}{1}%
  \Test{1}{32}{100000000000000000000000000000000}%
  \Test{1}{-100}{0}%
  \def\Op{Right}%
  \Test{0}{0}{0}%
  \Test{0}{1}{0}%
  \Test{0}{-1}{0}%
  \Test{1}{0}{1}%
  \Test{1}{1}{0}%
  \Test{1}{-1}{10}%
  \Test{10}{1}{1}%
  \Test{10}{-1}{100}%
  \Test{1}{-32}{100000000000000000000000000000000}%
  \Test{1}{100}{0}%
  \Test{110110110110110}{10}{11011}%
  \Test{110110110110110}{100}{0}%
  \Test{1}{100000}{0}%
\end{qstest}

\begin{qstest}{Profile: Set}{Profile: Set}
  \bitsetSet{abc}{4095}%
  \global\let\BS@global\BS@abc
\end{qstest}

\begin{qstest}{Profile: Get}{Profile: Get}
  \edef\x{\bitsetGet{global}{4095}}%
\end{qstest}

\begin{document}
\end{document}
%</test2>
%    \end{macrocode}
%
% \section{Installation}
%
% \subsection{Download}
%
% \paragraph{Package.} This package is available on
% CTAN\footnote{\url{ftp://ftp.ctan.org/tex-archive/}}:
% \begin{description}
% \item[\CTAN{macros/latex/contrib/oberdiek/bitset.dtx}] The source file.
% \item[\CTAN{macros/latex/contrib/oberdiek/bitset.pdf}] Documentation.
% \end{description}
%
%
% \paragraph{Bundle.} All the packages of the bundle `oberdiek'
% are also available in a TDS compliant ZIP archive. There
% the packages are already unpacked and the documentation files
% are generated. The files and directories obey the TDS standard.
% \begin{description}
% \item[\CTAN{install/macros/latex/contrib/oberdiek.tds.zip}]
% \end{description}
% \emph{TDS} refers to the standard ``A Directory Structure
% for \TeX\ Files'' (\CTAN{tds/tds.pdf}). Directories
% with \xfile{texmf} in their name are usually organized this way.
%
% \subsection{Bundle installation}
%
% \paragraph{Unpacking.} Unpack the \xfile{oberdiek.tds.zip} in the
% TDS tree (also known as \xfile{texmf} tree) of your choice.
% Example (linux):
% \begin{quote}
%   |unzip oberdiek.tds.zip -d ~/texmf|
% \end{quote}
%
% \paragraph{Script installation.}
% Check the directory \xfile{TDS:scripts/oberdiek/} for
% scripts that need further installation steps.
% Package \xpackage{attachfile2} comes with the Perl script
% \xfile{pdfatfi.pl} that should be installed in such a way
% that it can be called as \texttt{pdfatfi}.
% Example (linux):
% \begin{quote}
%   |chmod +x scripts/oberdiek/pdfatfi.pl|\\
%   |cp scripts/oberdiek/pdfatfi.pl /usr/local/bin/|
% \end{quote}
%
% \subsection{Package installation}
%
% \paragraph{Unpacking.} The \xfile{.dtx} file is a self-extracting
% \docstrip\ archive. The files are extracted by running the
% \xfile{.dtx} through \plainTeX:
% \begin{quote}
%   \verb|tex bitset.dtx|
% \end{quote}
%
% \paragraph{TDS.} Now the different files must be moved into
% the different directories in your installation TDS tree
% (also known as \xfile{texmf} tree):
% \begin{quote}
% \def\t{^^A
% \begin{tabular}{@{}>{\ttfamily}l@{ $\rightarrow$ }>{\ttfamily}l@{}}
%   bitset.sty & tex/generic/oberdiek/bitset.sty\\
%   bitset.pdf & doc/latex/oberdiek/bitset.pdf\\
%   test/bitset-test1.tex & doc/latex/oberdiek/test/bitset-test1.tex\\
%   test/bitset-test2.tex & doc/latex/oberdiek/test/bitset-test2.tex\\
%   test/bitset-test3.tex & doc/latex/oberdiek/test/bitset-test3.tex\\
%   bitset.dtx & source/latex/oberdiek/bitset.dtx\\
% \end{tabular}^^A
% }^^A
% \sbox0{\t}^^A
% \ifdim\wd0>\linewidth
%   \begingroup
%     \advance\linewidth by\leftmargin
%     \advance\linewidth by\rightmargin
%   \edef\x{\endgroup
%     \def\noexpand\lw{\the\linewidth}^^A
%   }\x
%   \def\lwbox{^^A
%     \leavevmode
%     \hbox to \linewidth{^^A
%       \kern-\leftmargin\relax
%       \hss
%       \usebox0
%       \hss
%       \kern-\rightmargin\relax
%     }^^A
%   }^^A
%   \ifdim\wd0>\lw
%     \sbox0{\small\t}^^A
%     \ifdim\wd0>\linewidth
%       \ifdim\wd0>\lw
%         \sbox0{\footnotesize\t}^^A
%         \ifdim\wd0>\linewidth
%           \ifdim\wd0>\lw
%             \sbox0{\scriptsize\t}^^A
%             \ifdim\wd0>\linewidth
%               \ifdim\wd0>\lw
%                 \sbox0{\tiny\t}^^A
%                 \ifdim\wd0>\linewidth
%                   \lwbox
%                 \else
%                   \usebox0
%                 \fi
%               \else
%                 \lwbox
%               \fi
%             \else
%               \usebox0
%             \fi
%           \else
%             \lwbox
%           \fi
%         \else
%           \usebox0
%         \fi
%       \else
%         \lwbox
%       \fi
%     \else
%       \usebox0
%     \fi
%   \else
%     \lwbox
%   \fi
% \else
%   \usebox0
% \fi
% \end{quote}
% If you have a \xfile{docstrip.cfg} that configures and enables \docstrip's
% TDS installing feature, then some files can already be in the right
% place, see the documentation of \docstrip.
%
% \subsection{Refresh file name databases}
%
% If your \TeX~distribution
% (\teTeX, \mikTeX, \dots) relies on file name databases, you must refresh
% these. For example, \teTeX\ users run \verb|texhash| or
% \verb|mktexlsr|.
%
% \subsection{Some details for the interested}
%
% \paragraph{Attached source.}
%
% The PDF documentation on CTAN also includes the
% \xfile{.dtx} source file. It can be extracted by
% AcrobatReader 6 or higher. Another option is \textsf{pdftk},
% e.g. unpack the file into the current directory:
% \begin{quote}
%   \verb|pdftk bitset.pdf unpack_files output .|
% \end{quote}
%
% \paragraph{Unpacking with \LaTeX.}
% The \xfile{.dtx} chooses its action depending on the format:
% \begin{description}
% \item[\plainTeX:] Run \docstrip\ and extract the files.
% \item[\LaTeX:] Generate the documentation.
% \end{description}
% If you insist on using \LaTeX\ for \docstrip\ (really,
% \docstrip\ does not need \LaTeX), then inform the autodetect routine
% about your intention:
% \begin{quote}
%   \verb|latex \let\install=y% \iffalse meta-comment
%
% File: bitset.dtx
% Version: 2011/01/30 v1.1
% Info: Handle bit-vector datatype
%
% Copyright (C) 2007, 2011 by
%    Heiko Oberdiek <heiko.oberdiek at googlemail.com>
%
% This work may be distributed and/or modified under the
% conditions of the LaTeX Project Public License, either
% version 1.3c of this license or (at your option) any later
% version. This version of this license is in
%    http://www.latex-project.org/lppl/lppl-1-3c.txt
% and the latest version of this license is in
%    http://www.latex-project.org/lppl.txt
% and version 1.3 or later is part of all distributions of
% LaTeX version 2005/12/01 or later.
%
% This work has the LPPL maintenance status "maintained".
%
% This Current Maintainer of this work is Heiko Oberdiek.
%
% The Base Interpreter refers to any `TeX-Format',
% because some files are installed in TDS:tex/generic//.
%
% This work consists of the main source file bitset.dtx
% and the derived files
%    bitset.sty, bitset.pdf, bitset.ins, bitset.drv, bitset-test1.tex,
%    bitset-test2.tex, bitset-test3.tex.
%
% Distribution:
%    CTAN:macros/latex/contrib/oberdiek/bitset.dtx
%    CTAN:macros/latex/contrib/oberdiek/bitset.pdf
%
% Unpacking:
%    (a) If bitset.ins is present:
%           tex bitset.ins
%    (b) Without bitset.ins:
%           tex bitset.dtx
%    (c) If you insist on using LaTeX
%           latex \let\install=y% \iffalse meta-comment
%
% File: bitset.dtx
% Version: 2011/01/30 v1.1
% Info: Handle bit-vector datatype
%
% Copyright (C) 2007, 2011 by
%    Heiko Oberdiek <heiko.oberdiek at googlemail.com>
%
% This work may be distributed and/or modified under the
% conditions of the LaTeX Project Public License, either
% version 1.3c of this license or (at your option) any later
% version. This version of this license is in
%    http://www.latex-project.org/lppl/lppl-1-3c.txt
% and the latest version of this license is in
%    http://www.latex-project.org/lppl.txt
% and version 1.3 or later is part of all distributions of
% LaTeX version 2005/12/01 or later.
%
% This work has the LPPL maintenance status "maintained".
%
% This Current Maintainer of this work is Heiko Oberdiek.
%
% The Base Interpreter refers to any `TeX-Format',
% because some files are installed in TDS:tex/generic//.
%
% This work consists of the main source file bitset.dtx
% and the derived files
%    bitset.sty, bitset.pdf, bitset.ins, bitset.drv, bitset-test1.tex,
%    bitset-test2.tex, bitset-test3.tex.
%
% Distribution:
%    CTAN:macros/latex/contrib/oberdiek/bitset.dtx
%    CTAN:macros/latex/contrib/oberdiek/bitset.pdf
%
% Unpacking:
%    (a) If bitset.ins is present:
%           tex bitset.ins
%    (b) Without bitset.ins:
%           tex bitset.dtx
%    (c) If you insist on using LaTeX
%           latex \let\install=y\input{bitset.dtx}
%        (quote the arguments according to the demands of your shell)
%
% Documentation:
%    (a) If bitset.drv is present:
%           latex bitset.drv
%    (b) Without bitset.drv:
%           latex bitset.dtx; ...
%    The class ltxdoc loads the configuration file ltxdoc.cfg
%    if available. Here you can specify further options, e.g.
%    use A4 as paper format:
%       \PassOptionsToClass{a4paper}{article}
%
%    Programm calls to get the documentation (example):
%       pdflatex bitset.dtx
%       makeindex -s gind.ist bitset.idx
%       pdflatex bitset.dtx
%       makeindex -s gind.ist bitset.idx
%       pdflatex bitset.dtx
%
% Installation:
%    TDS:tex/generic/oberdiek/bitset.sty
%    TDS:doc/latex/oberdiek/bitset.pdf
%    TDS:doc/latex/oberdiek/test/bitset-test1.tex
%    TDS:doc/latex/oberdiek/test/bitset-test2.tex
%    TDS:doc/latex/oberdiek/test/bitset-test3.tex
%    TDS:source/latex/oberdiek/bitset.dtx
%
%<*ignore>
\begingroup
  \catcode123=1 %
  \catcode125=2 %
  \def\x{LaTeX2e}%
\expandafter\endgroup
\ifcase 0\ifx\install y1\fi\expandafter
         \ifx\csname processbatchFile\endcsname\relax\else1\fi
         \ifx\fmtname\x\else 1\fi\relax
\else\csname fi\endcsname
%</ignore>
%<*install>
\input docstrip.tex
\Msg{************************************************************************}
\Msg{* Installation}
\Msg{* Package: bitset 2011/01/30 v1.1 Handle bit-vector datatype (HO)}
\Msg{************************************************************************}

\keepsilent
\askforoverwritefalse

\let\MetaPrefix\relax
\preamble

This is a generated file.

Project: bitset
Version: 2011/01/30 v1.1

Copyright (C) 2007, 2011 by
   Heiko Oberdiek <heiko.oberdiek at googlemail.com>

This work may be distributed and/or modified under the
conditions of the LaTeX Project Public License, either
version 1.3c of this license or (at your option) any later
version. This version of this license is in
   http://www.latex-project.org/lppl/lppl-1-3c.txt
and the latest version of this license is in
   http://www.latex-project.org/lppl.txt
and version 1.3 or later is part of all distributions of
LaTeX version 2005/12/01 or later.

This work has the LPPL maintenance status "maintained".

This Current Maintainer of this work is Heiko Oberdiek.

The Base Interpreter refers to any `TeX-Format',
because some files are installed in TDS:tex/generic//.

This work consists of the main source file bitset.dtx
and the derived files
   bitset.sty, bitset.pdf, bitset.ins, bitset.drv, bitset-test1.tex,
   bitset-test2.tex, bitset-test3.tex.

\endpreamble
\let\MetaPrefix\DoubleperCent

\generate{%
  \file{bitset.ins}{\from{bitset.dtx}{install}}%
  \file{bitset.drv}{\from{bitset.dtx}{driver}}%
  \usedir{tex/generic/oberdiek}%
  \file{bitset.sty}{\from{bitset.dtx}{package}}%
  \usedir{doc/latex/oberdiek/test}%
  \file{bitset-test1.tex}{\from{bitset.dtx}{test1}}%
  \file{bitset-test2.tex}{\from{bitset.dtx}{test2,etex}}%
  \file{bitset-test3.tex}{\from{bitset.dtx}{test2,noetex}}%
  \nopreamble
  \nopostamble
  \usedir{source/latex/oberdiek/catalogue}%
  \file{bitset.xml}{\from{bitset.dtx}{catalogue}}%
}

\catcode32=13\relax% active space
\let =\space%
\Msg{************************************************************************}
\Msg{*}
\Msg{* To finish the installation you have to move the following}
\Msg{* file into a directory searched by TeX:}
\Msg{*}
\Msg{*     bitset.sty}
\Msg{*}
\Msg{* To produce the documentation run the file `bitset.drv'}
\Msg{* through LaTeX.}
\Msg{*}
\Msg{* Happy TeXing!}
\Msg{*}
\Msg{************************************************************************}

\endbatchfile
%</install>
%<*ignore>
\fi
%</ignore>
%<*driver>
\NeedsTeXFormat{LaTeX2e}
\ProvidesFile{bitset.drv}%
  [2011/01/30 v1.1 Handle bit-vector datatype (HO)]%
\documentclass{ltxdoc}
\usepackage{holtxdoc}[2011/11/22]
\usepackage{amstext}
\usepackage{ifluatex}
\ifluatex
\else
  \usepackage[T1]{fontenc}%
  \usepackage{lmodern}%
\fi
\begin{document}
  \DocInput{bitset.dtx}%
\end{document}
%</driver>
% \fi
%
% \CheckSum{3510}
%
% \CharacterTable
%  {Upper-case    \A\B\C\D\E\F\G\H\I\J\K\L\M\N\O\P\Q\R\S\T\U\V\W\X\Y\Z
%   Lower-case    \a\b\c\d\e\f\g\h\i\j\k\l\m\n\o\p\q\r\s\t\u\v\w\x\y\z
%   Digits        \0\1\2\3\4\5\6\7\8\9
%   Exclamation   \!     Double quote  \"     Hash (number) \#
%   Dollar        \$     Percent       \%     Ampersand     \&
%   Acute accent  \'     Left paren    \(     Right paren   \)
%   Asterisk      \*     Plus          \+     Comma         \,
%   Minus         \-     Point         \.     Solidus       \/
%   Colon         \:     Semicolon     \;     Less than     \<
%   Equals        \=     Greater than  \>     Question mark \?
%   Commercial at \@     Left bracket  \[     Backslash     \\
%   Right bracket \]     Circumflex    \^     Underscore    \_
%   Grave accent  \`     Left brace    \{     Vertical bar  \|
%   Right brace   \}     Tilde         \~}
%
% \GetFileInfo{bitset.drv}
%
% \title{The \xpackage{bitset} package}
% \date{2011/01/30 v1.1}
% \author{Heiko Oberdiek\\\xemail{heiko.oberdiek at googlemail.com}}
%
% \maketitle
%
% \begin{abstract}
% This package defines and implements the data type bit set,
% a vector of bits. The size of the vector may grow dynamically.
% Individual bits can be manipulated.
% \end{abstract}
%
% \tableofcontents
%
% \clearpage
% \section{Documentation}
%
% \subsection{Introduction}
%
% Annotations in the PDF format know entries whose values are
% integers. This numbers are interpreted as set of flags
% specifying properties. For example, annotation dictionaries
% can have a key |/F|. The bits of its integer value
% are interpreted the following way:
% \begin{quote}
% \begin{tabular}{@{}l|l@{}}
%   Bit position & Property name\\
%   \hline
%   1 & Invisible\\
%   2 & Hidden\\
%   3 & Print\\
%   4 & NoZoom\\
%   5 & NoRotate\\
%   6 & NoView\\
%   7 & ReadOnly\\
%   \dots & \dots
% \end{tabular}
% \end{quote}
% Now, let's see how these values are set in package \xpackage{hyperref}
% before it uses this package (before v6.77a):
%\begin{quote}
%\begin{verbatim}
%\ifFld@hidden /F 6\else /F 4\fi
%\end{verbatim}
%\end{quote}
% Where are the other flags? The following example for key |/Ff|
% in a widget annotation
% supports at least three properties:
%\begin{quote}
%\begin{verbatim}
%\ifFld@multiline
%  \ifFld@readonly /Ff 4097\else /Ff 4096\fi
%\else
%  \ifFld@password
%    \ifFld@readonly /Ff 8193\else /Ff 8192\fi
%  \else
%    \ifFld@readonly /Ff 1\fi
%  \fi
%\fi
%\end{verbatim}
%\end{quote}
% But you see the point. It would be a nightmare to continue
% this way in supporting the missing flag settings. This kind
% of integers may have up to 32 bits.
%
% Therefore I wanted a data structure for setting and
% clearing individual bits. Also it should provide
% an export as decimal number. The snipsets above are
% executed in expansion contexts without \TeX's stomach
% commands. It would be convenient to have an expandable
% conversion from the data structure to the integer that
% gets written to the PDF file.
%
% This package \xpackage{bitset} implements such a data
% structure. The interface is quite close to Java's
% class |BitSet| in order not to learn to many interfaces
% for the same kind of data structure.
%
% \subsection{Glossary}
%
% \begin{description}
% \item[Bit set:]
%   A bit set is a vector of bits or flags.
%   The vector size is unlimited and
%   grows dynamically. An undefined bit set is treated as bit
%   set where all bits are cleared.
%
%   Bit sets are addressed by name. A name should consists of
%   letters or digits. Technically it must survive \cs{csname},
%   see \LaTeX's environment names for other names with such
%   a constraint. Package \xpackage{babel}'s shorthands are
%   not supported due to technical reasons. Shorthand support
%   would break expandable operations.
% \item[Size:]
%   A size of a bit set is the number of bits in use.
%   It's the number of the highest index, incremented by one.
%   Sizes are in the range |0| up to |2147483647|, the highest
%   number supported by \TeX.
% \item[Index:]
%   Bit positions in a bit set are addressed by an index number.
%   The bit vector is zero based. The first and least significant
%   bit is addressed by index |0| and the highest possible bit
%   by |2147483646|.
% \item[Bit:]
%   A bit is enoded as |0| for cleared/disabled or |1| for set/enabled.
% \end{description}
%
% \subsection{Design principles}
%
% \begin{description}
% \item[Name conventions:]
%   To avoid conflicts with existing macro names, the operations
%   are prefixed by the package name.
% \item[Zero based indexes:] The first bit is addressed by zero.
%   (Convention of array indexing in C, Java, \dots)
% \item[Unlimited size:]
%   There is no restriction on the size of a bit set other than
%   usual memory limitations. \cs{bitsetSetDec} and \cs{bitsetGetDec}
%   transparently switch to package \xpackage{bigintcalc} if the numbers
%   get too large for \TeX's number limit.
% \item[Expandibility:]
%   Any operation that does not change the bit set is expandable.
%   And all operations that extract or calculate some result
%   do this in exact two expansion steps. For example,
%   a macro \cs{Macro} wants a bit set as decimal number.
%   But the argument must be a plain number without macros.
%   Thus you could prefix \cs{bitsetGetDec} with \cs{number}.
%   However this won't work for bit sets with 31 or more bits
%   because of \TeX's number limit of $2^{31}-1$.
%   then just hit the operator with two \cs{expandafter}:
%   \begin{quote}
%   |\expandafter\expandafter\expandafter|\\
%   |\Macro\bitsetGetDec{foo}|
%   \end{quote}
%   |\bitsetGetDec| is hit first by the third \cs{expandafter}
%   and then by the second one.
% \item[Format independence:] This package is written as \LaTeX\ package,
%   but it does not depend on \LaTeX. It will also work for other formats
%   such as \plainTeX.
% \item[Independence from \TeX\ engines:]
%   Vanilla \TeX\ is all you need. Calculations are delegated to
%   packages \xpackage{intcalc} and \xpackage{bigintcalc}.
%   They don't need any special features, but they will switch
%   to a little more efficient implementation if features such
%   as \cs{numexpr} are available.
% \item[Numeric arguments:] Anything that is accepted by
%   \cs{number}. If \eTeX\ is detected, also expressions
%   for \cs{numexpr} are supported. The only exception so far
%   is the number for \cs{bitsetSetDec}. The number might be too
%   large for \cs{number} or \cs{numexpr}.
% \item[Error messages:]
%   In expandable contexts, only a limited set of \TeX\ primitive
%   commands work as expected. So called stomach commands behave
%   like \cs{relax} and don't get expanded or executed.
%   Unhappily also the error commands belong to this category.
%   The expandable operations will throw an unknown control
%   sequence instead to get \TeX's and user's attention.
%   The name of these control sequences starts with
%   \cs{BitSetError:} with the type of error after the colon.
% \end{description}
%
% \subsection{Operator overview}
%
% \begin{minipage}{\linewidth}
% \begin{description}
% \def\It#1#2{^^A
%   \item[^^A
%     {#1} ^^A
%     \textmd{(\hyperref[{sec:#2}]{section \ref*{sec:#2}})}^^A
%   ]^^A
%   \mbox{}\\^^A
% }
% \def\op#1{^^A
%   \cs{bitset}\texttt{\bfseries#1}^^A
% }
% \def\M#1{\meta{#1}}
% \def\Q{\hspace*{\fill}}
% \It{Miscellaneous}{misc}
%   \op{Reset}\Q \M{BitSet}\\
%   \op{Let}\Q \M{BitSet A} \M{BitSet B}
% \It{Import}{import}
%   \op{SetBin}, \op{SetOct}, \op{SetHex}\Q \M{BitSet} \M{Value}\\
%   \op{SetDec}\Q \M{BitSet} \M{Value}
% \It{Export\textsuperscript{a}}{export}
%   \op{GetBin}, \op{GetOct}, \op{GetHex}\Q \M{BitSet} \M{MinSize}\\
%   \op{GetDec}\Q \M{BitSet}
% \It{Logical operators}{log}
%   \op{And}, \op{AndNot}\Q \M{BitSet A} \M{BitSet B}\\
%   \op{Or}, \op{Xor}\Q \M{BitSet A} \M{BitSet B}
% \It{Shifting}{shift}
%   \op{ShiftLeft}, \op{ShiftRight}\Q \M{BitSet} \M{ShiftAmount}\\
% \It{Bit manipulation}{bitman}
%   \op{Clear}, \op{Set}, \op{Flip}\Q \M{BitSet} \M{Index}\\
%   \op{SetValue}\Q \M{BitSet} \M{Index} \M{Value}\\
%   \op{ClearRange}, \op{SetRange}, \op{FlipRange}\\
%     \Q \M{BitSet} \M{IndexFrom} \M{IndexTo}\\
%   \op{SetValueRange}\Q \M{BitSet} \M{IndexFrom} \M{IndexTo}
% \It{Bit retrieval\textsuperscript{a}}{bitret}
%   \op{Get}\Q \M{BitSet} \M{Index}\\
%   \op{NextClearBit}, \op{NextSetBit}\Q \M{BitSet} \M{Index}\\
%   \op{GetSetBitList}\Q \M{BitSet}
% \It{Bit set properties}{prop}
%   \op{Size}, \op{Cardinality}\Q \M{BitSet}
% \It{Queries\textsuperscript{b}}{query}
%   \op{IsDefined}, \op{IsEmpty}\Q \M{BitSet} \M{Then} \M{Else}\\
%   \op{Equals}, \op{Intersects}\Q \M{BitSet A} \M{BitSet B} \M{Then} \M{Else}\\
%   \op{Query}\Q \M{BitSet} \M{Index} \M{Then} \M{Else}
% \end{description}
% \footnotetext[1]{Macros are expandable, full expansion by two steps.}
% \footnotetext[2]{Macros are expandable.}
% \end{minipage}
%
% \subsection{Package loading}
%
% The package can be used as normal \LaTeX\ package:
%\begin{quote}
%|\usepackage{bitset}|
%\end{quote}
% Also \plainTeX\ is supported:
%\begin{quote}
%|\input bitset.sty\relax|
%\end{quote}
%
% \subsection{Operators}
%
% The following macros work on and with bit sets.
% A bit set \meta{BitSet} is represented by a name.
% The should consist of letters and digits. Technically
% it must survive \cs{csname}. It is the same constraint
% that must be satisfied by label or environment names
% in \LaTeX.
%
% However active characters that are shorthands
% of package \xpackage{babel} are not supported.
% Support for shorthands works by an assignment.
% But many operators
% such as \cs{bitsetGetDec} must be usable in expandable
% contexts. There assignments will not be executed in the
% best case or they will cause errors.
%
% The bits in a bit set are addressed by non-negative
% integers starting from zero. Thus negative index
% numbers cause an error message. Because index numbers
% are \TeX\ numbers. The largest index is 2147483647.
% But in practice memory limits and patience limits
% will be very likely reached much before.
%
% \subsubsection{Miscellaneous}\label{sec:misc}
%
% There isn't a separate operation for bit set creation.
% For simplicity an undefined bit set is treated as bit set
% with all bits cleared.
%
% \begin{declcs}{bitsetReset} \M{BitSet}
% \end{declcs}
% Macro \cs{bitsetReset} clears all bits. The result is an
% empty bit set. It may also be used as replacement for
% an operation ``new'', because an undefined bit set is defined
% afterwards.
%
% \begin{declcs}{bitsetLet} \M{BitSet A} \M{BitSet B}
% \end{declcs}
% Macro \cs{bitsetLet} performs a simple assignment similar
% to \TeX's \cs{let}. After the operation \meta{BitSet A} has
% the same value as \meta{BitSet B}. If \meta{BitSet B} is
% undefined, then \meta{BitSet A} will be the empty bit set.
%
% Note: If \meta{BitSet A} exists, it will be overwritten.
%
% \subsubsection{Import}\label{sec:import}
%
% \begin{declcs}{bitsetSetBin} \M{BitSet} \M{BinaryNumber}\\
%   \cs{bitsetSetOct} \M{BitSet} \M{OctalNumber}\\
%   \cs{bitsetSetHex} \M{BitSet} \M{HexadecimalNumber}
% \end{declcs}
% The numbers are interpreted as bit vectors and the flags in
% the bit \meta{BitSet} set are set accordingly.
% These numeric arguments are the only arguments where
% spaces are allowed. Then the numbers are easier to read.
%
% \begin{declcs}{bitsetSetDec} \M{BitSet} \M{DecimalNumber}
% \end{declcs}
% Macro \cs{bitsetSetDec} uses \meta{DecimalNumber} to
% set the bit set \meta{BitSet}.
% The numeric argument must expand
% to a plain number consisting of decimal digits without
% command tokens or spaces. Internally this argument is
% expanded only. It cannot be passed to \cs{number} or
% \cs{numexpr}, because the number may be too large for them.
% However \cs{number} or |\the\numexpr| may be used
% explicitely. This also helps for unexpandable number
% command tokens or registers (\cs{z@}, \cs{@ne}, \cs{count@}, \dots).
% Also \LaTeX'\ \cs{value} needs prefixing:
% \begin{quote}
% |\bitsetSetDec{foo}{\number\value{bar}}|
% \end{quote}
%
% \subsubsection{Export}\label{sec:export}
%
% \begin{declcs}{bitsetGetBin} \M{BitSet} \M{MinSize}\\
%   \cs{bitsetGetOct} \M{BitSet} \M{MinSize}\\
%   \cs{bitsetGetHex} \M{BitSet} \M{MinSize}
% \end{declcs}
% These macros returns the bit set as binary, octal or hexadecimal
% number. If the bit size is smaller than \meta{MinSize} the gap is
% filled with leading zeros.
% Example:
% \begin{quote}
% |\bitsetReset{abc}|\\
% |\bitsetSet{abc}{2}|\\
% |\bitsetGetBin{abc}{8}| $\rightarrow$ |00000100|\\
% |\bitsetSet{abc}{5}\bitsetSet{abc}{7}|\\
% |\bitsetGetHex{abc}{16}| $\rightarrow$ |00A2|
% \end{quote}
% Macro \cs{bitsetGetHex} uses the uppercase letters |A| to |F|.
% The catcode of the letters is one of 11 (letter) or 12 (other).
%
% \begin{declcs}{bitsetGetDec} \M{BitSet}
% \end{declcs}
% Macro \cs{bitsetGetDec} returns the bit set \meta{BitSet} as
% decimal number. The returned number can be larger than
% \TeX's number limit of $2^{31}-1$.
%
% \subsubsection{Logical operators}\label{sec:log}
%
% \begin{declcs}{bitsetAnd} \M{BitSet A} \M{BitSet B}
% \end{declcs}
% $A_{\text{new}} \mathrel{\mathop{:}}= A_{\text{old}}
% \mathbin{\text{and}} B \qquad (\forall\, \text{bits})$
%
% \begin{declcs}{bitsetAndNot} \M{BitSet A} \M{BitSet B}
% \end{declcs}
% $A_{\text{new}} \mathrel{\mathop{:}}= A_{\text{old}}
% \mathbin{\text{and}} (\text{not}\,B) \qquad (\forall\, \text{bits})$
%
% \begin{declcs}{bitsetOr} \M{BitSet A} \M{BitSet B}
% \end{declcs}
% $A_{\text{new}} \mathrel{\mathop{:}}= A_{\text{old}}
% \mathbin{\text{or}} B \qquad (\forall\, \text{bits})$
%
% \begin{declcs}{bitsetXor} \M{BitSet A} \M{BitSet B}
% \end{declcs}
% $A_{\text{new}} \mathrel{\mathop{:}}= A_{\text{old}}
% \mathbin{\text{xor}} B \qquad (\forall\, \text{bits})$
%
% \subsubsection{Shifting}\label{sec:shift}
%
% \begin{declcs}{bitsetShiftLeft} \M{BitSet} \M{ShiftAmount}\\
%   \cs{bitsetShiftRight} \M{BitSet} \M{ShiftAmount}
% \end{declcs}
% A left shift by one is a multiplication by two, thus left shifting
% moves the flags to higher positions. The new created low positions
% are filled by zeros.
%
% A right shift is the opposite, dividing by two, movint the bits
% to lower positions. The number will become smaller, the lowest
% bits are lost.
%
% If the \meta{ShiftAmount} is negative, it reverts the meaning of
% the shift operation. A left shift becomes a right shift.
% A \meta{ShiftAmount} of zero is ignored.
%
% \subsubsection{Bit manipulation}\label{sec:bitman}
%
% \begin{declcs}{bitsetClear} \M{BitSet} \M{Index}\\
%   \cs{bitsetSet} \M{BitSet} \M{Index}\\
%   \cs{bitsetFlip} \M{BitSet} \M{Index}
% \end{declcs}
% This macros manipulate a single bit in \meta{BitSet} addressed
% by \cs{Index}. Macro \cs{bitsetClear} disables the bit,
% \cs{bitsetSet} enables it and \cs{bitsetFlip} reverts the current
% setting of the bit.
%
% \begin{declcs}{bitsetSetValue} \M{BitSet} \M{Index} \M{Bit}
% \end{declcs}
% Macro \cs{bitsetSetValue} puts bit \meta{Bit} at position \meta{Index}
% in bit set \meta{BitSet}. \meta{Bit} must be a valid \TeX\ number
% equals to zero (disabled/cleared) or one (enabled/set).
%
% \subsubsection{Bit retrieval}\label{sec:bitret}
%
% \begin{declcs}{bitsetGet} \M{BitSet} \M{Index}
% \end{declcs}
% Macro \cs{bitsetGet} extracts the status of the bit at
% position \meta{Index} in bit set \meta{BitSet}.
% Digit |1| is returned if the bit is set/enabled. If the
% bit is cleared/disabled and in cases of an undefined bitset or
% an index number out of range the return value is |0|.
%
% \begin{declcs}{bitsetNextClearBit} \M{BitSet} \M{Index}
% \end{declcs}
% Starting at position \meta{Index} (inclusive) the bits
% are inspected. The first position without a set bit is returned.
% Possible results are decimal numbers: \meta{Index}, \meta{Index} + 1,
% \dots, ($\infty$)
%
% \begin{declcs}{bitsetNextSetBit} \M{BitSet} \M{Index}
% \end{declcs}
% Starting at position \meta{Index} (inclusive) the bits are
% inspected and the index position of the first found set bit
% is returned. If there isn't such a bit, then the result is
% |-1|. In summary possible results are decimal numbers:
% |-1|, \meta{Index}, \meta{Index} + 1, \dots, ($\infty$)
%
% \begin{declcs}{bitsetGetSetBitList} \M{BitSet}
% \end{declcs}
% Macro \cs{bitsetGetSetBitList} is an application for
% \cs{bitsetNextSetBit}. The set bits are iterated and
% returned as comma separated list of index positions
% in increasing order. The list is empty in case of
% an empty bit set.
%
% \subsubsection{Bit set properties}\label{sec:prop}
%
% \begin{declcs}{bitsetSize} \M{BitSet}
% \end{declcs}
% Macro \cs{bitsetSize} returns number of bits in use.
% It is the same as the index number of the highest set/enabled bit
% incremented by one.
%
% \begin{declcs}{bitsetCardinality} \M{BitSet}
% \end{declcs}
% Macro \cs{bitsetCardinality} counts the number of set/enabled bits.
%
% \subsubsection{Queries}\label{sec:query}
%
% Also the query procedures are expandable. They ask for a piece
% of information about a bit set and execute code depending
% on the answer.
%
% \begin{declcs}{bitsetIsDefined} \M{BitSet} \M{Then} \M{Else}
% \end{declcs}
% If the bit set with the name \meta{BitSet} exists the code
% given in \meta{Then} is executed, otherwise \meta{Else} is used.
%
% \begin{declcs}{bitsetIsEmpty} \M{BitSet} \M{Then} \M{Else}
% \end{declcs}
% If the bit set \meta{BitSet} exists and at least one bit
% is set/enabled, the code in \meta{Then} is executed,
% \meta{Else} otherwise.
%
% \begin{declcs}{bitsetEquals} \M{BitSet A} \M{BitSet B} \M{Then} \M{Else}
% \end{declcs}
% Both bit sets are equal if and only if either both are undefined
% or both are defined and represents the same bit values at the
% same positions. Thus this definition is reflexive, symmeric,
% and transitive, enough for an equivalent relation.
%
% \begin{declcs}{bitsetIntersects} \M{BitSet A} \M{BitSet B} \M{Then} \M{Else}
% \end{declcs}
% If and only if \meta{BitSet A} and \meta{BitSet B} have
% at least one bit at the same position that is set, then
% code part \meta{Then} is executed.
%
% \begin{declcs}{bitsetQuery} \M{BitSet} \M{Index} \M{Then} \M{Else}
% \end{declcs}
% It's just a wrapper for \cs{bitsetGet}. If the bit at position
% \meta{Index} is enabled, code \meta{Then} is called.
%
% \StopEventually{
% }
%
% \section{Implementation}
%
% The internal format of a bit set is quite simple, a sequence
% of digits |0| and |1|. The least significant bit is left.
% A bit set without any flag set is encoded by |0|. Also
% undefined bit sets are treated that way. After the
% highest bit that is set there are no further zeroes.
% A regular expression of valid bit sets values:
% \begin{quote}
%  |0|\string||[01]*1|
% \end{quote}
%
%    \begin{macrocode}
%<*package>
%    \end{macrocode}
%
% \subsection{Reload check and package identification}
%    Reload check, especially if the package is not used with \LaTeX.
%    \begin{macrocode}
\begingroup\catcode61\catcode48\catcode32=10\relax%
  \catcode13=5 % ^^M
  \endlinechar=13 %
  \catcode35=6 % #
  \catcode39=12 % '
  \catcode44=12 % ,
  \catcode45=12 % -
  \catcode46=12 % .
  \catcode58=12 % :
  \catcode64=11 % @
  \catcode123=1 % {
  \catcode125=2 % }
  \expandafter\let\expandafter\x\csname ver@bitset.sty\endcsname
  \ifx\x\relax % plain-TeX, first loading
  \else
    \def\empty{}%
    \ifx\x\empty % LaTeX, first loading,
      % variable is initialized, but \ProvidesPackage not yet seen
    \else
      \expandafter\ifx\csname PackageInfo\endcsname\relax
        \def\x#1#2{%
          \immediate\write-1{Package #1 Info: #2.}%
        }%
      \else
        \def\x#1#2{\PackageInfo{#1}{#2, stopped}}%
      \fi
      \x{bitset}{The package is already loaded}%
      \aftergroup\endinput
    \fi
  \fi
\endgroup%
%    \end{macrocode}
%    Package identification:
%    \begin{macrocode}
\begingroup\catcode61\catcode48\catcode32=10\relax%
  \catcode13=5 % ^^M
  \endlinechar=13 %
  \catcode35=6 % #
  \catcode39=12 % '
  \catcode40=12 % (
  \catcode41=12 % )
  \catcode44=12 % ,
  \catcode45=12 % -
  \catcode46=12 % .
  \catcode47=12 % /
  \catcode58=12 % :
  \catcode64=11 % @
  \catcode91=12 % [
  \catcode93=12 % ]
  \catcode123=1 % {
  \catcode125=2 % }
  \expandafter\ifx\csname ProvidesPackage\endcsname\relax
    \def\x#1#2#3[#4]{\endgroup
      \immediate\write-1{Package: #3 #4}%
      \xdef#1{#4}%
    }%
  \else
    \def\x#1#2[#3]{\endgroup
      #2[{#3}]%
      \ifx#1\@undefined
        \xdef#1{#3}%
      \fi
      \ifx#1\relax
        \xdef#1{#3}%
      \fi
    }%
  \fi
\expandafter\x\csname ver@bitset.sty\endcsname
\ProvidesPackage{bitset}%
  [2011/01/30 v1.1 Handle bit-vector datatype (HO)]%
%    \end{macrocode}
%
% \subsection{Catcodes}
%
%    \begin{macrocode}
\begingroup\catcode61\catcode48\catcode32=10\relax%
  \catcode13=5 % ^^M
  \endlinechar=13 %
  \catcode123=1 % {
  \catcode125=2 % }
  \catcode64=11 % @
  \def\x{\endgroup
    \expandafter\edef\csname BitSet@AtEnd\endcsname{%
      \endlinechar=\the\endlinechar\relax
      \catcode13=\the\catcode13\relax
      \catcode32=\the\catcode32\relax
      \catcode35=\the\catcode35\relax
      \catcode61=\the\catcode61\relax
      \catcode64=\the\catcode64\relax
      \catcode123=\the\catcode123\relax
      \catcode125=\the\catcode125\relax
    }%
  }%
\x\catcode61\catcode48\catcode32=10\relax%
\catcode13=5 % ^^M
\endlinechar=13 %
\catcode35=6 % #
\catcode64=11 % @
\catcode123=1 % {
\catcode125=2 % }
\def\TMP@EnsureCode#1#2{%
  \edef\BitSet@AtEnd{%
    \BitSet@AtEnd
    \catcode#1=\the\catcode#1\relax
  }%
  \catcode#1=#2\relax
}
\TMP@EnsureCode{33}{12}% !
\TMP@EnsureCode{39}{12}% '
\TMP@EnsureCode{40}{12}% (
\TMP@EnsureCode{41}{12}% )
\TMP@EnsureCode{42}{12}% *
\TMP@EnsureCode{43}{12}% +
\TMP@EnsureCode{44}{12}% ,
\TMP@EnsureCode{45}{12}% -
\TMP@EnsureCode{46}{12}% .
\TMP@EnsureCode{47}{12}% /
\TMP@EnsureCode{58}{11}% : (letter!)
\TMP@EnsureCode{60}{12}% <
\TMP@EnsureCode{62}{12}% >
\TMP@EnsureCode{63}{14}% ? (comment!)
\TMP@EnsureCode{91}{12}% [
\TMP@EnsureCode{93}{12}% ]
\TMP@EnsureCode{96}{12}% `
\edef\BitSet@AtEnd{\BitSet@AtEnd\noexpand\endinput}
\begingroup\expandafter\expandafter\expandafter\endgroup
\expandafter\ifx\csname BitSet@TestMode\endcsname\relax
\else
  \catcode63=9 % ? (ignore)
\fi
? \let\BitSet@@TestMode\BitSet@TestMode
%    \end{macrocode}
%
% \subsection{Package loading}
%
%    \begin{macrocode}
\begingroup\expandafter\expandafter\expandafter\endgroup
\expandafter\ifx\csname RequirePackage\endcsname\relax
  \def\TMP@RequirePackage#1[#2]{%
    \begingroup\expandafter\expandafter\expandafter\endgroup
    \expandafter\ifx\csname ver@#1.sty\endcsname\relax
      \input #1.sty\relax
    \fi
  }%
  \TMP@RequirePackage{infwarerr}[2007/09/09]%
  \TMP@RequirePackage{intcalc}[2007/09/27]%
  \TMP@RequirePackage{bigintcalc}[2007/09/27]%
\else
  \RequirePackage{infwarerr}[2007/09/09]%
  \RequirePackage{intcalc}[2007/09/27]%
  \RequirePackage{bigintcalc}[2007/09/27]%
\fi
%    \end{macrocode}
%
% \subsection{Help macros}
%
% \subsubsection{Number constant}
%
%    \begin{macro}{\BitSet@MaxSize}
%    \begin{macrocode}
\def\BitSet@MaxSize{2147483647}%
%    \end{macrocode}
%    \end{macro}
%
% \subsubsection{General basic macros}
%
%    \begin{macro}{\BitSet@Empty}
%    \begin{macrocode}
\def\BitSet@Empty{}
%    \end{macrocode}
%    \end{macro}
%    \begin{macro}{\BitSet@FirstOfOne}
%    \begin{macrocode}
\def\BitSet@FirstOfOne#1{#1}
%    \end{macrocode}
%    \end{macro}
%    \begin{macro}{\BitSet@Gobble}
%    \begin{macrocode}
\def\BitSet@Gobble#1{}
%    \end{macrocode}
%    \end{macro}
%    \begin{macro}{\BitSet@FirstOfTwo}
%    \begin{macrocode}
\def\BitSet@FirstOfTwo#1#2{#1}
%    \end{macrocode}
%    \end{macro}
%    \begin{macro}{\BitSet@SecondOfTwo}
%    \begin{macrocode}
\def\BitSet@SecondOfTwo#1#2{#2}
%    \end{macrocode}
%    \end{macro}
%
%    \begin{macro}{\BitSet@Space}
%    \begin{macrocode}
\def\BitSet@Space{ }
%    \end{macrocode}
%    \end{macro}
%    \begin{macro}{\BitSet@ZapSpace}
%    \begin{macrocode}
\def\BitSet@ZapSpace#1 #2{%
  #1%
  \ifx\BitSet@Empty#2%
  \else
    \expandafter\BitSet@ZapSpace
  \fi
  #2%
}
%    \end{macrocode}
%    \end{macro}
%
% \subsubsection{Tail recursion}
%
%    \begin{macro}{\BitSet@Fi}
%    \begin{macrocode}
\let\BitSet@Fi\fi
%    \end{macrocode}
%    \end{macro}
%    \begin{macro}{\BitSet@AfterFi}
%    \begin{macrocode}
\def\BitSet@AfterFi#1#2\BitSet@Fi{\fi#1}
%    \end{macrocode}
%    \end{macro}
%    \begin{macro}{\BitSet@AfterFiFi}
%    \begin{macrocode}
\def\BitSet@AfterFiFi#1#2\BitSet@Fi{\fi\fi#1}%
%    \end{macrocode}
%    \end{macro}
%    \begin{macro}{\BitSet@AfterFiFiFi}
%    \begin{macrocode}
\def\BitSet@AfterFiFiFi#1#2\BitSet@Fi{\fi\fi\fi#1}%
%    \end{macrocode}
%    \end{macro}
%
% \subsubsection{Check macros}
%
%    \begin{macro}{\BitSet@IfUndefined}
%    \begin{macrocode}
\def\BitSet@IfUndefined#1{%
  \expandafter\ifx\csname BS@#1\endcsname\relax
    \expandafter\BitSet@FirstOfTwo
  \else
    \expandafter\BitSet@SecondOfTwo
  \fi
}
%    \end{macrocode}
%    \end{macro}
%
%    \begin{macro}{\BitSet@CheckIndex}
%    |#1|: continuation code\\
%    |#2|: BitSet\\
%    |#3|: Index
%    \begin{macrocode}
\def\BitSet@CheckIndex#1#2#3{%
  \BitSet@IfUndefined{#2}{\bitsetReset{#2}}{}%
  \expandafter\expandafter\expandafter\BitSet@@CheckIndex
  \intcalcNum{#3}!%
  {#2}{#1}%
}
%    \end{macrocode}
%    \end{macro}
%    \begin{macro}{\BitSet@@CheckIndex}
%    |#1|: plain Index\\
%    |#2|: BitSet\\
%    |#3|: continuation code
%    \begin{macrocode}
\def\BitSet@@CheckIndex#1!#2#3{%
  \ifnum#1<0 %
    \BitSet@AfterFi{%
      \@PackageError{bitset}{%
        Invalid negative index (#1)%
      }\@ehc
    }%
  \else
    \BitSet@AfterFi{%
      #3{#2}{#1}%
    }%
  \BitSet@Fi
}
%    \end{macrocode}
%    \end{macro}
%
% \subsection{Miscellaneous}\label{src:misc}
%
%    \begin{macro}{\bitsetReset}
%    \begin{macrocode}
\def\bitsetReset#1{%
  \expandafter\def\csname BS@#1\endcsname{0}%
}
%    \end{macrocode}
%    \end{macro}
%
%    \begin{macro}{\bitsetLet}
%    \begin{macrocode}
\def\bitsetLet#1#2{%
  \BitSet@IfUndefined{#2}{%
    \bitsetReset{#1}%
  }{%
    \expandafter\let\csname BS@#1\expandafter\endcsname
                    \csname BS@#2\endcsname
  }%
}
%    \end{macrocode}
%    \end{macro}
%
% \subsection{Import}\label{src:import}
%
% \subsubsection{From binary number}
%
%    \begin{macro}{\bitsetSetBin}
%    \begin{macrocode}
\def\bitsetSetBin#1#2{%
  \edef\BitSet@Temp{#2}%
  \edef\BitSet@Temp{%
    \expandafter\expandafter\expandafter\BitSet@ZapSpace
    \expandafter\BitSet@Temp\BitSet@Space\BitSet@Empty
  }%
  \edef\BitSet@Temp{%
    \expandafter\BitSet@KillZeros\BitSet@Temp\BitSet@Empty
  }%
  \ifx\BitSet@Temp\BitSet@Empty
    \expandafter\let\csname BS@#1\endcsname\BitSet@Zero
  \else
    \expandafter\edef\csname BS@#1\endcsname{%
      \expandafter\BitSet@Reverse\BitSet@Temp!%
    }%
  \fi
}
%    \end{macrocode}
%    \end{macro}
%    \begin{macro}{\BitSet@KillZeros}
%    \begin{macrocode}
\def\BitSet@KillZeros#1{%
  \ifx#10%
    \expandafter\BitSet@KillZeros
  \else
    #1%
  \fi
}
%    \end{macrocode}
%    \end{macro}
%    \begin{macro}{\BitSet@Reverse}
%    \begin{macrocode}
\def\BitSet@Reverse#1#2!{%
  \ifx\\#2\\%
    #1%
  \else
    \BitSet@AfterFi{%
      \BitSet@Reverse#2!#1%
    }%
  \BitSet@Fi
}
%    \end{macrocode}
%    \end{macro}
%
% \subsubsection{From octal/hex number}
%
%    \begin{macro}{\bitsetSetOct}
%    \begin{macrocode}
\def\bitsetSetOct{%
  \BitSet@SetOctHex\BitSet@FromFirstOct
}
%    \end{macrocode}
%    \end{macro}
%    \begin{macro}{\bitsetSetHex}
%    \begin{macrocode}
\def\bitsetSetHex{%
  \BitSet@SetOctHex\BitSet@FromFirstHex
}
%    \end{macrocode}
%    \end{macro}
%    \begin{macro}{\BitSet@SetOctHex}
%    \begin{macrocode}
\def\BitSet@SetOctHex#1#2#3{%
  \edef\BitSet@Temp{#3}%
  \edef\BitSet@Temp{%
    \expandafter\expandafter\expandafter\BitSet@ZapSpace
    \expandafter\BitSet@Temp\BitSet@Space\BitSet@Empty
  }%
  \edef\BitSet@Temp{%
    \expandafter\BitSet@KillZeros\BitSet@Temp\BitSet@Empty
  }%
  \ifx\BitSet@Temp\BitSet@Empty
    \expandafter\let\csname BS@#2\endcsname\BitSet@Zero
  \else
    \edef\BitSet@Temp{%
      \expandafter#1\BitSet@Temp!%
    }%
    \ifx\BitSet@Temp\BitSet@Empty
      \expandafter\let\csname BS@#2\endcsname\BitSet@Zero
    \else
      \expandafter\edef\csname BS@#2\endcsname{%
        \expandafter\BitSet@Reverse\BitSet@Temp!%
      }%
    \fi
  \fi
}
%    \end{macrocode}
%    \end{macro}
%    \begin{macro}{\BitSet@FromFirstOct}
%    \begin{macrocode}
\def\BitSet@FromFirstOct#1{%
  \ifx#1!%
  \else
    \ifcase#1 \BitSet@AfterFiFi\BitSet@FromFirstOct
    \or 1%
    \or 10%
    \or 11%
    \or 100%
    \or 101%
    \or 110%
    \or 111%
    \else \BitSetError:WrongOctalDigit%
    \fi
    \expandafter\BitSet@FromOct
  \BitSet@Fi
}
%    \end{macrocode}
%    \end{macro}
%    \begin{macro}{\BitSet@FromOct}
%    \begin{macrocode}
\def\BitSet@FromOct#1{%
  \ifx#1!%
  \else
    \ifcase#1 000%
    \or 001%
    \or 010%
    \or 011%
    \or 100%
    \or 101%
    \or 110%
    \or 111%
    \else \BitSetError:WrongOctalDigit%
    \fi
    \expandafter\BitSet@FromOct
  \fi
}
%    \end{macrocode}
%    \end{macro}
%
%    \begin{macro}{\BitSet@FromFirstHex}
%    \begin{macrocode}
\def\BitSet@FromFirstHex#1{%
  \ifx#1!%
  \else
    \ifx#10%
      \BitSet@AfterFiFi\BitSet@FromFirstHex
    \fi
    \expandafter\ifx\csname BitSet@Hex#1\endcsname\relax
      \BitSetError:InvalidHexDigit%
    \else
      \expandafter\expandafter\expandafter\BitSet@KillZeros
      \csname BitSet@Hex#1\endcsname
    \fi
    \expandafter\BitSet@FromHex
  \BitSet@Fi
}
%    \end{macrocode}
%    \end{macro}
%    \begin{macro}{\BitSet@FromHex}
%    \begin{macrocode}
\def\BitSet@FromHex#1{%
  \ifx#1!%
  \else
    \expandafter\ifx\csname BitSet@Hex#1\endcsname\relax
      \BitSetError:InvalidHexDigit%
    \else
      \csname BitSet@Hex#1\endcsname
    \fi
    \expandafter\BitSet@FromHex
  \fi
}
%    \end{macrocode}
%    \end{macro}
%    \begin{macro}{\BitSet@Hex[0..F]}
%    \begin{macrocode}
\def\BitSet@Temp#1{%
  \expandafter\def\csname BitSet@Hex#1\endcsname
}
\BitSet@Temp 0{0000}%
\BitSet@Temp 1{0001}%
\BitSet@Temp 2{0010}%
\BitSet@Temp 3{0011}%
\BitSet@Temp 4{0100}%
\BitSet@Temp 5{0101}%
\BitSet@Temp 6{0110}%
\BitSet@Temp 7{0111}%
\BitSet@Temp 8{1000}%
\BitSet@Temp 9{1001}%
\BitSet@Temp A{1010}%
\BitSet@Temp B{1011}%
\BitSet@Temp C{1100}%
\BitSet@Temp D{1101}%
\BitSet@Temp E{1110}%
\BitSet@Temp F{1111}%
\BitSet@Temp a{1010}%
\BitSet@Temp b{1011}%
\BitSet@Temp c{1100}%
\BitSet@Temp d{1101}%
\BitSet@Temp e{1110}%
\BitSet@Temp f{1111}%
%    \end{macrocode}
%    \end{macro}
%
% \subsubsection{From decimal number}
%
%    \begin{macro}{\bitsetSetDec}
%    \begin{macrocode}
\def\bitsetSetDec#1#2{%
  \edef\BitSet@Temp{#2}%
  \edef\BitSet@Temp{%
    \expandafter\expandafter\expandafter\BitSet@ZapSpace
    \expandafter\BitSet@Temp\BitSet@Space\BitSet@Empty
  }%
  \edef\BitSet@Temp{%
    \expandafter\BitSet@KillZeros\BitSet@Temp\BitSet@Empty
  }%
  \ifx\BitSet@Temp\BitSet@Empty
    \expandafter\let\csname BS@#1\endcsname\BitSet@Zero
  \else
    \ifcase\bigintcalcSgn{\BitSet@Temp} %
      \expandafter\let\csname BS@#1\endcsname\BitSet@Zero
    \or
      \ifnum\bigintcalcCmp\BitSet@Temp\BitSet@MaxSize>0 %
        \expandafter\edef\csname BS@#1\endcsname{%
          \expandafter\BitSet@SetDecBig\BitSet@Temp!%
        }%
      \else
        \expandafter\edef\csname BS@#1\endcsname{%
          \expandafter\BitSet@SetDec\BitSet@Temp!%
        }%
      \fi
    \else
      \@PackageError{bitset}{%
        Bit sets cannot be negative%
      }\@ehc
    \fi
  \fi
}
%    \end{macrocode}
%    \end{macro}
%    \begin{macro}{\BitSet@SetDecBig}
%    \begin{macrocode}
\def\BitSet@SetDecBig#1#2#3#4#5#6#7#8#9!{%
  \ifx\\#9\\%
    \BitSet@SetDec#1#2#3#4#5#6#7#8!%
  \else
    \ifcase\BigIntCalcOdd#1#2#4#5#6#7#8#9! %
      0%
    \or
      1%
?   \else\BitSetError:ThisCannotHappen%
    \fi
    \BitSet@AfterFi{%
      \expandafter\expandafter\expandafter\BitSet@SetDecBig
      \BigIntCalcShr#1#2#3#4#5#6#7#8#9!!%
    }%
  \BitSet@Fi
}
%    \end{macrocode}
%    \end{macro}
%    \begin{macro}{\BitSet@SetDec}
%    \begin{macrocode}
\def\BitSet@SetDec#1!{%
  \ifcase#1 %
  \or 1%
  \else
    \ifodd#1 %
      1%
    \else
      0%
    \fi
    \BitSet@AfterFi{%
      \expandafter\expandafter\expandafter\BitSet@SetDec
      \IntCalcShr#1!!%
    }%
  \BitSet@Fi
}
%    \end{macrocode}
%    \end{macro}
%
% \subsection{Export}\label{src:export}
%
% \subsubsection{To binary number}
%
%    \begin{macro}{\bitsetGetBin}
%    \begin{macrocode}
\def\bitsetGetBin#1#2{%
  \romannumeral0%
  \expandafter\expandafter\expandafter\BitSet@@GetBin
  \intcalcNum{#2}!{#1}%
}
%    \end{macrocode}
%    \end{macro}
%    \begin{macro}{\BitSet@@GetBin}
%    \begin{macrocode}
\def\BitSet@@GetBin#1!#2{%
  \BitSet@IfUndefined{#2}{%
    \ifnum#1>1 %
      \BitSet@AfterFi{%
        \expandafter\expandafter\expandafter\BitSet@Fill
        \IntCalcDec#1!!0%
      }%
    \else
      \BitSet@AfterFi{ 0}%
    \BitSet@Fi
  }{%
    \expandafter\expandafter\expandafter\BitSet@NumBinRev
    \expandafter\expandafter\expandafter1%
    \expandafter\expandafter\expandafter!%
    \csname BS@#2\endcsname!!#1!%
  }%
}
%    \end{macrocode}
%    \end{macro}
%    \begin{macro}{\BitSet@Fill}
%    |#1|: number of leading digits |0|\\
%    |#2|: result
%    \begin{macrocode}
\def\BitSet@Fill#1!{%
  \ifnum#1>0 %
    \BitSet@AfterFi{%
      \expandafter\expandafter\expandafter\BitSet@Fill
      \IntCalcDec#1!!0%
    }%
  \else
    \BitSet@AfterFi{ }%
  \BitSet@Fi
}
%    \end{macrocode}
%    \end{macro}
%
%    \begin{macro}{\BitSet@NumBinRev}
%    |#1|: bit counter (including |#2|)\\
%    |#2#3|: reverted number\\
%    |#4|: result\\
%    |#5|: min size
%    \begin{macrocode}
\def\BitSet@NumBinRev#1!#2#3!{%
  \ifx\\#3\\%
    \BitSet@AfterFi{%
      \BitSet@NumBinFill#1!#2%
    }%
  \else
    \BitSet@AfterFi{%
      \expandafter\expandafter\expandafter\BitSet@NumBinRev
      \IntCalcInc#1!!#3!#2%
    }%
  \BitSet@Fi
}
%    \end{macrocode}
%    \end{macro}
%    \begin{macro}{\BitSet@NumBinFill}
%    \begin{macrocode}
\def\BitSet@NumBinFill#1!#2!#3!{%
  \ifnum#3>#1 %
    \BitSet@AfterFi{%
      \expandafter\expandafter\expandafter\BitSet@Fill
      \IntCalcSub#3!#1!!#2%
    }%
  \else
    \BitSet@AfterFi{ #2}%
  \BitSet@Fi
}
%    \end{macrocode}
%    \end{macro}
%
% \subsubsection{To octal/hexadecimal number}
%
%    \begin{macro}{\bitsetGetOct}
%    \begin{macrocode}
\def\bitsetGetOct#1#2{%
  \romannumeral0%
  \bitsetIsEmpty{#1}{%
    \expandafter\expandafter\expandafter\BitSet@@GetOctHex
    \intcalcNum{#2}!3!230%
  }{%
    \expandafter\expandafter\expandafter\BitSet@@GetOct
    \expandafter\expandafter\expandafter1%
    \expandafter\expandafter\expandafter!%
    \expandafter\expandafter\expandafter!%
    \csname BS@#1\endcsname00%
    \BitSet@Empty\BitSet@Empty\BitSet@Empty!{#2}%
  }%
}
%    \end{macrocode}
%    \end{macro}
%    \begin{macro}{\bitsetGetHex}
%    \begin{macrocode}
\def\bitsetGetHex#1#2{%
  \romannumeral0%
  \bitsetIsEmpty{#1}{%
    \expandafter\expandafter\expandafter\BitSet@@GetOctHex
    \intcalcNum{#2}!4!340%
  }{%
    \expandafter\expandafter\expandafter\BitSet@@GetHex
    \expandafter\expandafter\expandafter1%
    \expandafter\expandafter\expandafter!%
    \expandafter\expandafter\expandafter!%
    \csname BS@#1\endcsname000%
    \BitSet@Empty\BitSet@Empty\BitSet@Empty\BitSet@Empty!{#2}%
  }%
}
%    \end{macrocode}
%    \end{macro}
%    \begin{macro}{\BitSet@@GetOct}
%    |#1|: number of digits\\
%    |#2|: result\\
%    |#3#4#5|: bits
%    \begin{macrocode}
\def\BitSet@@GetOct#1!#2!#3#4#5{%
  \ifx#5\BitSet@Empty
    \BitSet@AfterFi{%
      \expandafter\expandafter\expandafter\BitSet@GetOctHex
      \IntCalcDec#1!!#2!23%
    }%
  \else
    \BitSet@AfterFi{%
      \expandafter\expandafter\expandafter\BitSet@@GetOct
      \number\IntCalcInc#1!\expandafter\expandafter\expandafter!%
      \csname BitSet@Oct#5#4#3\endcsname#2!%
    }%
  \BitSet@Fi
}
%    \end{macrocode}
%    \end{macro}
%    \begin{macro}{\BitSet@Oct[000..111]}
%    \begin{macrocode}
\def\BitSet@Temp#1#2#3#4{%
  \expandafter\def\csname BitSet@Oct#1#2#3\endcsname{#4}%
}
\BitSet@Temp0000%
\BitSet@Temp0011%
\BitSet@Temp0102%
\BitSet@Temp0113%
\BitSet@Temp1004%
\BitSet@Temp1015%
\BitSet@Temp1106%
\BitSet@Temp1117%
%    \end{macrocode}
%    \end{macro}
%    \begin{macro}{\BitSet@@GetHex}
%    |#1|: number of digits\\
%    |#2|: result\\
%    |#3#4#5#6|: bits
%    \begin{macrocode}
\def\BitSet@@GetHex#1!#2!#3#4#5#6{%
  \ifx#6\BitSet@Empty
    \BitSet@AfterFi{%
      \expandafter\expandafter\expandafter\BitSet@GetOctHex
      \IntCalcDec#1!!#2!34%
    }%
  \else
    \BitSet@AfterFi{%
      \expandafter\expandafter\expandafter\BitSet@@GetHex
      \number\IntCalcInc#1!\expandafter\expandafter\expandafter!%
      \csname BitSet@Hex#6#5#4#3\endcsname#2!%
    }%
  \BitSet@Fi
}
%    \end{macrocode}
%    \end{macro}
%    \begin{macro}{\BitSet@Hex[0000..1111]}
%    \begin{macrocode}
\def\BitSet@Temp#1#2#3#4#5{%
  \expandafter\def\csname BitSet@Hex#1#2#3#4\endcsname{#5}%
}
\BitSet@Temp00000%
\BitSet@Temp00011%
\BitSet@Temp00102%
\BitSet@Temp00113%
\BitSet@Temp01004%
\BitSet@Temp01015%
\BitSet@Temp01106%
\BitSet@Temp01117%
\BitSet@Temp10008%
\BitSet@Temp10019%
\BitSet@Temp1010A%
\BitSet@Temp1011B%
\BitSet@Temp1100C%
\BitSet@Temp1101D%
\BitSet@Temp1110E%
\BitSet@Temp1111F%
%    \end{macrocode}
%    \end{macro}
%    \begin{macro}{\BitSet@GetOctHex}
%    Leading zeros $ (|#4| - |#1| * 3 + 2)/3 $ if $ |#4| > |#1| * 3$\\
%    |#1|: digit size\\
%    |#2|: result\\
%    |#3|: bits per digit - 1\\
%    |#4|: bits per digit
%    |#5|: garbage\\
%    |#6|: min size
%    \begin{macrocode}
\def\BitSet@GetOctHex#1!#2!#3#4#5!#6{%
  \expandafter\BitSet@@GetOctHex
  \number\intcalcNum{#6}\expandafter\expandafter\expandafter!%
  \IntCalcMul#1!#4!!#3#4#2%
}
%    \end{macrocode}
%    \end{macro}
%    \begin{macro}{\BitSet@@GetOctHex}
%    |#1|: plain min size\\
%    |#2|: digits * (bits per digit)\\
%    |#3|: bits per digit - 1\\
%    |#4|: bits per digit
%    \begin{macrocode}
\def\BitSet@@GetOctHex#1!#2!#3#4{%
  \ifnum#1>#2 %
    \BitSet@AfterFi{%
      \expandafter\expandafter\expandafter\expandafter
      \expandafter\expandafter\expandafter\BitSet@Fill
      \expandafter\IntCalcDiv\number
      \expandafter\expandafter\expandafter\IntCalcAdd
      \IntCalcSub#1!#2!!#3!!#4!!%
    }%
  \else
    \BitSet@AfterFi{ }%
  \BitSet@Fi
}
%    \end{macrocode}
%    \end{macro}
%
% \subsubsection{To decimal number}
%
%    \begin{macro}{\bitsetGetDec}
%    \begin{macrocode}
\def\bitsetGetDec#1{%
  \romannumeral0%
  \BitSet@IfUndefined{#1}{ 0}{%
    \expandafter\expandafter\expandafter\BitSet@GetDec
    \csname BS@#1\endcsname!%
  }%
}
%    \end{macrocode}
%    \end{macro}
%    \begin{macro}{\BitSet@GetDec}
%    \begin{macrocode}
\def\BitSet@GetDec#1#2!{%
  \ifx\\#2\\%
    \BitSet@AfterFi{ #1}%
  \else
    \BitSet@AfterFi{%
      \BitSet@@GetDec2!#1!#2!%
    }%
  \BitSet@Fi
}
%    \end{macrocode}
%    \end{macro}
%    \begin{macro}{\BitSet@@GetDec}
%    |#1|: power of two\\
%    |#2|: result\\
%    |#3#4|: number
%    \begin{macrocode}
\def\BitSet@@GetDec#1!#2!#3#4!{%
  \ifx\\#4\\%
    \ifx#31%
      \BitSet@AfterFiFi{%
        \expandafter\expandafter\expandafter\BitSet@Space
        \IntCalcAdd#1!#2!%
      }%
    \else
      \BitSet@AfterFiFi{ #2}%
    \fi
  \else
    \ifx#31%
      \BitSet@AfterFiFi{%
        \csname BitSet@N#1%
        \expandafter\expandafter\expandafter\endcsname
        \IntCalcAdd#1!#2!!#4!%
      }%
    \else
      \BitSet@AfterFiFi{%
        \csname BitSet@N#1\endcsname#2!#4!%
      }%
    \fi
  \BitSet@Fi
}
%    \end{macrocode}
%    \end{macro}
%    \begin{macro}{\BitSet@N[1,2,4,...]}
%    \begin{macrocode}
\def\BitSet@Temp#1#2{%
  \expandafter\def\csname BitSet@N#1\endcsname{%
    \BitSet@@GetDec#2!%
  }%
}
\BitSet@Temp{1}{2}
\BitSet@Temp{2}{4}
\BitSet@Temp{4}{8}
\BitSet@Temp{8}{16}
\BitSet@Temp{16}{32}
\BitSet@Temp{32}{64}
\BitSet@Temp{64}{128}
\BitSet@Temp{128}{256}
\BitSet@Temp{256}{512}
\BitSet@Temp{512}{1024}
\BitSet@Temp{1024}{2048}
\BitSet@Temp{2048}{4096}
\BitSet@Temp{4096}{8192}
\BitSet@Temp{8192}{16384}
\BitSet@Temp{16384}{32768}
\BitSet@Temp{32768}{65536}
\BitSet@Temp{65536}{131072}
\BitSet@Temp{131072}{262144}
\BitSet@Temp{262144}{524288}
\BitSet@Temp{524288}{1048576}
\BitSet@Temp{1048576}{2097152}
\BitSet@Temp{2097152}{4194304}
\BitSet@Temp{4194304}{8388608}
\BitSet@Temp{8388608}{16777216}
\BitSet@Temp{16777216}{33554432}
\BitSet@Temp{33554432}{67108864}
\BitSet@Temp{67108864}{134217728}
\BitSet@Temp{134217728}{268435456}
\BitSet@Temp{268435456}{536870912}
\BitSet@Temp{536870912}{1073741824}
%    \end{macrocode}
%    \end{macro}
%    \begin{macro}{\BitSet@N1073741824}
%    \begin{macrocode}
\expandafter\def\csname BitSet@N1073741824\endcsname{%
  \BitSet@GetDecBig2147483648!%
}%
%    \end{macrocode}
%    \end{macro}
%
%    \begin{macro}{\BitSet@GetDecBig}
%    |#1|: current power of two\\
%    |#2|: result\\
%    |#3#4|: number
%    \begin{macrocode}
\def\BitSet@GetDecBig#1!#2!#3#4!{%
  \ifx\\#4\\%
    \ifx#31%
      \BitSet@AfterFiFi{%
        \expandafter\expandafter\expandafter\BitSet@Space
        \BigIntCalcAdd#1!#2!%
      }%
    \else
      \BitSet@AfterFiFi{ #2}%
    \fi
  \else
    \ifx#31%
      \BitSet@AfterFiFi{%
        \expandafter\expandafter\expandafter\BitSet@@GetDecBig
        \BigIntCalcAdd#1!#2!!#1!#4!%
      }%
    \else
      \BitSet@AfterFiFi{%
        \expandafter\expandafter\expandafter\BitSet@GetDecBig
        \BigIntCalcShl#1!!#2!#4!%
      }%
    \fi
  \BitSet@Fi
}
%    \end{macrocode}
%    \end{macro}
%    \begin{macro}{\BitSet@@GetDecBig}
%    |#1|: result\\
%    |#2|: power of two\\
%    |#3#4|: number
%    \begin{macrocode}
\def\BitSet@@GetDecBig#1!#2!{%
  \expandafter\expandafter\expandafter\BitSet@GetDecBig
  \BigIntCalcShl#2!!#1!%
}
%    \end{macrocode}
%    \end{macro}
%
% \subsection{Logical operators}\label{src:log}
%
% \subsubsection{\cs{bitsetAnd}}
%
%    \begin{macro}{\bitsetAnd}
%    Decision table for \cs{bitsetAnd}:
%    \begin{quote}
%    \begin{tabular}{l|l|l|l|}
%                         & undef(B)   & empty(B)   & cardinality(B)$>$0\\
%      \hline
%      undef(A)           & A := empty & A := empty & A := empty\\
%      \hline
%      empty(A)           &            &            &\\
%      \hline
%      cardinality(A)$>$0 & A := empty & A := empty & A \&= B\\
%      \hline
%    \end{tabular}
%    \end{quote}
%    \begin{macrocode}
\def\bitsetAnd#1#2{%
  \bitsetIsEmpty{#1}{%
    \bitsetReset{#1}%
  }{%
    \bitsetIsEmpty{#2}{%
      \bitsetReset{#1}%
    }{%
      \expandafter\edef\csname BS@#1\endcsname{%
        \expandafter\expandafter\expandafter\BitSet@And
        \csname BS@#1\expandafter\expandafter\expandafter\endcsname
        \expandafter\expandafter\expandafter!%
        \csname BS@#2\endcsname!!%
      }%
      \expandafter\ifx\csname BS@#1\endcsname\BitSet@Empty
        \bitsetReset{#1}%
      \fi
    }%
  }%
}
%    \end{macrocode}
%    \end{macro}
%    \begin{macro}{\BitSet@And}
%    \begin{macrocode}
\def\BitSet@And#1#2!#3#4!#5!{%
  \ifx\\#2\\%
    \ifnum#1#3=11 #51\fi
  \else
    \ifx\\#4\\%
      \ifnum#1#3=11 #51\fi
    \else
      \ifnum#1#3=11 %
        #51%
        \BitSet@AfterFiFiFi{%
          \BitSet@And#2!#4!!%
        }%
      \else
        \BitSet@AfterFiFiFi{%
          \BitSet@And#2!#4!#50!%
        }%
      \fi
    \fi
  \BitSet@Fi
}
%    \end{macrocode}
%    \end{macro}
%
% \subsubsection{\cs{bitsetAndNot}}
%
%    \begin{macro}{\bitsetAndNot}
%    Decision table for \cs{bitsetAndNot}:
%    \begin{quote}
%    \begin{tabular}{l|l|l|l|}
%                         & undef(B)   & empty(B)   & cardinality(B)$>$0\\
%      \hline
%      undef(A)           & A := empty & A := empty & A := empty\\
%      \hline
%      empty(A)           &            &            &\\
%      \hline
%      cardinality(A)$>$0 &            &            & A \&= !B\\
%      \hline
%    \end{tabular}
%    \end{quote}
%    \begin{macrocode}
\def\bitsetAndNot#1#2{%
  \bitsetIsEmpty{#1}{%
    \bitsetReset{#1}%
  }{%
    \bitsetIsEmpty{#2}{%
    }{%
      \expandafter\edef\csname BS@#1\endcsname{%
        \expandafter\expandafter\expandafter\BitSet@AndNot
        \csname BS@#1\expandafter\expandafter\expandafter\endcsname
        \expandafter\expandafter\expandafter!%
        \csname BS@#2\endcsname!!%
      }%
      \expandafter\ifx\csname BS@#1\endcsname\BitSet@Empty
        \bitsetReset{#1}%
      \fi
    }%
  }%
}
%    \end{macrocode}
%    \end{macro}
%    \begin{macro}{\BitSet@AndNot}
%    \begin{macrocode}
\def\BitSet@AndNot#1#2!#3#4!#5!{%
  \ifx\\#2\\%
    \ifnum#1#3=10 #51\fi
  \else
    \ifx\\#4\\%
      #5%
      \ifnum#1#3=10 1\else 0\fi
      #2%
    \else
      \ifnum#1#3=10 %
        #51%
        \BitSet@AfterFiFiFi{%
          \BitSet@AndNot#2!#4!!%
        }%
      \else
        \BitSet@AfterFiFiFi{%
          \BitSet@AndNot#2!#4!#50!%
        }%
      \fi
    \fi
  \BitSet@Fi
}
%    \end{macrocode}
%    \end{macro}
%
% \subsubsection{\cs{bitsetOr}}
%
%    \begin{macro}{\bitsetOr}
%    Decision table for \cs{bitsetOr}:
%    \begin{quote}
%    \begin{tabular}{l|l|l|l|}
%                         & undef(B)   & empty(B)   & cardinality(B)$>$0\\
%      \hline
%      undef(A)           & A := empty & A := empty & A := B\\
%      \hline
%      empty(A)           &            &            & A := B\\
%      \hline
%      cardinality(A)$>$0 &            &            & A \string|= B\\
%      \hline
%    \end{tabular}
%    \end{quote}
%    \begin{macrocode}
\def\bitsetOr#1#2{%
  \bitsetIsEmpty{#2}{%
    \BitSet@IfUndefined{#1}{\bitsetReset{#1}}{}%
  }{%
    \bitsetIsEmpty{#1}{%
      \expandafter\let\csname BS@#1\expandafter\endcsname
                      \csname BS@#2\endcsname
    }{%
      \expandafter\edef\csname BS@#1\endcsname{%
        \expandafter\expandafter\expandafter\BitSet@Or
        \csname BS@#1\expandafter\expandafter\expandafter\endcsname
        \expandafter\expandafter\expandafter!%
        \csname BS@#2\endcsname!%
      }%
    }%
  }%
}
%    \end{macrocode}
%    \end{macro}
%    \begin{macro}{\BitSet@Or}
%    \begin{macrocode}
\def\BitSet@Or#1#2!#3#4!{%
  \ifnum#1#3>0 1\else 0\fi
  \ifx\\#2\\%
    #4%
  \else
    \ifx\\#4\\%
      #2%
    \else
      \BitSet@AfterFiFi{%
        \BitSet@Or#2!#4!%
      }%
    \fi
  \BitSet@Fi
}
%    \end{macrocode}
%    \end{macro}
%
% \subsubsection{\cs{bitsetXor}}
%
%    \begin{macro}{\bitsetXor}
%    Decision table for \cs{bitsetXor}:
%    \begin{quote}
%    \begin{tabular}{l|l|l|l|}
%                         & undef(B)   & empty(B)   & cardinality(B)$>$0\\
%      \hline
%      undef(A)           & A := empty & A := empty & A := B\\
%      \hline
%      empty(A)           &            &            & A := B\\
%      \hline
%      cardinality(A)$>$0 &            &            & A \string^= B\\
%      \hline
%    \end{tabular}
%    \end{quote}
%    \begin{macrocode}
\def\bitsetXor#1#2{%
  \bitsetIsEmpty{#2}{%
    \BitSet@IfUndefined{#1}{\bitsetReset{#1}}{}%
  }{%
    \bitsetIsEmpty{#1}{%
      \expandafter\let\csname BS@#1\expandafter\endcsname
                      \csname BS@#2\endcsname
    }{%
      \expandafter\edef\csname BS@#1\endcsname{%
        \expandafter\expandafter\expandafter\BitSet@Xor
        \csname BS@#1\expandafter\expandafter\expandafter\endcsname
        \expandafter\expandafter\expandafter!%
        \csname BS@#2\endcsname!!%
      }%
      \expandafter\ifx\csname BS@#1\endcsname\BitSet@Empty
        \bitsetReset{#1}%
      \fi
    }%
  }%
}
%    \end{macrocode}
%    \end{macro}
%    \begin{macro}{\BitSet@Xor}
%    \begin{macrocode}
\def\BitSet@Xor#1#2!#3#4!#5!{%
  \ifx\\#2\\%
    \ifx#1#3%
      \ifx\\#4\\%
      \else
        #50#4%
      \fi
    \else
      #51#4%
    \fi
  \else
    \ifx\\#4\\%
      #5%
      \ifx#1#30\else 1\fi
      #2%
    \else
      \ifx#1#3%
        \BitSet@AfterFiFiFi{%
          \BitSet@Xor#2!#4!#50!%
        }%
      \else
        #51%
        \BitSet@AfterFiFiFi{%
          \BitSet@Xor#2!#4!!%
        }%
      \fi
    \fi
  \BitSet@Fi
}
%    \end{macrocode}
%    \end{macro}
%
% \subsubsection{Shifting}\label{src:shift}
%
% \subsubsection{\cs{bitsetShiftLeft}}
%
%    \begin{macro}{\bitsetShiftLeft}
%    \begin{macrocode}
\def\bitsetShiftLeft#1#2{%
  \BitSet@IfUndefined{#1}{%
    \bitsetReset{#1}%
  }{%
    \bitsetIsEmpty{#1}{%
    }{%
      \expandafter\expandafter\expandafter\BitSet@ShiftLeft
      \intcalcNum{#2}!{#1}%
    }%
  }%
}
%    \end{macrocode}
%    \end{macro}
%    \begin{macro}{\BitSet@ShiftLeft}
%    \begin{macrocode}
\def\BitSet@ShiftLeft#1!#2{%
  \ifcase\intcalcSgn{#1} %
  \or
    \begingroup
      \uccode`m=`0 %
    \uppercase\expandafter{\expandafter\endgroup
      \expandafter\edef\csname BS@#2\expandafter\endcsname
      \expandafter{%
        \romannumeral#1000\expandafter\BitSet@Space
        \csname BS@#2\endcsname
      }%
    }%
  \else
    \expandafter\BitSet@ShiftRight\BitSet@Gobble#1!{#2}%
  \fi
}
%    \end{macrocode}
%    \end{macro}
%
% \subsubsection{\cs{bitsetShiftRight}}
%
%    \begin{macro}{\bitsetShiftRight}
%    \begin{macrocode}
\def\bitsetShiftRight#1#2{%
  \BitSet@IfUndefined{#1}{%
    \bitsetReset{#1}%
  }{%
    \bitsetIsEmpty{#1}{%
    }{%
      \expandafter\expandafter\expandafter\BitSet@ShiftRight
      \intcalcNum{#2}!{#1}%
    }%
  }%
}
%    \end{macrocode}
%    \end{macro}
%    \begin{macro}{\BitSet@ShiftRight}
%    \begin{macrocode}
\def\BitSet@ShiftRight#1!#2{%
  \ifcase\intcalcSgn{#1} %
  \or
    \expandafter\edef\csname BS@#2\endcsname{%
      \expandafter\expandafter\expandafter\BitSet@Kill
      \csname BS@#2\expandafter\endcsname\expandafter\BitSet@Empty
      \expandafter=%
      \expandafter{\expandafter}\expandafter{\expandafter}%
      \romannumeral#1000!%
    }%
  \else
    \expandafter\BitSet@ShiftLeft\BitSet@Gobble#1!{#2}%
  \fi
}
%    \end{macrocode}
%    \end{macro}
%    \begin{macro}{\BitSet@Kill}
%    \begin{macrocode}
\def\BitSet@Kill#1#2=#3#4#5{%
  #3#4%
  \ifx#5!%
    \ifx#1\BitSet@Empty
      0%
    \else
      #1#2%
    \fi
  \else
    \ifx#1\BitSet@Empty
      0%
      \BitSet@AfterFiFi\BitSet@Cleanup
    \else
      \BitSet@Kill#2=%
    \fi
  \BitSet@Fi
}
%    \end{macrocode}
%    \end{macro}
%
% \subsection{Bit manipulation}\label{src:bitman}
%
%    \begin{macro}{\bitsetClear}
%    \begin{macrocode}
\def\bitsetClear{%
  \BitSet@CheckIndex\BitSet@Clear
}
%    \end{macrocode}
%    \end{macro}
%    \begin{macro}{\bitsetSet}
%    \begin{macrocode}
\def\bitsetSet{%
  \BitSet@CheckIndex\BitSet@Set
}
%    \end{macrocode}
%    \end{macro}
%    \begin{macro}{\bitsetFlip}
%    \begin{macrocode}
\def\bitsetFlip{%
  \BitSet@CheckIndex\BitSet@Flip
}
%    \end{macrocode}
%    \end{macro}
%    \begin{macro}{\bitsetSetValue}
%    \begin{macrocode}
\def\bitsetSetValue#1#2#3{%
  \expandafter\expandafter\expandafter\BitSet@SetValue
  \intcalcNum{#3}!{#1}{#2}%
}
%    \end{macrocode}
%    \end{macro}
%    \begin{macro}{\BitSet@SetValue}
%    |#1|: plain value\\
%    |#2|: BitSet\\
%    |#3|: Index
%    \begin{macrocode}
\def\BitSet@SetValue#1!{%
  \BitSet@CheckIndex{%
    \ifcase#1 %
      \expandafter\BitSet@Clear
    \or
      \expandafter\BitSet@Set
    \else
      \BitSet@ErrorInvalidBitValue{#1}%
      \expandafter\expandafter\expandafter\BitSet@Gobble
      \expandafter\BitSet@Gobble
    \fi
  }%
}
%    \end{macrocode}
%    \end{macro}
%    \begin{macro}{\BitSet@ErrorInvalidBitValue}
%    |#1|: Wrong bit value
%    \begin{macrocode}
\def\BitSet@ErrorInvalidBitValue#1{%
  \@PackageError{bitset}{%
    Invalid bit value (#1) not in range 0..1%
  }\@ehc
}
%    \end{macrocode}
%    \end{macro}
%
% \subsubsection{Clear operation}
%
%    \begin{macro}{\BitSet@Clear}
%    |#1|: BitSet\\
%    |#2|: plain and checked index
%    \begin{macrocode}
\def\BitSet@Clear#1#2{%
  \edef\BitSet@Temp{%
    \expandafter\expandafter\expandafter\BitSet@@Clear
    \csname BS@#1\expandafter\endcsname
    \expandafter\BitSet@Empty\expandafter=\expandafter!%
    \romannumeral#2000!%
  }%
  \expandafter\let\csname BS@#1\expandafter\endcsname
  \ifx\BitSet@Temp\BitSet@Empty
    \BitSet@Zero
  \else
    \BitSet@Temp
  \fi
}
%    \end{macrocode}
%    \end{macro}
%    \begin{macro}{\BitSet@@Clear}
%    \begin{macrocode}
\def\BitSet@@Clear#1#2=#3!#4{%
  \ifx#4!%
    \ifx#1\BitSet@Empty
    \else
      \ifx\BitSet@Empty#2%
      \else
        #30#2%
      \fi
    \fi
  \else
    \ifx#1\BitSet@Empty
      \BitSet@AfterFiFi\BitSet@Cleanup
    \else
      \ifx#10%
        \BitSet@AfterFiFiFi{%
          \BitSet@@Clear#2=#30!%
        }%
      \else
        #31%
        \BitSet@AfterFiFiFi{%
          \BitSet@@Clear#2=!%
        }%
      \fi
    \fi
  \BitSet@Fi
}
%    \end{macrocode}
%    \end{macro}
%
% \subsubsection{Set operation}
%
%    \begin{macro}{\BitSet@Set}
%    |#1|: BitSet\\
%    |#2|: plain and checked Index
%    \begin{macrocode}
\def\BitSet@Set#1#2{%
  \expandafter\edef\csname BS@#1\endcsname{%
    \expandafter\expandafter\expandafter\BitSet@@Set
    \csname BS@#1\expandafter\endcsname
    \expandafter\BitSet@Empty\expandafter=%
    \expandafter{\expandafter}\expandafter{\expandafter}%
    \romannumeral#2000!%
  }%
}
%    \end{macrocode}
%    \end{macro}
%    \begin{macro}{\BitSet@@Set}
%    \begin{macrocode}
\def\BitSet@@Set#1#2=#3#4#5{%
  #3#4%
  \ifx#5!%
    1#2%
  \else
    \ifx#1\BitSet@Empty
      0%
      \BitSet@AfterFiFi\BitSet@@@Set
    \else
      #1%
      \BitSet@@Set#2=%
    \fi
  \BitSet@Fi
}
%    \end{macrocode}
%    \end{macro}
%    \begin{macro}{\BitSet@@@Set}
%    \begin{macrocode}
\def\BitSet@@@Set#1{%
  \ifx#1!%
    1%
  \else
    0%
    \expandafter\BitSet@@@Set
  \fi
}
%    \end{macrocode}
%    \end{macro}
%
% \subsubsection{Flip operation}
%
%    \begin{macro}{\BitSet@Flip}
%    |#1|: BitSet\\
%    |#2|: plain and checked Index
%    \begin{macrocode}
\def\BitSet@Flip#1#2{%
  \edef\BitSet@Temp{%
    \expandafter\expandafter\expandafter\BitSet@@Flip
    \csname BS@#1\expandafter\endcsname
    \expandafter\BitSet@Empty\expandafter=\expandafter!%
    \romannumeral#2000!%
  }%
  \expandafter\let\csname BS@#1\expandafter\endcsname
  \ifx\BitSet@Temp\BitSet@Empty
    \BitSet@Zero
  \else
    \BitSet@Temp
  \fi
}
%    \end{macrocode}
%    \end{macro}
%    \begin{macro}{\BitSet@@Flip}
%    \begin{macrocode}
\def\BitSet@@Flip#1#2=#3!#4{%
  \ifx#4!%
    \ifx#11%
      \ifx\BitSet@Empty#2%
      \else
        #30#2%
      \fi
    \else
      #31#2%
    \fi
  \else
    \ifx#1\BitSet@Empty
      #30%
      \BitSet@AfterFiFi\BitSet@@@Set
    \else
      \ifx#10%
        \BitSet@AfterFiFiFi{%
          \BitSet@@Flip#2=#30!%
        }%
      \else
        #31%
        \BitSet@AfterFiFiFi{%
          \BitSet@@Flip#2=!%
        }%
      \fi
    \fi
  \BitSet@Fi
}
%    \end{macrocode}
%    \end{macro}
%
% \subsubsection{Range operators}
%
%    \begin{macro}{\bitsetClearRange}
%    \begin{macrocode}
\def\bitsetClearRange{%
  \BitSet@Range\BitSet@Clear
}
%    \end{macrocode}
%    \end{macro}
%    \begin{macro}{\bitsetSetRange}
%    \begin{macrocode}
\def\bitsetSetRange{%
  \BitSet@Range\BitSet@Set
}
%    \end{macrocode}
%    \end{macro}
%    \begin{macro}{\bitsetFlipRange}
%    \begin{macrocode}
\def\bitsetFlipRange{%
  \BitSet@Range\BitSet@Flip
}
%    \end{macrocode}
%    \end{macro}
%    \begin{macro}{\bitsetSetValueRange}
%    \begin{macrocode}
\def\bitsetSetValueRange#1#2#3#4{%
  \expandafter\expandafter\expandafter\BitSet@SetValueRange
  \intcalcNum{#4}!{#1}{#2}{#3}%
}
%    \end{macrocode}
%    \end{macro}
%    \begin{macro}{\BitSet@SetValueRange}
%    \begin{macrocode}
\def\BitSet@SetValueRange#1!#2#3#4{%
  \ifcase#1 %
    \BitSet@Range\BitSet@Clear{#2}{#3}{#4}%
  \or
    \BitSet@Range\BitSet@Set{#2}{#3}{#4}%
  \else
    \BitSet@ErrorInvalidBitValue{#1}%
  \fi
}
%    \end{macrocode}
%    \end{macro}
%
%    \begin{macro}{\BitSet@Range}
%    |#1|: clear/set/flip macro\\
%    |#2|: BitSet\\
%    |#3|: Index from\\
%    |#4|: Index to
%    \begin{macrocode}
\def\BitSet@Range#1#2#3#4{%
  \edef\BitSet@Temp{%
    \noexpand\BitSet@@Range\noexpand#1{#2}%
    \intcalcNum{#3}!\intcalcNum{#4}!%
  }%
  \BitSet@Temp
}
%    \end{macrocode}
%    \end{macro}
%    \begin{macro}{\BitSet@@Range}
%    |#1|: clear/set/flip macro\\
%    |#2|: BitSet\\
%    |#3|: Index from\\
%    |#4|: Index to
%    \begin{macrocode}
\def\BitSet@@Range#1#2#3!#4!{%
  \ifnum#3<0 %
    \BitSet@NegativeIndex#1{#2}#3!#4!0!#4!%
  \else
    \ifnum#4<0 %
      \BitSet@NegativeIndex#1{#2}#3!#4!#3!0!%
    \else
      \ifcase\intcalcCmp{#3}{#4} %
      \or
        \@PackageError{bitset}{%
          Wrong index numbers in range [#3..#4]\MessageBreak% hash-ok
          for clear/set/flip on bit set `#2'.\MessageBreak
          The lower index exceeds the upper index.\MessageBreak
          Canceling the operation as error recovery%
        }\@ehc
      \else
        \BitSet@@@Range#3!#4!#1{#2}%
      \fi
    \fi
  \fi
}
%    \end{macrocode}
%    \end{macro}
%    \begin{macro}{\BitSet@NegativeIndex}
%    \begin{macrocode}
\def\BitSet@NegativeIndex#1#2#3!#4!#5!#6!{%
  \@PackageError{bitset}{%
    Negative index in range [#3..#4]\MessageBreak % hash-ok
    for \string\bitset
    \ifx#1\BitSet@Clear
      Clear%
    \else
      \ifx#1\BitSet@Set
        Set%
      \else
        Flip%
      \fi
    \fi
    Range on bit set `#2'.\MessageBreak
    Using [#5..#6] as error recovery% hash-ok
  }\@ehc
  \BitSet@@Range#1{#2}#5!#6!%
}
%    \end{macrocode}
%    \end{macro}
%    \begin{macro}{\BitSet@@Range}
%    \begin{macrocode}
\def\BitSet@@@Range#1!#2!#3#4{%
  \ifnum#1<#2 %
    #3{#4}{#1}%
    \BitSet@AfterFi{%
      \expandafter\expandafter\expandafter\BitSet@@@Range
      \IntCalcInc#1!!#2!#3{#4}%
    }%
  \BitSet@Fi
}
%    \end{macrocode}
%    \end{macro}
%
% \subsection{Bit retrieval}\label{src:bitret}
%
% \subsubsection{\cs{bitsetGet}}
%
%    \begin{macro}{\bitsetGet}
%    \begin{macrocode}
\def\bitsetGet#1#2{%
  \number
  \expandafter\expandafter\expandafter\BitSet@Get
  \intcalcNum{#2}!{#1}%
}
%    \end{macrocode}
%    \end{macro}
%    \begin{macro}{\BitSet@Get}
%    |#1|: plain index\\
%    |#2|: BitSet
%    \begin{macrocode}
\def\BitSet@Get#1!#2{%
  \ifnum#1<0 %
    \BitSet@AfterFi{%
      0 \BitSetError:NegativeIndex%
    }%
  \else
    \BitSet@IfUndefined{#2}{0}{%
      \expandafter\expandafter\expandafter\BitSet@@Get
      \csname BS@#2\expandafter\endcsname
      \expandafter!\expandafter=%
      \expandafter{\expandafter}\expandafter{\expandafter}%
      \romannumeral\intcalcNum{#1}000!%
    }%
    \expandafter\BitSet@Space
  \BitSet@Fi
}
%    \end{macrocode}
%    \end{macro}
%    \begin{macro}{\BitSet@@Get}
%    \begin{macrocode}
\def\BitSet@@Get#1#2=#3#4#5{%
  #3#4%
  \ifx#5!%
    \ifx#1!%
      0%
    \else
      #1%
    \fi
  \else
    \ifx#1!%
      0%
      \BitSet@AfterFiFi\BitSet@Cleanup
    \else
      \BitSet@@Get#2=%
    \fi
  \BitSet@Fi
}
%    \end{macrocode}
%    \end{macro}
%
% \subsubsection{\cs{bitsetNextClearBit}, \cs{bitsetNextSetBit}}
%
%    \begin{macro}{\bitsetNextClearBit}
%    \begin{macrocode}
\def\bitsetNextClearBit#1#2{%
  \number
  \expandafter\expandafter\expandafter\BitSet@NextClearBit
  \intcalcNum{#2}!{#1} %
}
%    \end{macrocode}
%    \end{macro}
%    \begin{macro}{\BitSet@NextClearBit}
%    |#1|: Index\\
%    |#2|: BitSet
%    \begin{macrocode}
\def\BitSet@NextClearBit#1!#2{%
  \ifnum#1<0 %
    \BitSet@NextClearBit0!{#2}%
    \BitSet@AfterFi{%
      \expandafter\BitSet@Space
      \expandafter\BitSetError:NegativeIndex\romannumeral0%
    }%
  \else
    \bitsetIsEmpty{#2}{#1}{%
      \expandafter\BitSet@Skip
      \number#1\expandafter\expandafter\expandafter!%
      \csname BS@#2\endcsname!!!!!!!!!=%
      {\BitSet@@NextClearBit#1!}%
    }%
  \BitSet@Fi
}
%    \end{macrocode}
%    \end{macro}
%    \begin{macro}{\BitSet@@NextClearBit}
%    |#1|: index for next bit in |#2|\\
%    |#2|: next bit
%    \begin{macrocode}
\def\BitSet@@NextClearBit#1!#2{%
  \ifx#2!%
    #1%
  \else
    \ifx#20%
      #1%
      \BitSet@AfterFiFi\BitSet@Cleanup
    \else
      \BitSet@AfterFiFi{%
        \expandafter\expandafter\expandafter\BitSet@@NextClearBit
        \IntCalcInc#1!!%
      }%
    \fi
  \BitSet@Fi
}
%    \end{macrocode}
%    \end{macro}
%
%    \begin{macro}{\bitsetNextSetBit}
%    \begin{macrocode}
\def\bitsetNextSetBit#1#2{%
  \number
  \expandafter\expandafter\expandafter\BitSet@NextSetBit
  \intcalcNum{#2}!{#1} %
}
%    \end{macrocode}
%    \end{macro}
%    \begin{macro}{\BitSet@NextSetBit}
%    |#1|: Index\\
%    |#2|: BitSet
%    \begin{macrocode}
\def\BitSet@NextSetBit#1!#2{%
  \ifnum#1<0 %
    \BitSet@NextSetBit0!{#2}%
    \BitSet@AfterFi{%
      \expandafter\BitSet@Space
      \expandafter\BitSetError:NegativeIndex\romannumeral0%
    }%
  \else
    \bitsetIsEmpty{#2}{-1}{%
      \expandafter\BitSet@Skip
      \number#1\expandafter\expandafter\expandafter!%
      \csname BS@#2\endcsname!!!!!!!!!=%
      {\BitSet@@NextSetBit#1!}%
    }%
  \BitSet@Fi
}
%    \end{macrocode}
%    \end{macro}
%    \begin{macro}{\BitSet@@NextSetBit}
%    |#1|: index for next bit in |#2|\\
%    |#2|: next bit
%    \begin{macrocode}
\def\BitSet@@NextSetBit#1!#2{%
  \ifx#2!%
    -1%
  \else
    \ifx#21%
      #1%
      \BitSet@AfterFiFi\BitSet@Cleanup
    \else
      \BitSet@AfterFiFi{%
        \expandafter\expandafter\expandafter\BitSet@@NextSetBit
        \IntCalcInc#1!!%
      }%
    \fi
  \BitSet@Fi
}
%    \end{macrocode}
%    \end{macro}
%    \begin{macro}{\BitSet@Cleanup}
%    \begin{macrocode}
\def\BitSet@Cleanup#1!{}
%    \end{macrocode}
%    \end{macro}
%
%    \begin{macro}{\BitSet@Skip}
%    |#1|: number of bits to skip\\
%    |#2|: bits\\
%    |#3|: continuation code
%    \begin{macrocode}
\def\BitSet@Skip#1!#2{%
  \ifx#2!%
    \BitSet@AfterFi{%
      \BitSet@SkipContinue%
    }%
  \else
    \ifcase#1 %
      \BitSet@AfterFiFi{%
        \BitSet@SkipContinue#2%
      }%
    \or
      \BitSet@AfterFiFi\BitSet@SkipContinue
    \or
      \BitSet@AfterFiFi{%
        \expandafter\BitSet@SkipContinue\BitSet@Gobble
      }%
    \else
      \ifnum#1>8 %
        \BitSet@AfterFiFiFi{%
          \expandafter\BitSet@Skip
          \number\IntCalcSub#1!8!\expandafter!%
          \BitSet@GobbleSeven
        }%
      \else
        \BitSet@AfterFiFiFi{%
          \expandafter\expandafter\expandafter\BitSet@Skip
          \IntCalcDec#1!!%
        }%
      \fi
    \fi
  \BitSet@Fi
}
%    \end{macrocode}
%    \end{macro}
%    \begin{macro}{\BitSet@SkipContinue}
%    |#1|: remaining bits\\
%    |#2|: continuation code
%    \begin{macrocode}
\def\BitSet@SkipContinue#1!#2=#3{%
  #3#1!%
}
%    \end{macrocode}
%    \end{macro}
%    \begin{macro}{\BitSet@GobbleSeven}
%    \begin{macrocode}
\def\BitSet@GobbleSeven#1#2#3#4#5#6#7{}
%    \end{macrocode}
%    \end{macro}
%
% \subsubsection{\cs{bitsetGetSetBitList}}
%
%    \begin{macro}{\bitsetGetSetBitList}
%    It's just a wrapper for \cs{bitsetNextSetBit}.
%    \begin{macrocode}
\def\bitsetGetSetBitList#1{%
  \romannumeral0%
  \bitsetIsEmpty{#1}{ }{%
    \expandafter\BitSet@GetSetBitList
    \number\BitSet@NextSetBit0!{#1}!{#1}{}!%
  }%
}
%    \end{macrocode}
%    \end{macro}
%    \begin{macro}{\BitSet@GetSetBitList}
%    |#1|: found index\\
%    |#2|: BitSet\\
%    |#3|: comma
%    |#4|: result
%    \begin{macrocode}
\def\BitSet@GetSetBitList#1!#2#3#4!{%
  \ifnum#1<0 %
    \BitSet@AfterFi{ #4}%
  \else
    \BitSet@AfterFi{%
      \expandafter\BitSet@GetSetBitList\number
      \expandafter\expandafter\expandafter\BitSet@NextSetBit
      \IntCalcInc#1!!{#2}!{#2},#4#3#1!%
    }%
  \BitSet@Fi
}
%    \end{macrocode}
%    \end{macro}
%
% \subsection{Bit set properties}\label{src:prop}
%
%    \begin{macro}{\bitsetSize}
%    \begin{macrocode}
\def\bitsetSize#1{%
  \number
  \BitSet@IfUndefined{#1}{0 }{%
    \expandafter\expandafter\expandafter\BitSet@Size
    \expandafter\expandafter\expandafter1%
    \expandafter\expandafter\expandafter!%
    \csname BS@#1\endcsname!0!%
  }%
}
%    \end{macrocode}
%    \end{macro}
%    \begin{macro}{\BitSet@Size}
%    |#1|: counter\\
%    |#2#3|: bits\\
%    |#4|: result
%    \begin{macrocode}
\def\BitSet@Size#1!#2#3!#4!{%
  \ifx#21%
    \ifx\\#3\\%
      \BitSet@AfterFiFi{#1 }%
    \else
      \BitSet@AfterFiFi{%
        \expandafter\expandafter\expandafter\BitSet@Size
        \IntCalcInc#1!!#3!#1!%
      }%
    \fi
  \else
    \ifx\\#3\\%
      \BitSet@AfterFiFi{#4 }%
    \else
      \BitSet@AfterFiFi{%
        \expandafter\expandafter\expandafter\BitSet@Size
        \IntCalcInc#1!!#3!#4!%
      }%
    \fi
  \fi
  \BitSet@Fi
}
%    \end{macrocode}
%    \end{macro}
%
%    \begin{macro}{\bitsetCardinality}
%    \begin{macrocode}
\def\bitsetCardinality#1{%
  \number
  \BitSet@IfUndefined{#1}{0 }{%
    \expandafter\expandafter\expandafter\BitSet@Cardinality
    \expandafter\expandafter\expandafter0%
    \expandafter\expandafter\expandafter!%
    \csname BS@#1\endcsname!%
  }%
}
%    \end{macrocode}
%    \end{macro}
%    \begin{macro}{\BitSet@Cardinality}
%    |#1|: result\\
%    |#2#3|: bits
%    \begin{macrocode}
\def\BitSet@Cardinality#1!#2#3!{%
  \ifx#21%
    \ifx\\#3\\%
      \BitSet@AfterFiFi{\IntCalcInc#1! }%
    \else
      \BitSet@AfterFiFi{%
        \expandafter\expandafter\expandafter\BitSet@Cardinality
        \IntCalcInc#1!!#3!%
      }%
    \fi
  \else
    \ifx\\#3\\%
      \BitSet@AfterFiFi{#1 }%
    \else
      \BitSet@AfterFiFi{%
        \BitSet@Cardinality#1!#3!%
      }%
    \fi
  \fi
  \BitSet@Fi
}
%    \end{macrocode}
%    \end{macro}
%
% \subsection{Queries}\label{src:query}
%
%    \begin{macro}{\bitsetIsDefined}
%    \begin{macrocode}
\def\bitsetIsDefined#1{%
  \BitSet@IfUndefined{#1}%
  \BitSet@SecondOfTwo
  \BitSet@FirstOfTwo
}
%    \end{macrocode}
%    \end{macro}
%    \begin{macro}{\bitsetIsEmpty}
%    \begin{macrocode}
\def\bitsetIsEmpty#1{%
  \BitSet@IfUndefined{#1}\BitSet@FirstOfTwo{%
    \expandafter\ifx\csname BS@#1\endcsname\BitSet@Zero
      \expandafter\BitSet@FirstOfTwo
    \else
      \expandafter\BitSet@SecondOfTwo
    \fi
  }%
}
%    \end{macrocode}
%    \end{macro}
%    \begin{macro}{\BitSet@Zero}
%    \begin{macrocode}
\def\BitSet@Zero{0}
%    \end{macrocode}
%    \end{macro}
%
%    \begin{macro}{\bitsetQuery}
%    \begin{macrocode}
\def\bitsetQuery#1#2{%
  \ifnum\bitsetGet{#1}{#2}=1 %
    \expandafter\BitSet@FirstOfTwo
  \else
    \expandafter\BitSet@SecondOfTwo
  \fi
}
%    \end{macrocode}
%    \end{macro}
%
%    \begin{macro}{\bitsetEquals}
%    \begin{macrocode}
\def\bitsetEquals#1#2{%
  \BitSet@IfUndefined{#1}{%
    \BitSet@IfUndefined{#2}\BitSet@FirstOfTwo\BitSet@SecondOfTwo
  }{%
    \BitSet@IfUndefined{#2}\BitSet@SecondOfTwo{%
      \expandafter\ifx\csname BS@#1\expandafter\endcsname
                      \csname BS@#2\endcsname
        \expandafter\BitSet@FirstOfTwo
      \else
        \expandafter\BitSet@SecondOfTwo
      \fi
    }%
  }%
}
%    \end{macrocode}
%    \end{macro}
%
%    \begin{macro}{\bitsetIntersects}
%    \begin{macrocode}
\def\bitsetIntersects#1#2{%
  \bitsetIsEmpty{#1}\BitSet@SecondOfTwo{%
    \bitsetIsEmpty{#2}\BitSet@SecondOfTwo{%
      \expandafter\expandafter\expandafter\BitSet@Intersects
      \csname BS@#1\expandafter\expandafter\expandafter\endcsname
      \expandafter\expandafter\expandafter!%
      \csname BS@#2\endcsname!%
    }%
  }%
}
%    \end{macrocode}
%    \end{macro}
%    \begin{macro}{\BitSet@Intersects}
%    \begin{macrocode}
\def\BitSet@Intersects#1#2!#3#4!{%
  \ifnum#1#3=11 %
    \BitSet@AfterFi\BitSet@FirstOfTwo
  \else
    \ifx\\#2\\%
      \BitSet@AfterFiFi\BitSet@SecondOfTwo
    \else
      \ifx\\#4\\%
        \BitSet@AfterFiFiFi\BitSet@SecondOfTwo
      \else
        \BitSet@AfterFiFiFi{%
          \BitSet@Intersects#2!#4!%
        }%
      \fi
    \fi
  \BitSet@Fi
}
%    \end{macrocode}
%    \end{macro}
%
%    \begin{macrocode}
\BitSet@AtEnd%
%</package>
%    \end{macrocode}
%
% \section{Test}
%
% \subsection{Catcode checks for loading}
%
%    \begin{macrocode}
%<*test1>
%    \end{macrocode}
%    \begin{macrocode}
\catcode`\{=1 %
\catcode`\}=2 %
\catcode`\#=6 %
\catcode`\@=11 %
\expandafter\ifx\csname count@\endcsname\relax
  \countdef\count@=255 %
\fi
\expandafter\ifx\csname @gobble\endcsname\relax
  \long\def\@gobble#1{}%
\fi
\expandafter\ifx\csname @firstofone\endcsname\relax
  \long\def\@firstofone#1{#1}%
\fi
\expandafter\ifx\csname loop\endcsname\relax
  \expandafter\@firstofone
\else
  \expandafter\@gobble
\fi
{%
  \def\loop#1\repeat{%
    \def\body{#1}%
    \iterate
  }%
  \def\iterate{%
    \body
      \let\next\iterate
    \else
      \let\next\relax
    \fi
    \next
  }%
  \let\repeat=\fi
}%
\def\RestoreCatcodes{}
\count@=0 %
\loop
  \edef\RestoreCatcodes{%
    \RestoreCatcodes
    \catcode\the\count@=\the\catcode\count@\relax
  }%
\ifnum\count@<255 %
  \advance\count@ 1 %
\repeat

\def\RangeCatcodeInvalid#1#2{%
  \count@=#1\relax
  \loop
    \catcode\count@=15 %
  \ifnum\count@<#2\relax
    \advance\count@ 1 %
  \repeat
}
\def\RangeCatcodeCheck#1#2#3{%
  \count@=#1\relax
  \loop
    \ifnum#3=\catcode\count@
    \else
      \errmessage{%
        Character \the\count@\space
        with wrong catcode \the\catcode\count@\space
        instead of \number#3%
      }%
    \fi
  \ifnum\count@<#2\relax
    \advance\count@ 1 %
  \repeat
}
\def\space{ }
\expandafter\ifx\csname LoadCommand\endcsname\relax
  \def\LoadCommand{\input bitset.sty\relax}%
\fi
\def\Test{%
  \RangeCatcodeInvalid{0}{47}%
  \RangeCatcodeInvalid{58}{64}%
  \RangeCatcodeInvalid{91}{96}%
  \RangeCatcodeInvalid{123}{255}%
  \catcode`\@=12 %
  \catcode`\\=0 %
  \catcode`\%=14 %
  \LoadCommand
  \RangeCatcodeCheck{0}{36}{15}%
  \RangeCatcodeCheck{37}{37}{14}%
  \RangeCatcodeCheck{38}{47}{15}%
  \RangeCatcodeCheck{48}{57}{12}%
  \RangeCatcodeCheck{58}{63}{15}%
  \RangeCatcodeCheck{64}{64}{12}%
  \RangeCatcodeCheck{65}{90}{11}%
  \RangeCatcodeCheck{91}{91}{15}%
  \RangeCatcodeCheck{92}{92}{0}%
  \RangeCatcodeCheck{93}{96}{15}%
  \RangeCatcodeCheck{97}{122}{11}%
  \RangeCatcodeCheck{123}{255}{15}%
  \RestoreCatcodes
}
\Test
\csname @@end\endcsname
\end
%    \end{macrocode}
%    \begin{macrocode}
%</test1>
%    \end{macrocode}
%
% \subsection{Macro tests}
%
% \subsubsection{Preamble}
%
%    \begin{macrocode}
%<*test2>
\NeedsTeXFormat{LaTeX2e}
\nofiles
\documentclass{article}
\makeatletter
%<*noetex>
\let\SavedNumexpr\numexpr
\let\SavedIfcsname\ifcsname
\let\SavedCurrentgrouplevel\currentgrouplevel
\def\ETeXDisable{%
  \let\ifcsname\@undefined
  \let\numexpr\@undefined
  \let\currentgrouplevel\@undefined
}
\ETeXDisable
%</noetex>
\makeatletter
\chardef\BitSet@TestMode=1 %
\makeatother
\usepackage{bitset}[2011/01/30]
%<*noetex>
\def\ETeXEnable{%
  \let\numexpr\SavedNumexpr
  \let\ifcsname\SavedIfcsname
  \let\currentgrouplevel\SavedCurrentgrouplevel
}
\ETeXEnable
%</noetex>
\usepackage{qstest}
\IncludeTests{*}
\LogTests{log}{*}{*}
\makeatletter
%    \end{macrocode}
%
% \subsubsection{Time}
%
%    \begin{macrocode}
\begingroup\expandafter\expandafter\expandafter\endgroup
\expandafter\ifx\csname pdfresettimer\endcsname\relax
\else
  \newcount\SummaryTime
  \newcount\TestTime
  \SummaryTime=\z@
  \newcommand*{\PrintTime}[2]{%
    \typeout{%
      [Time #1: \strip@pt\dimexpr\number#2sp\relax\space s]%
    }%
  }%
  \newcommand*{\StartTime}[1]{%
    \renewcommand*{\TimeDescription}{#1}%
    \pdfresettimer
  }%
  \newcommand*{\TimeDescription}{}%
  \newcommand*{\StopTime}{%
    \TestTime=\pdfelapsedtime
    \global\advance\SummaryTime\TestTime
    \PrintTime\TimeDescription\TestTime
  }%
  \let\saved@qstest\qstest
  \let\saved@endqstest\endqstest
  \def\qstest#1#2{%
    \saved@qstest{#1}{#2}%
    \StartTime{#1}%
  }%
  \def\endqstest{%
    \StopTime
    \saved@endqstest
  }%
  \AtEndDocument{%
    \PrintTime{summary}\SummaryTime
  }%
\fi
%    \end{macrocode}
%
% \subsubsection{Detection of unwanted space}
%
%    \begin{macrocode}
\let\orig@qstest\qstest
\let\orig@endqstest\endqstest
\def\qstest#1#2{%
  \orig@qstest{#1}{#2}%
  \setbox0\hbox\bgroup\begingroup\ignorespaces
}
\def\endqstest{%
  \endgroup\egroup
  \Expect*{\the\wd0}{0.0pt}%
  \orig@endqstest
}
%    \end{macrocode}
%
% \subsubsection{Test macros}
%
%    \begin{macrocode}
\newcounter{Test}

\def\TestError#1#2{%
  \begingroup
    \setcounter{Test}{0}%
    \sbox0{%
      \def\@PackageError##1##2##3{%
        \stepcounter{Test}%
        \begingroup
          \let\MessageBreak\relax
%<*noetex>
          \ETeXEnable
%</noetex>
          \Expect{##1}{bitset}%
          \Expect*{##2}*{#1}%
        \endgroup
      }%
%<*noetex>
      \ETeXDisable
%</noetex>
      #2%
    }%
    \Expect*{\theTest}{1}%
    \Expect*{\the\wd0}{0.0pt}%
  \endgroup
}

\def\TestErrorNegativeIndex#1#2{%
  \TestError{Invalid negative index (#1)}{#2}%
}

\def\TestGetterUndefined#1{%
  \CheckUndef{dummy}%
  \expandafter\expandafter\expandafter\Expect
  \expandafter\expandafter\expandafter{#1{dummy}}{0}%
}

\def\ExpectBitSet#1#2{%
  \expandafter\expandafter\expandafter\Expect
  \expandafter\expandafter\expandafter
  {\csname BS@#1\endcsname}*{#2}%
}
\def\Check#1#2{%
  \ExpectBitSet{#1}{#2}%
}
\def\CheckUndef#1{%
  \begingroup
    \Expect*{%
      \expandafter
      \ifx\csname BS@#1\endcsname\relax true\else false\fi
    }{true}%
  \endgroup
}
\def\RevCheck#1#2{%
  \ExpectBitSet{#1}{\Reverse#2!!}%
}
\def\Set#1#2{%
  \expandafter\def\csname BS@#1\endcsname{#2}%
}
\def\RevSet#1#2{%
  \expandafter\edef\csname BS@#1\endcsname{%
    \Reverse#2!!%
  }%
}
\def\Reverse#1#2!#3!{%
  \ifx\\#2\\%
    #1#3%
    \expandafter\@gobble
  \else
    \expandafter\@firstofone
  \fi
  {\Reverse#2!#1#3!}%
}
%    \end{macrocode}
%
% \subsubsection{Test sets}
%
%    \begin{macrocode}
\begin{qstest}{Let}{Let}
  \CheckUndef{abc}%
  \CheckUndef{xyz}%
  \bitsetLet{xyz}{abc}%
  \CheckUndef{abc}%
  \Check{xyz}{0}%
  \Set{abc}{1}%
  \Check{abc}{1}%
  \Check{xyz}{0}%
  \bitsetLet{xyz}{abc}%
  \Check{abc}{1}%
  \Check{xyz}{1}%
  \Set{xyz}{11}%
  \Check{abc}{1}%
  \Check{xyz}{11}%
\end{qstest}

\begin{qstest}{Reset}{Reset}
  \bitsetReset{xyz}%
  \Check{xyz}{0}%
  \bitsetReset{abc}%
  \Check{abc}{0}%
  \Set{abc}{10101}%
  \bitsetReset{abc}%
  \Check{abc}{0}%
\end{qstest}

\begin{qstest}{Get/Query}{Get/Query}
  \expandafter\expandafter\expandafter\Expect
  \expandafter\expandafter\expandafter{%
    \bitsetGet{dummy}{0}%
  }{0}%
  \begingroup
    \expandafter\def\csname BitSetError:NegativeIndex\endcsname{}%
    \Set{abc}{1}%
    \Expect*{\bitsetQuery{abc}{-1}{true}{false}}{false}%
  \endgroup
  \def\Test#1#2#3{%
    \Set{abc}{#1}%
    \expandafter\expandafter\expandafter\Expect
    \expandafter\expandafter\expandafter{\bitsetGet{abc}{#2}}{#3}%
    \Expect*{\bitsetQuery{abc}{#2}{true}{false}}%
           *{\ifcase#3 false\or true\else error\fi}%
  }%
  \Test{1}{100}{0}%
  \Test{0}{0}{0}%
  \Test{1}{0}{1}%
  \Test{11}{1}{1}%
  \Test{111}{1}{1}%
  \Test{101}{1}{0}%
  \Test{101}{2}{1}%
  \Test{10100110011}{10}{1}%
\end{qstest}

\begin{qstest}{Size}{Size}
  \TestGetterUndefined\bitsetSize
  \def\Test#1#2{%
    \Set{abc}{#1}%
    \expandafter\expandafter\expandafter\Expect
    \expandafter\expandafter\expandafter{\bitsetSize{abc}}{#2}%
  }%
  \Test{0}{0}%
  \Test{1}{1}%
  \Test{00}{0}%
  \Test{0000000}{0}%
  \Test{10}{1}%
  \Test{01}{2}%
  \Test{11}{2}%
  \Test{010}{2}%
  \Test{011}{3}%
  \Test{100110011}{9}%
  \Test{0000011111000001111100000}{20}%
  \Test{000000000000000000000000011111111111111111111}{45}%
\end{qstest}

\begin{qstest}{Cardinality}{Cardinality}
  \TestGetterUndefined\bitsetCardinality
  \def\Test#1#2{%
    \Set{abc}{#1}%
    \expandafter\expandafter\expandafter\Expect
    \expandafter\expandafter\expandafter{%
      \bitsetCardinality{abc}%
    }{#2}%
  }%
  \Test{0}{0}%
  \Test{1}{1}%
  \Test{00}{0}%
  \Test{0000000}{0}%
  \Test{10}{1}%
  \Test{01}{1}%
  \Test{11}{2}%
  \Test{010}{1}%
  \Test{011}{2}%
  \Test{100110011}{5}%
  \Test{0000011111000001111100000}{10}%
  \Test{000000000000000000000000011111111111111111111}{20}%
\end{qstest}

\begin{qstest}{NextClearBit/NextSetBit}{NextClearBit/NextSetBit}
  \def\Test#1#2{%
    \expandafter\expandafter\expandafter\Expect
    \expandafter\expandafter\expandafter{%
      \TestOp{abc}{#1}%
    }{#2}%
  }%
  \def\Clear{\let\TestOp\bitsetNextClearBit}%
  \def\Set{\let\TestOp\bitsetNextSetBit}%
  \begingroup
    \catcode`\:=11 %
    \bitsetSetBin{abc}{1}%
    \Clear
    \Test{-1}{1\BitSetError:NegativeIndex}%
    \Set
    \Test{-1}{0\BitSetError:NegativeIndex}%
  \endgroup
  \let\BS@abc\@undefined
  \Clear
  \Test{0}{0}%
  \Test{1}{1}%
  \Test{2}{2}%
  \Test{100}{100}%
  \Set
  \Test{0}{-1}%
  \Test{1}{-1}%
  \Test{100}{-1}%
  \bitsetReset{abc}%
  \Clear
  \Test{0}{0}%
  \Test{1}{1}%
  \Test{2}{2}%
  \Test{100}{100}%
  \Set
  \Test{0}{-1}%
  \Test{1}{-1}%
  \Test{100}{-1}%
  \bitsetSetBin{abc}{1}%
  \Clear
  \Test{0}{1}%
  \Test{1}{1}%
  \Test{2}{2}%
  \Test{100}{100}%
  \Set
  \Test{0}{0}%
  \Test{1}{-1}%
  \Test{100}{-1}%
  \bitsetSetBin{abc}{111000111000111000111}%
  \Clear
  \Test{0}{3}%
  \Test{1}{3}%
  \Test{2}{3}%
  \Test{3}{3}%
  \Test{4}{4}%
  \Test{5}{5}%
  \Test{6}{9}%
  \Test{7}{9}%
  \Test{8}{9}%
  \Test{9}{9}%
  \Test{10}{10}%
  \Test{11}{11}%
  \Test{12}{15}%
  \Test{13}{15}%
  \Test{14}{15}%
  \Test{15}{15}%
  \Test{16}{16}%
  \Test{17}{17}%
  \Test{18}{21}%
  \Test{19}{21}%
  \Test{20}{21}%
  \Test{21}{21}%
  \Test{22}{22}%
  \Test{100}{100}%
  \Set
  \Test{0}{0}%
  \Test{1}{1}%
  \Test{2}{2}%
  \Test{3}{6}%
  \Test{4}{6}%
  \Test{5}{6}%
  \Test{6}{6}%
  \Test{7}{7}%
  \Test{8}{8}%
  \Test{9}{12}%
  \Test{10}{12}%
  \Test{11}{12}%
  \Test{12}{12}%
  \Test{13}{13}%
  \Test{14}{14}%
  \Test{15}{18}%
  \Test{16}{18}%
  \Test{17}{18}%
  \Test{18}{18}%
  \Test{19}{19}%
  \Test{20}{20}%
  \Test{21}{-1}%
  \Test{22}{-1}%
  \Test{100}{-1}%
  \bitsetSetBin{abc}{1111111}%
  \Clear
  \Test{6}{7}%
  \Test{7}{7}%
  \Test{8}{8}%
  \Test{100}{100}%
  \Set
  \Test{6}{6}%
  \Test{7}{-1}%
  \Test{8}{-1}%
  \Test{100}{-1}%
  \bitsetSetBin{abc}{11111111}%
  \Clear
  \Test{7}{8}%
  \Test{8}{8}%
  \Test{9}{9}%
  \Test{100}{100}%
  \Set
  \Test{7}{7}%
  \Test{8}{-1}%
  \Test{9}{-1}%
  \Test{100}{-1}%
  \bitsetSetBin{abc}{111111111}%
  \Clear
  \Test{8}{9}%
  \Test{9}{9}%
  \Test{10}{10}%
  \Test{100}{100}%
  \Set
  \Test{8}{8}%
  \Test{9}{-1}%
  \Test{10}{-1}%
  \Test{100}{-1}%
  \bitsetSetBin{abc}{1111111111}%
  \Clear
  \Test{9}{10}%
  \Test{10}{10}%
  \Test{11}{11}%
  \Test{100}{100}%
  \Set
  \Test{9}{9}%
  \Test{10}{-1}%
  \Test{11}{-1}%
  \Test{100}{-1}%
\end{qstest}

\begin{qstest}{GetSetBitList}{GetSetBitList}
  \let\BS@abc\@undefined
  \expandafter\expandafter\expandafter\Expect
  \expandafter\expandafter\expandafter{%
    \bitsetGetSetBitList{abc}%
  }{}%
  \def\Test#1#2{%
    \bitsetSetBin{abc}{#1}%
    \expandafter\expandafter\expandafter\Expect
    \expandafter\expandafter\expandafter{%
      \bitsetGetSetBitList{abc}%
    }{#2}%
  }%
  \Test{0}{}%
  \Test{1}{0}%
  \Test{10}{1}%
  \Test{11}{0,1}%
  \Test{10110100}{2,4,5,7}%
  \Test{101101001010011}{0,1,4,6,9,11,12,14}%
\end{qstest}

\begin{qstest}{GetDec}{GetDec}
  \TestGetterUndefined\bitsetGetDec
  \def\Test#1#2{%
    \RevSet{abc}{#1}%
%<*noetex>
    \begingroup\expandafter\expandafter\expandafter\endgroup
%</noetex>
    \expandafter\expandafter\expandafter\Expect
    \expandafter\expandafter\expandafter{%
      \bitsetGetDec{abc}%
    }{#2}%
  }%
  \Test{0}{0}%
  \Test{1}{1}%
  \Test{10}{2}%
  \Test{11}{3}%
  \Test{100}{4}%
  \Test{101}{5}%
  \Test{110}{6}%
  \Test{111}{7}%
  \Test{1000}{8}%
  \Test{000111}{7}%
  \Test{1111111111111111%
        111111111111111}{2147483647}%
  \Test{0001111111111111111%
        111111111111111}{2147483647}%
  \Test{1000000000000000%
        0000000000000000}{2147483648}%
  \Test{1000000000000000%
        00000000000000000}{4294967296}%
  \Test{0001000000000000000%
        00000000000000000}{4294967296}%
  \Test{1100000000000000%
        00000000000000011}{6442450947}%
\end{qstest}

\begin{qstest}{Clear}{Clear}
  \def\Test#1#2#3{%
    \RevSet{abc}{#1}%
    \bitsetClear{abc}{#2}%
    \Expect*{\BS@abc}*{\Reverse#3!!}%
  }%
  \bitsetClear{abc}{2}%
  \RevCheck{abc}{0}%
  \TestErrorNegativeIndex{-1}{\bitsetClear{abc}{-1}}%
  \RevCheck{abc}{0}%
  \Test{0}{0}{0}%
  \Test{1}{0}{0}%
  \Test{111}{1}{101}%
  \Test{111}{30}{111}%
  \Test{0000111}{5}{0000111}% 111 would also be ok
  \Test{10000111}{5}{10000111}%
  \Test{1001001}{3}{1000001}%
\end{qstest}

\begin{qstest}{Set}{Set}
  \def\Test#1#2#3{%
    \RevSet{abc}{#1}%
    \bitsetSet{abc}{#2}%
    \Expect*{\BS@abc}*{\Reverse#3!!}%
  }%
  \bitsetSet{abc}{2}%
  \RevCheck{abc}{100}%
  \TestErrorNegativeIndex{-1}{\bitsetSet{abc}{-1}}%
  \RevCheck{abc}{100}%
  \Test{0}{0}{1}%
  \Test{1}{0}{1}%
  \Test{100}{1}{110}%
  \Test{111}{1}{111}%
  \Test{11}{1}{11}%
  \Test{11}{2}{111}%
  \Test{11}{3}{1011}%
  \Test{111}{10}{10000000111}%
  \Test{0000111}{5}{0100111}% 100111 would also be ok
  \Test{10000111}{5}{10100111}%
  \Test{1000001}{3}{1001001}%
  \Test{1001001}{3}{1001001}%
\end{qstest}

\begin{qstest}{Flip}{Flip}
  \def\Test#1#2#3{%
    \RevSet{abc}{#1}%
    \bitsetFlip{abc}{#2}%
    \Expect*{\BS@abc}*{\Reverse#3!!}%
  }%
  \bitsetFlip{abc}{2}%
  \RevCheck{abc}{100}%
  \TestErrorNegativeIndex{-1}{\bitsetFlip{abc}{-1}}%
  \RevCheck{abc}{100}%
  \Test{0}{0}{1}%
  \Test{1}{0}{0}%
  \Test{0}{2}{100}%
  \Test{100}{1}{110}%
  \Test{111}{1}{101}%
  \Test{11}{1}{1}%
  \Test{11}{2}{111}%
  \Test{11}{3}{1011}%
  \Test{111}{10}{10000000111}%
  \Test{0000111}{5}{0100111}% 100111 would also be ok
  \Test{10000111}{5}{10100111}%
  \Test{1000001}{3}{1001001}%
  \Test{1001001}{3}{1000001}%
  \Test{11111}{2}{11011}%
\end{qstest}

\begin{qstest}{SetValue}{SetValue}
  \def\Test#1#2{%
    \TestError{Invalid bit value (#2) not in range 0..1}{%
      \bitsetSetValue{abc}{#1}{#2}%
    }%
  }%
  \Test{0}{-1}%
  \Test{0}{2}%
  \Test{0}{10}%
  \def\Test#1#2#3{%
    \let\BS@abc\@undefined
    \bitsetSetValue{abc}{#1}{#2}%
    \bitsetSetBin{result}{#3}%
    \Expect*{\BS@abc}*{\BS@result}%
  }%
  \Test{0}{0}{0}%
  \Test{0}{1}{1}%
  \Test{1}{0}{0}%
  \Test{1}{1}{10}%
  \def\Test#1#2#3#4{%
    \bitsetSetBin{abc}{#1}%
    \bitsetSetBin{result}{#4}%
    \bitsetSetValue{abc}{#2}{#3}%
    \Expect*{\BS@abc}*{\BS@result}%
  }%
  \Test{0}{0}{0}{0}%
  \Test{0}{0}{0}{0}%
  \Test{0}{0}{1}{1}%
  \Test{0}{1}{0}{0}%
  \Test{0}{1}{1}{10}%
  \Test{1010}{2}{1}{1110}%
  \Test{1010}{4}{1}{11010}%
  \Test{1010}{6}{1}{1001010}%
  \Test{1010}{1}{0}{1000}%
  \Test{1010}{2}{0}{1010}%
  \Test{1010}{3}{0}{10}%
  \Test{1010}{4}{0}{1010}%
  \Test{1010}{6}{0}{1010}%
  \Test{1010}{2}{\csname iffalse\endcsname 0\else 1\fi}{1110}%
  \Test{1010}{1}{\csname iffalse\endcsname 1\else 0\fi}{1000}%
\end{qstest}

\begin{qstest}{IsDefined}{IsDefined}
  \let\BS@abc\@undefined
  \Expect*{\bitsetIsDefined{abc}{true}{false}}{false}%
  \bitsetReset{abc}%
  \Expect*{\bitsetIsDefined{abc}{true}{false}}{true}%
\end{qstest}

\begin{qstest}{IsEmpty}{IsEmpty}
  \let\BS@abc\@undefined
  \Expect*{\bitsetIsEmpty{abc}{true}{false}}{true}%
  \bitsetReset{abc}%
  \Expect*{\bitsetIsEmpty{abc}{true}{false}}{true}%
  \bitsetSet{abc}{1}%
  \Expect*{\bitsetIsEmpty{abc}{true}{false}}{false}%
\end{qstest}

\begin{qstest}{Equals}{Equals}
  \def\Test#1#2#3{%
    \Expect*{\bitsetEquals{#1}{#2}{true}{false}}{#3}%
  }%
  \let\BS@abc\@undefined
  \Test{abc}{abc}{true}%
  \Test{abc}{foo}{true}%
  \Test{foo}{abc}{true}%
  \bitsetReset{abc}%
  \Test{abc}{abc}{true}%
  \Test{abc}{foo}{false}%
  \Test{foo}{abc}{false}%
  \bitsetReset{foo}%
  \Test{abc}{foo}{true}%
  \Test{foo}{abc}{true}%
  \bitsetSet{abc}{4}%
  \Test{abc}{foo}{false}%
  \Test{foo}{abc}{false}%
  \bitsetFlip{foo}{4}%
  \Test{abc}{foo}{true}%
  \Test{foo}{abc}{true}%
\end{qstest}

\begin{qstest}{Intersects}{Intersects}
  \def\Test#1{%
    \Expect*{\bitsetIntersects{abc}{foo}{true}{false}}{#1}%
  }%
  \let\BS@abc\@undefined
  \let\BS@foo\@undefined
  \Test{false}%
  \Set{abc}{0}%
  \Test{false}%
  \Set{foo}{0}%
  \Test{false}%
  \let\BS@abc\@undefined
  \Test{false}%
  \Set{foo}{1}%
  \Test{false}%
  \Set{abc}{0}%
  \Test{false}%
  \Set{abc}{1}%
  \Test{true}%
  \let\BS@foo\@undefined
  \Test{false}%
  \Set{foo}{0}%
  \Test{false}%
  \def\Test#1#2#3{%
    \bitsetSetBin{abc}{#1}%
    \bitsetSetBin{foo}{#2}%
    \Expect*{\bitsetIntersects{abc}{foo}{true}{false}}{#3}%
  }%
  \Test{1010}{0101}{false}%
  \Test{0}{10}{false}%
  \Test{1}{11}{true}%
  \Test{11}{1}{true}%
  \Test{10}{1}{false}%
\end{qstest}

\begin{qstest}{And/AndNot/Or/Xor}{And/AndNot/Or/Xor}
  \def\@Test#1#2#3#4#5{%
    \begingroup
      #5%
      \begingroup
        \let\BS@foo\@undefined
        \csname bitset#1\endcsname{abc}{foo}%
        \CheckUndef{foo}%
        \Check{abc}{#2}%
      \endgroup
      \begingroup
        \bitsetReset{foo}%
        \csname bitset#1\endcsname{abc}{foo}%
        \Check{foo}{0}%
        \Check{abc}{#3}%
      \endgroup
      \begingroup
        \def\BS@foo{0101}%
        \csname bitset#1\endcsname{abc}{foo}%
        \Check{foo}{0101}%
        \Check{abc}{#4}%
      \endgroup
    \endgroup
  }%
  \def\Test#1{%
    \def\Op{#1}%
    \Test@
  }%
  \def\Test@#1#2#3#4#5#6#7#8#9{%
    \@Test\Op{#1}{#2}{#3}{%
      \let\BS@abc\@undefined
    }%
    \@Test\Op{#4}{#5}{#6}{%
      \bitsetReset{abc}%
    }%
    \@Test\Op{#7}{#8}{#9}{%
      \def\BS@abc{1001}%
    }%
  }%
  \Test{And}%
       {0}{0}{0}%
       {0}{0}{0}%
       {0}{0}{0001}%
  \Test{AndNot}%
       {0}{0}{0}%
       {0}{0}{0}%
       {1001}{1001}{1}%
  \Test{Or}%
       {0}{0}{0101}%
       {0}{0}{0101}%
       {1001}{1001}{1101}%
  \Test{Xor}%
       {0}{0}{0101}%
       {0}{0}{0101}%
       {1001}{1001}{11}%
  \def\Test#1#2#3{%
    \bitsetSetBin{abc}{#1}%
    \bitsetSetBin{foo}{#2}%
    \csname bitset\Op\endcsname{abc}{foo}%
    \RevCheck{foo}{#2}%
    \RevCheck{abc}{#3}%
  }%
  \def\Op{And}%
  \Test{1}{111}{1}%
  \Test{111}{1}{1}%
  \Test{10}{111}{10}%
  \Test{111}{10}{10}%
  \Test{111}{1000}{0}%
  \Test{1000}{111}{0}%
  \def\Op{AndNot}%
  \Test{1010}{11}{1000}%
  \Test{100}{100}{0}%
  \Test{111}{1111}{0}%
  \Test{100}{111}{0}%
  \def\Op{Or}%
  \Test{0}{0}{0}%
  \Test{1}{0}{1}%
  \Test{0}{1}{1}%
  \Test{1}{1}{1}%
  \Test{1000}{10}{1010}%
  \Test{10}{1000}{1010}%
  \def\Op{Xor}%
  \Test{0}{0}{0}%
  \Test{1}{0}{1}%
  \Test{0}{1}{1}%
  \Test{1}{1}{0}%
  \Test{1000}{10}{1010}%
  \Test{10}{1000}{1010}%
  \Test   {110011001100}%
       {111000111000111}%
       {111110100001011}%
  \Test{111000111000111}%
          {110011001100}%
       {111110100001011}%
\end{qstest}

\begin{qstest}{GetUndef}{GetUndef, GetBin, GetOct, GetHex}
  \def\TestUndef#1#2{%
    \let\BS@abc\@undefined
    \expandafter\expandafter\expandafter\Expect
    \expandafter\expandafter\expandafter{%
      \x{abc}{#1}%
    }{#2}%
  }%
  \let\x\bitsetGetBin
  \TestUndef{-1}{0}%
  \TestUndef{0}{0}%
  \TestUndef{1}{0}%
  \TestUndef{2}{00}%
  \TestUndef{8}{00000000}%
  \let\x\bitsetGetOct
  \TestUndef{-1}{0}%
  \TestUndef{0}{0}%
  \TestUndef{1}{0}%
  \TestUndef{2}{0}%
  \TestUndef{3}{0}%
  \TestUndef{4}{00}%
  \TestUndef{5}{00}%
  \TestUndef{6}{00}%
  \TestUndef{7}{000}%
  \TestUndef{8}{000}%
  \TestUndef{9}{000}%
  \TestUndef{10}{0000}%
  \let\x\bitsetGetHex
  \TestUndef{-1}{0}%
  \TestUndef{0}{0}%
  \TestUndef{1}{0}%
  \TestUndef{2}{0}%
  \TestUndef{3}{0}%
  \TestUndef{4}{0}%
  \TestUndef{5}{00}%
  \TestUndef{6}{00}%
  \TestUndef{7}{00}%
  \TestUndef{8}{00}%
  \TestUndef{9}{000}%
  \TestUndef{10}{000}%
  \TestUndef{12}{000}%
  \TestUndef{13}{0000}%
  \TestUndef{16}{0000}%
  \TestUndef{17}{00000}%
\end{qstest}

\begin{qstest}{SetBin}{SetBin}
  \def\Test#1#2{%
    \let\BS@abc\@undefined
    \bitsetSetBin{abc}{#1}%
    \expandafter\Expect\expandafter{\BS@abc}{#2}%
  }%
  \Test{}{0}%
  \Test{0}{0}%
  \Test{1}{1}%
  \Test{10}{01}%
  \Test{11}{11}%
  \Test{010}{01}%
  \Test{011}{11}%
  \Test{0010}{01}%
  \Test{1010}{0101}%
\end{qstest}

\begin{qstest}{SetOct}{SetOct}
  \def\Test#1#2{%
    \bitsetSetOct{abc}{#1}%
    \expandafter\Expect\expandafter{\BS@abc}{#2}%
  }%
  \Test{}{0}%
  \Test{0}{0}%
  \Test{000}{0}%
  \Test{1}{1}%
  \Test{001}{1}%
  \Test{010}{0001}%
  \Test{020}{00001}%
  \Test{42}{010001}%
  \Test{377}{11111111}%
  \Test{0377}{11111111}%
  \Test{76543210}{000100010110001101011111}%
  \Test{ 0 7 0 7 1 }{100111000111}%
\end{qstest}

\begin{qstest}{SetHex}{SetHex}
  \def\Test#1#2{%
    \bitsetSetHex{abc}{#1}%
    \expandafter\Expect\expandafter{\BS@abc}{#2}%
  }%
  \Test{}{0}%
  \Test{0}{0}%
  \Test{000}{0}%
  \Test{1}{1}%
  \Test{001}{1}%
  \Test{010}{00001}%
  \Test{020}{000001}%
  \Test{42}{0100001}%
  \Test{3F}{111111}%
  \Test{03F}{111111}%
  \Test{43210}{0000100001001100001}%
  \Test{98765}{10100110111000011001}%
  \Test{FEDCBA}{010111010011101101111111}%
  \Test{ 0 F 0 F 1 }{1000111100001111}%
\end{qstest}

\begin{qstest}{SetDec}{SetDec}
  \def\Test#1#2{%
    \bitsetSetDec{abc}{#1}%
    \expandafter\Expect\expandafter{\BS@abc}{#2}%
  }%
  \Test{}{0}%
  \Test{0}{0}%
  \Test{000}{0}%
  \Test{1}{1}%
  \Test{7}{111}%
  \Test{8}{0001}%
  \Test{001}{1}%
  \Test{010}{0101}%
  \Test{020}{00101}%
  \Test{53}{101011}%
  \Test{255}{11111111}%
  \Test{256}{000000001}%
  \Test{999999999}{111111111001001101011001110111}%
  \Test{1000000000}{000000000101001101011001110111}%
  \Test{4210987654}{01100001010010010111111101011111}%
  \Test{2147483647}{1111111111111111111111111111111}%
  \Test{2147483648}{00000000000000000000000000000001}%
\end{qstest}

\begin{qstest}{GetBin}{GetBin}
  \def\TestUndef#1#2{%
    \let\BS@abc\@undefined
    \expandafter\expandafter\expandafter\Expect
    \expandafter\expandafter\expandafter{%
      \bitsetGetBin{abc}{#1}%
    }{#2}%
  }%
  \TestUndef{-1}{0}%
  \TestUndef{0}{0}%
  \TestUndef{1}{0}%
  \TestUndef{2}{00}%
  \TestUndef{8}{00000000}%
  \def\Test#1#2{%
    \bitsetSetBin{abc}{#2}%
    \expandafter\expandafter\expandafter\Expect
    \expandafter\expandafter\expandafter{%
      \bitsetGetBin{abc}{#1}%
    }{#2}%
  }%
  \Test{-1}{0}%
  \Test{0}{0}%
  \Test{1}{0}%
  \Test{1}{1}%
  \Test{2}{01}%
  \Test{2}{10}%
  \Test{3}{010}%
  \Test{2}{00}%
  \Test{2}{01}%
  \Test{8}{00101100}%
  \Test{2}{10101}%
  \Test{-100}{11011}%
\end{qstest}

\begin{qstest}{GetOct}{GetOct}
  \def\Test#1#2#3{%
    \edef\x{\zap@space#1 \@empty}%
    \edef\x{\noexpand\bitsetSetBin{abc}{\x}}%
    \x
    \expandafter\expandafter\expandafter\Expect
    \expandafter\expandafter\expandafter{%
      \bitsetGetOct{abc}{#2}%
    }{#3}%
  }%
  \Test{111 110 101 100 011 010 001 000}{0}{76543210}%
  \Test{000 111}{0}{7}%
  \Test{101 000}{-1}{50}%
  \Test{111}{-1}{7}%
  \Test{111}{0}{7}%
  \Test{111}{1}{7}%
  \Test{111}{3}{7}%
  \Test{111}{4}{07}%
  \Test{111}{6}{07}%
  \Test{111}{7}{007}%
  \Test{111 010}{6}{72}%
  \Test{111 010}{7}{072}%
  \Test{011 111}{0}{37}%
  \Test{011 111}{6}{37}%
  \Test{011 111}{7}{037}%
  \Test{001 111}{0}{17}%
  \Test{001 111}{6}{17}%
  \Test{001 111}{7}{017}%
\end{qstest}

\begin{qstest}{GetHex}{GetHex}
  \def\Test#1#2#3{%
    \bitsetSetBin{abc}{#1}%
    \expandafter\expandafter\expandafter\Expect
    \expandafter\expandafter\expandafter{%
      \bitsetGetHex{abc}{#2}%
    }{#3}%
  }%
  \Test{1111 1110 1101 1100 1011 1010 1001 1000}{0}{FEDCBA98}%
  \Test{0111 0110 0101 0100 0011 0010 0001 0000}{0}{76543210}%
  \Test{0000 1111}{0}{F}%
  \Test{0101 0000}{-1}{50}%
  \Test{1111}{-1}{F}%
  \Test{1111}{0}{F}%
  \Test{1111}{1}{F}%
  \Test{1111}{4}{F}%
  \Test{1111}{5}{0F}%
  \Test{1111}{8}{0F}%
  \Test{1111}{9}{00F}%
  \Test{1111 0010}{8}{F2}%
  \Test{1111 0010}{9}{0F2}%
  \Test{0111 1111}{0}{7F}%
  \Test{0111 1111}{8}{7F}%
  \Test{0111 1111}{9}{07F}%
  \Test{0011 1111}{0}{3F}%
  \Test{0011 1111}{8}{3F}%
  \Test{0011 1111}{9}{03F}%
  \Test{0001 1111}{0}{1F}%
  \Test{0001 1111}{8}{1F}%
  \Test{0001 1111}{9}{01F}%
\end{qstest}

\begin{qstest}{Range}{Range}
  \TestError{%
    Wrong index numbers in range [9..8]\MessageBreak% hash-ok
    for clear/set/flip on bit set `abc'.\MessageBreak
    The lower index exceeds the upper index.\MessageBreak
    Canceling the operation as error recovery%
  }{%
    \bitsetSetRange{abc}{9}{8}%
  }%
  \def\TestErrorNegInd#1#2#3#4#5#6{%
    \TestError{%
      Negative index in range [#2..#3]\MessageBreak % hash-ok
      for \string\bitset #1Range on bit set `abc'.\MessageBreak
      Using [#4..#5] as error recovery% hash-ok
    }{%
      \csname bitset#1Range\endcsname{abc}{#2}{#3}%
      \global\let\BS@global\BS@abc
    }%
    \Check{global}{#6}%
  }%
  \Set{abc}{111}%
  \TestErrorNegInd{Clear}{-1}{0}{0}{0}{111}%
  \TestErrorNegInd{Clear}{0}{-1}{0}{0}{111}%
  \TestErrorNegInd{Clear}{-2}{2}{0}{2}{001}%
  \bitsetReset{abc}%
  \TestErrorNegInd{Set}{-1}{0}{0}{0}{0}%
  \TestErrorNegInd{Set}{0}{-1}{0}{0}{0}%
  \TestErrorNegInd{Set}{-2}{2}{0}{2}{11}%
  \Set{abc}{101}%
  \TestErrorNegInd{Flip}{-1}{0}{0}{0}{101}%
  \TestErrorNegInd{Flip}{0}{-1}{0}{0}{101}%
  \TestErrorNegInd{Flip}{-2}{2}{0}{2}{011}%
  \def\Test#1#2#3#4{%
    \bitsetSetBin{abc}{#1}%
    \csname bitset\TestOp Range\endcsname{abc}{#2}{#3}%
    \Expect*{\bitsetGetBin{abc}{0}}{#4}%
  }%
  \def\TestOp{Clear}%
  \Test{0}{0}{1}{0}%
  \Test{1111}{1}{2}{1101}%
  \Test{1111}{1}{3}{1001}%
  \Test{1111111100000000}{12}{14}{1100111100000000}%
  \def\TestOp{Set}%
  \Test{0}{0}{1}{1}%
  \Test{1000}{1}{2}{1010}%
  \Test{0}{1}{2}{10}%
  \Test{1}{12}{15}{111000000000001}%
  \Test{1111}{1}{3}{1111}%
  \Test{1000000000000000}{12}{14}{1011000000000000}%
  \def\TestOp{Flip}%
  \Test{0}{0}{1}{1}%
  \Test{1}{0}{1}{0}%
  \Test{10101010}{1}{5}{10110100}%
  \def\Test#1#2#3#4#5{%
    \bitsetSetBin{abc}{#1}%
    \bitsetSetValueRange{abc}{#2}{#3}{#4}%
    \Expect*{\bitsetGetBin{abc}{0}}{#5}%
  }%
  \Test{0}{0}{1}{0}{0}%
  \Test{0}{0}{1}{1}{1}%
  \Test{1010}{1}{3}{0}{1000}%
  \Test{1010}{1}{3}{1}{1110}%
\end{qstest}

\begin{qstest}{ShiftLeft/ShiftRight}{ShiftLeft/ShiftRight}
  \def\@Test#1#2{%
    \let\BS@abc\@undefined
    \csname bitsetShift#1\endcsname{abc}{#2}%
    \Expect*{\BS@abc}{0}%
  }%
  \def\Test#1{%
    \@Test{Left}{#1}%
    \@Test{Right}{#1}%
  }%
  \Test{-16}%
  \Test{-1}%
  \Test{0}%
  \Test{1}%
  \Test{16}%
  \def\Test#1#2#3{%
    \bitsetSetBin{abc}{#1}%
    \bitsetSetBin{result}{#3}%
    \csname bitsetShift\Op\endcsname{abc}{#2}%
    \Expect*{\bitsetGetBin{abc}{0}}*{\bitsetGetBin{result}{0}}%
  }%
  \def\Op{Left}%
  \Test{0}{0}{0}%
  \Test{0}{1}{0}%
  \Test{0}{-1}{0}%
  \Test{1}{0}{1}%
  \Test{1}{1}{10}%
  \Test{1}{-1}{0}%
  \Test{10}{1}{100}%
  \Test{10}{-1}{1}%
  \Test{1}{32}{100000000000000000000000000000000}%
  \Test{1}{-100}{0}%
  \def\Op{Right}%
  \Test{0}{0}{0}%
  \Test{0}{1}{0}%
  \Test{0}{-1}{0}%
  \Test{1}{0}{1}%
  \Test{1}{1}{0}%
  \Test{1}{-1}{10}%
  \Test{10}{1}{1}%
  \Test{10}{-1}{100}%
  \Test{1}{-32}{100000000000000000000000000000000}%
  \Test{1}{100}{0}%
  \Test{110110110110110}{10}{11011}%
  \Test{110110110110110}{100}{0}%
  \Test{1}{100000}{0}%
\end{qstest}

\begin{qstest}{Profile: Set}{Profile: Set}
  \bitsetSet{abc}{4095}%
  \global\let\BS@global\BS@abc
\end{qstest}

\begin{qstest}{Profile: Get}{Profile: Get}
  \edef\x{\bitsetGet{global}{4095}}%
\end{qstest}

\begin{document}
\end{document}
%</test2>
%    \end{macrocode}
%
% \section{Installation}
%
% \subsection{Download}
%
% \paragraph{Package.} This package is available on
% CTAN\footnote{\url{ftp://ftp.ctan.org/tex-archive/}}:
% \begin{description}
% \item[\CTAN{macros/latex/contrib/oberdiek/bitset.dtx}] The source file.
% \item[\CTAN{macros/latex/contrib/oberdiek/bitset.pdf}] Documentation.
% \end{description}
%
%
% \paragraph{Bundle.} All the packages of the bundle `oberdiek'
% are also available in a TDS compliant ZIP archive. There
% the packages are already unpacked and the documentation files
% are generated. The files and directories obey the TDS standard.
% \begin{description}
% \item[\CTAN{install/macros/latex/contrib/oberdiek.tds.zip}]
% \end{description}
% \emph{TDS} refers to the standard ``A Directory Structure
% for \TeX\ Files'' (\CTAN{tds/tds.pdf}). Directories
% with \xfile{texmf} in their name are usually organized this way.
%
% \subsection{Bundle installation}
%
% \paragraph{Unpacking.} Unpack the \xfile{oberdiek.tds.zip} in the
% TDS tree (also known as \xfile{texmf} tree) of your choice.
% Example (linux):
% \begin{quote}
%   |unzip oberdiek.tds.zip -d ~/texmf|
% \end{quote}
%
% \paragraph{Script installation.}
% Check the directory \xfile{TDS:scripts/oberdiek/} for
% scripts that need further installation steps.
% Package \xpackage{attachfile2} comes with the Perl script
% \xfile{pdfatfi.pl} that should be installed in such a way
% that it can be called as \texttt{pdfatfi}.
% Example (linux):
% \begin{quote}
%   |chmod +x scripts/oberdiek/pdfatfi.pl|\\
%   |cp scripts/oberdiek/pdfatfi.pl /usr/local/bin/|
% \end{quote}
%
% \subsection{Package installation}
%
% \paragraph{Unpacking.} The \xfile{.dtx} file is a self-extracting
% \docstrip\ archive. The files are extracted by running the
% \xfile{.dtx} through \plainTeX:
% \begin{quote}
%   \verb|tex bitset.dtx|
% \end{quote}
%
% \paragraph{TDS.} Now the different files must be moved into
% the different directories in your installation TDS tree
% (also known as \xfile{texmf} tree):
% \begin{quote}
% \def\t{^^A
% \begin{tabular}{@{}>{\ttfamily}l@{ $\rightarrow$ }>{\ttfamily}l@{}}
%   bitset.sty & tex/generic/oberdiek/bitset.sty\\
%   bitset.pdf & doc/latex/oberdiek/bitset.pdf\\
%   test/bitset-test1.tex & doc/latex/oberdiek/test/bitset-test1.tex\\
%   test/bitset-test2.tex & doc/latex/oberdiek/test/bitset-test2.tex\\
%   test/bitset-test3.tex & doc/latex/oberdiek/test/bitset-test3.tex\\
%   bitset.dtx & source/latex/oberdiek/bitset.dtx\\
% \end{tabular}^^A
% }^^A
% \sbox0{\t}^^A
% \ifdim\wd0>\linewidth
%   \begingroup
%     \advance\linewidth by\leftmargin
%     \advance\linewidth by\rightmargin
%   \edef\x{\endgroup
%     \def\noexpand\lw{\the\linewidth}^^A
%   }\x
%   \def\lwbox{^^A
%     \leavevmode
%     \hbox to \linewidth{^^A
%       \kern-\leftmargin\relax
%       \hss
%       \usebox0
%       \hss
%       \kern-\rightmargin\relax
%     }^^A
%   }^^A
%   \ifdim\wd0>\lw
%     \sbox0{\small\t}^^A
%     \ifdim\wd0>\linewidth
%       \ifdim\wd0>\lw
%         \sbox0{\footnotesize\t}^^A
%         \ifdim\wd0>\linewidth
%           \ifdim\wd0>\lw
%             \sbox0{\scriptsize\t}^^A
%             \ifdim\wd0>\linewidth
%               \ifdim\wd0>\lw
%                 \sbox0{\tiny\t}^^A
%                 \ifdim\wd0>\linewidth
%                   \lwbox
%                 \else
%                   \usebox0
%                 \fi
%               \else
%                 \lwbox
%               \fi
%             \else
%               \usebox0
%             \fi
%           \else
%             \lwbox
%           \fi
%         \else
%           \usebox0
%         \fi
%       \else
%         \lwbox
%       \fi
%     \else
%       \usebox0
%     \fi
%   \else
%     \lwbox
%   \fi
% \else
%   \usebox0
% \fi
% \end{quote}
% If you have a \xfile{docstrip.cfg} that configures and enables \docstrip's
% TDS installing feature, then some files can already be in the right
% place, see the documentation of \docstrip.
%
% \subsection{Refresh file name databases}
%
% If your \TeX~distribution
% (\teTeX, \mikTeX, \dots) relies on file name databases, you must refresh
% these. For example, \teTeX\ users run \verb|texhash| or
% \verb|mktexlsr|.
%
% \subsection{Some details for the interested}
%
% \paragraph{Attached source.}
%
% The PDF documentation on CTAN also includes the
% \xfile{.dtx} source file. It can be extracted by
% AcrobatReader 6 or higher. Another option is \textsf{pdftk},
% e.g. unpack the file into the current directory:
% \begin{quote}
%   \verb|pdftk bitset.pdf unpack_files output .|
% \end{quote}
%
% \paragraph{Unpacking with \LaTeX.}
% The \xfile{.dtx} chooses its action depending on the format:
% \begin{description}
% \item[\plainTeX:] Run \docstrip\ and extract the files.
% \item[\LaTeX:] Generate the documentation.
% \end{description}
% If you insist on using \LaTeX\ for \docstrip\ (really,
% \docstrip\ does not need \LaTeX), then inform the autodetect routine
% about your intention:
% \begin{quote}
%   \verb|latex \let\install=y\input{bitset.dtx}|
% \end{quote}
% Do not forget to quote the argument according to the demands
% of your shell.
%
% \paragraph{Generating the documentation.}
% You can use both the \xfile{.dtx} or the \xfile{.drv} to generate
% the documentation. The process can be configured by the
% configuration file \xfile{ltxdoc.cfg}. For instance, put this
% line into this file, if you want to have A4 as paper format:
% \begin{quote}
%   \verb|\PassOptionsToClass{a4paper}{article}|
% \end{quote}
% An example follows how to generate the
% documentation with pdf\LaTeX:
% \begin{quote}
%\begin{verbatim}
%pdflatex bitset.dtx
%makeindex -s gind.ist bitset.idx
%pdflatex bitset.dtx
%makeindex -s gind.ist bitset.idx
%pdflatex bitset.dtx
%\end{verbatim}
% \end{quote}
%
% \section{Catalogue}
%
% The following XML file can be used as source for the
% \href{http://mirror.ctan.org/help/Catalogue/catalogue.html}{\TeX\ Catalogue}.
% The elements \texttt{caption} and \texttt{description} are imported
% from the original XML file from the Catalogue.
% The name of the XML file in the Catalogue is \xfile{bitset.xml}.
%    \begin{macrocode}
%<*catalogue>
<?xml version='1.0' encoding='us-ascii'?>
<!DOCTYPE entry SYSTEM 'catalogue.dtd'>
<entry datestamp='$Date$' modifier='$Author$' id='bitset'>
  <name>bitset</name>
  <caption>Handle bit-vector datatype.</caption>
  <authorref id='auth:oberdiek'/>
  <copyright owner='Heiko Oberdiek' year='2007,2011'/>
  <license type='lppl1.3'/>
  <version number='1.1'/>
  <description>
    This package defines and implements the data type bit set,
    a vector of bits. The size of the vector may grow dynamically.
    Individual bits can be manipulated.
    <p/>
    The package is part of the <xref refid='oberdiek'>oberdiek</xref> bundle.
  </description>
  <documentation details='Package documentation'
      href='ctan:/macros/latex/contrib/oberdiek/bitset.pdf'/>
  <ctan file='true' path='/macros/latex/contrib/oberdiek/bitset.dtx'/>
  <miktex location='oberdiek'/>
  <texlive location='oberdiek'/>
  <install path='/macros/latex/contrib/oberdiek/oberdiek.tds.zip'/>
</entry>
%</catalogue>
%    \end{macrocode}
%
% \begin{History}
%   \begin{Version}{2007/09/28 v1.0}
%   \item
%     First version.
%   \end{Version}
%   \begin{Version}{2011/01/30 v1.1}
%   \item
%     Already loaded package files are not input in \hologo{plainTeX}.
%   \end{Version}
% \end{History}
%
% \PrintIndex
%
% \Finale
\endinput

%        (quote the arguments according to the demands of your shell)
%
% Documentation:
%    (a) If bitset.drv is present:
%           latex bitset.drv
%    (b) Without bitset.drv:
%           latex bitset.dtx; ...
%    The class ltxdoc loads the configuration file ltxdoc.cfg
%    if available. Here you can specify further options, e.g.
%    use A4 as paper format:
%       \PassOptionsToClass{a4paper}{article}
%
%    Programm calls to get the documentation (example):
%       pdflatex bitset.dtx
%       makeindex -s gind.ist bitset.idx
%       pdflatex bitset.dtx
%       makeindex -s gind.ist bitset.idx
%       pdflatex bitset.dtx
%
% Installation:
%    TDS:tex/generic/oberdiek/bitset.sty
%    TDS:doc/latex/oberdiek/bitset.pdf
%    TDS:doc/latex/oberdiek/test/bitset-test1.tex
%    TDS:doc/latex/oberdiek/test/bitset-test2.tex
%    TDS:doc/latex/oberdiek/test/bitset-test3.tex
%    TDS:source/latex/oberdiek/bitset.dtx
%
%<*ignore>
\begingroup
  \catcode123=1 %
  \catcode125=2 %
  \def\x{LaTeX2e}%
\expandafter\endgroup
\ifcase 0\ifx\install y1\fi\expandafter
         \ifx\csname processbatchFile\endcsname\relax\else1\fi
         \ifx\fmtname\x\else 1\fi\relax
\else\csname fi\endcsname
%</ignore>
%<*install>
\input docstrip.tex
\Msg{************************************************************************}
\Msg{* Installation}
\Msg{* Package: bitset 2011/01/30 v1.1 Handle bit-vector datatype (HO)}
\Msg{************************************************************************}

\keepsilent
\askforoverwritefalse

\let\MetaPrefix\relax
\preamble

This is a generated file.

Project: bitset
Version: 2011/01/30 v1.1

Copyright (C) 2007, 2011 by
   Heiko Oberdiek <heiko.oberdiek at googlemail.com>

This work may be distributed and/or modified under the
conditions of the LaTeX Project Public License, either
version 1.3c of this license or (at your option) any later
version. This version of this license is in
   http://www.latex-project.org/lppl/lppl-1-3c.txt
and the latest version of this license is in
   http://www.latex-project.org/lppl.txt
and version 1.3 or later is part of all distributions of
LaTeX version 2005/12/01 or later.

This work has the LPPL maintenance status "maintained".

This Current Maintainer of this work is Heiko Oberdiek.

The Base Interpreter refers to any `TeX-Format',
because some files are installed in TDS:tex/generic//.

This work consists of the main source file bitset.dtx
and the derived files
   bitset.sty, bitset.pdf, bitset.ins, bitset.drv, bitset-test1.tex,
   bitset-test2.tex, bitset-test3.tex.

\endpreamble
\let\MetaPrefix\DoubleperCent

\generate{%
  \file{bitset.ins}{\from{bitset.dtx}{install}}%
  \file{bitset.drv}{\from{bitset.dtx}{driver}}%
  \usedir{tex/generic/oberdiek}%
  \file{bitset.sty}{\from{bitset.dtx}{package}}%
  \usedir{doc/latex/oberdiek/test}%
  \file{bitset-test1.tex}{\from{bitset.dtx}{test1}}%
  \file{bitset-test2.tex}{\from{bitset.dtx}{test2,etex}}%
  \file{bitset-test3.tex}{\from{bitset.dtx}{test2,noetex}}%
  \nopreamble
  \nopostamble
  \usedir{source/latex/oberdiek/catalogue}%
  \file{bitset.xml}{\from{bitset.dtx}{catalogue}}%
}

\catcode32=13\relax% active space
\let =\space%
\Msg{************************************************************************}
\Msg{*}
\Msg{* To finish the installation you have to move the following}
\Msg{* file into a directory searched by TeX:}
\Msg{*}
\Msg{*     bitset.sty}
\Msg{*}
\Msg{* To produce the documentation run the file `bitset.drv'}
\Msg{* through LaTeX.}
\Msg{*}
\Msg{* Happy TeXing!}
\Msg{*}
\Msg{************************************************************************}

\endbatchfile
%</install>
%<*ignore>
\fi
%</ignore>
%<*driver>
\NeedsTeXFormat{LaTeX2e}
\ProvidesFile{bitset.drv}%
  [2011/01/30 v1.1 Handle bit-vector datatype (HO)]%
\documentclass{ltxdoc}
\usepackage{holtxdoc}[2011/11/22]
\usepackage{amstext}
\usepackage{ifluatex}
\ifluatex
\else
  \usepackage[T1]{fontenc}%
  \usepackage{lmodern}%
\fi
\begin{document}
  \DocInput{bitset.dtx}%
\end{document}
%</driver>
% \fi
%
% \CheckSum{3510}
%
% \CharacterTable
%  {Upper-case    \A\B\C\D\E\F\G\H\I\J\K\L\M\N\O\P\Q\R\S\T\U\V\W\X\Y\Z
%   Lower-case    \a\b\c\d\e\f\g\h\i\j\k\l\m\n\o\p\q\r\s\t\u\v\w\x\y\z
%   Digits        \0\1\2\3\4\5\6\7\8\9
%   Exclamation   \!     Double quote  \"     Hash (number) \#
%   Dollar        \$     Percent       \%     Ampersand     \&
%   Acute accent  \'     Left paren    \(     Right paren   \)
%   Asterisk      \*     Plus          \+     Comma         \,
%   Minus         \-     Point         \.     Solidus       \/
%   Colon         \:     Semicolon     \;     Less than     \<
%   Equals        \=     Greater than  \>     Question mark \?
%   Commercial at \@     Left bracket  \[     Backslash     \\
%   Right bracket \]     Circumflex    \^     Underscore    \_
%   Grave accent  \`     Left brace    \{     Vertical bar  \|
%   Right brace   \}     Tilde         \~}
%
% \GetFileInfo{bitset.drv}
%
% \title{The \xpackage{bitset} package}
% \date{2011/01/30 v1.1}
% \author{Heiko Oberdiek\\\xemail{heiko.oberdiek at googlemail.com}}
%
% \maketitle
%
% \begin{abstract}
% This package defines and implements the data type bit set,
% a vector of bits. The size of the vector may grow dynamically.
% Individual bits can be manipulated.
% \end{abstract}
%
% \tableofcontents
%
% \clearpage
% \section{Documentation}
%
% \subsection{Introduction}
%
% Annotations in the PDF format know entries whose values are
% integers. This numbers are interpreted as set of flags
% specifying properties. For example, annotation dictionaries
% can have a key |/F|. The bits of its integer value
% are interpreted the following way:
% \begin{quote}
% \begin{tabular}{@{}l|l@{}}
%   Bit position & Property name\\
%   \hline
%   1 & Invisible\\
%   2 & Hidden\\
%   3 & Print\\
%   4 & NoZoom\\
%   5 & NoRotate\\
%   6 & NoView\\
%   7 & ReadOnly\\
%   \dots & \dots
% \end{tabular}
% \end{quote}
% Now, let's see how these values are set in package \xpackage{hyperref}
% before it uses this package (before v6.77a):
%\begin{quote}
%\begin{verbatim}
%\ifFld@hidden /F 6\else /F 4\fi
%\end{verbatim}
%\end{quote}
% Where are the other flags? The following example for key |/Ff|
% in a widget annotation
% supports at least three properties:
%\begin{quote}
%\begin{verbatim}
%\ifFld@multiline
%  \ifFld@readonly /Ff 4097\else /Ff 4096\fi
%\else
%  \ifFld@password
%    \ifFld@readonly /Ff 8193\else /Ff 8192\fi
%  \else
%    \ifFld@readonly /Ff 1\fi
%  \fi
%\fi
%\end{verbatim}
%\end{quote}
% But you see the point. It would be a nightmare to continue
% this way in supporting the missing flag settings. This kind
% of integers may have up to 32 bits.
%
% Therefore I wanted a data structure for setting and
% clearing individual bits. Also it should provide
% an export as decimal number. The snipsets above are
% executed in expansion contexts without \TeX's stomach
% commands. It would be convenient to have an expandable
% conversion from the data structure to the integer that
% gets written to the PDF file.
%
% This package \xpackage{bitset} implements such a data
% structure. The interface is quite close to Java's
% class |BitSet| in order not to learn to many interfaces
% for the same kind of data structure.
%
% \subsection{Glossary}
%
% \begin{description}
% \item[Bit set:]
%   A bit set is a vector of bits or flags.
%   The vector size is unlimited and
%   grows dynamically. An undefined bit set is treated as bit
%   set where all bits are cleared.
%
%   Bit sets are addressed by name. A name should consists of
%   letters or digits. Technically it must survive \cs{csname},
%   see \LaTeX's environment names for other names with such
%   a constraint. Package \xpackage{babel}'s shorthands are
%   not supported due to technical reasons. Shorthand support
%   would break expandable operations.
% \item[Size:]
%   A size of a bit set is the number of bits in use.
%   It's the number of the highest index, incremented by one.
%   Sizes are in the range |0| up to |2147483647|, the highest
%   number supported by \TeX.
% \item[Index:]
%   Bit positions in a bit set are addressed by an index number.
%   The bit vector is zero based. The first and least significant
%   bit is addressed by index |0| and the highest possible bit
%   by |2147483646|.
% \item[Bit:]
%   A bit is enoded as |0| for cleared/disabled or |1| for set/enabled.
% \end{description}
%
% \subsection{Design principles}
%
% \begin{description}
% \item[Name conventions:]
%   To avoid conflicts with existing macro names, the operations
%   are prefixed by the package name.
% \item[Zero based indexes:] The first bit is addressed by zero.
%   (Convention of array indexing in C, Java, \dots)
% \item[Unlimited size:]
%   There is no restriction on the size of a bit set other than
%   usual memory limitations. \cs{bitsetSetDec} and \cs{bitsetGetDec}
%   transparently switch to package \xpackage{bigintcalc} if the numbers
%   get too large for \TeX's number limit.
% \item[Expandibility:]
%   Any operation that does not change the bit set is expandable.
%   And all operations that extract or calculate some result
%   do this in exact two expansion steps. For example,
%   a macro \cs{Macro} wants a bit set as decimal number.
%   But the argument must be a plain number without macros.
%   Thus you could prefix \cs{bitsetGetDec} with \cs{number}.
%   However this won't work for bit sets with 31 or more bits
%   because of \TeX's number limit of $2^{31}-1$.
%   then just hit the operator with two \cs{expandafter}:
%   \begin{quote}
%   |\expandafter\expandafter\expandafter|\\
%   |\Macro\bitsetGetDec{foo}|
%   \end{quote}
%   |\bitsetGetDec| is hit first by the third \cs{expandafter}
%   and then by the second one.
% \item[Format independence:] This package is written as \LaTeX\ package,
%   but it does not depend on \LaTeX. It will also work for other formats
%   such as \plainTeX.
% \item[Independence from \TeX\ engines:]
%   Vanilla \TeX\ is all you need. Calculations are delegated to
%   packages \xpackage{intcalc} and \xpackage{bigintcalc}.
%   They don't need any special features, but they will switch
%   to a little more efficient implementation if features such
%   as \cs{numexpr} are available.
% \item[Numeric arguments:] Anything that is accepted by
%   \cs{number}. If \eTeX\ is detected, also expressions
%   for \cs{numexpr} are supported. The only exception so far
%   is the number for \cs{bitsetSetDec}. The number might be too
%   large for \cs{number} or \cs{numexpr}.
% \item[Error messages:]
%   In expandable contexts, only a limited set of \TeX\ primitive
%   commands work as expected. So called stomach commands behave
%   like \cs{relax} and don't get expanded or executed.
%   Unhappily also the error commands belong to this category.
%   The expandable operations will throw an unknown control
%   sequence instead to get \TeX's and user's attention.
%   The name of these control sequences starts with
%   \cs{BitSetError:} with the type of error after the colon.
% \end{description}
%
% \subsection{Operator overview}
%
% \begin{minipage}{\linewidth}
% \begin{description}
% \def\It#1#2{^^A
%   \item[^^A
%     {#1} ^^A
%     \textmd{(\hyperref[{sec:#2}]{section \ref*{sec:#2}})}^^A
%   ]^^A
%   \mbox{}\\^^A
% }
% \def\op#1{^^A
%   \cs{bitset}\texttt{\bfseries#1}^^A
% }
% \def\M#1{\meta{#1}}
% \def\Q{\hspace*{\fill}}
% \It{Miscellaneous}{misc}
%   \op{Reset}\Q \M{BitSet}\\
%   \op{Let}\Q \M{BitSet A} \M{BitSet B}
% \It{Import}{import}
%   \op{SetBin}, \op{SetOct}, \op{SetHex}\Q \M{BitSet} \M{Value}\\
%   \op{SetDec}\Q \M{BitSet} \M{Value}
% \It{Export\textsuperscript{a}}{export}
%   \op{GetBin}, \op{GetOct}, \op{GetHex}\Q \M{BitSet} \M{MinSize}\\
%   \op{GetDec}\Q \M{BitSet}
% \It{Logical operators}{log}
%   \op{And}, \op{AndNot}\Q \M{BitSet A} \M{BitSet B}\\
%   \op{Or}, \op{Xor}\Q \M{BitSet A} \M{BitSet B}
% \It{Shifting}{shift}
%   \op{ShiftLeft}, \op{ShiftRight}\Q \M{BitSet} \M{ShiftAmount}\\
% \It{Bit manipulation}{bitman}
%   \op{Clear}, \op{Set}, \op{Flip}\Q \M{BitSet} \M{Index}\\
%   \op{SetValue}\Q \M{BitSet} \M{Index} \M{Value}\\
%   \op{ClearRange}, \op{SetRange}, \op{FlipRange}\\
%     \Q \M{BitSet} \M{IndexFrom} \M{IndexTo}\\
%   \op{SetValueRange}\Q \M{BitSet} \M{IndexFrom} \M{IndexTo}
% \It{Bit retrieval\textsuperscript{a}}{bitret}
%   \op{Get}\Q \M{BitSet} \M{Index}\\
%   \op{NextClearBit}, \op{NextSetBit}\Q \M{BitSet} \M{Index}\\
%   \op{GetSetBitList}\Q \M{BitSet}
% \It{Bit set properties}{prop}
%   \op{Size}, \op{Cardinality}\Q \M{BitSet}
% \It{Queries\textsuperscript{b}}{query}
%   \op{IsDefined}, \op{IsEmpty}\Q \M{BitSet} \M{Then} \M{Else}\\
%   \op{Equals}, \op{Intersects}\Q \M{BitSet A} \M{BitSet B} \M{Then} \M{Else}\\
%   \op{Query}\Q \M{BitSet} \M{Index} \M{Then} \M{Else}
% \end{description}
% \footnotetext[1]{Macros are expandable, full expansion by two steps.}
% \footnotetext[2]{Macros are expandable.}
% \end{minipage}
%
% \subsection{Package loading}
%
% The package can be used as normal \LaTeX\ package:
%\begin{quote}
%|\usepackage{bitset}|
%\end{quote}
% Also \plainTeX\ is supported:
%\begin{quote}
%|\input bitset.sty\relax|
%\end{quote}
%
% \subsection{Operators}
%
% The following macros work on and with bit sets.
% A bit set \meta{BitSet} is represented by a name.
% The should consist of letters and digits. Technically
% it must survive \cs{csname}. It is the same constraint
% that must be satisfied by label or environment names
% in \LaTeX.
%
% However active characters that are shorthands
% of package \xpackage{babel} are not supported.
% Support for shorthands works by an assignment.
% But many operators
% such as \cs{bitsetGetDec} must be usable in expandable
% contexts. There assignments will not be executed in the
% best case or they will cause errors.
%
% The bits in a bit set are addressed by non-negative
% integers starting from zero. Thus negative index
% numbers cause an error message. Because index numbers
% are \TeX\ numbers. The largest index is 2147483647.
% But in practice memory limits and patience limits
% will be very likely reached much before.
%
% \subsubsection{Miscellaneous}\label{sec:misc}
%
% There isn't a separate operation for bit set creation.
% For simplicity an undefined bit set is treated as bit set
% with all bits cleared.
%
% \begin{declcs}{bitsetReset} \M{BitSet}
% \end{declcs}
% Macro \cs{bitsetReset} clears all bits. The result is an
% empty bit set. It may also be used as replacement for
% an operation ``new'', because an undefined bit set is defined
% afterwards.
%
% \begin{declcs}{bitsetLet} \M{BitSet A} \M{BitSet B}
% \end{declcs}
% Macro \cs{bitsetLet} performs a simple assignment similar
% to \TeX's \cs{let}. After the operation \meta{BitSet A} has
% the same value as \meta{BitSet B}. If \meta{BitSet B} is
% undefined, then \meta{BitSet A} will be the empty bit set.
%
% Note: If \meta{BitSet A} exists, it will be overwritten.
%
% \subsubsection{Import}\label{sec:import}
%
% \begin{declcs}{bitsetSetBin} \M{BitSet} \M{BinaryNumber}\\
%   \cs{bitsetSetOct} \M{BitSet} \M{OctalNumber}\\
%   \cs{bitsetSetHex} \M{BitSet} \M{HexadecimalNumber}
% \end{declcs}
% The numbers are interpreted as bit vectors and the flags in
% the bit \meta{BitSet} set are set accordingly.
% These numeric arguments are the only arguments where
% spaces are allowed. Then the numbers are easier to read.
%
% \begin{declcs}{bitsetSetDec} \M{BitSet} \M{DecimalNumber}
% \end{declcs}
% Macro \cs{bitsetSetDec} uses \meta{DecimalNumber} to
% set the bit set \meta{BitSet}.
% The numeric argument must expand
% to a plain number consisting of decimal digits without
% command tokens or spaces. Internally this argument is
% expanded only. It cannot be passed to \cs{number} or
% \cs{numexpr}, because the number may be too large for them.
% However \cs{number} or |\the\numexpr| may be used
% explicitely. This also helps for unexpandable number
% command tokens or registers (\cs{z@}, \cs{@ne}, \cs{count@}, \dots).
% Also \LaTeX'\ \cs{value} needs prefixing:
% \begin{quote}
% |\bitsetSetDec{foo}{\number\value{bar}}|
% \end{quote}
%
% \subsubsection{Export}\label{sec:export}
%
% \begin{declcs}{bitsetGetBin} \M{BitSet} \M{MinSize}\\
%   \cs{bitsetGetOct} \M{BitSet} \M{MinSize}\\
%   \cs{bitsetGetHex} \M{BitSet} \M{MinSize}
% \end{declcs}
% These macros returns the bit set as binary, octal or hexadecimal
% number. If the bit size is smaller than \meta{MinSize} the gap is
% filled with leading zeros.
% Example:
% \begin{quote}
% |\bitsetReset{abc}|\\
% |\bitsetSet{abc}{2}|\\
% |\bitsetGetBin{abc}{8}| $\rightarrow$ |00000100|\\
% |\bitsetSet{abc}{5}\bitsetSet{abc}{7}|\\
% |\bitsetGetHex{abc}{16}| $\rightarrow$ |00A2|
% \end{quote}
% Macro \cs{bitsetGetHex} uses the uppercase letters |A| to |F|.
% The catcode of the letters is one of 11 (letter) or 12 (other).
%
% \begin{declcs}{bitsetGetDec} \M{BitSet}
% \end{declcs}
% Macro \cs{bitsetGetDec} returns the bit set \meta{BitSet} as
% decimal number. The returned number can be larger than
% \TeX's number limit of $2^{31}-1$.
%
% \subsubsection{Logical operators}\label{sec:log}
%
% \begin{declcs}{bitsetAnd} \M{BitSet A} \M{BitSet B}
% \end{declcs}
% $A_{\text{new}} \mathrel{\mathop{:}}= A_{\text{old}}
% \mathbin{\text{and}} B \qquad (\forall\, \text{bits})$
%
% \begin{declcs}{bitsetAndNot} \M{BitSet A} \M{BitSet B}
% \end{declcs}
% $A_{\text{new}} \mathrel{\mathop{:}}= A_{\text{old}}
% \mathbin{\text{and}} (\text{not}\,B) \qquad (\forall\, \text{bits})$
%
% \begin{declcs}{bitsetOr} \M{BitSet A} \M{BitSet B}
% \end{declcs}
% $A_{\text{new}} \mathrel{\mathop{:}}= A_{\text{old}}
% \mathbin{\text{or}} B \qquad (\forall\, \text{bits})$
%
% \begin{declcs}{bitsetXor} \M{BitSet A} \M{BitSet B}
% \end{declcs}
% $A_{\text{new}} \mathrel{\mathop{:}}= A_{\text{old}}
% \mathbin{\text{xor}} B \qquad (\forall\, \text{bits})$
%
% \subsubsection{Shifting}\label{sec:shift}
%
% \begin{declcs}{bitsetShiftLeft} \M{BitSet} \M{ShiftAmount}\\
%   \cs{bitsetShiftRight} \M{BitSet} \M{ShiftAmount}
% \end{declcs}
% A left shift by one is a multiplication by two, thus left shifting
% moves the flags to higher positions. The new created low positions
% are filled by zeros.
%
% A right shift is the opposite, dividing by two, movint the bits
% to lower positions. The number will become smaller, the lowest
% bits are lost.
%
% If the \meta{ShiftAmount} is negative, it reverts the meaning of
% the shift operation. A left shift becomes a right shift.
% A \meta{ShiftAmount} of zero is ignored.
%
% \subsubsection{Bit manipulation}\label{sec:bitman}
%
% \begin{declcs}{bitsetClear} \M{BitSet} \M{Index}\\
%   \cs{bitsetSet} \M{BitSet} \M{Index}\\
%   \cs{bitsetFlip} \M{BitSet} \M{Index}
% \end{declcs}
% This macros manipulate a single bit in \meta{BitSet} addressed
% by \cs{Index}. Macro \cs{bitsetClear} disables the bit,
% \cs{bitsetSet} enables it and \cs{bitsetFlip} reverts the current
% setting of the bit.
%
% \begin{declcs}{bitsetSetValue} \M{BitSet} \M{Index} \M{Bit}
% \end{declcs}
% Macro \cs{bitsetSetValue} puts bit \meta{Bit} at position \meta{Index}
% in bit set \meta{BitSet}. \meta{Bit} must be a valid \TeX\ number
% equals to zero (disabled/cleared) or one (enabled/set).
%
% \subsubsection{Bit retrieval}\label{sec:bitret}
%
% \begin{declcs}{bitsetGet} \M{BitSet} \M{Index}
% \end{declcs}
% Macro \cs{bitsetGet} extracts the status of the bit at
% position \meta{Index} in bit set \meta{BitSet}.
% Digit |1| is returned if the bit is set/enabled. If the
% bit is cleared/disabled and in cases of an undefined bitset or
% an index number out of range the return value is |0|.
%
% \begin{declcs}{bitsetNextClearBit} \M{BitSet} \M{Index}
% \end{declcs}
% Starting at position \meta{Index} (inclusive) the bits
% are inspected. The first position without a set bit is returned.
% Possible results are decimal numbers: \meta{Index}, \meta{Index} + 1,
% \dots, ($\infty$)
%
% \begin{declcs}{bitsetNextSetBit} \M{BitSet} \M{Index}
% \end{declcs}
% Starting at position \meta{Index} (inclusive) the bits are
% inspected and the index position of the first found set bit
% is returned. If there isn't such a bit, then the result is
% |-1|. In summary possible results are decimal numbers:
% |-1|, \meta{Index}, \meta{Index} + 1, \dots, ($\infty$)
%
% \begin{declcs}{bitsetGetSetBitList} \M{BitSet}
% \end{declcs}
% Macro \cs{bitsetGetSetBitList} is an application for
% \cs{bitsetNextSetBit}. The set bits are iterated and
% returned as comma separated list of index positions
% in increasing order. The list is empty in case of
% an empty bit set.
%
% \subsubsection{Bit set properties}\label{sec:prop}
%
% \begin{declcs}{bitsetSize} \M{BitSet}
% \end{declcs}
% Macro \cs{bitsetSize} returns number of bits in use.
% It is the same as the index number of the highest set/enabled bit
% incremented by one.
%
% \begin{declcs}{bitsetCardinality} \M{BitSet}
% \end{declcs}
% Macro \cs{bitsetCardinality} counts the number of set/enabled bits.
%
% \subsubsection{Queries}\label{sec:query}
%
% Also the query procedures are expandable. They ask for a piece
% of information about a bit set and execute code depending
% on the answer.
%
% \begin{declcs}{bitsetIsDefined} \M{BitSet} \M{Then} \M{Else}
% \end{declcs}
% If the bit set with the name \meta{BitSet} exists the code
% given in \meta{Then} is executed, otherwise \meta{Else} is used.
%
% \begin{declcs}{bitsetIsEmpty} \M{BitSet} \M{Then} \M{Else}
% \end{declcs}
% If the bit set \meta{BitSet} exists and at least one bit
% is set/enabled, the code in \meta{Then} is executed,
% \meta{Else} otherwise.
%
% \begin{declcs}{bitsetEquals} \M{BitSet A} \M{BitSet B} \M{Then} \M{Else}
% \end{declcs}
% Both bit sets are equal if and only if either both are undefined
% or both are defined and represents the same bit values at the
% same positions. Thus this definition is reflexive, symmeric,
% and transitive, enough for an equivalent relation.
%
% \begin{declcs}{bitsetIntersects} \M{BitSet A} \M{BitSet B} \M{Then} \M{Else}
% \end{declcs}
% If and only if \meta{BitSet A} and \meta{BitSet B} have
% at least one bit at the same position that is set, then
% code part \meta{Then} is executed.
%
% \begin{declcs}{bitsetQuery} \M{BitSet} \M{Index} \M{Then} \M{Else}
% \end{declcs}
% It's just a wrapper for \cs{bitsetGet}. If the bit at position
% \meta{Index} is enabled, code \meta{Then} is called.
%
% \StopEventually{
% }
%
% \section{Implementation}
%
% The internal format of a bit set is quite simple, a sequence
% of digits |0| and |1|. The least significant bit is left.
% A bit set without any flag set is encoded by |0|. Also
% undefined bit sets are treated that way. After the
% highest bit that is set there are no further zeroes.
% A regular expression of valid bit sets values:
% \begin{quote}
%  |0|\string||[01]*1|
% \end{quote}
%
%    \begin{macrocode}
%<*package>
%    \end{macrocode}
%
% \subsection{Reload check and package identification}
%    Reload check, especially if the package is not used with \LaTeX.
%    \begin{macrocode}
\begingroup\catcode61\catcode48\catcode32=10\relax%
  \catcode13=5 % ^^M
  \endlinechar=13 %
  \catcode35=6 % #
  \catcode39=12 % '
  \catcode44=12 % ,
  \catcode45=12 % -
  \catcode46=12 % .
  \catcode58=12 % :
  \catcode64=11 % @
  \catcode123=1 % {
  \catcode125=2 % }
  \expandafter\let\expandafter\x\csname ver@bitset.sty\endcsname
  \ifx\x\relax % plain-TeX, first loading
  \else
    \def\empty{}%
    \ifx\x\empty % LaTeX, first loading,
      % variable is initialized, but \ProvidesPackage not yet seen
    \else
      \expandafter\ifx\csname PackageInfo\endcsname\relax
        \def\x#1#2{%
          \immediate\write-1{Package #1 Info: #2.}%
        }%
      \else
        \def\x#1#2{\PackageInfo{#1}{#2, stopped}}%
      \fi
      \x{bitset}{The package is already loaded}%
      \aftergroup\endinput
    \fi
  \fi
\endgroup%
%    \end{macrocode}
%    Package identification:
%    \begin{macrocode}
\begingroup\catcode61\catcode48\catcode32=10\relax%
  \catcode13=5 % ^^M
  \endlinechar=13 %
  \catcode35=6 % #
  \catcode39=12 % '
  \catcode40=12 % (
  \catcode41=12 % )
  \catcode44=12 % ,
  \catcode45=12 % -
  \catcode46=12 % .
  \catcode47=12 % /
  \catcode58=12 % :
  \catcode64=11 % @
  \catcode91=12 % [
  \catcode93=12 % ]
  \catcode123=1 % {
  \catcode125=2 % }
  \expandafter\ifx\csname ProvidesPackage\endcsname\relax
    \def\x#1#2#3[#4]{\endgroup
      \immediate\write-1{Package: #3 #4}%
      \xdef#1{#4}%
    }%
  \else
    \def\x#1#2[#3]{\endgroup
      #2[{#3}]%
      \ifx#1\@undefined
        \xdef#1{#3}%
      \fi
      \ifx#1\relax
        \xdef#1{#3}%
      \fi
    }%
  \fi
\expandafter\x\csname ver@bitset.sty\endcsname
\ProvidesPackage{bitset}%
  [2011/01/30 v1.1 Handle bit-vector datatype (HO)]%
%    \end{macrocode}
%
% \subsection{Catcodes}
%
%    \begin{macrocode}
\begingroup\catcode61\catcode48\catcode32=10\relax%
  \catcode13=5 % ^^M
  \endlinechar=13 %
  \catcode123=1 % {
  \catcode125=2 % }
  \catcode64=11 % @
  \def\x{\endgroup
    \expandafter\edef\csname BitSet@AtEnd\endcsname{%
      \endlinechar=\the\endlinechar\relax
      \catcode13=\the\catcode13\relax
      \catcode32=\the\catcode32\relax
      \catcode35=\the\catcode35\relax
      \catcode61=\the\catcode61\relax
      \catcode64=\the\catcode64\relax
      \catcode123=\the\catcode123\relax
      \catcode125=\the\catcode125\relax
    }%
  }%
\x\catcode61\catcode48\catcode32=10\relax%
\catcode13=5 % ^^M
\endlinechar=13 %
\catcode35=6 % #
\catcode64=11 % @
\catcode123=1 % {
\catcode125=2 % }
\def\TMP@EnsureCode#1#2{%
  \edef\BitSet@AtEnd{%
    \BitSet@AtEnd
    \catcode#1=\the\catcode#1\relax
  }%
  \catcode#1=#2\relax
}
\TMP@EnsureCode{33}{12}% !
\TMP@EnsureCode{39}{12}% '
\TMP@EnsureCode{40}{12}% (
\TMP@EnsureCode{41}{12}% )
\TMP@EnsureCode{42}{12}% *
\TMP@EnsureCode{43}{12}% +
\TMP@EnsureCode{44}{12}% ,
\TMP@EnsureCode{45}{12}% -
\TMP@EnsureCode{46}{12}% .
\TMP@EnsureCode{47}{12}% /
\TMP@EnsureCode{58}{11}% : (letter!)
\TMP@EnsureCode{60}{12}% <
\TMP@EnsureCode{62}{12}% >
\TMP@EnsureCode{63}{14}% ? (comment!)
\TMP@EnsureCode{91}{12}% [
\TMP@EnsureCode{93}{12}% ]
\TMP@EnsureCode{96}{12}% `
\edef\BitSet@AtEnd{\BitSet@AtEnd\noexpand\endinput}
\begingroup\expandafter\expandafter\expandafter\endgroup
\expandafter\ifx\csname BitSet@TestMode\endcsname\relax
\else
  \catcode63=9 % ? (ignore)
\fi
? \let\BitSet@@TestMode\BitSet@TestMode
%    \end{macrocode}
%
% \subsection{Package loading}
%
%    \begin{macrocode}
\begingroup\expandafter\expandafter\expandafter\endgroup
\expandafter\ifx\csname RequirePackage\endcsname\relax
  \def\TMP@RequirePackage#1[#2]{%
    \begingroup\expandafter\expandafter\expandafter\endgroup
    \expandafter\ifx\csname ver@#1.sty\endcsname\relax
      \input #1.sty\relax
    \fi
  }%
  \TMP@RequirePackage{infwarerr}[2007/09/09]%
  \TMP@RequirePackage{intcalc}[2007/09/27]%
  \TMP@RequirePackage{bigintcalc}[2007/09/27]%
\else
  \RequirePackage{infwarerr}[2007/09/09]%
  \RequirePackage{intcalc}[2007/09/27]%
  \RequirePackage{bigintcalc}[2007/09/27]%
\fi
%    \end{macrocode}
%
% \subsection{Help macros}
%
% \subsubsection{Number constant}
%
%    \begin{macro}{\BitSet@MaxSize}
%    \begin{macrocode}
\def\BitSet@MaxSize{2147483647}%
%    \end{macrocode}
%    \end{macro}
%
% \subsubsection{General basic macros}
%
%    \begin{macro}{\BitSet@Empty}
%    \begin{macrocode}
\def\BitSet@Empty{}
%    \end{macrocode}
%    \end{macro}
%    \begin{macro}{\BitSet@FirstOfOne}
%    \begin{macrocode}
\def\BitSet@FirstOfOne#1{#1}
%    \end{macrocode}
%    \end{macro}
%    \begin{macro}{\BitSet@Gobble}
%    \begin{macrocode}
\def\BitSet@Gobble#1{}
%    \end{macrocode}
%    \end{macro}
%    \begin{macro}{\BitSet@FirstOfTwo}
%    \begin{macrocode}
\def\BitSet@FirstOfTwo#1#2{#1}
%    \end{macrocode}
%    \end{macro}
%    \begin{macro}{\BitSet@SecondOfTwo}
%    \begin{macrocode}
\def\BitSet@SecondOfTwo#1#2{#2}
%    \end{macrocode}
%    \end{macro}
%
%    \begin{macro}{\BitSet@Space}
%    \begin{macrocode}
\def\BitSet@Space{ }
%    \end{macrocode}
%    \end{macro}
%    \begin{macro}{\BitSet@ZapSpace}
%    \begin{macrocode}
\def\BitSet@ZapSpace#1 #2{%
  #1%
  \ifx\BitSet@Empty#2%
  \else
    \expandafter\BitSet@ZapSpace
  \fi
  #2%
}
%    \end{macrocode}
%    \end{macro}
%
% \subsubsection{Tail recursion}
%
%    \begin{macro}{\BitSet@Fi}
%    \begin{macrocode}
\let\BitSet@Fi\fi
%    \end{macrocode}
%    \end{macro}
%    \begin{macro}{\BitSet@AfterFi}
%    \begin{macrocode}
\def\BitSet@AfterFi#1#2\BitSet@Fi{\fi#1}
%    \end{macrocode}
%    \end{macro}
%    \begin{macro}{\BitSet@AfterFiFi}
%    \begin{macrocode}
\def\BitSet@AfterFiFi#1#2\BitSet@Fi{\fi\fi#1}%
%    \end{macrocode}
%    \end{macro}
%    \begin{macro}{\BitSet@AfterFiFiFi}
%    \begin{macrocode}
\def\BitSet@AfterFiFiFi#1#2\BitSet@Fi{\fi\fi\fi#1}%
%    \end{macrocode}
%    \end{macro}
%
% \subsubsection{Check macros}
%
%    \begin{macro}{\BitSet@IfUndefined}
%    \begin{macrocode}
\def\BitSet@IfUndefined#1{%
  \expandafter\ifx\csname BS@#1\endcsname\relax
    \expandafter\BitSet@FirstOfTwo
  \else
    \expandafter\BitSet@SecondOfTwo
  \fi
}
%    \end{macrocode}
%    \end{macro}
%
%    \begin{macro}{\BitSet@CheckIndex}
%    |#1|: continuation code\\
%    |#2|: BitSet\\
%    |#3|: Index
%    \begin{macrocode}
\def\BitSet@CheckIndex#1#2#3{%
  \BitSet@IfUndefined{#2}{\bitsetReset{#2}}{}%
  \expandafter\expandafter\expandafter\BitSet@@CheckIndex
  \intcalcNum{#3}!%
  {#2}{#1}%
}
%    \end{macrocode}
%    \end{macro}
%    \begin{macro}{\BitSet@@CheckIndex}
%    |#1|: plain Index\\
%    |#2|: BitSet\\
%    |#3|: continuation code
%    \begin{macrocode}
\def\BitSet@@CheckIndex#1!#2#3{%
  \ifnum#1<0 %
    \BitSet@AfterFi{%
      \@PackageError{bitset}{%
        Invalid negative index (#1)%
      }\@ehc
    }%
  \else
    \BitSet@AfterFi{%
      #3{#2}{#1}%
    }%
  \BitSet@Fi
}
%    \end{macrocode}
%    \end{macro}
%
% \subsection{Miscellaneous}\label{src:misc}
%
%    \begin{macro}{\bitsetReset}
%    \begin{macrocode}
\def\bitsetReset#1{%
  \expandafter\def\csname BS@#1\endcsname{0}%
}
%    \end{macrocode}
%    \end{macro}
%
%    \begin{macro}{\bitsetLet}
%    \begin{macrocode}
\def\bitsetLet#1#2{%
  \BitSet@IfUndefined{#2}{%
    \bitsetReset{#1}%
  }{%
    \expandafter\let\csname BS@#1\expandafter\endcsname
                    \csname BS@#2\endcsname
  }%
}
%    \end{macrocode}
%    \end{macro}
%
% \subsection{Import}\label{src:import}
%
% \subsubsection{From binary number}
%
%    \begin{macro}{\bitsetSetBin}
%    \begin{macrocode}
\def\bitsetSetBin#1#2{%
  \edef\BitSet@Temp{#2}%
  \edef\BitSet@Temp{%
    \expandafter\expandafter\expandafter\BitSet@ZapSpace
    \expandafter\BitSet@Temp\BitSet@Space\BitSet@Empty
  }%
  \edef\BitSet@Temp{%
    \expandafter\BitSet@KillZeros\BitSet@Temp\BitSet@Empty
  }%
  \ifx\BitSet@Temp\BitSet@Empty
    \expandafter\let\csname BS@#1\endcsname\BitSet@Zero
  \else
    \expandafter\edef\csname BS@#1\endcsname{%
      \expandafter\BitSet@Reverse\BitSet@Temp!%
    }%
  \fi
}
%    \end{macrocode}
%    \end{macro}
%    \begin{macro}{\BitSet@KillZeros}
%    \begin{macrocode}
\def\BitSet@KillZeros#1{%
  \ifx#10%
    \expandafter\BitSet@KillZeros
  \else
    #1%
  \fi
}
%    \end{macrocode}
%    \end{macro}
%    \begin{macro}{\BitSet@Reverse}
%    \begin{macrocode}
\def\BitSet@Reverse#1#2!{%
  \ifx\\#2\\%
    #1%
  \else
    \BitSet@AfterFi{%
      \BitSet@Reverse#2!#1%
    }%
  \BitSet@Fi
}
%    \end{macrocode}
%    \end{macro}
%
% \subsubsection{From octal/hex number}
%
%    \begin{macro}{\bitsetSetOct}
%    \begin{macrocode}
\def\bitsetSetOct{%
  \BitSet@SetOctHex\BitSet@FromFirstOct
}
%    \end{macrocode}
%    \end{macro}
%    \begin{macro}{\bitsetSetHex}
%    \begin{macrocode}
\def\bitsetSetHex{%
  \BitSet@SetOctHex\BitSet@FromFirstHex
}
%    \end{macrocode}
%    \end{macro}
%    \begin{macro}{\BitSet@SetOctHex}
%    \begin{macrocode}
\def\BitSet@SetOctHex#1#2#3{%
  \edef\BitSet@Temp{#3}%
  \edef\BitSet@Temp{%
    \expandafter\expandafter\expandafter\BitSet@ZapSpace
    \expandafter\BitSet@Temp\BitSet@Space\BitSet@Empty
  }%
  \edef\BitSet@Temp{%
    \expandafter\BitSet@KillZeros\BitSet@Temp\BitSet@Empty
  }%
  \ifx\BitSet@Temp\BitSet@Empty
    \expandafter\let\csname BS@#2\endcsname\BitSet@Zero
  \else
    \edef\BitSet@Temp{%
      \expandafter#1\BitSet@Temp!%
    }%
    \ifx\BitSet@Temp\BitSet@Empty
      \expandafter\let\csname BS@#2\endcsname\BitSet@Zero
    \else
      \expandafter\edef\csname BS@#2\endcsname{%
        \expandafter\BitSet@Reverse\BitSet@Temp!%
      }%
    \fi
  \fi
}
%    \end{macrocode}
%    \end{macro}
%    \begin{macro}{\BitSet@FromFirstOct}
%    \begin{macrocode}
\def\BitSet@FromFirstOct#1{%
  \ifx#1!%
  \else
    \ifcase#1 \BitSet@AfterFiFi\BitSet@FromFirstOct
    \or 1%
    \or 10%
    \or 11%
    \or 100%
    \or 101%
    \or 110%
    \or 111%
    \else \BitSetError:WrongOctalDigit%
    \fi
    \expandafter\BitSet@FromOct
  \BitSet@Fi
}
%    \end{macrocode}
%    \end{macro}
%    \begin{macro}{\BitSet@FromOct}
%    \begin{macrocode}
\def\BitSet@FromOct#1{%
  \ifx#1!%
  \else
    \ifcase#1 000%
    \or 001%
    \or 010%
    \or 011%
    \or 100%
    \or 101%
    \or 110%
    \or 111%
    \else \BitSetError:WrongOctalDigit%
    \fi
    \expandafter\BitSet@FromOct
  \fi
}
%    \end{macrocode}
%    \end{macro}
%
%    \begin{macro}{\BitSet@FromFirstHex}
%    \begin{macrocode}
\def\BitSet@FromFirstHex#1{%
  \ifx#1!%
  \else
    \ifx#10%
      \BitSet@AfterFiFi\BitSet@FromFirstHex
    \fi
    \expandafter\ifx\csname BitSet@Hex#1\endcsname\relax
      \BitSetError:InvalidHexDigit%
    \else
      \expandafter\expandafter\expandafter\BitSet@KillZeros
      \csname BitSet@Hex#1\endcsname
    \fi
    \expandafter\BitSet@FromHex
  \BitSet@Fi
}
%    \end{macrocode}
%    \end{macro}
%    \begin{macro}{\BitSet@FromHex}
%    \begin{macrocode}
\def\BitSet@FromHex#1{%
  \ifx#1!%
  \else
    \expandafter\ifx\csname BitSet@Hex#1\endcsname\relax
      \BitSetError:InvalidHexDigit%
    \else
      \csname BitSet@Hex#1\endcsname
    \fi
    \expandafter\BitSet@FromHex
  \fi
}
%    \end{macrocode}
%    \end{macro}
%    \begin{macro}{\BitSet@Hex[0..F]}
%    \begin{macrocode}
\def\BitSet@Temp#1{%
  \expandafter\def\csname BitSet@Hex#1\endcsname
}
\BitSet@Temp 0{0000}%
\BitSet@Temp 1{0001}%
\BitSet@Temp 2{0010}%
\BitSet@Temp 3{0011}%
\BitSet@Temp 4{0100}%
\BitSet@Temp 5{0101}%
\BitSet@Temp 6{0110}%
\BitSet@Temp 7{0111}%
\BitSet@Temp 8{1000}%
\BitSet@Temp 9{1001}%
\BitSet@Temp A{1010}%
\BitSet@Temp B{1011}%
\BitSet@Temp C{1100}%
\BitSet@Temp D{1101}%
\BitSet@Temp E{1110}%
\BitSet@Temp F{1111}%
\BitSet@Temp a{1010}%
\BitSet@Temp b{1011}%
\BitSet@Temp c{1100}%
\BitSet@Temp d{1101}%
\BitSet@Temp e{1110}%
\BitSet@Temp f{1111}%
%    \end{macrocode}
%    \end{macro}
%
% \subsubsection{From decimal number}
%
%    \begin{macro}{\bitsetSetDec}
%    \begin{macrocode}
\def\bitsetSetDec#1#2{%
  \edef\BitSet@Temp{#2}%
  \edef\BitSet@Temp{%
    \expandafter\expandafter\expandafter\BitSet@ZapSpace
    \expandafter\BitSet@Temp\BitSet@Space\BitSet@Empty
  }%
  \edef\BitSet@Temp{%
    \expandafter\BitSet@KillZeros\BitSet@Temp\BitSet@Empty
  }%
  \ifx\BitSet@Temp\BitSet@Empty
    \expandafter\let\csname BS@#1\endcsname\BitSet@Zero
  \else
    \ifcase\bigintcalcSgn{\BitSet@Temp} %
      \expandafter\let\csname BS@#1\endcsname\BitSet@Zero
    \or
      \ifnum\bigintcalcCmp\BitSet@Temp\BitSet@MaxSize>0 %
        \expandafter\edef\csname BS@#1\endcsname{%
          \expandafter\BitSet@SetDecBig\BitSet@Temp!%
        }%
      \else
        \expandafter\edef\csname BS@#1\endcsname{%
          \expandafter\BitSet@SetDec\BitSet@Temp!%
        }%
      \fi
    \else
      \@PackageError{bitset}{%
        Bit sets cannot be negative%
      }\@ehc
    \fi
  \fi
}
%    \end{macrocode}
%    \end{macro}
%    \begin{macro}{\BitSet@SetDecBig}
%    \begin{macrocode}
\def\BitSet@SetDecBig#1#2#3#4#5#6#7#8#9!{%
  \ifx\\#9\\%
    \BitSet@SetDec#1#2#3#4#5#6#7#8!%
  \else
    \ifcase\BigIntCalcOdd#1#2#4#5#6#7#8#9! %
      0%
    \or
      1%
?   \else\BitSetError:ThisCannotHappen%
    \fi
    \BitSet@AfterFi{%
      \expandafter\expandafter\expandafter\BitSet@SetDecBig
      \BigIntCalcShr#1#2#3#4#5#6#7#8#9!!%
    }%
  \BitSet@Fi
}
%    \end{macrocode}
%    \end{macro}
%    \begin{macro}{\BitSet@SetDec}
%    \begin{macrocode}
\def\BitSet@SetDec#1!{%
  \ifcase#1 %
  \or 1%
  \else
    \ifodd#1 %
      1%
    \else
      0%
    \fi
    \BitSet@AfterFi{%
      \expandafter\expandafter\expandafter\BitSet@SetDec
      \IntCalcShr#1!!%
    }%
  \BitSet@Fi
}
%    \end{macrocode}
%    \end{macro}
%
% \subsection{Export}\label{src:export}
%
% \subsubsection{To binary number}
%
%    \begin{macro}{\bitsetGetBin}
%    \begin{macrocode}
\def\bitsetGetBin#1#2{%
  \romannumeral0%
  \expandafter\expandafter\expandafter\BitSet@@GetBin
  \intcalcNum{#2}!{#1}%
}
%    \end{macrocode}
%    \end{macro}
%    \begin{macro}{\BitSet@@GetBin}
%    \begin{macrocode}
\def\BitSet@@GetBin#1!#2{%
  \BitSet@IfUndefined{#2}{%
    \ifnum#1>1 %
      \BitSet@AfterFi{%
        \expandafter\expandafter\expandafter\BitSet@Fill
        \IntCalcDec#1!!0%
      }%
    \else
      \BitSet@AfterFi{ 0}%
    \BitSet@Fi
  }{%
    \expandafter\expandafter\expandafter\BitSet@NumBinRev
    \expandafter\expandafter\expandafter1%
    \expandafter\expandafter\expandafter!%
    \csname BS@#2\endcsname!!#1!%
  }%
}
%    \end{macrocode}
%    \end{macro}
%    \begin{macro}{\BitSet@Fill}
%    |#1|: number of leading digits |0|\\
%    |#2|: result
%    \begin{macrocode}
\def\BitSet@Fill#1!{%
  \ifnum#1>0 %
    \BitSet@AfterFi{%
      \expandafter\expandafter\expandafter\BitSet@Fill
      \IntCalcDec#1!!0%
    }%
  \else
    \BitSet@AfterFi{ }%
  \BitSet@Fi
}
%    \end{macrocode}
%    \end{macro}
%
%    \begin{macro}{\BitSet@NumBinRev}
%    |#1|: bit counter (including |#2|)\\
%    |#2#3|: reverted number\\
%    |#4|: result\\
%    |#5|: min size
%    \begin{macrocode}
\def\BitSet@NumBinRev#1!#2#3!{%
  \ifx\\#3\\%
    \BitSet@AfterFi{%
      \BitSet@NumBinFill#1!#2%
    }%
  \else
    \BitSet@AfterFi{%
      \expandafter\expandafter\expandafter\BitSet@NumBinRev
      \IntCalcInc#1!!#3!#2%
    }%
  \BitSet@Fi
}
%    \end{macrocode}
%    \end{macro}
%    \begin{macro}{\BitSet@NumBinFill}
%    \begin{macrocode}
\def\BitSet@NumBinFill#1!#2!#3!{%
  \ifnum#3>#1 %
    \BitSet@AfterFi{%
      \expandafter\expandafter\expandafter\BitSet@Fill
      \IntCalcSub#3!#1!!#2%
    }%
  \else
    \BitSet@AfterFi{ #2}%
  \BitSet@Fi
}
%    \end{macrocode}
%    \end{macro}
%
% \subsubsection{To octal/hexadecimal number}
%
%    \begin{macro}{\bitsetGetOct}
%    \begin{macrocode}
\def\bitsetGetOct#1#2{%
  \romannumeral0%
  \bitsetIsEmpty{#1}{%
    \expandafter\expandafter\expandafter\BitSet@@GetOctHex
    \intcalcNum{#2}!3!230%
  }{%
    \expandafter\expandafter\expandafter\BitSet@@GetOct
    \expandafter\expandafter\expandafter1%
    \expandafter\expandafter\expandafter!%
    \expandafter\expandafter\expandafter!%
    \csname BS@#1\endcsname00%
    \BitSet@Empty\BitSet@Empty\BitSet@Empty!{#2}%
  }%
}
%    \end{macrocode}
%    \end{macro}
%    \begin{macro}{\bitsetGetHex}
%    \begin{macrocode}
\def\bitsetGetHex#1#2{%
  \romannumeral0%
  \bitsetIsEmpty{#1}{%
    \expandafter\expandafter\expandafter\BitSet@@GetOctHex
    \intcalcNum{#2}!4!340%
  }{%
    \expandafter\expandafter\expandafter\BitSet@@GetHex
    \expandafter\expandafter\expandafter1%
    \expandafter\expandafter\expandafter!%
    \expandafter\expandafter\expandafter!%
    \csname BS@#1\endcsname000%
    \BitSet@Empty\BitSet@Empty\BitSet@Empty\BitSet@Empty!{#2}%
  }%
}
%    \end{macrocode}
%    \end{macro}
%    \begin{macro}{\BitSet@@GetOct}
%    |#1|: number of digits\\
%    |#2|: result\\
%    |#3#4#5|: bits
%    \begin{macrocode}
\def\BitSet@@GetOct#1!#2!#3#4#5{%
  \ifx#5\BitSet@Empty
    \BitSet@AfterFi{%
      \expandafter\expandafter\expandafter\BitSet@GetOctHex
      \IntCalcDec#1!!#2!23%
    }%
  \else
    \BitSet@AfterFi{%
      \expandafter\expandafter\expandafter\BitSet@@GetOct
      \number\IntCalcInc#1!\expandafter\expandafter\expandafter!%
      \csname BitSet@Oct#5#4#3\endcsname#2!%
    }%
  \BitSet@Fi
}
%    \end{macrocode}
%    \end{macro}
%    \begin{macro}{\BitSet@Oct[000..111]}
%    \begin{macrocode}
\def\BitSet@Temp#1#2#3#4{%
  \expandafter\def\csname BitSet@Oct#1#2#3\endcsname{#4}%
}
\BitSet@Temp0000%
\BitSet@Temp0011%
\BitSet@Temp0102%
\BitSet@Temp0113%
\BitSet@Temp1004%
\BitSet@Temp1015%
\BitSet@Temp1106%
\BitSet@Temp1117%
%    \end{macrocode}
%    \end{macro}
%    \begin{macro}{\BitSet@@GetHex}
%    |#1|: number of digits\\
%    |#2|: result\\
%    |#3#4#5#6|: bits
%    \begin{macrocode}
\def\BitSet@@GetHex#1!#2!#3#4#5#6{%
  \ifx#6\BitSet@Empty
    \BitSet@AfterFi{%
      \expandafter\expandafter\expandafter\BitSet@GetOctHex
      \IntCalcDec#1!!#2!34%
    }%
  \else
    \BitSet@AfterFi{%
      \expandafter\expandafter\expandafter\BitSet@@GetHex
      \number\IntCalcInc#1!\expandafter\expandafter\expandafter!%
      \csname BitSet@Hex#6#5#4#3\endcsname#2!%
    }%
  \BitSet@Fi
}
%    \end{macrocode}
%    \end{macro}
%    \begin{macro}{\BitSet@Hex[0000..1111]}
%    \begin{macrocode}
\def\BitSet@Temp#1#2#3#4#5{%
  \expandafter\def\csname BitSet@Hex#1#2#3#4\endcsname{#5}%
}
\BitSet@Temp00000%
\BitSet@Temp00011%
\BitSet@Temp00102%
\BitSet@Temp00113%
\BitSet@Temp01004%
\BitSet@Temp01015%
\BitSet@Temp01106%
\BitSet@Temp01117%
\BitSet@Temp10008%
\BitSet@Temp10019%
\BitSet@Temp1010A%
\BitSet@Temp1011B%
\BitSet@Temp1100C%
\BitSet@Temp1101D%
\BitSet@Temp1110E%
\BitSet@Temp1111F%
%    \end{macrocode}
%    \end{macro}
%    \begin{macro}{\BitSet@GetOctHex}
%    Leading zeros $ (|#4| - |#1| * 3 + 2)/3 $ if $ |#4| > |#1| * 3$\\
%    |#1|: digit size\\
%    |#2|: result\\
%    |#3|: bits per digit - 1\\
%    |#4|: bits per digit
%    |#5|: garbage\\
%    |#6|: min size
%    \begin{macrocode}
\def\BitSet@GetOctHex#1!#2!#3#4#5!#6{%
  \expandafter\BitSet@@GetOctHex
  \number\intcalcNum{#6}\expandafter\expandafter\expandafter!%
  \IntCalcMul#1!#4!!#3#4#2%
}
%    \end{macrocode}
%    \end{macro}
%    \begin{macro}{\BitSet@@GetOctHex}
%    |#1|: plain min size\\
%    |#2|: digits * (bits per digit)\\
%    |#3|: bits per digit - 1\\
%    |#4|: bits per digit
%    \begin{macrocode}
\def\BitSet@@GetOctHex#1!#2!#3#4{%
  \ifnum#1>#2 %
    \BitSet@AfterFi{%
      \expandafter\expandafter\expandafter\expandafter
      \expandafter\expandafter\expandafter\BitSet@Fill
      \expandafter\IntCalcDiv\number
      \expandafter\expandafter\expandafter\IntCalcAdd
      \IntCalcSub#1!#2!!#3!!#4!!%
    }%
  \else
    \BitSet@AfterFi{ }%
  \BitSet@Fi
}
%    \end{macrocode}
%    \end{macro}
%
% \subsubsection{To decimal number}
%
%    \begin{macro}{\bitsetGetDec}
%    \begin{macrocode}
\def\bitsetGetDec#1{%
  \romannumeral0%
  \BitSet@IfUndefined{#1}{ 0}{%
    \expandafter\expandafter\expandafter\BitSet@GetDec
    \csname BS@#1\endcsname!%
  }%
}
%    \end{macrocode}
%    \end{macro}
%    \begin{macro}{\BitSet@GetDec}
%    \begin{macrocode}
\def\BitSet@GetDec#1#2!{%
  \ifx\\#2\\%
    \BitSet@AfterFi{ #1}%
  \else
    \BitSet@AfterFi{%
      \BitSet@@GetDec2!#1!#2!%
    }%
  \BitSet@Fi
}
%    \end{macrocode}
%    \end{macro}
%    \begin{macro}{\BitSet@@GetDec}
%    |#1|: power of two\\
%    |#2|: result\\
%    |#3#4|: number
%    \begin{macrocode}
\def\BitSet@@GetDec#1!#2!#3#4!{%
  \ifx\\#4\\%
    \ifx#31%
      \BitSet@AfterFiFi{%
        \expandafter\expandafter\expandafter\BitSet@Space
        \IntCalcAdd#1!#2!%
      }%
    \else
      \BitSet@AfterFiFi{ #2}%
    \fi
  \else
    \ifx#31%
      \BitSet@AfterFiFi{%
        \csname BitSet@N#1%
        \expandafter\expandafter\expandafter\endcsname
        \IntCalcAdd#1!#2!!#4!%
      }%
    \else
      \BitSet@AfterFiFi{%
        \csname BitSet@N#1\endcsname#2!#4!%
      }%
    \fi
  \BitSet@Fi
}
%    \end{macrocode}
%    \end{macro}
%    \begin{macro}{\BitSet@N[1,2,4,...]}
%    \begin{macrocode}
\def\BitSet@Temp#1#2{%
  \expandafter\def\csname BitSet@N#1\endcsname{%
    \BitSet@@GetDec#2!%
  }%
}
\BitSet@Temp{1}{2}
\BitSet@Temp{2}{4}
\BitSet@Temp{4}{8}
\BitSet@Temp{8}{16}
\BitSet@Temp{16}{32}
\BitSet@Temp{32}{64}
\BitSet@Temp{64}{128}
\BitSet@Temp{128}{256}
\BitSet@Temp{256}{512}
\BitSet@Temp{512}{1024}
\BitSet@Temp{1024}{2048}
\BitSet@Temp{2048}{4096}
\BitSet@Temp{4096}{8192}
\BitSet@Temp{8192}{16384}
\BitSet@Temp{16384}{32768}
\BitSet@Temp{32768}{65536}
\BitSet@Temp{65536}{131072}
\BitSet@Temp{131072}{262144}
\BitSet@Temp{262144}{524288}
\BitSet@Temp{524288}{1048576}
\BitSet@Temp{1048576}{2097152}
\BitSet@Temp{2097152}{4194304}
\BitSet@Temp{4194304}{8388608}
\BitSet@Temp{8388608}{16777216}
\BitSet@Temp{16777216}{33554432}
\BitSet@Temp{33554432}{67108864}
\BitSet@Temp{67108864}{134217728}
\BitSet@Temp{134217728}{268435456}
\BitSet@Temp{268435456}{536870912}
\BitSet@Temp{536870912}{1073741824}
%    \end{macrocode}
%    \end{macro}
%    \begin{macro}{\BitSet@N1073741824}
%    \begin{macrocode}
\expandafter\def\csname BitSet@N1073741824\endcsname{%
  \BitSet@GetDecBig2147483648!%
}%
%    \end{macrocode}
%    \end{macro}
%
%    \begin{macro}{\BitSet@GetDecBig}
%    |#1|: current power of two\\
%    |#2|: result\\
%    |#3#4|: number
%    \begin{macrocode}
\def\BitSet@GetDecBig#1!#2!#3#4!{%
  \ifx\\#4\\%
    \ifx#31%
      \BitSet@AfterFiFi{%
        \expandafter\expandafter\expandafter\BitSet@Space
        \BigIntCalcAdd#1!#2!%
      }%
    \else
      \BitSet@AfterFiFi{ #2}%
    \fi
  \else
    \ifx#31%
      \BitSet@AfterFiFi{%
        \expandafter\expandafter\expandafter\BitSet@@GetDecBig
        \BigIntCalcAdd#1!#2!!#1!#4!%
      }%
    \else
      \BitSet@AfterFiFi{%
        \expandafter\expandafter\expandafter\BitSet@GetDecBig
        \BigIntCalcShl#1!!#2!#4!%
      }%
    \fi
  \BitSet@Fi
}
%    \end{macrocode}
%    \end{macro}
%    \begin{macro}{\BitSet@@GetDecBig}
%    |#1|: result\\
%    |#2|: power of two\\
%    |#3#4|: number
%    \begin{macrocode}
\def\BitSet@@GetDecBig#1!#2!{%
  \expandafter\expandafter\expandafter\BitSet@GetDecBig
  \BigIntCalcShl#2!!#1!%
}
%    \end{macrocode}
%    \end{macro}
%
% \subsection{Logical operators}\label{src:log}
%
% \subsubsection{\cs{bitsetAnd}}
%
%    \begin{macro}{\bitsetAnd}
%    Decision table for \cs{bitsetAnd}:
%    \begin{quote}
%    \begin{tabular}{l|l|l|l|}
%                         & undef(B)   & empty(B)   & cardinality(B)$>$0\\
%      \hline
%      undef(A)           & A := empty & A := empty & A := empty\\
%      \hline
%      empty(A)           &            &            &\\
%      \hline
%      cardinality(A)$>$0 & A := empty & A := empty & A \&= B\\
%      \hline
%    \end{tabular}
%    \end{quote}
%    \begin{macrocode}
\def\bitsetAnd#1#2{%
  \bitsetIsEmpty{#1}{%
    \bitsetReset{#1}%
  }{%
    \bitsetIsEmpty{#2}{%
      \bitsetReset{#1}%
    }{%
      \expandafter\edef\csname BS@#1\endcsname{%
        \expandafter\expandafter\expandafter\BitSet@And
        \csname BS@#1\expandafter\expandafter\expandafter\endcsname
        \expandafter\expandafter\expandafter!%
        \csname BS@#2\endcsname!!%
      }%
      \expandafter\ifx\csname BS@#1\endcsname\BitSet@Empty
        \bitsetReset{#1}%
      \fi
    }%
  }%
}
%    \end{macrocode}
%    \end{macro}
%    \begin{macro}{\BitSet@And}
%    \begin{macrocode}
\def\BitSet@And#1#2!#3#4!#5!{%
  \ifx\\#2\\%
    \ifnum#1#3=11 #51\fi
  \else
    \ifx\\#4\\%
      \ifnum#1#3=11 #51\fi
    \else
      \ifnum#1#3=11 %
        #51%
        \BitSet@AfterFiFiFi{%
          \BitSet@And#2!#4!!%
        }%
      \else
        \BitSet@AfterFiFiFi{%
          \BitSet@And#2!#4!#50!%
        }%
      \fi
    \fi
  \BitSet@Fi
}
%    \end{macrocode}
%    \end{macro}
%
% \subsubsection{\cs{bitsetAndNot}}
%
%    \begin{macro}{\bitsetAndNot}
%    Decision table for \cs{bitsetAndNot}:
%    \begin{quote}
%    \begin{tabular}{l|l|l|l|}
%                         & undef(B)   & empty(B)   & cardinality(B)$>$0\\
%      \hline
%      undef(A)           & A := empty & A := empty & A := empty\\
%      \hline
%      empty(A)           &            &            &\\
%      \hline
%      cardinality(A)$>$0 &            &            & A \&= !B\\
%      \hline
%    \end{tabular}
%    \end{quote}
%    \begin{macrocode}
\def\bitsetAndNot#1#2{%
  \bitsetIsEmpty{#1}{%
    \bitsetReset{#1}%
  }{%
    \bitsetIsEmpty{#2}{%
    }{%
      \expandafter\edef\csname BS@#1\endcsname{%
        \expandafter\expandafter\expandafter\BitSet@AndNot
        \csname BS@#1\expandafter\expandafter\expandafter\endcsname
        \expandafter\expandafter\expandafter!%
        \csname BS@#2\endcsname!!%
      }%
      \expandafter\ifx\csname BS@#1\endcsname\BitSet@Empty
        \bitsetReset{#1}%
      \fi
    }%
  }%
}
%    \end{macrocode}
%    \end{macro}
%    \begin{macro}{\BitSet@AndNot}
%    \begin{macrocode}
\def\BitSet@AndNot#1#2!#3#4!#5!{%
  \ifx\\#2\\%
    \ifnum#1#3=10 #51\fi
  \else
    \ifx\\#4\\%
      #5%
      \ifnum#1#3=10 1\else 0\fi
      #2%
    \else
      \ifnum#1#3=10 %
        #51%
        \BitSet@AfterFiFiFi{%
          \BitSet@AndNot#2!#4!!%
        }%
      \else
        \BitSet@AfterFiFiFi{%
          \BitSet@AndNot#2!#4!#50!%
        }%
      \fi
    \fi
  \BitSet@Fi
}
%    \end{macrocode}
%    \end{macro}
%
% \subsubsection{\cs{bitsetOr}}
%
%    \begin{macro}{\bitsetOr}
%    Decision table for \cs{bitsetOr}:
%    \begin{quote}
%    \begin{tabular}{l|l|l|l|}
%                         & undef(B)   & empty(B)   & cardinality(B)$>$0\\
%      \hline
%      undef(A)           & A := empty & A := empty & A := B\\
%      \hline
%      empty(A)           &            &            & A := B\\
%      \hline
%      cardinality(A)$>$0 &            &            & A \string|= B\\
%      \hline
%    \end{tabular}
%    \end{quote}
%    \begin{macrocode}
\def\bitsetOr#1#2{%
  \bitsetIsEmpty{#2}{%
    \BitSet@IfUndefined{#1}{\bitsetReset{#1}}{}%
  }{%
    \bitsetIsEmpty{#1}{%
      \expandafter\let\csname BS@#1\expandafter\endcsname
                      \csname BS@#2\endcsname
    }{%
      \expandafter\edef\csname BS@#1\endcsname{%
        \expandafter\expandafter\expandafter\BitSet@Or
        \csname BS@#1\expandafter\expandafter\expandafter\endcsname
        \expandafter\expandafter\expandafter!%
        \csname BS@#2\endcsname!%
      }%
    }%
  }%
}
%    \end{macrocode}
%    \end{macro}
%    \begin{macro}{\BitSet@Or}
%    \begin{macrocode}
\def\BitSet@Or#1#2!#3#4!{%
  \ifnum#1#3>0 1\else 0\fi
  \ifx\\#2\\%
    #4%
  \else
    \ifx\\#4\\%
      #2%
    \else
      \BitSet@AfterFiFi{%
        \BitSet@Or#2!#4!%
      }%
    \fi
  \BitSet@Fi
}
%    \end{macrocode}
%    \end{macro}
%
% \subsubsection{\cs{bitsetXor}}
%
%    \begin{macro}{\bitsetXor}
%    Decision table for \cs{bitsetXor}:
%    \begin{quote}
%    \begin{tabular}{l|l|l|l|}
%                         & undef(B)   & empty(B)   & cardinality(B)$>$0\\
%      \hline
%      undef(A)           & A := empty & A := empty & A := B\\
%      \hline
%      empty(A)           &            &            & A := B\\
%      \hline
%      cardinality(A)$>$0 &            &            & A \string^= B\\
%      \hline
%    \end{tabular}
%    \end{quote}
%    \begin{macrocode}
\def\bitsetXor#1#2{%
  \bitsetIsEmpty{#2}{%
    \BitSet@IfUndefined{#1}{\bitsetReset{#1}}{}%
  }{%
    \bitsetIsEmpty{#1}{%
      \expandafter\let\csname BS@#1\expandafter\endcsname
                      \csname BS@#2\endcsname
    }{%
      \expandafter\edef\csname BS@#1\endcsname{%
        \expandafter\expandafter\expandafter\BitSet@Xor
        \csname BS@#1\expandafter\expandafter\expandafter\endcsname
        \expandafter\expandafter\expandafter!%
        \csname BS@#2\endcsname!!%
      }%
      \expandafter\ifx\csname BS@#1\endcsname\BitSet@Empty
        \bitsetReset{#1}%
      \fi
    }%
  }%
}
%    \end{macrocode}
%    \end{macro}
%    \begin{macro}{\BitSet@Xor}
%    \begin{macrocode}
\def\BitSet@Xor#1#2!#3#4!#5!{%
  \ifx\\#2\\%
    \ifx#1#3%
      \ifx\\#4\\%
      \else
        #50#4%
      \fi
    \else
      #51#4%
    \fi
  \else
    \ifx\\#4\\%
      #5%
      \ifx#1#30\else 1\fi
      #2%
    \else
      \ifx#1#3%
        \BitSet@AfterFiFiFi{%
          \BitSet@Xor#2!#4!#50!%
        }%
      \else
        #51%
        \BitSet@AfterFiFiFi{%
          \BitSet@Xor#2!#4!!%
        }%
      \fi
    \fi
  \BitSet@Fi
}
%    \end{macrocode}
%    \end{macro}
%
% \subsubsection{Shifting}\label{src:shift}
%
% \subsubsection{\cs{bitsetShiftLeft}}
%
%    \begin{macro}{\bitsetShiftLeft}
%    \begin{macrocode}
\def\bitsetShiftLeft#1#2{%
  \BitSet@IfUndefined{#1}{%
    \bitsetReset{#1}%
  }{%
    \bitsetIsEmpty{#1}{%
    }{%
      \expandafter\expandafter\expandafter\BitSet@ShiftLeft
      \intcalcNum{#2}!{#1}%
    }%
  }%
}
%    \end{macrocode}
%    \end{macro}
%    \begin{macro}{\BitSet@ShiftLeft}
%    \begin{macrocode}
\def\BitSet@ShiftLeft#1!#2{%
  \ifcase\intcalcSgn{#1} %
  \or
    \begingroup
      \uccode`m=`0 %
    \uppercase\expandafter{\expandafter\endgroup
      \expandafter\edef\csname BS@#2\expandafter\endcsname
      \expandafter{%
        \romannumeral#1000\expandafter\BitSet@Space
        \csname BS@#2\endcsname
      }%
    }%
  \else
    \expandafter\BitSet@ShiftRight\BitSet@Gobble#1!{#2}%
  \fi
}
%    \end{macrocode}
%    \end{macro}
%
% \subsubsection{\cs{bitsetShiftRight}}
%
%    \begin{macro}{\bitsetShiftRight}
%    \begin{macrocode}
\def\bitsetShiftRight#1#2{%
  \BitSet@IfUndefined{#1}{%
    \bitsetReset{#1}%
  }{%
    \bitsetIsEmpty{#1}{%
    }{%
      \expandafter\expandafter\expandafter\BitSet@ShiftRight
      \intcalcNum{#2}!{#1}%
    }%
  }%
}
%    \end{macrocode}
%    \end{macro}
%    \begin{macro}{\BitSet@ShiftRight}
%    \begin{macrocode}
\def\BitSet@ShiftRight#1!#2{%
  \ifcase\intcalcSgn{#1} %
  \or
    \expandafter\edef\csname BS@#2\endcsname{%
      \expandafter\expandafter\expandafter\BitSet@Kill
      \csname BS@#2\expandafter\endcsname\expandafter\BitSet@Empty
      \expandafter=%
      \expandafter{\expandafter}\expandafter{\expandafter}%
      \romannumeral#1000!%
    }%
  \else
    \expandafter\BitSet@ShiftLeft\BitSet@Gobble#1!{#2}%
  \fi
}
%    \end{macrocode}
%    \end{macro}
%    \begin{macro}{\BitSet@Kill}
%    \begin{macrocode}
\def\BitSet@Kill#1#2=#3#4#5{%
  #3#4%
  \ifx#5!%
    \ifx#1\BitSet@Empty
      0%
    \else
      #1#2%
    \fi
  \else
    \ifx#1\BitSet@Empty
      0%
      \BitSet@AfterFiFi\BitSet@Cleanup
    \else
      \BitSet@Kill#2=%
    \fi
  \BitSet@Fi
}
%    \end{macrocode}
%    \end{macro}
%
% \subsection{Bit manipulation}\label{src:bitman}
%
%    \begin{macro}{\bitsetClear}
%    \begin{macrocode}
\def\bitsetClear{%
  \BitSet@CheckIndex\BitSet@Clear
}
%    \end{macrocode}
%    \end{macro}
%    \begin{macro}{\bitsetSet}
%    \begin{macrocode}
\def\bitsetSet{%
  \BitSet@CheckIndex\BitSet@Set
}
%    \end{macrocode}
%    \end{macro}
%    \begin{macro}{\bitsetFlip}
%    \begin{macrocode}
\def\bitsetFlip{%
  \BitSet@CheckIndex\BitSet@Flip
}
%    \end{macrocode}
%    \end{macro}
%    \begin{macro}{\bitsetSetValue}
%    \begin{macrocode}
\def\bitsetSetValue#1#2#3{%
  \expandafter\expandafter\expandafter\BitSet@SetValue
  \intcalcNum{#3}!{#1}{#2}%
}
%    \end{macrocode}
%    \end{macro}
%    \begin{macro}{\BitSet@SetValue}
%    |#1|: plain value\\
%    |#2|: BitSet\\
%    |#3|: Index
%    \begin{macrocode}
\def\BitSet@SetValue#1!{%
  \BitSet@CheckIndex{%
    \ifcase#1 %
      \expandafter\BitSet@Clear
    \or
      \expandafter\BitSet@Set
    \else
      \BitSet@ErrorInvalidBitValue{#1}%
      \expandafter\expandafter\expandafter\BitSet@Gobble
      \expandafter\BitSet@Gobble
    \fi
  }%
}
%    \end{macrocode}
%    \end{macro}
%    \begin{macro}{\BitSet@ErrorInvalidBitValue}
%    |#1|: Wrong bit value
%    \begin{macrocode}
\def\BitSet@ErrorInvalidBitValue#1{%
  \@PackageError{bitset}{%
    Invalid bit value (#1) not in range 0..1%
  }\@ehc
}
%    \end{macrocode}
%    \end{macro}
%
% \subsubsection{Clear operation}
%
%    \begin{macro}{\BitSet@Clear}
%    |#1|: BitSet\\
%    |#2|: plain and checked index
%    \begin{macrocode}
\def\BitSet@Clear#1#2{%
  \edef\BitSet@Temp{%
    \expandafter\expandafter\expandafter\BitSet@@Clear
    \csname BS@#1\expandafter\endcsname
    \expandafter\BitSet@Empty\expandafter=\expandafter!%
    \romannumeral#2000!%
  }%
  \expandafter\let\csname BS@#1\expandafter\endcsname
  \ifx\BitSet@Temp\BitSet@Empty
    \BitSet@Zero
  \else
    \BitSet@Temp
  \fi
}
%    \end{macrocode}
%    \end{macro}
%    \begin{macro}{\BitSet@@Clear}
%    \begin{macrocode}
\def\BitSet@@Clear#1#2=#3!#4{%
  \ifx#4!%
    \ifx#1\BitSet@Empty
    \else
      \ifx\BitSet@Empty#2%
      \else
        #30#2%
      \fi
    \fi
  \else
    \ifx#1\BitSet@Empty
      \BitSet@AfterFiFi\BitSet@Cleanup
    \else
      \ifx#10%
        \BitSet@AfterFiFiFi{%
          \BitSet@@Clear#2=#30!%
        }%
      \else
        #31%
        \BitSet@AfterFiFiFi{%
          \BitSet@@Clear#2=!%
        }%
      \fi
    \fi
  \BitSet@Fi
}
%    \end{macrocode}
%    \end{macro}
%
% \subsubsection{Set operation}
%
%    \begin{macro}{\BitSet@Set}
%    |#1|: BitSet\\
%    |#2|: plain and checked Index
%    \begin{macrocode}
\def\BitSet@Set#1#2{%
  \expandafter\edef\csname BS@#1\endcsname{%
    \expandafter\expandafter\expandafter\BitSet@@Set
    \csname BS@#1\expandafter\endcsname
    \expandafter\BitSet@Empty\expandafter=%
    \expandafter{\expandafter}\expandafter{\expandafter}%
    \romannumeral#2000!%
  }%
}
%    \end{macrocode}
%    \end{macro}
%    \begin{macro}{\BitSet@@Set}
%    \begin{macrocode}
\def\BitSet@@Set#1#2=#3#4#5{%
  #3#4%
  \ifx#5!%
    1#2%
  \else
    \ifx#1\BitSet@Empty
      0%
      \BitSet@AfterFiFi\BitSet@@@Set
    \else
      #1%
      \BitSet@@Set#2=%
    \fi
  \BitSet@Fi
}
%    \end{macrocode}
%    \end{macro}
%    \begin{macro}{\BitSet@@@Set}
%    \begin{macrocode}
\def\BitSet@@@Set#1{%
  \ifx#1!%
    1%
  \else
    0%
    \expandafter\BitSet@@@Set
  \fi
}
%    \end{macrocode}
%    \end{macro}
%
% \subsubsection{Flip operation}
%
%    \begin{macro}{\BitSet@Flip}
%    |#1|: BitSet\\
%    |#2|: plain and checked Index
%    \begin{macrocode}
\def\BitSet@Flip#1#2{%
  \edef\BitSet@Temp{%
    \expandafter\expandafter\expandafter\BitSet@@Flip
    \csname BS@#1\expandafter\endcsname
    \expandafter\BitSet@Empty\expandafter=\expandafter!%
    \romannumeral#2000!%
  }%
  \expandafter\let\csname BS@#1\expandafter\endcsname
  \ifx\BitSet@Temp\BitSet@Empty
    \BitSet@Zero
  \else
    \BitSet@Temp
  \fi
}
%    \end{macrocode}
%    \end{macro}
%    \begin{macro}{\BitSet@@Flip}
%    \begin{macrocode}
\def\BitSet@@Flip#1#2=#3!#4{%
  \ifx#4!%
    \ifx#11%
      \ifx\BitSet@Empty#2%
      \else
        #30#2%
      \fi
    \else
      #31#2%
    \fi
  \else
    \ifx#1\BitSet@Empty
      #30%
      \BitSet@AfterFiFi\BitSet@@@Set
    \else
      \ifx#10%
        \BitSet@AfterFiFiFi{%
          \BitSet@@Flip#2=#30!%
        }%
      \else
        #31%
        \BitSet@AfterFiFiFi{%
          \BitSet@@Flip#2=!%
        }%
      \fi
    \fi
  \BitSet@Fi
}
%    \end{macrocode}
%    \end{macro}
%
% \subsubsection{Range operators}
%
%    \begin{macro}{\bitsetClearRange}
%    \begin{macrocode}
\def\bitsetClearRange{%
  \BitSet@Range\BitSet@Clear
}
%    \end{macrocode}
%    \end{macro}
%    \begin{macro}{\bitsetSetRange}
%    \begin{macrocode}
\def\bitsetSetRange{%
  \BitSet@Range\BitSet@Set
}
%    \end{macrocode}
%    \end{macro}
%    \begin{macro}{\bitsetFlipRange}
%    \begin{macrocode}
\def\bitsetFlipRange{%
  \BitSet@Range\BitSet@Flip
}
%    \end{macrocode}
%    \end{macro}
%    \begin{macro}{\bitsetSetValueRange}
%    \begin{macrocode}
\def\bitsetSetValueRange#1#2#3#4{%
  \expandafter\expandafter\expandafter\BitSet@SetValueRange
  \intcalcNum{#4}!{#1}{#2}{#3}%
}
%    \end{macrocode}
%    \end{macro}
%    \begin{macro}{\BitSet@SetValueRange}
%    \begin{macrocode}
\def\BitSet@SetValueRange#1!#2#3#4{%
  \ifcase#1 %
    \BitSet@Range\BitSet@Clear{#2}{#3}{#4}%
  \or
    \BitSet@Range\BitSet@Set{#2}{#3}{#4}%
  \else
    \BitSet@ErrorInvalidBitValue{#1}%
  \fi
}
%    \end{macrocode}
%    \end{macro}
%
%    \begin{macro}{\BitSet@Range}
%    |#1|: clear/set/flip macro\\
%    |#2|: BitSet\\
%    |#3|: Index from\\
%    |#4|: Index to
%    \begin{macrocode}
\def\BitSet@Range#1#2#3#4{%
  \edef\BitSet@Temp{%
    \noexpand\BitSet@@Range\noexpand#1{#2}%
    \intcalcNum{#3}!\intcalcNum{#4}!%
  }%
  \BitSet@Temp
}
%    \end{macrocode}
%    \end{macro}
%    \begin{macro}{\BitSet@@Range}
%    |#1|: clear/set/flip macro\\
%    |#2|: BitSet\\
%    |#3|: Index from\\
%    |#4|: Index to
%    \begin{macrocode}
\def\BitSet@@Range#1#2#3!#4!{%
  \ifnum#3<0 %
    \BitSet@NegativeIndex#1{#2}#3!#4!0!#4!%
  \else
    \ifnum#4<0 %
      \BitSet@NegativeIndex#1{#2}#3!#4!#3!0!%
    \else
      \ifcase\intcalcCmp{#3}{#4} %
      \or
        \@PackageError{bitset}{%
          Wrong index numbers in range [#3..#4]\MessageBreak% hash-ok
          for clear/set/flip on bit set `#2'.\MessageBreak
          The lower index exceeds the upper index.\MessageBreak
          Canceling the operation as error recovery%
        }\@ehc
      \else
        \BitSet@@@Range#3!#4!#1{#2}%
      \fi
    \fi
  \fi
}
%    \end{macrocode}
%    \end{macro}
%    \begin{macro}{\BitSet@NegativeIndex}
%    \begin{macrocode}
\def\BitSet@NegativeIndex#1#2#3!#4!#5!#6!{%
  \@PackageError{bitset}{%
    Negative index in range [#3..#4]\MessageBreak % hash-ok
    for \string\bitset
    \ifx#1\BitSet@Clear
      Clear%
    \else
      \ifx#1\BitSet@Set
        Set%
      \else
        Flip%
      \fi
    \fi
    Range on bit set `#2'.\MessageBreak
    Using [#5..#6] as error recovery% hash-ok
  }\@ehc
  \BitSet@@Range#1{#2}#5!#6!%
}
%    \end{macrocode}
%    \end{macro}
%    \begin{macro}{\BitSet@@Range}
%    \begin{macrocode}
\def\BitSet@@@Range#1!#2!#3#4{%
  \ifnum#1<#2 %
    #3{#4}{#1}%
    \BitSet@AfterFi{%
      \expandafter\expandafter\expandafter\BitSet@@@Range
      \IntCalcInc#1!!#2!#3{#4}%
    }%
  \BitSet@Fi
}
%    \end{macrocode}
%    \end{macro}
%
% \subsection{Bit retrieval}\label{src:bitret}
%
% \subsubsection{\cs{bitsetGet}}
%
%    \begin{macro}{\bitsetGet}
%    \begin{macrocode}
\def\bitsetGet#1#2{%
  \number
  \expandafter\expandafter\expandafter\BitSet@Get
  \intcalcNum{#2}!{#1}%
}
%    \end{macrocode}
%    \end{macro}
%    \begin{macro}{\BitSet@Get}
%    |#1|: plain index\\
%    |#2|: BitSet
%    \begin{macrocode}
\def\BitSet@Get#1!#2{%
  \ifnum#1<0 %
    \BitSet@AfterFi{%
      0 \BitSetError:NegativeIndex%
    }%
  \else
    \BitSet@IfUndefined{#2}{0}{%
      \expandafter\expandafter\expandafter\BitSet@@Get
      \csname BS@#2\expandafter\endcsname
      \expandafter!\expandafter=%
      \expandafter{\expandafter}\expandafter{\expandafter}%
      \romannumeral\intcalcNum{#1}000!%
    }%
    \expandafter\BitSet@Space
  \BitSet@Fi
}
%    \end{macrocode}
%    \end{macro}
%    \begin{macro}{\BitSet@@Get}
%    \begin{macrocode}
\def\BitSet@@Get#1#2=#3#4#5{%
  #3#4%
  \ifx#5!%
    \ifx#1!%
      0%
    \else
      #1%
    \fi
  \else
    \ifx#1!%
      0%
      \BitSet@AfterFiFi\BitSet@Cleanup
    \else
      \BitSet@@Get#2=%
    \fi
  \BitSet@Fi
}
%    \end{macrocode}
%    \end{macro}
%
% \subsubsection{\cs{bitsetNextClearBit}, \cs{bitsetNextSetBit}}
%
%    \begin{macro}{\bitsetNextClearBit}
%    \begin{macrocode}
\def\bitsetNextClearBit#1#2{%
  \number
  \expandafter\expandafter\expandafter\BitSet@NextClearBit
  \intcalcNum{#2}!{#1} %
}
%    \end{macrocode}
%    \end{macro}
%    \begin{macro}{\BitSet@NextClearBit}
%    |#1|: Index\\
%    |#2|: BitSet
%    \begin{macrocode}
\def\BitSet@NextClearBit#1!#2{%
  \ifnum#1<0 %
    \BitSet@NextClearBit0!{#2}%
    \BitSet@AfterFi{%
      \expandafter\BitSet@Space
      \expandafter\BitSetError:NegativeIndex\romannumeral0%
    }%
  \else
    \bitsetIsEmpty{#2}{#1}{%
      \expandafter\BitSet@Skip
      \number#1\expandafter\expandafter\expandafter!%
      \csname BS@#2\endcsname!!!!!!!!!=%
      {\BitSet@@NextClearBit#1!}%
    }%
  \BitSet@Fi
}
%    \end{macrocode}
%    \end{macro}
%    \begin{macro}{\BitSet@@NextClearBit}
%    |#1|: index for next bit in |#2|\\
%    |#2|: next bit
%    \begin{macrocode}
\def\BitSet@@NextClearBit#1!#2{%
  \ifx#2!%
    #1%
  \else
    \ifx#20%
      #1%
      \BitSet@AfterFiFi\BitSet@Cleanup
    \else
      \BitSet@AfterFiFi{%
        \expandafter\expandafter\expandafter\BitSet@@NextClearBit
        \IntCalcInc#1!!%
      }%
    \fi
  \BitSet@Fi
}
%    \end{macrocode}
%    \end{macro}
%
%    \begin{macro}{\bitsetNextSetBit}
%    \begin{macrocode}
\def\bitsetNextSetBit#1#2{%
  \number
  \expandafter\expandafter\expandafter\BitSet@NextSetBit
  \intcalcNum{#2}!{#1} %
}
%    \end{macrocode}
%    \end{macro}
%    \begin{macro}{\BitSet@NextSetBit}
%    |#1|: Index\\
%    |#2|: BitSet
%    \begin{macrocode}
\def\BitSet@NextSetBit#1!#2{%
  \ifnum#1<0 %
    \BitSet@NextSetBit0!{#2}%
    \BitSet@AfterFi{%
      \expandafter\BitSet@Space
      \expandafter\BitSetError:NegativeIndex\romannumeral0%
    }%
  \else
    \bitsetIsEmpty{#2}{-1}{%
      \expandafter\BitSet@Skip
      \number#1\expandafter\expandafter\expandafter!%
      \csname BS@#2\endcsname!!!!!!!!!=%
      {\BitSet@@NextSetBit#1!}%
    }%
  \BitSet@Fi
}
%    \end{macrocode}
%    \end{macro}
%    \begin{macro}{\BitSet@@NextSetBit}
%    |#1|: index for next bit in |#2|\\
%    |#2|: next bit
%    \begin{macrocode}
\def\BitSet@@NextSetBit#1!#2{%
  \ifx#2!%
    -1%
  \else
    \ifx#21%
      #1%
      \BitSet@AfterFiFi\BitSet@Cleanup
    \else
      \BitSet@AfterFiFi{%
        \expandafter\expandafter\expandafter\BitSet@@NextSetBit
        \IntCalcInc#1!!%
      }%
    \fi
  \BitSet@Fi
}
%    \end{macrocode}
%    \end{macro}
%    \begin{macro}{\BitSet@Cleanup}
%    \begin{macrocode}
\def\BitSet@Cleanup#1!{}
%    \end{macrocode}
%    \end{macro}
%
%    \begin{macro}{\BitSet@Skip}
%    |#1|: number of bits to skip\\
%    |#2|: bits\\
%    |#3|: continuation code
%    \begin{macrocode}
\def\BitSet@Skip#1!#2{%
  \ifx#2!%
    \BitSet@AfterFi{%
      \BitSet@SkipContinue%
    }%
  \else
    \ifcase#1 %
      \BitSet@AfterFiFi{%
        \BitSet@SkipContinue#2%
      }%
    \or
      \BitSet@AfterFiFi\BitSet@SkipContinue
    \or
      \BitSet@AfterFiFi{%
        \expandafter\BitSet@SkipContinue\BitSet@Gobble
      }%
    \else
      \ifnum#1>8 %
        \BitSet@AfterFiFiFi{%
          \expandafter\BitSet@Skip
          \number\IntCalcSub#1!8!\expandafter!%
          \BitSet@GobbleSeven
        }%
      \else
        \BitSet@AfterFiFiFi{%
          \expandafter\expandafter\expandafter\BitSet@Skip
          \IntCalcDec#1!!%
        }%
      \fi
    \fi
  \BitSet@Fi
}
%    \end{macrocode}
%    \end{macro}
%    \begin{macro}{\BitSet@SkipContinue}
%    |#1|: remaining bits\\
%    |#2|: continuation code
%    \begin{macrocode}
\def\BitSet@SkipContinue#1!#2=#3{%
  #3#1!%
}
%    \end{macrocode}
%    \end{macro}
%    \begin{macro}{\BitSet@GobbleSeven}
%    \begin{macrocode}
\def\BitSet@GobbleSeven#1#2#3#4#5#6#7{}
%    \end{macrocode}
%    \end{macro}
%
% \subsubsection{\cs{bitsetGetSetBitList}}
%
%    \begin{macro}{\bitsetGetSetBitList}
%    It's just a wrapper for \cs{bitsetNextSetBit}.
%    \begin{macrocode}
\def\bitsetGetSetBitList#1{%
  \romannumeral0%
  \bitsetIsEmpty{#1}{ }{%
    \expandafter\BitSet@GetSetBitList
    \number\BitSet@NextSetBit0!{#1}!{#1}{}!%
  }%
}
%    \end{macrocode}
%    \end{macro}
%    \begin{macro}{\BitSet@GetSetBitList}
%    |#1|: found index\\
%    |#2|: BitSet\\
%    |#3|: comma
%    |#4|: result
%    \begin{macrocode}
\def\BitSet@GetSetBitList#1!#2#3#4!{%
  \ifnum#1<0 %
    \BitSet@AfterFi{ #4}%
  \else
    \BitSet@AfterFi{%
      \expandafter\BitSet@GetSetBitList\number
      \expandafter\expandafter\expandafter\BitSet@NextSetBit
      \IntCalcInc#1!!{#2}!{#2},#4#3#1!%
    }%
  \BitSet@Fi
}
%    \end{macrocode}
%    \end{macro}
%
% \subsection{Bit set properties}\label{src:prop}
%
%    \begin{macro}{\bitsetSize}
%    \begin{macrocode}
\def\bitsetSize#1{%
  \number
  \BitSet@IfUndefined{#1}{0 }{%
    \expandafter\expandafter\expandafter\BitSet@Size
    \expandafter\expandafter\expandafter1%
    \expandafter\expandafter\expandafter!%
    \csname BS@#1\endcsname!0!%
  }%
}
%    \end{macrocode}
%    \end{macro}
%    \begin{macro}{\BitSet@Size}
%    |#1|: counter\\
%    |#2#3|: bits\\
%    |#4|: result
%    \begin{macrocode}
\def\BitSet@Size#1!#2#3!#4!{%
  \ifx#21%
    \ifx\\#3\\%
      \BitSet@AfterFiFi{#1 }%
    \else
      \BitSet@AfterFiFi{%
        \expandafter\expandafter\expandafter\BitSet@Size
        \IntCalcInc#1!!#3!#1!%
      }%
    \fi
  \else
    \ifx\\#3\\%
      \BitSet@AfterFiFi{#4 }%
    \else
      \BitSet@AfterFiFi{%
        \expandafter\expandafter\expandafter\BitSet@Size
        \IntCalcInc#1!!#3!#4!%
      }%
    \fi
  \fi
  \BitSet@Fi
}
%    \end{macrocode}
%    \end{macro}
%
%    \begin{macro}{\bitsetCardinality}
%    \begin{macrocode}
\def\bitsetCardinality#1{%
  \number
  \BitSet@IfUndefined{#1}{0 }{%
    \expandafter\expandafter\expandafter\BitSet@Cardinality
    \expandafter\expandafter\expandafter0%
    \expandafter\expandafter\expandafter!%
    \csname BS@#1\endcsname!%
  }%
}
%    \end{macrocode}
%    \end{macro}
%    \begin{macro}{\BitSet@Cardinality}
%    |#1|: result\\
%    |#2#3|: bits
%    \begin{macrocode}
\def\BitSet@Cardinality#1!#2#3!{%
  \ifx#21%
    \ifx\\#3\\%
      \BitSet@AfterFiFi{\IntCalcInc#1! }%
    \else
      \BitSet@AfterFiFi{%
        \expandafter\expandafter\expandafter\BitSet@Cardinality
        \IntCalcInc#1!!#3!%
      }%
    \fi
  \else
    \ifx\\#3\\%
      \BitSet@AfterFiFi{#1 }%
    \else
      \BitSet@AfterFiFi{%
        \BitSet@Cardinality#1!#3!%
      }%
    \fi
  \fi
  \BitSet@Fi
}
%    \end{macrocode}
%    \end{macro}
%
% \subsection{Queries}\label{src:query}
%
%    \begin{macro}{\bitsetIsDefined}
%    \begin{macrocode}
\def\bitsetIsDefined#1{%
  \BitSet@IfUndefined{#1}%
  \BitSet@SecondOfTwo
  \BitSet@FirstOfTwo
}
%    \end{macrocode}
%    \end{macro}
%    \begin{macro}{\bitsetIsEmpty}
%    \begin{macrocode}
\def\bitsetIsEmpty#1{%
  \BitSet@IfUndefined{#1}\BitSet@FirstOfTwo{%
    \expandafter\ifx\csname BS@#1\endcsname\BitSet@Zero
      \expandafter\BitSet@FirstOfTwo
    \else
      \expandafter\BitSet@SecondOfTwo
    \fi
  }%
}
%    \end{macrocode}
%    \end{macro}
%    \begin{macro}{\BitSet@Zero}
%    \begin{macrocode}
\def\BitSet@Zero{0}
%    \end{macrocode}
%    \end{macro}
%
%    \begin{macro}{\bitsetQuery}
%    \begin{macrocode}
\def\bitsetQuery#1#2{%
  \ifnum\bitsetGet{#1}{#2}=1 %
    \expandafter\BitSet@FirstOfTwo
  \else
    \expandafter\BitSet@SecondOfTwo
  \fi
}
%    \end{macrocode}
%    \end{macro}
%
%    \begin{macro}{\bitsetEquals}
%    \begin{macrocode}
\def\bitsetEquals#1#2{%
  \BitSet@IfUndefined{#1}{%
    \BitSet@IfUndefined{#2}\BitSet@FirstOfTwo\BitSet@SecondOfTwo
  }{%
    \BitSet@IfUndefined{#2}\BitSet@SecondOfTwo{%
      \expandafter\ifx\csname BS@#1\expandafter\endcsname
                      \csname BS@#2\endcsname
        \expandafter\BitSet@FirstOfTwo
      \else
        \expandafter\BitSet@SecondOfTwo
      \fi
    }%
  }%
}
%    \end{macrocode}
%    \end{macro}
%
%    \begin{macro}{\bitsetIntersects}
%    \begin{macrocode}
\def\bitsetIntersects#1#2{%
  \bitsetIsEmpty{#1}\BitSet@SecondOfTwo{%
    \bitsetIsEmpty{#2}\BitSet@SecondOfTwo{%
      \expandafter\expandafter\expandafter\BitSet@Intersects
      \csname BS@#1\expandafter\expandafter\expandafter\endcsname
      \expandafter\expandafter\expandafter!%
      \csname BS@#2\endcsname!%
    }%
  }%
}
%    \end{macrocode}
%    \end{macro}
%    \begin{macro}{\BitSet@Intersects}
%    \begin{macrocode}
\def\BitSet@Intersects#1#2!#3#4!{%
  \ifnum#1#3=11 %
    \BitSet@AfterFi\BitSet@FirstOfTwo
  \else
    \ifx\\#2\\%
      \BitSet@AfterFiFi\BitSet@SecondOfTwo
    \else
      \ifx\\#4\\%
        \BitSet@AfterFiFiFi\BitSet@SecondOfTwo
      \else
        \BitSet@AfterFiFiFi{%
          \BitSet@Intersects#2!#4!%
        }%
      \fi
    \fi
  \BitSet@Fi
}
%    \end{macrocode}
%    \end{macro}
%
%    \begin{macrocode}
\BitSet@AtEnd%
%</package>
%    \end{macrocode}
%
% \section{Test}
%
% \subsection{Catcode checks for loading}
%
%    \begin{macrocode}
%<*test1>
%    \end{macrocode}
%    \begin{macrocode}
\catcode`\{=1 %
\catcode`\}=2 %
\catcode`\#=6 %
\catcode`\@=11 %
\expandafter\ifx\csname count@\endcsname\relax
  \countdef\count@=255 %
\fi
\expandafter\ifx\csname @gobble\endcsname\relax
  \long\def\@gobble#1{}%
\fi
\expandafter\ifx\csname @firstofone\endcsname\relax
  \long\def\@firstofone#1{#1}%
\fi
\expandafter\ifx\csname loop\endcsname\relax
  \expandafter\@firstofone
\else
  \expandafter\@gobble
\fi
{%
  \def\loop#1\repeat{%
    \def\body{#1}%
    \iterate
  }%
  \def\iterate{%
    \body
      \let\next\iterate
    \else
      \let\next\relax
    \fi
    \next
  }%
  \let\repeat=\fi
}%
\def\RestoreCatcodes{}
\count@=0 %
\loop
  \edef\RestoreCatcodes{%
    \RestoreCatcodes
    \catcode\the\count@=\the\catcode\count@\relax
  }%
\ifnum\count@<255 %
  \advance\count@ 1 %
\repeat

\def\RangeCatcodeInvalid#1#2{%
  \count@=#1\relax
  \loop
    \catcode\count@=15 %
  \ifnum\count@<#2\relax
    \advance\count@ 1 %
  \repeat
}
\def\RangeCatcodeCheck#1#2#3{%
  \count@=#1\relax
  \loop
    \ifnum#3=\catcode\count@
    \else
      \errmessage{%
        Character \the\count@\space
        with wrong catcode \the\catcode\count@\space
        instead of \number#3%
      }%
    \fi
  \ifnum\count@<#2\relax
    \advance\count@ 1 %
  \repeat
}
\def\space{ }
\expandafter\ifx\csname LoadCommand\endcsname\relax
  \def\LoadCommand{\input bitset.sty\relax}%
\fi
\def\Test{%
  \RangeCatcodeInvalid{0}{47}%
  \RangeCatcodeInvalid{58}{64}%
  \RangeCatcodeInvalid{91}{96}%
  \RangeCatcodeInvalid{123}{255}%
  \catcode`\@=12 %
  \catcode`\\=0 %
  \catcode`\%=14 %
  \LoadCommand
  \RangeCatcodeCheck{0}{36}{15}%
  \RangeCatcodeCheck{37}{37}{14}%
  \RangeCatcodeCheck{38}{47}{15}%
  \RangeCatcodeCheck{48}{57}{12}%
  \RangeCatcodeCheck{58}{63}{15}%
  \RangeCatcodeCheck{64}{64}{12}%
  \RangeCatcodeCheck{65}{90}{11}%
  \RangeCatcodeCheck{91}{91}{15}%
  \RangeCatcodeCheck{92}{92}{0}%
  \RangeCatcodeCheck{93}{96}{15}%
  \RangeCatcodeCheck{97}{122}{11}%
  \RangeCatcodeCheck{123}{255}{15}%
  \RestoreCatcodes
}
\Test
\csname @@end\endcsname
\end
%    \end{macrocode}
%    \begin{macrocode}
%</test1>
%    \end{macrocode}
%
% \subsection{Macro tests}
%
% \subsubsection{Preamble}
%
%    \begin{macrocode}
%<*test2>
\NeedsTeXFormat{LaTeX2e}
\nofiles
\documentclass{article}
\makeatletter
%<*noetex>
\let\SavedNumexpr\numexpr
\let\SavedIfcsname\ifcsname
\let\SavedCurrentgrouplevel\currentgrouplevel
\def\ETeXDisable{%
  \let\ifcsname\@undefined
  \let\numexpr\@undefined
  \let\currentgrouplevel\@undefined
}
\ETeXDisable
%</noetex>
\makeatletter
\chardef\BitSet@TestMode=1 %
\makeatother
\usepackage{bitset}[2011/01/30]
%<*noetex>
\def\ETeXEnable{%
  \let\numexpr\SavedNumexpr
  \let\ifcsname\SavedIfcsname
  \let\currentgrouplevel\SavedCurrentgrouplevel
}
\ETeXEnable
%</noetex>
\usepackage{qstest}
\IncludeTests{*}
\LogTests{log}{*}{*}
\makeatletter
%    \end{macrocode}
%
% \subsubsection{Time}
%
%    \begin{macrocode}
\begingroup\expandafter\expandafter\expandafter\endgroup
\expandafter\ifx\csname pdfresettimer\endcsname\relax
\else
  \newcount\SummaryTime
  \newcount\TestTime
  \SummaryTime=\z@
  \newcommand*{\PrintTime}[2]{%
    \typeout{%
      [Time #1: \strip@pt\dimexpr\number#2sp\relax\space s]%
    }%
  }%
  \newcommand*{\StartTime}[1]{%
    \renewcommand*{\TimeDescription}{#1}%
    \pdfresettimer
  }%
  \newcommand*{\TimeDescription}{}%
  \newcommand*{\StopTime}{%
    \TestTime=\pdfelapsedtime
    \global\advance\SummaryTime\TestTime
    \PrintTime\TimeDescription\TestTime
  }%
  \let\saved@qstest\qstest
  \let\saved@endqstest\endqstest
  \def\qstest#1#2{%
    \saved@qstest{#1}{#2}%
    \StartTime{#1}%
  }%
  \def\endqstest{%
    \StopTime
    \saved@endqstest
  }%
  \AtEndDocument{%
    \PrintTime{summary}\SummaryTime
  }%
\fi
%    \end{macrocode}
%
% \subsubsection{Detection of unwanted space}
%
%    \begin{macrocode}
\let\orig@qstest\qstest
\let\orig@endqstest\endqstest
\def\qstest#1#2{%
  \orig@qstest{#1}{#2}%
  \setbox0\hbox\bgroup\begingroup\ignorespaces
}
\def\endqstest{%
  \endgroup\egroup
  \Expect*{\the\wd0}{0.0pt}%
  \orig@endqstest
}
%    \end{macrocode}
%
% \subsubsection{Test macros}
%
%    \begin{macrocode}
\newcounter{Test}

\def\TestError#1#2{%
  \begingroup
    \setcounter{Test}{0}%
    \sbox0{%
      \def\@PackageError##1##2##3{%
        \stepcounter{Test}%
        \begingroup
          \let\MessageBreak\relax
%<*noetex>
          \ETeXEnable
%</noetex>
          \Expect{##1}{bitset}%
          \Expect*{##2}*{#1}%
        \endgroup
      }%
%<*noetex>
      \ETeXDisable
%</noetex>
      #2%
    }%
    \Expect*{\theTest}{1}%
    \Expect*{\the\wd0}{0.0pt}%
  \endgroup
}

\def\TestErrorNegativeIndex#1#2{%
  \TestError{Invalid negative index (#1)}{#2}%
}

\def\TestGetterUndefined#1{%
  \CheckUndef{dummy}%
  \expandafter\expandafter\expandafter\Expect
  \expandafter\expandafter\expandafter{#1{dummy}}{0}%
}

\def\ExpectBitSet#1#2{%
  \expandafter\expandafter\expandafter\Expect
  \expandafter\expandafter\expandafter
  {\csname BS@#1\endcsname}*{#2}%
}
\def\Check#1#2{%
  \ExpectBitSet{#1}{#2}%
}
\def\CheckUndef#1{%
  \begingroup
    \Expect*{%
      \expandafter
      \ifx\csname BS@#1\endcsname\relax true\else false\fi
    }{true}%
  \endgroup
}
\def\RevCheck#1#2{%
  \ExpectBitSet{#1}{\Reverse#2!!}%
}
\def\Set#1#2{%
  \expandafter\def\csname BS@#1\endcsname{#2}%
}
\def\RevSet#1#2{%
  \expandafter\edef\csname BS@#1\endcsname{%
    \Reverse#2!!%
  }%
}
\def\Reverse#1#2!#3!{%
  \ifx\\#2\\%
    #1#3%
    \expandafter\@gobble
  \else
    \expandafter\@firstofone
  \fi
  {\Reverse#2!#1#3!}%
}
%    \end{macrocode}
%
% \subsubsection{Test sets}
%
%    \begin{macrocode}
\begin{qstest}{Let}{Let}
  \CheckUndef{abc}%
  \CheckUndef{xyz}%
  \bitsetLet{xyz}{abc}%
  \CheckUndef{abc}%
  \Check{xyz}{0}%
  \Set{abc}{1}%
  \Check{abc}{1}%
  \Check{xyz}{0}%
  \bitsetLet{xyz}{abc}%
  \Check{abc}{1}%
  \Check{xyz}{1}%
  \Set{xyz}{11}%
  \Check{abc}{1}%
  \Check{xyz}{11}%
\end{qstest}

\begin{qstest}{Reset}{Reset}
  \bitsetReset{xyz}%
  \Check{xyz}{0}%
  \bitsetReset{abc}%
  \Check{abc}{0}%
  \Set{abc}{10101}%
  \bitsetReset{abc}%
  \Check{abc}{0}%
\end{qstest}

\begin{qstest}{Get/Query}{Get/Query}
  \expandafter\expandafter\expandafter\Expect
  \expandafter\expandafter\expandafter{%
    \bitsetGet{dummy}{0}%
  }{0}%
  \begingroup
    \expandafter\def\csname BitSetError:NegativeIndex\endcsname{}%
    \Set{abc}{1}%
    \Expect*{\bitsetQuery{abc}{-1}{true}{false}}{false}%
  \endgroup
  \def\Test#1#2#3{%
    \Set{abc}{#1}%
    \expandafter\expandafter\expandafter\Expect
    \expandafter\expandafter\expandafter{\bitsetGet{abc}{#2}}{#3}%
    \Expect*{\bitsetQuery{abc}{#2}{true}{false}}%
           *{\ifcase#3 false\or true\else error\fi}%
  }%
  \Test{1}{100}{0}%
  \Test{0}{0}{0}%
  \Test{1}{0}{1}%
  \Test{11}{1}{1}%
  \Test{111}{1}{1}%
  \Test{101}{1}{0}%
  \Test{101}{2}{1}%
  \Test{10100110011}{10}{1}%
\end{qstest}

\begin{qstest}{Size}{Size}
  \TestGetterUndefined\bitsetSize
  \def\Test#1#2{%
    \Set{abc}{#1}%
    \expandafter\expandafter\expandafter\Expect
    \expandafter\expandafter\expandafter{\bitsetSize{abc}}{#2}%
  }%
  \Test{0}{0}%
  \Test{1}{1}%
  \Test{00}{0}%
  \Test{0000000}{0}%
  \Test{10}{1}%
  \Test{01}{2}%
  \Test{11}{2}%
  \Test{010}{2}%
  \Test{011}{3}%
  \Test{100110011}{9}%
  \Test{0000011111000001111100000}{20}%
  \Test{000000000000000000000000011111111111111111111}{45}%
\end{qstest}

\begin{qstest}{Cardinality}{Cardinality}
  \TestGetterUndefined\bitsetCardinality
  \def\Test#1#2{%
    \Set{abc}{#1}%
    \expandafter\expandafter\expandafter\Expect
    \expandafter\expandafter\expandafter{%
      \bitsetCardinality{abc}%
    }{#2}%
  }%
  \Test{0}{0}%
  \Test{1}{1}%
  \Test{00}{0}%
  \Test{0000000}{0}%
  \Test{10}{1}%
  \Test{01}{1}%
  \Test{11}{2}%
  \Test{010}{1}%
  \Test{011}{2}%
  \Test{100110011}{5}%
  \Test{0000011111000001111100000}{10}%
  \Test{000000000000000000000000011111111111111111111}{20}%
\end{qstest}

\begin{qstest}{NextClearBit/NextSetBit}{NextClearBit/NextSetBit}
  \def\Test#1#2{%
    \expandafter\expandafter\expandafter\Expect
    \expandafter\expandafter\expandafter{%
      \TestOp{abc}{#1}%
    }{#2}%
  }%
  \def\Clear{\let\TestOp\bitsetNextClearBit}%
  \def\Set{\let\TestOp\bitsetNextSetBit}%
  \begingroup
    \catcode`\:=11 %
    \bitsetSetBin{abc}{1}%
    \Clear
    \Test{-1}{1\BitSetError:NegativeIndex}%
    \Set
    \Test{-1}{0\BitSetError:NegativeIndex}%
  \endgroup
  \let\BS@abc\@undefined
  \Clear
  \Test{0}{0}%
  \Test{1}{1}%
  \Test{2}{2}%
  \Test{100}{100}%
  \Set
  \Test{0}{-1}%
  \Test{1}{-1}%
  \Test{100}{-1}%
  \bitsetReset{abc}%
  \Clear
  \Test{0}{0}%
  \Test{1}{1}%
  \Test{2}{2}%
  \Test{100}{100}%
  \Set
  \Test{0}{-1}%
  \Test{1}{-1}%
  \Test{100}{-1}%
  \bitsetSetBin{abc}{1}%
  \Clear
  \Test{0}{1}%
  \Test{1}{1}%
  \Test{2}{2}%
  \Test{100}{100}%
  \Set
  \Test{0}{0}%
  \Test{1}{-1}%
  \Test{100}{-1}%
  \bitsetSetBin{abc}{111000111000111000111}%
  \Clear
  \Test{0}{3}%
  \Test{1}{3}%
  \Test{2}{3}%
  \Test{3}{3}%
  \Test{4}{4}%
  \Test{5}{5}%
  \Test{6}{9}%
  \Test{7}{9}%
  \Test{8}{9}%
  \Test{9}{9}%
  \Test{10}{10}%
  \Test{11}{11}%
  \Test{12}{15}%
  \Test{13}{15}%
  \Test{14}{15}%
  \Test{15}{15}%
  \Test{16}{16}%
  \Test{17}{17}%
  \Test{18}{21}%
  \Test{19}{21}%
  \Test{20}{21}%
  \Test{21}{21}%
  \Test{22}{22}%
  \Test{100}{100}%
  \Set
  \Test{0}{0}%
  \Test{1}{1}%
  \Test{2}{2}%
  \Test{3}{6}%
  \Test{4}{6}%
  \Test{5}{6}%
  \Test{6}{6}%
  \Test{7}{7}%
  \Test{8}{8}%
  \Test{9}{12}%
  \Test{10}{12}%
  \Test{11}{12}%
  \Test{12}{12}%
  \Test{13}{13}%
  \Test{14}{14}%
  \Test{15}{18}%
  \Test{16}{18}%
  \Test{17}{18}%
  \Test{18}{18}%
  \Test{19}{19}%
  \Test{20}{20}%
  \Test{21}{-1}%
  \Test{22}{-1}%
  \Test{100}{-1}%
  \bitsetSetBin{abc}{1111111}%
  \Clear
  \Test{6}{7}%
  \Test{7}{7}%
  \Test{8}{8}%
  \Test{100}{100}%
  \Set
  \Test{6}{6}%
  \Test{7}{-1}%
  \Test{8}{-1}%
  \Test{100}{-1}%
  \bitsetSetBin{abc}{11111111}%
  \Clear
  \Test{7}{8}%
  \Test{8}{8}%
  \Test{9}{9}%
  \Test{100}{100}%
  \Set
  \Test{7}{7}%
  \Test{8}{-1}%
  \Test{9}{-1}%
  \Test{100}{-1}%
  \bitsetSetBin{abc}{111111111}%
  \Clear
  \Test{8}{9}%
  \Test{9}{9}%
  \Test{10}{10}%
  \Test{100}{100}%
  \Set
  \Test{8}{8}%
  \Test{9}{-1}%
  \Test{10}{-1}%
  \Test{100}{-1}%
  \bitsetSetBin{abc}{1111111111}%
  \Clear
  \Test{9}{10}%
  \Test{10}{10}%
  \Test{11}{11}%
  \Test{100}{100}%
  \Set
  \Test{9}{9}%
  \Test{10}{-1}%
  \Test{11}{-1}%
  \Test{100}{-1}%
\end{qstest}

\begin{qstest}{GetSetBitList}{GetSetBitList}
  \let\BS@abc\@undefined
  \expandafter\expandafter\expandafter\Expect
  \expandafter\expandafter\expandafter{%
    \bitsetGetSetBitList{abc}%
  }{}%
  \def\Test#1#2{%
    \bitsetSetBin{abc}{#1}%
    \expandafter\expandafter\expandafter\Expect
    \expandafter\expandafter\expandafter{%
      \bitsetGetSetBitList{abc}%
    }{#2}%
  }%
  \Test{0}{}%
  \Test{1}{0}%
  \Test{10}{1}%
  \Test{11}{0,1}%
  \Test{10110100}{2,4,5,7}%
  \Test{101101001010011}{0,1,4,6,9,11,12,14}%
\end{qstest}

\begin{qstest}{GetDec}{GetDec}
  \TestGetterUndefined\bitsetGetDec
  \def\Test#1#2{%
    \RevSet{abc}{#1}%
%<*noetex>
    \begingroup\expandafter\expandafter\expandafter\endgroup
%</noetex>
    \expandafter\expandafter\expandafter\Expect
    \expandafter\expandafter\expandafter{%
      \bitsetGetDec{abc}%
    }{#2}%
  }%
  \Test{0}{0}%
  \Test{1}{1}%
  \Test{10}{2}%
  \Test{11}{3}%
  \Test{100}{4}%
  \Test{101}{5}%
  \Test{110}{6}%
  \Test{111}{7}%
  \Test{1000}{8}%
  \Test{000111}{7}%
  \Test{1111111111111111%
        111111111111111}{2147483647}%
  \Test{0001111111111111111%
        111111111111111}{2147483647}%
  \Test{1000000000000000%
        0000000000000000}{2147483648}%
  \Test{1000000000000000%
        00000000000000000}{4294967296}%
  \Test{0001000000000000000%
        00000000000000000}{4294967296}%
  \Test{1100000000000000%
        00000000000000011}{6442450947}%
\end{qstest}

\begin{qstest}{Clear}{Clear}
  \def\Test#1#2#3{%
    \RevSet{abc}{#1}%
    \bitsetClear{abc}{#2}%
    \Expect*{\BS@abc}*{\Reverse#3!!}%
  }%
  \bitsetClear{abc}{2}%
  \RevCheck{abc}{0}%
  \TestErrorNegativeIndex{-1}{\bitsetClear{abc}{-1}}%
  \RevCheck{abc}{0}%
  \Test{0}{0}{0}%
  \Test{1}{0}{0}%
  \Test{111}{1}{101}%
  \Test{111}{30}{111}%
  \Test{0000111}{5}{0000111}% 111 would also be ok
  \Test{10000111}{5}{10000111}%
  \Test{1001001}{3}{1000001}%
\end{qstest}

\begin{qstest}{Set}{Set}
  \def\Test#1#2#3{%
    \RevSet{abc}{#1}%
    \bitsetSet{abc}{#2}%
    \Expect*{\BS@abc}*{\Reverse#3!!}%
  }%
  \bitsetSet{abc}{2}%
  \RevCheck{abc}{100}%
  \TestErrorNegativeIndex{-1}{\bitsetSet{abc}{-1}}%
  \RevCheck{abc}{100}%
  \Test{0}{0}{1}%
  \Test{1}{0}{1}%
  \Test{100}{1}{110}%
  \Test{111}{1}{111}%
  \Test{11}{1}{11}%
  \Test{11}{2}{111}%
  \Test{11}{3}{1011}%
  \Test{111}{10}{10000000111}%
  \Test{0000111}{5}{0100111}% 100111 would also be ok
  \Test{10000111}{5}{10100111}%
  \Test{1000001}{3}{1001001}%
  \Test{1001001}{3}{1001001}%
\end{qstest}

\begin{qstest}{Flip}{Flip}
  \def\Test#1#2#3{%
    \RevSet{abc}{#1}%
    \bitsetFlip{abc}{#2}%
    \Expect*{\BS@abc}*{\Reverse#3!!}%
  }%
  \bitsetFlip{abc}{2}%
  \RevCheck{abc}{100}%
  \TestErrorNegativeIndex{-1}{\bitsetFlip{abc}{-1}}%
  \RevCheck{abc}{100}%
  \Test{0}{0}{1}%
  \Test{1}{0}{0}%
  \Test{0}{2}{100}%
  \Test{100}{1}{110}%
  \Test{111}{1}{101}%
  \Test{11}{1}{1}%
  \Test{11}{2}{111}%
  \Test{11}{3}{1011}%
  \Test{111}{10}{10000000111}%
  \Test{0000111}{5}{0100111}% 100111 would also be ok
  \Test{10000111}{5}{10100111}%
  \Test{1000001}{3}{1001001}%
  \Test{1001001}{3}{1000001}%
  \Test{11111}{2}{11011}%
\end{qstest}

\begin{qstest}{SetValue}{SetValue}
  \def\Test#1#2{%
    \TestError{Invalid bit value (#2) not in range 0..1}{%
      \bitsetSetValue{abc}{#1}{#2}%
    }%
  }%
  \Test{0}{-1}%
  \Test{0}{2}%
  \Test{0}{10}%
  \def\Test#1#2#3{%
    \let\BS@abc\@undefined
    \bitsetSetValue{abc}{#1}{#2}%
    \bitsetSetBin{result}{#3}%
    \Expect*{\BS@abc}*{\BS@result}%
  }%
  \Test{0}{0}{0}%
  \Test{0}{1}{1}%
  \Test{1}{0}{0}%
  \Test{1}{1}{10}%
  \def\Test#1#2#3#4{%
    \bitsetSetBin{abc}{#1}%
    \bitsetSetBin{result}{#4}%
    \bitsetSetValue{abc}{#2}{#3}%
    \Expect*{\BS@abc}*{\BS@result}%
  }%
  \Test{0}{0}{0}{0}%
  \Test{0}{0}{0}{0}%
  \Test{0}{0}{1}{1}%
  \Test{0}{1}{0}{0}%
  \Test{0}{1}{1}{10}%
  \Test{1010}{2}{1}{1110}%
  \Test{1010}{4}{1}{11010}%
  \Test{1010}{6}{1}{1001010}%
  \Test{1010}{1}{0}{1000}%
  \Test{1010}{2}{0}{1010}%
  \Test{1010}{3}{0}{10}%
  \Test{1010}{4}{0}{1010}%
  \Test{1010}{6}{0}{1010}%
  \Test{1010}{2}{\csname iffalse\endcsname 0\else 1\fi}{1110}%
  \Test{1010}{1}{\csname iffalse\endcsname 1\else 0\fi}{1000}%
\end{qstest}

\begin{qstest}{IsDefined}{IsDefined}
  \let\BS@abc\@undefined
  \Expect*{\bitsetIsDefined{abc}{true}{false}}{false}%
  \bitsetReset{abc}%
  \Expect*{\bitsetIsDefined{abc}{true}{false}}{true}%
\end{qstest}

\begin{qstest}{IsEmpty}{IsEmpty}
  \let\BS@abc\@undefined
  \Expect*{\bitsetIsEmpty{abc}{true}{false}}{true}%
  \bitsetReset{abc}%
  \Expect*{\bitsetIsEmpty{abc}{true}{false}}{true}%
  \bitsetSet{abc}{1}%
  \Expect*{\bitsetIsEmpty{abc}{true}{false}}{false}%
\end{qstest}

\begin{qstest}{Equals}{Equals}
  \def\Test#1#2#3{%
    \Expect*{\bitsetEquals{#1}{#2}{true}{false}}{#3}%
  }%
  \let\BS@abc\@undefined
  \Test{abc}{abc}{true}%
  \Test{abc}{foo}{true}%
  \Test{foo}{abc}{true}%
  \bitsetReset{abc}%
  \Test{abc}{abc}{true}%
  \Test{abc}{foo}{false}%
  \Test{foo}{abc}{false}%
  \bitsetReset{foo}%
  \Test{abc}{foo}{true}%
  \Test{foo}{abc}{true}%
  \bitsetSet{abc}{4}%
  \Test{abc}{foo}{false}%
  \Test{foo}{abc}{false}%
  \bitsetFlip{foo}{4}%
  \Test{abc}{foo}{true}%
  \Test{foo}{abc}{true}%
\end{qstest}

\begin{qstest}{Intersects}{Intersects}
  \def\Test#1{%
    \Expect*{\bitsetIntersects{abc}{foo}{true}{false}}{#1}%
  }%
  \let\BS@abc\@undefined
  \let\BS@foo\@undefined
  \Test{false}%
  \Set{abc}{0}%
  \Test{false}%
  \Set{foo}{0}%
  \Test{false}%
  \let\BS@abc\@undefined
  \Test{false}%
  \Set{foo}{1}%
  \Test{false}%
  \Set{abc}{0}%
  \Test{false}%
  \Set{abc}{1}%
  \Test{true}%
  \let\BS@foo\@undefined
  \Test{false}%
  \Set{foo}{0}%
  \Test{false}%
  \def\Test#1#2#3{%
    \bitsetSetBin{abc}{#1}%
    \bitsetSetBin{foo}{#2}%
    \Expect*{\bitsetIntersects{abc}{foo}{true}{false}}{#3}%
  }%
  \Test{1010}{0101}{false}%
  \Test{0}{10}{false}%
  \Test{1}{11}{true}%
  \Test{11}{1}{true}%
  \Test{10}{1}{false}%
\end{qstest}

\begin{qstest}{And/AndNot/Or/Xor}{And/AndNot/Or/Xor}
  \def\@Test#1#2#3#4#5{%
    \begingroup
      #5%
      \begingroup
        \let\BS@foo\@undefined
        \csname bitset#1\endcsname{abc}{foo}%
        \CheckUndef{foo}%
        \Check{abc}{#2}%
      \endgroup
      \begingroup
        \bitsetReset{foo}%
        \csname bitset#1\endcsname{abc}{foo}%
        \Check{foo}{0}%
        \Check{abc}{#3}%
      \endgroup
      \begingroup
        \def\BS@foo{0101}%
        \csname bitset#1\endcsname{abc}{foo}%
        \Check{foo}{0101}%
        \Check{abc}{#4}%
      \endgroup
    \endgroup
  }%
  \def\Test#1{%
    \def\Op{#1}%
    \Test@
  }%
  \def\Test@#1#2#3#4#5#6#7#8#9{%
    \@Test\Op{#1}{#2}{#3}{%
      \let\BS@abc\@undefined
    }%
    \@Test\Op{#4}{#5}{#6}{%
      \bitsetReset{abc}%
    }%
    \@Test\Op{#7}{#8}{#9}{%
      \def\BS@abc{1001}%
    }%
  }%
  \Test{And}%
       {0}{0}{0}%
       {0}{0}{0}%
       {0}{0}{0001}%
  \Test{AndNot}%
       {0}{0}{0}%
       {0}{0}{0}%
       {1001}{1001}{1}%
  \Test{Or}%
       {0}{0}{0101}%
       {0}{0}{0101}%
       {1001}{1001}{1101}%
  \Test{Xor}%
       {0}{0}{0101}%
       {0}{0}{0101}%
       {1001}{1001}{11}%
  \def\Test#1#2#3{%
    \bitsetSetBin{abc}{#1}%
    \bitsetSetBin{foo}{#2}%
    \csname bitset\Op\endcsname{abc}{foo}%
    \RevCheck{foo}{#2}%
    \RevCheck{abc}{#3}%
  }%
  \def\Op{And}%
  \Test{1}{111}{1}%
  \Test{111}{1}{1}%
  \Test{10}{111}{10}%
  \Test{111}{10}{10}%
  \Test{111}{1000}{0}%
  \Test{1000}{111}{0}%
  \def\Op{AndNot}%
  \Test{1010}{11}{1000}%
  \Test{100}{100}{0}%
  \Test{111}{1111}{0}%
  \Test{100}{111}{0}%
  \def\Op{Or}%
  \Test{0}{0}{0}%
  \Test{1}{0}{1}%
  \Test{0}{1}{1}%
  \Test{1}{1}{1}%
  \Test{1000}{10}{1010}%
  \Test{10}{1000}{1010}%
  \def\Op{Xor}%
  \Test{0}{0}{0}%
  \Test{1}{0}{1}%
  \Test{0}{1}{1}%
  \Test{1}{1}{0}%
  \Test{1000}{10}{1010}%
  \Test{10}{1000}{1010}%
  \Test   {110011001100}%
       {111000111000111}%
       {111110100001011}%
  \Test{111000111000111}%
          {110011001100}%
       {111110100001011}%
\end{qstest}

\begin{qstest}{GetUndef}{GetUndef, GetBin, GetOct, GetHex}
  \def\TestUndef#1#2{%
    \let\BS@abc\@undefined
    \expandafter\expandafter\expandafter\Expect
    \expandafter\expandafter\expandafter{%
      \x{abc}{#1}%
    }{#2}%
  }%
  \let\x\bitsetGetBin
  \TestUndef{-1}{0}%
  \TestUndef{0}{0}%
  \TestUndef{1}{0}%
  \TestUndef{2}{00}%
  \TestUndef{8}{00000000}%
  \let\x\bitsetGetOct
  \TestUndef{-1}{0}%
  \TestUndef{0}{0}%
  \TestUndef{1}{0}%
  \TestUndef{2}{0}%
  \TestUndef{3}{0}%
  \TestUndef{4}{00}%
  \TestUndef{5}{00}%
  \TestUndef{6}{00}%
  \TestUndef{7}{000}%
  \TestUndef{8}{000}%
  \TestUndef{9}{000}%
  \TestUndef{10}{0000}%
  \let\x\bitsetGetHex
  \TestUndef{-1}{0}%
  \TestUndef{0}{0}%
  \TestUndef{1}{0}%
  \TestUndef{2}{0}%
  \TestUndef{3}{0}%
  \TestUndef{4}{0}%
  \TestUndef{5}{00}%
  \TestUndef{6}{00}%
  \TestUndef{7}{00}%
  \TestUndef{8}{00}%
  \TestUndef{9}{000}%
  \TestUndef{10}{000}%
  \TestUndef{12}{000}%
  \TestUndef{13}{0000}%
  \TestUndef{16}{0000}%
  \TestUndef{17}{00000}%
\end{qstest}

\begin{qstest}{SetBin}{SetBin}
  \def\Test#1#2{%
    \let\BS@abc\@undefined
    \bitsetSetBin{abc}{#1}%
    \expandafter\Expect\expandafter{\BS@abc}{#2}%
  }%
  \Test{}{0}%
  \Test{0}{0}%
  \Test{1}{1}%
  \Test{10}{01}%
  \Test{11}{11}%
  \Test{010}{01}%
  \Test{011}{11}%
  \Test{0010}{01}%
  \Test{1010}{0101}%
\end{qstest}

\begin{qstest}{SetOct}{SetOct}
  \def\Test#1#2{%
    \bitsetSetOct{abc}{#1}%
    \expandafter\Expect\expandafter{\BS@abc}{#2}%
  }%
  \Test{}{0}%
  \Test{0}{0}%
  \Test{000}{0}%
  \Test{1}{1}%
  \Test{001}{1}%
  \Test{010}{0001}%
  \Test{020}{00001}%
  \Test{42}{010001}%
  \Test{377}{11111111}%
  \Test{0377}{11111111}%
  \Test{76543210}{000100010110001101011111}%
  \Test{ 0 7 0 7 1 }{100111000111}%
\end{qstest}

\begin{qstest}{SetHex}{SetHex}
  \def\Test#1#2{%
    \bitsetSetHex{abc}{#1}%
    \expandafter\Expect\expandafter{\BS@abc}{#2}%
  }%
  \Test{}{0}%
  \Test{0}{0}%
  \Test{000}{0}%
  \Test{1}{1}%
  \Test{001}{1}%
  \Test{010}{00001}%
  \Test{020}{000001}%
  \Test{42}{0100001}%
  \Test{3F}{111111}%
  \Test{03F}{111111}%
  \Test{43210}{0000100001001100001}%
  \Test{98765}{10100110111000011001}%
  \Test{FEDCBA}{010111010011101101111111}%
  \Test{ 0 F 0 F 1 }{1000111100001111}%
\end{qstest}

\begin{qstest}{SetDec}{SetDec}
  \def\Test#1#2{%
    \bitsetSetDec{abc}{#1}%
    \expandafter\Expect\expandafter{\BS@abc}{#2}%
  }%
  \Test{}{0}%
  \Test{0}{0}%
  \Test{000}{0}%
  \Test{1}{1}%
  \Test{7}{111}%
  \Test{8}{0001}%
  \Test{001}{1}%
  \Test{010}{0101}%
  \Test{020}{00101}%
  \Test{53}{101011}%
  \Test{255}{11111111}%
  \Test{256}{000000001}%
  \Test{999999999}{111111111001001101011001110111}%
  \Test{1000000000}{000000000101001101011001110111}%
  \Test{4210987654}{01100001010010010111111101011111}%
  \Test{2147483647}{1111111111111111111111111111111}%
  \Test{2147483648}{00000000000000000000000000000001}%
\end{qstest}

\begin{qstest}{GetBin}{GetBin}
  \def\TestUndef#1#2{%
    \let\BS@abc\@undefined
    \expandafter\expandafter\expandafter\Expect
    \expandafter\expandafter\expandafter{%
      \bitsetGetBin{abc}{#1}%
    }{#2}%
  }%
  \TestUndef{-1}{0}%
  \TestUndef{0}{0}%
  \TestUndef{1}{0}%
  \TestUndef{2}{00}%
  \TestUndef{8}{00000000}%
  \def\Test#1#2{%
    \bitsetSetBin{abc}{#2}%
    \expandafter\expandafter\expandafter\Expect
    \expandafter\expandafter\expandafter{%
      \bitsetGetBin{abc}{#1}%
    }{#2}%
  }%
  \Test{-1}{0}%
  \Test{0}{0}%
  \Test{1}{0}%
  \Test{1}{1}%
  \Test{2}{01}%
  \Test{2}{10}%
  \Test{3}{010}%
  \Test{2}{00}%
  \Test{2}{01}%
  \Test{8}{00101100}%
  \Test{2}{10101}%
  \Test{-100}{11011}%
\end{qstest}

\begin{qstest}{GetOct}{GetOct}
  \def\Test#1#2#3{%
    \edef\x{\zap@space#1 \@empty}%
    \edef\x{\noexpand\bitsetSetBin{abc}{\x}}%
    \x
    \expandafter\expandafter\expandafter\Expect
    \expandafter\expandafter\expandafter{%
      \bitsetGetOct{abc}{#2}%
    }{#3}%
  }%
  \Test{111 110 101 100 011 010 001 000}{0}{76543210}%
  \Test{000 111}{0}{7}%
  \Test{101 000}{-1}{50}%
  \Test{111}{-1}{7}%
  \Test{111}{0}{7}%
  \Test{111}{1}{7}%
  \Test{111}{3}{7}%
  \Test{111}{4}{07}%
  \Test{111}{6}{07}%
  \Test{111}{7}{007}%
  \Test{111 010}{6}{72}%
  \Test{111 010}{7}{072}%
  \Test{011 111}{0}{37}%
  \Test{011 111}{6}{37}%
  \Test{011 111}{7}{037}%
  \Test{001 111}{0}{17}%
  \Test{001 111}{6}{17}%
  \Test{001 111}{7}{017}%
\end{qstest}

\begin{qstest}{GetHex}{GetHex}
  \def\Test#1#2#3{%
    \bitsetSetBin{abc}{#1}%
    \expandafter\expandafter\expandafter\Expect
    \expandafter\expandafter\expandafter{%
      \bitsetGetHex{abc}{#2}%
    }{#3}%
  }%
  \Test{1111 1110 1101 1100 1011 1010 1001 1000}{0}{FEDCBA98}%
  \Test{0111 0110 0101 0100 0011 0010 0001 0000}{0}{76543210}%
  \Test{0000 1111}{0}{F}%
  \Test{0101 0000}{-1}{50}%
  \Test{1111}{-1}{F}%
  \Test{1111}{0}{F}%
  \Test{1111}{1}{F}%
  \Test{1111}{4}{F}%
  \Test{1111}{5}{0F}%
  \Test{1111}{8}{0F}%
  \Test{1111}{9}{00F}%
  \Test{1111 0010}{8}{F2}%
  \Test{1111 0010}{9}{0F2}%
  \Test{0111 1111}{0}{7F}%
  \Test{0111 1111}{8}{7F}%
  \Test{0111 1111}{9}{07F}%
  \Test{0011 1111}{0}{3F}%
  \Test{0011 1111}{8}{3F}%
  \Test{0011 1111}{9}{03F}%
  \Test{0001 1111}{0}{1F}%
  \Test{0001 1111}{8}{1F}%
  \Test{0001 1111}{9}{01F}%
\end{qstest}

\begin{qstest}{Range}{Range}
  \TestError{%
    Wrong index numbers in range [9..8]\MessageBreak% hash-ok
    for clear/set/flip on bit set `abc'.\MessageBreak
    The lower index exceeds the upper index.\MessageBreak
    Canceling the operation as error recovery%
  }{%
    \bitsetSetRange{abc}{9}{8}%
  }%
  \def\TestErrorNegInd#1#2#3#4#5#6{%
    \TestError{%
      Negative index in range [#2..#3]\MessageBreak % hash-ok
      for \string\bitset #1Range on bit set `abc'.\MessageBreak
      Using [#4..#5] as error recovery% hash-ok
    }{%
      \csname bitset#1Range\endcsname{abc}{#2}{#3}%
      \global\let\BS@global\BS@abc
    }%
    \Check{global}{#6}%
  }%
  \Set{abc}{111}%
  \TestErrorNegInd{Clear}{-1}{0}{0}{0}{111}%
  \TestErrorNegInd{Clear}{0}{-1}{0}{0}{111}%
  \TestErrorNegInd{Clear}{-2}{2}{0}{2}{001}%
  \bitsetReset{abc}%
  \TestErrorNegInd{Set}{-1}{0}{0}{0}{0}%
  \TestErrorNegInd{Set}{0}{-1}{0}{0}{0}%
  \TestErrorNegInd{Set}{-2}{2}{0}{2}{11}%
  \Set{abc}{101}%
  \TestErrorNegInd{Flip}{-1}{0}{0}{0}{101}%
  \TestErrorNegInd{Flip}{0}{-1}{0}{0}{101}%
  \TestErrorNegInd{Flip}{-2}{2}{0}{2}{011}%
  \def\Test#1#2#3#4{%
    \bitsetSetBin{abc}{#1}%
    \csname bitset\TestOp Range\endcsname{abc}{#2}{#3}%
    \Expect*{\bitsetGetBin{abc}{0}}{#4}%
  }%
  \def\TestOp{Clear}%
  \Test{0}{0}{1}{0}%
  \Test{1111}{1}{2}{1101}%
  \Test{1111}{1}{3}{1001}%
  \Test{1111111100000000}{12}{14}{1100111100000000}%
  \def\TestOp{Set}%
  \Test{0}{0}{1}{1}%
  \Test{1000}{1}{2}{1010}%
  \Test{0}{1}{2}{10}%
  \Test{1}{12}{15}{111000000000001}%
  \Test{1111}{1}{3}{1111}%
  \Test{1000000000000000}{12}{14}{1011000000000000}%
  \def\TestOp{Flip}%
  \Test{0}{0}{1}{1}%
  \Test{1}{0}{1}{0}%
  \Test{10101010}{1}{5}{10110100}%
  \def\Test#1#2#3#4#5{%
    \bitsetSetBin{abc}{#1}%
    \bitsetSetValueRange{abc}{#2}{#3}{#4}%
    \Expect*{\bitsetGetBin{abc}{0}}{#5}%
  }%
  \Test{0}{0}{1}{0}{0}%
  \Test{0}{0}{1}{1}{1}%
  \Test{1010}{1}{3}{0}{1000}%
  \Test{1010}{1}{3}{1}{1110}%
\end{qstest}

\begin{qstest}{ShiftLeft/ShiftRight}{ShiftLeft/ShiftRight}
  \def\@Test#1#2{%
    \let\BS@abc\@undefined
    \csname bitsetShift#1\endcsname{abc}{#2}%
    \Expect*{\BS@abc}{0}%
  }%
  \def\Test#1{%
    \@Test{Left}{#1}%
    \@Test{Right}{#1}%
  }%
  \Test{-16}%
  \Test{-1}%
  \Test{0}%
  \Test{1}%
  \Test{16}%
  \def\Test#1#2#3{%
    \bitsetSetBin{abc}{#1}%
    \bitsetSetBin{result}{#3}%
    \csname bitsetShift\Op\endcsname{abc}{#2}%
    \Expect*{\bitsetGetBin{abc}{0}}*{\bitsetGetBin{result}{0}}%
  }%
  \def\Op{Left}%
  \Test{0}{0}{0}%
  \Test{0}{1}{0}%
  \Test{0}{-1}{0}%
  \Test{1}{0}{1}%
  \Test{1}{1}{10}%
  \Test{1}{-1}{0}%
  \Test{10}{1}{100}%
  \Test{10}{-1}{1}%
  \Test{1}{32}{100000000000000000000000000000000}%
  \Test{1}{-100}{0}%
  \def\Op{Right}%
  \Test{0}{0}{0}%
  \Test{0}{1}{0}%
  \Test{0}{-1}{0}%
  \Test{1}{0}{1}%
  \Test{1}{1}{0}%
  \Test{1}{-1}{10}%
  \Test{10}{1}{1}%
  \Test{10}{-1}{100}%
  \Test{1}{-32}{100000000000000000000000000000000}%
  \Test{1}{100}{0}%
  \Test{110110110110110}{10}{11011}%
  \Test{110110110110110}{100}{0}%
  \Test{1}{100000}{0}%
\end{qstest}

\begin{qstest}{Profile: Set}{Profile: Set}
  \bitsetSet{abc}{4095}%
  \global\let\BS@global\BS@abc
\end{qstest}

\begin{qstest}{Profile: Get}{Profile: Get}
  \edef\x{\bitsetGet{global}{4095}}%
\end{qstest}

\begin{document}
\end{document}
%</test2>
%    \end{macrocode}
%
% \section{Installation}
%
% \subsection{Download}
%
% \paragraph{Package.} This package is available on
% CTAN\footnote{\url{ftp://ftp.ctan.org/tex-archive/}}:
% \begin{description}
% \item[\CTAN{macros/latex/contrib/oberdiek/bitset.dtx}] The source file.
% \item[\CTAN{macros/latex/contrib/oberdiek/bitset.pdf}] Documentation.
% \end{description}
%
%
% \paragraph{Bundle.} All the packages of the bundle `oberdiek'
% are also available in a TDS compliant ZIP archive. There
% the packages are already unpacked and the documentation files
% are generated. The files and directories obey the TDS standard.
% \begin{description}
% \item[\CTAN{install/macros/latex/contrib/oberdiek.tds.zip}]
% \end{description}
% \emph{TDS} refers to the standard ``A Directory Structure
% for \TeX\ Files'' (\CTAN{tds/tds.pdf}). Directories
% with \xfile{texmf} in their name are usually organized this way.
%
% \subsection{Bundle installation}
%
% \paragraph{Unpacking.} Unpack the \xfile{oberdiek.tds.zip} in the
% TDS tree (also known as \xfile{texmf} tree) of your choice.
% Example (linux):
% \begin{quote}
%   |unzip oberdiek.tds.zip -d ~/texmf|
% \end{quote}
%
% \paragraph{Script installation.}
% Check the directory \xfile{TDS:scripts/oberdiek/} for
% scripts that need further installation steps.
% Package \xpackage{attachfile2} comes with the Perl script
% \xfile{pdfatfi.pl} that should be installed in such a way
% that it can be called as \texttt{pdfatfi}.
% Example (linux):
% \begin{quote}
%   |chmod +x scripts/oberdiek/pdfatfi.pl|\\
%   |cp scripts/oberdiek/pdfatfi.pl /usr/local/bin/|
% \end{quote}
%
% \subsection{Package installation}
%
% \paragraph{Unpacking.} The \xfile{.dtx} file is a self-extracting
% \docstrip\ archive. The files are extracted by running the
% \xfile{.dtx} through \plainTeX:
% \begin{quote}
%   \verb|tex bitset.dtx|
% \end{quote}
%
% \paragraph{TDS.} Now the different files must be moved into
% the different directories in your installation TDS tree
% (also known as \xfile{texmf} tree):
% \begin{quote}
% \def\t{^^A
% \begin{tabular}{@{}>{\ttfamily}l@{ $\rightarrow$ }>{\ttfamily}l@{}}
%   bitset.sty & tex/generic/oberdiek/bitset.sty\\
%   bitset.pdf & doc/latex/oberdiek/bitset.pdf\\
%   test/bitset-test1.tex & doc/latex/oberdiek/test/bitset-test1.tex\\
%   test/bitset-test2.tex & doc/latex/oberdiek/test/bitset-test2.tex\\
%   test/bitset-test3.tex & doc/latex/oberdiek/test/bitset-test3.tex\\
%   bitset.dtx & source/latex/oberdiek/bitset.dtx\\
% \end{tabular}^^A
% }^^A
% \sbox0{\t}^^A
% \ifdim\wd0>\linewidth
%   \begingroup
%     \advance\linewidth by\leftmargin
%     \advance\linewidth by\rightmargin
%   \edef\x{\endgroup
%     \def\noexpand\lw{\the\linewidth}^^A
%   }\x
%   \def\lwbox{^^A
%     \leavevmode
%     \hbox to \linewidth{^^A
%       \kern-\leftmargin\relax
%       \hss
%       \usebox0
%       \hss
%       \kern-\rightmargin\relax
%     }^^A
%   }^^A
%   \ifdim\wd0>\lw
%     \sbox0{\small\t}^^A
%     \ifdim\wd0>\linewidth
%       \ifdim\wd0>\lw
%         \sbox0{\footnotesize\t}^^A
%         \ifdim\wd0>\linewidth
%           \ifdim\wd0>\lw
%             \sbox0{\scriptsize\t}^^A
%             \ifdim\wd0>\linewidth
%               \ifdim\wd0>\lw
%                 \sbox0{\tiny\t}^^A
%                 \ifdim\wd0>\linewidth
%                   \lwbox
%                 \else
%                   \usebox0
%                 \fi
%               \else
%                 \lwbox
%               \fi
%             \else
%               \usebox0
%             \fi
%           \else
%             \lwbox
%           \fi
%         \else
%           \usebox0
%         \fi
%       \else
%         \lwbox
%       \fi
%     \else
%       \usebox0
%     \fi
%   \else
%     \lwbox
%   \fi
% \else
%   \usebox0
% \fi
% \end{quote}
% If you have a \xfile{docstrip.cfg} that configures and enables \docstrip's
% TDS installing feature, then some files can already be in the right
% place, see the documentation of \docstrip.
%
% \subsection{Refresh file name databases}
%
% If your \TeX~distribution
% (\teTeX, \mikTeX, \dots) relies on file name databases, you must refresh
% these. For example, \teTeX\ users run \verb|texhash| or
% \verb|mktexlsr|.
%
% \subsection{Some details for the interested}
%
% \paragraph{Attached source.}
%
% The PDF documentation on CTAN also includes the
% \xfile{.dtx} source file. It can be extracted by
% AcrobatReader 6 or higher. Another option is \textsf{pdftk},
% e.g. unpack the file into the current directory:
% \begin{quote}
%   \verb|pdftk bitset.pdf unpack_files output .|
% \end{quote}
%
% \paragraph{Unpacking with \LaTeX.}
% The \xfile{.dtx} chooses its action depending on the format:
% \begin{description}
% \item[\plainTeX:] Run \docstrip\ and extract the files.
% \item[\LaTeX:] Generate the documentation.
% \end{description}
% If you insist on using \LaTeX\ for \docstrip\ (really,
% \docstrip\ does not need \LaTeX), then inform the autodetect routine
% about your intention:
% \begin{quote}
%   \verb|latex \let\install=y% \iffalse meta-comment
%
% File: bitset.dtx
% Version: 2011/01/30 v1.1
% Info: Handle bit-vector datatype
%
% Copyright (C) 2007, 2011 by
%    Heiko Oberdiek <heiko.oberdiek at googlemail.com>
%
% This work may be distributed and/or modified under the
% conditions of the LaTeX Project Public License, either
% version 1.3c of this license or (at your option) any later
% version. This version of this license is in
%    http://www.latex-project.org/lppl/lppl-1-3c.txt
% and the latest version of this license is in
%    http://www.latex-project.org/lppl.txt
% and version 1.3 or later is part of all distributions of
% LaTeX version 2005/12/01 or later.
%
% This work has the LPPL maintenance status "maintained".
%
% This Current Maintainer of this work is Heiko Oberdiek.
%
% The Base Interpreter refers to any `TeX-Format',
% because some files are installed in TDS:tex/generic//.
%
% This work consists of the main source file bitset.dtx
% and the derived files
%    bitset.sty, bitset.pdf, bitset.ins, bitset.drv, bitset-test1.tex,
%    bitset-test2.tex, bitset-test3.tex.
%
% Distribution:
%    CTAN:macros/latex/contrib/oberdiek/bitset.dtx
%    CTAN:macros/latex/contrib/oberdiek/bitset.pdf
%
% Unpacking:
%    (a) If bitset.ins is present:
%           tex bitset.ins
%    (b) Without bitset.ins:
%           tex bitset.dtx
%    (c) If you insist on using LaTeX
%           latex \let\install=y\input{bitset.dtx}
%        (quote the arguments according to the demands of your shell)
%
% Documentation:
%    (a) If bitset.drv is present:
%           latex bitset.drv
%    (b) Without bitset.drv:
%           latex bitset.dtx; ...
%    The class ltxdoc loads the configuration file ltxdoc.cfg
%    if available. Here you can specify further options, e.g.
%    use A4 as paper format:
%       \PassOptionsToClass{a4paper}{article}
%
%    Programm calls to get the documentation (example):
%       pdflatex bitset.dtx
%       makeindex -s gind.ist bitset.idx
%       pdflatex bitset.dtx
%       makeindex -s gind.ist bitset.idx
%       pdflatex bitset.dtx
%
% Installation:
%    TDS:tex/generic/oberdiek/bitset.sty
%    TDS:doc/latex/oberdiek/bitset.pdf
%    TDS:doc/latex/oberdiek/test/bitset-test1.tex
%    TDS:doc/latex/oberdiek/test/bitset-test2.tex
%    TDS:doc/latex/oberdiek/test/bitset-test3.tex
%    TDS:source/latex/oberdiek/bitset.dtx
%
%<*ignore>
\begingroup
  \catcode123=1 %
  \catcode125=2 %
  \def\x{LaTeX2e}%
\expandafter\endgroup
\ifcase 0\ifx\install y1\fi\expandafter
         \ifx\csname processbatchFile\endcsname\relax\else1\fi
         \ifx\fmtname\x\else 1\fi\relax
\else\csname fi\endcsname
%</ignore>
%<*install>
\input docstrip.tex
\Msg{************************************************************************}
\Msg{* Installation}
\Msg{* Package: bitset 2011/01/30 v1.1 Handle bit-vector datatype (HO)}
\Msg{************************************************************************}

\keepsilent
\askforoverwritefalse

\let\MetaPrefix\relax
\preamble

This is a generated file.

Project: bitset
Version: 2011/01/30 v1.1

Copyright (C) 2007, 2011 by
   Heiko Oberdiek <heiko.oberdiek at googlemail.com>

This work may be distributed and/or modified under the
conditions of the LaTeX Project Public License, either
version 1.3c of this license or (at your option) any later
version. This version of this license is in
   http://www.latex-project.org/lppl/lppl-1-3c.txt
and the latest version of this license is in
   http://www.latex-project.org/lppl.txt
and version 1.3 or later is part of all distributions of
LaTeX version 2005/12/01 or later.

This work has the LPPL maintenance status "maintained".

This Current Maintainer of this work is Heiko Oberdiek.

The Base Interpreter refers to any `TeX-Format',
because some files are installed in TDS:tex/generic//.

This work consists of the main source file bitset.dtx
and the derived files
   bitset.sty, bitset.pdf, bitset.ins, bitset.drv, bitset-test1.tex,
   bitset-test2.tex, bitset-test3.tex.

\endpreamble
\let\MetaPrefix\DoubleperCent

\generate{%
  \file{bitset.ins}{\from{bitset.dtx}{install}}%
  \file{bitset.drv}{\from{bitset.dtx}{driver}}%
  \usedir{tex/generic/oberdiek}%
  \file{bitset.sty}{\from{bitset.dtx}{package}}%
  \usedir{doc/latex/oberdiek/test}%
  \file{bitset-test1.tex}{\from{bitset.dtx}{test1}}%
  \file{bitset-test2.tex}{\from{bitset.dtx}{test2,etex}}%
  \file{bitset-test3.tex}{\from{bitset.dtx}{test2,noetex}}%
  \nopreamble
  \nopostamble
  \usedir{source/latex/oberdiek/catalogue}%
  \file{bitset.xml}{\from{bitset.dtx}{catalogue}}%
}

\catcode32=13\relax% active space
\let =\space%
\Msg{************************************************************************}
\Msg{*}
\Msg{* To finish the installation you have to move the following}
\Msg{* file into a directory searched by TeX:}
\Msg{*}
\Msg{*     bitset.sty}
\Msg{*}
\Msg{* To produce the documentation run the file `bitset.drv'}
\Msg{* through LaTeX.}
\Msg{*}
\Msg{* Happy TeXing!}
\Msg{*}
\Msg{************************************************************************}

\endbatchfile
%</install>
%<*ignore>
\fi
%</ignore>
%<*driver>
\NeedsTeXFormat{LaTeX2e}
\ProvidesFile{bitset.drv}%
  [2011/01/30 v1.1 Handle bit-vector datatype (HO)]%
\documentclass{ltxdoc}
\usepackage{holtxdoc}[2011/11/22]
\usepackage{amstext}
\usepackage{ifluatex}
\ifluatex
\else
  \usepackage[T1]{fontenc}%
  \usepackage{lmodern}%
\fi
\begin{document}
  \DocInput{bitset.dtx}%
\end{document}
%</driver>
% \fi
%
% \CheckSum{3510}
%
% \CharacterTable
%  {Upper-case    \A\B\C\D\E\F\G\H\I\J\K\L\M\N\O\P\Q\R\S\T\U\V\W\X\Y\Z
%   Lower-case    \a\b\c\d\e\f\g\h\i\j\k\l\m\n\o\p\q\r\s\t\u\v\w\x\y\z
%   Digits        \0\1\2\3\4\5\6\7\8\9
%   Exclamation   \!     Double quote  \"     Hash (number) \#
%   Dollar        \$     Percent       \%     Ampersand     \&
%   Acute accent  \'     Left paren    \(     Right paren   \)
%   Asterisk      \*     Plus          \+     Comma         \,
%   Minus         \-     Point         \.     Solidus       \/
%   Colon         \:     Semicolon     \;     Less than     \<
%   Equals        \=     Greater than  \>     Question mark \?
%   Commercial at \@     Left bracket  \[     Backslash     \\
%   Right bracket \]     Circumflex    \^     Underscore    \_
%   Grave accent  \`     Left brace    \{     Vertical bar  \|
%   Right brace   \}     Tilde         \~}
%
% \GetFileInfo{bitset.drv}
%
% \title{The \xpackage{bitset} package}
% \date{2011/01/30 v1.1}
% \author{Heiko Oberdiek\\\xemail{heiko.oberdiek at googlemail.com}}
%
% \maketitle
%
% \begin{abstract}
% This package defines and implements the data type bit set,
% a vector of bits. The size of the vector may grow dynamically.
% Individual bits can be manipulated.
% \end{abstract}
%
% \tableofcontents
%
% \clearpage
% \section{Documentation}
%
% \subsection{Introduction}
%
% Annotations in the PDF format know entries whose values are
% integers. This numbers are interpreted as set of flags
% specifying properties. For example, annotation dictionaries
% can have a key |/F|. The bits of its integer value
% are interpreted the following way:
% \begin{quote}
% \begin{tabular}{@{}l|l@{}}
%   Bit position & Property name\\
%   \hline
%   1 & Invisible\\
%   2 & Hidden\\
%   3 & Print\\
%   4 & NoZoom\\
%   5 & NoRotate\\
%   6 & NoView\\
%   7 & ReadOnly\\
%   \dots & \dots
% \end{tabular}
% \end{quote}
% Now, let's see how these values are set in package \xpackage{hyperref}
% before it uses this package (before v6.77a):
%\begin{quote}
%\begin{verbatim}
%\ifFld@hidden /F 6\else /F 4\fi
%\end{verbatim}
%\end{quote}
% Where are the other flags? The following example for key |/Ff|
% in a widget annotation
% supports at least three properties:
%\begin{quote}
%\begin{verbatim}
%\ifFld@multiline
%  \ifFld@readonly /Ff 4097\else /Ff 4096\fi
%\else
%  \ifFld@password
%    \ifFld@readonly /Ff 8193\else /Ff 8192\fi
%  \else
%    \ifFld@readonly /Ff 1\fi
%  \fi
%\fi
%\end{verbatim}
%\end{quote}
% But you see the point. It would be a nightmare to continue
% this way in supporting the missing flag settings. This kind
% of integers may have up to 32 bits.
%
% Therefore I wanted a data structure for setting and
% clearing individual bits. Also it should provide
% an export as decimal number. The snipsets above are
% executed in expansion contexts without \TeX's stomach
% commands. It would be convenient to have an expandable
% conversion from the data structure to the integer that
% gets written to the PDF file.
%
% This package \xpackage{bitset} implements such a data
% structure. The interface is quite close to Java's
% class |BitSet| in order not to learn to many interfaces
% for the same kind of data structure.
%
% \subsection{Glossary}
%
% \begin{description}
% \item[Bit set:]
%   A bit set is a vector of bits or flags.
%   The vector size is unlimited and
%   grows dynamically. An undefined bit set is treated as bit
%   set where all bits are cleared.
%
%   Bit sets are addressed by name. A name should consists of
%   letters or digits. Technically it must survive \cs{csname},
%   see \LaTeX's environment names for other names with such
%   a constraint. Package \xpackage{babel}'s shorthands are
%   not supported due to technical reasons. Shorthand support
%   would break expandable operations.
% \item[Size:]
%   A size of a bit set is the number of bits in use.
%   It's the number of the highest index, incremented by one.
%   Sizes are in the range |0| up to |2147483647|, the highest
%   number supported by \TeX.
% \item[Index:]
%   Bit positions in a bit set are addressed by an index number.
%   The bit vector is zero based. The first and least significant
%   bit is addressed by index |0| and the highest possible bit
%   by |2147483646|.
% \item[Bit:]
%   A bit is enoded as |0| for cleared/disabled or |1| for set/enabled.
% \end{description}
%
% \subsection{Design principles}
%
% \begin{description}
% \item[Name conventions:]
%   To avoid conflicts with existing macro names, the operations
%   are prefixed by the package name.
% \item[Zero based indexes:] The first bit is addressed by zero.
%   (Convention of array indexing in C, Java, \dots)
% \item[Unlimited size:]
%   There is no restriction on the size of a bit set other than
%   usual memory limitations. \cs{bitsetSetDec} and \cs{bitsetGetDec}
%   transparently switch to package \xpackage{bigintcalc} if the numbers
%   get too large for \TeX's number limit.
% \item[Expandibility:]
%   Any operation that does not change the bit set is expandable.
%   And all operations that extract or calculate some result
%   do this in exact two expansion steps. For example,
%   a macro \cs{Macro} wants a bit set as decimal number.
%   But the argument must be a plain number without macros.
%   Thus you could prefix \cs{bitsetGetDec} with \cs{number}.
%   However this won't work for bit sets with 31 or more bits
%   because of \TeX's number limit of $2^{31}-1$.
%   then just hit the operator with two \cs{expandafter}:
%   \begin{quote}
%   |\expandafter\expandafter\expandafter|\\
%   |\Macro\bitsetGetDec{foo}|
%   \end{quote}
%   |\bitsetGetDec| is hit first by the third \cs{expandafter}
%   and then by the second one.
% \item[Format independence:] This package is written as \LaTeX\ package,
%   but it does not depend on \LaTeX. It will also work for other formats
%   such as \plainTeX.
% \item[Independence from \TeX\ engines:]
%   Vanilla \TeX\ is all you need. Calculations are delegated to
%   packages \xpackage{intcalc} and \xpackage{bigintcalc}.
%   They don't need any special features, but they will switch
%   to a little more efficient implementation if features such
%   as \cs{numexpr} are available.
% \item[Numeric arguments:] Anything that is accepted by
%   \cs{number}. If \eTeX\ is detected, also expressions
%   for \cs{numexpr} are supported. The only exception so far
%   is the number for \cs{bitsetSetDec}. The number might be too
%   large for \cs{number} or \cs{numexpr}.
% \item[Error messages:]
%   In expandable contexts, only a limited set of \TeX\ primitive
%   commands work as expected. So called stomach commands behave
%   like \cs{relax} and don't get expanded or executed.
%   Unhappily also the error commands belong to this category.
%   The expandable operations will throw an unknown control
%   sequence instead to get \TeX's and user's attention.
%   The name of these control sequences starts with
%   \cs{BitSetError:} with the type of error after the colon.
% \end{description}
%
% \subsection{Operator overview}
%
% \begin{minipage}{\linewidth}
% \begin{description}
% \def\It#1#2{^^A
%   \item[^^A
%     {#1} ^^A
%     \textmd{(\hyperref[{sec:#2}]{section \ref*{sec:#2}})}^^A
%   ]^^A
%   \mbox{}\\^^A
% }
% \def\op#1{^^A
%   \cs{bitset}\texttt{\bfseries#1}^^A
% }
% \def\M#1{\meta{#1}}
% \def\Q{\hspace*{\fill}}
% \It{Miscellaneous}{misc}
%   \op{Reset}\Q \M{BitSet}\\
%   \op{Let}\Q \M{BitSet A} \M{BitSet B}
% \It{Import}{import}
%   \op{SetBin}, \op{SetOct}, \op{SetHex}\Q \M{BitSet} \M{Value}\\
%   \op{SetDec}\Q \M{BitSet} \M{Value}
% \It{Export\textsuperscript{a}}{export}
%   \op{GetBin}, \op{GetOct}, \op{GetHex}\Q \M{BitSet} \M{MinSize}\\
%   \op{GetDec}\Q \M{BitSet}
% \It{Logical operators}{log}
%   \op{And}, \op{AndNot}\Q \M{BitSet A} \M{BitSet B}\\
%   \op{Or}, \op{Xor}\Q \M{BitSet A} \M{BitSet B}
% \It{Shifting}{shift}
%   \op{ShiftLeft}, \op{ShiftRight}\Q \M{BitSet} \M{ShiftAmount}\\
% \It{Bit manipulation}{bitman}
%   \op{Clear}, \op{Set}, \op{Flip}\Q \M{BitSet} \M{Index}\\
%   \op{SetValue}\Q \M{BitSet} \M{Index} \M{Value}\\
%   \op{ClearRange}, \op{SetRange}, \op{FlipRange}\\
%     \Q \M{BitSet} \M{IndexFrom} \M{IndexTo}\\
%   \op{SetValueRange}\Q \M{BitSet} \M{IndexFrom} \M{IndexTo}
% \It{Bit retrieval\textsuperscript{a}}{bitret}
%   \op{Get}\Q \M{BitSet} \M{Index}\\
%   \op{NextClearBit}, \op{NextSetBit}\Q \M{BitSet} \M{Index}\\
%   \op{GetSetBitList}\Q \M{BitSet}
% \It{Bit set properties}{prop}
%   \op{Size}, \op{Cardinality}\Q \M{BitSet}
% \It{Queries\textsuperscript{b}}{query}
%   \op{IsDefined}, \op{IsEmpty}\Q \M{BitSet} \M{Then} \M{Else}\\
%   \op{Equals}, \op{Intersects}\Q \M{BitSet A} \M{BitSet B} \M{Then} \M{Else}\\
%   \op{Query}\Q \M{BitSet} \M{Index} \M{Then} \M{Else}
% \end{description}
% \footnotetext[1]{Macros are expandable, full expansion by two steps.}
% \footnotetext[2]{Macros are expandable.}
% \end{minipage}
%
% \subsection{Package loading}
%
% The package can be used as normal \LaTeX\ package:
%\begin{quote}
%|\usepackage{bitset}|
%\end{quote}
% Also \plainTeX\ is supported:
%\begin{quote}
%|\input bitset.sty\relax|
%\end{quote}
%
% \subsection{Operators}
%
% The following macros work on and with bit sets.
% A bit set \meta{BitSet} is represented by a name.
% The should consist of letters and digits. Technically
% it must survive \cs{csname}. It is the same constraint
% that must be satisfied by label or environment names
% in \LaTeX.
%
% However active characters that are shorthands
% of package \xpackage{babel} are not supported.
% Support for shorthands works by an assignment.
% But many operators
% such as \cs{bitsetGetDec} must be usable in expandable
% contexts. There assignments will not be executed in the
% best case or they will cause errors.
%
% The bits in a bit set are addressed by non-negative
% integers starting from zero. Thus negative index
% numbers cause an error message. Because index numbers
% are \TeX\ numbers. The largest index is 2147483647.
% But in practice memory limits and patience limits
% will be very likely reached much before.
%
% \subsubsection{Miscellaneous}\label{sec:misc}
%
% There isn't a separate operation for bit set creation.
% For simplicity an undefined bit set is treated as bit set
% with all bits cleared.
%
% \begin{declcs}{bitsetReset} \M{BitSet}
% \end{declcs}
% Macro \cs{bitsetReset} clears all bits. The result is an
% empty bit set. It may also be used as replacement for
% an operation ``new'', because an undefined bit set is defined
% afterwards.
%
% \begin{declcs}{bitsetLet} \M{BitSet A} \M{BitSet B}
% \end{declcs}
% Macro \cs{bitsetLet} performs a simple assignment similar
% to \TeX's \cs{let}. After the operation \meta{BitSet A} has
% the same value as \meta{BitSet B}. If \meta{BitSet B} is
% undefined, then \meta{BitSet A} will be the empty bit set.
%
% Note: If \meta{BitSet A} exists, it will be overwritten.
%
% \subsubsection{Import}\label{sec:import}
%
% \begin{declcs}{bitsetSetBin} \M{BitSet} \M{BinaryNumber}\\
%   \cs{bitsetSetOct} \M{BitSet} \M{OctalNumber}\\
%   \cs{bitsetSetHex} \M{BitSet} \M{HexadecimalNumber}
% \end{declcs}
% The numbers are interpreted as bit vectors and the flags in
% the bit \meta{BitSet} set are set accordingly.
% These numeric arguments are the only arguments where
% spaces are allowed. Then the numbers are easier to read.
%
% \begin{declcs}{bitsetSetDec} \M{BitSet} \M{DecimalNumber}
% \end{declcs}
% Macro \cs{bitsetSetDec} uses \meta{DecimalNumber} to
% set the bit set \meta{BitSet}.
% The numeric argument must expand
% to a plain number consisting of decimal digits without
% command tokens or spaces. Internally this argument is
% expanded only. It cannot be passed to \cs{number} or
% \cs{numexpr}, because the number may be too large for them.
% However \cs{number} or |\the\numexpr| may be used
% explicitely. This also helps for unexpandable number
% command tokens or registers (\cs{z@}, \cs{@ne}, \cs{count@}, \dots).
% Also \LaTeX'\ \cs{value} needs prefixing:
% \begin{quote}
% |\bitsetSetDec{foo}{\number\value{bar}}|
% \end{quote}
%
% \subsubsection{Export}\label{sec:export}
%
% \begin{declcs}{bitsetGetBin} \M{BitSet} \M{MinSize}\\
%   \cs{bitsetGetOct} \M{BitSet} \M{MinSize}\\
%   \cs{bitsetGetHex} \M{BitSet} \M{MinSize}
% \end{declcs}
% These macros returns the bit set as binary, octal or hexadecimal
% number. If the bit size is smaller than \meta{MinSize} the gap is
% filled with leading zeros.
% Example:
% \begin{quote}
% |\bitsetReset{abc}|\\
% |\bitsetSet{abc}{2}|\\
% |\bitsetGetBin{abc}{8}| $\rightarrow$ |00000100|\\
% |\bitsetSet{abc}{5}\bitsetSet{abc}{7}|\\
% |\bitsetGetHex{abc}{16}| $\rightarrow$ |00A2|
% \end{quote}
% Macro \cs{bitsetGetHex} uses the uppercase letters |A| to |F|.
% The catcode of the letters is one of 11 (letter) or 12 (other).
%
% \begin{declcs}{bitsetGetDec} \M{BitSet}
% \end{declcs}
% Macro \cs{bitsetGetDec} returns the bit set \meta{BitSet} as
% decimal number. The returned number can be larger than
% \TeX's number limit of $2^{31}-1$.
%
% \subsubsection{Logical operators}\label{sec:log}
%
% \begin{declcs}{bitsetAnd} \M{BitSet A} \M{BitSet B}
% \end{declcs}
% $A_{\text{new}} \mathrel{\mathop{:}}= A_{\text{old}}
% \mathbin{\text{and}} B \qquad (\forall\, \text{bits})$
%
% \begin{declcs}{bitsetAndNot} \M{BitSet A} \M{BitSet B}
% \end{declcs}
% $A_{\text{new}} \mathrel{\mathop{:}}= A_{\text{old}}
% \mathbin{\text{and}} (\text{not}\,B) \qquad (\forall\, \text{bits})$
%
% \begin{declcs}{bitsetOr} \M{BitSet A} \M{BitSet B}
% \end{declcs}
% $A_{\text{new}} \mathrel{\mathop{:}}= A_{\text{old}}
% \mathbin{\text{or}} B \qquad (\forall\, \text{bits})$
%
% \begin{declcs}{bitsetXor} \M{BitSet A} \M{BitSet B}
% \end{declcs}
% $A_{\text{new}} \mathrel{\mathop{:}}= A_{\text{old}}
% \mathbin{\text{xor}} B \qquad (\forall\, \text{bits})$
%
% \subsubsection{Shifting}\label{sec:shift}
%
% \begin{declcs}{bitsetShiftLeft} \M{BitSet} \M{ShiftAmount}\\
%   \cs{bitsetShiftRight} \M{BitSet} \M{ShiftAmount}
% \end{declcs}
% A left shift by one is a multiplication by two, thus left shifting
% moves the flags to higher positions. The new created low positions
% are filled by zeros.
%
% A right shift is the opposite, dividing by two, movint the bits
% to lower positions. The number will become smaller, the lowest
% bits are lost.
%
% If the \meta{ShiftAmount} is negative, it reverts the meaning of
% the shift operation. A left shift becomes a right shift.
% A \meta{ShiftAmount} of zero is ignored.
%
% \subsubsection{Bit manipulation}\label{sec:bitman}
%
% \begin{declcs}{bitsetClear} \M{BitSet} \M{Index}\\
%   \cs{bitsetSet} \M{BitSet} \M{Index}\\
%   \cs{bitsetFlip} \M{BitSet} \M{Index}
% \end{declcs}
% This macros manipulate a single bit in \meta{BitSet} addressed
% by \cs{Index}. Macro \cs{bitsetClear} disables the bit,
% \cs{bitsetSet} enables it and \cs{bitsetFlip} reverts the current
% setting of the bit.
%
% \begin{declcs}{bitsetSetValue} \M{BitSet} \M{Index} \M{Bit}
% \end{declcs}
% Macro \cs{bitsetSetValue} puts bit \meta{Bit} at position \meta{Index}
% in bit set \meta{BitSet}. \meta{Bit} must be a valid \TeX\ number
% equals to zero (disabled/cleared) or one (enabled/set).
%
% \subsubsection{Bit retrieval}\label{sec:bitret}
%
% \begin{declcs}{bitsetGet} \M{BitSet} \M{Index}
% \end{declcs}
% Macro \cs{bitsetGet} extracts the status of the bit at
% position \meta{Index} in bit set \meta{BitSet}.
% Digit |1| is returned if the bit is set/enabled. If the
% bit is cleared/disabled and in cases of an undefined bitset or
% an index number out of range the return value is |0|.
%
% \begin{declcs}{bitsetNextClearBit} \M{BitSet} \M{Index}
% \end{declcs}
% Starting at position \meta{Index} (inclusive) the bits
% are inspected. The first position without a set bit is returned.
% Possible results are decimal numbers: \meta{Index}, \meta{Index} + 1,
% \dots, ($\infty$)
%
% \begin{declcs}{bitsetNextSetBit} \M{BitSet} \M{Index}
% \end{declcs}
% Starting at position \meta{Index} (inclusive) the bits are
% inspected and the index position of the first found set bit
% is returned. If there isn't such a bit, then the result is
% |-1|. In summary possible results are decimal numbers:
% |-1|, \meta{Index}, \meta{Index} + 1, \dots, ($\infty$)
%
% \begin{declcs}{bitsetGetSetBitList} \M{BitSet}
% \end{declcs}
% Macro \cs{bitsetGetSetBitList} is an application for
% \cs{bitsetNextSetBit}. The set bits are iterated and
% returned as comma separated list of index positions
% in increasing order. The list is empty in case of
% an empty bit set.
%
% \subsubsection{Bit set properties}\label{sec:prop}
%
% \begin{declcs}{bitsetSize} \M{BitSet}
% \end{declcs}
% Macro \cs{bitsetSize} returns number of bits in use.
% It is the same as the index number of the highest set/enabled bit
% incremented by one.
%
% \begin{declcs}{bitsetCardinality} \M{BitSet}
% \end{declcs}
% Macro \cs{bitsetCardinality} counts the number of set/enabled bits.
%
% \subsubsection{Queries}\label{sec:query}
%
% Also the query procedures are expandable. They ask for a piece
% of information about a bit set and execute code depending
% on the answer.
%
% \begin{declcs}{bitsetIsDefined} \M{BitSet} \M{Then} \M{Else}
% \end{declcs}
% If the bit set with the name \meta{BitSet} exists the code
% given in \meta{Then} is executed, otherwise \meta{Else} is used.
%
% \begin{declcs}{bitsetIsEmpty} \M{BitSet} \M{Then} \M{Else}
% \end{declcs}
% If the bit set \meta{BitSet} exists and at least one bit
% is set/enabled, the code in \meta{Then} is executed,
% \meta{Else} otherwise.
%
% \begin{declcs}{bitsetEquals} \M{BitSet A} \M{BitSet B} \M{Then} \M{Else}
% \end{declcs}
% Both bit sets are equal if and only if either both are undefined
% or both are defined and represents the same bit values at the
% same positions. Thus this definition is reflexive, symmeric,
% and transitive, enough for an equivalent relation.
%
% \begin{declcs}{bitsetIntersects} \M{BitSet A} \M{BitSet B} \M{Then} \M{Else}
% \end{declcs}
% If and only if \meta{BitSet A} and \meta{BitSet B} have
% at least one bit at the same position that is set, then
% code part \meta{Then} is executed.
%
% \begin{declcs}{bitsetQuery} \M{BitSet} \M{Index} \M{Then} \M{Else}
% \end{declcs}
% It's just a wrapper for \cs{bitsetGet}. If the bit at position
% \meta{Index} is enabled, code \meta{Then} is called.
%
% \StopEventually{
% }
%
% \section{Implementation}
%
% The internal format of a bit set is quite simple, a sequence
% of digits |0| and |1|. The least significant bit is left.
% A bit set without any flag set is encoded by |0|. Also
% undefined bit sets are treated that way. After the
% highest bit that is set there are no further zeroes.
% A regular expression of valid bit sets values:
% \begin{quote}
%  |0|\string||[01]*1|
% \end{quote}
%
%    \begin{macrocode}
%<*package>
%    \end{macrocode}
%
% \subsection{Reload check and package identification}
%    Reload check, especially if the package is not used with \LaTeX.
%    \begin{macrocode}
\begingroup\catcode61\catcode48\catcode32=10\relax%
  \catcode13=5 % ^^M
  \endlinechar=13 %
  \catcode35=6 % #
  \catcode39=12 % '
  \catcode44=12 % ,
  \catcode45=12 % -
  \catcode46=12 % .
  \catcode58=12 % :
  \catcode64=11 % @
  \catcode123=1 % {
  \catcode125=2 % }
  \expandafter\let\expandafter\x\csname ver@bitset.sty\endcsname
  \ifx\x\relax % plain-TeX, first loading
  \else
    \def\empty{}%
    \ifx\x\empty % LaTeX, first loading,
      % variable is initialized, but \ProvidesPackage not yet seen
    \else
      \expandafter\ifx\csname PackageInfo\endcsname\relax
        \def\x#1#2{%
          \immediate\write-1{Package #1 Info: #2.}%
        }%
      \else
        \def\x#1#2{\PackageInfo{#1}{#2, stopped}}%
      \fi
      \x{bitset}{The package is already loaded}%
      \aftergroup\endinput
    \fi
  \fi
\endgroup%
%    \end{macrocode}
%    Package identification:
%    \begin{macrocode}
\begingroup\catcode61\catcode48\catcode32=10\relax%
  \catcode13=5 % ^^M
  \endlinechar=13 %
  \catcode35=6 % #
  \catcode39=12 % '
  \catcode40=12 % (
  \catcode41=12 % )
  \catcode44=12 % ,
  \catcode45=12 % -
  \catcode46=12 % .
  \catcode47=12 % /
  \catcode58=12 % :
  \catcode64=11 % @
  \catcode91=12 % [
  \catcode93=12 % ]
  \catcode123=1 % {
  \catcode125=2 % }
  \expandafter\ifx\csname ProvidesPackage\endcsname\relax
    \def\x#1#2#3[#4]{\endgroup
      \immediate\write-1{Package: #3 #4}%
      \xdef#1{#4}%
    }%
  \else
    \def\x#1#2[#3]{\endgroup
      #2[{#3}]%
      \ifx#1\@undefined
        \xdef#1{#3}%
      \fi
      \ifx#1\relax
        \xdef#1{#3}%
      \fi
    }%
  \fi
\expandafter\x\csname ver@bitset.sty\endcsname
\ProvidesPackage{bitset}%
  [2011/01/30 v1.1 Handle bit-vector datatype (HO)]%
%    \end{macrocode}
%
% \subsection{Catcodes}
%
%    \begin{macrocode}
\begingroup\catcode61\catcode48\catcode32=10\relax%
  \catcode13=5 % ^^M
  \endlinechar=13 %
  \catcode123=1 % {
  \catcode125=2 % }
  \catcode64=11 % @
  \def\x{\endgroup
    \expandafter\edef\csname BitSet@AtEnd\endcsname{%
      \endlinechar=\the\endlinechar\relax
      \catcode13=\the\catcode13\relax
      \catcode32=\the\catcode32\relax
      \catcode35=\the\catcode35\relax
      \catcode61=\the\catcode61\relax
      \catcode64=\the\catcode64\relax
      \catcode123=\the\catcode123\relax
      \catcode125=\the\catcode125\relax
    }%
  }%
\x\catcode61\catcode48\catcode32=10\relax%
\catcode13=5 % ^^M
\endlinechar=13 %
\catcode35=6 % #
\catcode64=11 % @
\catcode123=1 % {
\catcode125=2 % }
\def\TMP@EnsureCode#1#2{%
  \edef\BitSet@AtEnd{%
    \BitSet@AtEnd
    \catcode#1=\the\catcode#1\relax
  }%
  \catcode#1=#2\relax
}
\TMP@EnsureCode{33}{12}% !
\TMP@EnsureCode{39}{12}% '
\TMP@EnsureCode{40}{12}% (
\TMP@EnsureCode{41}{12}% )
\TMP@EnsureCode{42}{12}% *
\TMP@EnsureCode{43}{12}% +
\TMP@EnsureCode{44}{12}% ,
\TMP@EnsureCode{45}{12}% -
\TMP@EnsureCode{46}{12}% .
\TMP@EnsureCode{47}{12}% /
\TMP@EnsureCode{58}{11}% : (letter!)
\TMP@EnsureCode{60}{12}% <
\TMP@EnsureCode{62}{12}% >
\TMP@EnsureCode{63}{14}% ? (comment!)
\TMP@EnsureCode{91}{12}% [
\TMP@EnsureCode{93}{12}% ]
\TMP@EnsureCode{96}{12}% `
\edef\BitSet@AtEnd{\BitSet@AtEnd\noexpand\endinput}
\begingroup\expandafter\expandafter\expandafter\endgroup
\expandafter\ifx\csname BitSet@TestMode\endcsname\relax
\else
  \catcode63=9 % ? (ignore)
\fi
? \let\BitSet@@TestMode\BitSet@TestMode
%    \end{macrocode}
%
% \subsection{Package loading}
%
%    \begin{macrocode}
\begingroup\expandafter\expandafter\expandafter\endgroup
\expandafter\ifx\csname RequirePackage\endcsname\relax
  \def\TMP@RequirePackage#1[#2]{%
    \begingroup\expandafter\expandafter\expandafter\endgroup
    \expandafter\ifx\csname ver@#1.sty\endcsname\relax
      \input #1.sty\relax
    \fi
  }%
  \TMP@RequirePackage{infwarerr}[2007/09/09]%
  \TMP@RequirePackage{intcalc}[2007/09/27]%
  \TMP@RequirePackage{bigintcalc}[2007/09/27]%
\else
  \RequirePackage{infwarerr}[2007/09/09]%
  \RequirePackage{intcalc}[2007/09/27]%
  \RequirePackage{bigintcalc}[2007/09/27]%
\fi
%    \end{macrocode}
%
% \subsection{Help macros}
%
% \subsubsection{Number constant}
%
%    \begin{macro}{\BitSet@MaxSize}
%    \begin{macrocode}
\def\BitSet@MaxSize{2147483647}%
%    \end{macrocode}
%    \end{macro}
%
% \subsubsection{General basic macros}
%
%    \begin{macro}{\BitSet@Empty}
%    \begin{macrocode}
\def\BitSet@Empty{}
%    \end{macrocode}
%    \end{macro}
%    \begin{macro}{\BitSet@FirstOfOne}
%    \begin{macrocode}
\def\BitSet@FirstOfOne#1{#1}
%    \end{macrocode}
%    \end{macro}
%    \begin{macro}{\BitSet@Gobble}
%    \begin{macrocode}
\def\BitSet@Gobble#1{}
%    \end{macrocode}
%    \end{macro}
%    \begin{macro}{\BitSet@FirstOfTwo}
%    \begin{macrocode}
\def\BitSet@FirstOfTwo#1#2{#1}
%    \end{macrocode}
%    \end{macro}
%    \begin{macro}{\BitSet@SecondOfTwo}
%    \begin{macrocode}
\def\BitSet@SecondOfTwo#1#2{#2}
%    \end{macrocode}
%    \end{macro}
%
%    \begin{macro}{\BitSet@Space}
%    \begin{macrocode}
\def\BitSet@Space{ }
%    \end{macrocode}
%    \end{macro}
%    \begin{macro}{\BitSet@ZapSpace}
%    \begin{macrocode}
\def\BitSet@ZapSpace#1 #2{%
  #1%
  \ifx\BitSet@Empty#2%
  \else
    \expandafter\BitSet@ZapSpace
  \fi
  #2%
}
%    \end{macrocode}
%    \end{macro}
%
% \subsubsection{Tail recursion}
%
%    \begin{macro}{\BitSet@Fi}
%    \begin{macrocode}
\let\BitSet@Fi\fi
%    \end{macrocode}
%    \end{macro}
%    \begin{macro}{\BitSet@AfterFi}
%    \begin{macrocode}
\def\BitSet@AfterFi#1#2\BitSet@Fi{\fi#1}
%    \end{macrocode}
%    \end{macro}
%    \begin{macro}{\BitSet@AfterFiFi}
%    \begin{macrocode}
\def\BitSet@AfterFiFi#1#2\BitSet@Fi{\fi\fi#1}%
%    \end{macrocode}
%    \end{macro}
%    \begin{macro}{\BitSet@AfterFiFiFi}
%    \begin{macrocode}
\def\BitSet@AfterFiFiFi#1#2\BitSet@Fi{\fi\fi\fi#1}%
%    \end{macrocode}
%    \end{macro}
%
% \subsubsection{Check macros}
%
%    \begin{macro}{\BitSet@IfUndefined}
%    \begin{macrocode}
\def\BitSet@IfUndefined#1{%
  \expandafter\ifx\csname BS@#1\endcsname\relax
    \expandafter\BitSet@FirstOfTwo
  \else
    \expandafter\BitSet@SecondOfTwo
  \fi
}
%    \end{macrocode}
%    \end{macro}
%
%    \begin{macro}{\BitSet@CheckIndex}
%    |#1|: continuation code\\
%    |#2|: BitSet\\
%    |#3|: Index
%    \begin{macrocode}
\def\BitSet@CheckIndex#1#2#3{%
  \BitSet@IfUndefined{#2}{\bitsetReset{#2}}{}%
  \expandafter\expandafter\expandafter\BitSet@@CheckIndex
  \intcalcNum{#3}!%
  {#2}{#1}%
}
%    \end{macrocode}
%    \end{macro}
%    \begin{macro}{\BitSet@@CheckIndex}
%    |#1|: plain Index\\
%    |#2|: BitSet\\
%    |#3|: continuation code
%    \begin{macrocode}
\def\BitSet@@CheckIndex#1!#2#3{%
  \ifnum#1<0 %
    \BitSet@AfterFi{%
      \@PackageError{bitset}{%
        Invalid negative index (#1)%
      }\@ehc
    }%
  \else
    \BitSet@AfterFi{%
      #3{#2}{#1}%
    }%
  \BitSet@Fi
}
%    \end{macrocode}
%    \end{macro}
%
% \subsection{Miscellaneous}\label{src:misc}
%
%    \begin{macro}{\bitsetReset}
%    \begin{macrocode}
\def\bitsetReset#1{%
  \expandafter\def\csname BS@#1\endcsname{0}%
}
%    \end{macrocode}
%    \end{macro}
%
%    \begin{macro}{\bitsetLet}
%    \begin{macrocode}
\def\bitsetLet#1#2{%
  \BitSet@IfUndefined{#2}{%
    \bitsetReset{#1}%
  }{%
    \expandafter\let\csname BS@#1\expandafter\endcsname
                    \csname BS@#2\endcsname
  }%
}
%    \end{macrocode}
%    \end{macro}
%
% \subsection{Import}\label{src:import}
%
% \subsubsection{From binary number}
%
%    \begin{macro}{\bitsetSetBin}
%    \begin{macrocode}
\def\bitsetSetBin#1#2{%
  \edef\BitSet@Temp{#2}%
  \edef\BitSet@Temp{%
    \expandafter\expandafter\expandafter\BitSet@ZapSpace
    \expandafter\BitSet@Temp\BitSet@Space\BitSet@Empty
  }%
  \edef\BitSet@Temp{%
    \expandafter\BitSet@KillZeros\BitSet@Temp\BitSet@Empty
  }%
  \ifx\BitSet@Temp\BitSet@Empty
    \expandafter\let\csname BS@#1\endcsname\BitSet@Zero
  \else
    \expandafter\edef\csname BS@#1\endcsname{%
      \expandafter\BitSet@Reverse\BitSet@Temp!%
    }%
  \fi
}
%    \end{macrocode}
%    \end{macro}
%    \begin{macro}{\BitSet@KillZeros}
%    \begin{macrocode}
\def\BitSet@KillZeros#1{%
  \ifx#10%
    \expandafter\BitSet@KillZeros
  \else
    #1%
  \fi
}
%    \end{macrocode}
%    \end{macro}
%    \begin{macro}{\BitSet@Reverse}
%    \begin{macrocode}
\def\BitSet@Reverse#1#2!{%
  \ifx\\#2\\%
    #1%
  \else
    \BitSet@AfterFi{%
      \BitSet@Reverse#2!#1%
    }%
  \BitSet@Fi
}
%    \end{macrocode}
%    \end{macro}
%
% \subsubsection{From octal/hex number}
%
%    \begin{macro}{\bitsetSetOct}
%    \begin{macrocode}
\def\bitsetSetOct{%
  \BitSet@SetOctHex\BitSet@FromFirstOct
}
%    \end{macrocode}
%    \end{macro}
%    \begin{macro}{\bitsetSetHex}
%    \begin{macrocode}
\def\bitsetSetHex{%
  \BitSet@SetOctHex\BitSet@FromFirstHex
}
%    \end{macrocode}
%    \end{macro}
%    \begin{macro}{\BitSet@SetOctHex}
%    \begin{macrocode}
\def\BitSet@SetOctHex#1#2#3{%
  \edef\BitSet@Temp{#3}%
  \edef\BitSet@Temp{%
    \expandafter\expandafter\expandafter\BitSet@ZapSpace
    \expandafter\BitSet@Temp\BitSet@Space\BitSet@Empty
  }%
  \edef\BitSet@Temp{%
    \expandafter\BitSet@KillZeros\BitSet@Temp\BitSet@Empty
  }%
  \ifx\BitSet@Temp\BitSet@Empty
    \expandafter\let\csname BS@#2\endcsname\BitSet@Zero
  \else
    \edef\BitSet@Temp{%
      \expandafter#1\BitSet@Temp!%
    }%
    \ifx\BitSet@Temp\BitSet@Empty
      \expandafter\let\csname BS@#2\endcsname\BitSet@Zero
    \else
      \expandafter\edef\csname BS@#2\endcsname{%
        \expandafter\BitSet@Reverse\BitSet@Temp!%
      }%
    \fi
  \fi
}
%    \end{macrocode}
%    \end{macro}
%    \begin{macro}{\BitSet@FromFirstOct}
%    \begin{macrocode}
\def\BitSet@FromFirstOct#1{%
  \ifx#1!%
  \else
    \ifcase#1 \BitSet@AfterFiFi\BitSet@FromFirstOct
    \or 1%
    \or 10%
    \or 11%
    \or 100%
    \or 101%
    \or 110%
    \or 111%
    \else \BitSetError:WrongOctalDigit%
    \fi
    \expandafter\BitSet@FromOct
  \BitSet@Fi
}
%    \end{macrocode}
%    \end{macro}
%    \begin{macro}{\BitSet@FromOct}
%    \begin{macrocode}
\def\BitSet@FromOct#1{%
  \ifx#1!%
  \else
    \ifcase#1 000%
    \or 001%
    \or 010%
    \or 011%
    \or 100%
    \or 101%
    \or 110%
    \or 111%
    \else \BitSetError:WrongOctalDigit%
    \fi
    \expandafter\BitSet@FromOct
  \fi
}
%    \end{macrocode}
%    \end{macro}
%
%    \begin{macro}{\BitSet@FromFirstHex}
%    \begin{macrocode}
\def\BitSet@FromFirstHex#1{%
  \ifx#1!%
  \else
    \ifx#10%
      \BitSet@AfterFiFi\BitSet@FromFirstHex
    \fi
    \expandafter\ifx\csname BitSet@Hex#1\endcsname\relax
      \BitSetError:InvalidHexDigit%
    \else
      \expandafter\expandafter\expandafter\BitSet@KillZeros
      \csname BitSet@Hex#1\endcsname
    \fi
    \expandafter\BitSet@FromHex
  \BitSet@Fi
}
%    \end{macrocode}
%    \end{macro}
%    \begin{macro}{\BitSet@FromHex}
%    \begin{macrocode}
\def\BitSet@FromHex#1{%
  \ifx#1!%
  \else
    \expandafter\ifx\csname BitSet@Hex#1\endcsname\relax
      \BitSetError:InvalidHexDigit%
    \else
      \csname BitSet@Hex#1\endcsname
    \fi
    \expandafter\BitSet@FromHex
  \fi
}
%    \end{macrocode}
%    \end{macro}
%    \begin{macro}{\BitSet@Hex[0..F]}
%    \begin{macrocode}
\def\BitSet@Temp#1{%
  \expandafter\def\csname BitSet@Hex#1\endcsname
}
\BitSet@Temp 0{0000}%
\BitSet@Temp 1{0001}%
\BitSet@Temp 2{0010}%
\BitSet@Temp 3{0011}%
\BitSet@Temp 4{0100}%
\BitSet@Temp 5{0101}%
\BitSet@Temp 6{0110}%
\BitSet@Temp 7{0111}%
\BitSet@Temp 8{1000}%
\BitSet@Temp 9{1001}%
\BitSet@Temp A{1010}%
\BitSet@Temp B{1011}%
\BitSet@Temp C{1100}%
\BitSet@Temp D{1101}%
\BitSet@Temp E{1110}%
\BitSet@Temp F{1111}%
\BitSet@Temp a{1010}%
\BitSet@Temp b{1011}%
\BitSet@Temp c{1100}%
\BitSet@Temp d{1101}%
\BitSet@Temp e{1110}%
\BitSet@Temp f{1111}%
%    \end{macrocode}
%    \end{macro}
%
% \subsubsection{From decimal number}
%
%    \begin{macro}{\bitsetSetDec}
%    \begin{macrocode}
\def\bitsetSetDec#1#2{%
  \edef\BitSet@Temp{#2}%
  \edef\BitSet@Temp{%
    \expandafter\expandafter\expandafter\BitSet@ZapSpace
    \expandafter\BitSet@Temp\BitSet@Space\BitSet@Empty
  }%
  \edef\BitSet@Temp{%
    \expandafter\BitSet@KillZeros\BitSet@Temp\BitSet@Empty
  }%
  \ifx\BitSet@Temp\BitSet@Empty
    \expandafter\let\csname BS@#1\endcsname\BitSet@Zero
  \else
    \ifcase\bigintcalcSgn{\BitSet@Temp} %
      \expandafter\let\csname BS@#1\endcsname\BitSet@Zero
    \or
      \ifnum\bigintcalcCmp\BitSet@Temp\BitSet@MaxSize>0 %
        \expandafter\edef\csname BS@#1\endcsname{%
          \expandafter\BitSet@SetDecBig\BitSet@Temp!%
        }%
      \else
        \expandafter\edef\csname BS@#1\endcsname{%
          \expandafter\BitSet@SetDec\BitSet@Temp!%
        }%
      \fi
    \else
      \@PackageError{bitset}{%
        Bit sets cannot be negative%
      }\@ehc
    \fi
  \fi
}
%    \end{macrocode}
%    \end{macro}
%    \begin{macro}{\BitSet@SetDecBig}
%    \begin{macrocode}
\def\BitSet@SetDecBig#1#2#3#4#5#6#7#8#9!{%
  \ifx\\#9\\%
    \BitSet@SetDec#1#2#3#4#5#6#7#8!%
  \else
    \ifcase\BigIntCalcOdd#1#2#4#5#6#7#8#9! %
      0%
    \or
      1%
?   \else\BitSetError:ThisCannotHappen%
    \fi
    \BitSet@AfterFi{%
      \expandafter\expandafter\expandafter\BitSet@SetDecBig
      \BigIntCalcShr#1#2#3#4#5#6#7#8#9!!%
    }%
  \BitSet@Fi
}
%    \end{macrocode}
%    \end{macro}
%    \begin{macro}{\BitSet@SetDec}
%    \begin{macrocode}
\def\BitSet@SetDec#1!{%
  \ifcase#1 %
  \or 1%
  \else
    \ifodd#1 %
      1%
    \else
      0%
    \fi
    \BitSet@AfterFi{%
      \expandafter\expandafter\expandafter\BitSet@SetDec
      \IntCalcShr#1!!%
    }%
  \BitSet@Fi
}
%    \end{macrocode}
%    \end{macro}
%
% \subsection{Export}\label{src:export}
%
% \subsubsection{To binary number}
%
%    \begin{macro}{\bitsetGetBin}
%    \begin{macrocode}
\def\bitsetGetBin#1#2{%
  \romannumeral0%
  \expandafter\expandafter\expandafter\BitSet@@GetBin
  \intcalcNum{#2}!{#1}%
}
%    \end{macrocode}
%    \end{macro}
%    \begin{macro}{\BitSet@@GetBin}
%    \begin{macrocode}
\def\BitSet@@GetBin#1!#2{%
  \BitSet@IfUndefined{#2}{%
    \ifnum#1>1 %
      \BitSet@AfterFi{%
        \expandafter\expandafter\expandafter\BitSet@Fill
        \IntCalcDec#1!!0%
      }%
    \else
      \BitSet@AfterFi{ 0}%
    \BitSet@Fi
  }{%
    \expandafter\expandafter\expandafter\BitSet@NumBinRev
    \expandafter\expandafter\expandafter1%
    \expandafter\expandafter\expandafter!%
    \csname BS@#2\endcsname!!#1!%
  }%
}
%    \end{macrocode}
%    \end{macro}
%    \begin{macro}{\BitSet@Fill}
%    |#1|: number of leading digits |0|\\
%    |#2|: result
%    \begin{macrocode}
\def\BitSet@Fill#1!{%
  \ifnum#1>0 %
    \BitSet@AfterFi{%
      \expandafter\expandafter\expandafter\BitSet@Fill
      \IntCalcDec#1!!0%
    }%
  \else
    \BitSet@AfterFi{ }%
  \BitSet@Fi
}
%    \end{macrocode}
%    \end{macro}
%
%    \begin{macro}{\BitSet@NumBinRev}
%    |#1|: bit counter (including |#2|)\\
%    |#2#3|: reverted number\\
%    |#4|: result\\
%    |#5|: min size
%    \begin{macrocode}
\def\BitSet@NumBinRev#1!#2#3!{%
  \ifx\\#3\\%
    \BitSet@AfterFi{%
      \BitSet@NumBinFill#1!#2%
    }%
  \else
    \BitSet@AfterFi{%
      \expandafter\expandafter\expandafter\BitSet@NumBinRev
      \IntCalcInc#1!!#3!#2%
    }%
  \BitSet@Fi
}
%    \end{macrocode}
%    \end{macro}
%    \begin{macro}{\BitSet@NumBinFill}
%    \begin{macrocode}
\def\BitSet@NumBinFill#1!#2!#3!{%
  \ifnum#3>#1 %
    \BitSet@AfterFi{%
      \expandafter\expandafter\expandafter\BitSet@Fill
      \IntCalcSub#3!#1!!#2%
    }%
  \else
    \BitSet@AfterFi{ #2}%
  \BitSet@Fi
}
%    \end{macrocode}
%    \end{macro}
%
% \subsubsection{To octal/hexadecimal number}
%
%    \begin{macro}{\bitsetGetOct}
%    \begin{macrocode}
\def\bitsetGetOct#1#2{%
  \romannumeral0%
  \bitsetIsEmpty{#1}{%
    \expandafter\expandafter\expandafter\BitSet@@GetOctHex
    \intcalcNum{#2}!3!230%
  }{%
    \expandafter\expandafter\expandafter\BitSet@@GetOct
    \expandafter\expandafter\expandafter1%
    \expandafter\expandafter\expandafter!%
    \expandafter\expandafter\expandafter!%
    \csname BS@#1\endcsname00%
    \BitSet@Empty\BitSet@Empty\BitSet@Empty!{#2}%
  }%
}
%    \end{macrocode}
%    \end{macro}
%    \begin{macro}{\bitsetGetHex}
%    \begin{macrocode}
\def\bitsetGetHex#1#2{%
  \romannumeral0%
  \bitsetIsEmpty{#1}{%
    \expandafter\expandafter\expandafter\BitSet@@GetOctHex
    \intcalcNum{#2}!4!340%
  }{%
    \expandafter\expandafter\expandafter\BitSet@@GetHex
    \expandafter\expandafter\expandafter1%
    \expandafter\expandafter\expandafter!%
    \expandafter\expandafter\expandafter!%
    \csname BS@#1\endcsname000%
    \BitSet@Empty\BitSet@Empty\BitSet@Empty\BitSet@Empty!{#2}%
  }%
}
%    \end{macrocode}
%    \end{macro}
%    \begin{macro}{\BitSet@@GetOct}
%    |#1|: number of digits\\
%    |#2|: result\\
%    |#3#4#5|: bits
%    \begin{macrocode}
\def\BitSet@@GetOct#1!#2!#3#4#5{%
  \ifx#5\BitSet@Empty
    \BitSet@AfterFi{%
      \expandafter\expandafter\expandafter\BitSet@GetOctHex
      \IntCalcDec#1!!#2!23%
    }%
  \else
    \BitSet@AfterFi{%
      \expandafter\expandafter\expandafter\BitSet@@GetOct
      \number\IntCalcInc#1!\expandafter\expandafter\expandafter!%
      \csname BitSet@Oct#5#4#3\endcsname#2!%
    }%
  \BitSet@Fi
}
%    \end{macrocode}
%    \end{macro}
%    \begin{macro}{\BitSet@Oct[000..111]}
%    \begin{macrocode}
\def\BitSet@Temp#1#2#3#4{%
  \expandafter\def\csname BitSet@Oct#1#2#3\endcsname{#4}%
}
\BitSet@Temp0000%
\BitSet@Temp0011%
\BitSet@Temp0102%
\BitSet@Temp0113%
\BitSet@Temp1004%
\BitSet@Temp1015%
\BitSet@Temp1106%
\BitSet@Temp1117%
%    \end{macrocode}
%    \end{macro}
%    \begin{macro}{\BitSet@@GetHex}
%    |#1|: number of digits\\
%    |#2|: result\\
%    |#3#4#5#6|: bits
%    \begin{macrocode}
\def\BitSet@@GetHex#1!#2!#3#4#5#6{%
  \ifx#6\BitSet@Empty
    \BitSet@AfterFi{%
      \expandafter\expandafter\expandafter\BitSet@GetOctHex
      \IntCalcDec#1!!#2!34%
    }%
  \else
    \BitSet@AfterFi{%
      \expandafter\expandafter\expandafter\BitSet@@GetHex
      \number\IntCalcInc#1!\expandafter\expandafter\expandafter!%
      \csname BitSet@Hex#6#5#4#3\endcsname#2!%
    }%
  \BitSet@Fi
}
%    \end{macrocode}
%    \end{macro}
%    \begin{macro}{\BitSet@Hex[0000..1111]}
%    \begin{macrocode}
\def\BitSet@Temp#1#2#3#4#5{%
  \expandafter\def\csname BitSet@Hex#1#2#3#4\endcsname{#5}%
}
\BitSet@Temp00000%
\BitSet@Temp00011%
\BitSet@Temp00102%
\BitSet@Temp00113%
\BitSet@Temp01004%
\BitSet@Temp01015%
\BitSet@Temp01106%
\BitSet@Temp01117%
\BitSet@Temp10008%
\BitSet@Temp10019%
\BitSet@Temp1010A%
\BitSet@Temp1011B%
\BitSet@Temp1100C%
\BitSet@Temp1101D%
\BitSet@Temp1110E%
\BitSet@Temp1111F%
%    \end{macrocode}
%    \end{macro}
%    \begin{macro}{\BitSet@GetOctHex}
%    Leading zeros $ (|#4| - |#1| * 3 + 2)/3 $ if $ |#4| > |#1| * 3$\\
%    |#1|: digit size\\
%    |#2|: result\\
%    |#3|: bits per digit - 1\\
%    |#4|: bits per digit
%    |#5|: garbage\\
%    |#6|: min size
%    \begin{macrocode}
\def\BitSet@GetOctHex#1!#2!#3#4#5!#6{%
  \expandafter\BitSet@@GetOctHex
  \number\intcalcNum{#6}\expandafter\expandafter\expandafter!%
  \IntCalcMul#1!#4!!#3#4#2%
}
%    \end{macrocode}
%    \end{macro}
%    \begin{macro}{\BitSet@@GetOctHex}
%    |#1|: plain min size\\
%    |#2|: digits * (bits per digit)\\
%    |#3|: bits per digit - 1\\
%    |#4|: bits per digit
%    \begin{macrocode}
\def\BitSet@@GetOctHex#1!#2!#3#4{%
  \ifnum#1>#2 %
    \BitSet@AfterFi{%
      \expandafter\expandafter\expandafter\expandafter
      \expandafter\expandafter\expandafter\BitSet@Fill
      \expandafter\IntCalcDiv\number
      \expandafter\expandafter\expandafter\IntCalcAdd
      \IntCalcSub#1!#2!!#3!!#4!!%
    }%
  \else
    \BitSet@AfterFi{ }%
  \BitSet@Fi
}
%    \end{macrocode}
%    \end{macro}
%
% \subsubsection{To decimal number}
%
%    \begin{macro}{\bitsetGetDec}
%    \begin{macrocode}
\def\bitsetGetDec#1{%
  \romannumeral0%
  \BitSet@IfUndefined{#1}{ 0}{%
    \expandafter\expandafter\expandafter\BitSet@GetDec
    \csname BS@#1\endcsname!%
  }%
}
%    \end{macrocode}
%    \end{macro}
%    \begin{macro}{\BitSet@GetDec}
%    \begin{macrocode}
\def\BitSet@GetDec#1#2!{%
  \ifx\\#2\\%
    \BitSet@AfterFi{ #1}%
  \else
    \BitSet@AfterFi{%
      \BitSet@@GetDec2!#1!#2!%
    }%
  \BitSet@Fi
}
%    \end{macrocode}
%    \end{macro}
%    \begin{macro}{\BitSet@@GetDec}
%    |#1|: power of two\\
%    |#2|: result\\
%    |#3#4|: number
%    \begin{macrocode}
\def\BitSet@@GetDec#1!#2!#3#4!{%
  \ifx\\#4\\%
    \ifx#31%
      \BitSet@AfterFiFi{%
        \expandafter\expandafter\expandafter\BitSet@Space
        \IntCalcAdd#1!#2!%
      }%
    \else
      \BitSet@AfterFiFi{ #2}%
    \fi
  \else
    \ifx#31%
      \BitSet@AfterFiFi{%
        \csname BitSet@N#1%
        \expandafter\expandafter\expandafter\endcsname
        \IntCalcAdd#1!#2!!#4!%
      }%
    \else
      \BitSet@AfterFiFi{%
        \csname BitSet@N#1\endcsname#2!#4!%
      }%
    \fi
  \BitSet@Fi
}
%    \end{macrocode}
%    \end{macro}
%    \begin{macro}{\BitSet@N[1,2,4,...]}
%    \begin{macrocode}
\def\BitSet@Temp#1#2{%
  \expandafter\def\csname BitSet@N#1\endcsname{%
    \BitSet@@GetDec#2!%
  }%
}
\BitSet@Temp{1}{2}
\BitSet@Temp{2}{4}
\BitSet@Temp{4}{8}
\BitSet@Temp{8}{16}
\BitSet@Temp{16}{32}
\BitSet@Temp{32}{64}
\BitSet@Temp{64}{128}
\BitSet@Temp{128}{256}
\BitSet@Temp{256}{512}
\BitSet@Temp{512}{1024}
\BitSet@Temp{1024}{2048}
\BitSet@Temp{2048}{4096}
\BitSet@Temp{4096}{8192}
\BitSet@Temp{8192}{16384}
\BitSet@Temp{16384}{32768}
\BitSet@Temp{32768}{65536}
\BitSet@Temp{65536}{131072}
\BitSet@Temp{131072}{262144}
\BitSet@Temp{262144}{524288}
\BitSet@Temp{524288}{1048576}
\BitSet@Temp{1048576}{2097152}
\BitSet@Temp{2097152}{4194304}
\BitSet@Temp{4194304}{8388608}
\BitSet@Temp{8388608}{16777216}
\BitSet@Temp{16777216}{33554432}
\BitSet@Temp{33554432}{67108864}
\BitSet@Temp{67108864}{134217728}
\BitSet@Temp{134217728}{268435456}
\BitSet@Temp{268435456}{536870912}
\BitSet@Temp{536870912}{1073741824}
%    \end{macrocode}
%    \end{macro}
%    \begin{macro}{\BitSet@N1073741824}
%    \begin{macrocode}
\expandafter\def\csname BitSet@N1073741824\endcsname{%
  \BitSet@GetDecBig2147483648!%
}%
%    \end{macrocode}
%    \end{macro}
%
%    \begin{macro}{\BitSet@GetDecBig}
%    |#1|: current power of two\\
%    |#2|: result\\
%    |#3#4|: number
%    \begin{macrocode}
\def\BitSet@GetDecBig#1!#2!#3#4!{%
  \ifx\\#4\\%
    \ifx#31%
      \BitSet@AfterFiFi{%
        \expandafter\expandafter\expandafter\BitSet@Space
        \BigIntCalcAdd#1!#2!%
      }%
    \else
      \BitSet@AfterFiFi{ #2}%
    \fi
  \else
    \ifx#31%
      \BitSet@AfterFiFi{%
        \expandafter\expandafter\expandafter\BitSet@@GetDecBig
        \BigIntCalcAdd#1!#2!!#1!#4!%
      }%
    \else
      \BitSet@AfterFiFi{%
        \expandafter\expandafter\expandafter\BitSet@GetDecBig
        \BigIntCalcShl#1!!#2!#4!%
      }%
    \fi
  \BitSet@Fi
}
%    \end{macrocode}
%    \end{macro}
%    \begin{macro}{\BitSet@@GetDecBig}
%    |#1|: result\\
%    |#2|: power of two\\
%    |#3#4|: number
%    \begin{macrocode}
\def\BitSet@@GetDecBig#1!#2!{%
  \expandafter\expandafter\expandafter\BitSet@GetDecBig
  \BigIntCalcShl#2!!#1!%
}
%    \end{macrocode}
%    \end{macro}
%
% \subsection{Logical operators}\label{src:log}
%
% \subsubsection{\cs{bitsetAnd}}
%
%    \begin{macro}{\bitsetAnd}
%    Decision table for \cs{bitsetAnd}:
%    \begin{quote}
%    \begin{tabular}{l|l|l|l|}
%                         & undef(B)   & empty(B)   & cardinality(B)$>$0\\
%      \hline
%      undef(A)           & A := empty & A := empty & A := empty\\
%      \hline
%      empty(A)           &            &            &\\
%      \hline
%      cardinality(A)$>$0 & A := empty & A := empty & A \&= B\\
%      \hline
%    \end{tabular}
%    \end{quote}
%    \begin{macrocode}
\def\bitsetAnd#1#2{%
  \bitsetIsEmpty{#1}{%
    \bitsetReset{#1}%
  }{%
    \bitsetIsEmpty{#2}{%
      \bitsetReset{#1}%
    }{%
      \expandafter\edef\csname BS@#1\endcsname{%
        \expandafter\expandafter\expandafter\BitSet@And
        \csname BS@#1\expandafter\expandafter\expandafter\endcsname
        \expandafter\expandafter\expandafter!%
        \csname BS@#2\endcsname!!%
      }%
      \expandafter\ifx\csname BS@#1\endcsname\BitSet@Empty
        \bitsetReset{#1}%
      \fi
    }%
  }%
}
%    \end{macrocode}
%    \end{macro}
%    \begin{macro}{\BitSet@And}
%    \begin{macrocode}
\def\BitSet@And#1#2!#3#4!#5!{%
  \ifx\\#2\\%
    \ifnum#1#3=11 #51\fi
  \else
    \ifx\\#4\\%
      \ifnum#1#3=11 #51\fi
    \else
      \ifnum#1#3=11 %
        #51%
        \BitSet@AfterFiFiFi{%
          \BitSet@And#2!#4!!%
        }%
      \else
        \BitSet@AfterFiFiFi{%
          \BitSet@And#2!#4!#50!%
        }%
      \fi
    \fi
  \BitSet@Fi
}
%    \end{macrocode}
%    \end{macro}
%
% \subsubsection{\cs{bitsetAndNot}}
%
%    \begin{macro}{\bitsetAndNot}
%    Decision table for \cs{bitsetAndNot}:
%    \begin{quote}
%    \begin{tabular}{l|l|l|l|}
%                         & undef(B)   & empty(B)   & cardinality(B)$>$0\\
%      \hline
%      undef(A)           & A := empty & A := empty & A := empty\\
%      \hline
%      empty(A)           &            &            &\\
%      \hline
%      cardinality(A)$>$0 &            &            & A \&= !B\\
%      \hline
%    \end{tabular}
%    \end{quote}
%    \begin{macrocode}
\def\bitsetAndNot#1#2{%
  \bitsetIsEmpty{#1}{%
    \bitsetReset{#1}%
  }{%
    \bitsetIsEmpty{#2}{%
    }{%
      \expandafter\edef\csname BS@#1\endcsname{%
        \expandafter\expandafter\expandafter\BitSet@AndNot
        \csname BS@#1\expandafter\expandafter\expandafter\endcsname
        \expandafter\expandafter\expandafter!%
        \csname BS@#2\endcsname!!%
      }%
      \expandafter\ifx\csname BS@#1\endcsname\BitSet@Empty
        \bitsetReset{#1}%
      \fi
    }%
  }%
}
%    \end{macrocode}
%    \end{macro}
%    \begin{macro}{\BitSet@AndNot}
%    \begin{macrocode}
\def\BitSet@AndNot#1#2!#3#4!#5!{%
  \ifx\\#2\\%
    \ifnum#1#3=10 #51\fi
  \else
    \ifx\\#4\\%
      #5%
      \ifnum#1#3=10 1\else 0\fi
      #2%
    \else
      \ifnum#1#3=10 %
        #51%
        \BitSet@AfterFiFiFi{%
          \BitSet@AndNot#2!#4!!%
        }%
      \else
        \BitSet@AfterFiFiFi{%
          \BitSet@AndNot#2!#4!#50!%
        }%
      \fi
    \fi
  \BitSet@Fi
}
%    \end{macrocode}
%    \end{macro}
%
% \subsubsection{\cs{bitsetOr}}
%
%    \begin{macro}{\bitsetOr}
%    Decision table for \cs{bitsetOr}:
%    \begin{quote}
%    \begin{tabular}{l|l|l|l|}
%                         & undef(B)   & empty(B)   & cardinality(B)$>$0\\
%      \hline
%      undef(A)           & A := empty & A := empty & A := B\\
%      \hline
%      empty(A)           &            &            & A := B\\
%      \hline
%      cardinality(A)$>$0 &            &            & A \string|= B\\
%      \hline
%    \end{tabular}
%    \end{quote}
%    \begin{macrocode}
\def\bitsetOr#1#2{%
  \bitsetIsEmpty{#2}{%
    \BitSet@IfUndefined{#1}{\bitsetReset{#1}}{}%
  }{%
    \bitsetIsEmpty{#1}{%
      \expandafter\let\csname BS@#1\expandafter\endcsname
                      \csname BS@#2\endcsname
    }{%
      \expandafter\edef\csname BS@#1\endcsname{%
        \expandafter\expandafter\expandafter\BitSet@Or
        \csname BS@#1\expandafter\expandafter\expandafter\endcsname
        \expandafter\expandafter\expandafter!%
        \csname BS@#2\endcsname!%
      }%
    }%
  }%
}
%    \end{macrocode}
%    \end{macro}
%    \begin{macro}{\BitSet@Or}
%    \begin{macrocode}
\def\BitSet@Or#1#2!#3#4!{%
  \ifnum#1#3>0 1\else 0\fi
  \ifx\\#2\\%
    #4%
  \else
    \ifx\\#4\\%
      #2%
    \else
      \BitSet@AfterFiFi{%
        \BitSet@Or#2!#4!%
      }%
    \fi
  \BitSet@Fi
}
%    \end{macrocode}
%    \end{macro}
%
% \subsubsection{\cs{bitsetXor}}
%
%    \begin{macro}{\bitsetXor}
%    Decision table for \cs{bitsetXor}:
%    \begin{quote}
%    \begin{tabular}{l|l|l|l|}
%                         & undef(B)   & empty(B)   & cardinality(B)$>$0\\
%      \hline
%      undef(A)           & A := empty & A := empty & A := B\\
%      \hline
%      empty(A)           &            &            & A := B\\
%      \hline
%      cardinality(A)$>$0 &            &            & A \string^= B\\
%      \hline
%    \end{tabular}
%    \end{quote}
%    \begin{macrocode}
\def\bitsetXor#1#2{%
  \bitsetIsEmpty{#2}{%
    \BitSet@IfUndefined{#1}{\bitsetReset{#1}}{}%
  }{%
    \bitsetIsEmpty{#1}{%
      \expandafter\let\csname BS@#1\expandafter\endcsname
                      \csname BS@#2\endcsname
    }{%
      \expandafter\edef\csname BS@#1\endcsname{%
        \expandafter\expandafter\expandafter\BitSet@Xor
        \csname BS@#1\expandafter\expandafter\expandafter\endcsname
        \expandafter\expandafter\expandafter!%
        \csname BS@#2\endcsname!!%
      }%
      \expandafter\ifx\csname BS@#1\endcsname\BitSet@Empty
        \bitsetReset{#1}%
      \fi
    }%
  }%
}
%    \end{macrocode}
%    \end{macro}
%    \begin{macro}{\BitSet@Xor}
%    \begin{macrocode}
\def\BitSet@Xor#1#2!#3#4!#5!{%
  \ifx\\#2\\%
    \ifx#1#3%
      \ifx\\#4\\%
      \else
        #50#4%
      \fi
    \else
      #51#4%
    \fi
  \else
    \ifx\\#4\\%
      #5%
      \ifx#1#30\else 1\fi
      #2%
    \else
      \ifx#1#3%
        \BitSet@AfterFiFiFi{%
          \BitSet@Xor#2!#4!#50!%
        }%
      \else
        #51%
        \BitSet@AfterFiFiFi{%
          \BitSet@Xor#2!#4!!%
        }%
      \fi
    \fi
  \BitSet@Fi
}
%    \end{macrocode}
%    \end{macro}
%
% \subsubsection{Shifting}\label{src:shift}
%
% \subsubsection{\cs{bitsetShiftLeft}}
%
%    \begin{macro}{\bitsetShiftLeft}
%    \begin{macrocode}
\def\bitsetShiftLeft#1#2{%
  \BitSet@IfUndefined{#1}{%
    \bitsetReset{#1}%
  }{%
    \bitsetIsEmpty{#1}{%
    }{%
      \expandafter\expandafter\expandafter\BitSet@ShiftLeft
      \intcalcNum{#2}!{#1}%
    }%
  }%
}
%    \end{macrocode}
%    \end{macro}
%    \begin{macro}{\BitSet@ShiftLeft}
%    \begin{macrocode}
\def\BitSet@ShiftLeft#1!#2{%
  \ifcase\intcalcSgn{#1} %
  \or
    \begingroup
      \uccode`m=`0 %
    \uppercase\expandafter{\expandafter\endgroup
      \expandafter\edef\csname BS@#2\expandafter\endcsname
      \expandafter{%
        \romannumeral#1000\expandafter\BitSet@Space
        \csname BS@#2\endcsname
      }%
    }%
  \else
    \expandafter\BitSet@ShiftRight\BitSet@Gobble#1!{#2}%
  \fi
}
%    \end{macrocode}
%    \end{macro}
%
% \subsubsection{\cs{bitsetShiftRight}}
%
%    \begin{macro}{\bitsetShiftRight}
%    \begin{macrocode}
\def\bitsetShiftRight#1#2{%
  \BitSet@IfUndefined{#1}{%
    \bitsetReset{#1}%
  }{%
    \bitsetIsEmpty{#1}{%
    }{%
      \expandafter\expandafter\expandafter\BitSet@ShiftRight
      \intcalcNum{#2}!{#1}%
    }%
  }%
}
%    \end{macrocode}
%    \end{macro}
%    \begin{macro}{\BitSet@ShiftRight}
%    \begin{macrocode}
\def\BitSet@ShiftRight#1!#2{%
  \ifcase\intcalcSgn{#1} %
  \or
    \expandafter\edef\csname BS@#2\endcsname{%
      \expandafter\expandafter\expandafter\BitSet@Kill
      \csname BS@#2\expandafter\endcsname\expandafter\BitSet@Empty
      \expandafter=%
      \expandafter{\expandafter}\expandafter{\expandafter}%
      \romannumeral#1000!%
    }%
  \else
    \expandafter\BitSet@ShiftLeft\BitSet@Gobble#1!{#2}%
  \fi
}
%    \end{macrocode}
%    \end{macro}
%    \begin{macro}{\BitSet@Kill}
%    \begin{macrocode}
\def\BitSet@Kill#1#2=#3#4#5{%
  #3#4%
  \ifx#5!%
    \ifx#1\BitSet@Empty
      0%
    \else
      #1#2%
    \fi
  \else
    \ifx#1\BitSet@Empty
      0%
      \BitSet@AfterFiFi\BitSet@Cleanup
    \else
      \BitSet@Kill#2=%
    \fi
  \BitSet@Fi
}
%    \end{macrocode}
%    \end{macro}
%
% \subsection{Bit manipulation}\label{src:bitman}
%
%    \begin{macro}{\bitsetClear}
%    \begin{macrocode}
\def\bitsetClear{%
  \BitSet@CheckIndex\BitSet@Clear
}
%    \end{macrocode}
%    \end{macro}
%    \begin{macro}{\bitsetSet}
%    \begin{macrocode}
\def\bitsetSet{%
  \BitSet@CheckIndex\BitSet@Set
}
%    \end{macrocode}
%    \end{macro}
%    \begin{macro}{\bitsetFlip}
%    \begin{macrocode}
\def\bitsetFlip{%
  \BitSet@CheckIndex\BitSet@Flip
}
%    \end{macrocode}
%    \end{macro}
%    \begin{macro}{\bitsetSetValue}
%    \begin{macrocode}
\def\bitsetSetValue#1#2#3{%
  \expandafter\expandafter\expandafter\BitSet@SetValue
  \intcalcNum{#3}!{#1}{#2}%
}
%    \end{macrocode}
%    \end{macro}
%    \begin{macro}{\BitSet@SetValue}
%    |#1|: plain value\\
%    |#2|: BitSet\\
%    |#3|: Index
%    \begin{macrocode}
\def\BitSet@SetValue#1!{%
  \BitSet@CheckIndex{%
    \ifcase#1 %
      \expandafter\BitSet@Clear
    \or
      \expandafter\BitSet@Set
    \else
      \BitSet@ErrorInvalidBitValue{#1}%
      \expandafter\expandafter\expandafter\BitSet@Gobble
      \expandafter\BitSet@Gobble
    \fi
  }%
}
%    \end{macrocode}
%    \end{macro}
%    \begin{macro}{\BitSet@ErrorInvalidBitValue}
%    |#1|: Wrong bit value
%    \begin{macrocode}
\def\BitSet@ErrorInvalidBitValue#1{%
  \@PackageError{bitset}{%
    Invalid bit value (#1) not in range 0..1%
  }\@ehc
}
%    \end{macrocode}
%    \end{macro}
%
% \subsubsection{Clear operation}
%
%    \begin{macro}{\BitSet@Clear}
%    |#1|: BitSet\\
%    |#2|: plain and checked index
%    \begin{macrocode}
\def\BitSet@Clear#1#2{%
  \edef\BitSet@Temp{%
    \expandafter\expandafter\expandafter\BitSet@@Clear
    \csname BS@#1\expandafter\endcsname
    \expandafter\BitSet@Empty\expandafter=\expandafter!%
    \romannumeral#2000!%
  }%
  \expandafter\let\csname BS@#1\expandafter\endcsname
  \ifx\BitSet@Temp\BitSet@Empty
    \BitSet@Zero
  \else
    \BitSet@Temp
  \fi
}
%    \end{macrocode}
%    \end{macro}
%    \begin{macro}{\BitSet@@Clear}
%    \begin{macrocode}
\def\BitSet@@Clear#1#2=#3!#4{%
  \ifx#4!%
    \ifx#1\BitSet@Empty
    \else
      \ifx\BitSet@Empty#2%
      \else
        #30#2%
      \fi
    \fi
  \else
    \ifx#1\BitSet@Empty
      \BitSet@AfterFiFi\BitSet@Cleanup
    \else
      \ifx#10%
        \BitSet@AfterFiFiFi{%
          \BitSet@@Clear#2=#30!%
        }%
      \else
        #31%
        \BitSet@AfterFiFiFi{%
          \BitSet@@Clear#2=!%
        }%
      \fi
    \fi
  \BitSet@Fi
}
%    \end{macrocode}
%    \end{macro}
%
% \subsubsection{Set operation}
%
%    \begin{macro}{\BitSet@Set}
%    |#1|: BitSet\\
%    |#2|: plain and checked Index
%    \begin{macrocode}
\def\BitSet@Set#1#2{%
  \expandafter\edef\csname BS@#1\endcsname{%
    \expandafter\expandafter\expandafter\BitSet@@Set
    \csname BS@#1\expandafter\endcsname
    \expandafter\BitSet@Empty\expandafter=%
    \expandafter{\expandafter}\expandafter{\expandafter}%
    \romannumeral#2000!%
  }%
}
%    \end{macrocode}
%    \end{macro}
%    \begin{macro}{\BitSet@@Set}
%    \begin{macrocode}
\def\BitSet@@Set#1#2=#3#4#5{%
  #3#4%
  \ifx#5!%
    1#2%
  \else
    \ifx#1\BitSet@Empty
      0%
      \BitSet@AfterFiFi\BitSet@@@Set
    \else
      #1%
      \BitSet@@Set#2=%
    \fi
  \BitSet@Fi
}
%    \end{macrocode}
%    \end{macro}
%    \begin{macro}{\BitSet@@@Set}
%    \begin{macrocode}
\def\BitSet@@@Set#1{%
  \ifx#1!%
    1%
  \else
    0%
    \expandafter\BitSet@@@Set
  \fi
}
%    \end{macrocode}
%    \end{macro}
%
% \subsubsection{Flip operation}
%
%    \begin{macro}{\BitSet@Flip}
%    |#1|: BitSet\\
%    |#2|: plain and checked Index
%    \begin{macrocode}
\def\BitSet@Flip#1#2{%
  \edef\BitSet@Temp{%
    \expandafter\expandafter\expandafter\BitSet@@Flip
    \csname BS@#1\expandafter\endcsname
    \expandafter\BitSet@Empty\expandafter=\expandafter!%
    \romannumeral#2000!%
  }%
  \expandafter\let\csname BS@#1\expandafter\endcsname
  \ifx\BitSet@Temp\BitSet@Empty
    \BitSet@Zero
  \else
    \BitSet@Temp
  \fi
}
%    \end{macrocode}
%    \end{macro}
%    \begin{macro}{\BitSet@@Flip}
%    \begin{macrocode}
\def\BitSet@@Flip#1#2=#3!#4{%
  \ifx#4!%
    \ifx#11%
      \ifx\BitSet@Empty#2%
      \else
        #30#2%
      \fi
    \else
      #31#2%
    \fi
  \else
    \ifx#1\BitSet@Empty
      #30%
      \BitSet@AfterFiFi\BitSet@@@Set
    \else
      \ifx#10%
        \BitSet@AfterFiFiFi{%
          \BitSet@@Flip#2=#30!%
        }%
      \else
        #31%
        \BitSet@AfterFiFiFi{%
          \BitSet@@Flip#2=!%
        }%
      \fi
    \fi
  \BitSet@Fi
}
%    \end{macrocode}
%    \end{macro}
%
% \subsubsection{Range operators}
%
%    \begin{macro}{\bitsetClearRange}
%    \begin{macrocode}
\def\bitsetClearRange{%
  \BitSet@Range\BitSet@Clear
}
%    \end{macrocode}
%    \end{macro}
%    \begin{macro}{\bitsetSetRange}
%    \begin{macrocode}
\def\bitsetSetRange{%
  \BitSet@Range\BitSet@Set
}
%    \end{macrocode}
%    \end{macro}
%    \begin{macro}{\bitsetFlipRange}
%    \begin{macrocode}
\def\bitsetFlipRange{%
  \BitSet@Range\BitSet@Flip
}
%    \end{macrocode}
%    \end{macro}
%    \begin{macro}{\bitsetSetValueRange}
%    \begin{macrocode}
\def\bitsetSetValueRange#1#2#3#4{%
  \expandafter\expandafter\expandafter\BitSet@SetValueRange
  \intcalcNum{#4}!{#1}{#2}{#3}%
}
%    \end{macrocode}
%    \end{macro}
%    \begin{macro}{\BitSet@SetValueRange}
%    \begin{macrocode}
\def\BitSet@SetValueRange#1!#2#3#4{%
  \ifcase#1 %
    \BitSet@Range\BitSet@Clear{#2}{#3}{#4}%
  \or
    \BitSet@Range\BitSet@Set{#2}{#3}{#4}%
  \else
    \BitSet@ErrorInvalidBitValue{#1}%
  \fi
}
%    \end{macrocode}
%    \end{macro}
%
%    \begin{macro}{\BitSet@Range}
%    |#1|: clear/set/flip macro\\
%    |#2|: BitSet\\
%    |#3|: Index from\\
%    |#4|: Index to
%    \begin{macrocode}
\def\BitSet@Range#1#2#3#4{%
  \edef\BitSet@Temp{%
    \noexpand\BitSet@@Range\noexpand#1{#2}%
    \intcalcNum{#3}!\intcalcNum{#4}!%
  }%
  \BitSet@Temp
}
%    \end{macrocode}
%    \end{macro}
%    \begin{macro}{\BitSet@@Range}
%    |#1|: clear/set/flip macro\\
%    |#2|: BitSet\\
%    |#3|: Index from\\
%    |#4|: Index to
%    \begin{macrocode}
\def\BitSet@@Range#1#2#3!#4!{%
  \ifnum#3<0 %
    \BitSet@NegativeIndex#1{#2}#3!#4!0!#4!%
  \else
    \ifnum#4<0 %
      \BitSet@NegativeIndex#1{#2}#3!#4!#3!0!%
    \else
      \ifcase\intcalcCmp{#3}{#4} %
      \or
        \@PackageError{bitset}{%
          Wrong index numbers in range [#3..#4]\MessageBreak% hash-ok
          for clear/set/flip on bit set `#2'.\MessageBreak
          The lower index exceeds the upper index.\MessageBreak
          Canceling the operation as error recovery%
        }\@ehc
      \else
        \BitSet@@@Range#3!#4!#1{#2}%
      \fi
    \fi
  \fi
}
%    \end{macrocode}
%    \end{macro}
%    \begin{macro}{\BitSet@NegativeIndex}
%    \begin{macrocode}
\def\BitSet@NegativeIndex#1#2#3!#4!#5!#6!{%
  \@PackageError{bitset}{%
    Negative index in range [#3..#4]\MessageBreak % hash-ok
    for \string\bitset
    \ifx#1\BitSet@Clear
      Clear%
    \else
      \ifx#1\BitSet@Set
        Set%
      \else
        Flip%
      \fi
    \fi
    Range on bit set `#2'.\MessageBreak
    Using [#5..#6] as error recovery% hash-ok
  }\@ehc
  \BitSet@@Range#1{#2}#5!#6!%
}
%    \end{macrocode}
%    \end{macro}
%    \begin{macro}{\BitSet@@Range}
%    \begin{macrocode}
\def\BitSet@@@Range#1!#2!#3#4{%
  \ifnum#1<#2 %
    #3{#4}{#1}%
    \BitSet@AfterFi{%
      \expandafter\expandafter\expandafter\BitSet@@@Range
      \IntCalcInc#1!!#2!#3{#4}%
    }%
  \BitSet@Fi
}
%    \end{macrocode}
%    \end{macro}
%
% \subsection{Bit retrieval}\label{src:bitret}
%
% \subsubsection{\cs{bitsetGet}}
%
%    \begin{macro}{\bitsetGet}
%    \begin{macrocode}
\def\bitsetGet#1#2{%
  \number
  \expandafter\expandafter\expandafter\BitSet@Get
  \intcalcNum{#2}!{#1}%
}
%    \end{macrocode}
%    \end{macro}
%    \begin{macro}{\BitSet@Get}
%    |#1|: plain index\\
%    |#2|: BitSet
%    \begin{macrocode}
\def\BitSet@Get#1!#2{%
  \ifnum#1<0 %
    \BitSet@AfterFi{%
      0 \BitSetError:NegativeIndex%
    }%
  \else
    \BitSet@IfUndefined{#2}{0}{%
      \expandafter\expandafter\expandafter\BitSet@@Get
      \csname BS@#2\expandafter\endcsname
      \expandafter!\expandafter=%
      \expandafter{\expandafter}\expandafter{\expandafter}%
      \romannumeral\intcalcNum{#1}000!%
    }%
    \expandafter\BitSet@Space
  \BitSet@Fi
}
%    \end{macrocode}
%    \end{macro}
%    \begin{macro}{\BitSet@@Get}
%    \begin{macrocode}
\def\BitSet@@Get#1#2=#3#4#5{%
  #3#4%
  \ifx#5!%
    \ifx#1!%
      0%
    \else
      #1%
    \fi
  \else
    \ifx#1!%
      0%
      \BitSet@AfterFiFi\BitSet@Cleanup
    \else
      \BitSet@@Get#2=%
    \fi
  \BitSet@Fi
}
%    \end{macrocode}
%    \end{macro}
%
% \subsubsection{\cs{bitsetNextClearBit}, \cs{bitsetNextSetBit}}
%
%    \begin{macro}{\bitsetNextClearBit}
%    \begin{macrocode}
\def\bitsetNextClearBit#1#2{%
  \number
  \expandafter\expandafter\expandafter\BitSet@NextClearBit
  \intcalcNum{#2}!{#1} %
}
%    \end{macrocode}
%    \end{macro}
%    \begin{macro}{\BitSet@NextClearBit}
%    |#1|: Index\\
%    |#2|: BitSet
%    \begin{macrocode}
\def\BitSet@NextClearBit#1!#2{%
  \ifnum#1<0 %
    \BitSet@NextClearBit0!{#2}%
    \BitSet@AfterFi{%
      \expandafter\BitSet@Space
      \expandafter\BitSetError:NegativeIndex\romannumeral0%
    }%
  \else
    \bitsetIsEmpty{#2}{#1}{%
      \expandafter\BitSet@Skip
      \number#1\expandafter\expandafter\expandafter!%
      \csname BS@#2\endcsname!!!!!!!!!=%
      {\BitSet@@NextClearBit#1!}%
    }%
  \BitSet@Fi
}
%    \end{macrocode}
%    \end{macro}
%    \begin{macro}{\BitSet@@NextClearBit}
%    |#1|: index for next bit in |#2|\\
%    |#2|: next bit
%    \begin{macrocode}
\def\BitSet@@NextClearBit#1!#2{%
  \ifx#2!%
    #1%
  \else
    \ifx#20%
      #1%
      \BitSet@AfterFiFi\BitSet@Cleanup
    \else
      \BitSet@AfterFiFi{%
        \expandafter\expandafter\expandafter\BitSet@@NextClearBit
        \IntCalcInc#1!!%
      }%
    \fi
  \BitSet@Fi
}
%    \end{macrocode}
%    \end{macro}
%
%    \begin{macro}{\bitsetNextSetBit}
%    \begin{macrocode}
\def\bitsetNextSetBit#1#2{%
  \number
  \expandafter\expandafter\expandafter\BitSet@NextSetBit
  \intcalcNum{#2}!{#1} %
}
%    \end{macrocode}
%    \end{macro}
%    \begin{macro}{\BitSet@NextSetBit}
%    |#1|: Index\\
%    |#2|: BitSet
%    \begin{macrocode}
\def\BitSet@NextSetBit#1!#2{%
  \ifnum#1<0 %
    \BitSet@NextSetBit0!{#2}%
    \BitSet@AfterFi{%
      \expandafter\BitSet@Space
      \expandafter\BitSetError:NegativeIndex\romannumeral0%
    }%
  \else
    \bitsetIsEmpty{#2}{-1}{%
      \expandafter\BitSet@Skip
      \number#1\expandafter\expandafter\expandafter!%
      \csname BS@#2\endcsname!!!!!!!!!=%
      {\BitSet@@NextSetBit#1!}%
    }%
  \BitSet@Fi
}
%    \end{macrocode}
%    \end{macro}
%    \begin{macro}{\BitSet@@NextSetBit}
%    |#1|: index for next bit in |#2|\\
%    |#2|: next bit
%    \begin{macrocode}
\def\BitSet@@NextSetBit#1!#2{%
  \ifx#2!%
    -1%
  \else
    \ifx#21%
      #1%
      \BitSet@AfterFiFi\BitSet@Cleanup
    \else
      \BitSet@AfterFiFi{%
        \expandafter\expandafter\expandafter\BitSet@@NextSetBit
        \IntCalcInc#1!!%
      }%
    \fi
  \BitSet@Fi
}
%    \end{macrocode}
%    \end{macro}
%    \begin{macro}{\BitSet@Cleanup}
%    \begin{macrocode}
\def\BitSet@Cleanup#1!{}
%    \end{macrocode}
%    \end{macro}
%
%    \begin{macro}{\BitSet@Skip}
%    |#1|: number of bits to skip\\
%    |#2|: bits\\
%    |#3|: continuation code
%    \begin{macrocode}
\def\BitSet@Skip#1!#2{%
  \ifx#2!%
    \BitSet@AfterFi{%
      \BitSet@SkipContinue%
    }%
  \else
    \ifcase#1 %
      \BitSet@AfterFiFi{%
        \BitSet@SkipContinue#2%
      }%
    \or
      \BitSet@AfterFiFi\BitSet@SkipContinue
    \or
      \BitSet@AfterFiFi{%
        \expandafter\BitSet@SkipContinue\BitSet@Gobble
      }%
    \else
      \ifnum#1>8 %
        \BitSet@AfterFiFiFi{%
          \expandafter\BitSet@Skip
          \number\IntCalcSub#1!8!\expandafter!%
          \BitSet@GobbleSeven
        }%
      \else
        \BitSet@AfterFiFiFi{%
          \expandafter\expandafter\expandafter\BitSet@Skip
          \IntCalcDec#1!!%
        }%
      \fi
    \fi
  \BitSet@Fi
}
%    \end{macrocode}
%    \end{macro}
%    \begin{macro}{\BitSet@SkipContinue}
%    |#1|: remaining bits\\
%    |#2|: continuation code
%    \begin{macrocode}
\def\BitSet@SkipContinue#1!#2=#3{%
  #3#1!%
}
%    \end{macrocode}
%    \end{macro}
%    \begin{macro}{\BitSet@GobbleSeven}
%    \begin{macrocode}
\def\BitSet@GobbleSeven#1#2#3#4#5#6#7{}
%    \end{macrocode}
%    \end{macro}
%
% \subsubsection{\cs{bitsetGetSetBitList}}
%
%    \begin{macro}{\bitsetGetSetBitList}
%    It's just a wrapper for \cs{bitsetNextSetBit}.
%    \begin{macrocode}
\def\bitsetGetSetBitList#1{%
  \romannumeral0%
  \bitsetIsEmpty{#1}{ }{%
    \expandafter\BitSet@GetSetBitList
    \number\BitSet@NextSetBit0!{#1}!{#1}{}!%
  }%
}
%    \end{macrocode}
%    \end{macro}
%    \begin{macro}{\BitSet@GetSetBitList}
%    |#1|: found index\\
%    |#2|: BitSet\\
%    |#3|: comma
%    |#4|: result
%    \begin{macrocode}
\def\BitSet@GetSetBitList#1!#2#3#4!{%
  \ifnum#1<0 %
    \BitSet@AfterFi{ #4}%
  \else
    \BitSet@AfterFi{%
      \expandafter\BitSet@GetSetBitList\number
      \expandafter\expandafter\expandafter\BitSet@NextSetBit
      \IntCalcInc#1!!{#2}!{#2},#4#3#1!%
    }%
  \BitSet@Fi
}
%    \end{macrocode}
%    \end{macro}
%
% \subsection{Bit set properties}\label{src:prop}
%
%    \begin{macro}{\bitsetSize}
%    \begin{macrocode}
\def\bitsetSize#1{%
  \number
  \BitSet@IfUndefined{#1}{0 }{%
    \expandafter\expandafter\expandafter\BitSet@Size
    \expandafter\expandafter\expandafter1%
    \expandafter\expandafter\expandafter!%
    \csname BS@#1\endcsname!0!%
  }%
}
%    \end{macrocode}
%    \end{macro}
%    \begin{macro}{\BitSet@Size}
%    |#1|: counter\\
%    |#2#3|: bits\\
%    |#4|: result
%    \begin{macrocode}
\def\BitSet@Size#1!#2#3!#4!{%
  \ifx#21%
    \ifx\\#3\\%
      \BitSet@AfterFiFi{#1 }%
    \else
      \BitSet@AfterFiFi{%
        \expandafter\expandafter\expandafter\BitSet@Size
        \IntCalcInc#1!!#3!#1!%
      }%
    \fi
  \else
    \ifx\\#3\\%
      \BitSet@AfterFiFi{#4 }%
    \else
      \BitSet@AfterFiFi{%
        \expandafter\expandafter\expandafter\BitSet@Size
        \IntCalcInc#1!!#3!#4!%
      }%
    \fi
  \fi
  \BitSet@Fi
}
%    \end{macrocode}
%    \end{macro}
%
%    \begin{macro}{\bitsetCardinality}
%    \begin{macrocode}
\def\bitsetCardinality#1{%
  \number
  \BitSet@IfUndefined{#1}{0 }{%
    \expandafter\expandafter\expandafter\BitSet@Cardinality
    \expandafter\expandafter\expandafter0%
    \expandafter\expandafter\expandafter!%
    \csname BS@#1\endcsname!%
  }%
}
%    \end{macrocode}
%    \end{macro}
%    \begin{macro}{\BitSet@Cardinality}
%    |#1|: result\\
%    |#2#3|: bits
%    \begin{macrocode}
\def\BitSet@Cardinality#1!#2#3!{%
  \ifx#21%
    \ifx\\#3\\%
      \BitSet@AfterFiFi{\IntCalcInc#1! }%
    \else
      \BitSet@AfterFiFi{%
        \expandafter\expandafter\expandafter\BitSet@Cardinality
        \IntCalcInc#1!!#3!%
      }%
    \fi
  \else
    \ifx\\#3\\%
      \BitSet@AfterFiFi{#1 }%
    \else
      \BitSet@AfterFiFi{%
        \BitSet@Cardinality#1!#3!%
      }%
    \fi
  \fi
  \BitSet@Fi
}
%    \end{macrocode}
%    \end{macro}
%
% \subsection{Queries}\label{src:query}
%
%    \begin{macro}{\bitsetIsDefined}
%    \begin{macrocode}
\def\bitsetIsDefined#1{%
  \BitSet@IfUndefined{#1}%
  \BitSet@SecondOfTwo
  \BitSet@FirstOfTwo
}
%    \end{macrocode}
%    \end{macro}
%    \begin{macro}{\bitsetIsEmpty}
%    \begin{macrocode}
\def\bitsetIsEmpty#1{%
  \BitSet@IfUndefined{#1}\BitSet@FirstOfTwo{%
    \expandafter\ifx\csname BS@#1\endcsname\BitSet@Zero
      \expandafter\BitSet@FirstOfTwo
    \else
      \expandafter\BitSet@SecondOfTwo
    \fi
  }%
}
%    \end{macrocode}
%    \end{macro}
%    \begin{macro}{\BitSet@Zero}
%    \begin{macrocode}
\def\BitSet@Zero{0}
%    \end{macrocode}
%    \end{macro}
%
%    \begin{macro}{\bitsetQuery}
%    \begin{macrocode}
\def\bitsetQuery#1#2{%
  \ifnum\bitsetGet{#1}{#2}=1 %
    \expandafter\BitSet@FirstOfTwo
  \else
    \expandafter\BitSet@SecondOfTwo
  \fi
}
%    \end{macrocode}
%    \end{macro}
%
%    \begin{macro}{\bitsetEquals}
%    \begin{macrocode}
\def\bitsetEquals#1#2{%
  \BitSet@IfUndefined{#1}{%
    \BitSet@IfUndefined{#2}\BitSet@FirstOfTwo\BitSet@SecondOfTwo
  }{%
    \BitSet@IfUndefined{#2}\BitSet@SecondOfTwo{%
      \expandafter\ifx\csname BS@#1\expandafter\endcsname
                      \csname BS@#2\endcsname
        \expandafter\BitSet@FirstOfTwo
      \else
        \expandafter\BitSet@SecondOfTwo
      \fi
    }%
  }%
}
%    \end{macrocode}
%    \end{macro}
%
%    \begin{macro}{\bitsetIntersects}
%    \begin{macrocode}
\def\bitsetIntersects#1#2{%
  \bitsetIsEmpty{#1}\BitSet@SecondOfTwo{%
    \bitsetIsEmpty{#2}\BitSet@SecondOfTwo{%
      \expandafter\expandafter\expandafter\BitSet@Intersects
      \csname BS@#1\expandafter\expandafter\expandafter\endcsname
      \expandafter\expandafter\expandafter!%
      \csname BS@#2\endcsname!%
    }%
  }%
}
%    \end{macrocode}
%    \end{macro}
%    \begin{macro}{\BitSet@Intersects}
%    \begin{macrocode}
\def\BitSet@Intersects#1#2!#3#4!{%
  \ifnum#1#3=11 %
    \BitSet@AfterFi\BitSet@FirstOfTwo
  \else
    \ifx\\#2\\%
      \BitSet@AfterFiFi\BitSet@SecondOfTwo
    \else
      \ifx\\#4\\%
        \BitSet@AfterFiFiFi\BitSet@SecondOfTwo
      \else
        \BitSet@AfterFiFiFi{%
          \BitSet@Intersects#2!#4!%
        }%
      \fi
    \fi
  \BitSet@Fi
}
%    \end{macrocode}
%    \end{macro}
%
%    \begin{macrocode}
\BitSet@AtEnd%
%</package>
%    \end{macrocode}
%
% \section{Test}
%
% \subsection{Catcode checks for loading}
%
%    \begin{macrocode}
%<*test1>
%    \end{macrocode}
%    \begin{macrocode}
\catcode`\{=1 %
\catcode`\}=2 %
\catcode`\#=6 %
\catcode`\@=11 %
\expandafter\ifx\csname count@\endcsname\relax
  \countdef\count@=255 %
\fi
\expandafter\ifx\csname @gobble\endcsname\relax
  \long\def\@gobble#1{}%
\fi
\expandafter\ifx\csname @firstofone\endcsname\relax
  \long\def\@firstofone#1{#1}%
\fi
\expandafter\ifx\csname loop\endcsname\relax
  \expandafter\@firstofone
\else
  \expandafter\@gobble
\fi
{%
  \def\loop#1\repeat{%
    \def\body{#1}%
    \iterate
  }%
  \def\iterate{%
    \body
      \let\next\iterate
    \else
      \let\next\relax
    \fi
    \next
  }%
  \let\repeat=\fi
}%
\def\RestoreCatcodes{}
\count@=0 %
\loop
  \edef\RestoreCatcodes{%
    \RestoreCatcodes
    \catcode\the\count@=\the\catcode\count@\relax
  }%
\ifnum\count@<255 %
  \advance\count@ 1 %
\repeat

\def\RangeCatcodeInvalid#1#2{%
  \count@=#1\relax
  \loop
    \catcode\count@=15 %
  \ifnum\count@<#2\relax
    \advance\count@ 1 %
  \repeat
}
\def\RangeCatcodeCheck#1#2#3{%
  \count@=#1\relax
  \loop
    \ifnum#3=\catcode\count@
    \else
      \errmessage{%
        Character \the\count@\space
        with wrong catcode \the\catcode\count@\space
        instead of \number#3%
      }%
    \fi
  \ifnum\count@<#2\relax
    \advance\count@ 1 %
  \repeat
}
\def\space{ }
\expandafter\ifx\csname LoadCommand\endcsname\relax
  \def\LoadCommand{\input bitset.sty\relax}%
\fi
\def\Test{%
  \RangeCatcodeInvalid{0}{47}%
  \RangeCatcodeInvalid{58}{64}%
  \RangeCatcodeInvalid{91}{96}%
  \RangeCatcodeInvalid{123}{255}%
  \catcode`\@=12 %
  \catcode`\\=0 %
  \catcode`\%=14 %
  \LoadCommand
  \RangeCatcodeCheck{0}{36}{15}%
  \RangeCatcodeCheck{37}{37}{14}%
  \RangeCatcodeCheck{38}{47}{15}%
  \RangeCatcodeCheck{48}{57}{12}%
  \RangeCatcodeCheck{58}{63}{15}%
  \RangeCatcodeCheck{64}{64}{12}%
  \RangeCatcodeCheck{65}{90}{11}%
  \RangeCatcodeCheck{91}{91}{15}%
  \RangeCatcodeCheck{92}{92}{0}%
  \RangeCatcodeCheck{93}{96}{15}%
  \RangeCatcodeCheck{97}{122}{11}%
  \RangeCatcodeCheck{123}{255}{15}%
  \RestoreCatcodes
}
\Test
\csname @@end\endcsname
\end
%    \end{macrocode}
%    \begin{macrocode}
%</test1>
%    \end{macrocode}
%
% \subsection{Macro tests}
%
% \subsubsection{Preamble}
%
%    \begin{macrocode}
%<*test2>
\NeedsTeXFormat{LaTeX2e}
\nofiles
\documentclass{article}
\makeatletter
%<*noetex>
\let\SavedNumexpr\numexpr
\let\SavedIfcsname\ifcsname
\let\SavedCurrentgrouplevel\currentgrouplevel
\def\ETeXDisable{%
  \let\ifcsname\@undefined
  \let\numexpr\@undefined
  \let\currentgrouplevel\@undefined
}
\ETeXDisable
%</noetex>
\makeatletter
\chardef\BitSet@TestMode=1 %
\makeatother
\usepackage{bitset}[2011/01/30]
%<*noetex>
\def\ETeXEnable{%
  \let\numexpr\SavedNumexpr
  \let\ifcsname\SavedIfcsname
  \let\currentgrouplevel\SavedCurrentgrouplevel
}
\ETeXEnable
%</noetex>
\usepackage{qstest}
\IncludeTests{*}
\LogTests{log}{*}{*}
\makeatletter
%    \end{macrocode}
%
% \subsubsection{Time}
%
%    \begin{macrocode}
\begingroup\expandafter\expandafter\expandafter\endgroup
\expandafter\ifx\csname pdfresettimer\endcsname\relax
\else
  \newcount\SummaryTime
  \newcount\TestTime
  \SummaryTime=\z@
  \newcommand*{\PrintTime}[2]{%
    \typeout{%
      [Time #1: \strip@pt\dimexpr\number#2sp\relax\space s]%
    }%
  }%
  \newcommand*{\StartTime}[1]{%
    \renewcommand*{\TimeDescription}{#1}%
    \pdfresettimer
  }%
  \newcommand*{\TimeDescription}{}%
  \newcommand*{\StopTime}{%
    \TestTime=\pdfelapsedtime
    \global\advance\SummaryTime\TestTime
    \PrintTime\TimeDescription\TestTime
  }%
  \let\saved@qstest\qstest
  \let\saved@endqstest\endqstest
  \def\qstest#1#2{%
    \saved@qstest{#1}{#2}%
    \StartTime{#1}%
  }%
  \def\endqstest{%
    \StopTime
    \saved@endqstest
  }%
  \AtEndDocument{%
    \PrintTime{summary}\SummaryTime
  }%
\fi
%    \end{macrocode}
%
% \subsubsection{Detection of unwanted space}
%
%    \begin{macrocode}
\let\orig@qstest\qstest
\let\orig@endqstest\endqstest
\def\qstest#1#2{%
  \orig@qstest{#1}{#2}%
  \setbox0\hbox\bgroup\begingroup\ignorespaces
}
\def\endqstest{%
  \endgroup\egroup
  \Expect*{\the\wd0}{0.0pt}%
  \orig@endqstest
}
%    \end{macrocode}
%
% \subsubsection{Test macros}
%
%    \begin{macrocode}
\newcounter{Test}

\def\TestError#1#2{%
  \begingroup
    \setcounter{Test}{0}%
    \sbox0{%
      \def\@PackageError##1##2##3{%
        \stepcounter{Test}%
        \begingroup
          \let\MessageBreak\relax
%<*noetex>
          \ETeXEnable
%</noetex>
          \Expect{##1}{bitset}%
          \Expect*{##2}*{#1}%
        \endgroup
      }%
%<*noetex>
      \ETeXDisable
%</noetex>
      #2%
    }%
    \Expect*{\theTest}{1}%
    \Expect*{\the\wd0}{0.0pt}%
  \endgroup
}

\def\TestErrorNegativeIndex#1#2{%
  \TestError{Invalid negative index (#1)}{#2}%
}

\def\TestGetterUndefined#1{%
  \CheckUndef{dummy}%
  \expandafter\expandafter\expandafter\Expect
  \expandafter\expandafter\expandafter{#1{dummy}}{0}%
}

\def\ExpectBitSet#1#2{%
  \expandafter\expandafter\expandafter\Expect
  \expandafter\expandafter\expandafter
  {\csname BS@#1\endcsname}*{#2}%
}
\def\Check#1#2{%
  \ExpectBitSet{#1}{#2}%
}
\def\CheckUndef#1{%
  \begingroup
    \Expect*{%
      \expandafter
      \ifx\csname BS@#1\endcsname\relax true\else false\fi
    }{true}%
  \endgroup
}
\def\RevCheck#1#2{%
  \ExpectBitSet{#1}{\Reverse#2!!}%
}
\def\Set#1#2{%
  \expandafter\def\csname BS@#1\endcsname{#2}%
}
\def\RevSet#1#2{%
  \expandafter\edef\csname BS@#1\endcsname{%
    \Reverse#2!!%
  }%
}
\def\Reverse#1#2!#3!{%
  \ifx\\#2\\%
    #1#3%
    \expandafter\@gobble
  \else
    \expandafter\@firstofone
  \fi
  {\Reverse#2!#1#3!}%
}
%    \end{macrocode}
%
% \subsubsection{Test sets}
%
%    \begin{macrocode}
\begin{qstest}{Let}{Let}
  \CheckUndef{abc}%
  \CheckUndef{xyz}%
  \bitsetLet{xyz}{abc}%
  \CheckUndef{abc}%
  \Check{xyz}{0}%
  \Set{abc}{1}%
  \Check{abc}{1}%
  \Check{xyz}{0}%
  \bitsetLet{xyz}{abc}%
  \Check{abc}{1}%
  \Check{xyz}{1}%
  \Set{xyz}{11}%
  \Check{abc}{1}%
  \Check{xyz}{11}%
\end{qstest}

\begin{qstest}{Reset}{Reset}
  \bitsetReset{xyz}%
  \Check{xyz}{0}%
  \bitsetReset{abc}%
  \Check{abc}{0}%
  \Set{abc}{10101}%
  \bitsetReset{abc}%
  \Check{abc}{0}%
\end{qstest}

\begin{qstest}{Get/Query}{Get/Query}
  \expandafter\expandafter\expandafter\Expect
  \expandafter\expandafter\expandafter{%
    \bitsetGet{dummy}{0}%
  }{0}%
  \begingroup
    \expandafter\def\csname BitSetError:NegativeIndex\endcsname{}%
    \Set{abc}{1}%
    \Expect*{\bitsetQuery{abc}{-1}{true}{false}}{false}%
  \endgroup
  \def\Test#1#2#3{%
    \Set{abc}{#1}%
    \expandafter\expandafter\expandafter\Expect
    \expandafter\expandafter\expandafter{\bitsetGet{abc}{#2}}{#3}%
    \Expect*{\bitsetQuery{abc}{#2}{true}{false}}%
           *{\ifcase#3 false\or true\else error\fi}%
  }%
  \Test{1}{100}{0}%
  \Test{0}{0}{0}%
  \Test{1}{0}{1}%
  \Test{11}{1}{1}%
  \Test{111}{1}{1}%
  \Test{101}{1}{0}%
  \Test{101}{2}{1}%
  \Test{10100110011}{10}{1}%
\end{qstest}

\begin{qstest}{Size}{Size}
  \TestGetterUndefined\bitsetSize
  \def\Test#1#2{%
    \Set{abc}{#1}%
    \expandafter\expandafter\expandafter\Expect
    \expandafter\expandafter\expandafter{\bitsetSize{abc}}{#2}%
  }%
  \Test{0}{0}%
  \Test{1}{1}%
  \Test{00}{0}%
  \Test{0000000}{0}%
  \Test{10}{1}%
  \Test{01}{2}%
  \Test{11}{2}%
  \Test{010}{2}%
  \Test{011}{3}%
  \Test{100110011}{9}%
  \Test{0000011111000001111100000}{20}%
  \Test{000000000000000000000000011111111111111111111}{45}%
\end{qstest}

\begin{qstest}{Cardinality}{Cardinality}
  \TestGetterUndefined\bitsetCardinality
  \def\Test#1#2{%
    \Set{abc}{#1}%
    \expandafter\expandafter\expandafter\Expect
    \expandafter\expandafter\expandafter{%
      \bitsetCardinality{abc}%
    }{#2}%
  }%
  \Test{0}{0}%
  \Test{1}{1}%
  \Test{00}{0}%
  \Test{0000000}{0}%
  \Test{10}{1}%
  \Test{01}{1}%
  \Test{11}{2}%
  \Test{010}{1}%
  \Test{011}{2}%
  \Test{100110011}{5}%
  \Test{0000011111000001111100000}{10}%
  \Test{000000000000000000000000011111111111111111111}{20}%
\end{qstest}

\begin{qstest}{NextClearBit/NextSetBit}{NextClearBit/NextSetBit}
  \def\Test#1#2{%
    \expandafter\expandafter\expandafter\Expect
    \expandafter\expandafter\expandafter{%
      \TestOp{abc}{#1}%
    }{#2}%
  }%
  \def\Clear{\let\TestOp\bitsetNextClearBit}%
  \def\Set{\let\TestOp\bitsetNextSetBit}%
  \begingroup
    \catcode`\:=11 %
    \bitsetSetBin{abc}{1}%
    \Clear
    \Test{-1}{1\BitSetError:NegativeIndex}%
    \Set
    \Test{-1}{0\BitSetError:NegativeIndex}%
  \endgroup
  \let\BS@abc\@undefined
  \Clear
  \Test{0}{0}%
  \Test{1}{1}%
  \Test{2}{2}%
  \Test{100}{100}%
  \Set
  \Test{0}{-1}%
  \Test{1}{-1}%
  \Test{100}{-1}%
  \bitsetReset{abc}%
  \Clear
  \Test{0}{0}%
  \Test{1}{1}%
  \Test{2}{2}%
  \Test{100}{100}%
  \Set
  \Test{0}{-1}%
  \Test{1}{-1}%
  \Test{100}{-1}%
  \bitsetSetBin{abc}{1}%
  \Clear
  \Test{0}{1}%
  \Test{1}{1}%
  \Test{2}{2}%
  \Test{100}{100}%
  \Set
  \Test{0}{0}%
  \Test{1}{-1}%
  \Test{100}{-1}%
  \bitsetSetBin{abc}{111000111000111000111}%
  \Clear
  \Test{0}{3}%
  \Test{1}{3}%
  \Test{2}{3}%
  \Test{3}{3}%
  \Test{4}{4}%
  \Test{5}{5}%
  \Test{6}{9}%
  \Test{7}{9}%
  \Test{8}{9}%
  \Test{9}{9}%
  \Test{10}{10}%
  \Test{11}{11}%
  \Test{12}{15}%
  \Test{13}{15}%
  \Test{14}{15}%
  \Test{15}{15}%
  \Test{16}{16}%
  \Test{17}{17}%
  \Test{18}{21}%
  \Test{19}{21}%
  \Test{20}{21}%
  \Test{21}{21}%
  \Test{22}{22}%
  \Test{100}{100}%
  \Set
  \Test{0}{0}%
  \Test{1}{1}%
  \Test{2}{2}%
  \Test{3}{6}%
  \Test{4}{6}%
  \Test{5}{6}%
  \Test{6}{6}%
  \Test{7}{7}%
  \Test{8}{8}%
  \Test{9}{12}%
  \Test{10}{12}%
  \Test{11}{12}%
  \Test{12}{12}%
  \Test{13}{13}%
  \Test{14}{14}%
  \Test{15}{18}%
  \Test{16}{18}%
  \Test{17}{18}%
  \Test{18}{18}%
  \Test{19}{19}%
  \Test{20}{20}%
  \Test{21}{-1}%
  \Test{22}{-1}%
  \Test{100}{-1}%
  \bitsetSetBin{abc}{1111111}%
  \Clear
  \Test{6}{7}%
  \Test{7}{7}%
  \Test{8}{8}%
  \Test{100}{100}%
  \Set
  \Test{6}{6}%
  \Test{7}{-1}%
  \Test{8}{-1}%
  \Test{100}{-1}%
  \bitsetSetBin{abc}{11111111}%
  \Clear
  \Test{7}{8}%
  \Test{8}{8}%
  \Test{9}{9}%
  \Test{100}{100}%
  \Set
  \Test{7}{7}%
  \Test{8}{-1}%
  \Test{9}{-1}%
  \Test{100}{-1}%
  \bitsetSetBin{abc}{111111111}%
  \Clear
  \Test{8}{9}%
  \Test{9}{9}%
  \Test{10}{10}%
  \Test{100}{100}%
  \Set
  \Test{8}{8}%
  \Test{9}{-1}%
  \Test{10}{-1}%
  \Test{100}{-1}%
  \bitsetSetBin{abc}{1111111111}%
  \Clear
  \Test{9}{10}%
  \Test{10}{10}%
  \Test{11}{11}%
  \Test{100}{100}%
  \Set
  \Test{9}{9}%
  \Test{10}{-1}%
  \Test{11}{-1}%
  \Test{100}{-1}%
\end{qstest}

\begin{qstest}{GetSetBitList}{GetSetBitList}
  \let\BS@abc\@undefined
  \expandafter\expandafter\expandafter\Expect
  \expandafter\expandafter\expandafter{%
    \bitsetGetSetBitList{abc}%
  }{}%
  \def\Test#1#2{%
    \bitsetSetBin{abc}{#1}%
    \expandafter\expandafter\expandafter\Expect
    \expandafter\expandafter\expandafter{%
      \bitsetGetSetBitList{abc}%
    }{#2}%
  }%
  \Test{0}{}%
  \Test{1}{0}%
  \Test{10}{1}%
  \Test{11}{0,1}%
  \Test{10110100}{2,4,5,7}%
  \Test{101101001010011}{0,1,4,6,9,11,12,14}%
\end{qstest}

\begin{qstest}{GetDec}{GetDec}
  \TestGetterUndefined\bitsetGetDec
  \def\Test#1#2{%
    \RevSet{abc}{#1}%
%<*noetex>
    \begingroup\expandafter\expandafter\expandafter\endgroup
%</noetex>
    \expandafter\expandafter\expandafter\Expect
    \expandafter\expandafter\expandafter{%
      \bitsetGetDec{abc}%
    }{#2}%
  }%
  \Test{0}{0}%
  \Test{1}{1}%
  \Test{10}{2}%
  \Test{11}{3}%
  \Test{100}{4}%
  \Test{101}{5}%
  \Test{110}{6}%
  \Test{111}{7}%
  \Test{1000}{8}%
  \Test{000111}{7}%
  \Test{1111111111111111%
        111111111111111}{2147483647}%
  \Test{0001111111111111111%
        111111111111111}{2147483647}%
  \Test{1000000000000000%
        0000000000000000}{2147483648}%
  \Test{1000000000000000%
        00000000000000000}{4294967296}%
  \Test{0001000000000000000%
        00000000000000000}{4294967296}%
  \Test{1100000000000000%
        00000000000000011}{6442450947}%
\end{qstest}

\begin{qstest}{Clear}{Clear}
  \def\Test#1#2#3{%
    \RevSet{abc}{#1}%
    \bitsetClear{abc}{#2}%
    \Expect*{\BS@abc}*{\Reverse#3!!}%
  }%
  \bitsetClear{abc}{2}%
  \RevCheck{abc}{0}%
  \TestErrorNegativeIndex{-1}{\bitsetClear{abc}{-1}}%
  \RevCheck{abc}{0}%
  \Test{0}{0}{0}%
  \Test{1}{0}{0}%
  \Test{111}{1}{101}%
  \Test{111}{30}{111}%
  \Test{0000111}{5}{0000111}% 111 would also be ok
  \Test{10000111}{5}{10000111}%
  \Test{1001001}{3}{1000001}%
\end{qstest}

\begin{qstest}{Set}{Set}
  \def\Test#1#2#3{%
    \RevSet{abc}{#1}%
    \bitsetSet{abc}{#2}%
    \Expect*{\BS@abc}*{\Reverse#3!!}%
  }%
  \bitsetSet{abc}{2}%
  \RevCheck{abc}{100}%
  \TestErrorNegativeIndex{-1}{\bitsetSet{abc}{-1}}%
  \RevCheck{abc}{100}%
  \Test{0}{0}{1}%
  \Test{1}{0}{1}%
  \Test{100}{1}{110}%
  \Test{111}{1}{111}%
  \Test{11}{1}{11}%
  \Test{11}{2}{111}%
  \Test{11}{3}{1011}%
  \Test{111}{10}{10000000111}%
  \Test{0000111}{5}{0100111}% 100111 would also be ok
  \Test{10000111}{5}{10100111}%
  \Test{1000001}{3}{1001001}%
  \Test{1001001}{3}{1001001}%
\end{qstest}

\begin{qstest}{Flip}{Flip}
  \def\Test#1#2#3{%
    \RevSet{abc}{#1}%
    \bitsetFlip{abc}{#2}%
    \Expect*{\BS@abc}*{\Reverse#3!!}%
  }%
  \bitsetFlip{abc}{2}%
  \RevCheck{abc}{100}%
  \TestErrorNegativeIndex{-1}{\bitsetFlip{abc}{-1}}%
  \RevCheck{abc}{100}%
  \Test{0}{0}{1}%
  \Test{1}{0}{0}%
  \Test{0}{2}{100}%
  \Test{100}{1}{110}%
  \Test{111}{1}{101}%
  \Test{11}{1}{1}%
  \Test{11}{2}{111}%
  \Test{11}{3}{1011}%
  \Test{111}{10}{10000000111}%
  \Test{0000111}{5}{0100111}% 100111 would also be ok
  \Test{10000111}{5}{10100111}%
  \Test{1000001}{3}{1001001}%
  \Test{1001001}{3}{1000001}%
  \Test{11111}{2}{11011}%
\end{qstest}

\begin{qstest}{SetValue}{SetValue}
  \def\Test#1#2{%
    \TestError{Invalid bit value (#2) not in range 0..1}{%
      \bitsetSetValue{abc}{#1}{#2}%
    }%
  }%
  \Test{0}{-1}%
  \Test{0}{2}%
  \Test{0}{10}%
  \def\Test#1#2#3{%
    \let\BS@abc\@undefined
    \bitsetSetValue{abc}{#1}{#2}%
    \bitsetSetBin{result}{#3}%
    \Expect*{\BS@abc}*{\BS@result}%
  }%
  \Test{0}{0}{0}%
  \Test{0}{1}{1}%
  \Test{1}{0}{0}%
  \Test{1}{1}{10}%
  \def\Test#1#2#3#4{%
    \bitsetSetBin{abc}{#1}%
    \bitsetSetBin{result}{#4}%
    \bitsetSetValue{abc}{#2}{#3}%
    \Expect*{\BS@abc}*{\BS@result}%
  }%
  \Test{0}{0}{0}{0}%
  \Test{0}{0}{0}{0}%
  \Test{0}{0}{1}{1}%
  \Test{0}{1}{0}{0}%
  \Test{0}{1}{1}{10}%
  \Test{1010}{2}{1}{1110}%
  \Test{1010}{4}{1}{11010}%
  \Test{1010}{6}{1}{1001010}%
  \Test{1010}{1}{0}{1000}%
  \Test{1010}{2}{0}{1010}%
  \Test{1010}{3}{0}{10}%
  \Test{1010}{4}{0}{1010}%
  \Test{1010}{6}{0}{1010}%
  \Test{1010}{2}{\csname iffalse\endcsname 0\else 1\fi}{1110}%
  \Test{1010}{1}{\csname iffalse\endcsname 1\else 0\fi}{1000}%
\end{qstest}

\begin{qstest}{IsDefined}{IsDefined}
  \let\BS@abc\@undefined
  \Expect*{\bitsetIsDefined{abc}{true}{false}}{false}%
  \bitsetReset{abc}%
  \Expect*{\bitsetIsDefined{abc}{true}{false}}{true}%
\end{qstest}

\begin{qstest}{IsEmpty}{IsEmpty}
  \let\BS@abc\@undefined
  \Expect*{\bitsetIsEmpty{abc}{true}{false}}{true}%
  \bitsetReset{abc}%
  \Expect*{\bitsetIsEmpty{abc}{true}{false}}{true}%
  \bitsetSet{abc}{1}%
  \Expect*{\bitsetIsEmpty{abc}{true}{false}}{false}%
\end{qstest}

\begin{qstest}{Equals}{Equals}
  \def\Test#1#2#3{%
    \Expect*{\bitsetEquals{#1}{#2}{true}{false}}{#3}%
  }%
  \let\BS@abc\@undefined
  \Test{abc}{abc}{true}%
  \Test{abc}{foo}{true}%
  \Test{foo}{abc}{true}%
  \bitsetReset{abc}%
  \Test{abc}{abc}{true}%
  \Test{abc}{foo}{false}%
  \Test{foo}{abc}{false}%
  \bitsetReset{foo}%
  \Test{abc}{foo}{true}%
  \Test{foo}{abc}{true}%
  \bitsetSet{abc}{4}%
  \Test{abc}{foo}{false}%
  \Test{foo}{abc}{false}%
  \bitsetFlip{foo}{4}%
  \Test{abc}{foo}{true}%
  \Test{foo}{abc}{true}%
\end{qstest}

\begin{qstest}{Intersects}{Intersects}
  \def\Test#1{%
    \Expect*{\bitsetIntersects{abc}{foo}{true}{false}}{#1}%
  }%
  \let\BS@abc\@undefined
  \let\BS@foo\@undefined
  \Test{false}%
  \Set{abc}{0}%
  \Test{false}%
  \Set{foo}{0}%
  \Test{false}%
  \let\BS@abc\@undefined
  \Test{false}%
  \Set{foo}{1}%
  \Test{false}%
  \Set{abc}{0}%
  \Test{false}%
  \Set{abc}{1}%
  \Test{true}%
  \let\BS@foo\@undefined
  \Test{false}%
  \Set{foo}{0}%
  \Test{false}%
  \def\Test#1#2#3{%
    \bitsetSetBin{abc}{#1}%
    \bitsetSetBin{foo}{#2}%
    \Expect*{\bitsetIntersects{abc}{foo}{true}{false}}{#3}%
  }%
  \Test{1010}{0101}{false}%
  \Test{0}{10}{false}%
  \Test{1}{11}{true}%
  \Test{11}{1}{true}%
  \Test{10}{1}{false}%
\end{qstest}

\begin{qstest}{And/AndNot/Or/Xor}{And/AndNot/Or/Xor}
  \def\@Test#1#2#3#4#5{%
    \begingroup
      #5%
      \begingroup
        \let\BS@foo\@undefined
        \csname bitset#1\endcsname{abc}{foo}%
        \CheckUndef{foo}%
        \Check{abc}{#2}%
      \endgroup
      \begingroup
        \bitsetReset{foo}%
        \csname bitset#1\endcsname{abc}{foo}%
        \Check{foo}{0}%
        \Check{abc}{#3}%
      \endgroup
      \begingroup
        \def\BS@foo{0101}%
        \csname bitset#1\endcsname{abc}{foo}%
        \Check{foo}{0101}%
        \Check{abc}{#4}%
      \endgroup
    \endgroup
  }%
  \def\Test#1{%
    \def\Op{#1}%
    \Test@
  }%
  \def\Test@#1#2#3#4#5#6#7#8#9{%
    \@Test\Op{#1}{#2}{#3}{%
      \let\BS@abc\@undefined
    }%
    \@Test\Op{#4}{#5}{#6}{%
      \bitsetReset{abc}%
    }%
    \@Test\Op{#7}{#8}{#9}{%
      \def\BS@abc{1001}%
    }%
  }%
  \Test{And}%
       {0}{0}{0}%
       {0}{0}{0}%
       {0}{0}{0001}%
  \Test{AndNot}%
       {0}{0}{0}%
       {0}{0}{0}%
       {1001}{1001}{1}%
  \Test{Or}%
       {0}{0}{0101}%
       {0}{0}{0101}%
       {1001}{1001}{1101}%
  \Test{Xor}%
       {0}{0}{0101}%
       {0}{0}{0101}%
       {1001}{1001}{11}%
  \def\Test#1#2#3{%
    \bitsetSetBin{abc}{#1}%
    \bitsetSetBin{foo}{#2}%
    \csname bitset\Op\endcsname{abc}{foo}%
    \RevCheck{foo}{#2}%
    \RevCheck{abc}{#3}%
  }%
  \def\Op{And}%
  \Test{1}{111}{1}%
  \Test{111}{1}{1}%
  \Test{10}{111}{10}%
  \Test{111}{10}{10}%
  \Test{111}{1000}{0}%
  \Test{1000}{111}{0}%
  \def\Op{AndNot}%
  \Test{1010}{11}{1000}%
  \Test{100}{100}{0}%
  \Test{111}{1111}{0}%
  \Test{100}{111}{0}%
  \def\Op{Or}%
  \Test{0}{0}{0}%
  \Test{1}{0}{1}%
  \Test{0}{1}{1}%
  \Test{1}{1}{1}%
  \Test{1000}{10}{1010}%
  \Test{10}{1000}{1010}%
  \def\Op{Xor}%
  \Test{0}{0}{0}%
  \Test{1}{0}{1}%
  \Test{0}{1}{1}%
  \Test{1}{1}{0}%
  \Test{1000}{10}{1010}%
  \Test{10}{1000}{1010}%
  \Test   {110011001100}%
       {111000111000111}%
       {111110100001011}%
  \Test{111000111000111}%
          {110011001100}%
       {111110100001011}%
\end{qstest}

\begin{qstest}{GetUndef}{GetUndef, GetBin, GetOct, GetHex}
  \def\TestUndef#1#2{%
    \let\BS@abc\@undefined
    \expandafter\expandafter\expandafter\Expect
    \expandafter\expandafter\expandafter{%
      \x{abc}{#1}%
    }{#2}%
  }%
  \let\x\bitsetGetBin
  \TestUndef{-1}{0}%
  \TestUndef{0}{0}%
  \TestUndef{1}{0}%
  \TestUndef{2}{00}%
  \TestUndef{8}{00000000}%
  \let\x\bitsetGetOct
  \TestUndef{-1}{0}%
  \TestUndef{0}{0}%
  \TestUndef{1}{0}%
  \TestUndef{2}{0}%
  \TestUndef{3}{0}%
  \TestUndef{4}{00}%
  \TestUndef{5}{00}%
  \TestUndef{6}{00}%
  \TestUndef{7}{000}%
  \TestUndef{8}{000}%
  \TestUndef{9}{000}%
  \TestUndef{10}{0000}%
  \let\x\bitsetGetHex
  \TestUndef{-1}{0}%
  \TestUndef{0}{0}%
  \TestUndef{1}{0}%
  \TestUndef{2}{0}%
  \TestUndef{3}{0}%
  \TestUndef{4}{0}%
  \TestUndef{5}{00}%
  \TestUndef{6}{00}%
  \TestUndef{7}{00}%
  \TestUndef{8}{00}%
  \TestUndef{9}{000}%
  \TestUndef{10}{000}%
  \TestUndef{12}{000}%
  \TestUndef{13}{0000}%
  \TestUndef{16}{0000}%
  \TestUndef{17}{00000}%
\end{qstest}

\begin{qstest}{SetBin}{SetBin}
  \def\Test#1#2{%
    \let\BS@abc\@undefined
    \bitsetSetBin{abc}{#1}%
    \expandafter\Expect\expandafter{\BS@abc}{#2}%
  }%
  \Test{}{0}%
  \Test{0}{0}%
  \Test{1}{1}%
  \Test{10}{01}%
  \Test{11}{11}%
  \Test{010}{01}%
  \Test{011}{11}%
  \Test{0010}{01}%
  \Test{1010}{0101}%
\end{qstest}

\begin{qstest}{SetOct}{SetOct}
  \def\Test#1#2{%
    \bitsetSetOct{abc}{#1}%
    \expandafter\Expect\expandafter{\BS@abc}{#2}%
  }%
  \Test{}{0}%
  \Test{0}{0}%
  \Test{000}{0}%
  \Test{1}{1}%
  \Test{001}{1}%
  \Test{010}{0001}%
  \Test{020}{00001}%
  \Test{42}{010001}%
  \Test{377}{11111111}%
  \Test{0377}{11111111}%
  \Test{76543210}{000100010110001101011111}%
  \Test{ 0 7 0 7 1 }{100111000111}%
\end{qstest}

\begin{qstest}{SetHex}{SetHex}
  \def\Test#1#2{%
    \bitsetSetHex{abc}{#1}%
    \expandafter\Expect\expandafter{\BS@abc}{#2}%
  }%
  \Test{}{0}%
  \Test{0}{0}%
  \Test{000}{0}%
  \Test{1}{1}%
  \Test{001}{1}%
  \Test{010}{00001}%
  \Test{020}{000001}%
  \Test{42}{0100001}%
  \Test{3F}{111111}%
  \Test{03F}{111111}%
  \Test{43210}{0000100001001100001}%
  \Test{98765}{10100110111000011001}%
  \Test{FEDCBA}{010111010011101101111111}%
  \Test{ 0 F 0 F 1 }{1000111100001111}%
\end{qstest}

\begin{qstest}{SetDec}{SetDec}
  \def\Test#1#2{%
    \bitsetSetDec{abc}{#1}%
    \expandafter\Expect\expandafter{\BS@abc}{#2}%
  }%
  \Test{}{0}%
  \Test{0}{0}%
  \Test{000}{0}%
  \Test{1}{1}%
  \Test{7}{111}%
  \Test{8}{0001}%
  \Test{001}{1}%
  \Test{010}{0101}%
  \Test{020}{00101}%
  \Test{53}{101011}%
  \Test{255}{11111111}%
  \Test{256}{000000001}%
  \Test{999999999}{111111111001001101011001110111}%
  \Test{1000000000}{000000000101001101011001110111}%
  \Test{4210987654}{01100001010010010111111101011111}%
  \Test{2147483647}{1111111111111111111111111111111}%
  \Test{2147483648}{00000000000000000000000000000001}%
\end{qstest}

\begin{qstest}{GetBin}{GetBin}
  \def\TestUndef#1#2{%
    \let\BS@abc\@undefined
    \expandafter\expandafter\expandafter\Expect
    \expandafter\expandafter\expandafter{%
      \bitsetGetBin{abc}{#1}%
    }{#2}%
  }%
  \TestUndef{-1}{0}%
  \TestUndef{0}{0}%
  \TestUndef{1}{0}%
  \TestUndef{2}{00}%
  \TestUndef{8}{00000000}%
  \def\Test#1#2{%
    \bitsetSetBin{abc}{#2}%
    \expandafter\expandafter\expandafter\Expect
    \expandafter\expandafter\expandafter{%
      \bitsetGetBin{abc}{#1}%
    }{#2}%
  }%
  \Test{-1}{0}%
  \Test{0}{0}%
  \Test{1}{0}%
  \Test{1}{1}%
  \Test{2}{01}%
  \Test{2}{10}%
  \Test{3}{010}%
  \Test{2}{00}%
  \Test{2}{01}%
  \Test{8}{00101100}%
  \Test{2}{10101}%
  \Test{-100}{11011}%
\end{qstest}

\begin{qstest}{GetOct}{GetOct}
  \def\Test#1#2#3{%
    \edef\x{\zap@space#1 \@empty}%
    \edef\x{\noexpand\bitsetSetBin{abc}{\x}}%
    \x
    \expandafter\expandafter\expandafter\Expect
    \expandafter\expandafter\expandafter{%
      \bitsetGetOct{abc}{#2}%
    }{#3}%
  }%
  \Test{111 110 101 100 011 010 001 000}{0}{76543210}%
  \Test{000 111}{0}{7}%
  \Test{101 000}{-1}{50}%
  \Test{111}{-1}{7}%
  \Test{111}{0}{7}%
  \Test{111}{1}{7}%
  \Test{111}{3}{7}%
  \Test{111}{4}{07}%
  \Test{111}{6}{07}%
  \Test{111}{7}{007}%
  \Test{111 010}{6}{72}%
  \Test{111 010}{7}{072}%
  \Test{011 111}{0}{37}%
  \Test{011 111}{6}{37}%
  \Test{011 111}{7}{037}%
  \Test{001 111}{0}{17}%
  \Test{001 111}{6}{17}%
  \Test{001 111}{7}{017}%
\end{qstest}

\begin{qstest}{GetHex}{GetHex}
  \def\Test#1#2#3{%
    \bitsetSetBin{abc}{#1}%
    \expandafter\expandafter\expandafter\Expect
    \expandafter\expandafter\expandafter{%
      \bitsetGetHex{abc}{#2}%
    }{#3}%
  }%
  \Test{1111 1110 1101 1100 1011 1010 1001 1000}{0}{FEDCBA98}%
  \Test{0111 0110 0101 0100 0011 0010 0001 0000}{0}{76543210}%
  \Test{0000 1111}{0}{F}%
  \Test{0101 0000}{-1}{50}%
  \Test{1111}{-1}{F}%
  \Test{1111}{0}{F}%
  \Test{1111}{1}{F}%
  \Test{1111}{4}{F}%
  \Test{1111}{5}{0F}%
  \Test{1111}{8}{0F}%
  \Test{1111}{9}{00F}%
  \Test{1111 0010}{8}{F2}%
  \Test{1111 0010}{9}{0F2}%
  \Test{0111 1111}{0}{7F}%
  \Test{0111 1111}{8}{7F}%
  \Test{0111 1111}{9}{07F}%
  \Test{0011 1111}{0}{3F}%
  \Test{0011 1111}{8}{3F}%
  \Test{0011 1111}{9}{03F}%
  \Test{0001 1111}{0}{1F}%
  \Test{0001 1111}{8}{1F}%
  \Test{0001 1111}{9}{01F}%
\end{qstest}

\begin{qstest}{Range}{Range}
  \TestError{%
    Wrong index numbers in range [9..8]\MessageBreak% hash-ok
    for clear/set/flip on bit set `abc'.\MessageBreak
    The lower index exceeds the upper index.\MessageBreak
    Canceling the operation as error recovery%
  }{%
    \bitsetSetRange{abc}{9}{8}%
  }%
  \def\TestErrorNegInd#1#2#3#4#5#6{%
    \TestError{%
      Negative index in range [#2..#3]\MessageBreak % hash-ok
      for \string\bitset #1Range on bit set `abc'.\MessageBreak
      Using [#4..#5] as error recovery% hash-ok
    }{%
      \csname bitset#1Range\endcsname{abc}{#2}{#3}%
      \global\let\BS@global\BS@abc
    }%
    \Check{global}{#6}%
  }%
  \Set{abc}{111}%
  \TestErrorNegInd{Clear}{-1}{0}{0}{0}{111}%
  \TestErrorNegInd{Clear}{0}{-1}{0}{0}{111}%
  \TestErrorNegInd{Clear}{-2}{2}{0}{2}{001}%
  \bitsetReset{abc}%
  \TestErrorNegInd{Set}{-1}{0}{0}{0}{0}%
  \TestErrorNegInd{Set}{0}{-1}{0}{0}{0}%
  \TestErrorNegInd{Set}{-2}{2}{0}{2}{11}%
  \Set{abc}{101}%
  \TestErrorNegInd{Flip}{-1}{0}{0}{0}{101}%
  \TestErrorNegInd{Flip}{0}{-1}{0}{0}{101}%
  \TestErrorNegInd{Flip}{-2}{2}{0}{2}{011}%
  \def\Test#1#2#3#4{%
    \bitsetSetBin{abc}{#1}%
    \csname bitset\TestOp Range\endcsname{abc}{#2}{#3}%
    \Expect*{\bitsetGetBin{abc}{0}}{#4}%
  }%
  \def\TestOp{Clear}%
  \Test{0}{0}{1}{0}%
  \Test{1111}{1}{2}{1101}%
  \Test{1111}{1}{3}{1001}%
  \Test{1111111100000000}{12}{14}{1100111100000000}%
  \def\TestOp{Set}%
  \Test{0}{0}{1}{1}%
  \Test{1000}{1}{2}{1010}%
  \Test{0}{1}{2}{10}%
  \Test{1}{12}{15}{111000000000001}%
  \Test{1111}{1}{3}{1111}%
  \Test{1000000000000000}{12}{14}{1011000000000000}%
  \def\TestOp{Flip}%
  \Test{0}{0}{1}{1}%
  \Test{1}{0}{1}{0}%
  \Test{10101010}{1}{5}{10110100}%
  \def\Test#1#2#3#4#5{%
    \bitsetSetBin{abc}{#1}%
    \bitsetSetValueRange{abc}{#2}{#3}{#4}%
    \Expect*{\bitsetGetBin{abc}{0}}{#5}%
  }%
  \Test{0}{0}{1}{0}{0}%
  \Test{0}{0}{1}{1}{1}%
  \Test{1010}{1}{3}{0}{1000}%
  \Test{1010}{1}{3}{1}{1110}%
\end{qstest}

\begin{qstest}{ShiftLeft/ShiftRight}{ShiftLeft/ShiftRight}
  \def\@Test#1#2{%
    \let\BS@abc\@undefined
    \csname bitsetShift#1\endcsname{abc}{#2}%
    \Expect*{\BS@abc}{0}%
  }%
  \def\Test#1{%
    \@Test{Left}{#1}%
    \@Test{Right}{#1}%
  }%
  \Test{-16}%
  \Test{-1}%
  \Test{0}%
  \Test{1}%
  \Test{16}%
  \def\Test#1#2#3{%
    \bitsetSetBin{abc}{#1}%
    \bitsetSetBin{result}{#3}%
    \csname bitsetShift\Op\endcsname{abc}{#2}%
    \Expect*{\bitsetGetBin{abc}{0}}*{\bitsetGetBin{result}{0}}%
  }%
  \def\Op{Left}%
  \Test{0}{0}{0}%
  \Test{0}{1}{0}%
  \Test{0}{-1}{0}%
  \Test{1}{0}{1}%
  \Test{1}{1}{10}%
  \Test{1}{-1}{0}%
  \Test{10}{1}{100}%
  \Test{10}{-1}{1}%
  \Test{1}{32}{100000000000000000000000000000000}%
  \Test{1}{-100}{0}%
  \def\Op{Right}%
  \Test{0}{0}{0}%
  \Test{0}{1}{0}%
  \Test{0}{-1}{0}%
  \Test{1}{0}{1}%
  \Test{1}{1}{0}%
  \Test{1}{-1}{10}%
  \Test{10}{1}{1}%
  \Test{10}{-1}{100}%
  \Test{1}{-32}{100000000000000000000000000000000}%
  \Test{1}{100}{0}%
  \Test{110110110110110}{10}{11011}%
  \Test{110110110110110}{100}{0}%
  \Test{1}{100000}{0}%
\end{qstest}

\begin{qstest}{Profile: Set}{Profile: Set}
  \bitsetSet{abc}{4095}%
  \global\let\BS@global\BS@abc
\end{qstest}

\begin{qstest}{Profile: Get}{Profile: Get}
  \edef\x{\bitsetGet{global}{4095}}%
\end{qstest}

\begin{document}
\end{document}
%</test2>
%    \end{macrocode}
%
% \section{Installation}
%
% \subsection{Download}
%
% \paragraph{Package.} This package is available on
% CTAN\footnote{\url{ftp://ftp.ctan.org/tex-archive/}}:
% \begin{description}
% \item[\CTAN{macros/latex/contrib/oberdiek/bitset.dtx}] The source file.
% \item[\CTAN{macros/latex/contrib/oberdiek/bitset.pdf}] Documentation.
% \end{description}
%
%
% \paragraph{Bundle.} All the packages of the bundle `oberdiek'
% are also available in a TDS compliant ZIP archive. There
% the packages are already unpacked and the documentation files
% are generated. The files and directories obey the TDS standard.
% \begin{description}
% \item[\CTAN{install/macros/latex/contrib/oberdiek.tds.zip}]
% \end{description}
% \emph{TDS} refers to the standard ``A Directory Structure
% for \TeX\ Files'' (\CTAN{tds/tds.pdf}). Directories
% with \xfile{texmf} in their name are usually organized this way.
%
% \subsection{Bundle installation}
%
% \paragraph{Unpacking.} Unpack the \xfile{oberdiek.tds.zip} in the
% TDS tree (also known as \xfile{texmf} tree) of your choice.
% Example (linux):
% \begin{quote}
%   |unzip oberdiek.tds.zip -d ~/texmf|
% \end{quote}
%
% \paragraph{Script installation.}
% Check the directory \xfile{TDS:scripts/oberdiek/} for
% scripts that need further installation steps.
% Package \xpackage{attachfile2} comes with the Perl script
% \xfile{pdfatfi.pl} that should be installed in such a way
% that it can be called as \texttt{pdfatfi}.
% Example (linux):
% \begin{quote}
%   |chmod +x scripts/oberdiek/pdfatfi.pl|\\
%   |cp scripts/oberdiek/pdfatfi.pl /usr/local/bin/|
% \end{quote}
%
% \subsection{Package installation}
%
% \paragraph{Unpacking.} The \xfile{.dtx} file is a self-extracting
% \docstrip\ archive. The files are extracted by running the
% \xfile{.dtx} through \plainTeX:
% \begin{quote}
%   \verb|tex bitset.dtx|
% \end{quote}
%
% \paragraph{TDS.} Now the different files must be moved into
% the different directories in your installation TDS tree
% (also known as \xfile{texmf} tree):
% \begin{quote}
% \def\t{^^A
% \begin{tabular}{@{}>{\ttfamily}l@{ $\rightarrow$ }>{\ttfamily}l@{}}
%   bitset.sty & tex/generic/oberdiek/bitset.sty\\
%   bitset.pdf & doc/latex/oberdiek/bitset.pdf\\
%   test/bitset-test1.tex & doc/latex/oberdiek/test/bitset-test1.tex\\
%   test/bitset-test2.tex & doc/latex/oberdiek/test/bitset-test2.tex\\
%   test/bitset-test3.tex & doc/latex/oberdiek/test/bitset-test3.tex\\
%   bitset.dtx & source/latex/oberdiek/bitset.dtx\\
% \end{tabular}^^A
% }^^A
% \sbox0{\t}^^A
% \ifdim\wd0>\linewidth
%   \begingroup
%     \advance\linewidth by\leftmargin
%     \advance\linewidth by\rightmargin
%   \edef\x{\endgroup
%     \def\noexpand\lw{\the\linewidth}^^A
%   }\x
%   \def\lwbox{^^A
%     \leavevmode
%     \hbox to \linewidth{^^A
%       \kern-\leftmargin\relax
%       \hss
%       \usebox0
%       \hss
%       \kern-\rightmargin\relax
%     }^^A
%   }^^A
%   \ifdim\wd0>\lw
%     \sbox0{\small\t}^^A
%     \ifdim\wd0>\linewidth
%       \ifdim\wd0>\lw
%         \sbox0{\footnotesize\t}^^A
%         \ifdim\wd0>\linewidth
%           \ifdim\wd0>\lw
%             \sbox0{\scriptsize\t}^^A
%             \ifdim\wd0>\linewidth
%               \ifdim\wd0>\lw
%                 \sbox0{\tiny\t}^^A
%                 \ifdim\wd0>\linewidth
%                   \lwbox
%                 \else
%                   \usebox0
%                 \fi
%               \else
%                 \lwbox
%               \fi
%             \else
%               \usebox0
%             \fi
%           \else
%             \lwbox
%           \fi
%         \else
%           \usebox0
%         \fi
%       \else
%         \lwbox
%       \fi
%     \else
%       \usebox0
%     \fi
%   \else
%     \lwbox
%   \fi
% \else
%   \usebox0
% \fi
% \end{quote}
% If you have a \xfile{docstrip.cfg} that configures and enables \docstrip's
% TDS installing feature, then some files can already be in the right
% place, see the documentation of \docstrip.
%
% \subsection{Refresh file name databases}
%
% If your \TeX~distribution
% (\teTeX, \mikTeX, \dots) relies on file name databases, you must refresh
% these. For example, \teTeX\ users run \verb|texhash| or
% \verb|mktexlsr|.
%
% \subsection{Some details for the interested}
%
% \paragraph{Attached source.}
%
% The PDF documentation on CTAN also includes the
% \xfile{.dtx} source file. It can be extracted by
% AcrobatReader 6 or higher. Another option is \textsf{pdftk},
% e.g. unpack the file into the current directory:
% \begin{quote}
%   \verb|pdftk bitset.pdf unpack_files output .|
% \end{quote}
%
% \paragraph{Unpacking with \LaTeX.}
% The \xfile{.dtx} chooses its action depending on the format:
% \begin{description}
% \item[\plainTeX:] Run \docstrip\ and extract the files.
% \item[\LaTeX:] Generate the documentation.
% \end{description}
% If you insist on using \LaTeX\ for \docstrip\ (really,
% \docstrip\ does not need \LaTeX), then inform the autodetect routine
% about your intention:
% \begin{quote}
%   \verb|latex \let\install=y\input{bitset.dtx}|
% \end{quote}
% Do not forget to quote the argument according to the demands
% of your shell.
%
% \paragraph{Generating the documentation.}
% You can use both the \xfile{.dtx} or the \xfile{.drv} to generate
% the documentation. The process can be configured by the
% configuration file \xfile{ltxdoc.cfg}. For instance, put this
% line into this file, if you want to have A4 as paper format:
% \begin{quote}
%   \verb|\PassOptionsToClass{a4paper}{article}|
% \end{quote}
% An example follows how to generate the
% documentation with pdf\LaTeX:
% \begin{quote}
%\begin{verbatim}
%pdflatex bitset.dtx
%makeindex -s gind.ist bitset.idx
%pdflatex bitset.dtx
%makeindex -s gind.ist bitset.idx
%pdflatex bitset.dtx
%\end{verbatim}
% \end{quote}
%
% \section{Catalogue}
%
% The following XML file can be used as source for the
% \href{http://mirror.ctan.org/help/Catalogue/catalogue.html}{\TeX\ Catalogue}.
% The elements \texttt{caption} and \texttt{description} are imported
% from the original XML file from the Catalogue.
% The name of the XML file in the Catalogue is \xfile{bitset.xml}.
%    \begin{macrocode}
%<*catalogue>
<?xml version='1.0' encoding='us-ascii'?>
<!DOCTYPE entry SYSTEM 'catalogue.dtd'>
<entry datestamp='$Date$' modifier='$Author$' id='bitset'>
  <name>bitset</name>
  <caption>Handle bit-vector datatype.</caption>
  <authorref id='auth:oberdiek'/>
  <copyright owner='Heiko Oberdiek' year='2007,2011'/>
  <license type='lppl1.3'/>
  <version number='1.1'/>
  <description>
    This package defines and implements the data type bit set,
    a vector of bits. The size of the vector may grow dynamically.
    Individual bits can be manipulated.
    <p/>
    The package is part of the <xref refid='oberdiek'>oberdiek</xref> bundle.
  </description>
  <documentation details='Package documentation'
      href='ctan:/macros/latex/contrib/oberdiek/bitset.pdf'/>
  <ctan file='true' path='/macros/latex/contrib/oberdiek/bitset.dtx'/>
  <miktex location='oberdiek'/>
  <texlive location='oberdiek'/>
  <install path='/macros/latex/contrib/oberdiek/oberdiek.tds.zip'/>
</entry>
%</catalogue>
%    \end{macrocode}
%
% \begin{History}
%   \begin{Version}{2007/09/28 v1.0}
%   \item
%     First version.
%   \end{Version}
%   \begin{Version}{2011/01/30 v1.1}
%   \item
%     Already loaded package files are not input in \hologo{plainTeX}.
%   \end{Version}
% \end{History}
%
% \PrintIndex
%
% \Finale
\endinput
|
% \end{quote}
% Do not forget to quote the argument according to the demands
% of your shell.
%
% \paragraph{Generating the documentation.}
% You can use both the \xfile{.dtx} or the \xfile{.drv} to generate
% the documentation. The process can be configured by the
% configuration file \xfile{ltxdoc.cfg}. For instance, put this
% line into this file, if you want to have A4 as paper format:
% \begin{quote}
%   \verb|\PassOptionsToClass{a4paper}{article}|
% \end{quote}
% An example follows how to generate the
% documentation with pdf\LaTeX:
% \begin{quote}
%\begin{verbatim}
%pdflatex bitset.dtx
%makeindex -s gind.ist bitset.idx
%pdflatex bitset.dtx
%makeindex -s gind.ist bitset.idx
%pdflatex bitset.dtx
%\end{verbatim}
% \end{quote}
%
% \section{Catalogue}
%
% The following XML file can be used as source for the
% \href{http://mirror.ctan.org/help/Catalogue/catalogue.html}{\TeX\ Catalogue}.
% The elements \texttt{caption} and \texttt{description} are imported
% from the original XML file from the Catalogue.
% The name of the XML file in the Catalogue is \xfile{bitset.xml}.
%    \begin{macrocode}
%<*catalogue>
<?xml version='1.0' encoding='us-ascii'?>
<!DOCTYPE entry SYSTEM 'catalogue.dtd'>
<entry datestamp='$Date$' modifier='$Author$' id='bitset'>
  <name>bitset</name>
  <caption>Handle bit-vector datatype.</caption>
  <authorref id='auth:oberdiek'/>
  <copyright owner='Heiko Oberdiek' year='2007,2011'/>
  <license type='lppl1.3'/>
  <version number='1.1'/>
  <description>
    This package defines and implements the data type bit set,
    a vector of bits. The size of the vector may grow dynamically.
    Individual bits can be manipulated.
    <p/>
    The package is part of the <xref refid='oberdiek'>oberdiek</xref> bundle.
  </description>
  <documentation details='Package documentation'
      href='ctan:/macros/latex/contrib/oberdiek/bitset.pdf'/>
  <ctan file='true' path='/macros/latex/contrib/oberdiek/bitset.dtx'/>
  <miktex location='oberdiek'/>
  <texlive location='oberdiek'/>
  <install path='/macros/latex/contrib/oberdiek/oberdiek.tds.zip'/>
</entry>
%</catalogue>
%    \end{macrocode}
%
% \begin{History}
%   \begin{Version}{2007/09/28 v1.0}
%   \item
%     First version.
%   \end{Version}
%   \begin{Version}{2011/01/30 v1.1}
%   \item
%     Already loaded package files are not input in \hologo{plainTeX}.
%   \end{Version}
% \end{History}
%
% \PrintIndex
%
% \Finale
\endinput
|
% \end{quote}
% Do not forget to quote the argument according to the demands
% of your shell.
%
% \paragraph{Generating the documentation.}
% You can use both the \xfile{.dtx} or the \xfile{.drv} to generate
% the documentation. The process can be configured by the
% configuration file \xfile{ltxdoc.cfg}. For instance, put this
% line into this file, if you want to have A4 as paper format:
% \begin{quote}
%   \verb|\PassOptionsToClass{a4paper}{article}|
% \end{quote}
% An example follows how to generate the
% documentation with pdf\LaTeX:
% \begin{quote}
%\begin{verbatim}
%pdflatex bitset.dtx
%makeindex -s gind.ist bitset.idx
%pdflatex bitset.dtx
%makeindex -s gind.ist bitset.idx
%pdflatex bitset.dtx
%\end{verbatim}
% \end{quote}
%
% \section{Catalogue}
%
% The following XML file can be used as source for the
% \href{http://mirror.ctan.org/help/Catalogue/catalogue.html}{\TeX\ Catalogue}.
% The elements \texttt{caption} and \texttt{description} are imported
% from the original XML file from the Catalogue.
% The name of the XML file in the Catalogue is \xfile{bitset.xml}.
%    \begin{macrocode}
%<*catalogue>
<?xml version='1.0' encoding='us-ascii'?>
<!DOCTYPE entry SYSTEM 'catalogue.dtd'>
<entry datestamp='$Date$' modifier='$Author$' id='bitset'>
  <name>bitset</name>
  <caption>Handle bit-vector datatype.</caption>
  <authorref id='auth:oberdiek'/>
  <copyright owner='Heiko Oberdiek' year='2007,2011'/>
  <license type='lppl1.3'/>
  <version number='1.1'/>
  <description>
    This package defines and implements the data type bit set,
    a vector of bits. The size of the vector may grow dynamically.
    Individual bits can be manipulated.
    <p/>
    The package is part of the <xref refid='oberdiek'>oberdiek</xref> bundle.
  </description>
  <documentation details='Package documentation'
      href='ctan:/macros/latex/contrib/oberdiek/bitset.pdf'/>
  <ctan file='true' path='/macros/latex/contrib/oberdiek/bitset.dtx'/>
  <miktex location='oberdiek'/>
  <texlive location='oberdiek'/>
  <install path='/macros/latex/contrib/oberdiek/oberdiek.tds.zip'/>
</entry>
%</catalogue>
%    \end{macrocode}
%
% \begin{History}
%   \begin{Version}{2007/09/28 v1.0}
%   \item
%     First version.
%   \end{Version}
%   \begin{Version}{2011/01/30 v1.1}
%   \item
%     Already loaded package files are not input in \hologo{plainTeX}.
%   \end{Version}
% \end{History}
%
% \PrintIndex
%
% \Finale
\endinput

%        (quote the arguments according to the demands of your shell)
%
% Documentation:
%    (a) If bitset.drv is present:
%           latex bitset.drv
%    (b) Without bitset.drv:
%           latex bitset.dtx; ...
%    The class ltxdoc loads the configuration file ltxdoc.cfg
%    if available. Here you can specify further options, e.g.
%    use A4 as paper format:
%       \PassOptionsToClass{a4paper}{article}
%
%    Programm calls to get the documentation (example):
%       pdflatex bitset.dtx
%       makeindex -s gind.ist bitset.idx
%       pdflatex bitset.dtx
%       makeindex -s gind.ist bitset.idx
%       pdflatex bitset.dtx
%
% Installation:
%    TDS:tex/generic/oberdiek/bitset.sty
%    TDS:doc/latex/oberdiek/bitset.pdf
%    TDS:doc/latex/oberdiek/test/bitset-test1.tex
%    TDS:doc/latex/oberdiek/test/bitset-test2.tex
%    TDS:doc/latex/oberdiek/test/bitset-test3.tex
%    TDS:source/latex/oberdiek/bitset.dtx
%
%<*ignore>
\begingroup
  \catcode123=1 %
  \catcode125=2 %
  \def\x{LaTeX2e}%
\expandafter\endgroup
\ifcase 0\ifx\install y1\fi\expandafter
         \ifx\csname processbatchFile\endcsname\relax\else1\fi
         \ifx\fmtname\x\else 1\fi\relax
\else\csname fi\endcsname
%</ignore>
%<*install>
\input docstrip.tex
\Msg{************************************************************************}
\Msg{* Installation}
\Msg{* Package: bitset 2011/01/30 v1.1 Handle bit-vector datatype (HO)}
\Msg{************************************************************************}

\keepsilent
\askforoverwritefalse

\let\MetaPrefix\relax
\preamble

This is a generated file.

Project: bitset
Version: 2011/01/30 v1.1

Copyright (C) 2007, 2011 by
   Heiko Oberdiek <heiko.oberdiek at googlemail.com>

This work may be distributed and/or modified under the
conditions of the LaTeX Project Public License, either
version 1.3c of this license or (at your option) any later
version. This version of this license is in
   http://www.latex-project.org/lppl/lppl-1-3c.txt
and the latest version of this license is in
   http://www.latex-project.org/lppl.txt
and version 1.3 or later is part of all distributions of
LaTeX version 2005/12/01 or later.

This work has the LPPL maintenance status "maintained".

This Current Maintainer of this work is Heiko Oberdiek.

The Base Interpreter refers to any `TeX-Format',
because some files are installed in TDS:tex/generic//.

This work consists of the main source file bitset.dtx
and the derived files
   bitset.sty, bitset.pdf, bitset.ins, bitset.drv, bitset-test1.tex,
   bitset-test2.tex, bitset-test3.tex.

\endpreamble
\let\MetaPrefix\DoubleperCent

\generate{%
  \file{bitset.ins}{\from{bitset.dtx}{install}}%
  \file{bitset.drv}{\from{bitset.dtx}{driver}}%
  \usedir{tex/generic/oberdiek}%
  \file{bitset.sty}{\from{bitset.dtx}{package}}%
  \usedir{doc/latex/oberdiek/test}%
  \file{bitset-test1.tex}{\from{bitset.dtx}{test1}}%
  \file{bitset-test2.tex}{\from{bitset.dtx}{test2,etex}}%
  \file{bitset-test3.tex}{\from{bitset.dtx}{test2,noetex}}%
  \nopreamble
  \nopostamble
  \usedir{source/latex/oberdiek/catalogue}%
  \file{bitset.xml}{\from{bitset.dtx}{catalogue}}%
}

\catcode32=13\relax% active space
\let =\space%
\Msg{************************************************************************}
\Msg{*}
\Msg{* To finish the installation you have to move the following}
\Msg{* file into a directory searched by TeX:}
\Msg{*}
\Msg{*     bitset.sty}
\Msg{*}
\Msg{* To produce the documentation run the file `bitset.drv'}
\Msg{* through LaTeX.}
\Msg{*}
\Msg{* Happy TeXing!}
\Msg{*}
\Msg{************************************************************************}

\endbatchfile
%</install>
%<*ignore>
\fi
%</ignore>
%<*driver>
\NeedsTeXFormat{LaTeX2e}
\ProvidesFile{bitset.drv}%
  [2011/01/30 v1.1 Handle bit-vector datatype (HO)]%
\documentclass{ltxdoc}
\usepackage{holtxdoc}[2011/11/22]
\usepackage{amstext}
\usepackage{ifluatex}
\ifluatex
\else
  \usepackage[T1]{fontenc}%
  \usepackage{lmodern}%
\fi
\begin{document}
  \DocInput{bitset.dtx}%
\end{document}
%</driver>
% \fi
%
% \CheckSum{3510}
%
% \CharacterTable
%  {Upper-case    \A\B\C\D\E\F\G\H\I\J\K\L\M\N\O\P\Q\R\S\T\U\V\W\X\Y\Z
%   Lower-case    \a\b\c\d\e\f\g\h\i\j\k\l\m\n\o\p\q\r\s\t\u\v\w\x\y\z
%   Digits        \0\1\2\3\4\5\6\7\8\9
%   Exclamation   \!     Double quote  \"     Hash (number) \#
%   Dollar        \$     Percent       \%     Ampersand     \&
%   Acute accent  \'     Left paren    \(     Right paren   \)
%   Asterisk      \*     Plus          \+     Comma         \,
%   Minus         \-     Point         \.     Solidus       \/
%   Colon         \:     Semicolon     \;     Less than     \<
%   Equals        \=     Greater than  \>     Question mark \?
%   Commercial at \@     Left bracket  \[     Backslash     \\
%   Right bracket \]     Circumflex    \^     Underscore    \_
%   Grave accent  \`     Left brace    \{     Vertical bar  \|
%   Right brace   \}     Tilde         \~}
%
% \GetFileInfo{bitset.drv}
%
% \title{The \xpackage{bitset} package}
% \date{2011/01/30 v1.1}
% \author{Heiko Oberdiek\\\xemail{heiko.oberdiek at googlemail.com}}
%
% \maketitle
%
% \begin{abstract}
% This package defines and implements the data type bit set,
% a vector of bits. The size of the vector may grow dynamically.
% Individual bits can be manipulated.
% \end{abstract}
%
% \tableofcontents
%
% \clearpage
% \section{Documentation}
%
% \subsection{Introduction}
%
% Annotations in the PDF format know entries whose values are
% integers. This numbers are interpreted as set of flags
% specifying properties. For example, annotation dictionaries
% can have a key |/F|. The bits of its integer value
% are interpreted the following way:
% \begin{quote}
% \begin{tabular}{@{}l|l@{}}
%   Bit position & Property name\\
%   \hline
%   1 & Invisible\\
%   2 & Hidden\\
%   3 & Print\\
%   4 & NoZoom\\
%   5 & NoRotate\\
%   6 & NoView\\
%   7 & ReadOnly\\
%   \dots & \dots
% \end{tabular}
% \end{quote}
% Now, let's see how these values are set in package \xpackage{hyperref}
% before it uses this package (before v6.77a):
%\begin{quote}
%\begin{verbatim}
%\ifFld@hidden /F 6\else /F 4\fi
%\end{verbatim}
%\end{quote}
% Where are the other flags? The following example for key |/Ff|
% in a widget annotation
% supports at least three properties:
%\begin{quote}
%\begin{verbatim}
%\ifFld@multiline
%  \ifFld@readonly /Ff 4097\else /Ff 4096\fi
%\else
%  \ifFld@password
%    \ifFld@readonly /Ff 8193\else /Ff 8192\fi
%  \else
%    \ifFld@readonly /Ff 1\fi
%  \fi
%\fi
%\end{verbatim}
%\end{quote}
% But you see the point. It would be a nightmare to continue
% this way in supporting the missing flag settings. This kind
% of integers may have up to 32 bits.
%
% Therefore I wanted a data structure for setting and
% clearing individual bits. Also it should provide
% an export as decimal number. The snipsets above are
% executed in expansion contexts without \TeX's stomach
% commands. It would be convenient to have an expandable
% conversion from the data structure to the integer that
% gets written to the PDF file.
%
% This package \xpackage{bitset} implements such a data
% structure. The interface is quite close to Java's
% class |BitSet| in order not to learn to many interfaces
% for the same kind of data structure.
%
% \subsection{Glossary}
%
% \begin{description}
% \item[Bit set:]
%   A bit set is a vector of bits or flags.
%   The vector size is unlimited and
%   grows dynamically. An undefined bit set is treated as bit
%   set where all bits are cleared.
%
%   Bit sets are addressed by name. A name should consists of
%   letters or digits. Technically it must survive \cs{csname},
%   see \LaTeX's environment names for other names with such
%   a constraint. Package \xpackage{babel}'s shorthands are
%   not supported due to technical reasons. Shorthand support
%   would break expandable operations.
% \item[Size:]
%   A size of a bit set is the number of bits in use.
%   It's the number of the highest index, incremented by one.
%   Sizes are in the range |0| up to |2147483647|, the highest
%   number supported by \TeX.
% \item[Index:]
%   Bit positions in a bit set are addressed by an index number.
%   The bit vector is zero based. The first and least significant
%   bit is addressed by index |0| and the highest possible bit
%   by |2147483646|.
% \item[Bit:]
%   A bit is enoded as |0| for cleared/disabled or |1| for set/enabled.
% \end{description}
%
% \subsection{Design principles}
%
% \begin{description}
% \item[Name conventions:]
%   To avoid conflicts with existing macro names, the operations
%   are prefixed by the package name.
% \item[Zero based indexes:] The first bit is addressed by zero.
%   (Convention of array indexing in C, Java, \dots)
% \item[Unlimited size:]
%   There is no restriction on the size of a bit set other than
%   usual memory limitations. \cs{bitsetSetDec} and \cs{bitsetGetDec}
%   transparently switch to package \xpackage{bigintcalc} if the numbers
%   get too large for \TeX's number limit.
% \item[Expandibility:]
%   Any operation that does not change the bit set is expandable.
%   And all operations that extract or calculate some result
%   do this in exact two expansion steps. For example,
%   a macro \cs{Macro} wants a bit set as decimal number.
%   But the argument must be a plain number without macros.
%   Thus you could prefix \cs{bitsetGetDec} with \cs{number}.
%   However this won't work for bit sets with 31 or more bits
%   because of \TeX's number limit of $2^{31}-1$.
%   then just hit the operator with two \cs{expandafter}:
%   \begin{quote}
%   |\expandafter\expandafter\expandafter|\\
%   |\Macro\bitsetGetDec{foo}|
%   \end{quote}
%   |\bitsetGetDec| is hit first by the third \cs{expandafter}
%   and then by the second one.
% \item[Format independence:] This package is written as \LaTeX\ package,
%   but it does not depend on \LaTeX. It will also work for other formats
%   such as \plainTeX.
% \item[Independence from \TeX\ engines:]
%   Vanilla \TeX\ is all you need. Calculations are delegated to
%   packages \xpackage{intcalc} and \xpackage{bigintcalc}.
%   They don't need any special features, but they will switch
%   to a little more efficient implementation if features such
%   as \cs{numexpr} are available.
% \item[Numeric arguments:] Anything that is accepted by
%   \cs{number}. If \eTeX\ is detected, also expressions
%   for \cs{numexpr} are supported. The only exception so far
%   is the number for \cs{bitsetSetDec}. The number might be too
%   large for \cs{number} or \cs{numexpr}.
% \item[Error messages:]
%   In expandable contexts, only a limited set of \TeX\ primitive
%   commands work as expected. So called stomach commands behave
%   like \cs{relax} and don't get expanded or executed.
%   Unhappily also the error commands belong to this category.
%   The expandable operations will throw an unknown control
%   sequence instead to get \TeX's and user's attention.
%   The name of these control sequences starts with
%   \cs{BitSetError:} with the type of error after the colon.
% \end{description}
%
% \subsection{Operator overview}
%
% \begin{minipage}{\linewidth}
% \begin{description}
% \def\It#1#2{^^A
%   \item[^^A
%     {#1} ^^A
%     \textmd{(\hyperref[{sec:#2}]{section \ref*{sec:#2}})}^^A
%   ]^^A
%   \mbox{}\\^^A
% }
% \def\op#1{^^A
%   \cs{bitset}\texttt{\bfseries#1}^^A
% }
% \def\M#1{\meta{#1}}
% \def\Q{\hspace*{\fill}}
% \It{Miscellaneous}{misc}
%   \op{Reset}\Q \M{BitSet}\\
%   \op{Let}\Q \M{BitSet A} \M{BitSet B}
% \It{Import}{import}
%   \op{SetBin}, \op{SetOct}, \op{SetHex}\Q \M{BitSet} \M{Value}\\
%   \op{SetDec}\Q \M{BitSet} \M{Value}
% \It{Export\textsuperscript{a}}{export}
%   \op{GetBin}, \op{GetOct}, \op{GetHex}\Q \M{BitSet} \M{MinSize}\\
%   \op{GetDec}\Q \M{BitSet}
% \It{Logical operators}{log}
%   \op{And}, \op{AndNot}\Q \M{BitSet A} \M{BitSet B}\\
%   \op{Or}, \op{Xor}\Q \M{BitSet A} \M{BitSet B}
% \It{Shifting}{shift}
%   \op{ShiftLeft}, \op{ShiftRight}\Q \M{BitSet} \M{ShiftAmount}\\
% \It{Bit manipulation}{bitman}
%   \op{Clear}, \op{Set}, \op{Flip}\Q \M{BitSet} \M{Index}\\
%   \op{SetValue}\Q \M{BitSet} \M{Index} \M{Value}\\
%   \op{ClearRange}, \op{SetRange}, \op{FlipRange}\\
%     \Q \M{BitSet} \M{IndexFrom} \M{IndexTo}\\
%   \op{SetValueRange}\Q \M{BitSet} \M{IndexFrom} \M{IndexTo}
% \It{Bit retrieval\textsuperscript{a}}{bitret}
%   \op{Get}\Q \M{BitSet} \M{Index}\\
%   \op{NextClearBit}, \op{NextSetBit}\Q \M{BitSet} \M{Index}\\
%   \op{GetSetBitList}\Q \M{BitSet}
% \It{Bit set properties}{prop}
%   \op{Size}, \op{Cardinality}\Q \M{BitSet}
% \It{Queries\textsuperscript{b}}{query}
%   \op{IsDefined}, \op{IsEmpty}\Q \M{BitSet} \M{Then} \M{Else}\\
%   \op{Equals}, \op{Intersects}\Q \M{BitSet A} \M{BitSet B} \M{Then} \M{Else}\\
%   \op{Query}\Q \M{BitSet} \M{Index} \M{Then} \M{Else}
% \end{description}
% \footnotetext[1]{Macros are expandable, full expansion by two steps.}
% \footnotetext[2]{Macros are expandable.}
% \end{minipage}
%
% \subsection{Package loading}
%
% The package can be used as normal \LaTeX\ package:
%\begin{quote}
%|\usepackage{bitset}|
%\end{quote}
% Also \plainTeX\ is supported:
%\begin{quote}
%|\input bitset.sty\relax|
%\end{quote}
%
% \subsection{Operators}
%
% The following macros work on and with bit sets.
% A bit set \meta{BitSet} is represented by a name.
% The should consist of letters and digits. Technically
% it must survive \cs{csname}. It is the same constraint
% that must be satisfied by label or environment names
% in \LaTeX.
%
% However active characters that are shorthands
% of package \xpackage{babel} are not supported.
% Support for shorthands works by an assignment.
% But many operators
% such as \cs{bitsetGetDec} must be usable in expandable
% contexts. There assignments will not be executed in the
% best case or they will cause errors.
%
% The bits in a bit set are addressed by non-negative
% integers starting from zero. Thus negative index
% numbers cause an error message. Because index numbers
% are \TeX\ numbers. The largest index is 2147483647.
% But in practice memory limits and patience limits
% will be very likely reached much before.
%
% \subsubsection{Miscellaneous}\label{sec:misc}
%
% There isn't a separate operation for bit set creation.
% For simplicity an undefined bit set is treated as bit set
% with all bits cleared.
%
% \begin{declcs}{bitsetReset} \M{BitSet}
% \end{declcs}
% Macro \cs{bitsetReset} clears all bits. The result is an
% empty bit set. It may also be used as replacement for
% an operation ``new'', because an undefined bit set is defined
% afterwards.
%
% \begin{declcs}{bitsetLet} \M{BitSet A} \M{BitSet B}
% \end{declcs}
% Macro \cs{bitsetLet} performs a simple assignment similar
% to \TeX's \cs{let}. After the operation \meta{BitSet A} has
% the same value as \meta{BitSet B}. If \meta{BitSet B} is
% undefined, then \meta{BitSet A} will be the empty bit set.
%
% Note: If \meta{BitSet A} exists, it will be overwritten.
%
% \subsubsection{Import}\label{sec:import}
%
% \begin{declcs}{bitsetSetBin} \M{BitSet} \M{BinaryNumber}\\
%   \cs{bitsetSetOct} \M{BitSet} \M{OctalNumber}\\
%   \cs{bitsetSetHex} \M{BitSet} \M{HexadecimalNumber}
% \end{declcs}
% The numbers are interpreted as bit vectors and the flags in
% the bit \meta{BitSet} set are set accordingly.
% These numeric arguments are the only arguments where
% spaces are allowed. Then the numbers are easier to read.
%
% \begin{declcs}{bitsetSetDec} \M{BitSet} \M{DecimalNumber}
% \end{declcs}
% Macro \cs{bitsetSetDec} uses \meta{DecimalNumber} to
% set the bit set \meta{BitSet}.
% The numeric argument must expand
% to a plain number consisting of decimal digits without
% command tokens or spaces. Internally this argument is
% expanded only. It cannot be passed to \cs{number} or
% \cs{numexpr}, because the number may be too large for them.
% However \cs{number} or |\the\numexpr| may be used
% explicitely. This also helps for unexpandable number
% command tokens or registers (\cs{z@}, \cs{@ne}, \cs{count@}, \dots).
% Also \LaTeX'\ \cs{value} needs prefixing:
% \begin{quote}
% |\bitsetSetDec{foo}{\number\value{bar}}|
% \end{quote}
%
% \subsubsection{Export}\label{sec:export}
%
% \begin{declcs}{bitsetGetBin} \M{BitSet} \M{MinSize}\\
%   \cs{bitsetGetOct} \M{BitSet} \M{MinSize}\\
%   \cs{bitsetGetHex} \M{BitSet} \M{MinSize}
% \end{declcs}
% These macros returns the bit set as binary, octal or hexadecimal
% number. If the bit size is smaller than \meta{MinSize} the gap is
% filled with leading zeros.
% Example:
% \begin{quote}
% |\bitsetReset{abc}|\\
% |\bitsetSet{abc}{2}|\\
% |\bitsetGetBin{abc}{8}| $\rightarrow$ |00000100|\\
% |\bitsetSet{abc}{5}\bitsetSet{abc}{7}|\\
% |\bitsetGetHex{abc}{16}| $\rightarrow$ |00A2|
% \end{quote}
% Macro \cs{bitsetGetHex} uses the uppercase letters |A| to |F|.
% The catcode of the letters is one of 11 (letter) or 12 (other).
%
% \begin{declcs}{bitsetGetDec} \M{BitSet}
% \end{declcs}
% Macro \cs{bitsetGetDec} returns the bit set \meta{BitSet} as
% decimal number. The returned number can be larger than
% \TeX's number limit of $2^{31}-1$.
%
% \subsubsection{Logical operators}\label{sec:log}
%
% \begin{declcs}{bitsetAnd} \M{BitSet A} \M{BitSet B}
% \end{declcs}
% $A_{\text{new}} \mathrel{\mathop{:}}= A_{\text{old}}
% \mathbin{\text{and}} B \qquad (\forall\, \text{bits})$
%
% \begin{declcs}{bitsetAndNot} \M{BitSet A} \M{BitSet B}
% \end{declcs}
% $A_{\text{new}} \mathrel{\mathop{:}}= A_{\text{old}}
% \mathbin{\text{and}} (\text{not}\,B) \qquad (\forall\, \text{bits})$
%
% \begin{declcs}{bitsetOr} \M{BitSet A} \M{BitSet B}
% \end{declcs}
% $A_{\text{new}} \mathrel{\mathop{:}}= A_{\text{old}}
% \mathbin{\text{or}} B \qquad (\forall\, \text{bits})$
%
% \begin{declcs}{bitsetXor} \M{BitSet A} \M{BitSet B}
% \end{declcs}
% $A_{\text{new}} \mathrel{\mathop{:}}= A_{\text{old}}
% \mathbin{\text{xor}} B \qquad (\forall\, \text{bits})$
%
% \subsubsection{Shifting}\label{sec:shift}
%
% \begin{declcs}{bitsetShiftLeft} \M{BitSet} \M{ShiftAmount}\\
%   \cs{bitsetShiftRight} \M{BitSet} \M{ShiftAmount}
% \end{declcs}
% A left shift by one is a multiplication by two, thus left shifting
% moves the flags to higher positions. The new created low positions
% are filled by zeros.
%
% A right shift is the opposite, dividing by two, movint the bits
% to lower positions. The number will become smaller, the lowest
% bits are lost.
%
% If the \meta{ShiftAmount} is negative, it reverts the meaning of
% the shift operation. A left shift becomes a right shift.
% A \meta{ShiftAmount} of zero is ignored.
%
% \subsubsection{Bit manipulation}\label{sec:bitman}
%
% \begin{declcs}{bitsetClear} \M{BitSet} \M{Index}\\
%   \cs{bitsetSet} \M{BitSet} \M{Index}\\
%   \cs{bitsetFlip} \M{BitSet} \M{Index}
% \end{declcs}
% This macros manipulate a single bit in \meta{BitSet} addressed
% by \cs{Index}. Macro \cs{bitsetClear} disables the bit,
% \cs{bitsetSet} enables it and \cs{bitsetFlip} reverts the current
% setting of the bit.
%
% \begin{declcs}{bitsetSetValue} \M{BitSet} \M{Index} \M{Bit}
% \end{declcs}
% Macro \cs{bitsetSetValue} puts bit \meta{Bit} at position \meta{Index}
% in bit set \meta{BitSet}. \meta{Bit} must be a valid \TeX\ number
% equals to zero (disabled/cleared) or one (enabled/set).
%
% \subsubsection{Bit retrieval}\label{sec:bitret}
%
% \begin{declcs}{bitsetGet} \M{BitSet} \M{Index}
% \end{declcs}
% Macro \cs{bitsetGet} extracts the status of the bit at
% position \meta{Index} in bit set \meta{BitSet}.
% Digit |1| is returned if the bit is set/enabled. If the
% bit is cleared/disabled and in cases of an undefined bitset or
% an index number out of range the return value is |0|.
%
% \begin{declcs}{bitsetNextClearBit} \M{BitSet} \M{Index}
% \end{declcs}
% Starting at position \meta{Index} (inclusive) the bits
% are inspected. The first position without a set bit is returned.
% Possible results are decimal numbers: \meta{Index}, \meta{Index} + 1,
% \dots, ($\infty$)
%
% \begin{declcs}{bitsetNextSetBit} \M{BitSet} \M{Index}
% \end{declcs}
% Starting at position \meta{Index} (inclusive) the bits are
% inspected and the index position of the first found set bit
% is returned. If there isn't such a bit, then the result is
% |-1|. In summary possible results are decimal numbers:
% |-1|, \meta{Index}, \meta{Index} + 1, \dots, ($\infty$)
%
% \begin{declcs}{bitsetGetSetBitList} \M{BitSet}
% \end{declcs}
% Macro \cs{bitsetGetSetBitList} is an application for
% \cs{bitsetNextSetBit}. The set bits are iterated and
% returned as comma separated list of index positions
% in increasing order. The list is empty in case of
% an empty bit set.
%
% \subsubsection{Bit set properties}\label{sec:prop}
%
% \begin{declcs}{bitsetSize} \M{BitSet}
% \end{declcs}
% Macro \cs{bitsetSize} returns number of bits in use.
% It is the same as the index number of the highest set/enabled bit
% incremented by one.
%
% \begin{declcs}{bitsetCardinality} \M{BitSet}
% \end{declcs}
% Macro \cs{bitsetCardinality} counts the number of set/enabled bits.
%
% \subsubsection{Queries}\label{sec:query}
%
% Also the query procedures are expandable. They ask for a piece
% of information about a bit set and execute code depending
% on the answer.
%
% \begin{declcs}{bitsetIsDefined} \M{BitSet} \M{Then} \M{Else}
% \end{declcs}
% If the bit set with the name \meta{BitSet} exists the code
% given in \meta{Then} is executed, otherwise \meta{Else} is used.
%
% \begin{declcs}{bitsetIsEmpty} \M{BitSet} \M{Then} \M{Else}
% \end{declcs}
% If the bit set \meta{BitSet} exists and at least one bit
% is set/enabled, the code in \meta{Then} is executed,
% \meta{Else} otherwise.
%
% \begin{declcs}{bitsetEquals} \M{BitSet A} \M{BitSet B} \M{Then} \M{Else}
% \end{declcs}
% Both bit sets are equal if and only if either both are undefined
% or both are defined and represents the same bit values at the
% same positions. Thus this definition is reflexive, symmeric,
% and transitive, enough for an equivalent relation.
%
% \begin{declcs}{bitsetIntersects} \M{BitSet A} \M{BitSet B} \M{Then} \M{Else}
% \end{declcs}
% If and only if \meta{BitSet A} and \meta{BitSet B} have
% at least one bit at the same position that is set, then
% code part \meta{Then} is executed.
%
% \begin{declcs}{bitsetQuery} \M{BitSet} \M{Index} \M{Then} \M{Else}
% \end{declcs}
% It's just a wrapper for \cs{bitsetGet}. If the bit at position
% \meta{Index} is enabled, code \meta{Then} is called.
%
% \StopEventually{
% }
%
% \section{Implementation}
%
% The internal format of a bit set is quite simple, a sequence
% of digits |0| and |1|. The least significant bit is left.
% A bit set without any flag set is encoded by |0|. Also
% undefined bit sets are treated that way. After the
% highest bit that is set there are no further zeroes.
% A regular expression of valid bit sets values:
% \begin{quote}
%  |0|\string||[01]*1|
% \end{quote}
%
%    \begin{macrocode}
%<*package>
%    \end{macrocode}
%
% \subsection{Reload check and package identification}
%    Reload check, especially if the package is not used with \LaTeX.
%    \begin{macrocode}
\begingroup\catcode61\catcode48\catcode32=10\relax%
  \catcode13=5 % ^^M
  \endlinechar=13 %
  \catcode35=6 % #
  \catcode39=12 % '
  \catcode44=12 % ,
  \catcode45=12 % -
  \catcode46=12 % .
  \catcode58=12 % :
  \catcode64=11 % @
  \catcode123=1 % {
  \catcode125=2 % }
  \expandafter\let\expandafter\x\csname ver@bitset.sty\endcsname
  \ifx\x\relax % plain-TeX, first loading
  \else
    \def\empty{}%
    \ifx\x\empty % LaTeX, first loading,
      % variable is initialized, but \ProvidesPackage not yet seen
    \else
      \expandafter\ifx\csname PackageInfo\endcsname\relax
        \def\x#1#2{%
          \immediate\write-1{Package #1 Info: #2.}%
        }%
      \else
        \def\x#1#2{\PackageInfo{#1}{#2, stopped}}%
      \fi
      \x{bitset}{The package is already loaded}%
      \aftergroup\endinput
    \fi
  \fi
\endgroup%
%    \end{macrocode}
%    Package identification:
%    \begin{macrocode}
\begingroup\catcode61\catcode48\catcode32=10\relax%
  \catcode13=5 % ^^M
  \endlinechar=13 %
  \catcode35=6 % #
  \catcode39=12 % '
  \catcode40=12 % (
  \catcode41=12 % )
  \catcode44=12 % ,
  \catcode45=12 % -
  \catcode46=12 % .
  \catcode47=12 % /
  \catcode58=12 % :
  \catcode64=11 % @
  \catcode91=12 % [
  \catcode93=12 % ]
  \catcode123=1 % {
  \catcode125=2 % }
  \expandafter\ifx\csname ProvidesPackage\endcsname\relax
    \def\x#1#2#3[#4]{\endgroup
      \immediate\write-1{Package: #3 #4}%
      \xdef#1{#4}%
    }%
  \else
    \def\x#1#2[#3]{\endgroup
      #2[{#3}]%
      \ifx#1\@undefined
        \xdef#1{#3}%
      \fi
      \ifx#1\relax
        \xdef#1{#3}%
      \fi
    }%
  \fi
\expandafter\x\csname ver@bitset.sty\endcsname
\ProvidesPackage{bitset}%
  [2011/01/30 v1.1 Handle bit-vector datatype (HO)]%
%    \end{macrocode}
%
% \subsection{Catcodes}
%
%    \begin{macrocode}
\begingroup\catcode61\catcode48\catcode32=10\relax%
  \catcode13=5 % ^^M
  \endlinechar=13 %
  \catcode123=1 % {
  \catcode125=2 % }
  \catcode64=11 % @
  \def\x{\endgroup
    \expandafter\edef\csname BitSet@AtEnd\endcsname{%
      \endlinechar=\the\endlinechar\relax
      \catcode13=\the\catcode13\relax
      \catcode32=\the\catcode32\relax
      \catcode35=\the\catcode35\relax
      \catcode61=\the\catcode61\relax
      \catcode64=\the\catcode64\relax
      \catcode123=\the\catcode123\relax
      \catcode125=\the\catcode125\relax
    }%
  }%
\x\catcode61\catcode48\catcode32=10\relax%
\catcode13=5 % ^^M
\endlinechar=13 %
\catcode35=6 % #
\catcode64=11 % @
\catcode123=1 % {
\catcode125=2 % }
\def\TMP@EnsureCode#1#2{%
  \edef\BitSet@AtEnd{%
    \BitSet@AtEnd
    \catcode#1=\the\catcode#1\relax
  }%
  \catcode#1=#2\relax
}
\TMP@EnsureCode{33}{12}% !
\TMP@EnsureCode{39}{12}% '
\TMP@EnsureCode{40}{12}% (
\TMP@EnsureCode{41}{12}% )
\TMP@EnsureCode{42}{12}% *
\TMP@EnsureCode{43}{12}% +
\TMP@EnsureCode{44}{12}% ,
\TMP@EnsureCode{45}{12}% -
\TMP@EnsureCode{46}{12}% .
\TMP@EnsureCode{47}{12}% /
\TMP@EnsureCode{58}{11}% : (letter!)
\TMP@EnsureCode{60}{12}% <
\TMP@EnsureCode{62}{12}% >
\TMP@EnsureCode{63}{14}% ? (comment!)
\TMP@EnsureCode{91}{12}% [
\TMP@EnsureCode{93}{12}% ]
\TMP@EnsureCode{96}{12}% `
\edef\BitSet@AtEnd{\BitSet@AtEnd\noexpand\endinput}
\begingroup\expandafter\expandafter\expandafter\endgroup
\expandafter\ifx\csname BitSet@TestMode\endcsname\relax
\else
  \catcode63=9 % ? (ignore)
\fi
? \let\BitSet@@TestMode\BitSet@TestMode
%    \end{macrocode}
%
% \subsection{Package loading}
%
%    \begin{macrocode}
\begingroup\expandafter\expandafter\expandafter\endgroup
\expandafter\ifx\csname RequirePackage\endcsname\relax
  \def\TMP@RequirePackage#1[#2]{%
    \begingroup\expandafter\expandafter\expandafter\endgroup
    \expandafter\ifx\csname ver@#1.sty\endcsname\relax
      \input #1.sty\relax
    \fi
  }%
  \TMP@RequirePackage{infwarerr}[2007/09/09]%
  \TMP@RequirePackage{intcalc}[2007/09/27]%
  \TMP@RequirePackage{bigintcalc}[2007/09/27]%
\else
  \RequirePackage{infwarerr}[2007/09/09]%
  \RequirePackage{intcalc}[2007/09/27]%
  \RequirePackage{bigintcalc}[2007/09/27]%
\fi
%    \end{macrocode}
%
% \subsection{Help macros}
%
% \subsubsection{Number constant}
%
%    \begin{macro}{\BitSet@MaxSize}
%    \begin{macrocode}
\def\BitSet@MaxSize{2147483647}%
%    \end{macrocode}
%    \end{macro}
%
% \subsubsection{General basic macros}
%
%    \begin{macro}{\BitSet@Empty}
%    \begin{macrocode}
\def\BitSet@Empty{}
%    \end{macrocode}
%    \end{macro}
%    \begin{macro}{\BitSet@FirstOfOne}
%    \begin{macrocode}
\def\BitSet@FirstOfOne#1{#1}
%    \end{macrocode}
%    \end{macro}
%    \begin{macro}{\BitSet@Gobble}
%    \begin{macrocode}
\def\BitSet@Gobble#1{}
%    \end{macrocode}
%    \end{macro}
%    \begin{macro}{\BitSet@FirstOfTwo}
%    \begin{macrocode}
\def\BitSet@FirstOfTwo#1#2{#1}
%    \end{macrocode}
%    \end{macro}
%    \begin{macro}{\BitSet@SecondOfTwo}
%    \begin{macrocode}
\def\BitSet@SecondOfTwo#1#2{#2}
%    \end{macrocode}
%    \end{macro}
%
%    \begin{macro}{\BitSet@Space}
%    \begin{macrocode}
\def\BitSet@Space{ }
%    \end{macrocode}
%    \end{macro}
%    \begin{macro}{\BitSet@ZapSpace}
%    \begin{macrocode}
\def\BitSet@ZapSpace#1 #2{%
  #1%
  \ifx\BitSet@Empty#2%
  \else
    \expandafter\BitSet@ZapSpace
  \fi
  #2%
}
%    \end{macrocode}
%    \end{macro}
%
% \subsubsection{Tail recursion}
%
%    \begin{macro}{\BitSet@Fi}
%    \begin{macrocode}
\let\BitSet@Fi\fi
%    \end{macrocode}
%    \end{macro}
%    \begin{macro}{\BitSet@AfterFi}
%    \begin{macrocode}
\def\BitSet@AfterFi#1#2\BitSet@Fi{\fi#1}
%    \end{macrocode}
%    \end{macro}
%    \begin{macro}{\BitSet@AfterFiFi}
%    \begin{macrocode}
\def\BitSet@AfterFiFi#1#2\BitSet@Fi{\fi\fi#1}%
%    \end{macrocode}
%    \end{macro}
%    \begin{macro}{\BitSet@AfterFiFiFi}
%    \begin{macrocode}
\def\BitSet@AfterFiFiFi#1#2\BitSet@Fi{\fi\fi\fi#1}%
%    \end{macrocode}
%    \end{macro}
%
% \subsubsection{Check macros}
%
%    \begin{macro}{\BitSet@IfUndefined}
%    \begin{macrocode}
\def\BitSet@IfUndefined#1{%
  \expandafter\ifx\csname BS@#1\endcsname\relax
    \expandafter\BitSet@FirstOfTwo
  \else
    \expandafter\BitSet@SecondOfTwo
  \fi
}
%    \end{macrocode}
%    \end{macro}
%
%    \begin{macro}{\BitSet@CheckIndex}
%    |#1|: continuation code\\
%    |#2|: BitSet\\
%    |#3|: Index
%    \begin{macrocode}
\def\BitSet@CheckIndex#1#2#3{%
  \BitSet@IfUndefined{#2}{\bitsetReset{#2}}{}%
  \expandafter\expandafter\expandafter\BitSet@@CheckIndex
  \intcalcNum{#3}!%
  {#2}{#1}%
}
%    \end{macrocode}
%    \end{macro}
%    \begin{macro}{\BitSet@@CheckIndex}
%    |#1|: plain Index\\
%    |#2|: BitSet\\
%    |#3|: continuation code
%    \begin{macrocode}
\def\BitSet@@CheckIndex#1!#2#3{%
  \ifnum#1<0 %
    \BitSet@AfterFi{%
      \@PackageError{bitset}{%
        Invalid negative index (#1)%
      }\@ehc
    }%
  \else
    \BitSet@AfterFi{%
      #3{#2}{#1}%
    }%
  \BitSet@Fi
}
%    \end{macrocode}
%    \end{macro}
%
% \subsection{Miscellaneous}\label{src:misc}
%
%    \begin{macro}{\bitsetReset}
%    \begin{macrocode}
\def\bitsetReset#1{%
  \expandafter\def\csname BS@#1\endcsname{0}%
}
%    \end{macrocode}
%    \end{macro}
%
%    \begin{macro}{\bitsetLet}
%    \begin{macrocode}
\def\bitsetLet#1#2{%
  \BitSet@IfUndefined{#2}{%
    \bitsetReset{#1}%
  }{%
    \expandafter\let\csname BS@#1\expandafter\endcsname
                    \csname BS@#2\endcsname
  }%
}
%    \end{macrocode}
%    \end{macro}
%
% \subsection{Import}\label{src:import}
%
% \subsubsection{From binary number}
%
%    \begin{macro}{\bitsetSetBin}
%    \begin{macrocode}
\def\bitsetSetBin#1#2{%
  \edef\BitSet@Temp{#2}%
  \edef\BitSet@Temp{%
    \expandafter\expandafter\expandafter\BitSet@ZapSpace
    \expandafter\BitSet@Temp\BitSet@Space\BitSet@Empty
  }%
  \edef\BitSet@Temp{%
    \expandafter\BitSet@KillZeros\BitSet@Temp\BitSet@Empty
  }%
  \ifx\BitSet@Temp\BitSet@Empty
    \expandafter\let\csname BS@#1\endcsname\BitSet@Zero
  \else
    \expandafter\edef\csname BS@#1\endcsname{%
      \expandafter\BitSet@Reverse\BitSet@Temp!%
    }%
  \fi
}
%    \end{macrocode}
%    \end{macro}
%    \begin{macro}{\BitSet@KillZeros}
%    \begin{macrocode}
\def\BitSet@KillZeros#1{%
  \ifx#10%
    \expandafter\BitSet@KillZeros
  \else
    #1%
  \fi
}
%    \end{macrocode}
%    \end{macro}
%    \begin{macro}{\BitSet@Reverse}
%    \begin{macrocode}
\def\BitSet@Reverse#1#2!{%
  \ifx\\#2\\%
    #1%
  \else
    \BitSet@AfterFi{%
      \BitSet@Reverse#2!#1%
    }%
  \BitSet@Fi
}
%    \end{macrocode}
%    \end{macro}
%
% \subsubsection{From octal/hex number}
%
%    \begin{macro}{\bitsetSetOct}
%    \begin{macrocode}
\def\bitsetSetOct{%
  \BitSet@SetOctHex\BitSet@FromFirstOct
}
%    \end{macrocode}
%    \end{macro}
%    \begin{macro}{\bitsetSetHex}
%    \begin{macrocode}
\def\bitsetSetHex{%
  \BitSet@SetOctHex\BitSet@FromFirstHex
}
%    \end{macrocode}
%    \end{macro}
%    \begin{macro}{\BitSet@SetOctHex}
%    \begin{macrocode}
\def\BitSet@SetOctHex#1#2#3{%
  \edef\BitSet@Temp{#3}%
  \edef\BitSet@Temp{%
    \expandafter\expandafter\expandafter\BitSet@ZapSpace
    \expandafter\BitSet@Temp\BitSet@Space\BitSet@Empty
  }%
  \edef\BitSet@Temp{%
    \expandafter\BitSet@KillZeros\BitSet@Temp\BitSet@Empty
  }%
  \ifx\BitSet@Temp\BitSet@Empty
    \expandafter\let\csname BS@#2\endcsname\BitSet@Zero
  \else
    \edef\BitSet@Temp{%
      \expandafter#1\BitSet@Temp!%
    }%
    \ifx\BitSet@Temp\BitSet@Empty
      \expandafter\let\csname BS@#2\endcsname\BitSet@Zero
    \else
      \expandafter\edef\csname BS@#2\endcsname{%
        \expandafter\BitSet@Reverse\BitSet@Temp!%
      }%
    \fi
  \fi
}
%    \end{macrocode}
%    \end{macro}
%    \begin{macro}{\BitSet@FromFirstOct}
%    \begin{macrocode}
\def\BitSet@FromFirstOct#1{%
  \ifx#1!%
  \else
    \ifcase#1 \BitSet@AfterFiFi\BitSet@FromFirstOct
    \or 1%
    \or 10%
    \or 11%
    \or 100%
    \or 101%
    \or 110%
    \or 111%
    \else \BitSetError:WrongOctalDigit%
    \fi
    \expandafter\BitSet@FromOct
  \BitSet@Fi
}
%    \end{macrocode}
%    \end{macro}
%    \begin{macro}{\BitSet@FromOct}
%    \begin{macrocode}
\def\BitSet@FromOct#1{%
  \ifx#1!%
  \else
    \ifcase#1 000%
    \or 001%
    \or 010%
    \or 011%
    \or 100%
    \or 101%
    \or 110%
    \or 111%
    \else \BitSetError:WrongOctalDigit%
    \fi
    \expandafter\BitSet@FromOct
  \fi
}
%    \end{macrocode}
%    \end{macro}
%
%    \begin{macro}{\BitSet@FromFirstHex}
%    \begin{macrocode}
\def\BitSet@FromFirstHex#1{%
  \ifx#1!%
  \else
    \ifx#10%
      \BitSet@AfterFiFi\BitSet@FromFirstHex
    \fi
    \expandafter\ifx\csname BitSet@Hex#1\endcsname\relax
      \BitSetError:InvalidHexDigit%
    \else
      \expandafter\expandafter\expandafter\BitSet@KillZeros
      \csname BitSet@Hex#1\endcsname
    \fi
    \expandafter\BitSet@FromHex
  \BitSet@Fi
}
%    \end{macrocode}
%    \end{macro}
%    \begin{macro}{\BitSet@FromHex}
%    \begin{macrocode}
\def\BitSet@FromHex#1{%
  \ifx#1!%
  \else
    \expandafter\ifx\csname BitSet@Hex#1\endcsname\relax
      \BitSetError:InvalidHexDigit%
    \else
      \csname BitSet@Hex#1\endcsname
    \fi
    \expandafter\BitSet@FromHex
  \fi
}
%    \end{macrocode}
%    \end{macro}
%    \begin{macro}{\BitSet@Hex[0..F]}
%    \begin{macrocode}
\def\BitSet@Temp#1{%
  \expandafter\def\csname BitSet@Hex#1\endcsname
}
\BitSet@Temp 0{0000}%
\BitSet@Temp 1{0001}%
\BitSet@Temp 2{0010}%
\BitSet@Temp 3{0011}%
\BitSet@Temp 4{0100}%
\BitSet@Temp 5{0101}%
\BitSet@Temp 6{0110}%
\BitSet@Temp 7{0111}%
\BitSet@Temp 8{1000}%
\BitSet@Temp 9{1001}%
\BitSet@Temp A{1010}%
\BitSet@Temp B{1011}%
\BitSet@Temp C{1100}%
\BitSet@Temp D{1101}%
\BitSet@Temp E{1110}%
\BitSet@Temp F{1111}%
\BitSet@Temp a{1010}%
\BitSet@Temp b{1011}%
\BitSet@Temp c{1100}%
\BitSet@Temp d{1101}%
\BitSet@Temp e{1110}%
\BitSet@Temp f{1111}%
%    \end{macrocode}
%    \end{macro}
%
% \subsubsection{From decimal number}
%
%    \begin{macro}{\bitsetSetDec}
%    \begin{macrocode}
\def\bitsetSetDec#1#2{%
  \edef\BitSet@Temp{#2}%
  \edef\BitSet@Temp{%
    \expandafter\expandafter\expandafter\BitSet@ZapSpace
    \expandafter\BitSet@Temp\BitSet@Space\BitSet@Empty
  }%
  \edef\BitSet@Temp{%
    \expandafter\BitSet@KillZeros\BitSet@Temp\BitSet@Empty
  }%
  \ifx\BitSet@Temp\BitSet@Empty
    \expandafter\let\csname BS@#1\endcsname\BitSet@Zero
  \else
    \ifcase\bigintcalcSgn{\BitSet@Temp} %
      \expandafter\let\csname BS@#1\endcsname\BitSet@Zero
    \or
      \ifnum\bigintcalcCmp\BitSet@Temp\BitSet@MaxSize>0 %
        \expandafter\edef\csname BS@#1\endcsname{%
          \expandafter\BitSet@SetDecBig\BitSet@Temp!%
        }%
      \else
        \expandafter\edef\csname BS@#1\endcsname{%
          \expandafter\BitSet@SetDec\BitSet@Temp!%
        }%
      \fi
    \else
      \@PackageError{bitset}{%
        Bit sets cannot be negative%
      }\@ehc
    \fi
  \fi
}
%    \end{macrocode}
%    \end{macro}
%    \begin{macro}{\BitSet@SetDecBig}
%    \begin{macrocode}
\def\BitSet@SetDecBig#1#2#3#4#5#6#7#8#9!{%
  \ifx\\#9\\%
    \BitSet@SetDec#1#2#3#4#5#6#7#8!%
  \else
    \ifcase\BigIntCalcOdd#1#2#4#5#6#7#8#9! %
      0%
    \or
      1%
?   \else\BitSetError:ThisCannotHappen%
    \fi
    \BitSet@AfterFi{%
      \expandafter\expandafter\expandafter\BitSet@SetDecBig
      \BigIntCalcShr#1#2#3#4#5#6#7#8#9!!%
    }%
  \BitSet@Fi
}
%    \end{macrocode}
%    \end{macro}
%    \begin{macro}{\BitSet@SetDec}
%    \begin{macrocode}
\def\BitSet@SetDec#1!{%
  \ifcase#1 %
  \or 1%
  \else
    \ifodd#1 %
      1%
    \else
      0%
    \fi
    \BitSet@AfterFi{%
      \expandafter\expandafter\expandafter\BitSet@SetDec
      \IntCalcShr#1!!%
    }%
  \BitSet@Fi
}
%    \end{macrocode}
%    \end{macro}
%
% \subsection{Export}\label{src:export}
%
% \subsubsection{To binary number}
%
%    \begin{macro}{\bitsetGetBin}
%    \begin{macrocode}
\def\bitsetGetBin#1#2{%
  \romannumeral0%
  \expandafter\expandafter\expandafter\BitSet@@GetBin
  \intcalcNum{#2}!{#1}%
}
%    \end{macrocode}
%    \end{macro}
%    \begin{macro}{\BitSet@@GetBin}
%    \begin{macrocode}
\def\BitSet@@GetBin#1!#2{%
  \BitSet@IfUndefined{#2}{%
    \ifnum#1>1 %
      \BitSet@AfterFi{%
        \expandafter\expandafter\expandafter\BitSet@Fill
        \IntCalcDec#1!!0%
      }%
    \else
      \BitSet@AfterFi{ 0}%
    \BitSet@Fi
  }{%
    \expandafter\expandafter\expandafter\BitSet@NumBinRev
    \expandafter\expandafter\expandafter1%
    \expandafter\expandafter\expandafter!%
    \csname BS@#2\endcsname!!#1!%
  }%
}
%    \end{macrocode}
%    \end{macro}
%    \begin{macro}{\BitSet@Fill}
%    |#1|: number of leading digits |0|\\
%    |#2|: result
%    \begin{macrocode}
\def\BitSet@Fill#1!{%
  \ifnum#1>0 %
    \BitSet@AfterFi{%
      \expandafter\expandafter\expandafter\BitSet@Fill
      \IntCalcDec#1!!0%
    }%
  \else
    \BitSet@AfterFi{ }%
  \BitSet@Fi
}
%    \end{macrocode}
%    \end{macro}
%
%    \begin{macro}{\BitSet@NumBinRev}
%    |#1|: bit counter (including |#2|)\\
%    |#2#3|: reverted number\\
%    |#4|: result\\
%    |#5|: min size
%    \begin{macrocode}
\def\BitSet@NumBinRev#1!#2#3!{%
  \ifx\\#3\\%
    \BitSet@AfterFi{%
      \BitSet@NumBinFill#1!#2%
    }%
  \else
    \BitSet@AfterFi{%
      \expandafter\expandafter\expandafter\BitSet@NumBinRev
      \IntCalcInc#1!!#3!#2%
    }%
  \BitSet@Fi
}
%    \end{macrocode}
%    \end{macro}
%    \begin{macro}{\BitSet@NumBinFill}
%    \begin{macrocode}
\def\BitSet@NumBinFill#1!#2!#3!{%
  \ifnum#3>#1 %
    \BitSet@AfterFi{%
      \expandafter\expandafter\expandafter\BitSet@Fill
      \IntCalcSub#3!#1!!#2%
    }%
  \else
    \BitSet@AfterFi{ #2}%
  \BitSet@Fi
}
%    \end{macrocode}
%    \end{macro}
%
% \subsubsection{To octal/hexadecimal number}
%
%    \begin{macro}{\bitsetGetOct}
%    \begin{macrocode}
\def\bitsetGetOct#1#2{%
  \romannumeral0%
  \bitsetIsEmpty{#1}{%
    \expandafter\expandafter\expandafter\BitSet@@GetOctHex
    \intcalcNum{#2}!3!230%
  }{%
    \expandafter\expandafter\expandafter\BitSet@@GetOct
    \expandafter\expandafter\expandafter1%
    \expandafter\expandafter\expandafter!%
    \expandafter\expandafter\expandafter!%
    \csname BS@#1\endcsname00%
    \BitSet@Empty\BitSet@Empty\BitSet@Empty!{#2}%
  }%
}
%    \end{macrocode}
%    \end{macro}
%    \begin{macro}{\bitsetGetHex}
%    \begin{macrocode}
\def\bitsetGetHex#1#2{%
  \romannumeral0%
  \bitsetIsEmpty{#1}{%
    \expandafter\expandafter\expandafter\BitSet@@GetOctHex
    \intcalcNum{#2}!4!340%
  }{%
    \expandafter\expandafter\expandafter\BitSet@@GetHex
    \expandafter\expandafter\expandafter1%
    \expandafter\expandafter\expandafter!%
    \expandafter\expandafter\expandafter!%
    \csname BS@#1\endcsname000%
    \BitSet@Empty\BitSet@Empty\BitSet@Empty\BitSet@Empty!{#2}%
  }%
}
%    \end{macrocode}
%    \end{macro}
%    \begin{macro}{\BitSet@@GetOct}
%    |#1|: number of digits\\
%    |#2|: result\\
%    |#3#4#5|: bits
%    \begin{macrocode}
\def\BitSet@@GetOct#1!#2!#3#4#5{%
  \ifx#5\BitSet@Empty
    \BitSet@AfterFi{%
      \expandafter\expandafter\expandafter\BitSet@GetOctHex
      \IntCalcDec#1!!#2!23%
    }%
  \else
    \BitSet@AfterFi{%
      \expandafter\expandafter\expandafter\BitSet@@GetOct
      \number\IntCalcInc#1!\expandafter\expandafter\expandafter!%
      \csname BitSet@Oct#5#4#3\endcsname#2!%
    }%
  \BitSet@Fi
}
%    \end{macrocode}
%    \end{macro}
%    \begin{macro}{\BitSet@Oct[000..111]}
%    \begin{macrocode}
\def\BitSet@Temp#1#2#3#4{%
  \expandafter\def\csname BitSet@Oct#1#2#3\endcsname{#4}%
}
\BitSet@Temp0000%
\BitSet@Temp0011%
\BitSet@Temp0102%
\BitSet@Temp0113%
\BitSet@Temp1004%
\BitSet@Temp1015%
\BitSet@Temp1106%
\BitSet@Temp1117%
%    \end{macrocode}
%    \end{macro}
%    \begin{macro}{\BitSet@@GetHex}
%    |#1|: number of digits\\
%    |#2|: result\\
%    |#3#4#5#6|: bits
%    \begin{macrocode}
\def\BitSet@@GetHex#1!#2!#3#4#5#6{%
  \ifx#6\BitSet@Empty
    \BitSet@AfterFi{%
      \expandafter\expandafter\expandafter\BitSet@GetOctHex
      \IntCalcDec#1!!#2!34%
    }%
  \else
    \BitSet@AfterFi{%
      \expandafter\expandafter\expandafter\BitSet@@GetHex
      \number\IntCalcInc#1!\expandafter\expandafter\expandafter!%
      \csname BitSet@Hex#6#5#4#3\endcsname#2!%
    }%
  \BitSet@Fi
}
%    \end{macrocode}
%    \end{macro}
%    \begin{macro}{\BitSet@Hex[0000..1111]}
%    \begin{macrocode}
\def\BitSet@Temp#1#2#3#4#5{%
  \expandafter\def\csname BitSet@Hex#1#2#3#4\endcsname{#5}%
}
\BitSet@Temp00000%
\BitSet@Temp00011%
\BitSet@Temp00102%
\BitSet@Temp00113%
\BitSet@Temp01004%
\BitSet@Temp01015%
\BitSet@Temp01106%
\BitSet@Temp01117%
\BitSet@Temp10008%
\BitSet@Temp10019%
\BitSet@Temp1010A%
\BitSet@Temp1011B%
\BitSet@Temp1100C%
\BitSet@Temp1101D%
\BitSet@Temp1110E%
\BitSet@Temp1111F%
%    \end{macrocode}
%    \end{macro}
%    \begin{macro}{\BitSet@GetOctHex}
%    Leading zeros $ (|#4| - |#1| * 3 + 2)/3 $ if $ |#4| > |#1| * 3$\\
%    |#1|: digit size\\
%    |#2|: result\\
%    |#3|: bits per digit - 1\\
%    |#4|: bits per digit
%    |#5|: garbage\\
%    |#6|: min size
%    \begin{macrocode}
\def\BitSet@GetOctHex#1!#2!#3#4#5!#6{%
  \expandafter\BitSet@@GetOctHex
  \number\intcalcNum{#6}\expandafter\expandafter\expandafter!%
  \IntCalcMul#1!#4!!#3#4#2%
}
%    \end{macrocode}
%    \end{macro}
%    \begin{macro}{\BitSet@@GetOctHex}
%    |#1|: plain min size\\
%    |#2|: digits * (bits per digit)\\
%    |#3|: bits per digit - 1\\
%    |#4|: bits per digit
%    \begin{macrocode}
\def\BitSet@@GetOctHex#1!#2!#3#4{%
  \ifnum#1>#2 %
    \BitSet@AfterFi{%
      \expandafter\expandafter\expandafter\expandafter
      \expandafter\expandafter\expandafter\BitSet@Fill
      \expandafter\IntCalcDiv\number
      \expandafter\expandafter\expandafter\IntCalcAdd
      \IntCalcSub#1!#2!!#3!!#4!!%
    }%
  \else
    \BitSet@AfterFi{ }%
  \BitSet@Fi
}
%    \end{macrocode}
%    \end{macro}
%
% \subsubsection{To decimal number}
%
%    \begin{macro}{\bitsetGetDec}
%    \begin{macrocode}
\def\bitsetGetDec#1{%
  \romannumeral0%
  \BitSet@IfUndefined{#1}{ 0}{%
    \expandafter\expandafter\expandafter\BitSet@GetDec
    \csname BS@#1\endcsname!%
  }%
}
%    \end{macrocode}
%    \end{macro}
%    \begin{macro}{\BitSet@GetDec}
%    \begin{macrocode}
\def\BitSet@GetDec#1#2!{%
  \ifx\\#2\\%
    \BitSet@AfterFi{ #1}%
  \else
    \BitSet@AfterFi{%
      \BitSet@@GetDec2!#1!#2!%
    }%
  \BitSet@Fi
}
%    \end{macrocode}
%    \end{macro}
%    \begin{macro}{\BitSet@@GetDec}
%    |#1|: power of two\\
%    |#2|: result\\
%    |#3#4|: number
%    \begin{macrocode}
\def\BitSet@@GetDec#1!#2!#3#4!{%
  \ifx\\#4\\%
    \ifx#31%
      \BitSet@AfterFiFi{%
        \expandafter\expandafter\expandafter\BitSet@Space
        \IntCalcAdd#1!#2!%
      }%
    \else
      \BitSet@AfterFiFi{ #2}%
    \fi
  \else
    \ifx#31%
      \BitSet@AfterFiFi{%
        \csname BitSet@N#1%
        \expandafter\expandafter\expandafter\endcsname
        \IntCalcAdd#1!#2!!#4!%
      }%
    \else
      \BitSet@AfterFiFi{%
        \csname BitSet@N#1\endcsname#2!#4!%
      }%
    \fi
  \BitSet@Fi
}
%    \end{macrocode}
%    \end{macro}
%    \begin{macro}{\BitSet@N[1,2,4,...]}
%    \begin{macrocode}
\def\BitSet@Temp#1#2{%
  \expandafter\def\csname BitSet@N#1\endcsname{%
    \BitSet@@GetDec#2!%
  }%
}
\BitSet@Temp{1}{2}
\BitSet@Temp{2}{4}
\BitSet@Temp{4}{8}
\BitSet@Temp{8}{16}
\BitSet@Temp{16}{32}
\BitSet@Temp{32}{64}
\BitSet@Temp{64}{128}
\BitSet@Temp{128}{256}
\BitSet@Temp{256}{512}
\BitSet@Temp{512}{1024}
\BitSet@Temp{1024}{2048}
\BitSet@Temp{2048}{4096}
\BitSet@Temp{4096}{8192}
\BitSet@Temp{8192}{16384}
\BitSet@Temp{16384}{32768}
\BitSet@Temp{32768}{65536}
\BitSet@Temp{65536}{131072}
\BitSet@Temp{131072}{262144}
\BitSet@Temp{262144}{524288}
\BitSet@Temp{524288}{1048576}
\BitSet@Temp{1048576}{2097152}
\BitSet@Temp{2097152}{4194304}
\BitSet@Temp{4194304}{8388608}
\BitSet@Temp{8388608}{16777216}
\BitSet@Temp{16777216}{33554432}
\BitSet@Temp{33554432}{67108864}
\BitSet@Temp{67108864}{134217728}
\BitSet@Temp{134217728}{268435456}
\BitSet@Temp{268435456}{536870912}
\BitSet@Temp{536870912}{1073741824}
%    \end{macrocode}
%    \end{macro}
%    \begin{macro}{\BitSet@N1073741824}
%    \begin{macrocode}
\expandafter\def\csname BitSet@N1073741824\endcsname{%
  \BitSet@GetDecBig2147483648!%
}%
%    \end{macrocode}
%    \end{macro}
%
%    \begin{macro}{\BitSet@GetDecBig}
%    |#1|: current power of two\\
%    |#2|: result\\
%    |#3#4|: number
%    \begin{macrocode}
\def\BitSet@GetDecBig#1!#2!#3#4!{%
  \ifx\\#4\\%
    \ifx#31%
      \BitSet@AfterFiFi{%
        \expandafter\expandafter\expandafter\BitSet@Space
        \BigIntCalcAdd#1!#2!%
      }%
    \else
      \BitSet@AfterFiFi{ #2}%
    \fi
  \else
    \ifx#31%
      \BitSet@AfterFiFi{%
        \expandafter\expandafter\expandafter\BitSet@@GetDecBig
        \BigIntCalcAdd#1!#2!!#1!#4!%
      }%
    \else
      \BitSet@AfterFiFi{%
        \expandafter\expandafter\expandafter\BitSet@GetDecBig
        \BigIntCalcShl#1!!#2!#4!%
      }%
    \fi
  \BitSet@Fi
}
%    \end{macrocode}
%    \end{macro}
%    \begin{macro}{\BitSet@@GetDecBig}
%    |#1|: result\\
%    |#2|: power of two\\
%    |#3#4|: number
%    \begin{macrocode}
\def\BitSet@@GetDecBig#1!#2!{%
  \expandafter\expandafter\expandafter\BitSet@GetDecBig
  \BigIntCalcShl#2!!#1!%
}
%    \end{macrocode}
%    \end{macro}
%
% \subsection{Logical operators}\label{src:log}
%
% \subsubsection{\cs{bitsetAnd}}
%
%    \begin{macro}{\bitsetAnd}
%    Decision table for \cs{bitsetAnd}:
%    \begin{quote}
%    \begin{tabular}{l|l|l|l|}
%                         & undef(B)   & empty(B)   & cardinality(B)$>$0\\
%      \hline
%      undef(A)           & A := empty & A := empty & A := empty\\
%      \hline
%      empty(A)           &            &            &\\
%      \hline
%      cardinality(A)$>$0 & A := empty & A := empty & A \&= B\\
%      \hline
%    \end{tabular}
%    \end{quote}
%    \begin{macrocode}
\def\bitsetAnd#1#2{%
  \bitsetIsEmpty{#1}{%
    \bitsetReset{#1}%
  }{%
    \bitsetIsEmpty{#2}{%
      \bitsetReset{#1}%
    }{%
      \expandafter\edef\csname BS@#1\endcsname{%
        \expandafter\expandafter\expandafter\BitSet@And
        \csname BS@#1\expandafter\expandafter\expandafter\endcsname
        \expandafter\expandafter\expandafter!%
        \csname BS@#2\endcsname!!%
      }%
      \expandafter\ifx\csname BS@#1\endcsname\BitSet@Empty
        \bitsetReset{#1}%
      \fi
    }%
  }%
}
%    \end{macrocode}
%    \end{macro}
%    \begin{macro}{\BitSet@And}
%    \begin{macrocode}
\def\BitSet@And#1#2!#3#4!#5!{%
  \ifx\\#2\\%
    \ifnum#1#3=11 #51\fi
  \else
    \ifx\\#4\\%
      \ifnum#1#3=11 #51\fi
    \else
      \ifnum#1#3=11 %
        #51%
        \BitSet@AfterFiFiFi{%
          \BitSet@And#2!#4!!%
        }%
      \else
        \BitSet@AfterFiFiFi{%
          \BitSet@And#2!#4!#50!%
        }%
      \fi
    \fi
  \BitSet@Fi
}
%    \end{macrocode}
%    \end{macro}
%
% \subsubsection{\cs{bitsetAndNot}}
%
%    \begin{macro}{\bitsetAndNot}
%    Decision table for \cs{bitsetAndNot}:
%    \begin{quote}
%    \begin{tabular}{l|l|l|l|}
%                         & undef(B)   & empty(B)   & cardinality(B)$>$0\\
%      \hline
%      undef(A)           & A := empty & A := empty & A := empty\\
%      \hline
%      empty(A)           &            &            &\\
%      \hline
%      cardinality(A)$>$0 &            &            & A \&= !B\\
%      \hline
%    \end{tabular}
%    \end{quote}
%    \begin{macrocode}
\def\bitsetAndNot#1#2{%
  \bitsetIsEmpty{#1}{%
    \bitsetReset{#1}%
  }{%
    \bitsetIsEmpty{#2}{%
    }{%
      \expandafter\edef\csname BS@#1\endcsname{%
        \expandafter\expandafter\expandafter\BitSet@AndNot
        \csname BS@#1\expandafter\expandafter\expandafter\endcsname
        \expandafter\expandafter\expandafter!%
        \csname BS@#2\endcsname!!%
      }%
      \expandafter\ifx\csname BS@#1\endcsname\BitSet@Empty
        \bitsetReset{#1}%
      \fi
    }%
  }%
}
%    \end{macrocode}
%    \end{macro}
%    \begin{macro}{\BitSet@AndNot}
%    \begin{macrocode}
\def\BitSet@AndNot#1#2!#3#4!#5!{%
  \ifx\\#2\\%
    \ifnum#1#3=10 #51\fi
  \else
    \ifx\\#4\\%
      #5%
      \ifnum#1#3=10 1\else 0\fi
      #2%
    \else
      \ifnum#1#3=10 %
        #51%
        \BitSet@AfterFiFiFi{%
          \BitSet@AndNot#2!#4!!%
        }%
      \else
        \BitSet@AfterFiFiFi{%
          \BitSet@AndNot#2!#4!#50!%
        }%
      \fi
    \fi
  \BitSet@Fi
}
%    \end{macrocode}
%    \end{macro}
%
% \subsubsection{\cs{bitsetOr}}
%
%    \begin{macro}{\bitsetOr}
%    Decision table for \cs{bitsetOr}:
%    \begin{quote}
%    \begin{tabular}{l|l|l|l|}
%                         & undef(B)   & empty(B)   & cardinality(B)$>$0\\
%      \hline
%      undef(A)           & A := empty & A := empty & A := B\\
%      \hline
%      empty(A)           &            &            & A := B\\
%      \hline
%      cardinality(A)$>$0 &            &            & A \string|= B\\
%      \hline
%    \end{tabular}
%    \end{quote}
%    \begin{macrocode}
\def\bitsetOr#1#2{%
  \bitsetIsEmpty{#2}{%
    \BitSet@IfUndefined{#1}{\bitsetReset{#1}}{}%
  }{%
    \bitsetIsEmpty{#1}{%
      \expandafter\let\csname BS@#1\expandafter\endcsname
                      \csname BS@#2\endcsname
    }{%
      \expandafter\edef\csname BS@#1\endcsname{%
        \expandafter\expandafter\expandafter\BitSet@Or
        \csname BS@#1\expandafter\expandafter\expandafter\endcsname
        \expandafter\expandafter\expandafter!%
        \csname BS@#2\endcsname!%
      }%
    }%
  }%
}
%    \end{macrocode}
%    \end{macro}
%    \begin{macro}{\BitSet@Or}
%    \begin{macrocode}
\def\BitSet@Or#1#2!#3#4!{%
  \ifnum#1#3>0 1\else 0\fi
  \ifx\\#2\\%
    #4%
  \else
    \ifx\\#4\\%
      #2%
    \else
      \BitSet@AfterFiFi{%
        \BitSet@Or#2!#4!%
      }%
    \fi
  \BitSet@Fi
}
%    \end{macrocode}
%    \end{macro}
%
% \subsubsection{\cs{bitsetXor}}
%
%    \begin{macro}{\bitsetXor}
%    Decision table for \cs{bitsetXor}:
%    \begin{quote}
%    \begin{tabular}{l|l|l|l|}
%                         & undef(B)   & empty(B)   & cardinality(B)$>$0\\
%      \hline
%      undef(A)           & A := empty & A := empty & A := B\\
%      \hline
%      empty(A)           &            &            & A := B\\
%      \hline
%      cardinality(A)$>$0 &            &            & A \string^= B\\
%      \hline
%    \end{tabular}
%    \end{quote}
%    \begin{macrocode}
\def\bitsetXor#1#2{%
  \bitsetIsEmpty{#2}{%
    \BitSet@IfUndefined{#1}{\bitsetReset{#1}}{}%
  }{%
    \bitsetIsEmpty{#1}{%
      \expandafter\let\csname BS@#1\expandafter\endcsname
                      \csname BS@#2\endcsname
    }{%
      \expandafter\edef\csname BS@#1\endcsname{%
        \expandafter\expandafter\expandafter\BitSet@Xor
        \csname BS@#1\expandafter\expandafter\expandafter\endcsname
        \expandafter\expandafter\expandafter!%
        \csname BS@#2\endcsname!!%
      }%
      \expandafter\ifx\csname BS@#1\endcsname\BitSet@Empty
        \bitsetReset{#1}%
      \fi
    }%
  }%
}
%    \end{macrocode}
%    \end{macro}
%    \begin{macro}{\BitSet@Xor}
%    \begin{macrocode}
\def\BitSet@Xor#1#2!#3#4!#5!{%
  \ifx\\#2\\%
    \ifx#1#3%
      \ifx\\#4\\%
      \else
        #50#4%
      \fi
    \else
      #51#4%
    \fi
  \else
    \ifx\\#4\\%
      #5%
      \ifx#1#30\else 1\fi
      #2%
    \else
      \ifx#1#3%
        \BitSet@AfterFiFiFi{%
          \BitSet@Xor#2!#4!#50!%
        }%
      \else
        #51%
        \BitSet@AfterFiFiFi{%
          \BitSet@Xor#2!#4!!%
        }%
      \fi
    \fi
  \BitSet@Fi
}
%    \end{macrocode}
%    \end{macro}
%
% \subsubsection{Shifting}\label{src:shift}
%
% \subsubsection{\cs{bitsetShiftLeft}}
%
%    \begin{macro}{\bitsetShiftLeft}
%    \begin{macrocode}
\def\bitsetShiftLeft#1#2{%
  \BitSet@IfUndefined{#1}{%
    \bitsetReset{#1}%
  }{%
    \bitsetIsEmpty{#1}{%
    }{%
      \expandafter\expandafter\expandafter\BitSet@ShiftLeft
      \intcalcNum{#2}!{#1}%
    }%
  }%
}
%    \end{macrocode}
%    \end{macro}
%    \begin{macro}{\BitSet@ShiftLeft}
%    \begin{macrocode}
\def\BitSet@ShiftLeft#1!#2{%
  \ifcase\intcalcSgn{#1} %
  \or
    \begingroup
      \uccode`m=`0 %
    \uppercase\expandafter{\expandafter\endgroup
      \expandafter\edef\csname BS@#2\expandafter\endcsname
      \expandafter{%
        \romannumeral#1000\expandafter\BitSet@Space
        \csname BS@#2\endcsname
      }%
    }%
  \else
    \expandafter\BitSet@ShiftRight\BitSet@Gobble#1!{#2}%
  \fi
}
%    \end{macrocode}
%    \end{macro}
%
% \subsubsection{\cs{bitsetShiftRight}}
%
%    \begin{macro}{\bitsetShiftRight}
%    \begin{macrocode}
\def\bitsetShiftRight#1#2{%
  \BitSet@IfUndefined{#1}{%
    \bitsetReset{#1}%
  }{%
    \bitsetIsEmpty{#1}{%
    }{%
      \expandafter\expandafter\expandafter\BitSet@ShiftRight
      \intcalcNum{#2}!{#1}%
    }%
  }%
}
%    \end{macrocode}
%    \end{macro}
%    \begin{macro}{\BitSet@ShiftRight}
%    \begin{macrocode}
\def\BitSet@ShiftRight#1!#2{%
  \ifcase\intcalcSgn{#1} %
  \or
    \expandafter\edef\csname BS@#2\endcsname{%
      \expandafter\expandafter\expandafter\BitSet@Kill
      \csname BS@#2\expandafter\endcsname\expandafter\BitSet@Empty
      \expandafter=%
      \expandafter{\expandafter}\expandafter{\expandafter}%
      \romannumeral#1000!%
    }%
  \else
    \expandafter\BitSet@ShiftLeft\BitSet@Gobble#1!{#2}%
  \fi
}
%    \end{macrocode}
%    \end{macro}
%    \begin{macro}{\BitSet@Kill}
%    \begin{macrocode}
\def\BitSet@Kill#1#2=#3#4#5{%
  #3#4%
  \ifx#5!%
    \ifx#1\BitSet@Empty
      0%
    \else
      #1#2%
    \fi
  \else
    \ifx#1\BitSet@Empty
      0%
      \BitSet@AfterFiFi\BitSet@Cleanup
    \else
      \BitSet@Kill#2=%
    \fi
  \BitSet@Fi
}
%    \end{macrocode}
%    \end{macro}
%
% \subsection{Bit manipulation}\label{src:bitman}
%
%    \begin{macro}{\bitsetClear}
%    \begin{macrocode}
\def\bitsetClear{%
  \BitSet@CheckIndex\BitSet@Clear
}
%    \end{macrocode}
%    \end{macro}
%    \begin{macro}{\bitsetSet}
%    \begin{macrocode}
\def\bitsetSet{%
  \BitSet@CheckIndex\BitSet@Set
}
%    \end{macrocode}
%    \end{macro}
%    \begin{macro}{\bitsetFlip}
%    \begin{macrocode}
\def\bitsetFlip{%
  \BitSet@CheckIndex\BitSet@Flip
}
%    \end{macrocode}
%    \end{macro}
%    \begin{macro}{\bitsetSetValue}
%    \begin{macrocode}
\def\bitsetSetValue#1#2#3{%
  \expandafter\expandafter\expandafter\BitSet@SetValue
  \intcalcNum{#3}!{#1}{#2}%
}
%    \end{macrocode}
%    \end{macro}
%    \begin{macro}{\BitSet@SetValue}
%    |#1|: plain value\\
%    |#2|: BitSet\\
%    |#3|: Index
%    \begin{macrocode}
\def\BitSet@SetValue#1!{%
  \BitSet@CheckIndex{%
    \ifcase#1 %
      \expandafter\BitSet@Clear
    \or
      \expandafter\BitSet@Set
    \else
      \BitSet@ErrorInvalidBitValue{#1}%
      \expandafter\expandafter\expandafter\BitSet@Gobble
      \expandafter\BitSet@Gobble
    \fi
  }%
}
%    \end{macrocode}
%    \end{macro}
%    \begin{macro}{\BitSet@ErrorInvalidBitValue}
%    |#1|: Wrong bit value
%    \begin{macrocode}
\def\BitSet@ErrorInvalidBitValue#1{%
  \@PackageError{bitset}{%
    Invalid bit value (#1) not in range 0..1%
  }\@ehc
}
%    \end{macrocode}
%    \end{macro}
%
% \subsubsection{Clear operation}
%
%    \begin{macro}{\BitSet@Clear}
%    |#1|: BitSet\\
%    |#2|: plain and checked index
%    \begin{macrocode}
\def\BitSet@Clear#1#2{%
  \edef\BitSet@Temp{%
    \expandafter\expandafter\expandafter\BitSet@@Clear
    \csname BS@#1\expandafter\endcsname
    \expandafter\BitSet@Empty\expandafter=\expandafter!%
    \romannumeral#2000!%
  }%
  \expandafter\let\csname BS@#1\expandafter\endcsname
  \ifx\BitSet@Temp\BitSet@Empty
    \BitSet@Zero
  \else
    \BitSet@Temp
  \fi
}
%    \end{macrocode}
%    \end{macro}
%    \begin{macro}{\BitSet@@Clear}
%    \begin{macrocode}
\def\BitSet@@Clear#1#2=#3!#4{%
  \ifx#4!%
    \ifx#1\BitSet@Empty
    \else
      \ifx\BitSet@Empty#2%
      \else
        #30#2%
      \fi
    \fi
  \else
    \ifx#1\BitSet@Empty
      \BitSet@AfterFiFi\BitSet@Cleanup
    \else
      \ifx#10%
        \BitSet@AfterFiFiFi{%
          \BitSet@@Clear#2=#30!%
        }%
      \else
        #31%
        \BitSet@AfterFiFiFi{%
          \BitSet@@Clear#2=!%
        }%
      \fi
    \fi
  \BitSet@Fi
}
%    \end{macrocode}
%    \end{macro}
%
% \subsubsection{Set operation}
%
%    \begin{macro}{\BitSet@Set}
%    |#1|: BitSet\\
%    |#2|: plain and checked Index
%    \begin{macrocode}
\def\BitSet@Set#1#2{%
  \expandafter\edef\csname BS@#1\endcsname{%
    \expandafter\expandafter\expandafter\BitSet@@Set
    \csname BS@#1\expandafter\endcsname
    \expandafter\BitSet@Empty\expandafter=%
    \expandafter{\expandafter}\expandafter{\expandafter}%
    \romannumeral#2000!%
  }%
}
%    \end{macrocode}
%    \end{macro}
%    \begin{macro}{\BitSet@@Set}
%    \begin{macrocode}
\def\BitSet@@Set#1#2=#3#4#5{%
  #3#4%
  \ifx#5!%
    1#2%
  \else
    \ifx#1\BitSet@Empty
      0%
      \BitSet@AfterFiFi\BitSet@@@Set
    \else
      #1%
      \BitSet@@Set#2=%
    \fi
  \BitSet@Fi
}
%    \end{macrocode}
%    \end{macro}
%    \begin{macro}{\BitSet@@@Set}
%    \begin{macrocode}
\def\BitSet@@@Set#1{%
  \ifx#1!%
    1%
  \else
    0%
    \expandafter\BitSet@@@Set
  \fi
}
%    \end{macrocode}
%    \end{macro}
%
% \subsubsection{Flip operation}
%
%    \begin{macro}{\BitSet@Flip}
%    |#1|: BitSet\\
%    |#2|: plain and checked Index
%    \begin{macrocode}
\def\BitSet@Flip#1#2{%
  \edef\BitSet@Temp{%
    \expandafter\expandafter\expandafter\BitSet@@Flip
    \csname BS@#1\expandafter\endcsname
    \expandafter\BitSet@Empty\expandafter=\expandafter!%
    \romannumeral#2000!%
  }%
  \expandafter\let\csname BS@#1\expandafter\endcsname
  \ifx\BitSet@Temp\BitSet@Empty
    \BitSet@Zero
  \else
    \BitSet@Temp
  \fi
}
%    \end{macrocode}
%    \end{macro}
%    \begin{macro}{\BitSet@@Flip}
%    \begin{macrocode}
\def\BitSet@@Flip#1#2=#3!#4{%
  \ifx#4!%
    \ifx#11%
      \ifx\BitSet@Empty#2%
      \else
        #30#2%
      \fi
    \else
      #31#2%
    \fi
  \else
    \ifx#1\BitSet@Empty
      #30%
      \BitSet@AfterFiFi\BitSet@@@Set
    \else
      \ifx#10%
        \BitSet@AfterFiFiFi{%
          \BitSet@@Flip#2=#30!%
        }%
      \else
        #31%
        \BitSet@AfterFiFiFi{%
          \BitSet@@Flip#2=!%
        }%
      \fi
    \fi
  \BitSet@Fi
}
%    \end{macrocode}
%    \end{macro}
%
% \subsubsection{Range operators}
%
%    \begin{macro}{\bitsetClearRange}
%    \begin{macrocode}
\def\bitsetClearRange{%
  \BitSet@Range\BitSet@Clear
}
%    \end{macrocode}
%    \end{macro}
%    \begin{macro}{\bitsetSetRange}
%    \begin{macrocode}
\def\bitsetSetRange{%
  \BitSet@Range\BitSet@Set
}
%    \end{macrocode}
%    \end{macro}
%    \begin{macro}{\bitsetFlipRange}
%    \begin{macrocode}
\def\bitsetFlipRange{%
  \BitSet@Range\BitSet@Flip
}
%    \end{macrocode}
%    \end{macro}
%    \begin{macro}{\bitsetSetValueRange}
%    \begin{macrocode}
\def\bitsetSetValueRange#1#2#3#4{%
  \expandafter\expandafter\expandafter\BitSet@SetValueRange
  \intcalcNum{#4}!{#1}{#2}{#3}%
}
%    \end{macrocode}
%    \end{macro}
%    \begin{macro}{\BitSet@SetValueRange}
%    \begin{macrocode}
\def\BitSet@SetValueRange#1!#2#3#4{%
  \ifcase#1 %
    \BitSet@Range\BitSet@Clear{#2}{#3}{#4}%
  \or
    \BitSet@Range\BitSet@Set{#2}{#3}{#4}%
  \else
    \BitSet@ErrorInvalidBitValue{#1}%
  \fi
}
%    \end{macrocode}
%    \end{macro}
%
%    \begin{macro}{\BitSet@Range}
%    |#1|: clear/set/flip macro\\
%    |#2|: BitSet\\
%    |#3|: Index from\\
%    |#4|: Index to
%    \begin{macrocode}
\def\BitSet@Range#1#2#3#4{%
  \edef\BitSet@Temp{%
    \noexpand\BitSet@@Range\noexpand#1{#2}%
    \intcalcNum{#3}!\intcalcNum{#4}!%
  }%
  \BitSet@Temp
}
%    \end{macrocode}
%    \end{macro}
%    \begin{macro}{\BitSet@@Range}
%    |#1|: clear/set/flip macro\\
%    |#2|: BitSet\\
%    |#3|: Index from\\
%    |#4|: Index to
%    \begin{macrocode}
\def\BitSet@@Range#1#2#3!#4!{%
  \ifnum#3<0 %
    \BitSet@NegativeIndex#1{#2}#3!#4!0!#4!%
  \else
    \ifnum#4<0 %
      \BitSet@NegativeIndex#1{#2}#3!#4!#3!0!%
    \else
      \ifcase\intcalcCmp{#3}{#4} %
      \or
        \@PackageError{bitset}{%
          Wrong index numbers in range [#3..#4]\MessageBreak% hash-ok
          for clear/set/flip on bit set `#2'.\MessageBreak
          The lower index exceeds the upper index.\MessageBreak
          Canceling the operation as error recovery%
        }\@ehc
      \else
        \BitSet@@@Range#3!#4!#1{#2}%
      \fi
    \fi
  \fi
}
%    \end{macrocode}
%    \end{macro}
%    \begin{macro}{\BitSet@NegativeIndex}
%    \begin{macrocode}
\def\BitSet@NegativeIndex#1#2#3!#4!#5!#6!{%
  \@PackageError{bitset}{%
    Negative index in range [#3..#4]\MessageBreak % hash-ok
    for \string\bitset
    \ifx#1\BitSet@Clear
      Clear%
    \else
      \ifx#1\BitSet@Set
        Set%
      \else
        Flip%
      \fi
    \fi
    Range on bit set `#2'.\MessageBreak
    Using [#5..#6] as error recovery% hash-ok
  }\@ehc
  \BitSet@@Range#1{#2}#5!#6!%
}
%    \end{macrocode}
%    \end{macro}
%    \begin{macro}{\BitSet@@Range}
%    \begin{macrocode}
\def\BitSet@@@Range#1!#2!#3#4{%
  \ifnum#1<#2 %
    #3{#4}{#1}%
    \BitSet@AfterFi{%
      \expandafter\expandafter\expandafter\BitSet@@@Range
      \IntCalcInc#1!!#2!#3{#4}%
    }%
  \BitSet@Fi
}
%    \end{macrocode}
%    \end{macro}
%
% \subsection{Bit retrieval}\label{src:bitret}
%
% \subsubsection{\cs{bitsetGet}}
%
%    \begin{macro}{\bitsetGet}
%    \begin{macrocode}
\def\bitsetGet#1#2{%
  \number
  \expandafter\expandafter\expandafter\BitSet@Get
  \intcalcNum{#2}!{#1}%
}
%    \end{macrocode}
%    \end{macro}
%    \begin{macro}{\BitSet@Get}
%    |#1|: plain index\\
%    |#2|: BitSet
%    \begin{macrocode}
\def\BitSet@Get#1!#2{%
  \ifnum#1<0 %
    \BitSet@AfterFi{%
      0 \BitSetError:NegativeIndex%
    }%
  \else
    \BitSet@IfUndefined{#2}{0}{%
      \expandafter\expandafter\expandafter\BitSet@@Get
      \csname BS@#2\expandafter\endcsname
      \expandafter!\expandafter=%
      \expandafter{\expandafter}\expandafter{\expandafter}%
      \romannumeral\intcalcNum{#1}000!%
    }%
    \expandafter\BitSet@Space
  \BitSet@Fi
}
%    \end{macrocode}
%    \end{macro}
%    \begin{macro}{\BitSet@@Get}
%    \begin{macrocode}
\def\BitSet@@Get#1#2=#3#4#5{%
  #3#4%
  \ifx#5!%
    \ifx#1!%
      0%
    \else
      #1%
    \fi
  \else
    \ifx#1!%
      0%
      \BitSet@AfterFiFi\BitSet@Cleanup
    \else
      \BitSet@@Get#2=%
    \fi
  \BitSet@Fi
}
%    \end{macrocode}
%    \end{macro}
%
% \subsubsection{\cs{bitsetNextClearBit}, \cs{bitsetNextSetBit}}
%
%    \begin{macro}{\bitsetNextClearBit}
%    \begin{macrocode}
\def\bitsetNextClearBit#1#2{%
  \number
  \expandafter\expandafter\expandafter\BitSet@NextClearBit
  \intcalcNum{#2}!{#1} %
}
%    \end{macrocode}
%    \end{macro}
%    \begin{macro}{\BitSet@NextClearBit}
%    |#1|: Index\\
%    |#2|: BitSet
%    \begin{macrocode}
\def\BitSet@NextClearBit#1!#2{%
  \ifnum#1<0 %
    \BitSet@NextClearBit0!{#2}%
    \BitSet@AfterFi{%
      \expandafter\BitSet@Space
      \expandafter\BitSetError:NegativeIndex\romannumeral0%
    }%
  \else
    \bitsetIsEmpty{#2}{#1}{%
      \expandafter\BitSet@Skip
      \number#1\expandafter\expandafter\expandafter!%
      \csname BS@#2\endcsname!!!!!!!!!=%
      {\BitSet@@NextClearBit#1!}%
    }%
  \BitSet@Fi
}
%    \end{macrocode}
%    \end{macro}
%    \begin{macro}{\BitSet@@NextClearBit}
%    |#1|: index for next bit in |#2|\\
%    |#2|: next bit
%    \begin{macrocode}
\def\BitSet@@NextClearBit#1!#2{%
  \ifx#2!%
    #1%
  \else
    \ifx#20%
      #1%
      \BitSet@AfterFiFi\BitSet@Cleanup
    \else
      \BitSet@AfterFiFi{%
        \expandafter\expandafter\expandafter\BitSet@@NextClearBit
        \IntCalcInc#1!!%
      }%
    \fi
  \BitSet@Fi
}
%    \end{macrocode}
%    \end{macro}
%
%    \begin{macro}{\bitsetNextSetBit}
%    \begin{macrocode}
\def\bitsetNextSetBit#1#2{%
  \number
  \expandafter\expandafter\expandafter\BitSet@NextSetBit
  \intcalcNum{#2}!{#1} %
}
%    \end{macrocode}
%    \end{macro}
%    \begin{macro}{\BitSet@NextSetBit}
%    |#1|: Index\\
%    |#2|: BitSet
%    \begin{macrocode}
\def\BitSet@NextSetBit#1!#2{%
  \ifnum#1<0 %
    \BitSet@NextSetBit0!{#2}%
    \BitSet@AfterFi{%
      \expandafter\BitSet@Space
      \expandafter\BitSetError:NegativeIndex\romannumeral0%
    }%
  \else
    \bitsetIsEmpty{#2}{-1}{%
      \expandafter\BitSet@Skip
      \number#1\expandafter\expandafter\expandafter!%
      \csname BS@#2\endcsname!!!!!!!!!=%
      {\BitSet@@NextSetBit#1!}%
    }%
  \BitSet@Fi
}
%    \end{macrocode}
%    \end{macro}
%    \begin{macro}{\BitSet@@NextSetBit}
%    |#1|: index for next bit in |#2|\\
%    |#2|: next bit
%    \begin{macrocode}
\def\BitSet@@NextSetBit#1!#2{%
  \ifx#2!%
    -1%
  \else
    \ifx#21%
      #1%
      \BitSet@AfterFiFi\BitSet@Cleanup
    \else
      \BitSet@AfterFiFi{%
        \expandafter\expandafter\expandafter\BitSet@@NextSetBit
        \IntCalcInc#1!!%
      }%
    \fi
  \BitSet@Fi
}
%    \end{macrocode}
%    \end{macro}
%    \begin{macro}{\BitSet@Cleanup}
%    \begin{macrocode}
\def\BitSet@Cleanup#1!{}
%    \end{macrocode}
%    \end{macro}
%
%    \begin{macro}{\BitSet@Skip}
%    |#1|: number of bits to skip\\
%    |#2|: bits\\
%    |#3|: continuation code
%    \begin{macrocode}
\def\BitSet@Skip#1!#2{%
  \ifx#2!%
    \BitSet@AfterFi{%
      \BitSet@SkipContinue%
    }%
  \else
    \ifcase#1 %
      \BitSet@AfterFiFi{%
        \BitSet@SkipContinue#2%
      }%
    \or
      \BitSet@AfterFiFi\BitSet@SkipContinue
    \or
      \BitSet@AfterFiFi{%
        \expandafter\BitSet@SkipContinue\BitSet@Gobble
      }%
    \else
      \ifnum#1>8 %
        \BitSet@AfterFiFiFi{%
          \expandafter\BitSet@Skip
          \number\IntCalcSub#1!8!\expandafter!%
          \BitSet@GobbleSeven
        }%
      \else
        \BitSet@AfterFiFiFi{%
          \expandafter\expandafter\expandafter\BitSet@Skip
          \IntCalcDec#1!!%
        }%
      \fi
    \fi
  \BitSet@Fi
}
%    \end{macrocode}
%    \end{macro}
%    \begin{macro}{\BitSet@SkipContinue}
%    |#1|: remaining bits\\
%    |#2|: continuation code
%    \begin{macrocode}
\def\BitSet@SkipContinue#1!#2=#3{%
  #3#1!%
}
%    \end{macrocode}
%    \end{macro}
%    \begin{macro}{\BitSet@GobbleSeven}
%    \begin{macrocode}
\def\BitSet@GobbleSeven#1#2#3#4#5#6#7{}
%    \end{macrocode}
%    \end{macro}
%
% \subsubsection{\cs{bitsetGetSetBitList}}
%
%    \begin{macro}{\bitsetGetSetBitList}
%    It's just a wrapper for \cs{bitsetNextSetBit}.
%    \begin{macrocode}
\def\bitsetGetSetBitList#1{%
  \romannumeral0%
  \bitsetIsEmpty{#1}{ }{%
    \expandafter\BitSet@GetSetBitList
    \number\BitSet@NextSetBit0!{#1}!{#1}{}!%
  }%
}
%    \end{macrocode}
%    \end{macro}
%    \begin{macro}{\BitSet@GetSetBitList}
%    |#1|: found index\\
%    |#2|: BitSet\\
%    |#3|: comma
%    |#4|: result
%    \begin{macrocode}
\def\BitSet@GetSetBitList#1!#2#3#4!{%
  \ifnum#1<0 %
    \BitSet@AfterFi{ #4}%
  \else
    \BitSet@AfterFi{%
      \expandafter\BitSet@GetSetBitList\number
      \expandafter\expandafter\expandafter\BitSet@NextSetBit
      \IntCalcInc#1!!{#2}!{#2},#4#3#1!%
    }%
  \BitSet@Fi
}
%    \end{macrocode}
%    \end{macro}
%
% \subsection{Bit set properties}\label{src:prop}
%
%    \begin{macro}{\bitsetSize}
%    \begin{macrocode}
\def\bitsetSize#1{%
  \number
  \BitSet@IfUndefined{#1}{0 }{%
    \expandafter\expandafter\expandafter\BitSet@Size
    \expandafter\expandafter\expandafter1%
    \expandafter\expandafter\expandafter!%
    \csname BS@#1\endcsname!0!%
  }%
}
%    \end{macrocode}
%    \end{macro}
%    \begin{macro}{\BitSet@Size}
%    |#1|: counter\\
%    |#2#3|: bits\\
%    |#4|: result
%    \begin{macrocode}
\def\BitSet@Size#1!#2#3!#4!{%
  \ifx#21%
    \ifx\\#3\\%
      \BitSet@AfterFiFi{#1 }%
    \else
      \BitSet@AfterFiFi{%
        \expandafter\expandafter\expandafter\BitSet@Size
        \IntCalcInc#1!!#3!#1!%
      }%
    \fi
  \else
    \ifx\\#3\\%
      \BitSet@AfterFiFi{#4 }%
    \else
      \BitSet@AfterFiFi{%
        \expandafter\expandafter\expandafter\BitSet@Size
        \IntCalcInc#1!!#3!#4!%
      }%
    \fi
  \fi
  \BitSet@Fi
}
%    \end{macrocode}
%    \end{macro}
%
%    \begin{macro}{\bitsetCardinality}
%    \begin{macrocode}
\def\bitsetCardinality#1{%
  \number
  \BitSet@IfUndefined{#1}{0 }{%
    \expandafter\expandafter\expandafter\BitSet@Cardinality
    \expandafter\expandafter\expandafter0%
    \expandafter\expandafter\expandafter!%
    \csname BS@#1\endcsname!%
  }%
}
%    \end{macrocode}
%    \end{macro}
%    \begin{macro}{\BitSet@Cardinality}
%    |#1|: result\\
%    |#2#3|: bits
%    \begin{macrocode}
\def\BitSet@Cardinality#1!#2#3!{%
  \ifx#21%
    \ifx\\#3\\%
      \BitSet@AfterFiFi{\IntCalcInc#1! }%
    \else
      \BitSet@AfterFiFi{%
        \expandafter\expandafter\expandafter\BitSet@Cardinality
        \IntCalcInc#1!!#3!%
      }%
    \fi
  \else
    \ifx\\#3\\%
      \BitSet@AfterFiFi{#1 }%
    \else
      \BitSet@AfterFiFi{%
        \BitSet@Cardinality#1!#3!%
      }%
    \fi
  \fi
  \BitSet@Fi
}
%    \end{macrocode}
%    \end{macro}
%
% \subsection{Queries}\label{src:query}
%
%    \begin{macro}{\bitsetIsDefined}
%    \begin{macrocode}
\def\bitsetIsDefined#1{%
  \BitSet@IfUndefined{#1}%
  \BitSet@SecondOfTwo
  \BitSet@FirstOfTwo
}
%    \end{macrocode}
%    \end{macro}
%    \begin{macro}{\bitsetIsEmpty}
%    \begin{macrocode}
\def\bitsetIsEmpty#1{%
  \BitSet@IfUndefined{#1}\BitSet@FirstOfTwo{%
    \expandafter\ifx\csname BS@#1\endcsname\BitSet@Zero
      \expandafter\BitSet@FirstOfTwo
    \else
      \expandafter\BitSet@SecondOfTwo
    \fi
  }%
}
%    \end{macrocode}
%    \end{macro}
%    \begin{macro}{\BitSet@Zero}
%    \begin{macrocode}
\def\BitSet@Zero{0}
%    \end{macrocode}
%    \end{macro}
%
%    \begin{macro}{\bitsetQuery}
%    \begin{macrocode}
\def\bitsetQuery#1#2{%
  \ifnum\bitsetGet{#1}{#2}=1 %
    \expandafter\BitSet@FirstOfTwo
  \else
    \expandafter\BitSet@SecondOfTwo
  \fi
}
%    \end{macrocode}
%    \end{macro}
%
%    \begin{macro}{\bitsetEquals}
%    \begin{macrocode}
\def\bitsetEquals#1#2{%
  \BitSet@IfUndefined{#1}{%
    \BitSet@IfUndefined{#2}\BitSet@FirstOfTwo\BitSet@SecondOfTwo
  }{%
    \BitSet@IfUndefined{#2}\BitSet@SecondOfTwo{%
      \expandafter\ifx\csname BS@#1\expandafter\endcsname
                      \csname BS@#2\endcsname
        \expandafter\BitSet@FirstOfTwo
      \else
        \expandafter\BitSet@SecondOfTwo
      \fi
    }%
  }%
}
%    \end{macrocode}
%    \end{macro}
%
%    \begin{macro}{\bitsetIntersects}
%    \begin{macrocode}
\def\bitsetIntersects#1#2{%
  \bitsetIsEmpty{#1}\BitSet@SecondOfTwo{%
    \bitsetIsEmpty{#2}\BitSet@SecondOfTwo{%
      \expandafter\expandafter\expandafter\BitSet@Intersects
      \csname BS@#1\expandafter\expandafter\expandafter\endcsname
      \expandafter\expandafter\expandafter!%
      \csname BS@#2\endcsname!%
    }%
  }%
}
%    \end{macrocode}
%    \end{macro}
%    \begin{macro}{\BitSet@Intersects}
%    \begin{macrocode}
\def\BitSet@Intersects#1#2!#3#4!{%
  \ifnum#1#3=11 %
    \BitSet@AfterFi\BitSet@FirstOfTwo
  \else
    \ifx\\#2\\%
      \BitSet@AfterFiFi\BitSet@SecondOfTwo
    \else
      \ifx\\#4\\%
        \BitSet@AfterFiFiFi\BitSet@SecondOfTwo
      \else
        \BitSet@AfterFiFiFi{%
          \BitSet@Intersects#2!#4!%
        }%
      \fi
    \fi
  \BitSet@Fi
}
%    \end{macrocode}
%    \end{macro}
%
%    \begin{macrocode}
\BitSet@AtEnd%
%</package>
%    \end{macrocode}
%
% \section{Test}
%
% \subsection{Catcode checks for loading}
%
%    \begin{macrocode}
%<*test1>
%    \end{macrocode}
%    \begin{macrocode}
\catcode`\{=1 %
\catcode`\}=2 %
\catcode`\#=6 %
\catcode`\@=11 %
\expandafter\ifx\csname count@\endcsname\relax
  \countdef\count@=255 %
\fi
\expandafter\ifx\csname @gobble\endcsname\relax
  \long\def\@gobble#1{}%
\fi
\expandafter\ifx\csname @firstofone\endcsname\relax
  \long\def\@firstofone#1{#1}%
\fi
\expandafter\ifx\csname loop\endcsname\relax
  \expandafter\@firstofone
\else
  \expandafter\@gobble
\fi
{%
  \def\loop#1\repeat{%
    \def\body{#1}%
    \iterate
  }%
  \def\iterate{%
    \body
      \let\next\iterate
    \else
      \let\next\relax
    \fi
    \next
  }%
  \let\repeat=\fi
}%
\def\RestoreCatcodes{}
\count@=0 %
\loop
  \edef\RestoreCatcodes{%
    \RestoreCatcodes
    \catcode\the\count@=\the\catcode\count@\relax
  }%
\ifnum\count@<255 %
  \advance\count@ 1 %
\repeat

\def\RangeCatcodeInvalid#1#2{%
  \count@=#1\relax
  \loop
    \catcode\count@=15 %
  \ifnum\count@<#2\relax
    \advance\count@ 1 %
  \repeat
}
\def\RangeCatcodeCheck#1#2#3{%
  \count@=#1\relax
  \loop
    \ifnum#3=\catcode\count@
    \else
      \errmessage{%
        Character \the\count@\space
        with wrong catcode \the\catcode\count@\space
        instead of \number#3%
      }%
    \fi
  \ifnum\count@<#2\relax
    \advance\count@ 1 %
  \repeat
}
\def\space{ }
\expandafter\ifx\csname LoadCommand\endcsname\relax
  \def\LoadCommand{\input bitset.sty\relax}%
\fi
\def\Test{%
  \RangeCatcodeInvalid{0}{47}%
  \RangeCatcodeInvalid{58}{64}%
  \RangeCatcodeInvalid{91}{96}%
  \RangeCatcodeInvalid{123}{255}%
  \catcode`\@=12 %
  \catcode`\\=0 %
  \catcode`\%=14 %
  \LoadCommand
  \RangeCatcodeCheck{0}{36}{15}%
  \RangeCatcodeCheck{37}{37}{14}%
  \RangeCatcodeCheck{38}{47}{15}%
  \RangeCatcodeCheck{48}{57}{12}%
  \RangeCatcodeCheck{58}{63}{15}%
  \RangeCatcodeCheck{64}{64}{12}%
  \RangeCatcodeCheck{65}{90}{11}%
  \RangeCatcodeCheck{91}{91}{15}%
  \RangeCatcodeCheck{92}{92}{0}%
  \RangeCatcodeCheck{93}{96}{15}%
  \RangeCatcodeCheck{97}{122}{11}%
  \RangeCatcodeCheck{123}{255}{15}%
  \RestoreCatcodes
}
\Test
\csname @@end\endcsname
\end
%    \end{macrocode}
%    \begin{macrocode}
%</test1>
%    \end{macrocode}
%
% \subsection{Macro tests}
%
% \subsubsection{Preamble}
%
%    \begin{macrocode}
%<*test2>
\NeedsTeXFormat{LaTeX2e}
\nofiles
\documentclass{article}
\makeatletter
%<*noetex>
\let\SavedNumexpr\numexpr
\let\SavedIfcsname\ifcsname
\let\SavedCurrentgrouplevel\currentgrouplevel
\def\ETeXDisable{%
  \let\ifcsname\@undefined
  \let\numexpr\@undefined
  \let\currentgrouplevel\@undefined
}
\ETeXDisable
%</noetex>
\makeatletter
\chardef\BitSet@TestMode=1 %
\makeatother
\usepackage{bitset}[2011/01/30]
%<*noetex>
\def\ETeXEnable{%
  \let\numexpr\SavedNumexpr
  \let\ifcsname\SavedIfcsname
  \let\currentgrouplevel\SavedCurrentgrouplevel
}
\ETeXEnable
%</noetex>
\usepackage{qstest}
\IncludeTests{*}
\LogTests{log}{*}{*}
\makeatletter
%    \end{macrocode}
%
% \subsubsection{Time}
%
%    \begin{macrocode}
\begingroup\expandafter\expandafter\expandafter\endgroup
\expandafter\ifx\csname pdfresettimer\endcsname\relax
\else
  \newcount\SummaryTime
  \newcount\TestTime
  \SummaryTime=\z@
  \newcommand*{\PrintTime}[2]{%
    \typeout{%
      [Time #1: \strip@pt\dimexpr\number#2sp\relax\space s]%
    }%
  }%
  \newcommand*{\StartTime}[1]{%
    \renewcommand*{\TimeDescription}{#1}%
    \pdfresettimer
  }%
  \newcommand*{\TimeDescription}{}%
  \newcommand*{\StopTime}{%
    \TestTime=\pdfelapsedtime
    \global\advance\SummaryTime\TestTime
    \PrintTime\TimeDescription\TestTime
  }%
  \let\saved@qstest\qstest
  \let\saved@endqstest\endqstest
  \def\qstest#1#2{%
    \saved@qstest{#1}{#2}%
    \StartTime{#1}%
  }%
  \def\endqstest{%
    \StopTime
    \saved@endqstest
  }%
  \AtEndDocument{%
    \PrintTime{summary}\SummaryTime
  }%
\fi
%    \end{macrocode}
%
% \subsubsection{Detection of unwanted space}
%
%    \begin{macrocode}
\let\orig@qstest\qstest
\let\orig@endqstest\endqstest
\def\qstest#1#2{%
  \orig@qstest{#1}{#2}%
  \setbox0\hbox\bgroup\begingroup\ignorespaces
}
\def\endqstest{%
  \endgroup\egroup
  \Expect*{\the\wd0}{0.0pt}%
  \orig@endqstest
}
%    \end{macrocode}
%
% \subsubsection{Test macros}
%
%    \begin{macrocode}
\newcounter{Test}

\def\TestError#1#2{%
  \begingroup
    \setcounter{Test}{0}%
    \sbox0{%
      \def\@PackageError##1##2##3{%
        \stepcounter{Test}%
        \begingroup
          \let\MessageBreak\relax
%<*noetex>
          \ETeXEnable
%</noetex>
          \Expect{##1}{bitset}%
          \Expect*{##2}*{#1}%
        \endgroup
      }%
%<*noetex>
      \ETeXDisable
%</noetex>
      #2%
    }%
    \Expect*{\theTest}{1}%
    \Expect*{\the\wd0}{0.0pt}%
  \endgroup
}

\def\TestErrorNegativeIndex#1#2{%
  \TestError{Invalid negative index (#1)}{#2}%
}

\def\TestGetterUndefined#1{%
  \CheckUndef{dummy}%
  \expandafter\expandafter\expandafter\Expect
  \expandafter\expandafter\expandafter{#1{dummy}}{0}%
}

\def\ExpectBitSet#1#2{%
  \expandafter\expandafter\expandafter\Expect
  \expandafter\expandafter\expandafter
  {\csname BS@#1\endcsname}*{#2}%
}
\def\Check#1#2{%
  \ExpectBitSet{#1}{#2}%
}
\def\CheckUndef#1{%
  \begingroup
    \Expect*{%
      \expandafter
      \ifx\csname BS@#1\endcsname\relax true\else false\fi
    }{true}%
  \endgroup
}
\def\RevCheck#1#2{%
  \ExpectBitSet{#1}{\Reverse#2!!}%
}
\def\Set#1#2{%
  \expandafter\def\csname BS@#1\endcsname{#2}%
}
\def\RevSet#1#2{%
  \expandafter\edef\csname BS@#1\endcsname{%
    \Reverse#2!!%
  }%
}
\def\Reverse#1#2!#3!{%
  \ifx\\#2\\%
    #1#3%
    \expandafter\@gobble
  \else
    \expandafter\@firstofone
  \fi
  {\Reverse#2!#1#3!}%
}
%    \end{macrocode}
%
% \subsubsection{Test sets}
%
%    \begin{macrocode}
\begin{qstest}{Let}{Let}
  \CheckUndef{abc}%
  \CheckUndef{xyz}%
  \bitsetLet{xyz}{abc}%
  \CheckUndef{abc}%
  \Check{xyz}{0}%
  \Set{abc}{1}%
  \Check{abc}{1}%
  \Check{xyz}{0}%
  \bitsetLet{xyz}{abc}%
  \Check{abc}{1}%
  \Check{xyz}{1}%
  \Set{xyz}{11}%
  \Check{abc}{1}%
  \Check{xyz}{11}%
\end{qstest}

\begin{qstest}{Reset}{Reset}
  \bitsetReset{xyz}%
  \Check{xyz}{0}%
  \bitsetReset{abc}%
  \Check{abc}{0}%
  \Set{abc}{10101}%
  \bitsetReset{abc}%
  \Check{abc}{0}%
\end{qstest}

\begin{qstest}{Get/Query}{Get/Query}
  \expandafter\expandafter\expandafter\Expect
  \expandafter\expandafter\expandafter{%
    \bitsetGet{dummy}{0}%
  }{0}%
  \begingroup
    \expandafter\def\csname BitSetError:NegativeIndex\endcsname{}%
    \Set{abc}{1}%
    \Expect*{\bitsetQuery{abc}{-1}{true}{false}}{false}%
  \endgroup
  \def\Test#1#2#3{%
    \Set{abc}{#1}%
    \expandafter\expandafter\expandafter\Expect
    \expandafter\expandafter\expandafter{\bitsetGet{abc}{#2}}{#3}%
    \Expect*{\bitsetQuery{abc}{#2}{true}{false}}%
           *{\ifcase#3 false\or true\else error\fi}%
  }%
  \Test{1}{100}{0}%
  \Test{0}{0}{0}%
  \Test{1}{0}{1}%
  \Test{11}{1}{1}%
  \Test{111}{1}{1}%
  \Test{101}{1}{0}%
  \Test{101}{2}{1}%
  \Test{10100110011}{10}{1}%
\end{qstest}

\begin{qstest}{Size}{Size}
  \TestGetterUndefined\bitsetSize
  \def\Test#1#2{%
    \Set{abc}{#1}%
    \expandafter\expandafter\expandafter\Expect
    \expandafter\expandafter\expandafter{\bitsetSize{abc}}{#2}%
  }%
  \Test{0}{0}%
  \Test{1}{1}%
  \Test{00}{0}%
  \Test{0000000}{0}%
  \Test{10}{1}%
  \Test{01}{2}%
  \Test{11}{2}%
  \Test{010}{2}%
  \Test{011}{3}%
  \Test{100110011}{9}%
  \Test{0000011111000001111100000}{20}%
  \Test{000000000000000000000000011111111111111111111}{45}%
\end{qstest}

\begin{qstest}{Cardinality}{Cardinality}
  \TestGetterUndefined\bitsetCardinality
  \def\Test#1#2{%
    \Set{abc}{#1}%
    \expandafter\expandafter\expandafter\Expect
    \expandafter\expandafter\expandafter{%
      \bitsetCardinality{abc}%
    }{#2}%
  }%
  \Test{0}{0}%
  \Test{1}{1}%
  \Test{00}{0}%
  \Test{0000000}{0}%
  \Test{10}{1}%
  \Test{01}{1}%
  \Test{11}{2}%
  \Test{010}{1}%
  \Test{011}{2}%
  \Test{100110011}{5}%
  \Test{0000011111000001111100000}{10}%
  \Test{000000000000000000000000011111111111111111111}{20}%
\end{qstest}

\begin{qstest}{NextClearBit/NextSetBit}{NextClearBit/NextSetBit}
  \def\Test#1#2{%
    \expandafter\expandafter\expandafter\Expect
    \expandafter\expandafter\expandafter{%
      \TestOp{abc}{#1}%
    }{#2}%
  }%
  \def\Clear{\let\TestOp\bitsetNextClearBit}%
  \def\Set{\let\TestOp\bitsetNextSetBit}%
  \begingroup
    \catcode`\:=11 %
    \bitsetSetBin{abc}{1}%
    \Clear
    \Test{-1}{1\BitSetError:NegativeIndex}%
    \Set
    \Test{-1}{0\BitSetError:NegativeIndex}%
  \endgroup
  \let\BS@abc\@undefined
  \Clear
  \Test{0}{0}%
  \Test{1}{1}%
  \Test{2}{2}%
  \Test{100}{100}%
  \Set
  \Test{0}{-1}%
  \Test{1}{-1}%
  \Test{100}{-1}%
  \bitsetReset{abc}%
  \Clear
  \Test{0}{0}%
  \Test{1}{1}%
  \Test{2}{2}%
  \Test{100}{100}%
  \Set
  \Test{0}{-1}%
  \Test{1}{-1}%
  \Test{100}{-1}%
  \bitsetSetBin{abc}{1}%
  \Clear
  \Test{0}{1}%
  \Test{1}{1}%
  \Test{2}{2}%
  \Test{100}{100}%
  \Set
  \Test{0}{0}%
  \Test{1}{-1}%
  \Test{100}{-1}%
  \bitsetSetBin{abc}{111000111000111000111}%
  \Clear
  \Test{0}{3}%
  \Test{1}{3}%
  \Test{2}{3}%
  \Test{3}{3}%
  \Test{4}{4}%
  \Test{5}{5}%
  \Test{6}{9}%
  \Test{7}{9}%
  \Test{8}{9}%
  \Test{9}{9}%
  \Test{10}{10}%
  \Test{11}{11}%
  \Test{12}{15}%
  \Test{13}{15}%
  \Test{14}{15}%
  \Test{15}{15}%
  \Test{16}{16}%
  \Test{17}{17}%
  \Test{18}{21}%
  \Test{19}{21}%
  \Test{20}{21}%
  \Test{21}{21}%
  \Test{22}{22}%
  \Test{100}{100}%
  \Set
  \Test{0}{0}%
  \Test{1}{1}%
  \Test{2}{2}%
  \Test{3}{6}%
  \Test{4}{6}%
  \Test{5}{6}%
  \Test{6}{6}%
  \Test{7}{7}%
  \Test{8}{8}%
  \Test{9}{12}%
  \Test{10}{12}%
  \Test{11}{12}%
  \Test{12}{12}%
  \Test{13}{13}%
  \Test{14}{14}%
  \Test{15}{18}%
  \Test{16}{18}%
  \Test{17}{18}%
  \Test{18}{18}%
  \Test{19}{19}%
  \Test{20}{20}%
  \Test{21}{-1}%
  \Test{22}{-1}%
  \Test{100}{-1}%
  \bitsetSetBin{abc}{1111111}%
  \Clear
  \Test{6}{7}%
  \Test{7}{7}%
  \Test{8}{8}%
  \Test{100}{100}%
  \Set
  \Test{6}{6}%
  \Test{7}{-1}%
  \Test{8}{-1}%
  \Test{100}{-1}%
  \bitsetSetBin{abc}{11111111}%
  \Clear
  \Test{7}{8}%
  \Test{8}{8}%
  \Test{9}{9}%
  \Test{100}{100}%
  \Set
  \Test{7}{7}%
  \Test{8}{-1}%
  \Test{9}{-1}%
  \Test{100}{-1}%
  \bitsetSetBin{abc}{111111111}%
  \Clear
  \Test{8}{9}%
  \Test{9}{9}%
  \Test{10}{10}%
  \Test{100}{100}%
  \Set
  \Test{8}{8}%
  \Test{9}{-1}%
  \Test{10}{-1}%
  \Test{100}{-1}%
  \bitsetSetBin{abc}{1111111111}%
  \Clear
  \Test{9}{10}%
  \Test{10}{10}%
  \Test{11}{11}%
  \Test{100}{100}%
  \Set
  \Test{9}{9}%
  \Test{10}{-1}%
  \Test{11}{-1}%
  \Test{100}{-1}%
\end{qstest}

\begin{qstest}{GetSetBitList}{GetSetBitList}
  \let\BS@abc\@undefined
  \expandafter\expandafter\expandafter\Expect
  \expandafter\expandafter\expandafter{%
    \bitsetGetSetBitList{abc}%
  }{}%
  \def\Test#1#2{%
    \bitsetSetBin{abc}{#1}%
    \expandafter\expandafter\expandafter\Expect
    \expandafter\expandafter\expandafter{%
      \bitsetGetSetBitList{abc}%
    }{#2}%
  }%
  \Test{0}{}%
  \Test{1}{0}%
  \Test{10}{1}%
  \Test{11}{0,1}%
  \Test{10110100}{2,4,5,7}%
  \Test{101101001010011}{0,1,4,6,9,11,12,14}%
\end{qstest}

\begin{qstest}{GetDec}{GetDec}
  \TestGetterUndefined\bitsetGetDec
  \def\Test#1#2{%
    \RevSet{abc}{#1}%
%<*noetex>
    \begingroup\expandafter\expandafter\expandafter\endgroup
%</noetex>
    \expandafter\expandafter\expandafter\Expect
    \expandafter\expandafter\expandafter{%
      \bitsetGetDec{abc}%
    }{#2}%
  }%
  \Test{0}{0}%
  \Test{1}{1}%
  \Test{10}{2}%
  \Test{11}{3}%
  \Test{100}{4}%
  \Test{101}{5}%
  \Test{110}{6}%
  \Test{111}{7}%
  \Test{1000}{8}%
  \Test{000111}{7}%
  \Test{1111111111111111%
        111111111111111}{2147483647}%
  \Test{0001111111111111111%
        111111111111111}{2147483647}%
  \Test{1000000000000000%
        0000000000000000}{2147483648}%
  \Test{1000000000000000%
        00000000000000000}{4294967296}%
  \Test{0001000000000000000%
        00000000000000000}{4294967296}%
  \Test{1100000000000000%
        00000000000000011}{6442450947}%
\end{qstest}

\begin{qstest}{Clear}{Clear}
  \def\Test#1#2#3{%
    \RevSet{abc}{#1}%
    \bitsetClear{abc}{#2}%
    \Expect*{\BS@abc}*{\Reverse#3!!}%
  }%
  \bitsetClear{abc}{2}%
  \RevCheck{abc}{0}%
  \TestErrorNegativeIndex{-1}{\bitsetClear{abc}{-1}}%
  \RevCheck{abc}{0}%
  \Test{0}{0}{0}%
  \Test{1}{0}{0}%
  \Test{111}{1}{101}%
  \Test{111}{30}{111}%
  \Test{0000111}{5}{0000111}% 111 would also be ok
  \Test{10000111}{5}{10000111}%
  \Test{1001001}{3}{1000001}%
\end{qstest}

\begin{qstest}{Set}{Set}
  \def\Test#1#2#3{%
    \RevSet{abc}{#1}%
    \bitsetSet{abc}{#2}%
    \Expect*{\BS@abc}*{\Reverse#3!!}%
  }%
  \bitsetSet{abc}{2}%
  \RevCheck{abc}{100}%
  \TestErrorNegativeIndex{-1}{\bitsetSet{abc}{-1}}%
  \RevCheck{abc}{100}%
  \Test{0}{0}{1}%
  \Test{1}{0}{1}%
  \Test{100}{1}{110}%
  \Test{111}{1}{111}%
  \Test{11}{1}{11}%
  \Test{11}{2}{111}%
  \Test{11}{3}{1011}%
  \Test{111}{10}{10000000111}%
  \Test{0000111}{5}{0100111}% 100111 would also be ok
  \Test{10000111}{5}{10100111}%
  \Test{1000001}{3}{1001001}%
  \Test{1001001}{3}{1001001}%
\end{qstest}

\begin{qstest}{Flip}{Flip}
  \def\Test#1#2#3{%
    \RevSet{abc}{#1}%
    \bitsetFlip{abc}{#2}%
    \Expect*{\BS@abc}*{\Reverse#3!!}%
  }%
  \bitsetFlip{abc}{2}%
  \RevCheck{abc}{100}%
  \TestErrorNegativeIndex{-1}{\bitsetFlip{abc}{-1}}%
  \RevCheck{abc}{100}%
  \Test{0}{0}{1}%
  \Test{1}{0}{0}%
  \Test{0}{2}{100}%
  \Test{100}{1}{110}%
  \Test{111}{1}{101}%
  \Test{11}{1}{1}%
  \Test{11}{2}{111}%
  \Test{11}{3}{1011}%
  \Test{111}{10}{10000000111}%
  \Test{0000111}{5}{0100111}% 100111 would also be ok
  \Test{10000111}{5}{10100111}%
  \Test{1000001}{3}{1001001}%
  \Test{1001001}{3}{1000001}%
  \Test{11111}{2}{11011}%
\end{qstest}

\begin{qstest}{SetValue}{SetValue}
  \def\Test#1#2{%
    \TestError{Invalid bit value (#2) not in range 0..1}{%
      \bitsetSetValue{abc}{#1}{#2}%
    }%
  }%
  \Test{0}{-1}%
  \Test{0}{2}%
  \Test{0}{10}%
  \def\Test#1#2#3{%
    \let\BS@abc\@undefined
    \bitsetSetValue{abc}{#1}{#2}%
    \bitsetSetBin{result}{#3}%
    \Expect*{\BS@abc}*{\BS@result}%
  }%
  \Test{0}{0}{0}%
  \Test{0}{1}{1}%
  \Test{1}{0}{0}%
  \Test{1}{1}{10}%
  \def\Test#1#2#3#4{%
    \bitsetSetBin{abc}{#1}%
    \bitsetSetBin{result}{#4}%
    \bitsetSetValue{abc}{#2}{#3}%
    \Expect*{\BS@abc}*{\BS@result}%
  }%
  \Test{0}{0}{0}{0}%
  \Test{0}{0}{0}{0}%
  \Test{0}{0}{1}{1}%
  \Test{0}{1}{0}{0}%
  \Test{0}{1}{1}{10}%
  \Test{1010}{2}{1}{1110}%
  \Test{1010}{4}{1}{11010}%
  \Test{1010}{6}{1}{1001010}%
  \Test{1010}{1}{0}{1000}%
  \Test{1010}{2}{0}{1010}%
  \Test{1010}{3}{0}{10}%
  \Test{1010}{4}{0}{1010}%
  \Test{1010}{6}{0}{1010}%
  \Test{1010}{2}{\csname iffalse\endcsname 0\else 1\fi}{1110}%
  \Test{1010}{1}{\csname iffalse\endcsname 1\else 0\fi}{1000}%
\end{qstest}

\begin{qstest}{IsDefined}{IsDefined}
  \let\BS@abc\@undefined
  \Expect*{\bitsetIsDefined{abc}{true}{false}}{false}%
  \bitsetReset{abc}%
  \Expect*{\bitsetIsDefined{abc}{true}{false}}{true}%
\end{qstest}

\begin{qstest}{IsEmpty}{IsEmpty}
  \let\BS@abc\@undefined
  \Expect*{\bitsetIsEmpty{abc}{true}{false}}{true}%
  \bitsetReset{abc}%
  \Expect*{\bitsetIsEmpty{abc}{true}{false}}{true}%
  \bitsetSet{abc}{1}%
  \Expect*{\bitsetIsEmpty{abc}{true}{false}}{false}%
\end{qstest}

\begin{qstest}{Equals}{Equals}
  \def\Test#1#2#3{%
    \Expect*{\bitsetEquals{#1}{#2}{true}{false}}{#3}%
  }%
  \let\BS@abc\@undefined
  \Test{abc}{abc}{true}%
  \Test{abc}{foo}{true}%
  \Test{foo}{abc}{true}%
  \bitsetReset{abc}%
  \Test{abc}{abc}{true}%
  \Test{abc}{foo}{false}%
  \Test{foo}{abc}{false}%
  \bitsetReset{foo}%
  \Test{abc}{foo}{true}%
  \Test{foo}{abc}{true}%
  \bitsetSet{abc}{4}%
  \Test{abc}{foo}{false}%
  \Test{foo}{abc}{false}%
  \bitsetFlip{foo}{4}%
  \Test{abc}{foo}{true}%
  \Test{foo}{abc}{true}%
\end{qstest}

\begin{qstest}{Intersects}{Intersects}
  \def\Test#1{%
    \Expect*{\bitsetIntersects{abc}{foo}{true}{false}}{#1}%
  }%
  \let\BS@abc\@undefined
  \let\BS@foo\@undefined
  \Test{false}%
  \Set{abc}{0}%
  \Test{false}%
  \Set{foo}{0}%
  \Test{false}%
  \let\BS@abc\@undefined
  \Test{false}%
  \Set{foo}{1}%
  \Test{false}%
  \Set{abc}{0}%
  \Test{false}%
  \Set{abc}{1}%
  \Test{true}%
  \let\BS@foo\@undefined
  \Test{false}%
  \Set{foo}{0}%
  \Test{false}%
  \def\Test#1#2#3{%
    \bitsetSetBin{abc}{#1}%
    \bitsetSetBin{foo}{#2}%
    \Expect*{\bitsetIntersects{abc}{foo}{true}{false}}{#3}%
  }%
  \Test{1010}{0101}{false}%
  \Test{0}{10}{false}%
  \Test{1}{11}{true}%
  \Test{11}{1}{true}%
  \Test{10}{1}{false}%
\end{qstest}

\begin{qstest}{And/AndNot/Or/Xor}{And/AndNot/Or/Xor}
  \def\@Test#1#2#3#4#5{%
    \begingroup
      #5%
      \begingroup
        \let\BS@foo\@undefined
        \csname bitset#1\endcsname{abc}{foo}%
        \CheckUndef{foo}%
        \Check{abc}{#2}%
      \endgroup
      \begingroup
        \bitsetReset{foo}%
        \csname bitset#1\endcsname{abc}{foo}%
        \Check{foo}{0}%
        \Check{abc}{#3}%
      \endgroup
      \begingroup
        \def\BS@foo{0101}%
        \csname bitset#1\endcsname{abc}{foo}%
        \Check{foo}{0101}%
        \Check{abc}{#4}%
      \endgroup
    \endgroup
  }%
  \def\Test#1{%
    \def\Op{#1}%
    \Test@
  }%
  \def\Test@#1#2#3#4#5#6#7#8#9{%
    \@Test\Op{#1}{#2}{#3}{%
      \let\BS@abc\@undefined
    }%
    \@Test\Op{#4}{#5}{#6}{%
      \bitsetReset{abc}%
    }%
    \@Test\Op{#7}{#8}{#9}{%
      \def\BS@abc{1001}%
    }%
  }%
  \Test{And}%
       {0}{0}{0}%
       {0}{0}{0}%
       {0}{0}{0001}%
  \Test{AndNot}%
       {0}{0}{0}%
       {0}{0}{0}%
       {1001}{1001}{1}%
  \Test{Or}%
       {0}{0}{0101}%
       {0}{0}{0101}%
       {1001}{1001}{1101}%
  \Test{Xor}%
       {0}{0}{0101}%
       {0}{0}{0101}%
       {1001}{1001}{11}%
  \def\Test#1#2#3{%
    \bitsetSetBin{abc}{#1}%
    \bitsetSetBin{foo}{#2}%
    \csname bitset\Op\endcsname{abc}{foo}%
    \RevCheck{foo}{#2}%
    \RevCheck{abc}{#3}%
  }%
  \def\Op{And}%
  \Test{1}{111}{1}%
  \Test{111}{1}{1}%
  \Test{10}{111}{10}%
  \Test{111}{10}{10}%
  \Test{111}{1000}{0}%
  \Test{1000}{111}{0}%
  \def\Op{AndNot}%
  \Test{1010}{11}{1000}%
  \Test{100}{100}{0}%
  \Test{111}{1111}{0}%
  \Test{100}{111}{0}%
  \def\Op{Or}%
  \Test{0}{0}{0}%
  \Test{1}{0}{1}%
  \Test{0}{1}{1}%
  \Test{1}{1}{1}%
  \Test{1000}{10}{1010}%
  \Test{10}{1000}{1010}%
  \def\Op{Xor}%
  \Test{0}{0}{0}%
  \Test{1}{0}{1}%
  \Test{0}{1}{1}%
  \Test{1}{1}{0}%
  \Test{1000}{10}{1010}%
  \Test{10}{1000}{1010}%
  \Test   {110011001100}%
       {111000111000111}%
       {111110100001011}%
  \Test{111000111000111}%
          {110011001100}%
       {111110100001011}%
\end{qstest}

\begin{qstest}{GetUndef}{GetUndef, GetBin, GetOct, GetHex}
  \def\TestUndef#1#2{%
    \let\BS@abc\@undefined
    \expandafter\expandafter\expandafter\Expect
    \expandafter\expandafter\expandafter{%
      \x{abc}{#1}%
    }{#2}%
  }%
  \let\x\bitsetGetBin
  \TestUndef{-1}{0}%
  \TestUndef{0}{0}%
  \TestUndef{1}{0}%
  \TestUndef{2}{00}%
  \TestUndef{8}{00000000}%
  \let\x\bitsetGetOct
  \TestUndef{-1}{0}%
  \TestUndef{0}{0}%
  \TestUndef{1}{0}%
  \TestUndef{2}{0}%
  \TestUndef{3}{0}%
  \TestUndef{4}{00}%
  \TestUndef{5}{00}%
  \TestUndef{6}{00}%
  \TestUndef{7}{000}%
  \TestUndef{8}{000}%
  \TestUndef{9}{000}%
  \TestUndef{10}{0000}%
  \let\x\bitsetGetHex
  \TestUndef{-1}{0}%
  \TestUndef{0}{0}%
  \TestUndef{1}{0}%
  \TestUndef{2}{0}%
  \TestUndef{3}{0}%
  \TestUndef{4}{0}%
  \TestUndef{5}{00}%
  \TestUndef{6}{00}%
  \TestUndef{7}{00}%
  \TestUndef{8}{00}%
  \TestUndef{9}{000}%
  \TestUndef{10}{000}%
  \TestUndef{12}{000}%
  \TestUndef{13}{0000}%
  \TestUndef{16}{0000}%
  \TestUndef{17}{00000}%
\end{qstest}

\begin{qstest}{SetBin}{SetBin}
  \def\Test#1#2{%
    \let\BS@abc\@undefined
    \bitsetSetBin{abc}{#1}%
    \expandafter\Expect\expandafter{\BS@abc}{#2}%
  }%
  \Test{}{0}%
  \Test{0}{0}%
  \Test{1}{1}%
  \Test{10}{01}%
  \Test{11}{11}%
  \Test{010}{01}%
  \Test{011}{11}%
  \Test{0010}{01}%
  \Test{1010}{0101}%
\end{qstest}

\begin{qstest}{SetOct}{SetOct}
  \def\Test#1#2{%
    \bitsetSetOct{abc}{#1}%
    \expandafter\Expect\expandafter{\BS@abc}{#2}%
  }%
  \Test{}{0}%
  \Test{0}{0}%
  \Test{000}{0}%
  \Test{1}{1}%
  \Test{001}{1}%
  \Test{010}{0001}%
  \Test{020}{00001}%
  \Test{42}{010001}%
  \Test{377}{11111111}%
  \Test{0377}{11111111}%
  \Test{76543210}{000100010110001101011111}%
  \Test{ 0 7 0 7 1 }{100111000111}%
\end{qstest}

\begin{qstest}{SetHex}{SetHex}
  \def\Test#1#2{%
    \bitsetSetHex{abc}{#1}%
    \expandafter\Expect\expandafter{\BS@abc}{#2}%
  }%
  \Test{}{0}%
  \Test{0}{0}%
  \Test{000}{0}%
  \Test{1}{1}%
  \Test{001}{1}%
  \Test{010}{00001}%
  \Test{020}{000001}%
  \Test{42}{0100001}%
  \Test{3F}{111111}%
  \Test{03F}{111111}%
  \Test{43210}{0000100001001100001}%
  \Test{98765}{10100110111000011001}%
  \Test{FEDCBA}{010111010011101101111111}%
  \Test{ 0 F 0 F 1 }{1000111100001111}%
\end{qstest}

\begin{qstest}{SetDec}{SetDec}
  \def\Test#1#2{%
    \bitsetSetDec{abc}{#1}%
    \expandafter\Expect\expandafter{\BS@abc}{#2}%
  }%
  \Test{}{0}%
  \Test{0}{0}%
  \Test{000}{0}%
  \Test{1}{1}%
  \Test{7}{111}%
  \Test{8}{0001}%
  \Test{001}{1}%
  \Test{010}{0101}%
  \Test{020}{00101}%
  \Test{53}{101011}%
  \Test{255}{11111111}%
  \Test{256}{000000001}%
  \Test{999999999}{111111111001001101011001110111}%
  \Test{1000000000}{000000000101001101011001110111}%
  \Test{4210987654}{01100001010010010111111101011111}%
  \Test{2147483647}{1111111111111111111111111111111}%
  \Test{2147483648}{00000000000000000000000000000001}%
\end{qstest}

\begin{qstest}{GetBin}{GetBin}
  \def\TestUndef#1#2{%
    \let\BS@abc\@undefined
    \expandafter\expandafter\expandafter\Expect
    \expandafter\expandafter\expandafter{%
      \bitsetGetBin{abc}{#1}%
    }{#2}%
  }%
  \TestUndef{-1}{0}%
  \TestUndef{0}{0}%
  \TestUndef{1}{0}%
  \TestUndef{2}{00}%
  \TestUndef{8}{00000000}%
  \def\Test#1#2{%
    \bitsetSetBin{abc}{#2}%
    \expandafter\expandafter\expandafter\Expect
    \expandafter\expandafter\expandafter{%
      \bitsetGetBin{abc}{#1}%
    }{#2}%
  }%
  \Test{-1}{0}%
  \Test{0}{0}%
  \Test{1}{0}%
  \Test{1}{1}%
  \Test{2}{01}%
  \Test{2}{10}%
  \Test{3}{010}%
  \Test{2}{00}%
  \Test{2}{01}%
  \Test{8}{00101100}%
  \Test{2}{10101}%
  \Test{-100}{11011}%
\end{qstest}

\begin{qstest}{GetOct}{GetOct}
  \def\Test#1#2#3{%
    \edef\x{\zap@space#1 \@empty}%
    \edef\x{\noexpand\bitsetSetBin{abc}{\x}}%
    \x
    \expandafter\expandafter\expandafter\Expect
    \expandafter\expandafter\expandafter{%
      \bitsetGetOct{abc}{#2}%
    }{#3}%
  }%
  \Test{111 110 101 100 011 010 001 000}{0}{76543210}%
  \Test{000 111}{0}{7}%
  \Test{101 000}{-1}{50}%
  \Test{111}{-1}{7}%
  \Test{111}{0}{7}%
  \Test{111}{1}{7}%
  \Test{111}{3}{7}%
  \Test{111}{4}{07}%
  \Test{111}{6}{07}%
  \Test{111}{7}{007}%
  \Test{111 010}{6}{72}%
  \Test{111 010}{7}{072}%
  \Test{011 111}{0}{37}%
  \Test{011 111}{6}{37}%
  \Test{011 111}{7}{037}%
  \Test{001 111}{0}{17}%
  \Test{001 111}{6}{17}%
  \Test{001 111}{7}{017}%
\end{qstest}

\begin{qstest}{GetHex}{GetHex}
  \def\Test#1#2#3{%
    \bitsetSetBin{abc}{#1}%
    \expandafter\expandafter\expandafter\Expect
    \expandafter\expandafter\expandafter{%
      \bitsetGetHex{abc}{#2}%
    }{#3}%
  }%
  \Test{1111 1110 1101 1100 1011 1010 1001 1000}{0}{FEDCBA98}%
  \Test{0111 0110 0101 0100 0011 0010 0001 0000}{0}{76543210}%
  \Test{0000 1111}{0}{F}%
  \Test{0101 0000}{-1}{50}%
  \Test{1111}{-1}{F}%
  \Test{1111}{0}{F}%
  \Test{1111}{1}{F}%
  \Test{1111}{4}{F}%
  \Test{1111}{5}{0F}%
  \Test{1111}{8}{0F}%
  \Test{1111}{9}{00F}%
  \Test{1111 0010}{8}{F2}%
  \Test{1111 0010}{9}{0F2}%
  \Test{0111 1111}{0}{7F}%
  \Test{0111 1111}{8}{7F}%
  \Test{0111 1111}{9}{07F}%
  \Test{0011 1111}{0}{3F}%
  \Test{0011 1111}{8}{3F}%
  \Test{0011 1111}{9}{03F}%
  \Test{0001 1111}{0}{1F}%
  \Test{0001 1111}{8}{1F}%
  \Test{0001 1111}{9}{01F}%
\end{qstest}

\begin{qstest}{Range}{Range}
  \TestError{%
    Wrong index numbers in range [9..8]\MessageBreak% hash-ok
    for clear/set/flip on bit set `abc'.\MessageBreak
    The lower index exceeds the upper index.\MessageBreak
    Canceling the operation as error recovery%
  }{%
    \bitsetSetRange{abc}{9}{8}%
  }%
  \def\TestErrorNegInd#1#2#3#4#5#6{%
    \TestError{%
      Negative index in range [#2..#3]\MessageBreak % hash-ok
      for \string\bitset #1Range on bit set `abc'.\MessageBreak
      Using [#4..#5] as error recovery% hash-ok
    }{%
      \csname bitset#1Range\endcsname{abc}{#2}{#3}%
      \global\let\BS@global\BS@abc
    }%
    \Check{global}{#6}%
  }%
  \Set{abc}{111}%
  \TestErrorNegInd{Clear}{-1}{0}{0}{0}{111}%
  \TestErrorNegInd{Clear}{0}{-1}{0}{0}{111}%
  \TestErrorNegInd{Clear}{-2}{2}{0}{2}{001}%
  \bitsetReset{abc}%
  \TestErrorNegInd{Set}{-1}{0}{0}{0}{0}%
  \TestErrorNegInd{Set}{0}{-1}{0}{0}{0}%
  \TestErrorNegInd{Set}{-2}{2}{0}{2}{11}%
  \Set{abc}{101}%
  \TestErrorNegInd{Flip}{-1}{0}{0}{0}{101}%
  \TestErrorNegInd{Flip}{0}{-1}{0}{0}{101}%
  \TestErrorNegInd{Flip}{-2}{2}{0}{2}{011}%
  \def\Test#1#2#3#4{%
    \bitsetSetBin{abc}{#1}%
    \csname bitset\TestOp Range\endcsname{abc}{#2}{#3}%
    \Expect*{\bitsetGetBin{abc}{0}}{#4}%
  }%
  \def\TestOp{Clear}%
  \Test{0}{0}{1}{0}%
  \Test{1111}{1}{2}{1101}%
  \Test{1111}{1}{3}{1001}%
  \Test{1111111100000000}{12}{14}{1100111100000000}%
  \def\TestOp{Set}%
  \Test{0}{0}{1}{1}%
  \Test{1000}{1}{2}{1010}%
  \Test{0}{1}{2}{10}%
  \Test{1}{12}{15}{111000000000001}%
  \Test{1111}{1}{3}{1111}%
  \Test{1000000000000000}{12}{14}{1011000000000000}%
  \def\TestOp{Flip}%
  \Test{0}{0}{1}{1}%
  \Test{1}{0}{1}{0}%
  \Test{10101010}{1}{5}{10110100}%
  \def\Test#1#2#3#4#5{%
    \bitsetSetBin{abc}{#1}%
    \bitsetSetValueRange{abc}{#2}{#3}{#4}%
    \Expect*{\bitsetGetBin{abc}{0}}{#5}%
  }%
  \Test{0}{0}{1}{0}{0}%
  \Test{0}{0}{1}{1}{1}%
  \Test{1010}{1}{3}{0}{1000}%
  \Test{1010}{1}{3}{1}{1110}%
\end{qstest}

\begin{qstest}{ShiftLeft/ShiftRight}{ShiftLeft/ShiftRight}
  \def\@Test#1#2{%
    \let\BS@abc\@undefined
    \csname bitsetShift#1\endcsname{abc}{#2}%
    \Expect*{\BS@abc}{0}%
  }%
  \def\Test#1{%
    \@Test{Left}{#1}%
    \@Test{Right}{#1}%
  }%
  \Test{-16}%
  \Test{-1}%
  \Test{0}%
  \Test{1}%
  \Test{16}%
  \def\Test#1#2#3{%
    \bitsetSetBin{abc}{#1}%
    \bitsetSetBin{result}{#3}%
    \csname bitsetShift\Op\endcsname{abc}{#2}%
    \Expect*{\bitsetGetBin{abc}{0}}*{\bitsetGetBin{result}{0}}%
  }%
  \def\Op{Left}%
  \Test{0}{0}{0}%
  \Test{0}{1}{0}%
  \Test{0}{-1}{0}%
  \Test{1}{0}{1}%
  \Test{1}{1}{10}%
  \Test{1}{-1}{0}%
  \Test{10}{1}{100}%
  \Test{10}{-1}{1}%
  \Test{1}{32}{100000000000000000000000000000000}%
  \Test{1}{-100}{0}%
  \def\Op{Right}%
  \Test{0}{0}{0}%
  \Test{0}{1}{0}%
  \Test{0}{-1}{0}%
  \Test{1}{0}{1}%
  \Test{1}{1}{0}%
  \Test{1}{-1}{10}%
  \Test{10}{1}{1}%
  \Test{10}{-1}{100}%
  \Test{1}{-32}{100000000000000000000000000000000}%
  \Test{1}{100}{0}%
  \Test{110110110110110}{10}{11011}%
  \Test{110110110110110}{100}{0}%
  \Test{1}{100000}{0}%
\end{qstest}

\begin{qstest}{Profile: Set}{Profile: Set}
  \bitsetSet{abc}{4095}%
  \global\let\BS@global\BS@abc
\end{qstest}

\begin{qstest}{Profile: Get}{Profile: Get}
  \edef\x{\bitsetGet{global}{4095}}%
\end{qstest}

\begin{document}
\end{document}
%</test2>
%    \end{macrocode}
%
% \section{Installation}
%
% \subsection{Download}
%
% \paragraph{Package.} This package is available on
% CTAN\footnote{\url{ftp://ftp.ctan.org/tex-archive/}}:
% \begin{description}
% \item[\CTAN{macros/latex/contrib/oberdiek/bitset.dtx}] The source file.
% \item[\CTAN{macros/latex/contrib/oberdiek/bitset.pdf}] Documentation.
% \end{description}
%
%
% \paragraph{Bundle.} All the packages of the bundle `oberdiek'
% are also available in a TDS compliant ZIP archive. There
% the packages are already unpacked and the documentation files
% are generated. The files and directories obey the TDS standard.
% \begin{description}
% \item[\CTAN{install/macros/latex/contrib/oberdiek.tds.zip}]
% \end{description}
% \emph{TDS} refers to the standard ``A Directory Structure
% for \TeX\ Files'' (\CTAN{tds/tds.pdf}). Directories
% with \xfile{texmf} in their name are usually organized this way.
%
% \subsection{Bundle installation}
%
% \paragraph{Unpacking.} Unpack the \xfile{oberdiek.tds.zip} in the
% TDS tree (also known as \xfile{texmf} tree) of your choice.
% Example (linux):
% \begin{quote}
%   |unzip oberdiek.tds.zip -d ~/texmf|
% \end{quote}
%
% \paragraph{Script installation.}
% Check the directory \xfile{TDS:scripts/oberdiek/} for
% scripts that need further installation steps.
% Package \xpackage{attachfile2} comes with the Perl script
% \xfile{pdfatfi.pl} that should be installed in such a way
% that it can be called as \texttt{pdfatfi}.
% Example (linux):
% \begin{quote}
%   |chmod +x scripts/oberdiek/pdfatfi.pl|\\
%   |cp scripts/oberdiek/pdfatfi.pl /usr/local/bin/|
% \end{quote}
%
% \subsection{Package installation}
%
% \paragraph{Unpacking.} The \xfile{.dtx} file is a self-extracting
% \docstrip\ archive. The files are extracted by running the
% \xfile{.dtx} through \plainTeX:
% \begin{quote}
%   \verb|tex bitset.dtx|
% \end{quote}
%
% \paragraph{TDS.} Now the different files must be moved into
% the different directories in your installation TDS tree
% (also known as \xfile{texmf} tree):
% \begin{quote}
% \def\t{^^A
% \begin{tabular}{@{}>{\ttfamily}l@{ $\rightarrow$ }>{\ttfamily}l@{}}
%   bitset.sty & tex/generic/oberdiek/bitset.sty\\
%   bitset.pdf & doc/latex/oberdiek/bitset.pdf\\
%   test/bitset-test1.tex & doc/latex/oberdiek/test/bitset-test1.tex\\
%   test/bitset-test2.tex & doc/latex/oberdiek/test/bitset-test2.tex\\
%   test/bitset-test3.tex & doc/latex/oberdiek/test/bitset-test3.tex\\
%   bitset.dtx & source/latex/oberdiek/bitset.dtx\\
% \end{tabular}^^A
% }^^A
% \sbox0{\t}^^A
% \ifdim\wd0>\linewidth
%   \begingroup
%     \advance\linewidth by\leftmargin
%     \advance\linewidth by\rightmargin
%   \edef\x{\endgroup
%     \def\noexpand\lw{\the\linewidth}^^A
%   }\x
%   \def\lwbox{^^A
%     \leavevmode
%     \hbox to \linewidth{^^A
%       \kern-\leftmargin\relax
%       \hss
%       \usebox0
%       \hss
%       \kern-\rightmargin\relax
%     }^^A
%   }^^A
%   \ifdim\wd0>\lw
%     \sbox0{\small\t}^^A
%     \ifdim\wd0>\linewidth
%       \ifdim\wd0>\lw
%         \sbox0{\footnotesize\t}^^A
%         \ifdim\wd0>\linewidth
%           \ifdim\wd0>\lw
%             \sbox0{\scriptsize\t}^^A
%             \ifdim\wd0>\linewidth
%               \ifdim\wd0>\lw
%                 \sbox0{\tiny\t}^^A
%                 \ifdim\wd0>\linewidth
%                   \lwbox
%                 \else
%                   \usebox0
%                 \fi
%               \else
%                 \lwbox
%               \fi
%             \else
%               \usebox0
%             \fi
%           \else
%             \lwbox
%           \fi
%         \else
%           \usebox0
%         \fi
%       \else
%         \lwbox
%       \fi
%     \else
%       \usebox0
%     \fi
%   \else
%     \lwbox
%   \fi
% \else
%   \usebox0
% \fi
% \end{quote}
% If you have a \xfile{docstrip.cfg} that configures and enables \docstrip's
% TDS installing feature, then some files can already be in the right
% place, see the documentation of \docstrip.
%
% \subsection{Refresh file name databases}
%
% If your \TeX~distribution
% (\teTeX, \mikTeX, \dots) relies on file name databases, you must refresh
% these. For example, \teTeX\ users run \verb|texhash| or
% \verb|mktexlsr|.
%
% \subsection{Some details for the interested}
%
% \paragraph{Attached source.}
%
% The PDF documentation on CTAN also includes the
% \xfile{.dtx} source file. It can be extracted by
% AcrobatReader 6 or higher. Another option is \textsf{pdftk},
% e.g. unpack the file into the current directory:
% \begin{quote}
%   \verb|pdftk bitset.pdf unpack_files output .|
% \end{quote}
%
% \paragraph{Unpacking with \LaTeX.}
% The \xfile{.dtx} chooses its action depending on the format:
% \begin{description}
% \item[\plainTeX:] Run \docstrip\ and extract the files.
% \item[\LaTeX:] Generate the documentation.
% \end{description}
% If you insist on using \LaTeX\ for \docstrip\ (really,
% \docstrip\ does not need \LaTeX), then inform the autodetect routine
% about your intention:
% \begin{quote}
%   \verb|latex \let\install=y% \iffalse meta-comment
%
% File: bitset.dtx
% Version: 2011/01/30 v1.1
% Info: Handle bit-vector datatype
%
% Copyright (C) 2007, 2011 by
%    Heiko Oberdiek <heiko.oberdiek at googlemail.com>
%
% This work may be distributed and/or modified under the
% conditions of the LaTeX Project Public License, either
% version 1.3c of this license or (at your option) any later
% version. This version of this license is in
%    http://www.latex-project.org/lppl/lppl-1-3c.txt
% and the latest version of this license is in
%    http://www.latex-project.org/lppl.txt
% and version 1.3 or later is part of all distributions of
% LaTeX version 2005/12/01 or later.
%
% This work has the LPPL maintenance status "maintained".
%
% This Current Maintainer of this work is Heiko Oberdiek.
%
% The Base Interpreter refers to any `TeX-Format',
% because some files are installed in TDS:tex/generic//.
%
% This work consists of the main source file bitset.dtx
% and the derived files
%    bitset.sty, bitset.pdf, bitset.ins, bitset.drv, bitset-test1.tex,
%    bitset-test2.tex, bitset-test3.tex.
%
% Distribution:
%    CTAN:macros/latex/contrib/oberdiek/bitset.dtx
%    CTAN:macros/latex/contrib/oberdiek/bitset.pdf
%
% Unpacking:
%    (a) If bitset.ins is present:
%           tex bitset.ins
%    (b) Without bitset.ins:
%           tex bitset.dtx
%    (c) If you insist on using LaTeX
%           latex \let\install=y% \iffalse meta-comment
%
% File: bitset.dtx
% Version: 2011/01/30 v1.1
% Info: Handle bit-vector datatype
%
% Copyright (C) 2007, 2011 by
%    Heiko Oberdiek <heiko.oberdiek at googlemail.com>
%
% This work may be distributed and/or modified under the
% conditions of the LaTeX Project Public License, either
% version 1.3c of this license or (at your option) any later
% version. This version of this license is in
%    http://www.latex-project.org/lppl/lppl-1-3c.txt
% and the latest version of this license is in
%    http://www.latex-project.org/lppl.txt
% and version 1.3 or later is part of all distributions of
% LaTeX version 2005/12/01 or later.
%
% This work has the LPPL maintenance status "maintained".
%
% This Current Maintainer of this work is Heiko Oberdiek.
%
% The Base Interpreter refers to any `TeX-Format',
% because some files are installed in TDS:tex/generic//.
%
% This work consists of the main source file bitset.dtx
% and the derived files
%    bitset.sty, bitset.pdf, bitset.ins, bitset.drv, bitset-test1.tex,
%    bitset-test2.tex, bitset-test3.tex.
%
% Distribution:
%    CTAN:macros/latex/contrib/oberdiek/bitset.dtx
%    CTAN:macros/latex/contrib/oberdiek/bitset.pdf
%
% Unpacking:
%    (a) If bitset.ins is present:
%           tex bitset.ins
%    (b) Without bitset.ins:
%           tex bitset.dtx
%    (c) If you insist on using LaTeX
%           latex \let\install=y% \iffalse meta-comment
%
% File: bitset.dtx
% Version: 2011/01/30 v1.1
% Info: Handle bit-vector datatype
%
% Copyright (C) 2007, 2011 by
%    Heiko Oberdiek <heiko.oberdiek at googlemail.com>
%
% This work may be distributed and/or modified under the
% conditions of the LaTeX Project Public License, either
% version 1.3c of this license or (at your option) any later
% version. This version of this license is in
%    http://www.latex-project.org/lppl/lppl-1-3c.txt
% and the latest version of this license is in
%    http://www.latex-project.org/lppl.txt
% and version 1.3 or later is part of all distributions of
% LaTeX version 2005/12/01 or later.
%
% This work has the LPPL maintenance status "maintained".
%
% This Current Maintainer of this work is Heiko Oberdiek.
%
% The Base Interpreter refers to any `TeX-Format',
% because some files are installed in TDS:tex/generic//.
%
% This work consists of the main source file bitset.dtx
% and the derived files
%    bitset.sty, bitset.pdf, bitset.ins, bitset.drv, bitset-test1.tex,
%    bitset-test2.tex, bitset-test3.tex.
%
% Distribution:
%    CTAN:macros/latex/contrib/oberdiek/bitset.dtx
%    CTAN:macros/latex/contrib/oberdiek/bitset.pdf
%
% Unpacking:
%    (a) If bitset.ins is present:
%           tex bitset.ins
%    (b) Without bitset.ins:
%           tex bitset.dtx
%    (c) If you insist on using LaTeX
%           latex \let\install=y\input{bitset.dtx}
%        (quote the arguments according to the demands of your shell)
%
% Documentation:
%    (a) If bitset.drv is present:
%           latex bitset.drv
%    (b) Without bitset.drv:
%           latex bitset.dtx; ...
%    The class ltxdoc loads the configuration file ltxdoc.cfg
%    if available. Here you can specify further options, e.g.
%    use A4 as paper format:
%       \PassOptionsToClass{a4paper}{article}
%
%    Programm calls to get the documentation (example):
%       pdflatex bitset.dtx
%       makeindex -s gind.ist bitset.idx
%       pdflatex bitset.dtx
%       makeindex -s gind.ist bitset.idx
%       pdflatex bitset.dtx
%
% Installation:
%    TDS:tex/generic/oberdiek/bitset.sty
%    TDS:doc/latex/oberdiek/bitset.pdf
%    TDS:doc/latex/oberdiek/test/bitset-test1.tex
%    TDS:doc/latex/oberdiek/test/bitset-test2.tex
%    TDS:doc/latex/oberdiek/test/bitset-test3.tex
%    TDS:source/latex/oberdiek/bitset.dtx
%
%<*ignore>
\begingroup
  \catcode123=1 %
  \catcode125=2 %
  \def\x{LaTeX2e}%
\expandafter\endgroup
\ifcase 0\ifx\install y1\fi\expandafter
         \ifx\csname processbatchFile\endcsname\relax\else1\fi
         \ifx\fmtname\x\else 1\fi\relax
\else\csname fi\endcsname
%</ignore>
%<*install>
\input docstrip.tex
\Msg{************************************************************************}
\Msg{* Installation}
\Msg{* Package: bitset 2011/01/30 v1.1 Handle bit-vector datatype (HO)}
\Msg{************************************************************************}

\keepsilent
\askforoverwritefalse

\let\MetaPrefix\relax
\preamble

This is a generated file.

Project: bitset
Version: 2011/01/30 v1.1

Copyright (C) 2007, 2011 by
   Heiko Oberdiek <heiko.oberdiek at googlemail.com>

This work may be distributed and/or modified under the
conditions of the LaTeX Project Public License, either
version 1.3c of this license or (at your option) any later
version. This version of this license is in
   http://www.latex-project.org/lppl/lppl-1-3c.txt
and the latest version of this license is in
   http://www.latex-project.org/lppl.txt
and version 1.3 or later is part of all distributions of
LaTeX version 2005/12/01 or later.

This work has the LPPL maintenance status "maintained".

This Current Maintainer of this work is Heiko Oberdiek.

The Base Interpreter refers to any `TeX-Format',
because some files are installed in TDS:tex/generic//.

This work consists of the main source file bitset.dtx
and the derived files
   bitset.sty, bitset.pdf, bitset.ins, bitset.drv, bitset-test1.tex,
   bitset-test2.tex, bitset-test3.tex.

\endpreamble
\let\MetaPrefix\DoubleperCent

\generate{%
  \file{bitset.ins}{\from{bitset.dtx}{install}}%
  \file{bitset.drv}{\from{bitset.dtx}{driver}}%
  \usedir{tex/generic/oberdiek}%
  \file{bitset.sty}{\from{bitset.dtx}{package}}%
  \usedir{doc/latex/oberdiek/test}%
  \file{bitset-test1.tex}{\from{bitset.dtx}{test1}}%
  \file{bitset-test2.tex}{\from{bitset.dtx}{test2,etex}}%
  \file{bitset-test3.tex}{\from{bitset.dtx}{test2,noetex}}%
  \nopreamble
  \nopostamble
  \usedir{source/latex/oberdiek/catalogue}%
  \file{bitset.xml}{\from{bitset.dtx}{catalogue}}%
}

\catcode32=13\relax% active space
\let =\space%
\Msg{************************************************************************}
\Msg{*}
\Msg{* To finish the installation you have to move the following}
\Msg{* file into a directory searched by TeX:}
\Msg{*}
\Msg{*     bitset.sty}
\Msg{*}
\Msg{* To produce the documentation run the file `bitset.drv'}
\Msg{* through LaTeX.}
\Msg{*}
\Msg{* Happy TeXing!}
\Msg{*}
\Msg{************************************************************************}

\endbatchfile
%</install>
%<*ignore>
\fi
%</ignore>
%<*driver>
\NeedsTeXFormat{LaTeX2e}
\ProvidesFile{bitset.drv}%
  [2011/01/30 v1.1 Handle bit-vector datatype (HO)]%
\documentclass{ltxdoc}
\usepackage{holtxdoc}[2011/11/22]
\usepackage{amstext}
\usepackage{ifluatex}
\ifluatex
\else
  \usepackage[T1]{fontenc}%
  \usepackage{lmodern}%
\fi
\begin{document}
  \DocInput{bitset.dtx}%
\end{document}
%</driver>
% \fi
%
% \CheckSum{3510}
%
% \CharacterTable
%  {Upper-case    \A\B\C\D\E\F\G\H\I\J\K\L\M\N\O\P\Q\R\S\T\U\V\W\X\Y\Z
%   Lower-case    \a\b\c\d\e\f\g\h\i\j\k\l\m\n\o\p\q\r\s\t\u\v\w\x\y\z
%   Digits        \0\1\2\3\4\5\6\7\8\9
%   Exclamation   \!     Double quote  \"     Hash (number) \#
%   Dollar        \$     Percent       \%     Ampersand     \&
%   Acute accent  \'     Left paren    \(     Right paren   \)
%   Asterisk      \*     Plus          \+     Comma         \,
%   Minus         \-     Point         \.     Solidus       \/
%   Colon         \:     Semicolon     \;     Less than     \<
%   Equals        \=     Greater than  \>     Question mark \?
%   Commercial at \@     Left bracket  \[     Backslash     \\
%   Right bracket \]     Circumflex    \^     Underscore    \_
%   Grave accent  \`     Left brace    \{     Vertical bar  \|
%   Right brace   \}     Tilde         \~}
%
% \GetFileInfo{bitset.drv}
%
% \title{The \xpackage{bitset} package}
% \date{2011/01/30 v1.1}
% \author{Heiko Oberdiek\\\xemail{heiko.oberdiek at googlemail.com}}
%
% \maketitle
%
% \begin{abstract}
% This package defines and implements the data type bit set,
% a vector of bits. The size of the vector may grow dynamically.
% Individual bits can be manipulated.
% \end{abstract}
%
% \tableofcontents
%
% \clearpage
% \section{Documentation}
%
% \subsection{Introduction}
%
% Annotations in the PDF format know entries whose values are
% integers. This numbers are interpreted as set of flags
% specifying properties. For example, annotation dictionaries
% can have a key |/F|. The bits of its integer value
% are interpreted the following way:
% \begin{quote}
% \begin{tabular}{@{}l|l@{}}
%   Bit position & Property name\\
%   \hline
%   1 & Invisible\\
%   2 & Hidden\\
%   3 & Print\\
%   4 & NoZoom\\
%   5 & NoRotate\\
%   6 & NoView\\
%   7 & ReadOnly\\
%   \dots & \dots
% \end{tabular}
% \end{quote}
% Now, let's see how these values are set in package \xpackage{hyperref}
% before it uses this package (before v6.77a):
%\begin{quote}
%\begin{verbatim}
%\ifFld@hidden /F 6\else /F 4\fi
%\end{verbatim}
%\end{quote}
% Where are the other flags? The following example for key |/Ff|
% in a widget annotation
% supports at least three properties:
%\begin{quote}
%\begin{verbatim}
%\ifFld@multiline
%  \ifFld@readonly /Ff 4097\else /Ff 4096\fi
%\else
%  \ifFld@password
%    \ifFld@readonly /Ff 8193\else /Ff 8192\fi
%  \else
%    \ifFld@readonly /Ff 1\fi
%  \fi
%\fi
%\end{verbatim}
%\end{quote}
% But you see the point. It would be a nightmare to continue
% this way in supporting the missing flag settings. This kind
% of integers may have up to 32 bits.
%
% Therefore I wanted a data structure for setting and
% clearing individual bits. Also it should provide
% an export as decimal number. The snipsets above are
% executed in expansion contexts without \TeX's stomach
% commands. It would be convenient to have an expandable
% conversion from the data structure to the integer that
% gets written to the PDF file.
%
% This package \xpackage{bitset} implements such a data
% structure. The interface is quite close to Java's
% class |BitSet| in order not to learn to many interfaces
% for the same kind of data structure.
%
% \subsection{Glossary}
%
% \begin{description}
% \item[Bit set:]
%   A bit set is a vector of bits or flags.
%   The vector size is unlimited and
%   grows dynamically. An undefined bit set is treated as bit
%   set where all bits are cleared.
%
%   Bit sets are addressed by name. A name should consists of
%   letters or digits. Technically it must survive \cs{csname},
%   see \LaTeX's environment names for other names with such
%   a constraint. Package \xpackage{babel}'s shorthands are
%   not supported due to technical reasons. Shorthand support
%   would break expandable operations.
% \item[Size:]
%   A size of a bit set is the number of bits in use.
%   It's the number of the highest index, incremented by one.
%   Sizes are in the range |0| up to |2147483647|, the highest
%   number supported by \TeX.
% \item[Index:]
%   Bit positions in a bit set are addressed by an index number.
%   The bit vector is zero based. The first and least significant
%   bit is addressed by index |0| and the highest possible bit
%   by |2147483646|.
% \item[Bit:]
%   A bit is enoded as |0| for cleared/disabled or |1| for set/enabled.
% \end{description}
%
% \subsection{Design principles}
%
% \begin{description}
% \item[Name conventions:]
%   To avoid conflicts with existing macro names, the operations
%   are prefixed by the package name.
% \item[Zero based indexes:] The first bit is addressed by zero.
%   (Convention of array indexing in C, Java, \dots)
% \item[Unlimited size:]
%   There is no restriction on the size of a bit set other than
%   usual memory limitations. \cs{bitsetSetDec} and \cs{bitsetGetDec}
%   transparently switch to package \xpackage{bigintcalc} if the numbers
%   get too large for \TeX's number limit.
% \item[Expandibility:]
%   Any operation that does not change the bit set is expandable.
%   And all operations that extract or calculate some result
%   do this in exact two expansion steps. For example,
%   a macro \cs{Macro} wants a bit set as decimal number.
%   But the argument must be a plain number without macros.
%   Thus you could prefix \cs{bitsetGetDec} with \cs{number}.
%   However this won't work for bit sets with 31 or more bits
%   because of \TeX's number limit of $2^{31}-1$.
%   then just hit the operator with two \cs{expandafter}:
%   \begin{quote}
%   |\expandafter\expandafter\expandafter|\\
%   |\Macro\bitsetGetDec{foo}|
%   \end{quote}
%   |\bitsetGetDec| is hit first by the third \cs{expandafter}
%   and then by the second one.
% \item[Format independence:] This package is written as \LaTeX\ package,
%   but it does not depend on \LaTeX. It will also work for other formats
%   such as \plainTeX.
% \item[Independence from \TeX\ engines:]
%   Vanilla \TeX\ is all you need. Calculations are delegated to
%   packages \xpackage{intcalc} and \xpackage{bigintcalc}.
%   They don't need any special features, but they will switch
%   to a little more efficient implementation if features such
%   as \cs{numexpr} are available.
% \item[Numeric arguments:] Anything that is accepted by
%   \cs{number}. If \eTeX\ is detected, also expressions
%   for \cs{numexpr} are supported. The only exception so far
%   is the number for \cs{bitsetSetDec}. The number might be too
%   large for \cs{number} or \cs{numexpr}.
% \item[Error messages:]
%   In expandable contexts, only a limited set of \TeX\ primitive
%   commands work as expected. So called stomach commands behave
%   like \cs{relax} and don't get expanded or executed.
%   Unhappily also the error commands belong to this category.
%   The expandable operations will throw an unknown control
%   sequence instead to get \TeX's and user's attention.
%   The name of these control sequences starts with
%   \cs{BitSetError:} with the type of error after the colon.
% \end{description}
%
% \subsection{Operator overview}
%
% \begin{minipage}{\linewidth}
% \begin{description}
% \def\It#1#2{^^A
%   \item[^^A
%     {#1} ^^A
%     \textmd{(\hyperref[{sec:#2}]{section \ref*{sec:#2}})}^^A
%   ]^^A
%   \mbox{}\\^^A
% }
% \def\op#1{^^A
%   \cs{bitset}\texttt{\bfseries#1}^^A
% }
% \def\M#1{\meta{#1}}
% \def\Q{\hspace*{\fill}}
% \It{Miscellaneous}{misc}
%   \op{Reset}\Q \M{BitSet}\\
%   \op{Let}\Q \M{BitSet A} \M{BitSet B}
% \It{Import}{import}
%   \op{SetBin}, \op{SetOct}, \op{SetHex}\Q \M{BitSet} \M{Value}\\
%   \op{SetDec}\Q \M{BitSet} \M{Value}
% \It{Export\textsuperscript{a}}{export}
%   \op{GetBin}, \op{GetOct}, \op{GetHex}\Q \M{BitSet} \M{MinSize}\\
%   \op{GetDec}\Q \M{BitSet}
% \It{Logical operators}{log}
%   \op{And}, \op{AndNot}\Q \M{BitSet A} \M{BitSet B}\\
%   \op{Or}, \op{Xor}\Q \M{BitSet A} \M{BitSet B}
% \It{Shifting}{shift}
%   \op{ShiftLeft}, \op{ShiftRight}\Q \M{BitSet} \M{ShiftAmount}\\
% \It{Bit manipulation}{bitman}
%   \op{Clear}, \op{Set}, \op{Flip}\Q \M{BitSet} \M{Index}\\
%   \op{SetValue}\Q \M{BitSet} \M{Index} \M{Value}\\
%   \op{ClearRange}, \op{SetRange}, \op{FlipRange}\\
%     \Q \M{BitSet} \M{IndexFrom} \M{IndexTo}\\
%   \op{SetValueRange}\Q \M{BitSet} \M{IndexFrom} \M{IndexTo}
% \It{Bit retrieval\textsuperscript{a}}{bitret}
%   \op{Get}\Q \M{BitSet} \M{Index}\\
%   \op{NextClearBit}, \op{NextSetBit}\Q \M{BitSet} \M{Index}\\
%   \op{GetSetBitList}\Q \M{BitSet}
% \It{Bit set properties}{prop}
%   \op{Size}, \op{Cardinality}\Q \M{BitSet}
% \It{Queries\textsuperscript{b}}{query}
%   \op{IsDefined}, \op{IsEmpty}\Q \M{BitSet} \M{Then} \M{Else}\\
%   \op{Equals}, \op{Intersects}\Q \M{BitSet A} \M{BitSet B} \M{Then} \M{Else}\\
%   \op{Query}\Q \M{BitSet} \M{Index} \M{Then} \M{Else}
% \end{description}
% \footnotetext[1]{Macros are expandable, full expansion by two steps.}
% \footnotetext[2]{Macros are expandable.}
% \end{minipage}
%
% \subsection{Package loading}
%
% The package can be used as normal \LaTeX\ package:
%\begin{quote}
%|\usepackage{bitset}|
%\end{quote}
% Also \plainTeX\ is supported:
%\begin{quote}
%|\input bitset.sty\relax|
%\end{quote}
%
% \subsection{Operators}
%
% The following macros work on and with bit sets.
% A bit set \meta{BitSet} is represented by a name.
% The should consist of letters and digits. Technically
% it must survive \cs{csname}. It is the same constraint
% that must be satisfied by label or environment names
% in \LaTeX.
%
% However active characters that are shorthands
% of package \xpackage{babel} are not supported.
% Support for shorthands works by an assignment.
% But many operators
% such as \cs{bitsetGetDec} must be usable in expandable
% contexts. There assignments will not be executed in the
% best case or they will cause errors.
%
% The bits in a bit set are addressed by non-negative
% integers starting from zero. Thus negative index
% numbers cause an error message. Because index numbers
% are \TeX\ numbers. The largest index is 2147483647.
% But in practice memory limits and patience limits
% will be very likely reached much before.
%
% \subsubsection{Miscellaneous}\label{sec:misc}
%
% There isn't a separate operation for bit set creation.
% For simplicity an undefined bit set is treated as bit set
% with all bits cleared.
%
% \begin{declcs}{bitsetReset} \M{BitSet}
% \end{declcs}
% Macro \cs{bitsetReset} clears all bits. The result is an
% empty bit set. It may also be used as replacement for
% an operation ``new'', because an undefined bit set is defined
% afterwards.
%
% \begin{declcs}{bitsetLet} \M{BitSet A} \M{BitSet B}
% \end{declcs}
% Macro \cs{bitsetLet} performs a simple assignment similar
% to \TeX's \cs{let}. After the operation \meta{BitSet A} has
% the same value as \meta{BitSet B}. If \meta{BitSet B} is
% undefined, then \meta{BitSet A} will be the empty bit set.
%
% Note: If \meta{BitSet A} exists, it will be overwritten.
%
% \subsubsection{Import}\label{sec:import}
%
% \begin{declcs}{bitsetSetBin} \M{BitSet} \M{BinaryNumber}\\
%   \cs{bitsetSetOct} \M{BitSet} \M{OctalNumber}\\
%   \cs{bitsetSetHex} \M{BitSet} \M{HexadecimalNumber}
% \end{declcs}
% The numbers are interpreted as bit vectors and the flags in
% the bit \meta{BitSet} set are set accordingly.
% These numeric arguments are the only arguments where
% spaces are allowed. Then the numbers are easier to read.
%
% \begin{declcs}{bitsetSetDec} \M{BitSet} \M{DecimalNumber}
% \end{declcs}
% Macro \cs{bitsetSetDec} uses \meta{DecimalNumber} to
% set the bit set \meta{BitSet}.
% The numeric argument must expand
% to a plain number consisting of decimal digits without
% command tokens or spaces. Internally this argument is
% expanded only. It cannot be passed to \cs{number} or
% \cs{numexpr}, because the number may be too large for them.
% However \cs{number} or |\the\numexpr| may be used
% explicitely. This also helps for unexpandable number
% command tokens or registers (\cs{z@}, \cs{@ne}, \cs{count@}, \dots).
% Also \LaTeX'\ \cs{value} needs prefixing:
% \begin{quote}
% |\bitsetSetDec{foo}{\number\value{bar}}|
% \end{quote}
%
% \subsubsection{Export}\label{sec:export}
%
% \begin{declcs}{bitsetGetBin} \M{BitSet} \M{MinSize}\\
%   \cs{bitsetGetOct} \M{BitSet} \M{MinSize}\\
%   \cs{bitsetGetHex} \M{BitSet} \M{MinSize}
% \end{declcs}
% These macros returns the bit set as binary, octal or hexadecimal
% number. If the bit size is smaller than \meta{MinSize} the gap is
% filled with leading zeros.
% Example:
% \begin{quote}
% |\bitsetReset{abc}|\\
% |\bitsetSet{abc}{2}|\\
% |\bitsetGetBin{abc}{8}| $\rightarrow$ |00000100|\\
% |\bitsetSet{abc}{5}\bitsetSet{abc}{7}|\\
% |\bitsetGetHex{abc}{16}| $\rightarrow$ |00A2|
% \end{quote}
% Macro \cs{bitsetGetHex} uses the uppercase letters |A| to |F|.
% The catcode of the letters is one of 11 (letter) or 12 (other).
%
% \begin{declcs}{bitsetGetDec} \M{BitSet}
% \end{declcs}
% Macro \cs{bitsetGetDec} returns the bit set \meta{BitSet} as
% decimal number. The returned number can be larger than
% \TeX's number limit of $2^{31}-1$.
%
% \subsubsection{Logical operators}\label{sec:log}
%
% \begin{declcs}{bitsetAnd} \M{BitSet A} \M{BitSet B}
% \end{declcs}
% $A_{\text{new}} \mathrel{\mathop{:}}= A_{\text{old}}
% \mathbin{\text{and}} B \qquad (\forall\, \text{bits})$
%
% \begin{declcs}{bitsetAndNot} \M{BitSet A} \M{BitSet B}
% \end{declcs}
% $A_{\text{new}} \mathrel{\mathop{:}}= A_{\text{old}}
% \mathbin{\text{and}} (\text{not}\,B) \qquad (\forall\, \text{bits})$
%
% \begin{declcs}{bitsetOr} \M{BitSet A} \M{BitSet B}
% \end{declcs}
% $A_{\text{new}} \mathrel{\mathop{:}}= A_{\text{old}}
% \mathbin{\text{or}} B \qquad (\forall\, \text{bits})$
%
% \begin{declcs}{bitsetXor} \M{BitSet A} \M{BitSet B}
% \end{declcs}
% $A_{\text{new}} \mathrel{\mathop{:}}= A_{\text{old}}
% \mathbin{\text{xor}} B \qquad (\forall\, \text{bits})$
%
% \subsubsection{Shifting}\label{sec:shift}
%
% \begin{declcs}{bitsetShiftLeft} \M{BitSet} \M{ShiftAmount}\\
%   \cs{bitsetShiftRight} \M{BitSet} \M{ShiftAmount}
% \end{declcs}
% A left shift by one is a multiplication by two, thus left shifting
% moves the flags to higher positions. The new created low positions
% are filled by zeros.
%
% A right shift is the opposite, dividing by two, movint the bits
% to lower positions. The number will become smaller, the lowest
% bits are lost.
%
% If the \meta{ShiftAmount} is negative, it reverts the meaning of
% the shift operation. A left shift becomes a right shift.
% A \meta{ShiftAmount} of zero is ignored.
%
% \subsubsection{Bit manipulation}\label{sec:bitman}
%
% \begin{declcs}{bitsetClear} \M{BitSet} \M{Index}\\
%   \cs{bitsetSet} \M{BitSet} \M{Index}\\
%   \cs{bitsetFlip} \M{BitSet} \M{Index}
% \end{declcs}
% This macros manipulate a single bit in \meta{BitSet} addressed
% by \cs{Index}. Macro \cs{bitsetClear} disables the bit,
% \cs{bitsetSet} enables it and \cs{bitsetFlip} reverts the current
% setting of the bit.
%
% \begin{declcs}{bitsetSetValue} \M{BitSet} \M{Index} \M{Bit}
% \end{declcs}
% Macro \cs{bitsetSetValue} puts bit \meta{Bit} at position \meta{Index}
% in bit set \meta{BitSet}. \meta{Bit} must be a valid \TeX\ number
% equals to zero (disabled/cleared) or one (enabled/set).
%
% \subsubsection{Bit retrieval}\label{sec:bitret}
%
% \begin{declcs}{bitsetGet} \M{BitSet} \M{Index}
% \end{declcs}
% Macro \cs{bitsetGet} extracts the status of the bit at
% position \meta{Index} in bit set \meta{BitSet}.
% Digit |1| is returned if the bit is set/enabled. If the
% bit is cleared/disabled and in cases of an undefined bitset or
% an index number out of range the return value is |0|.
%
% \begin{declcs}{bitsetNextClearBit} \M{BitSet} \M{Index}
% \end{declcs}
% Starting at position \meta{Index} (inclusive) the bits
% are inspected. The first position without a set bit is returned.
% Possible results are decimal numbers: \meta{Index}, \meta{Index} + 1,
% \dots, ($\infty$)
%
% \begin{declcs}{bitsetNextSetBit} \M{BitSet} \M{Index}
% \end{declcs}
% Starting at position \meta{Index} (inclusive) the bits are
% inspected and the index position of the first found set bit
% is returned. If there isn't such a bit, then the result is
% |-1|. In summary possible results are decimal numbers:
% |-1|, \meta{Index}, \meta{Index} + 1, \dots, ($\infty$)
%
% \begin{declcs}{bitsetGetSetBitList} \M{BitSet}
% \end{declcs}
% Macro \cs{bitsetGetSetBitList} is an application for
% \cs{bitsetNextSetBit}. The set bits are iterated and
% returned as comma separated list of index positions
% in increasing order. The list is empty in case of
% an empty bit set.
%
% \subsubsection{Bit set properties}\label{sec:prop}
%
% \begin{declcs}{bitsetSize} \M{BitSet}
% \end{declcs}
% Macro \cs{bitsetSize} returns number of bits in use.
% It is the same as the index number of the highest set/enabled bit
% incremented by one.
%
% \begin{declcs}{bitsetCardinality} \M{BitSet}
% \end{declcs}
% Macro \cs{bitsetCardinality} counts the number of set/enabled bits.
%
% \subsubsection{Queries}\label{sec:query}
%
% Also the query procedures are expandable. They ask for a piece
% of information about a bit set and execute code depending
% on the answer.
%
% \begin{declcs}{bitsetIsDefined} \M{BitSet} \M{Then} \M{Else}
% \end{declcs}
% If the bit set with the name \meta{BitSet} exists the code
% given in \meta{Then} is executed, otherwise \meta{Else} is used.
%
% \begin{declcs}{bitsetIsEmpty} \M{BitSet} \M{Then} \M{Else}
% \end{declcs}
% If the bit set \meta{BitSet} exists and at least one bit
% is set/enabled, the code in \meta{Then} is executed,
% \meta{Else} otherwise.
%
% \begin{declcs}{bitsetEquals} \M{BitSet A} \M{BitSet B} \M{Then} \M{Else}
% \end{declcs}
% Both bit sets are equal if and only if either both are undefined
% or both are defined and represents the same bit values at the
% same positions. Thus this definition is reflexive, symmeric,
% and transitive, enough for an equivalent relation.
%
% \begin{declcs}{bitsetIntersects} \M{BitSet A} \M{BitSet B} \M{Then} \M{Else}
% \end{declcs}
% If and only if \meta{BitSet A} and \meta{BitSet B} have
% at least one bit at the same position that is set, then
% code part \meta{Then} is executed.
%
% \begin{declcs}{bitsetQuery} \M{BitSet} \M{Index} \M{Then} \M{Else}
% \end{declcs}
% It's just a wrapper for \cs{bitsetGet}. If the bit at position
% \meta{Index} is enabled, code \meta{Then} is called.
%
% \StopEventually{
% }
%
% \section{Implementation}
%
% The internal format of a bit set is quite simple, a sequence
% of digits |0| and |1|. The least significant bit is left.
% A bit set without any flag set is encoded by |0|. Also
% undefined bit sets are treated that way. After the
% highest bit that is set there are no further zeroes.
% A regular expression of valid bit sets values:
% \begin{quote}
%  |0|\string||[01]*1|
% \end{quote}
%
%    \begin{macrocode}
%<*package>
%    \end{macrocode}
%
% \subsection{Reload check and package identification}
%    Reload check, especially if the package is not used with \LaTeX.
%    \begin{macrocode}
\begingroup\catcode61\catcode48\catcode32=10\relax%
  \catcode13=5 % ^^M
  \endlinechar=13 %
  \catcode35=6 % #
  \catcode39=12 % '
  \catcode44=12 % ,
  \catcode45=12 % -
  \catcode46=12 % .
  \catcode58=12 % :
  \catcode64=11 % @
  \catcode123=1 % {
  \catcode125=2 % }
  \expandafter\let\expandafter\x\csname ver@bitset.sty\endcsname
  \ifx\x\relax % plain-TeX, first loading
  \else
    \def\empty{}%
    \ifx\x\empty % LaTeX, first loading,
      % variable is initialized, but \ProvidesPackage not yet seen
    \else
      \expandafter\ifx\csname PackageInfo\endcsname\relax
        \def\x#1#2{%
          \immediate\write-1{Package #1 Info: #2.}%
        }%
      \else
        \def\x#1#2{\PackageInfo{#1}{#2, stopped}}%
      \fi
      \x{bitset}{The package is already loaded}%
      \aftergroup\endinput
    \fi
  \fi
\endgroup%
%    \end{macrocode}
%    Package identification:
%    \begin{macrocode}
\begingroup\catcode61\catcode48\catcode32=10\relax%
  \catcode13=5 % ^^M
  \endlinechar=13 %
  \catcode35=6 % #
  \catcode39=12 % '
  \catcode40=12 % (
  \catcode41=12 % )
  \catcode44=12 % ,
  \catcode45=12 % -
  \catcode46=12 % .
  \catcode47=12 % /
  \catcode58=12 % :
  \catcode64=11 % @
  \catcode91=12 % [
  \catcode93=12 % ]
  \catcode123=1 % {
  \catcode125=2 % }
  \expandafter\ifx\csname ProvidesPackage\endcsname\relax
    \def\x#1#2#3[#4]{\endgroup
      \immediate\write-1{Package: #3 #4}%
      \xdef#1{#4}%
    }%
  \else
    \def\x#1#2[#3]{\endgroup
      #2[{#3}]%
      \ifx#1\@undefined
        \xdef#1{#3}%
      \fi
      \ifx#1\relax
        \xdef#1{#3}%
      \fi
    }%
  \fi
\expandafter\x\csname ver@bitset.sty\endcsname
\ProvidesPackage{bitset}%
  [2011/01/30 v1.1 Handle bit-vector datatype (HO)]%
%    \end{macrocode}
%
% \subsection{Catcodes}
%
%    \begin{macrocode}
\begingroup\catcode61\catcode48\catcode32=10\relax%
  \catcode13=5 % ^^M
  \endlinechar=13 %
  \catcode123=1 % {
  \catcode125=2 % }
  \catcode64=11 % @
  \def\x{\endgroup
    \expandafter\edef\csname BitSet@AtEnd\endcsname{%
      \endlinechar=\the\endlinechar\relax
      \catcode13=\the\catcode13\relax
      \catcode32=\the\catcode32\relax
      \catcode35=\the\catcode35\relax
      \catcode61=\the\catcode61\relax
      \catcode64=\the\catcode64\relax
      \catcode123=\the\catcode123\relax
      \catcode125=\the\catcode125\relax
    }%
  }%
\x\catcode61\catcode48\catcode32=10\relax%
\catcode13=5 % ^^M
\endlinechar=13 %
\catcode35=6 % #
\catcode64=11 % @
\catcode123=1 % {
\catcode125=2 % }
\def\TMP@EnsureCode#1#2{%
  \edef\BitSet@AtEnd{%
    \BitSet@AtEnd
    \catcode#1=\the\catcode#1\relax
  }%
  \catcode#1=#2\relax
}
\TMP@EnsureCode{33}{12}% !
\TMP@EnsureCode{39}{12}% '
\TMP@EnsureCode{40}{12}% (
\TMP@EnsureCode{41}{12}% )
\TMP@EnsureCode{42}{12}% *
\TMP@EnsureCode{43}{12}% +
\TMP@EnsureCode{44}{12}% ,
\TMP@EnsureCode{45}{12}% -
\TMP@EnsureCode{46}{12}% .
\TMP@EnsureCode{47}{12}% /
\TMP@EnsureCode{58}{11}% : (letter!)
\TMP@EnsureCode{60}{12}% <
\TMP@EnsureCode{62}{12}% >
\TMP@EnsureCode{63}{14}% ? (comment!)
\TMP@EnsureCode{91}{12}% [
\TMP@EnsureCode{93}{12}% ]
\TMP@EnsureCode{96}{12}% `
\edef\BitSet@AtEnd{\BitSet@AtEnd\noexpand\endinput}
\begingroup\expandafter\expandafter\expandafter\endgroup
\expandafter\ifx\csname BitSet@TestMode\endcsname\relax
\else
  \catcode63=9 % ? (ignore)
\fi
? \let\BitSet@@TestMode\BitSet@TestMode
%    \end{macrocode}
%
% \subsection{Package loading}
%
%    \begin{macrocode}
\begingroup\expandafter\expandafter\expandafter\endgroup
\expandafter\ifx\csname RequirePackage\endcsname\relax
  \def\TMP@RequirePackage#1[#2]{%
    \begingroup\expandafter\expandafter\expandafter\endgroup
    \expandafter\ifx\csname ver@#1.sty\endcsname\relax
      \input #1.sty\relax
    \fi
  }%
  \TMP@RequirePackage{infwarerr}[2007/09/09]%
  \TMP@RequirePackage{intcalc}[2007/09/27]%
  \TMP@RequirePackage{bigintcalc}[2007/09/27]%
\else
  \RequirePackage{infwarerr}[2007/09/09]%
  \RequirePackage{intcalc}[2007/09/27]%
  \RequirePackage{bigintcalc}[2007/09/27]%
\fi
%    \end{macrocode}
%
% \subsection{Help macros}
%
% \subsubsection{Number constant}
%
%    \begin{macro}{\BitSet@MaxSize}
%    \begin{macrocode}
\def\BitSet@MaxSize{2147483647}%
%    \end{macrocode}
%    \end{macro}
%
% \subsubsection{General basic macros}
%
%    \begin{macro}{\BitSet@Empty}
%    \begin{macrocode}
\def\BitSet@Empty{}
%    \end{macrocode}
%    \end{macro}
%    \begin{macro}{\BitSet@FirstOfOne}
%    \begin{macrocode}
\def\BitSet@FirstOfOne#1{#1}
%    \end{macrocode}
%    \end{macro}
%    \begin{macro}{\BitSet@Gobble}
%    \begin{macrocode}
\def\BitSet@Gobble#1{}
%    \end{macrocode}
%    \end{macro}
%    \begin{macro}{\BitSet@FirstOfTwo}
%    \begin{macrocode}
\def\BitSet@FirstOfTwo#1#2{#1}
%    \end{macrocode}
%    \end{macro}
%    \begin{macro}{\BitSet@SecondOfTwo}
%    \begin{macrocode}
\def\BitSet@SecondOfTwo#1#2{#2}
%    \end{macrocode}
%    \end{macro}
%
%    \begin{macro}{\BitSet@Space}
%    \begin{macrocode}
\def\BitSet@Space{ }
%    \end{macrocode}
%    \end{macro}
%    \begin{macro}{\BitSet@ZapSpace}
%    \begin{macrocode}
\def\BitSet@ZapSpace#1 #2{%
  #1%
  \ifx\BitSet@Empty#2%
  \else
    \expandafter\BitSet@ZapSpace
  \fi
  #2%
}
%    \end{macrocode}
%    \end{macro}
%
% \subsubsection{Tail recursion}
%
%    \begin{macro}{\BitSet@Fi}
%    \begin{macrocode}
\let\BitSet@Fi\fi
%    \end{macrocode}
%    \end{macro}
%    \begin{macro}{\BitSet@AfterFi}
%    \begin{macrocode}
\def\BitSet@AfterFi#1#2\BitSet@Fi{\fi#1}
%    \end{macrocode}
%    \end{macro}
%    \begin{macro}{\BitSet@AfterFiFi}
%    \begin{macrocode}
\def\BitSet@AfterFiFi#1#2\BitSet@Fi{\fi\fi#1}%
%    \end{macrocode}
%    \end{macro}
%    \begin{macro}{\BitSet@AfterFiFiFi}
%    \begin{macrocode}
\def\BitSet@AfterFiFiFi#1#2\BitSet@Fi{\fi\fi\fi#1}%
%    \end{macrocode}
%    \end{macro}
%
% \subsubsection{Check macros}
%
%    \begin{macro}{\BitSet@IfUndefined}
%    \begin{macrocode}
\def\BitSet@IfUndefined#1{%
  \expandafter\ifx\csname BS@#1\endcsname\relax
    \expandafter\BitSet@FirstOfTwo
  \else
    \expandafter\BitSet@SecondOfTwo
  \fi
}
%    \end{macrocode}
%    \end{macro}
%
%    \begin{macro}{\BitSet@CheckIndex}
%    |#1|: continuation code\\
%    |#2|: BitSet\\
%    |#3|: Index
%    \begin{macrocode}
\def\BitSet@CheckIndex#1#2#3{%
  \BitSet@IfUndefined{#2}{\bitsetReset{#2}}{}%
  \expandafter\expandafter\expandafter\BitSet@@CheckIndex
  \intcalcNum{#3}!%
  {#2}{#1}%
}
%    \end{macrocode}
%    \end{macro}
%    \begin{macro}{\BitSet@@CheckIndex}
%    |#1|: plain Index\\
%    |#2|: BitSet\\
%    |#3|: continuation code
%    \begin{macrocode}
\def\BitSet@@CheckIndex#1!#2#3{%
  \ifnum#1<0 %
    \BitSet@AfterFi{%
      \@PackageError{bitset}{%
        Invalid negative index (#1)%
      }\@ehc
    }%
  \else
    \BitSet@AfterFi{%
      #3{#2}{#1}%
    }%
  \BitSet@Fi
}
%    \end{macrocode}
%    \end{macro}
%
% \subsection{Miscellaneous}\label{src:misc}
%
%    \begin{macro}{\bitsetReset}
%    \begin{macrocode}
\def\bitsetReset#1{%
  \expandafter\def\csname BS@#1\endcsname{0}%
}
%    \end{macrocode}
%    \end{macro}
%
%    \begin{macro}{\bitsetLet}
%    \begin{macrocode}
\def\bitsetLet#1#2{%
  \BitSet@IfUndefined{#2}{%
    \bitsetReset{#1}%
  }{%
    \expandafter\let\csname BS@#1\expandafter\endcsname
                    \csname BS@#2\endcsname
  }%
}
%    \end{macrocode}
%    \end{macro}
%
% \subsection{Import}\label{src:import}
%
% \subsubsection{From binary number}
%
%    \begin{macro}{\bitsetSetBin}
%    \begin{macrocode}
\def\bitsetSetBin#1#2{%
  \edef\BitSet@Temp{#2}%
  \edef\BitSet@Temp{%
    \expandafter\expandafter\expandafter\BitSet@ZapSpace
    \expandafter\BitSet@Temp\BitSet@Space\BitSet@Empty
  }%
  \edef\BitSet@Temp{%
    \expandafter\BitSet@KillZeros\BitSet@Temp\BitSet@Empty
  }%
  \ifx\BitSet@Temp\BitSet@Empty
    \expandafter\let\csname BS@#1\endcsname\BitSet@Zero
  \else
    \expandafter\edef\csname BS@#1\endcsname{%
      \expandafter\BitSet@Reverse\BitSet@Temp!%
    }%
  \fi
}
%    \end{macrocode}
%    \end{macro}
%    \begin{macro}{\BitSet@KillZeros}
%    \begin{macrocode}
\def\BitSet@KillZeros#1{%
  \ifx#10%
    \expandafter\BitSet@KillZeros
  \else
    #1%
  \fi
}
%    \end{macrocode}
%    \end{macro}
%    \begin{macro}{\BitSet@Reverse}
%    \begin{macrocode}
\def\BitSet@Reverse#1#2!{%
  \ifx\\#2\\%
    #1%
  \else
    \BitSet@AfterFi{%
      \BitSet@Reverse#2!#1%
    }%
  \BitSet@Fi
}
%    \end{macrocode}
%    \end{macro}
%
% \subsubsection{From octal/hex number}
%
%    \begin{macro}{\bitsetSetOct}
%    \begin{macrocode}
\def\bitsetSetOct{%
  \BitSet@SetOctHex\BitSet@FromFirstOct
}
%    \end{macrocode}
%    \end{macro}
%    \begin{macro}{\bitsetSetHex}
%    \begin{macrocode}
\def\bitsetSetHex{%
  \BitSet@SetOctHex\BitSet@FromFirstHex
}
%    \end{macrocode}
%    \end{macro}
%    \begin{macro}{\BitSet@SetOctHex}
%    \begin{macrocode}
\def\BitSet@SetOctHex#1#2#3{%
  \edef\BitSet@Temp{#3}%
  \edef\BitSet@Temp{%
    \expandafter\expandafter\expandafter\BitSet@ZapSpace
    \expandafter\BitSet@Temp\BitSet@Space\BitSet@Empty
  }%
  \edef\BitSet@Temp{%
    \expandafter\BitSet@KillZeros\BitSet@Temp\BitSet@Empty
  }%
  \ifx\BitSet@Temp\BitSet@Empty
    \expandafter\let\csname BS@#2\endcsname\BitSet@Zero
  \else
    \edef\BitSet@Temp{%
      \expandafter#1\BitSet@Temp!%
    }%
    \ifx\BitSet@Temp\BitSet@Empty
      \expandafter\let\csname BS@#2\endcsname\BitSet@Zero
    \else
      \expandafter\edef\csname BS@#2\endcsname{%
        \expandafter\BitSet@Reverse\BitSet@Temp!%
      }%
    \fi
  \fi
}
%    \end{macrocode}
%    \end{macro}
%    \begin{macro}{\BitSet@FromFirstOct}
%    \begin{macrocode}
\def\BitSet@FromFirstOct#1{%
  \ifx#1!%
  \else
    \ifcase#1 \BitSet@AfterFiFi\BitSet@FromFirstOct
    \or 1%
    \or 10%
    \or 11%
    \or 100%
    \or 101%
    \or 110%
    \or 111%
    \else \BitSetError:WrongOctalDigit%
    \fi
    \expandafter\BitSet@FromOct
  \BitSet@Fi
}
%    \end{macrocode}
%    \end{macro}
%    \begin{macro}{\BitSet@FromOct}
%    \begin{macrocode}
\def\BitSet@FromOct#1{%
  \ifx#1!%
  \else
    \ifcase#1 000%
    \or 001%
    \or 010%
    \or 011%
    \or 100%
    \or 101%
    \or 110%
    \or 111%
    \else \BitSetError:WrongOctalDigit%
    \fi
    \expandafter\BitSet@FromOct
  \fi
}
%    \end{macrocode}
%    \end{macro}
%
%    \begin{macro}{\BitSet@FromFirstHex}
%    \begin{macrocode}
\def\BitSet@FromFirstHex#1{%
  \ifx#1!%
  \else
    \ifx#10%
      \BitSet@AfterFiFi\BitSet@FromFirstHex
    \fi
    \expandafter\ifx\csname BitSet@Hex#1\endcsname\relax
      \BitSetError:InvalidHexDigit%
    \else
      \expandafter\expandafter\expandafter\BitSet@KillZeros
      \csname BitSet@Hex#1\endcsname
    \fi
    \expandafter\BitSet@FromHex
  \BitSet@Fi
}
%    \end{macrocode}
%    \end{macro}
%    \begin{macro}{\BitSet@FromHex}
%    \begin{macrocode}
\def\BitSet@FromHex#1{%
  \ifx#1!%
  \else
    \expandafter\ifx\csname BitSet@Hex#1\endcsname\relax
      \BitSetError:InvalidHexDigit%
    \else
      \csname BitSet@Hex#1\endcsname
    \fi
    \expandafter\BitSet@FromHex
  \fi
}
%    \end{macrocode}
%    \end{macro}
%    \begin{macro}{\BitSet@Hex[0..F]}
%    \begin{macrocode}
\def\BitSet@Temp#1{%
  \expandafter\def\csname BitSet@Hex#1\endcsname
}
\BitSet@Temp 0{0000}%
\BitSet@Temp 1{0001}%
\BitSet@Temp 2{0010}%
\BitSet@Temp 3{0011}%
\BitSet@Temp 4{0100}%
\BitSet@Temp 5{0101}%
\BitSet@Temp 6{0110}%
\BitSet@Temp 7{0111}%
\BitSet@Temp 8{1000}%
\BitSet@Temp 9{1001}%
\BitSet@Temp A{1010}%
\BitSet@Temp B{1011}%
\BitSet@Temp C{1100}%
\BitSet@Temp D{1101}%
\BitSet@Temp E{1110}%
\BitSet@Temp F{1111}%
\BitSet@Temp a{1010}%
\BitSet@Temp b{1011}%
\BitSet@Temp c{1100}%
\BitSet@Temp d{1101}%
\BitSet@Temp e{1110}%
\BitSet@Temp f{1111}%
%    \end{macrocode}
%    \end{macro}
%
% \subsubsection{From decimal number}
%
%    \begin{macro}{\bitsetSetDec}
%    \begin{macrocode}
\def\bitsetSetDec#1#2{%
  \edef\BitSet@Temp{#2}%
  \edef\BitSet@Temp{%
    \expandafter\expandafter\expandafter\BitSet@ZapSpace
    \expandafter\BitSet@Temp\BitSet@Space\BitSet@Empty
  }%
  \edef\BitSet@Temp{%
    \expandafter\BitSet@KillZeros\BitSet@Temp\BitSet@Empty
  }%
  \ifx\BitSet@Temp\BitSet@Empty
    \expandafter\let\csname BS@#1\endcsname\BitSet@Zero
  \else
    \ifcase\bigintcalcSgn{\BitSet@Temp} %
      \expandafter\let\csname BS@#1\endcsname\BitSet@Zero
    \or
      \ifnum\bigintcalcCmp\BitSet@Temp\BitSet@MaxSize>0 %
        \expandafter\edef\csname BS@#1\endcsname{%
          \expandafter\BitSet@SetDecBig\BitSet@Temp!%
        }%
      \else
        \expandafter\edef\csname BS@#1\endcsname{%
          \expandafter\BitSet@SetDec\BitSet@Temp!%
        }%
      \fi
    \else
      \@PackageError{bitset}{%
        Bit sets cannot be negative%
      }\@ehc
    \fi
  \fi
}
%    \end{macrocode}
%    \end{macro}
%    \begin{macro}{\BitSet@SetDecBig}
%    \begin{macrocode}
\def\BitSet@SetDecBig#1#2#3#4#5#6#7#8#9!{%
  \ifx\\#9\\%
    \BitSet@SetDec#1#2#3#4#5#6#7#8!%
  \else
    \ifcase\BigIntCalcOdd#1#2#4#5#6#7#8#9! %
      0%
    \or
      1%
?   \else\BitSetError:ThisCannotHappen%
    \fi
    \BitSet@AfterFi{%
      \expandafter\expandafter\expandafter\BitSet@SetDecBig
      \BigIntCalcShr#1#2#3#4#5#6#7#8#9!!%
    }%
  \BitSet@Fi
}
%    \end{macrocode}
%    \end{macro}
%    \begin{macro}{\BitSet@SetDec}
%    \begin{macrocode}
\def\BitSet@SetDec#1!{%
  \ifcase#1 %
  \or 1%
  \else
    \ifodd#1 %
      1%
    \else
      0%
    \fi
    \BitSet@AfterFi{%
      \expandafter\expandafter\expandafter\BitSet@SetDec
      \IntCalcShr#1!!%
    }%
  \BitSet@Fi
}
%    \end{macrocode}
%    \end{macro}
%
% \subsection{Export}\label{src:export}
%
% \subsubsection{To binary number}
%
%    \begin{macro}{\bitsetGetBin}
%    \begin{macrocode}
\def\bitsetGetBin#1#2{%
  \romannumeral0%
  \expandafter\expandafter\expandafter\BitSet@@GetBin
  \intcalcNum{#2}!{#1}%
}
%    \end{macrocode}
%    \end{macro}
%    \begin{macro}{\BitSet@@GetBin}
%    \begin{macrocode}
\def\BitSet@@GetBin#1!#2{%
  \BitSet@IfUndefined{#2}{%
    \ifnum#1>1 %
      \BitSet@AfterFi{%
        \expandafter\expandafter\expandafter\BitSet@Fill
        \IntCalcDec#1!!0%
      }%
    \else
      \BitSet@AfterFi{ 0}%
    \BitSet@Fi
  }{%
    \expandafter\expandafter\expandafter\BitSet@NumBinRev
    \expandafter\expandafter\expandafter1%
    \expandafter\expandafter\expandafter!%
    \csname BS@#2\endcsname!!#1!%
  }%
}
%    \end{macrocode}
%    \end{macro}
%    \begin{macro}{\BitSet@Fill}
%    |#1|: number of leading digits |0|\\
%    |#2|: result
%    \begin{macrocode}
\def\BitSet@Fill#1!{%
  \ifnum#1>0 %
    \BitSet@AfterFi{%
      \expandafter\expandafter\expandafter\BitSet@Fill
      \IntCalcDec#1!!0%
    }%
  \else
    \BitSet@AfterFi{ }%
  \BitSet@Fi
}
%    \end{macrocode}
%    \end{macro}
%
%    \begin{macro}{\BitSet@NumBinRev}
%    |#1|: bit counter (including |#2|)\\
%    |#2#3|: reverted number\\
%    |#4|: result\\
%    |#5|: min size
%    \begin{macrocode}
\def\BitSet@NumBinRev#1!#2#3!{%
  \ifx\\#3\\%
    \BitSet@AfterFi{%
      \BitSet@NumBinFill#1!#2%
    }%
  \else
    \BitSet@AfterFi{%
      \expandafter\expandafter\expandafter\BitSet@NumBinRev
      \IntCalcInc#1!!#3!#2%
    }%
  \BitSet@Fi
}
%    \end{macrocode}
%    \end{macro}
%    \begin{macro}{\BitSet@NumBinFill}
%    \begin{macrocode}
\def\BitSet@NumBinFill#1!#2!#3!{%
  \ifnum#3>#1 %
    \BitSet@AfterFi{%
      \expandafter\expandafter\expandafter\BitSet@Fill
      \IntCalcSub#3!#1!!#2%
    }%
  \else
    \BitSet@AfterFi{ #2}%
  \BitSet@Fi
}
%    \end{macrocode}
%    \end{macro}
%
% \subsubsection{To octal/hexadecimal number}
%
%    \begin{macro}{\bitsetGetOct}
%    \begin{macrocode}
\def\bitsetGetOct#1#2{%
  \romannumeral0%
  \bitsetIsEmpty{#1}{%
    \expandafter\expandafter\expandafter\BitSet@@GetOctHex
    \intcalcNum{#2}!3!230%
  }{%
    \expandafter\expandafter\expandafter\BitSet@@GetOct
    \expandafter\expandafter\expandafter1%
    \expandafter\expandafter\expandafter!%
    \expandafter\expandafter\expandafter!%
    \csname BS@#1\endcsname00%
    \BitSet@Empty\BitSet@Empty\BitSet@Empty!{#2}%
  }%
}
%    \end{macrocode}
%    \end{macro}
%    \begin{macro}{\bitsetGetHex}
%    \begin{macrocode}
\def\bitsetGetHex#1#2{%
  \romannumeral0%
  \bitsetIsEmpty{#1}{%
    \expandafter\expandafter\expandafter\BitSet@@GetOctHex
    \intcalcNum{#2}!4!340%
  }{%
    \expandafter\expandafter\expandafter\BitSet@@GetHex
    \expandafter\expandafter\expandafter1%
    \expandafter\expandafter\expandafter!%
    \expandafter\expandafter\expandafter!%
    \csname BS@#1\endcsname000%
    \BitSet@Empty\BitSet@Empty\BitSet@Empty\BitSet@Empty!{#2}%
  }%
}
%    \end{macrocode}
%    \end{macro}
%    \begin{macro}{\BitSet@@GetOct}
%    |#1|: number of digits\\
%    |#2|: result\\
%    |#3#4#5|: bits
%    \begin{macrocode}
\def\BitSet@@GetOct#1!#2!#3#4#5{%
  \ifx#5\BitSet@Empty
    \BitSet@AfterFi{%
      \expandafter\expandafter\expandafter\BitSet@GetOctHex
      \IntCalcDec#1!!#2!23%
    }%
  \else
    \BitSet@AfterFi{%
      \expandafter\expandafter\expandafter\BitSet@@GetOct
      \number\IntCalcInc#1!\expandafter\expandafter\expandafter!%
      \csname BitSet@Oct#5#4#3\endcsname#2!%
    }%
  \BitSet@Fi
}
%    \end{macrocode}
%    \end{macro}
%    \begin{macro}{\BitSet@Oct[000..111]}
%    \begin{macrocode}
\def\BitSet@Temp#1#2#3#4{%
  \expandafter\def\csname BitSet@Oct#1#2#3\endcsname{#4}%
}
\BitSet@Temp0000%
\BitSet@Temp0011%
\BitSet@Temp0102%
\BitSet@Temp0113%
\BitSet@Temp1004%
\BitSet@Temp1015%
\BitSet@Temp1106%
\BitSet@Temp1117%
%    \end{macrocode}
%    \end{macro}
%    \begin{macro}{\BitSet@@GetHex}
%    |#1|: number of digits\\
%    |#2|: result\\
%    |#3#4#5#6|: bits
%    \begin{macrocode}
\def\BitSet@@GetHex#1!#2!#3#4#5#6{%
  \ifx#6\BitSet@Empty
    \BitSet@AfterFi{%
      \expandafter\expandafter\expandafter\BitSet@GetOctHex
      \IntCalcDec#1!!#2!34%
    }%
  \else
    \BitSet@AfterFi{%
      \expandafter\expandafter\expandafter\BitSet@@GetHex
      \number\IntCalcInc#1!\expandafter\expandafter\expandafter!%
      \csname BitSet@Hex#6#5#4#3\endcsname#2!%
    }%
  \BitSet@Fi
}
%    \end{macrocode}
%    \end{macro}
%    \begin{macro}{\BitSet@Hex[0000..1111]}
%    \begin{macrocode}
\def\BitSet@Temp#1#2#3#4#5{%
  \expandafter\def\csname BitSet@Hex#1#2#3#4\endcsname{#5}%
}
\BitSet@Temp00000%
\BitSet@Temp00011%
\BitSet@Temp00102%
\BitSet@Temp00113%
\BitSet@Temp01004%
\BitSet@Temp01015%
\BitSet@Temp01106%
\BitSet@Temp01117%
\BitSet@Temp10008%
\BitSet@Temp10019%
\BitSet@Temp1010A%
\BitSet@Temp1011B%
\BitSet@Temp1100C%
\BitSet@Temp1101D%
\BitSet@Temp1110E%
\BitSet@Temp1111F%
%    \end{macrocode}
%    \end{macro}
%    \begin{macro}{\BitSet@GetOctHex}
%    Leading zeros $ (|#4| - |#1| * 3 + 2)/3 $ if $ |#4| > |#1| * 3$\\
%    |#1|: digit size\\
%    |#2|: result\\
%    |#3|: bits per digit - 1\\
%    |#4|: bits per digit
%    |#5|: garbage\\
%    |#6|: min size
%    \begin{macrocode}
\def\BitSet@GetOctHex#1!#2!#3#4#5!#6{%
  \expandafter\BitSet@@GetOctHex
  \number\intcalcNum{#6}\expandafter\expandafter\expandafter!%
  \IntCalcMul#1!#4!!#3#4#2%
}
%    \end{macrocode}
%    \end{macro}
%    \begin{macro}{\BitSet@@GetOctHex}
%    |#1|: plain min size\\
%    |#2|: digits * (bits per digit)\\
%    |#3|: bits per digit - 1\\
%    |#4|: bits per digit
%    \begin{macrocode}
\def\BitSet@@GetOctHex#1!#2!#3#4{%
  \ifnum#1>#2 %
    \BitSet@AfterFi{%
      \expandafter\expandafter\expandafter\expandafter
      \expandafter\expandafter\expandafter\BitSet@Fill
      \expandafter\IntCalcDiv\number
      \expandafter\expandafter\expandafter\IntCalcAdd
      \IntCalcSub#1!#2!!#3!!#4!!%
    }%
  \else
    \BitSet@AfterFi{ }%
  \BitSet@Fi
}
%    \end{macrocode}
%    \end{macro}
%
% \subsubsection{To decimal number}
%
%    \begin{macro}{\bitsetGetDec}
%    \begin{macrocode}
\def\bitsetGetDec#1{%
  \romannumeral0%
  \BitSet@IfUndefined{#1}{ 0}{%
    \expandafter\expandafter\expandafter\BitSet@GetDec
    \csname BS@#1\endcsname!%
  }%
}
%    \end{macrocode}
%    \end{macro}
%    \begin{macro}{\BitSet@GetDec}
%    \begin{macrocode}
\def\BitSet@GetDec#1#2!{%
  \ifx\\#2\\%
    \BitSet@AfterFi{ #1}%
  \else
    \BitSet@AfterFi{%
      \BitSet@@GetDec2!#1!#2!%
    }%
  \BitSet@Fi
}
%    \end{macrocode}
%    \end{macro}
%    \begin{macro}{\BitSet@@GetDec}
%    |#1|: power of two\\
%    |#2|: result\\
%    |#3#4|: number
%    \begin{macrocode}
\def\BitSet@@GetDec#1!#2!#3#4!{%
  \ifx\\#4\\%
    \ifx#31%
      \BitSet@AfterFiFi{%
        \expandafter\expandafter\expandafter\BitSet@Space
        \IntCalcAdd#1!#2!%
      }%
    \else
      \BitSet@AfterFiFi{ #2}%
    \fi
  \else
    \ifx#31%
      \BitSet@AfterFiFi{%
        \csname BitSet@N#1%
        \expandafter\expandafter\expandafter\endcsname
        \IntCalcAdd#1!#2!!#4!%
      }%
    \else
      \BitSet@AfterFiFi{%
        \csname BitSet@N#1\endcsname#2!#4!%
      }%
    \fi
  \BitSet@Fi
}
%    \end{macrocode}
%    \end{macro}
%    \begin{macro}{\BitSet@N[1,2,4,...]}
%    \begin{macrocode}
\def\BitSet@Temp#1#2{%
  \expandafter\def\csname BitSet@N#1\endcsname{%
    \BitSet@@GetDec#2!%
  }%
}
\BitSet@Temp{1}{2}
\BitSet@Temp{2}{4}
\BitSet@Temp{4}{8}
\BitSet@Temp{8}{16}
\BitSet@Temp{16}{32}
\BitSet@Temp{32}{64}
\BitSet@Temp{64}{128}
\BitSet@Temp{128}{256}
\BitSet@Temp{256}{512}
\BitSet@Temp{512}{1024}
\BitSet@Temp{1024}{2048}
\BitSet@Temp{2048}{4096}
\BitSet@Temp{4096}{8192}
\BitSet@Temp{8192}{16384}
\BitSet@Temp{16384}{32768}
\BitSet@Temp{32768}{65536}
\BitSet@Temp{65536}{131072}
\BitSet@Temp{131072}{262144}
\BitSet@Temp{262144}{524288}
\BitSet@Temp{524288}{1048576}
\BitSet@Temp{1048576}{2097152}
\BitSet@Temp{2097152}{4194304}
\BitSet@Temp{4194304}{8388608}
\BitSet@Temp{8388608}{16777216}
\BitSet@Temp{16777216}{33554432}
\BitSet@Temp{33554432}{67108864}
\BitSet@Temp{67108864}{134217728}
\BitSet@Temp{134217728}{268435456}
\BitSet@Temp{268435456}{536870912}
\BitSet@Temp{536870912}{1073741824}
%    \end{macrocode}
%    \end{macro}
%    \begin{macro}{\BitSet@N1073741824}
%    \begin{macrocode}
\expandafter\def\csname BitSet@N1073741824\endcsname{%
  \BitSet@GetDecBig2147483648!%
}%
%    \end{macrocode}
%    \end{macro}
%
%    \begin{macro}{\BitSet@GetDecBig}
%    |#1|: current power of two\\
%    |#2|: result\\
%    |#3#4|: number
%    \begin{macrocode}
\def\BitSet@GetDecBig#1!#2!#3#4!{%
  \ifx\\#4\\%
    \ifx#31%
      \BitSet@AfterFiFi{%
        \expandafter\expandafter\expandafter\BitSet@Space
        \BigIntCalcAdd#1!#2!%
      }%
    \else
      \BitSet@AfterFiFi{ #2}%
    \fi
  \else
    \ifx#31%
      \BitSet@AfterFiFi{%
        \expandafter\expandafter\expandafter\BitSet@@GetDecBig
        \BigIntCalcAdd#1!#2!!#1!#4!%
      }%
    \else
      \BitSet@AfterFiFi{%
        \expandafter\expandafter\expandafter\BitSet@GetDecBig
        \BigIntCalcShl#1!!#2!#4!%
      }%
    \fi
  \BitSet@Fi
}
%    \end{macrocode}
%    \end{macro}
%    \begin{macro}{\BitSet@@GetDecBig}
%    |#1|: result\\
%    |#2|: power of two\\
%    |#3#4|: number
%    \begin{macrocode}
\def\BitSet@@GetDecBig#1!#2!{%
  \expandafter\expandafter\expandafter\BitSet@GetDecBig
  \BigIntCalcShl#2!!#1!%
}
%    \end{macrocode}
%    \end{macro}
%
% \subsection{Logical operators}\label{src:log}
%
% \subsubsection{\cs{bitsetAnd}}
%
%    \begin{macro}{\bitsetAnd}
%    Decision table for \cs{bitsetAnd}:
%    \begin{quote}
%    \begin{tabular}{l|l|l|l|}
%                         & undef(B)   & empty(B)   & cardinality(B)$>$0\\
%      \hline
%      undef(A)           & A := empty & A := empty & A := empty\\
%      \hline
%      empty(A)           &            &            &\\
%      \hline
%      cardinality(A)$>$0 & A := empty & A := empty & A \&= B\\
%      \hline
%    \end{tabular}
%    \end{quote}
%    \begin{macrocode}
\def\bitsetAnd#1#2{%
  \bitsetIsEmpty{#1}{%
    \bitsetReset{#1}%
  }{%
    \bitsetIsEmpty{#2}{%
      \bitsetReset{#1}%
    }{%
      \expandafter\edef\csname BS@#1\endcsname{%
        \expandafter\expandafter\expandafter\BitSet@And
        \csname BS@#1\expandafter\expandafter\expandafter\endcsname
        \expandafter\expandafter\expandafter!%
        \csname BS@#2\endcsname!!%
      }%
      \expandafter\ifx\csname BS@#1\endcsname\BitSet@Empty
        \bitsetReset{#1}%
      \fi
    }%
  }%
}
%    \end{macrocode}
%    \end{macro}
%    \begin{macro}{\BitSet@And}
%    \begin{macrocode}
\def\BitSet@And#1#2!#3#4!#5!{%
  \ifx\\#2\\%
    \ifnum#1#3=11 #51\fi
  \else
    \ifx\\#4\\%
      \ifnum#1#3=11 #51\fi
    \else
      \ifnum#1#3=11 %
        #51%
        \BitSet@AfterFiFiFi{%
          \BitSet@And#2!#4!!%
        }%
      \else
        \BitSet@AfterFiFiFi{%
          \BitSet@And#2!#4!#50!%
        }%
      \fi
    \fi
  \BitSet@Fi
}
%    \end{macrocode}
%    \end{macro}
%
% \subsubsection{\cs{bitsetAndNot}}
%
%    \begin{macro}{\bitsetAndNot}
%    Decision table for \cs{bitsetAndNot}:
%    \begin{quote}
%    \begin{tabular}{l|l|l|l|}
%                         & undef(B)   & empty(B)   & cardinality(B)$>$0\\
%      \hline
%      undef(A)           & A := empty & A := empty & A := empty\\
%      \hline
%      empty(A)           &            &            &\\
%      \hline
%      cardinality(A)$>$0 &            &            & A \&= !B\\
%      \hline
%    \end{tabular}
%    \end{quote}
%    \begin{macrocode}
\def\bitsetAndNot#1#2{%
  \bitsetIsEmpty{#1}{%
    \bitsetReset{#1}%
  }{%
    \bitsetIsEmpty{#2}{%
    }{%
      \expandafter\edef\csname BS@#1\endcsname{%
        \expandafter\expandafter\expandafter\BitSet@AndNot
        \csname BS@#1\expandafter\expandafter\expandafter\endcsname
        \expandafter\expandafter\expandafter!%
        \csname BS@#2\endcsname!!%
      }%
      \expandafter\ifx\csname BS@#1\endcsname\BitSet@Empty
        \bitsetReset{#1}%
      \fi
    }%
  }%
}
%    \end{macrocode}
%    \end{macro}
%    \begin{macro}{\BitSet@AndNot}
%    \begin{macrocode}
\def\BitSet@AndNot#1#2!#3#4!#5!{%
  \ifx\\#2\\%
    \ifnum#1#3=10 #51\fi
  \else
    \ifx\\#4\\%
      #5%
      \ifnum#1#3=10 1\else 0\fi
      #2%
    \else
      \ifnum#1#3=10 %
        #51%
        \BitSet@AfterFiFiFi{%
          \BitSet@AndNot#2!#4!!%
        }%
      \else
        \BitSet@AfterFiFiFi{%
          \BitSet@AndNot#2!#4!#50!%
        }%
      \fi
    \fi
  \BitSet@Fi
}
%    \end{macrocode}
%    \end{macro}
%
% \subsubsection{\cs{bitsetOr}}
%
%    \begin{macro}{\bitsetOr}
%    Decision table for \cs{bitsetOr}:
%    \begin{quote}
%    \begin{tabular}{l|l|l|l|}
%                         & undef(B)   & empty(B)   & cardinality(B)$>$0\\
%      \hline
%      undef(A)           & A := empty & A := empty & A := B\\
%      \hline
%      empty(A)           &            &            & A := B\\
%      \hline
%      cardinality(A)$>$0 &            &            & A \string|= B\\
%      \hline
%    \end{tabular}
%    \end{quote}
%    \begin{macrocode}
\def\bitsetOr#1#2{%
  \bitsetIsEmpty{#2}{%
    \BitSet@IfUndefined{#1}{\bitsetReset{#1}}{}%
  }{%
    \bitsetIsEmpty{#1}{%
      \expandafter\let\csname BS@#1\expandafter\endcsname
                      \csname BS@#2\endcsname
    }{%
      \expandafter\edef\csname BS@#1\endcsname{%
        \expandafter\expandafter\expandafter\BitSet@Or
        \csname BS@#1\expandafter\expandafter\expandafter\endcsname
        \expandafter\expandafter\expandafter!%
        \csname BS@#2\endcsname!%
      }%
    }%
  }%
}
%    \end{macrocode}
%    \end{macro}
%    \begin{macro}{\BitSet@Or}
%    \begin{macrocode}
\def\BitSet@Or#1#2!#3#4!{%
  \ifnum#1#3>0 1\else 0\fi
  \ifx\\#2\\%
    #4%
  \else
    \ifx\\#4\\%
      #2%
    \else
      \BitSet@AfterFiFi{%
        \BitSet@Or#2!#4!%
      }%
    \fi
  \BitSet@Fi
}
%    \end{macrocode}
%    \end{macro}
%
% \subsubsection{\cs{bitsetXor}}
%
%    \begin{macro}{\bitsetXor}
%    Decision table for \cs{bitsetXor}:
%    \begin{quote}
%    \begin{tabular}{l|l|l|l|}
%                         & undef(B)   & empty(B)   & cardinality(B)$>$0\\
%      \hline
%      undef(A)           & A := empty & A := empty & A := B\\
%      \hline
%      empty(A)           &            &            & A := B\\
%      \hline
%      cardinality(A)$>$0 &            &            & A \string^= B\\
%      \hline
%    \end{tabular}
%    \end{quote}
%    \begin{macrocode}
\def\bitsetXor#1#2{%
  \bitsetIsEmpty{#2}{%
    \BitSet@IfUndefined{#1}{\bitsetReset{#1}}{}%
  }{%
    \bitsetIsEmpty{#1}{%
      \expandafter\let\csname BS@#1\expandafter\endcsname
                      \csname BS@#2\endcsname
    }{%
      \expandafter\edef\csname BS@#1\endcsname{%
        \expandafter\expandafter\expandafter\BitSet@Xor
        \csname BS@#1\expandafter\expandafter\expandafter\endcsname
        \expandafter\expandafter\expandafter!%
        \csname BS@#2\endcsname!!%
      }%
      \expandafter\ifx\csname BS@#1\endcsname\BitSet@Empty
        \bitsetReset{#1}%
      \fi
    }%
  }%
}
%    \end{macrocode}
%    \end{macro}
%    \begin{macro}{\BitSet@Xor}
%    \begin{macrocode}
\def\BitSet@Xor#1#2!#3#4!#5!{%
  \ifx\\#2\\%
    \ifx#1#3%
      \ifx\\#4\\%
      \else
        #50#4%
      \fi
    \else
      #51#4%
    \fi
  \else
    \ifx\\#4\\%
      #5%
      \ifx#1#30\else 1\fi
      #2%
    \else
      \ifx#1#3%
        \BitSet@AfterFiFiFi{%
          \BitSet@Xor#2!#4!#50!%
        }%
      \else
        #51%
        \BitSet@AfterFiFiFi{%
          \BitSet@Xor#2!#4!!%
        }%
      \fi
    \fi
  \BitSet@Fi
}
%    \end{macrocode}
%    \end{macro}
%
% \subsubsection{Shifting}\label{src:shift}
%
% \subsubsection{\cs{bitsetShiftLeft}}
%
%    \begin{macro}{\bitsetShiftLeft}
%    \begin{macrocode}
\def\bitsetShiftLeft#1#2{%
  \BitSet@IfUndefined{#1}{%
    \bitsetReset{#1}%
  }{%
    \bitsetIsEmpty{#1}{%
    }{%
      \expandafter\expandafter\expandafter\BitSet@ShiftLeft
      \intcalcNum{#2}!{#1}%
    }%
  }%
}
%    \end{macrocode}
%    \end{macro}
%    \begin{macro}{\BitSet@ShiftLeft}
%    \begin{macrocode}
\def\BitSet@ShiftLeft#1!#2{%
  \ifcase\intcalcSgn{#1} %
  \or
    \begingroup
      \uccode`m=`0 %
    \uppercase\expandafter{\expandafter\endgroup
      \expandafter\edef\csname BS@#2\expandafter\endcsname
      \expandafter{%
        \romannumeral#1000\expandafter\BitSet@Space
        \csname BS@#2\endcsname
      }%
    }%
  \else
    \expandafter\BitSet@ShiftRight\BitSet@Gobble#1!{#2}%
  \fi
}
%    \end{macrocode}
%    \end{macro}
%
% \subsubsection{\cs{bitsetShiftRight}}
%
%    \begin{macro}{\bitsetShiftRight}
%    \begin{macrocode}
\def\bitsetShiftRight#1#2{%
  \BitSet@IfUndefined{#1}{%
    \bitsetReset{#1}%
  }{%
    \bitsetIsEmpty{#1}{%
    }{%
      \expandafter\expandafter\expandafter\BitSet@ShiftRight
      \intcalcNum{#2}!{#1}%
    }%
  }%
}
%    \end{macrocode}
%    \end{macro}
%    \begin{macro}{\BitSet@ShiftRight}
%    \begin{macrocode}
\def\BitSet@ShiftRight#1!#2{%
  \ifcase\intcalcSgn{#1} %
  \or
    \expandafter\edef\csname BS@#2\endcsname{%
      \expandafter\expandafter\expandafter\BitSet@Kill
      \csname BS@#2\expandafter\endcsname\expandafter\BitSet@Empty
      \expandafter=%
      \expandafter{\expandafter}\expandafter{\expandafter}%
      \romannumeral#1000!%
    }%
  \else
    \expandafter\BitSet@ShiftLeft\BitSet@Gobble#1!{#2}%
  \fi
}
%    \end{macrocode}
%    \end{macro}
%    \begin{macro}{\BitSet@Kill}
%    \begin{macrocode}
\def\BitSet@Kill#1#2=#3#4#5{%
  #3#4%
  \ifx#5!%
    \ifx#1\BitSet@Empty
      0%
    \else
      #1#2%
    \fi
  \else
    \ifx#1\BitSet@Empty
      0%
      \BitSet@AfterFiFi\BitSet@Cleanup
    \else
      \BitSet@Kill#2=%
    \fi
  \BitSet@Fi
}
%    \end{macrocode}
%    \end{macro}
%
% \subsection{Bit manipulation}\label{src:bitman}
%
%    \begin{macro}{\bitsetClear}
%    \begin{macrocode}
\def\bitsetClear{%
  \BitSet@CheckIndex\BitSet@Clear
}
%    \end{macrocode}
%    \end{macro}
%    \begin{macro}{\bitsetSet}
%    \begin{macrocode}
\def\bitsetSet{%
  \BitSet@CheckIndex\BitSet@Set
}
%    \end{macrocode}
%    \end{macro}
%    \begin{macro}{\bitsetFlip}
%    \begin{macrocode}
\def\bitsetFlip{%
  \BitSet@CheckIndex\BitSet@Flip
}
%    \end{macrocode}
%    \end{macro}
%    \begin{macro}{\bitsetSetValue}
%    \begin{macrocode}
\def\bitsetSetValue#1#2#3{%
  \expandafter\expandafter\expandafter\BitSet@SetValue
  \intcalcNum{#3}!{#1}{#2}%
}
%    \end{macrocode}
%    \end{macro}
%    \begin{macro}{\BitSet@SetValue}
%    |#1|: plain value\\
%    |#2|: BitSet\\
%    |#3|: Index
%    \begin{macrocode}
\def\BitSet@SetValue#1!{%
  \BitSet@CheckIndex{%
    \ifcase#1 %
      \expandafter\BitSet@Clear
    \or
      \expandafter\BitSet@Set
    \else
      \BitSet@ErrorInvalidBitValue{#1}%
      \expandafter\expandafter\expandafter\BitSet@Gobble
      \expandafter\BitSet@Gobble
    \fi
  }%
}
%    \end{macrocode}
%    \end{macro}
%    \begin{macro}{\BitSet@ErrorInvalidBitValue}
%    |#1|: Wrong bit value
%    \begin{macrocode}
\def\BitSet@ErrorInvalidBitValue#1{%
  \@PackageError{bitset}{%
    Invalid bit value (#1) not in range 0..1%
  }\@ehc
}
%    \end{macrocode}
%    \end{macro}
%
% \subsubsection{Clear operation}
%
%    \begin{macro}{\BitSet@Clear}
%    |#1|: BitSet\\
%    |#2|: plain and checked index
%    \begin{macrocode}
\def\BitSet@Clear#1#2{%
  \edef\BitSet@Temp{%
    \expandafter\expandafter\expandafter\BitSet@@Clear
    \csname BS@#1\expandafter\endcsname
    \expandafter\BitSet@Empty\expandafter=\expandafter!%
    \romannumeral#2000!%
  }%
  \expandafter\let\csname BS@#1\expandafter\endcsname
  \ifx\BitSet@Temp\BitSet@Empty
    \BitSet@Zero
  \else
    \BitSet@Temp
  \fi
}
%    \end{macrocode}
%    \end{macro}
%    \begin{macro}{\BitSet@@Clear}
%    \begin{macrocode}
\def\BitSet@@Clear#1#2=#3!#4{%
  \ifx#4!%
    \ifx#1\BitSet@Empty
    \else
      \ifx\BitSet@Empty#2%
      \else
        #30#2%
      \fi
    \fi
  \else
    \ifx#1\BitSet@Empty
      \BitSet@AfterFiFi\BitSet@Cleanup
    \else
      \ifx#10%
        \BitSet@AfterFiFiFi{%
          \BitSet@@Clear#2=#30!%
        }%
      \else
        #31%
        \BitSet@AfterFiFiFi{%
          \BitSet@@Clear#2=!%
        }%
      \fi
    \fi
  \BitSet@Fi
}
%    \end{macrocode}
%    \end{macro}
%
% \subsubsection{Set operation}
%
%    \begin{macro}{\BitSet@Set}
%    |#1|: BitSet\\
%    |#2|: plain and checked Index
%    \begin{macrocode}
\def\BitSet@Set#1#2{%
  \expandafter\edef\csname BS@#1\endcsname{%
    \expandafter\expandafter\expandafter\BitSet@@Set
    \csname BS@#1\expandafter\endcsname
    \expandafter\BitSet@Empty\expandafter=%
    \expandafter{\expandafter}\expandafter{\expandafter}%
    \romannumeral#2000!%
  }%
}
%    \end{macrocode}
%    \end{macro}
%    \begin{macro}{\BitSet@@Set}
%    \begin{macrocode}
\def\BitSet@@Set#1#2=#3#4#5{%
  #3#4%
  \ifx#5!%
    1#2%
  \else
    \ifx#1\BitSet@Empty
      0%
      \BitSet@AfterFiFi\BitSet@@@Set
    \else
      #1%
      \BitSet@@Set#2=%
    \fi
  \BitSet@Fi
}
%    \end{macrocode}
%    \end{macro}
%    \begin{macro}{\BitSet@@@Set}
%    \begin{macrocode}
\def\BitSet@@@Set#1{%
  \ifx#1!%
    1%
  \else
    0%
    \expandafter\BitSet@@@Set
  \fi
}
%    \end{macrocode}
%    \end{macro}
%
% \subsubsection{Flip operation}
%
%    \begin{macro}{\BitSet@Flip}
%    |#1|: BitSet\\
%    |#2|: plain and checked Index
%    \begin{macrocode}
\def\BitSet@Flip#1#2{%
  \edef\BitSet@Temp{%
    \expandafter\expandafter\expandafter\BitSet@@Flip
    \csname BS@#1\expandafter\endcsname
    \expandafter\BitSet@Empty\expandafter=\expandafter!%
    \romannumeral#2000!%
  }%
  \expandafter\let\csname BS@#1\expandafter\endcsname
  \ifx\BitSet@Temp\BitSet@Empty
    \BitSet@Zero
  \else
    \BitSet@Temp
  \fi
}
%    \end{macrocode}
%    \end{macro}
%    \begin{macro}{\BitSet@@Flip}
%    \begin{macrocode}
\def\BitSet@@Flip#1#2=#3!#4{%
  \ifx#4!%
    \ifx#11%
      \ifx\BitSet@Empty#2%
      \else
        #30#2%
      \fi
    \else
      #31#2%
    \fi
  \else
    \ifx#1\BitSet@Empty
      #30%
      \BitSet@AfterFiFi\BitSet@@@Set
    \else
      \ifx#10%
        \BitSet@AfterFiFiFi{%
          \BitSet@@Flip#2=#30!%
        }%
      \else
        #31%
        \BitSet@AfterFiFiFi{%
          \BitSet@@Flip#2=!%
        }%
      \fi
    \fi
  \BitSet@Fi
}
%    \end{macrocode}
%    \end{macro}
%
% \subsubsection{Range operators}
%
%    \begin{macro}{\bitsetClearRange}
%    \begin{macrocode}
\def\bitsetClearRange{%
  \BitSet@Range\BitSet@Clear
}
%    \end{macrocode}
%    \end{macro}
%    \begin{macro}{\bitsetSetRange}
%    \begin{macrocode}
\def\bitsetSetRange{%
  \BitSet@Range\BitSet@Set
}
%    \end{macrocode}
%    \end{macro}
%    \begin{macro}{\bitsetFlipRange}
%    \begin{macrocode}
\def\bitsetFlipRange{%
  \BitSet@Range\BitSet@Flip
}
%    \end{macrocode}
%    \end{macro}
%    \begin{macro}{\bitsetSetValueRange}
%    \begin{macrocode}
\def\bitsetSetValueRange#1#2#3#4{%
  \expandafter\expandafter\expandafter\BitSet@SetValueRange
  \intcalcNum{#4}!{#1}{#2}{#3}%
}
%    \end{macrocode}
%    \end{macro}
%    \begin{macro}{\BitSet@SetValueRange}
%    \begin{macrocode}
\def\BitSet@SetValueRange#1!#2#3#4{%
  \ifcase#1 %
    \BitSet@Range\BitSet@Clear{#2}{#3}{#4}%
  \or
    \BitSet@Range\BitSet@Set{#2}{#3}{#4}%
  \else
    \BitSet@ErrorInvalidBitValue{#1}%
  \fi
}
%    \end{macrocode}
%    \end{macro}
%
%    \begin{macro}{\BitSet@Range}
%    |#1|: clear/set/flip macro\\
%    |#2|: BitSet\\
%    |#3|: Index from\\
%    |#4|: Index to
%    \begin{macrocode}
\def\BitSet@Range#1#2#3#4{%
  \edef\BitSet@Temp{%
    \noexpand\BitSet@@Range\noexpand#1{#2}%
    \intcalcNum{#3}!\intcalcNum{#4}!%
  }%
  \BitSet@Temp
}
%    \end{macrocode}
%    \end{macro}
%    \begin{macro}{\BitSet@@Range}
%    |#1|: clear/set/flip macro\\
%    |#2|: BitSet\\
%    |#3|: Index from\\
%    |#4|: Index to
%    \begin{macrocode}
\def\BitSet@@Range#1#2#3!#4!{%
  \ifnum#3<0 %
    \BitSet@NegativeIndex#1{#2}#3!#4!0!#4!%
  \else
    \ifnum#4<0 %
      \BitSet@NegativeIndex#1{#2}#3!#4!#3!0!%
    \else
      \ifcase\intcalcCmp{#3}{#4} %
      \or
        \@PackageError{bitset}{%
          Wrong index numbers in range [#3..#4]\MessageBreak% hash-ok
          for clear/set/flip on bit set `#2'.\MessageBreak
          The lower index exceeds the upper index.\MessageBreak
          Canceling the operation as error recovery%
        }\@ehc
      \else
        \BitSet@@@Range#3!#4!#1{#2}%
      \fi
    \fi
  \fi
}
%    \end{macrocode}
%    \end{macro}
%    \begin{macro}{\BitSet@NegativeIndex}
%    \begin{macrocode}
\def\BitSet@NegativeIndex#1#2#3!#4!#5!#6!{%
  \@PackageError{bitset}{%
    Negative index in range [#3..#4]\MessageBreak % hash-ok
    for \string\bitset
    \ifx#1\BitSet@Clear
      Clear%
    \else
      \ifx#1\BitSet@Set
        Set%
      \else
        Flip%
      \fi
    \fi
    Range on bit set `#2'.\MessageBreak
    Using [#5..#6] as error recovery% hash-ok
  }\@ehc
  \BitSet@@Range#1{#2}#5!#6!%
}
%    \end{macrocode}
%    \end{macro}
%    \begin{macro}{\BitSet@@Range}
%    \begin{macrocode}
\def\BitSet@@@Range#1!#2!#3#4{%
  \ifnum#1<#2 %
    #3{#4}{#1}%
    \BitSet@AfterFi{%
      \expandafter\expandafter\expandafter\BitSet@@@Range
      \IntCalcInc#1!!#2!#3{#4}%
    }%
  \BitSet@Fi
}
%    \end{macrocode}
%    \end{macro}
%
% \subsection{Bit retrieval}\label{src:bitret}
%
% \subsubsection{\cs{bitsetGet}}
%
%    \begin{macro}{\bitsetGet}
%    \begin{macrocode}
\def\bitsetGet#1#2{%
  \number
  \expandafter\expandafter\expandafter\BitSet@Get
  \intcalcNum{#2}!{#1}%
}
%    \end{macrocode}
%    \end{macro}
%    \begin{macro}{\BitSet@Get}
%    |#1|: plain index\\
%    |#2|: BitSet
%    \begin{macrocode}
\def\BitSet@Get#1!#2{%
  \ifnum#1<0 %
    \BitSet@AfterFi{%
      0 \BitSetError:NegativeIndex%
    }%
  \else
    \BitSet@IfUndefined{#2}{0}{%
      \expandafter\expandafter\expandafter\BitSet@@Get
      \csname BS@#2\expandafter\endcsname
      \expandafter!\expandafter=%
      \expandafter{\expandafter}\expandafter{\expandafter}%
      \romannumeral\intcalcNum{#1}000!%
    }%
    \expandafter\BitSet@Space
  \BitSet@Fi
}
%    \end{macrocode}
%    \end{macro}
%    \begin{macro}{\BitSet@@Get}
%    \begin{macrocode}
\def\BitSet@@Get#1#2=#3#4#5{%
  #3#4%
  \ifx#5!%
    \ifx#1!%
      0%
    \else
      #1%
    \fi
  \else
    \ifx#1!%
      0%
      \BitSet@AfterFiFi\BitSet@Cleanup
    \else
      \BitSet@@Get#2=%
    \fi
  \BitSet@Fi
}
%    \end{macrocode}
%    \end{macro}
%
% \subsubsection{\cs{bitsetNextClearBit}, \cs{bitsetNextSetBit}}
%
%    \begin{macro}{\bitsetNextClearBit}
%    \begin{macrocode}
\def\bitsetNextClearBit#1#2{%
  \number
  \expandafter\expandafter\expandafter\BitSet@NextClearBit
  \intcalcNum{#2}!{#1} %
}
%    \end{macrocode}
%    \end{macro}
%    \begin{macro}{\BitSet@NextClearBit}
%    |#1|: Index\\
%    |#2|: BitSet
%    \begin{macrocode}
\def\BitSet@NextClearBit#1!#2{%
  \ifnum#1<0 %
    \BitSet@NextClearBit0!{#2}%
    \BitSet@AfterFi{%
      \expandafter\BitSet@Space
      \expandafter\BitSetError:NegativeIndex\romannumeral0%
    }%
  \else
    \bitsetIsEmpty{#2}{#1}{%
      \expandafter\BitSet@Skip
      \number#1\expandafter\expandafter\expandafter!%
      \csname BS@#2\endcsname!!!!!!!!!=%
      {\BitSet@@NextClearBit#1!}%
    }%
  \BitSet@Fi
}
%    \end{macrocode}
%    \end{macro}
%    \begin{macro}{\BitSet@@NextClearBit}
%    |#1|: index for next bit in |#2|\\
%    |#2|: next bit
%    \begin{macrocode}
\def\BitSet@@NextClearBit#1!#2{%
  \ifx#2!%
    #1%
  \else
    \ifx#20%
      #1%
      \BitSet@AfterFiFi\BitSet@Cleanup
    \else
      \BitSet@AfterFiFi{%
        \expandafter\expandafter\expandafter\BitSet@@NextClearBit
        \IntCalcInc#1!!%
      }%
    \fi
  \BitSet@Fi
}
%    \end{macrocode}
%    \end{macro}
%
%    \begin{macro}{\bitsetNextSetBit}
%    \begin{macrocode}
\def\bitsetNextSetBit#1#2{%
  \number
  \expandafter\expandafter\expandafter\BitSet@NextSetBit
  \intcalcNum{#2}!{#1} %
}
%    \end{macrocode}
%    \end{macro}
%    \begin{macro}{\BitSet@NextSetBit}
%    |#1|: Index\\
%    |#2|: BitSet
%    \begin{macrocode}
\def\BitSet@NextSetBit#1!#2{%
  \ifnum#1<0 %
    \BitSet@NextSetBit0!{#2}%
    \BitSet@AfterFi{%
      \expandafter\BitSet@Space
      \expandafter\BitSetError:NegativeIndex\romannumeral0%
    }%
  \else
    \bitsetIsEmpty{#2}{-1}{%
      \expandafter\BitSet@Skip
      \number#1\expandafter\expandafter\expandafter!%
      \csname BS@#2\endcsname!!!!!!!!!=%
      {\BitSet@@NextSetBit#1!}%
    }%
  \BitSet@Fi
}
%    \end{macrocode}
%    \end{macro}
%    \begin{macro}{\BitSet@@NextSetBit}
%    |#1|: index for next bit in |#2|\\
%    |#2|: next bit
%    \begin{macrocode}
\def\BitSet@@NextSetBit#1!#2{%
  \ifx#2!%
    -1%
  \else
    \ifx#21%
      #1%
      \BitSet@AfterFiFi\BitSet@Cleanup
    \else
      \BitSet@AfterFiFi{%
        \expandafter\expandafter\expandafter\BitSet@@NextSetBit
        \IntCalcInc#1!!%
      }%
    \fi
  \BitSet@Fi
}
%    \end{macrocode}
%    \end{macro}
%    \begin{macro}{\BitSet@Cleanup}
%    \begin{macrocode}
\def\BitSet@Cleanup#1!{}
%    \end{macrocode}
%    \end{macro}
%
%    \begin{macro}{\BitSet@Skip}
%    |#1|: number of bits to skip\\
%    |#2|: bits\\
%    |#3|: continuation code
%    \begin{macrocode}
\def\BitSet@Skip#1!#2{%
  \ifx#2!%
    \BitSet@AfterFi{%
      \BitSet@SkipContinue%
    }%
  \else
    \ifcase#1 %
      \BitSet@AfterFiFi{%
        \BitSet@SkipContinue#2%
      }%
    \or
      \BitSet@AfterFiFi\BitSet@SkipContinue
    \or
      \BitSet@AfterFiFi{%
        \expandafter\BitSet@SkipContinue\BitSet@Gobble
      }%
    \else
      \ifnum#1>8 %
        \BitSet@AfterFiFiFi{%
          \expandafter\BitSet@Skip
          \number\IntCalcSub#1!8!\expandafter!%
          \BitSet@GobbleSeven
        }%
      \else
        \BitSet@AfterFiFiFi{%
          \expandafter\expandafter\expandafter\BitSet@Skip
          \IntCalcDec#1!!%
        }%
      \fi
    \fi
  \BitSet@Fi
}
%    \end{macrocode}
%    \end{macro}
%    \begin{macro}{\BitSet@SkipContinue}
%    |#1|: remaining bits\\
%    |#2|: continuation code
%    \begin{macrocode}
\def\BitSet@SkipContinue#1!#2=#3{%
  #3#1!%
}
%    \end{macrocode}
%    \end{macro}
%    \begin{macro}{\BitSet@GobbleSeven}
%    \begin{macrocode}
\def\BitSet@GobbleSeven#1#2#3#4#5#6#7{}
%    \end{macrocode}
%    \end{macro}
%
% \subsubsection{\cs{bitsetGetSetBitList}}
%
%    \begin{macro}{\bitsetGetSetBitList}
%    It's just a wrapper for \cs{bitsetNextSetBit}.
%    \begin{macrocode}
\def\bitsetGetSetBitList#1{%
  \romannumeral0%
  \bitsetIsEmpty{#1}{ }{%
    \expandafter\BitSet@GetSetBitList
    \number\BitSet@NextSetBit0!{#1}!{#1}{}!%
  }%
}
%    \end{macrocode}
%    \end{macro}
%    \begin{macro}{\BitSet@GetSetBitList}
%    |#1|: found index\\
%    |#2|: BitSet\\
%    |#3|: comma
%    |#4|: result
%    \begin{macrocode}
\def\BitSet@GetSetBitList#1!#2#3#4!{%
  \ifnum#1<0 %
    \BitSet@AfterFi{ #4}%
  \else
    \BitSet@AfterFi{%
      \expandafter\BitSet@GetSetBitList\number
      \expandafter\expandafter\expandafter\BitSet@NextSetBit
      \IntCalcInc#1!!{#2}!{#2},#4#3#1!%
    }%
  \BitSet@Fi
}
%    \end{macrocode}
%    \end{macro}
%
% \subsection{Bit set properties}\label{src:prop}
%
%    \begin{macro}{\bitsetSize}
%    \begin{macrocode}
\def\bitsetSize#1{%
  \number
  \BitSet@IfUndefined{#1}{0 }{%
    \expandafter\expandafter\expandafter\BitSet@Size
    \expandafter\expandafter\expandafter1%
    \expandafter\expandafter\expandafter!%
    \csname BS@#1\endcsname!0!%
  }%
}
%    \end{macrocode}
%    \end{macro}
%    \begin{macro}{\BitSet@Size}
%    |#1|: counter\\
%    |#2#3|: bits\\
%    |#4|: result
%    \begin{macrocode}
\def\BitSet@Size#1!#2#3!#4!{%
  \ifx#21%
    \ifx\\#3\\%
      \BitSet@AfterFiFi{#1 }%
    \else
      \BitSet@AfterFiFi{%
        \expandafter\expandafter\expandafter\BitSet@Size
        \IntCalcInc#1!!#3!#1!%
      }%
    \fi
  \else
    \ifx\\#3\\%
      \BitSet@AfterFiFi{#4 }%
    \else
      \BitSet@AfterFiFi{%
        \expandafter\expandafter\expandafter\BitSet@Size
        \IntCalcInc#1!!#3!#4!%
      }%
    \fi
  \fi
  \BitSet@Fi
}
%    \end{macrocode}
%    \end{macro}
%
%    \begin{macro}{\bitsetCardinality}
%    \begin{macrocode}
\def\bitsetCardinality#1{%
  \number
  \BitSet@IfUndefined{#1}{0 }{%
    \expandafter\expandafter\expandafter\BitSet@Cardinality
    \expandafter\expandafter\expandafter0%
    \expandafter\expandafter\expandafter!%
    \csname BS@#1\endcsname!%
  }%
}
%    \end{macrocode}
%    \end{macro}
%    \begin{macro}{\BitSet@Cardinality}
%    |#1|: result\\
%    |#2#3|: bits
%    \begin{macrocode}
\def\BitSet@Cardinality#1!#2#3!{%
  \ifx#21%
    \ifx\\#3\\%
      \BitSet@AfterFiFi{\IntCalcInc#1! }%
    \else
      \BitSet@AfterFiFi{%
        \expandafter\expandafter\expandafter\BitSet@Cardinality
        \IntCalcInc#1!!#3!%
      }%
    \fi
  \else
    \ifx\\#3\\%
      \BitSet@AfterFiFi{#1 }%
    \else
      \BitSet@AfterFiFi{%
        \BitSet@Cardinality#1!#3!%
      }%
    \fi
  \fi
  \BitSet@Fi
}
%    \end{macrocode}
%    \end{macro}
%
% \subsection{Queries}\label{src:query}
%
%    \begin{macro}{\bitsetIsDefined}
%    \begin{macrocode}
\def\bitsetIsDefined#1{%
  \BitSet@IfUndefined{#1}%
  \BitSet@SecondOfTwo
  \BitSet@FirstOfTwo
}
%    \end{macrocode}
%    \end{macro}
%    \begin{macro}{\bitsetIsEmpty}
%    \begin{macrocode}
\def\bitsetIsEmpty#1{%
  \BitSet@IfUndefined{#1}\BitSet@FirstOfTwo{%
    \expandafter\ifx\csname BS@#1\endcsname\BitSet@Zero
      \expandafter\BitSet@FirstOfTwo
    \else
      \expandafter\BitSet@SecondOfTwo
    \fi
  }%
}
%    \end{macrocode}
%    \end{macro}
%    \begin{macro}{\BitSet@Zero}
%    \begin{macrocode}
\def\BitSet@Zero{0}
%    \end{macrocode}
%    \end{macro}
%
%    \begin{macro}{\bitsetQuery}
%    \begin{macrocode}
\def\bitsetQuery#1#2{%
  \ifnum\bitsetGet{#1}{#2}=1 %
    \expandafter\BitSet@FirstOfTwo
  \else
    \expandafter\BitSet@SecondOfTwo
  \fi
}
%    \end{macrocode}
%    \end{macro}
%
%    \begin{macro}{\bitsetEquals}
%    \begin{macrocode}
\def\bitsetEquals#1#2{%
  \BitSet@IfUndefined{#1}{%
    \BitSet@IfUndefined{#2}\BitSet@FirstOfTwo\BitSet@SecondOfTwo
  }{%
    \BitSet@IfUndefined{#2}\BitSet@SecondOfTwo{%
      \expandafter\ifx\csname BS@#1\expandafter\endcsname
                      \csname BS@#2\endcsname
        \expandafter\BitSet@FirstOfTwo
      \else
        \expandafter\BitSet@SecondOfTwo
      \fi
    }%
  }%
}
%    \end{macrocode}
%    \end{macro}
%
%    \begin{macro}{\bitsetIntersects}
%    \begin{macrocode}
\def\bitsetIntersects#1#2{%
  \bitsetIsEmpty{#1}\BitSet@SecondOfTwo{%
    \bitsetIsEmpty{#2}\BitSet@SecondOfTwo{%
      \expandafter\expandafter\expandafter\BitSet@Intersects
      \csname BS@#1\expandafter\expandafter\expandafter\endcsname
      \expandafter\expandafter\expandafter!%
      \csname BS@#2\endcsname!%
    }%
  }%
}
%    \end{macrocode}
%    \end{macro}
%    \begin{macro}{\BitSet@Intersects}
%    \begin{macrocode}
\def\BitSet@Intersects#1#2!#3#4!{%
  \ifnum#1#3=11 %
    \BitSet@AfterFi\BitSet@FirstOfTwo
  \else
    \ifx\\#2\\%
      \BitSet@AfterFiFi\BitSet@SecondOfTwo
    \else
      \ifx\\#4\\%
        \BitSet@AfterFiFiFi\BitSet@SecondOfTwo
      \else
        \BitSet@AfterFiFiFi{%
          \BitSet@Intersects#2!#4!%
        }%
      \fi
    \fi
  \BitSet@Fi
}
%    \end{macrocode}
%    \end{macro}
%
%    \begin{macrocode}
\BitSet@AtEnd%
%</package>
%    \end{macrocode}
%
% \section{Test}
%
% \subsection{Catcode checks for loading}
%
%    \begin{macrocode}
%<*test1>
%    \end{macrocode}
%    \begin{macrocode}
\catcode`\{=1 %
\catcode`\}=2 %
\catcode`\#=6 %
\catcode`\@=11 %
\expandafter\ifx\csname count@\endcsname\relax
  \countdef\count@=255 %
\fi
\expandafter\ifx\csname @gobble\endcsname\relax
  \long\def\@gobble#1{}%
\fi
\expandafter\ifx\csname @firstofone\endcsname\relax
  \long\def\@firstofone#1{#1}%
\fi
\expandafter\ifx\csname loop\endcsname\relax
  \expandafter\@firstofone
\else
  \expandafter\@gobble
\fi
{%
  \def\loop#1\repeat{%
    \def\body{#1}%
    \iterate
  }%
  \def\iterate{%
    \body
      \let\next\iterate
    \else
      \let\next\relax
    \fi
    \next
  }%
  \let\repeat=\fi
}%
\def\RestoreCatcodes{}
\count@=0 %
\loop
  \edef\RestoreCatcodes{%
    \RestoreCatcodes
    \catcode\the\count@=\the\catcode\count@\relax
  }%
\ifnum\count@<255 %
  \advance\count@ 1 %
\repeat

\def\RangeCatcodeInvalid#1#2{%
  \count@=#1\relax
  \loop
    \catcode\count@=15 %
  \ifnum\count@<#2\relax
    \advance\count@ 1 %
  \repeat
}
\def\RangeCatcodeCheck#1#2#3{%
  \count@=#1\relax
  \loop
    \ifnum#3=\catcode\count@
    \else
      \errmessage{%
        Character \the\count@\space
        with wrong catcode \the\catcode\count@\space
        instead of \number#3%
      }%
    \fi
  \ifnum\count@<#2\relax
    \advance\count@ 1 %
  \repeat
}
\def\space{ }
\expandafter\ifx\csname LoadCommand\endcsname\relax
  \def\LoadCommand{\input bitset.sty\relax}%
\fi
\def\Test{%
  \RangeCatcodeInvalid{0}{47}%
  \RangeCatcodeInvalid{58}{64}%
  \RangeCatcodeInvalid{91}{96}%
  \RangeCatcodeInvalid{123}{255}%
  \catcode`\@=12 %
  \catcode`\\=0 %
  \catcode`\%=14 %
  \LoadCommand
  \RangeCatcodeCheck{0}{36}{15}%
  \RangeCatcodeCheck{37}{37}{14}%
  \RangeCatcodeCheck{38}{47}{15}%
  \RangeCatcodeCheck{48}{57}{12}%
  \RangeCatcodeCheck{58}{63}{15}%
  \RangeCatcodeCheck{64}{64}{12}%
  \RangeCatcodeCheck{65}{90}{11}%
  \RangeCatcodeCheck{91}{91}{15}%
  \RangeCatcodeCheck{92}{92}{0}%
  \RangeCatcodeCheck{93}{96}{15}%
  \RangeCatcodeCheck{97}{122}{11}%
  \RangeCatcodeCheck{123}{255}{15}%
  \RestoreCatcodes
}
\Test
\csname @@end\endcsname
\end
%    \end{macrocode}
%    \begin{macrocode}
%</test1>
%    \end{macrocode}
%
% \subsection{Macro tests}
%
% \subsubsection{Preamble}
%
%    \begin{macrocode}
%<*test2>
\NeedsTeXFormat{LaTeX2e}
\nofiles
\documentclass{article}
\makeatletter
%<*noetex>
\let\SavedNumexpr\numexpr
\let\SavedIfcsname\ifcsname
\let\SavedCurrentgrouplevel\currentgrouplevel
\def\ETeXDisable{%
  \let\ifcsname\@undefined
  \let\numexpr\@undefined
  \let\currentgrouplevel\@undefined
}
\ETeXDisable
%</noetex>
\makeatletter
\chardef\BitSet@TestMode=1 %
\makeatother
\usepackage{bitset}[2011/01/30]
%<*noetex>
\def\ETeXEnable{%
  \let\numexpr\SavedNumexpr
  \let\ifcsname\SavedIfcsname
  \let\currentgrouplevel\SavedCurrentgrouplevel
}
\ETeXEnable
%</noetex>
\usepackage{qstest}
\IncludeTests{*}
\LogTests{log}{*}{*}
\makeatletter
%    \end{macrocode}
%
% \subsubsection{Time}
%
%    \begin{macrocode}
\begingroup\expandafter\expandafter\expandafter\endgroup
\expandafter\ifx\csname pdfresettimer\endcsname\relax
\else
  \newcount\SummaryTime
  \newcount\TestTime
  \SummaryTime=\z@
  \newcommand*{\PrintTime}[2]{%
    \typeout{%
      [Time #1: \strip@pt\dimexpr\number#2sp\relax\space s]%
    }%
  }%
  \newcommand*{\StartTime}[1]{%
    \renewcommand*{\TimeDescription}{#1}%
    \pdfresettimer
  }%
  \newcommand*{\TimeDescription}{}%
  \newcommand*{\StopTime}{%
    \TestTime=\pdfelapsedtime
    \global\advance\SummaryTime\TestTime
    \PrintTime\TimeDescription\TestTime
  }%
  \let\saved@qstest\qstest
  \let\saved@endqstest\endqstest
  \def\qstest#1#2{%
    \saved@qstest{#1}{#2}%
    \StartTime{#1}%
  }%
  \def\endqstest{%
    \StopTime
    \saved@endqstest
  }%
  \AtEndDocument{%
    \PrintTime{summary}\SummaryTime
  }%
\fi
%    \end{macrocode}
%
% \subsubsection{Detection of unwanted space}
%
%    \begin{macrocode}
\let\orig@qstest\qstest
\let\orig@endqstest\endqstest
\def\qstest#1#2{%
  \orig@qstest{#1}{#2}%
  \setbox0\hbox\bgroup\begingroup\ignorespaces
}
\def\endqstest{%
  \endgroup\egroup
  \Expect*{\the\wd0}{0.0pt}%
  \orig@endqstest
}
%    \end{macrocode}
%
% \subsubsection{Test macros}
%
%    \begin{macrocode}
\newcounter{Test}

\def\TestError#1#2{%
  \begingroup
    \setcounter{Test}{0}%
    \sbox0{%
      \def\@PackageError##1##2##3{%
        \stepcounter{Test}%
        \begingroup
          \let\MessageBreak\relax
%<*noetex>
          \ETeXEnable
%</noetex>
          \Expect{##1}{bitset}%
          \Expect*{##2}*{#1}%
        \endgroup
      }%
%<*noetex>
      \ETeXDisable
%</noetex>
      #2%
    }%
    \Expect*{\theTest}{1}%
    \Expect*{\the\wd0}{0.0pt}%
  \endgroup
}

\def\TestErrorNegativeIndex#1#2{%
  \TestError{Invalid negative index (#1)}{#2}%
}

\def\TestGetterUndefined#1{%
  \CheckUndef{dummy}%
  \expandafter\expandafter\expandafter\Expect
  \expandafter\expandafter\expandafter{#1{dummy}}{0}%
}

\def\ExpectBitSet#1#2{%
  \expandafter\expandafter\expandafter\Expect
  \expandafter\expandafter\expandafter
  {\csname BS@#1\endcsname}*{#2}%
}
\def\Check#1#2{%
  \ExpectBitSet{#1}{#2}%
}
\def\CheckUndef#1{%
  \begingroup
    \Expect*{%
      \expandafter
      \ifx\csname BS@#1\endcsname\relax true\else false\fi
    }{true}%
  \endgroup
}
\def\RevCheck#1#2{%
  \ExpectBitSet{#1}{\Reverse#2!!}%
}
\def\Set#1#2{%
  \expandafter\def\csname BS@#1\endcsname{#2}%
}
\def\RevSet#1#2{%
  \expandafter\edef\csname BS@#1\endcsname{%
    \Reverse#2!!%
  }%
}
\def\Reverse#1#2!#3!{%
  \ifx\\#2\\%
    #1#3%
    \expandafter\@gobble
  \else
    \expandafter\@firstofone
  \fi
  {\Reverse#2!#1#3!}%
}
%    \end{macrocode}
%
% \subsubsection{Test sets}
%
%    \begin{macrocode}
\begin{qstest}{Let}{Let}
  \CheckUndef{abc}%
  \CheckUndef{xyz}%
  \bitsetLet{xyz}{abc}%
  \CheckUndef{abc}%
  \Check{xyz}{0}%
  \Set{abc}{1}%
  \Check{abc}{1}%
  \Check{xyz}{0}%
  \bitsetLet{xyz}{abc}%
  \Check{abc}{1}%
  \Check{xyz}{1}%
  \Set{xyz}{11}%
  \Check{abc}{1}%
  \Check{xyz}{11}%
\end{qstest}

\begin{qstest}{Reset}{Reset}
  \bitsetReset{xyz}%
  \Check{xyz}{0}%
  \bitsetReset{abc}%
  \Check{abc}{0}%
  \Set{abc}{10101}%
  \bitsetReset{abc}%
  \Check{abc}{0}%
\end{qstest}

\begin{qstest}{Get/Query}{Get/Query}
  \expandafter\expandafter\expandafter\Expect
  \expandafter\expandafter\expandafter{%
    \bitsetGet{dummy}{0}%
  }{0}%
  \begingroup
    \expandafter\def\csname BitSetError:NegativeIndex\endcsname{}%
    \Set{abc}{1}%
    \Expect*{\bitsetQuery{abc}{-1}{true}{false}}{false}%
  \endgroup
  \def\Test#1#2#3{%
    \Set{abc}{#1}%
    \expandafter\expandafter\expandafter\Expect
    \expandafter\expandafter\expandafter{\bitsetGet{abc}{#2}}{#3}%
    \Expect*{\bitsetQuery{abc}{#2}{true}{false}}%
           *{\ifcase#3 false\or true\else error\fi}%
  }%
  \Test{1}{100}{0}%
  \Test{0}{0}{0}%
  \Test{1}{0}{1}%
  \Test{11}{1}{1}%
  \Test{111}{1}{1}%
  \Test{101}{1}{0}%
  \Test{101}{2}{1}%
  \Test{10100110011}{10}{1}%
\end{qstest}

\begin{qstest}{Size}{Size}
  \TestGetterUndefined\bitsetSize
  \def\Test#1#2{%
    \Set{abc}{#1}%
    \expandafter\expandafter\expandafter\Expect
    \expandafter\expandafter\expandafter{\bitsetSize{abc}}{#2}%
  }%
  \Test{0}{0}%
  \Test{1}{1}%
  \Test{00}{0}%
  \Test{0000000}{0}%
  \Test{10}{1}%
  \Test{01}{2}%
  \Test{11}{2}%
  \Test{010}{2}%
  \Test{011}{3}%
  \Test{100110011}{9}%
  \Test{0000011111000001111100000}{20}%
  \Test{000000000000000000000000011111111111111111111}{45}%
\end{qstest}

\begin{qstest}{Cardinality}{Cardinality}
  \TestGetterUndefined\bitsetCardinality
  \def\Test#1#2{%
    \Set{abc}{#1}%
    \expandafter\expandafter\expandafter\Expect
    \expandafter\expandafter\expandafter{%
      \bitsetCardinality{abc}%
    }{#2}%
  }%
  \Test{0}{0}%
  \Test{1}{1}%
  \Test{00}{0}%
  \Test{0000000}{0}%
  \Test{10}{1}%
  \Test{01}{1}%
  \Test{11}{2}%
  \Test{010}{1}%
  \Test{011}{2}%
  \Test{100110011}{5}%
  \Test{0000011111000001111100000}{10}%
  \Test{000000000000000000000000011111111111111111111}{20}%
\end{qstest}

\begin{qstest}{NextClearBit/NextSetBit}{NextClearBit/NextSetBit}
  \def\Test#1#2{%
    \expandafter\expandafter\expandafter\Expect
    \expandafter\expandafter\expandafter{%
      \TestOp{abc}{#1}%
    }{#2}%
  }%
  \def\Clear{\let\TestOp\bitsetNextClearBit}%
  \def\Set{\let\TestOp\bitsetNextSetBit}%
  \begingroup
    \catcode`\:=11 %
    \bitsetSetBin{abc}{1}%
    \Clear
    \Test{-1}{1\BitSetError:NegativeIndex}%
    \Set
    \Test{-1}{0\BitSetError:NegativeIndex}%
  \endgroup
  \let\BS@abc\@undefined
  \Clear
  \Test{0}{0}%
  \Test{1}{1}%
  \Test{2}{2}%
  \Test{100}{100}%
  \Set
  \Test{0}{-1}%
  \Test{1}{-1}%
  \Test{100}{-1}%
  \bitsetReset{abc}%
  \Clear
  \Test{0}{0}%
  \Test{1}{1}%
  \Test{2}{2}%
  \Test{100}{100}%
  \Set
  \Test{0}{-1}%
  \Test{1}{-1}%
  \Test{100}{-1}%
  \bitsetSetBin{abc}{1}%
  \Clear
  \Test{0}{1}%
  \Test{1}{1}%
  \Test{2}{2}%
  \Test{100}{100}%
  \Set
  \Test{0}{0}%
  \Test{1}{-1}%
  \Test{100}{-1}%
  \bitsetSetBin{abc}{111000111000111000111}%
  \Clear
  \Test{0}{3}%
  \Test{1}{3}%
  \Test{2}{3}%
  \Test{3}{3}%
  \Test{4}{4}%
  \Test{5}{5}%
  \Test{6}{9}%
  \Test{7}{9}%
  \Test{8}{9}%
  \Test{9}{9}%
  \Test{10}{10}%
  \Test{11}{11}%
  \Test{12}{15}%
  \Test{13}{15}%
  \Test{14}{15}%
  \Test{15}{15}%
  \Test{16}{16}%
  \Test{17}{17}%
  \Test{18}{21}%
  \Test{19}{21}%
  \Test{20}{21}%
  \Test{21}{21}%
  \Test{22}{22}%
  \Test{100}{100}%
  \Set
  \Test{0}{0}%
  \Test{1}{1}%
  \Test{2}{2}%
  \Test{3}{6}%
  \Test{4}{6}%
  \Test{5}{6}%
  \Test{6}{6}%
  \Test{7}{7}%
  \Test{8}{8}%
  \Test{9}{12}%
  \Test{10}{12}%
  \Test{11}{12}%
  \Test{12}{12}%
  \Test{13}{13}%
  \Test{14}{14}%
  \Test{15}{18}%
  \Test{16}{18}%
  \Test{17}{18}%
  \Test{18}{18}%
  \Test{19}{19}%
  \Test{20}{20}%
  \Test{21}{-1}%
  \Test{22}{-1}%
  \Test{100}{-1}%
  \bitsetSetBin{abc}{1111111}%
  \Clear
  \Test{6}{7}%
  \Test{7}{7}%
  \Test{8}{8}%
  \Test{100}{100}%
  \Set
  \Test{6}{6}%
  \Test{7}{-1}%
  \Test{8}{-1}%
  \Test{100}{-1}%
  \bitsetSetBin{abc}{11111111}%
  \Clear
  \Test{7}{8}%
  \Test{8}{8}%
  \Test{9}{9}%
  \Test{100}{100}%
  \Set
  \Test{7}{7}%
  \Test{8}{-1}%
  \Test{9}{-1}%
  \Test{100}{-1}%
  \bitsetSetBin{abc}{111111111}%
  \Clear
  \Test{8}{9}%
  \Test{9}{9}%
  \Test{10}{10}%
  \Test{100}{100}%
  \Set
  \Test{8}{8}%
  \Test{9}{-1}%
  \Test{10}{-1}%
  \Test{100}{-1}%
  \bitsetSetBin{abc}{1111111111}%
  \Clear
  \Test{9}{10}%
  \Test{10}{10}%
  \Test{11}{11}%
  \Test{100}{100}%
  \Set
  \Test{9}{9}%
  \Test{10}{-1}%
  \Test{11}{-1}%
  \Test{100}{-1}%
\end{qstest}

\begin{qstest}{GetSetBitList}{GetSetBitList}
  \let\BS@abc\@undefined
  \expandafter\expandafter\expandafter\Expect
  \expandafter\expandafter\expandafter{%
    \bitsetGetSetBitList{abc}%
  }{}%
  \def\Test#1#2{%
    \bitsetSetBin{abc}{#1}%
    \expandafter\expandafter\expandafter\Expect
    \expandafter\expandafter\expandafter{%
      \bitsetGetSetBitList{abc}%
    }{#2}%
  }%
  \Test{0}{}%
  \Test{1}{0}%
  \Test{10}{1}%
  \Test{11}{0,1}%
  \Test{10110100}{2,4,5,7}%
  \Test{101101001010011}{0,1,4,6,9,11,12,14}%
\end{qstest}

\begin{qstest}{GetDec}{GetDec}
  \TestGetterUndefined\bitsetGetDec
  \def\Test#1#2{%
    \RevSet{abc}{#1}%
%<*noetex>
    \begingroup\expandafter\expandafter\expandafter\endgroup
%</noetex>
    \expandafter\expandafter\expandafter\Expect
    \expandafter\expandafter\expandafter{%
      \bitsetGetDec{abc}%
    }{#2}%
  }%
  \Test{0}{0}%
  \Test{1}{1}%
  \Test{10}{2}%
  \Test{11}{3}%
  \Test{100}{4}%
  \Test{101}{5}%
  \Test{110}{6}%
  \Test{111}{7}%
  \Test{1000}{8}%
  \Test{000111}{7}%
  \Test{1111111111111111%
        111111111111111}{2147483647}%
  \Test{0001111111111111111%
        111111111111111}{2147483647}%
  \Test{1000000000000000%
        0000000000000000}{2147483648}%
  \Test{1000000000000000%
        00000000000000000}{4294967296}%
  \Test{0001000000000000000%
        00000000000000000}{4294967296}%
  \Test{1100000000000000%
        00000000000000011}{6442450947}%
\end{qstest}

\begin{qstest}{Clear}{Clear}
  \def\Test#1#2#3{%
    \RevSet{abc}{#1}%
    \bitsetClear{abc}{#2}%
    \Expect*{\BS@abc}*{\Reverse#3!!}%
  }%
  \bitsetClear{abc}{2}%
  \RevCheck{abc}{0}%
  \TestErrorNegativeIndex{-1}{\bitsetClear{abc}{-1}}%
  \RevCheck{abc}{0}%
  \Test{0}{0}{0}%
  \Test{1}{0}{0}%
  \Test{111}{1}{101}%
  \Test{111}{30}{111}%
  \Test{0000111}{5}{0000111}% 111 would also be ok
  \Test{10000111}{5}{10000111}%
  \Test{1001001}{3}{1000001}%
\end{qstest}

\begin{qstest}{Set}{Set}
  \def\Test#1#2#3{%
    \RevSet{abc}{#1}%
    \bitsetSet{abc}{#2}%
    \Expect*{\BS@abc}*{\Reverse#3!!}%
  }%
  \bitsetSet{abc}{2}%
  \RevCheck{abc}{100}%
  \TestErrorNegativeIndex{-1}{\bitsetSet{abc}{-1}}%
  \RevCheck{abc}{100}%
  \Test{0}{0}{1}%
  \Test{1}{0}{1}%
  \Test{100}{1}{110}%
  \Test{111}{1}{111}%
  \Test{11}{1}{11}%
  \Test{11}{2}{111}%
  \Test{11}{3}{1011}%
  \Test{111}{10}{10000000111}%
  \Test{0000111}{5}{0100111}% 100111 would also be ok
  \Test{10000111}{5}{10100111}%
  \Test{1000001}{3}{1001001}%
  \Test{1001001}{3}{1001001}%
\end{qstest}

\begin{qstest}{Flip}{Flip}
  \def\Test#1#2#3{%
    \RevSet{abc}{#1}%
    \bitsetFlip{abc}{#2}%
    \Expect*{\BS@abc}*{\Reverse#3!!}%
  }%
  \bitsetFlip{abc}{2}%
  \RevCheck{abc}{100}%
  \TestErrorNegativeIndex{-1}{\bitsetFlip{abc}{-1}}%
  \RevCheck{abc}{100}%
  \Test{0}{0}{1}%
  \Test{1}{0}{0}%
  \Test{0}{2}{100}%
  \Test{100}{1}{110}%
  \Test{111}{1}{101}%
  \Test{11}{1}{1}%
  \Test{11}{2}{111}%
  \Test{11}{3}{1011}%
  \Test{111}{10}{10000000111}%
  \Test{0000111}{5}{0100111}% 100111 would also be ok
  \Test{10000111}{5}{10100111}%
  \Test{1000001}{3}{1001001}%
  \Test{1001001}{3}{1000001}%
  \Test{11111}{2}{11011}%
\end{qstest}

\begin{qstest}{SetValue}{SetValue}
  \def\Test#1#2{%
    \TestError{Invalid bit value (#2) not in range 0..1}{%
      \bitsetSetValue{abc}{#1}{#2}%
    }%
  }%
  \Test{0}{-1}%
  \Test{0}{2}%
  \Test{0}{10}%
  \def\Test#1#2#3{%
    \let\BS@abc\@undefined
    \bitsetSetValue{abc}{#1}{#2}%
    \bitsetSetBin{result}{#3}%
    \Expect*{\BS@abc}*{\BS@result}%
  }%
  \Test{0}{0}{0}%
  \Test{0}{1}{1}%
  \Test{1}{0}{0}%
  \Test{1}{1}{10}%
  \def\Test#1#2#3#4{%
    \bitsetSetBin{abc}{#1}%
    \bitsetSetBin{result}{#4}%
    \bitsetSetValue{abc}{#2}{#3}%
    \Expect*{\BS@abc}*{\BS@result}%
  }%
  \Test{0}{0}{0}{0}%
  \Test{0}{0}{0}{0}%
  \Test{0}{0}{1}{1}%
  \Test{0}{1}{0}{0}%
  \Test{0}{1}{1}{10}%
  \Test{1010}{2}{1}{1110}%
  \Test{1010}{4}{1}{11010}%
  \Test{1010}{6}{1}{1001010}%
  \Test{1010}{1}{0}{1000}%
  \Test{1010}{2}{0}{1010}%
  \Test{1010}{3}{0}{10}%
  \Test{1010}{4}{0}{1010}%
  \Test{1010}{6}{0}{1010}%
  \Test{1010}{2}{\csname iffalse\endcsname 0\else 1\fi}{1110}%
  \Test{1010}{1}{\csname iffalse\endcsname 1\else 0\fi}{1000}%
\end{qstest}

\begin{qstest}{IsDefined}{IsDefined}
  \let\BS@abc\@undefined
  \Expect*{\bitsetIsDefined{abc}{true}{false}}{false}%
  \bitsetReset{abc}%
  \Expect*{\bitsetIsDefined{abc}{true}{false}}{true}%
\end{qstest}

\begin{qstest}{IsEmpty}{IsEmpty}
  \let\BS@abc\@undefined
  \Expect*{\bitsetIsEmpty{abc}{true}{false}}{true}%
  \bitsetReset{abc}%
  \Expect*{\bitsetIsEmpty{abc}{true}{false}}{true}%
  \bitsetSet{abc}{1}%
  \Expect*{\bitsetIsEmpty{abc}{true}{false}}{false}%
\end{qstest}

\begin{qstest}{Equals}{Equals}
  \def\Test#1#2#3{%
    \Expect*{\bitsetEquals{#1}{#2}{true}{false}}{#3}%
  }%
  \let\BS@abc\@undefined
  \Test{abc}{abc}{true}%
  \Test{abc}{foo}{true}%
  \Test{foo}{abc}{true}%
  \bitsetReset{abc}%
  \Test{abc}{abc}{true}%
  \Test{abc}{foo}{false}%
  \Test{foo}{abc}{false}%
  \bitsetReset{foo}%
  \Test{abc}{foo}{true}%
  \Test{foo}{abc}{true}%
  \bitsetSet{abc}{4}%
  \Test{abc}{foo}{false}%
  \Test{foo}{abc}{false}%
  \bitsetFlip{foo}{4}%
  \Test{abc}{foo}{true}%
  \Test{foo}{abc}{true}%
\end{qstest}

\begin{qstest}{Intersects}{Intersects}
  \def\Test#1{%
    \Expect*{\bitsetIntersects{abc}{foo}{true}{false}}{#1}%
  }%
  \let\BS@abc\@undefined
  \let\BS@foo\@undefined
  \Test{false}%
  \Set{abc}{0}%
  \Test{false}%
  \Set{foo}{0}%
  \Test{false}%
  \let\BS@abc\@undefined
  \Test{false}%
  \Set{foo}{1}%
  \Test{false}%
  \Set{abc}{0}%
  \Test{false}%
  \Set{abc}{1}%
  \Test{true}%
  \let\BS@foo\@undefined
  \Test{false}%
  \Set{foo}{0}%
  \Test{false}%
  \def\Test#1#2#3{%
    \bitsetSetBin{abc}{#1}%
    \bitsetSetBin{foo}{#2}%
    \Expect*{\bitsetIntersects{abc}{foo}{true}{false}}{#3}%
  }%
  \Test{1010}{0101}{false}%
  \Test{0}{10}{false}%
  \Test{1}{11}{true}%
  \Test{11}{1}{true}%
  \Test{10}{1}{false}%
\end{qstest}

\begin{qstest}{And/AndNot/Or/Xor}{And/AndNot/Or/Xor}
  \def\@Test#1#2#3#4#5{%
    \begingroup
      #5%
      \begingroup
        \let\BS@foo\@undefined
        \csname bitset#1\endcsname{abc}{foo}%
        \CheckUndef{foo}%
        \Check{abc}{#2}%
      \endgroup
      \begingroup
        \bitsetReset{foo}%
        \csname bitset#1\endcsname{abc}{foo}%
        \Check{foo}{0}%
        \Check{abc}{#3}%
      \endgroup
      \begingroup
        \def\BS@foo{0101}%
        \csname bitset#1\endcsname{abc}{foo}%
        \Check{foo}{0101}%
        \Check{abc}{#4}%
      \endgroup
    \endgroup
  }%
  \def\Test#1{%
    \def\Op{#1}%
    \Test@
  }%
  \def\Test@#1#2#3#4#5#6#7#8#9{%
    \@Test\Op{#1}{#2}{#3}{%
      \let\BS@abc\@undefined
    }%
    \@Test\Op{#4}{#5}{#6}{%
      \bitsetReset{abc}%
    }%
    \@Test\Op{#7}{#8}{#9}{%
      \def\BS@abc{1001}%
    }%
  }%
  \Test{And}%
       {0}{0}{0}%
       {0}{0}{0}%
       {0}{0}{0001}%
  \Test{AndNot}%
       {0}{0}{0}%
       {0}{0}{0}%
       {1001}{1001}{1}%
  \Test{Or}%
       {0}{0}{0101}%
       {0}{0}{0101}%
       {1001}{1001}{1101}%
  \Test{Xor}%
       {0}{0}{0101}%
       {0}{0}{0101}%
       {1001}{1001}{11}%
  \def\Test#1#2#3{%
    \bitsetSetBin{abc}{#1}%
    \bitsetSetBin{foo}{#2}%
    \csname bitset\Op\endcsname{abc}{foo}%
    \RevCheck{foo}{#2}%
    \RevCheck{abc}{#3}%
  }%
  \def\Op{And}%
  \Test{1}{111}{1}%
  \Test{111}{1}{1}%
  \Test{10}{111}{10}%
  \Test{111}{10}{10}%
  \Test{111}{1000}{0}%
  \Test{1000}{111}{0}%
  \def\Op{AndNot}%
  \Test{1010}{11}{1000}%
  \Test{100}{100}{0}%
  \Test{111}{1111}{0}%
  \Test{100}{111}{0}%
  \def\Op{Or}%
  \Test{0}{0}{0}%
  \Test{1}{0}{1}%
  \Test{0}{1}{1}%
  \Test{1}{1}{1}%
  \Test{1000}{10}{1010}%
  \Test{10}{1000}{1010}%
  \def\Op{Xor}%
  \Test{0}{0}{0}%
  \Test{1}{0}{1}%
  \Test{0}{1}{1}%
  \Test{1}{1}{0}%
  \Test{1000}{10}{1010}%
  \Test{10}{1000}{1010}%
  \Test   {110011001100}%
       {111000111000111}%
       {111110100001011}%
  \Test{111000111000111}%
          {110011001100}%
       {111110100001011}%
\end{qstest}

\begin{qstest}{GetUndef}{GetUndef, GetBin, GetOct, GetHex}
  \def\TestUndef#1#2{%
    \let\BS@abc\@undefined
    \expandafter\expandafter\expandafter\Expect
    \expandafter\expandafter\expandafter{%
      \x{abc}{#1}%
    }{#2}%
  }%
  \let\x\bitsetGetBin
  \TestUndef{-1}{0}%
  \TestUndef{0}{0}%
  \TestUndef{1}{0}%
  \TestUndef{2}{00}%
  \TestUndef{8}{00000000}%
  \let\x\bitsetGetOct
  \TestUndef{-1}{0}%
  \TestUndef{0}{0}%
  \TestUndef{1}{0}%
  \TestUndef{2}{0}%
  \TestUndef{3}{0}%
  \TestUndef{4}{00}%
  \TestUndef{5}{00}%
  \TestUndef{6}{00}%
  \TestUndef{7}{000}%
  \TestUndef{8}{000}%
  \TestUndef{9}{000}%
  \TestUndef{10}{0000}%
  \let\x\bitsetGetHex
  \TestUndef{-1}{0}%
  \TestUndef{0}{0}%
  \TestUndef{1}{0}%
  \TestUndef{2}{0}%
  \TestUndef{3}{0}%
  \TestUndef{4}{0}%
  \TestUndef{5}{00}%
  \TestUndef{6}{00}%
  \TestUndef{7}{00}%
  \TestUndef{8}{00}%
  \TestUndef{9}{000}%
  \TestUndef{10}{000}%
  \TestUndef{12}{000}%
  \TestUndef{13}{0000}%
  \TestUndef{16}{0000}%
  \TestUndef{17}{00000}%
\end{qstest}

\begin{qstest}{SetBin}{SetBin}
  \def\Test#1#2{%
    \let\BS@abc\@undefined
    \bitsetSetBin{abc}{#1}%
    \expandafter\Expect\expandafter{\BS@abc}{#2}%
  }%
  \Test{}{0}%
  \Test{0}{0}%
  \Test{1}{1}%
  \Test{10}{01}%
  \Test{11}{11}%
  \Test{010}{01}%
  \Test{011}{11}%
  \Test{0010}{01}%
  \Test{1010}{0101}%
\end{qstest}

\begin{qstest}{SetOct}{SetOct}
  \def\Test#1#2{%
    \bitsetSetOct{abc}{#1}%
    \expandafter\Expect\expandafter{\BS@abc}{#2}%
  }%
  \Test{}{0}%
  \Test{0}{0}%
  \Test{000}{0}%
  \Test{1}{1}%
  \Test{001}{1}%
  \Test{010}{0001}%
  \Test{020}{00001}%
  \Test{42}{010001}%
  \Test{377}{11111111}%
  \Test{0377}{11111111}%
  \Test{76543210}{000100010110001101011111}%
  \Test{ 0 7 0 7 1 }{100111000111}%
\end{qstest}

\begin{qstest}{SetHex}{SetHex}
  \def\Test#1#2{%
    \bitsetSetHex{abc}{#1}%
    \expandafter\Expect\expandafter{\BS@abc}{#2}%
  }%
  \Test{}{0}%
  \Test{0}{0}%
  \Test{000}{0}%
  \Test{1}{1}%
  \Test{001}{1}%
  \Test{010}{00001}%
  \Test{020}{000001}%
  \Test{42}{0100001}%
  \Test{3F}{111111}%
  \Test{03F}{111111}%
  \Test{43210}{0000100001001100001}%
  \Test{98765}{10100110111000011001}%
  \Test{FEDCBA}{010111010011101101111111}%
  \Test{ 0 F 0 F 1 }{1000111100001111}%
\end{qstest}

\begin{qstest}{SetDec}{SetDec}
  \def\Test#1#2{%
    \bitsetSetDec{abc}{#1}%
    \expandafter\Expect\expandafter{\BS@abc}{#2}%
  }%
  \Test{}{0}%
  \Test{0}{0}%
  \Test{000}{0}%
  \Test{1}{1}%
  \Test{7}{111}%
  \Test{8}{0001}%
  \Test{001}{1}%
  \Test{010}{0101}%
  \Test{020}{00101}%
  \Test{53}{101011}%
  \Test{255}{11111111}%
  \Test{256}{000000001}%
  \Test{999999999}{111111111001001101011001110111}%
  \Test{1000000000}{000000000101001101011001110111}%
  \Test{4210987654}{01100001010010010111111101011111}%
  \Test{2147483647}{1111111111111111111111111111111}%
  \Test{2147483648}{00000000000000000000000000000001}%
\end{qstest}

\begin{qstest}{GetBin}{GetBin}
  \def\TestUndef#1#2{%
    \let\BS@abc\@undefined
    \expandafter\expandafter\expandafter\Expect
    \expandafter\expandafter\expandafter{%
      \bitsetGetBin{abc}{#1}%
    }{#2}%
  }%
  \TestUndef{-1}{0}%
  \TestUndef{0}{0}%
  \TestUndef{1}{0}%
  \TestUndef{2}{00}%
  \TestUndef{8}{00000000}%
  \def\Test#1#2{%
    \bitsetSetBin{abc}{#2}%
    \expandafter\expandafter\expandafter\Expect
    \expandafter\expandafter\expandafter{%
      \bitsetGetBin{abc}{#1}%
    }{#2}%
  }%
  \Test{-1}{0}%
  \Test{0}{0}%
  \Test{1}{0}%
  \Test{1}{1}%
  \Test{2}{01}%
  \Test{2}{10}%
  \Test{3}{010}%
  \Test{2}{00}%
  \Test{2}{01}%
  \Test{8}{00101100}%
  \Test{2}{10101}%
  \Test{-100}{11011}%
\end{qstest}

\begin{qstest}{GetOct}{GetOct}
  \def\Test#1#2#3{%
    \edef\x{\zap@space#1 \@empty}%
    \edef\x{\noexpand\bitsetSetBin{abc}{\x}}%
    \x
    \expandafter\expandafter\expandafter\Expect
    \expandafter\expandafter\expandafter{%
      \bitsetGetOct{abc}{#2}%
    }{#3}%
  }%
  \Test{111 110 101 100 011 010 001 000}{0}{76543210}%
  \Test{000 111}{0}{7}%
  \Test{101 000}{-1}{50}%
  \Test{111}{-1}{7}%
  \Test{111}{0}{7}%
  \Test{111}{1}{7}%
  \Test{111}{3}{7}%
  \Test{111}{4}{07}%
  \Test{111}{6}{07}%
  \Test{111}{7}{007}%
  \Test{111 010}{6}{72}%
  \Test{111 010}{7}{072}%
  \Test{011 111}{0}{37}%
  \Test{011 111}{6}{37}%
  \Test{011 111}{7}{037}%
  \Test{001 111}{0}{17}%
  \Test{001 111}{6}{17}%
  \Test{001 111}{7}{017}%
\end{qstest}

\begin{qstest}{GetHex}{GetHex}
  \def\Test#1#2#3{%
    \bitsetSetBin{abc}{#1}%
    \expandafter\expandafter\expandafter\Expect
    \expandafter\expandafter\expandafter{%
      \bitsetGetHex{abc}{#2}%
    }{#3}%
  }%
  \Test{1111 1110 1101 1100 1011 1010 1001 1000}{0}{FEDCBA98}%
  \Test{0111 0110 0101 0100 0011 0010 0001 0000}{0}{76543210}%
  \Test{0000 1111}{0}{F}%
  \Test{0101 0000}{-1}{50}%
  \Test{1111}{-1}{F}%
  \Test{1111}{0}{F}%
  \Test{1111}{1}{F}%
  \Test{1111}{4}{F}%
  \Test{1111}{5}{0F}%
  \Test{1111}{8}{0F}%
  \Test{1111}{9}{00F}%
  \Test{1111 0010}{8}{F2}%
  \Test{1111 0010}{9}{0F2}%
  \Test{0111 1111}{0}{7F}%
  \Test{0111 1111}{8}{7F}%
  \Test{0111 1111}{9}{07F}%
  \Test{0011 1111}{0}{3F}%
  \Test{0011 1111}{8}{3F}%
  \Test{0011 1111}{9}{03F}%
  \Test{0001 1111}{0}{1F}%
  \Test{0001 1111}{8}{1F}%
  \Test{0001 1111}{9}{01F}%
\end{qstest}

\begin{qstest}{Range}{Range}
  \TestError{%
    Wrong index numbers in range [9..8]\MessageBreak% hash-ok
    for clear/set/flip on bit set `abc'.\MessageBreak
    The lower index exceeds the upper index.\MessageBreak
    Canceling the operation as error recovery%
  }{%
    \bitsetSetRange{abc}{9}{8}%
  }%
  \def\TestErrorNegInd#1#2#3#4#5#6{%
    \TestError{%
      Negative index in range [#2..#3]\MessageBreak % hash-ok
      for \string\bitset #1Range on bit set `abc'.\MessageBreak
      Using [#4..#5] as error recovery% hash-ok
    }{%
      \csname bitset#1Range\endcsname{abc}{#2}{#3}%
      \global\let\BS@global\BS@abc
    }%
    \Check{global}{#6}%
  }%
  \Set{abc}{111}%
  \TestErrorNegInd{Clear}{-1}{0}{0}{0}{111}%
  \TestErrorNegInd{Clear}{0}{-1}{0}{0}{111}%
  \TestErrorNegInd{Clear}{-2}{2}{0}{2}{001}%
  \bitsetReset{abc}%
  \TestErrorNegInd{Set}{-1}{0}{0}{0}{0}%
  \TestErrorNegInd{Set}{0}{-1}{0}{0}{0}%
  \TestErrorNegInd{Set}{-2}{2}{0}{2}{11}%
  \Set{abc}{101}%
  \TestErrorNegInd{Flip}{-1}{0}{0}{0}{101}%
  \TestErrorNegInd{Flip}{0}{-1}{0}{0}{101}%
  \TestErrorNegInd{Flip}{-2}{2}{0}{2}{011}%
  \def\Test#1#2#3#4{%
    \bitsetSetBin{abc}{#1}%
    \csname bitset\TestOp Range\endcsname{abc}{#2}{#3}%
    \Expect*{\bitsetGetBin{abc}{0}}{#4}%
  }%
  \def\TestOp{Clear}%
  \Test{0}{0}{1}{0}%
  \Test{1111}{1}{2}{1101}%
  \Test{1111}{1}{3}{1001}%
  \Test{1111111100000000}{12}{14}{1100111100000000}%
  \def\TestOp{Set}%
  \Test{0}{0}{1}{1}%
  \Test{1000}{1}{2}{1010}%
  \Test{0}{1}{2}{10}%
  \Test{1}{12}{15}{111000000000001}%
  \Test{1111}{1}{3}{1111}%
  \Test{1000000000000000}{12}{14}{1011000000000000}%
  \def\TestOp{Flip}%
  \Test{0}{0}{1}{1}%
  \Test{1}{0}{1}{0}%
  \Test{10101010}{1}{5}{10110100}%
  \def\Test#1#2#3#4#5{%
    \bitsetSetBin{abc}{#1}%
    \bitsetSetValueRange{abc}{#2}{#3}{#4}%
    \Expect*{\bitsetGetBin{abc}{0}}{#5}%
  }%
  \Test{0}{0}{1}{0}{0}%
  \Test{0}{0}{1}{1}{1}%
  \Test{1010}{1}{3}{0}{1000}%
  \Test{1010}{1}{3}{1}{1110}%
\end{qstest}

\begin{qstest}{ShiftLeft/ShiftRight}{ShiftLeft/ShiftRight}
  \def\@Test#1#2{%
    \let\BS@abc\@undefined
    \csname bitsetShift#1\endcsname{abc}{#2}%
    \Expect*{\BS@abc}{0}%
  }%
  \def\Test#1{%
    \@Test{Left}{#1}%
    \@Test{Right}{#1}%
  }%
  \Test{-16}%
  \Test{-1}%
  \Test{0}%
  \Test{1}%
  \Test{16}%
  \def\Test#1#2#3{%
    \bitsetSetBin{abc}{#1}%
    \bitsetSetBin{result}{#3}%
    \csname bitsetShift\Op\endcsname{abc}{#2}%
    \Expect*{\bitsetGetBin{abc}{0}}*{\bitsetGetBin{result}{0}}%
  }%
  \def\Op{Left}%
  \Test{0}{0}{0}%
  \Test{0}{1}{0}%
  \Test{0}{-1}{0}%
  \Test{1}{0}{1}%
  \Test{1}{1}{10}%
  \Test{1}{-1}{0}%
  \Test{10}{1}{100}%
  \Test{10}{-1}{1}%
  \Test{1}{32}{100000000000000000000000000000000}%
  \Test{1}{-100}{0}%
  \def\Op{Right}%
  \Test{0}{0}{0}%
  \Test{0}{1}{0}%
  \Test{0}{-1}{0}%
  \Test{1}{0}{1}%
  \Test{1}{1}{0}%
  \Test{1}{-1}{10}%
  \Test{10}{1}{1}%
  \Test{10}{-1}{100}%
  \Test{1}{-32}{100000000000000000000000000000000}%
  \Test{1}{100}{0}%
  \Test{110110110110110}{10}{11011}%
  \Test{110110110110110}{100}{0}%
  \Test{1}{100000}{0}%
\end{qstest}

\begin{qstest}{Profile: Set}{Profile: Set}
  \bitsetSet{abc}{4095}%
  \global\let\BS@global\BS@abc
\end{qstest}

\begin{qstest}{Profile: Get}{Profile: Get}
  \edef\x{\bitsetGet{global}{4095}}%
\end{qstest}

\begin{document}
\end{document}
%</test2>
%    \end{macrocode}
%
% \section{Installation}
%
% \subsection{Download}
%
% \paragraph{Package.} This package is available on
% CTAN\footnote{\url{ftp://ftp.ctan.org/tex-archive/}}:
% \begin{description}
% \item[\CTAN{macros/latex/contrib/oberdiek/bitset.dtx}] The source file.
% \item[\CTAN{macros/latex/contrib/oberdiek/bitset.pdf}] Documentation.
% \end{description}
%
%
% \paragraph{Bundle.} All the packages of the bundle `oberdiek'
% are also available in a TDS compliant ZIP archive. There
% the packages are already unpacked and the documentation files
% are generated. The files and directories obey the TDS standard.
% \begin{description}
% \item[\CTAN{install/macros/latex/contrib/oberdiek.tds.zip}]
% \end{description}
% \emph{TDS} refers to the standard ``A Directory Structure
% for \TeX\ Files'' (\CTAN{tds/tds.pdf}). Directories
% with \xfile{texmf} in their name are usually organized this way.
%
% \subsection{Bundle installation}
%
% \paragraph{Unpacking.} Unpack the \xfile{oberdiek.tds.zip} in the
% TDS tree (also known as \xfile{texmf} tree) of your choice.
% Example (linux):
% \begin{quote}
%   |unzip oberdiek.tds.zip -d ~/texmf|
% \end{quote}
%
% \paragraph{Script installation.}
% Check the directory \xfile{TDS:scripts/oberdiek/} for
% scripts that need further installation steps.
% Package \xpackage{attachfile2} comes with the Perl script
% \xfile{pdfatfi.pl} that should be installed in such a way
% that it can be called as \texttt{pdfatfi}.
% Example (linux):
% \begin{quote}
%   |chmod +x scripts/oberdiek/pdfatfi.pl|\\
%   |cp scripts/oberdiek/pdfatfi.pl /usr/local/bin/|
% \end{quote}
%
% \subsection{Package installation}
%
% \paragraph{Unpacking.} The \xfile{.dtx} file is a self-extracting
% \docstrip\ archive. The files are extracted by running the
% \xfile{.dtx} through \plainTeX:
% \begin{quote}
%   \verb|tex bitset.dtx|
% \end{quote}
%
% \paragraph{TDS.} Now the different files must be moved into
% the different directories in your installation TDS tree
% (also known as \xfile{texmf} tree):
% \begin{quote}
% \def\t{^^A
% \begin{tabular}{@{}>{\ttfamily}l@{ $\rightarrow$ }>{\ttfamily}l@{}}
%   bitset.sty & tex/generic/oberdiek/bitset.sty\\
%   bitset.pdf & doc/latex/oberdiek/bitset.pdf\\
%   test/bitset-test1.tex & doc/latex/oberdiek/test/bitset-test1.tex\\
%   test/bitset-test2.tex & doc/latex/oberdiek/test/bitset-test2.tex\\
%   test/bitset-test3.tex & doc/latex/oberdiek/test/bitset-test3.tex\\
%   bitset.dtx & source/latex/oberdiek/bitset.dtx\\
% \end{tabular}^^A
% }^^A
% \sbox0{\t}^^A
% \ifdim\wd0>\linewidth
%   \begingroup
%     \advance\linewidth by\leftmargin
%     \advance\linewidth by\rightmargin
%   \edef\x{\endgroup
%     \def\noexpand\lw{\the\linewidth}^^A
%   }\x
%   \def\lwbox{^^A
%     \leavevmode
%     \hbox to \linewidth{^^A
%       \kern-\leftmargin\relax
%       \hss
%       \usebox0
%       \hss
%       \kern-\rightmargin\relax
%     }^^A
%   }^^A
%   \ifdim\wd0>\lw
%     \sbox0{\small\t}^^A
%     \ifdim\wd0>\linewidth
%       \ifdim\wd0>\lw
%         \sbox0{\footnotesize\t}^^A
%         \ifdim\wd0>\linewidth
%           \ifdim\wd0>\lw
%             \sbox0{\scriptsize\t}^^A
%             \ifdim\wd0>\linewidth
%               \ifdim\wd0>\lw
%                 \sbox0{\tiny\t}^^A
%                 \ifdim\wd0>\linewidth
%                   \lwbox
%                 \else
%                   \usebox0
%                 \fi
%               \else
%                 \lwbox
%               \fi
%             \else
%               \usebox0
%             \fi
%           \else
%             \lwbox
%           \fi
%         \else
%           \usebox0
%         \fi
%       \else
%         \lwbox
%       \fi
%     \else
%       \usebox0
%     \fi
%   \else
%     \lwbox
%   \fi
% \else
%   \usebox0
% \fi
% \end{quote}
% If you have a \xfile{docstrip.cfg} that configures and enables \docstrip's
% TDS installing feature, then some files can already be in the right
% place, see the documentation of \docstrip.
%
% \subsection{Refresh file name databases}
%
% If your \TeX~distribution
% (\teTeX, \mikTeX, \dots) relies on file name databases, you must refresh
% these. For example, \teTeX\ users run \verb|texhash| or
% \verb|mktexlsr|.
%
% \subsection{Some details for the interested}
%
% \paragraph{Attached source.}
%
% The PDF documentation on CTAN also includes the
% \xfile{.dtx} source file. It can be extracted by
% AcrobatReader 6 or higher. Another option is \textsf{pdftk},
% e.g. unpack the file into the current directory:
% \begin{quote}
%   \verb|pdftk bitset.pdf unpack_files output .|
% \end{quote}
%
% \paragraph{Unpacking with \LaTeX.}
% The \xfile{.dtx} chooses its action depending on the format:
% \begin{description}
% \item[\plainTeX:] Run \docstrip\ and extract the files.
% \item[\LaTeX:] Generate the documentation.
% \end{description}
% If you insist on using \LaTeX\ for \docstrip\ (really,
% \docstrip\ does not need \LaTeX), then inform the autodetect routine
% about your intention:
% \begin{quote}
%   \verb|latex \let\install=y\input{bitset.dtx}|
% \end{quote}
% Do not forget to quote the argument according to the demands
% of your shell.
%
% \paragraph{Generating the documentation.}
% You can use both the \xfile{.dtx} or the \xfile{.drv} to generate
% the documentation. The process can be configured by the
% configuration file \xfile{ltxdoc.cfg}. For instance, put this
% line into this file, if you want to have A4 as paper format:
% \begin{quote}
%   \verb|\PassOptionsToClass{a4paper}{article}|
% \end{quote}
% An example follows how to generate the
% documentation with pdf\LaTeX:
% \begin{quote}
%\begin{verbatim}
%pdflatex bitset.dtx
%makeindex -s gind.ist bitset.idx
%pdflatex bitset.dtx
%makeindex -s gind.ist bitset.idx
%pdflatex bitset.dtx
%\end{verbatim}
% \end{quote}
%
% \section{Catalogue}
%
% The following XML file can be used as source for the
% \href{http://mirror.ctan.org/help/Catalogue/catalogue.html}{\TeX\ Catalogue}.
% The elements \texttt{caption} and \texttt{description} are imported
% from the original XML file from the Catalogue.
% The name of the XML file in the Catalogue is \xfile{bitset.xml}.
%    \begin{macrocode}
%<*catalogue>
<?xml version='1.0' encoding='us-ascii'?>
<!DOCTYPE entry SYSTEM 'catalogue.dtd'>
<entry datestamp='$Date$' modifier='$Author$' id='bitset'>
  <name>bitset</name>
  <caption>Handle bit-vector datatype.</caption>
  <authorref id='auth:oberdiek'/>
  <copyright owner='Heiko Oberdiek' year='2007,2011'/>
  <license type='lppl1.3'/>
  <version number='1.1'/>
  <description>
    This package defines and implements the data type bit set,
    a vector of bits. The size of the vector may grow dynamically.
    Individual bits can be manipulated.
    <p/>
    The package is part of the <xref refid='oberdiek'>oberdiek</xref> bundle.
  </description>
  <documentation details='Package documentation'
      href='ctan:/macros/latex/contrib/oberdiek/bitset.pdf'/>
  <ctan file='true' path='/macros/latex/contrib/oberdiek/bitset.dtx'/>
  <miktex location='oberdiek'/>
  <texlive location='oberdiek'/>
  <install path='/macros/latex/contrib/oberdiek/oberdiek.tds.zip'/>
</entry>
%</catalogue>
%    \end{macrocode}
%
% \begin{History}
%   \begin{Version}{2007/09/28 v1.0}
%   \item
%     First version.
%   \end{Version}
%   \begin{Version}{2011/01/30 v1.1}
%   \item
%     Already loaded package files are not input in \hologo{plainTeX}.
%   \end{Version}
% \end{History}
%
% \PrintIndex
%
% \Finale
\endinput

%        (quote the arguments according to the demands of your shell)
%
% Documentation:
%    (a) If bitset.drv is present:
%           latex bitset.drv
%    (b) Without bitset.drv:
%           latex bitset.dtx; ...
%    The class ltxdoc loads the configuration file ltxdoc.cfg
%    if available. Here you can specify further options, e.g.
%    use A4 as paper format:
%       \PassOptionsToClass{a4paper}{article}
%
%    Programm calls to get the documentation (example):
%       pdflatex bitset.dtx
%       makeindex -s gind.ist bitset.idx
%       pdflatex bitset.dtx
%       makeindex -s gind.ist bitset.idx
%       pdflatex bitset.dtx
%
% Installation:
%    TDS:tex/generic/oberdiek/bitset.sty
%    TDS:doc/latex/oberdiek/bitset.pdf
%    TDS:doc/latex/oberdiek/test/bitset-test1.tex
%    TDS:doc/latex/oberdiek/test/bitset-test2.tex
%    TDS:doc/latex/oberdiek/test/bitset-test3.tex
%    TDS:source/latex/oberdiek/bitset.dtx
%
%<*ignore>
\begingroup
  \catcode123=1 %
  \catcode125=2 %
  \def\x{LaTeX2e}%
\expandafter\endgroup
\ifcase 0\ifx\install y1\fi\expandafter
         \ifx\csname processbatchFile\endcsname\relax\else1\fi
         \ifx\fmtname\x\else 1\fi\relax
\else\csname fi\endcsname
%</ignore>
%<*install>
\input docstrip.tex
\Msg{************************************************************************}
\Msg{* Installation}
\Msg{* Package: bitset 2011/01/30 v1.1 Handle bit-vector datatype (HO)}
\Msg{************************************************************************}

\keepsilent
\askforoverwritefalse

\let\MetaPrefix\relax
\preamble

This is a generated file.

Project: bitset
Version: 2011/01/30 v1.1

Copyright (C) 2007, 2011 by
   Heiko Oberdiek <heiko.oberdiek at googlemail.com>

This work may be distributed and/or modified under the
conditions of the LaTeX Project Public License, either
version 1.3c of this license or (at your option) any later
version. This version of this license is in
   http://www.latex-project.org/lppl/lppl-1-3c.txt
and the latest version of this license is in
   http://www.latex-project.org/lppl.txt
and version 1.3 or later is part of all distributions of
LaTeX version 2005/12/01 or later.

This work has the LPPL maintenance status "maintained".

This Current Maintainer of this work is Heiko Oberdiek.

The Base Interpreter refers to any `TeX-Format',
because some files are installed in TDS:tex/generic//.

This work consists of the main source file bitset.dtx
and the derived files
   bitset.sty, bitset.pdf, bitset.ins, bitset.drv, bitset-test1.tex,
   bitset-test2.tex, bitset-test3.tex.

\endpreamble
\let\MetaPrefix\DoubleperCent

\generate{%
  \file{bitset.ins}{\from{bitset.dtx}{install}}%
  \file{bitset.drv}{\from{bitset.dtx}{driver}}%
  \usedir{tex/generic/oberdiek}%
  \file{bitset.sty}{\from{bitset.dtx}{package}}%
  \usedir{doc/latex/oberdiek/test}%
  \file{bitset-test1.tex}{\from{bitset.dtx}{test1}}%
  \file{bitset-test2.tex}{\from{bitset.dtx}{test2,etex}}%
  \file{bitset-test3.tex}{\from{bitset.dtx}{test2,noetex}}%
  \nopreamble
  \nopostamble
  \usedir{source/latex/oberdiek/catalogue}%
  \file{bitset.xml}{\from{bitset.dtx}{catalogue}}%
}

\catcode32=13\relax% active space
\let =\space%
\Msg{************************************************************************}
\Msg{*}
\Msg{* To finish the installation you have to move the following}
\Msg{* file into a directory searched by TeX:}
\Msg{*}
\Msg{*     bitset.sty}
\Msg{*}
\Msg{* To produce the documentation run the file `bitset.drv'}
\Msg{* through LaTeX.}
\Msg{*}
\Msg{* Happy TeXing!}
\Msg{*}
\Msg{************************************************************************}

\endbatchfile
%</install>
%<*ignore>
\fi
%</ignore>
%<*driver>
\NeedsTeXFormat{LaTeX2e}
\ProvidesFile{bitset.drv}%
  [2011/01/30 v1.1 Handle bit-vector datatype (HO)]%
\documentclass{ltxdoc}
\usepackage{holtxdoc}[2011/11/22]
\usepackage{amstext}
\usepackage{ifluatex}
\ifluatex
\else
  \usepackage[T1]{fontenc}%
  \usepackage{lmodern}%
\fi
\begin{document}
  \DocInput{bitset.dtx}%
\end{document}
%</driver>
% \fi
%
% \CheckSum{3510}
%
% \CharacterTable
%  {Upper-case    \A\B\C\D\E\F\G\H\I\J\K\L\M\N\O\P\Q\R\S\T\U\V\W\X\Y\Z
%   Lower-case    \a\b\c\d\e\f\g\h\i\j\k\l\m\n\o\p\q\r\s\t\u\v\w\x\y\z
%   Digits        \0\1\2\3\4\5\6\7\8\9
%   Exclamation   \!     Double quote  \"     Hash (number) \#
%   Dollar        \$     Percent       \%     Ampersand     \&
%   Acute accent  \'     Left paren    \(     Right paren   \)
%   Asterisk      \*     Plus          \+     Comma         \,
%   Minus         \-     Point         \.     Solidus       \/
%   Colon         \:     Semicolon     \;     Less than     \<
%   Equals        \=     Greater than  \>     Question mark \?
%   Commercial at \@     Left bracket  \[     Backslash     \\
%   Right bracket \]     Circumflex    \^     Underscore    \_
%   Grave accent  \`     Left brace    \{     Vertical bar  \|
%   Right brace   \}     Tilde         \~}
%
% \GetFileInfo{bitset.drv}
%
% \title{The \xpackage{bitset} package}
% \date{2011/01/30 v1.1}
% \author{Heiko Oberdiek\\\xemail{heiko.oberdiek at googlemail.com}}
%
% \maketitle
%
% \begin{abstract}
% This package defines and implements the data type bit set,
% a vector of bits. The size of the vector may grow dynamically.
% Individual bits can be manipulated.
% \end{abstract}
%
% \tableofcontents
%
% \clearpage
% \section{Documentation}
%
% \subsection{Introduction}
%
% Annotations in the PDF format know entries whose values are
% integers. This numbers are interpreted as set of flags
% specifying properties. For example, annotation dictionaries
% can have a key |/F|. The bits of its integer value
% are interpreted the following way:
% \begin{quote}
% \begin{tabular}{@{}l|l@{}}
%   Bit position & Property name\\
%   \hline
%   1 & Invisible\\
%   2 & Hidden\\
%   3 & Print\\
%   4 & NoZoom\\
%   5 & NoRotate\\
%   6 & NoView\\
%   7 & ReadOnly\\
%   \dots & \dots
% \end{tabular}
% \end{quote}
% Now, let's see how these values are set in package \xpackage{hyperref}
% before it uses this package (before v6.77a):
%\begin{quote}
%\begin{verbatim}
%\ifFld@hidden /F 6\else /F 4\fi
%\end{verbatim}
%\end{quote}
% Where are the other flags? The following example for key |/Ff|
% in a widget annotation
% supports at least three properties:
%\begin{quote}
%\begin{verbatim}
%\ifFld@multiline
%  \ifFld@readonly /Ff 4097\else /Ff 4096\fi
%\else
%  \ifFld@password
%    \ifFld@readonly /Ff 8193\else /Ff 8192\fi
%  \else
%    \ifFld@readonly /Ff 1\fi
%  \fi
%\fi
%\end{verbatim}
%\end{quote}
% But you see the point. It would be a nightmare to continue
% this way in supporting the missing flag settings. This kind
% of integers may have up to 32 bits.
%
% Therefore I wanted a data structure for setting and
% clearing individual bits. Also it should provide
% an export as decimal number. The snipsets above are
% executed in expansion contexts without \TeX's stomach
% commands. It would be convenient to have an expandable
% conversion from the data structure to the integer that
% gets written to the PDF file.
%
% This package \xpackage{bitset} implements such a data
% structure. The interface is quite close to Java's
% class |BitSet| in order not to learn to many interfaces
% for the same kind of data structure.
%
% \subsection{Glossary}
%
% \begin{description}
% \item[Bit set:]
%   A bit set is a vector of bits or flags.
%   The vector size is unlimited and
%   grows dynamically. An undefined bit set is treated as bit
%   set where all bits are cleared.
%
%   Bit sets are addressed by name. A name should consists of
%   letters or digits. Technically it must survive \cs{csname},
%   see \LaTeX's environment names for other names with such
%   a constraint. Package \xpackage{babel}'s shorthands are
%   not supported due to technical reasons. Shorthand support
%   would break expandable operations.
% \item[Size:]
%   A size of a bit set is the number of bits in use.
%   It's the number of the highest index, incremented by one.
%   Sizes are in the range |0| up to |2147483647|, the highest
%   number supported by \TeX.
% \item[Index:]
%   Bit positions in a bit set are addressed by an index number.
%   The bit vector is zero based. The first and least significant
%   bit is addressed by index |0| and the highest possible bit
%   by |2147483646|.
% \item[Bit:]
%   A bit is enoded as |0| for cleared/disabled or |1| for set/enabled.
% \end{description}
%
% \subsection{Design principles}
%
% \begin{description}
% \item[Name conventions:]
%   To avoid conflicts with existing macro names, the operations
%   are prefixed by the package name.
% \item[Zero based indexes:] The first bit is addressed by zero.
%   (Convention of array indexing in C, Java, \dots)
% \item[Unlimited size:]
%   There is no restriction on the size of a bit set other than
%   usual memory limitations. \cs{bitsetSetDec} and \cs{bitsetGetDec}
%   transparently switch to package \xpackage{bigintcalc} if the numbers
%   get too large for \TeX's number limit.
% \item[Expandibility:]
%   Any operation that does not change the bit set is expandable.
%   And all operations that extract or calculate some result
%   do this in exact two expansion steps. For example,
%   a macro \cs{Macro} wants a bit set as decimal number.
%   But the argument must be a plain number without macros.
%   Thus you could prefix \cs{bitsetGetDec} with \cs{number}.
%   However this won't work for bit sets with 31 or more bits
%   because of \TeX's number limit of $2^{31}-1$.
%   then just hit the operator with two \cs{expandafter}:
%   \begin{quote}
%   |\expandafter\expandafter\expandafter|\\
%   |\Macro\bitsetGetDec{foo}|
%   \end{quote}
%   |\bitsetGetDec| is hit first by the third \cs{expandafter}
%   and then by the second one.
% \item[Format independence:] This package is written as \LaTeX\ package,
%   but it does not depend on \LaTeX. It will also work for other formats
%   such as \plainTeX.
% \item[Independence from \TeX\ engines:]
%   Vanilla \TeX\ is all you need. Calculations are delegated to
%   packages \xpackage{intcalc} and \xpackage{bigintcalc}.
%   They don't need any special features, but they will switch
%   to a little more efficient implementation if features such
%   as \cs{numexpr} are available.
% \item[Numeric arguments:] Anything that is accepted by
%   \cs{number}. If \eTeX\ is detected, also expressions
%   for \cs{numexpr} are supported. The only exception so far
%   is the number for \cs{bitsetSetDec}. The number might be too
%   large for \cs{number} or \cs{numexpr}.
% \item[Error messages:]
%   In expandable contexts, only a limited set of \TeX\ primitive
%   commands work as expected. So called stomach commands behave
%   like \cs{relax} and don't get expanded or executed.
%   Unhappily also the error commands belong to this category.
%   The expandable operations will throw an unknown control
%   sequence instead to get \TeX's and user's attention.
%   The name of these control sequences starts with
%   \cs{BitSetError:} with the type of error after the colon.
% \end{description}
%
% \subsection{Operator overview}
%
% \begin{minipage}{\linewidth}
% \begin{description}
% \def\It#1#2{^^A
%   \item[^^A
%     {#1} ^^A
%     \textmd{(\hyperref[{sec:#2}]{section \ref*{sec:#2}})}^^A
%   ]^^A
%   \mbox{}\\^^A
% }
% \def\op#1{^^A
%   \cs{bitset}\texttt{\bfseries#1}^^A
% }
% \def\M#1{\meta{#1}}
% \def\Q{\hspace*{\fill}}
% \It{Miscellaneous}{misc}
%   \op{Reset}\Q \M{BitSet}\\
%   \op{Let}\Q \M{BitSet A} \M{BitSet B}
% \It{Import}{import}
%   \op{SetBin}, \op{SetOct}, \op{SetHex}\Q \M{BitSet} \M{Value}\\
%   \op{SetDec}\Q \M{BitSet} \M{Value}
% \It{Export\textsuperscript{a}}{export}
%   \op{GetBin}, \op{GetOct}, \op{GetHex}\Q \M{BitSet} \M{MinSize}\\
%   \op{GetDec}\Q \M{BitSet}
% \It{Logical operators}{log}
%   \op{And}, \op{AndNot}\Q \M{BitSet A} \M{BitSet B}\\
%   \op{Or}, \op{Xor}\Q \M{BitSet A} \M{BitSet B}
% \It{Shifting}{shift}
%   \op{ShiftLeft}, \op{ShiftRight}\Q \M{BitSet} \M{ShiftAmount}\\
% \It{Bit manipulation}{bitman}
%   \op{Clear}, \op{Set}, \op{Flip}\Q \M{BitSet} \M{Index}\\
%   \op{SetValue}\Q \M{BitSet} \M{Index} \M{Value}\\
%   \op{ClearRange}, \op{SetRange}, \op{FlipRange}\\
%     \Q \M{BitSet} \M{IndexFrom} \M{IndexTo}\\
%   \op{SetValueRange}\Q \M{BitSet} \M{IndexFrom} \M{IndexTo}
% \It{Bit retrieval\textsuperscript{a}}{bitret}
%   \op{Get}\Q \M{BitSet} \M{Index}\\
%   \op{NextClearBit}, \op{NextSetBit}\Q \M{BitSet} \M{Index}\\
%   \op{GetSetBitList}\Q \M{BitSet}
% \It{Bit set properties}{prop}
%   \op{Size}, \op{Cardinality}\Q \M{BitSet}
% \It{Queries\textsuperscript{b}}{query}
%   \op{IsDefined}, \op{IsEmpty}\Q \M{BitSet} \M{Then} \M{Else}\\
%   \op{Equals}, \op{Intersects}\Q \M{BitSet A} \M{BitSet B} \M{Then} \M{Else}\\
%   \op{Query}\Q \M{BitSet} \M{Index} \M{Then} \M{Else}
% \end{description}
% \footnotetext[1]{Macros are expandable, full expansion by two steps.}
% \footnotetext[2]{Macros are expandable.}
% \end{minipage}
%
% \subsection{Package loading}
%
% The package can be used as normal \LaTeX\ package:
%\begin{quote}
%|\usepackage{bitset}|
%\end{quote}
% Also \plainTeX\ is supported:
%\begin{quote}
%|\input bitset.sty\relax|
%\end{quote}
%
% \subsection{Operators}
%
% The following macros work on and with bit sets.
% A bit set \meta{BitSet} is represented by a name.
% The should consist of letters and digits. Technically
% it must survive \cs{csname}. It is the same constraint
% that must be satisfied by label or environment names
% in \LaTeX.
%
% However active characters that are shorthands
% of package \xpackage{babel} are not supported.
% Support for shorthands works by an assignment.
% But many operators
% such as \cs{bitsetGetDec} must be usable in expandable
% contexts. There assignments will not be executed in the
% best case or they will cause errors.
%
% The bits in a bit set are addressed by non-negative
% integers starting from zero. Thus negative index
% numbers cause an error message. Because index numbers
% are \TeX\ numbers. The largest index is 2147483647.
% But in practice memory limits and patience limits
% will be very likely reached much before.
%
% \subsubsection{Miscellaneous}\label{sec:misc}
%
% There isn't a separate operation for bit set creation.
% For simplicity an undefined bit set is treated as bit set
% with all bits cleared.
%
% \begin{declcs}{bitsetReset} \M{BitSet}
% \end{declcs}
% Macro \cs{bitsetReset} clears all bits. The result is an
% empty bit set. It may also be used as replacement for
% an operation ``new'', because an undefined bit set is defined
% afterwards.
%
% \begin{declcs}{bitsetLet} \M{BitSet A} \M{BitSet B}
% \end{declcs}
% Macro \cs{bitsetLet} performs a simple assignment similar
% to \TeX's \cs{let}. After the operation \meta{BitSet A} has
% the same value as \meta{BitSet B}. If \meta{BitSet B} is
% undefined, then \meta{BitSet A} will be the empty bit set.
%
% Note: If \meta{BitSet A} exists, it will be overwritten.
%
% \subsubsection{Import}\label{sec:import}
%
% \begin{declcs}{bitsetSetBin} \M{BitSet} \M{BinaryNumber}\\
%   \cs{bitsetSetOct} \M{BitSet} \M{OctalNumber}\\
%   \cs{bitsetSetHex} \M{BitSet} \M{HexadecimalNumber}
% \end{declcs}
% The numbers are interpreted as bit vectors and the flags in
% the bit \meta{BitSet} set are set accordingly.
% These numeric arguments are the only arguments where
% spaces are allowed. Then the numbers are easier to read.
%
% \begin{declcs}{bitsetSetDec} \M{BitSet} \M{DecimalNumber}
% \end{declcs}
% Macro \cs{bitsetSetDec} uses \meta{DecimalNumber} to
% set the bit set \meta{BitSet}.
% The numeric argument must expand
% to a plain number consisting of decimal digits without
% command tokens or spaces. Internally this argument is
% expanded only. It cannot be passed to \cs{number} or
% \cs{numexpr}, because the number may be too large for them.
% However \cs{number} or |\the\numexpr| may be used
% explicitely. This also helps for unexpandable number
% command tokens or registers (\cs{z@}, \cs{@ne}, \cs{count@}, \dots).
% Also \LaTeX'\ \cs{value} needs prefixing:
% \begin{quote}
% |\bitsetSetDec{foo}{\number\value{bar}}|
% \end{quote}
%
% \subsubsection{Export}\label{sec:export}
%
% \begin{declcs}{bitsetGetBin} \M{BitSet} \M{MinSize}\\
%   \cs{bitsetGetOct} \M{BitSet} \M{MinSize}\\
%   \cs{bitsetGetHex} \M{BitSet} \M{MinSize}
% \end{declcs}
% These macros returns the bit set as binary, octal or hexadecimal
% number. If the bit size is smaller than \meta{MinSize} the gap is
% filled with leading zeros.
% Example:
% \begin{quote}
% |\bitsetReset{abc}|\\
% |\bitsetSet{abc}{2}|\\
% |\bitsetGetBin{abc}{8}| $\rightarrow$ |00000100|\\
% |\bitsetSet{abc}{5}\bitsetSet{abc}{7}|\\
% |\bitsetGetHex{abc}{16}| $\rightarrow$ |00A2|
% \end{quote}
% Macro \cs{bitsetGetHex} uses the uppercase letters |A| to |F|.
% The catcode of the letters is one of 11 (letter) or 12 (other).
%
% \begin{declcs}{bitsetGetDec} \M{BitSet}
% \end{declcs}
% Macro \cs{bitsetGetDec} returns the bit set \meta{BitSet} as
% decimal number. The returned number can be larger than
% \TeX's number limit of $2^{31}-1$.
%
% \subsubsection{Logical operators}\label{sec:log}
%
% \begin{declcs}{bitsetAnd} \M{BitSet A} \M{BitSet B}
% \end{declcs}
% $A_{\text{new}} \mathrel{\mathop{:}}= A_{\text{old}}
% \mathbin{\text{and}} B \qquad (\forall\, \text{bits})$
%
% \begin{declcs}{bitsetAndNot} \M{BitSet A} \M{BitSet B}
% \end{declcs}
% $A_{\text{new}} \mathrel{\mathop{:}}= A_{\text{old}}
% \mathbin{\text{and}} (\text{not}\,B) \qquad (\forall\, \text{bits})$
%
% \begin{declcs}{bitsetOr} \M{BitSet A} \M{BitSet B}
% \end{declcs}
% $A_{\text{new}} \mathrel{\mathop{:}}= A_{\text{old}}
% \mathbin{\text{or}} B \qquad (\forall\, \text{bits})$
%
% \begin{declcs}{bitsetXor} \M{BitSet A} \M{BitSet B}
% \end{declcs}
% $A_{\text{new}} \mathrel{\mathop{:}}= A_{\text{old}}
% \mathbin{\text{xor}} B \qquad (\forall\, \text{bits})$
%
% \subsubsection{Shifting}\label{sec:shift}
%
% \begin{declcs}{bitsetShiftLeft} \M{BitSet} \M{ShiftAmount}\\
%   \cs{bitsetShiftRight} \M{BitSet} \M{ShiftAmount}
% \end{declcs}
% A left shift by one is a multiplication by two, thus left shifting
% moves the flags to higher positions. The new created low positions
% are filled by zeros.
%
% A right shift is the opposite, dividing by two, movint the bits
% to lower positions. The number will become smaller, the lowest
% bits are lost.
%
% If the \meta{ShiftAmount} is negative, it reverts the meaning of
% the shift operation. A left shift becomes a right shift.
% A \meta{ShiftAmount} of zero is ignored.
%
% \subsubsection{Bit manipulation}\label{sec:bitman}
%
% \begin{declcs}{bitsetClear} \M{BitSet} \M{Index}\\
%   \cs{bitsetSet} \M{BitSet} \M{Index}\\
%   \cs{bitsetFlip} \M{BitSet} \M{Index}
% \end{declcs}
% This macros manipulate a single bit in \meta{BitSet} addressed
% by \cs{Index}. Macro \cs{bitsetClear} disables the bit,
% \cs{bitsetSet} enables it and \cs{bitsetFlip} reverts the current
% setting of the bit.
%
% \begin{declcs}{bitsetSetValue} \M{BitSet} \M{Index} \M{Bit}
% \end{declcs}
% Macro \cs{bitsetSetValue} puts bit \meta{Bit} at position \meta{Index}
% in bit set \meta{BitSet}. \meta{Bit} must be a valid \TeX\ number
% equals to zero (disabled/cleared) or one (enabled/set).
%
% \subsubsection{Bit retrieval}\label{sec:bitret}
%
% \begin{declcs}{bitsetGet} \M{BitSet} \M{Index}
% \end{declcs}
% Macro \cs{bitsetGet} extracts the status of the bit at
% position \meta{Index} in bit set \meta{BitSet}.
% Digit |1| is returned if the bit is set/enabled. If the
% bit is cleared/disabled and in cases of an undefined bitset or
% an index number out of range the return value is |0|.
%
% \begin{declcs}{bitsetNextClearBit} \M{BitSet} \M{Index}
% \end{declcs}
% Starting at position \meta{Index} (inclusive) the bits
% are inspected. The first position without a set bit is returned.
% Possible results are decimal numbers: \meta{Index}, \meta{Index} + 1,
% \dots, ($\infty$)
%
% \begin{declcs}{bitsetNextSetBit} \M{BitSet} \M{Index}
% \end{declcs}
% Starting at position \meta{Index} (inclusive) the bits are
% inspected and the index position of the first found set bit
% is returned. If there isn't such a bit, then the result is
% |-1|. In summary possible results are decimal numbers:
% |-1|, \meta{Index}, \meta{Index} + 1, \dots, ($\infty$)
%
% \begin{declcs}{bitsetGetSetBitList} \M{BitSet}
% \end{declcs}
% Macro \cs{bitsetGetSetBitList} is an application for
% \cs{bitsetNextSetBit}. The set bits are iterated and
% returned as comma separated list of index positions
% in increasing order. The list is empty in case of
% an empty bit set.
%
% \subsubsection{Bit set properties}\label{sec:prop}
%
% \begin{declcs}{bitsetSize} \M{BitSet}
% \end{declcs}
% Macro \cs{bitsetSize} returns number of bits in use.
% It is the same as the index number of the highest set/enabled bit
% incremented by one.
%
% \begin{declcs}{bitsetCardinality} \M{BitSet}
% \end{declcs}
% Macro \cs{bitsetCardinality} counts the number of set/enabled bits.
%
% \subsubsection{Queries}\label{sec:query}
%
% Also the query procedures are expandable. They ask for a piece
% of information about a bit set and execute code depending
% on the answer.
%
% \begin{declcs}{bitsetIsDefined} \M{BitSet} \M{Then} \M{Else}
% \end{declcs}
% If the bit set with the name \meta{BitSet} exists the code
% given in \meta{Then} is executed, otherwise \meta{Else} is used.
%
% \begin{declcs}{bitsetIsEmpty} \M{BitSet} \M{Then} \M{Else}
% \end{declcs}
% If the bit set \meta{BitSet} exists and at least one bit
% is set/enabled, the code in \meta{Then} is executed,
% \meta{Else} otherwise.
%
% \begin{declcs}{bitsetEquals} \M{BitSet A} \M{BitSet B} \M{Then} \M{Else}
% \end{declcs}
% Both bit sets are equal if and only if either both are undefined
% or both are defined and represents the same bit values at the
% same positions. Thus this definition is reflexive, symmeric,
% and transitive, enough for an equivalent relation.
%
% \begin{declcs}{bitsetIntersects} \M{BitSet A} \M{BitSet B} \M{Then} \M{Else}
% \end{declcs}
% If and only if \meta{BitSet A} and \meta{BitSet B} have
% at least one bit at the same position that is set, then
% code part \meta{Then} is executed.
%
% \begin{declcs}{bitsetQuery} \M{BitSet} \M{Index} \M{Then} \M{Else}
% \end{declcs}
% It's just a wrapper for \cs{bitsetGet}. If the bit at position
% \meta{Index} is enabled, code \meta{Then} is called.
%
% \StopEventually{
% }
%
% \section{Implementation}
%
% The internal format of a bit set is quite simple, a sequence
% of digits |0| and |1|. The least significant bit is left.
% A bit set without any flag set is encoded by |0|. Also
% undefined bit sets are treated that way. After the
% highest bit that is set there are no further zeroes.
% A regular expression of valid bit sets values:
% \begin{quote}
%  |0|\string||[01]*1|
% \end{quote}
%
%    \begin{macrocode}
%<*package>
%    \end{macrocode}
%
% \subsection{Reload check and package identification}
%    Reload check, especially if the package is not used with \LaTeX.
%    \begin{macrocode}
\begingroup\catcode61\catcode48\catcode32=10\relax%
  \catcode13=5 % ^^M
  \endlinechar=13 %
  \catcode35=6 % #
  \catcode39=12 % '
  \catcode44=12 % ,
  \catcode45=12 % -
  \catcode46=12 % .
  \catcode58=12 % :
  \catcode64=11 % @
  \catcode123=1 % {
  \catcode125=2 % }
  \expandafter\let\expandafter\x\csname ver@bitset.sty\endcsname
  \ifx\x\relax % plain-TeX, first loading
  \else
    \def\empty{}%
    \ifx\x\empty % LaTeX, first loading,
      % variable is initialized, but \ProvidesPackage not yet seen
    \else
      \expandafter\ifx\csname PackageInfo\endcsname\relax
        \def\x#1#2{%
          \immediate\write-1{Package #1 Info: #2.}%
        }%
      \else
        \def\x#1#2{\PackageInfo{#1}{#2, stopped}}%
      \fi
      \x{bitset}{The package is already loaded}%
      \aftergroup\endinput
    \fi
  \fi
\endgroup%
%    \end{macrocode}
%    Package identification:
%    \begin{macrocode}
\begingroup\catcode61\catcode48\catcode32=10\relax%
  \catcode13=5 % ^^M
  \endlinechar=13 %
  \catcode35=6 % #
  \catcode39=12 % '
  \catcode40=12 % (
  \catcode41=12 % )
  \catcode44=12 % ,
  \catcode45=12 % -
  \catcode46=12 % .
  \catcode47=12 % /
  \catcode58=12 % :
  \catcode64=11 % @
  \catcode91=12 % [
  \catcode93=12 % ]
  \catcode123=1 % {
  \catcode125=2 % }
  \expandafter\ifx\csname ProvidesPackage\endcsname\relax
    \def\x#1#2#3[#4]{\endgroup
      \immediate\write-1{Package: #3 #4}%
      \xdef#1{#4}%
    }%
  \else
    \def\x#1#2[#3]{\endgroup
      #2[{#3}]%
      \ifx#1\@undefined
        \xdef#1{#3}%
      \fi
      \ifx#1\relax
        \xdef#1{#3}%
      \fi
    }%
  \fi
\expandafter\x\csname ver@bitset.sty\endcsname
\ProvidesPackage{bitset}%
  [2011/01/30 v1.1 Handle bit-vector datatype (HO)]%
%    \end{macrocode}
%
% \subsection{Catcodes}
%
%    \begin{macrocode}
\begingroup\catcode61\catcode48\catcode32=10\relax%
  \catcode13=5 % ^^M
  \endlinechar=13 %
  \catcode123=1 % {
  \catcode125=2 % }
  \catcode64=11 % @
  \def\x{\endgroup
    \expandafter\edef\csname BitSet@AtEnd\endcsname{%
      \endlinechar=\the\endlinechar\relax
      \catcode13=\the\catcode13\relax
      \catcode32=\the\catcode32\relax
      \catcode35=\the\catcode35\relax
      \catcode61=\the\catcode61\relax
      \catcode64=\the\catcode64\relax
      \catcode123=\the\catcode123\relax
      \catcode125=\the\catcode125\relax
    }%
  }%
\x\catcode61\catcode48\catcode32=10\relax%
\catcode13=5 % ^^M
\endlinechar=13 %
\catcode35=6 % #
\catcode64=11 % @
\catcode123=1 % {
\catcode125=2 % }
\def\TMP@EnsureCode#1#2{%
  \edef\BitSet@AtEnd{%
    \BitSet@AtEnd
    \catcode#1=\the\catcode#1\relax
  }%
  \catcode#1=#2\relax
}
\TMP@EnsureCode{33}{12}% !
\TMP@EnsureCode{39}{12}% '
\TMP@EnsureCode{40}{12}% (
\TMP@EnsureCode{41}{12}% )
\TMP@EnsureCode{42}{12}% *
\TMP@EnsureCode{43}{12}% +
\TMP@EnsureCode{44}{12}% ,
\TMP@EnsureCode{45}{12}% -
\TMP@EnsureCode{46}{12}% .
\TMP@EnsureCode{47}{12}% /
\TMP@EnsureCode{58}{11}% : (letter!)
\TMP@EnsureCode{60}{12}% <
\TMP@EnsureCode{62}{12}% >
\TMP@EnsureCode{63}{14}% ? (comment!)
\TMP@EnsureCode{91}{12}% [
\TMP@EnsureCode{93}{12}% ]
\TMP@EnsureCode{96}{12}% `
\edef\BitSet@AtEnd{\BitSet@AtEnd\noexpand\endinput}
\begingroup\expandafter\expandafter\expandafter\endgroup
\expandafter\ifx\csname BitSet@TestMode\endcsname\relax
\else
  \catcode63=9 % ? (ignore)
\fi
? \let\BitSet@@TestMode\BitSet@TestMode
%    \end{macrocode}
%
% \subsection{Package loading}
%
%    \begin{macrocode}
\begingroup\expandafter\expandafter\expandafter\endgroup
\expandafter\ifx\csname RequirePackage\endcsname\relax
  \def\TMP@RequirePackage#1[#2]{%
    \begingroup\expandafter\expandafter\expandafter\endgroup
    \expandafter\ifx\csname ver@#1.sty\endcsname\relax
      \input #1.sty\relax
    \fi
  }%
  \TMP@RequirePackage{infwarerr}[2007/09/09]%
  \TMP@RequirePackage{intcalc}[2007/09/27]%
  \TMP@RequirePackage{bigintcalc}[2007/09/27]%
\else
  \RequirePackage{infwarerr}[2007/09/09]%
  \RequirePackage{intcalc}[2007/09/27]%
  \RequirePackage{bigintcalc}[2007/09/27]%
\fi
%    \end{macrocode}
%
% \subsection{Help macros}
%
% \subsubsection{Number constant}
%
%    \begin{macro}{\BitSet@MaxSize}
%    \begin{macrocode}
\def\BitSet@MaxSize{2147483647}%
%    \end{macrocode}
%    \end{macro}
%
% \subsubsection{General basic macros}
%
%    \begin{macro}{\BitSet@Empty}
%    \begin{macrocode}
\def\BitSet@Empty{}
%    \end{macrocode}
%    \end{macro}
%    \begin{macro}{\BitSet@FirstOfOne}
%    \begin{macrocode}
\def\BitSet@FirstOfOne#1{#1}
%    \end{macrocode}
%    \end{macro}
%    \begin{macro}{\BitSet@Gobble}
%    \begin{macrocode}
\def\BitSet@Gobble#1{}
%    \end{macrocode}
%    \end{macro}
%    \begin{macro}{\BitSet@FirstOfTwo}
%    \begin{macrocode}
\def\BitSet@FirstOfTwo#1#2{#1}
%    \end{macrocode}
%    \end{macro}
%    \begin{macro}{\BitSet@SecondOfTwo}
%    \begin{macrocode}
\def\BitSet@SecondOfTwo#1#2{#2}
%    \end{macrocode}
%    \end{macro}
%
%    \begin{macro}{\BitSet@Space}
%    \begin{macrocode}
\def\BitSet@Space{ }
%    \end{macrocode}
%    \end{macro}
%    \begin{macro}{\BitSet@ZapSpace}
%    \begin{macrocode}
\def\BitSet@ZapSpace#1 #2{%
  #1%
  \ifx\BitSet@Empty#2%
  \else
    \expandafter\BitSet@ZapSpace
  \fi
  #2%
}
%    \end{macrocode}
%    \end{macro}
%
% \subsubsection{Tail recursion}
%
%    \begin{macro}{\BitSet@Fi}
%    \begin{macrocode}
\let\BitSet@Fi\fi
%    \end{macrocode}
%    \end{macro}
%    \begin{macro}{\BitSet@AfterFi}
%    \begin{macrocode}
\def\BitSet@AfterFi#1#2\BitSet@Fi{\fi#1}
%    \end{macrocode}
%    \end{macro}
%    \begin{macro}{\BitSet@AfterFiFi}
%    \begin{macrocode}
\def\BitSet@AfterFiFi#1#2\BitSet@Fi{\fi\fi#1}%
%    \end{macrocode}
%    \end{macro}
%    \begin{macro}{\BitSet@AfterFiFiFi}
%    \begin{macrocode}
\def\BitSet@AfterFiFiFi#1#2\BitSet@Fi{\fi\fi\fi#1}%
%    \end{macrocode}
%    \end{macro}
%
% \subsubsection{Check macros}
%
%    \begin{macro}{\BitSet@IfUndefined}
%    \begin{macrocode}
\def\BitSet@IfUndefined#1{%
  \expandafter\ifx\csname BS@#1\endcsname\relax
    \expandafter\BitSet@FirstOfTwo
  \else
    \expandafter\BitSet@SecondOfTwo
  \fi
}
%    \end{macrocode}
%    \end{macro}
%
%    \begin{macro}{\BitSet@CheckIndex}
%    |#1|: continuation code\\
%    |#2|: BitSet\\
%    |#3|: Index
%    \begin{macrocode}
\def\BitSet@CheckIndex#1#2#3{%
  \BitSet@IfUndefined{#2}{\bitsetReset{#2}}{}%
  \expandafter\expandafter\expandafter\BitSet@@CheckIndex
  \intcalcNum{#3}!%
  {#2}{#1}%
}
%    \end{macrocode}
%    \end{macro}
%    \begin{macro}{\BitSet@@CheckIndex}
%    |#1|: plain Index\\
%    |#2|: BitSet\\
%    |#3|: continuation code
%    \begin{macrocode}
\def\BitSet@@CheckIndex#1!#2#3{%
  \ifnum#1<0 %
    \BitSet@AfterFi{%
      \@PackageError{bitset}{%
        Invalid negative index (#1)%
      }\@ehc
    }%
  \else
    \BitSet@AfterFi{%
      #3{#2}{#1}%
    }%
  \BitSet@Fi
}
%    \end{macrocode}
%    \end{macro}
%
% \subsection{Miscellaneous}\label{src:misc}
%
%    \begin{macro}{\bitsetReset}
%    \begin{macrocode}
\def\bitsetReset#1{%
  \expandafter\def\csname BS@#1\endcsname{0}%
}
%    \end{macrocode}
%    \end{macro}
%
%    \begin{macro}{\bitsetLet}
%    \begin{macrocode}
\def\bitsetLet#1#2{%
  \BitSet@IfUndefined{#2}{%
    \bitsetReset{#1}%
  }{%
    \expandafter\let\csname BS@#1\expandafter\endcsname
                    \csname BS@#2\endcsname
  }%
}
%    \end{macrocode}
%    \end{macro}
%
% \subsection{Import}\label{src:import}
%
% \subsubsection{From binary number}
%
%    \begin{macro}{\bitsetSetBin}
%    \begin{macrocode}
\def\bitsetSetBin#1#2{%
  \edef\BitSet@Temp{#2}%
  \edef\BitSet@Temp{%
    \expandafter\expandafter\expandafter\BitSet@ZapSpace
    \expandafter\BitSet@Temp\BitSet@Space\BitSet@Empty
  }%
  \edef\BitSet@Temp{%
    \expandafter\BitSet@KillZeros\BitSet@Temp\BitSet@Empty
  }%
  \ifx\BitSet@Temp\BitSet@Empty
    \expandafter\let\csname BS@#1\endcsname\BitSet@Zero
  \else
    \expandafter\edef\csname BS@#1\endcsname{%
      \expandafter\BitSet@Reverse\BitSet@Temp!%
    }%
  \fi
}
%    \end{macrocode}
%    \end{macro}
%    \begin{macro}{\BitSet@KillZeros}
%    \begin{macrocode}
\def\BitSet@KillZeros#1{%
  \ifx#10%
    \expandafter\BitSet@KillZeros
  \else
    #1%
  \fi
}
%    \end{macrocode}
%    \end{macro}
%    \begin{macro}{\BitSet@Reverse}
%    \begin{macrocode}
\def\BitSet@Reverse#1#2!{%
  \ifx\\#2\\%
    #1%
  \else
    \BitSet@AfterFi{%
      \BitSet@Reverse#2!#1%
    }%
  \BitSet@Fi
}
%    \end{macrocode}
%    \end{macro}
%
% \subsubsection{From octal/hex number}
%
%    \begin{macro}{\bitsetSetOct}
%    \begin{macrocode}
\def\bitsetSetOct{%
  \BitSet@SetOctHex\BitSet@FromFirstOct
}
%    \end{macrocode}
%    \end{macro}
%    \begin{macro}{\bitsetSetHex}
%    \begin{macrocode}
\def\bitsetSetHex{%
  \BitSet@SetOctHex\BitSet@FromFirstHex
}
%    \end{macrocode}
%    \end{macro}
%    \begin{macro}{\BitSet@SetOctHex}
%    \begin{macrocode}
\def\BitSet@SetOctHex#1#2#3{%
  \edef\BitSet@Temp{#3}%
  \edef\BitSet@Temp{%
    \expandafter\expandafter\expandafter\BitSet@ZapSpace
    \expandafter\BitSet@Temp\BitSet@Space\BitSet@Empty
  }%
  \edef\BitSet@Temp{%
    \expandafter\BitSet@KillZeros\BitSet@Temp\BitSet@Empty
  }%
  \ifx\BitSet@Temp\BitSet@Empty
    \expandafter\let\csname BS@#2\endcsname\BitSet@Zero
  \else
    \edef\BitSet@Temp{%
      \expandafter#1\BitSet@Temp!%
    }%
    \ifx\BitSet@Temp\BitSet@Empty
      \expandafter\let\csname BS@#2\endcsname\BitSet@Zero
    \else
      \expandafter\edef\csname BS@#2\endcsname{%
        \expandafter\BitSet@Reverse\BitSet@Temp!%
      }%
    \fi
  \fi
}
%    \end{macrocode}
%    \end{macro}
%    \begin{macro}{\BitSet@FromFirstOct}
%    \begin{macrocode}
\def\BitSet@FromFirstOct#1{%
  \ifx#1!%
  \else
    \ifcase#1 \BitSet@AfterFiFi\BitSet@FromFirstOct
    \or 1%
    \or 10%
    \or 11%
    \or 100%
    \or 101%
    \or 110%
    \or 111%
    \else \BitSetError:WrongOctalDigit%
    \fi
    \expandafter\BitSet@FromOct
  \BitSet@Fi
}
%    \end{macrocode}
%    \end{macro}
%    \begin{macro}{\BitSet@FromOct}
%    \begin{macrocode}
\def\BitSet@FromOct#1{%
  \ifx#1!%
  \else
    \ifcase#1 000%
    \or 001%
    \or 010%
    \or 011%
    \or 100%
    \or 101%
    \or 110%
    \or 111%
    \else \BitSetError:WrongOctalDigit%
    \fi
    \expandafter\BitSet@FromOct
  \fi
}
%    \end{macrocode}
%    \end{macro}
%
%    \begin{macro}{\BitSet@FromFirstHex}
%    \begin{macrocode}
\def\BitSet@FromFirstHex#1{%
  \ifx#1!%
  \else
    \ifx#10%
      \BitSet@AfterFiFi\BitSet@FromFirstHex
    \fi
    \expandafter\ifx\csname BitSet@Hex#1\endcsname\relax
      \BitSetError:InvalidHexDigit%
    \else
      \expandafter\expandafter\expandafter\BitSet@KillZeros
      \csname BitSet@Hex#1\endcsname
    \fi
    \expandafter\BitSet@FromHex
  \BitSet@Fi
}
%    \end{macrocode}
%    \end{macro}
%    \begin{macro}{\BitSet@FromHex}
%    \begin{macrocode}
\def\BitSet@FromHex#1{%
  \ifx#1!%
  \else
    \expandafter\ifx\csname BitSet@Hex#1\endcsname\relax
      \BitSetError:InvalidHexDigit%
    \else
      \csname BitSet@Hex#1\endcsname
    \fi
    \expandafter\BitSet@FromHex
  \fi
}
%    \end{macrocode}
%    \end{macro}
%    \begin{macro}{\BitSet@Hex[0..F]}
%    \begin{macrocode}
\def\BitSet@Temp#1{%
  \expandafter\def\csname BitSet@Hex#1\endcsname
}
\BitSet@Temp 0{0000}%
\BitSet@Temp 1{0001}%
\BitSet@Temp 2{0010}%
\BitSet@Temp 3{0011}%
\BitSet@Temp 4{0100}%
\BitSet@Temp 5{0101}%
\BitSet@Temp 6{0110}%
\BitSet@Temp 7{0111}%
\BitSet@Temp 8{1000}%
\BitSet@Temp 9{1001}%
\BitSet@Temp A{1010}%
\BitSet@Temp B{1011}%
\BitSet@Temp C{1100}%
\BitSet@Temp D{1101}%
\BitSet@Temp E{1110}%
\BitSet@Temp F{1111}%
\BitSet@Temp a{1010}%
\BitSet@Temp b{1011}%
\BitSet@Temp c{1100}%
\BitSet@Temp d{1101}%
\BitSet@Temp e{1110}%
\BitSet@Temp f{1111}%
%    \end{macrocode}
%    \end{macro}
%
% \subsubsection{From decimal number}
%
%    \begin{macro}{\bitsetSetDec}
%    \begin{macrocode}
\def\bitsetSetDec#1#2{%
  \edef\BitSet@Temp{#2}%
  \edef\BitSet@Temp{%
    \expandafter\expandafter\expandafter\BitSet@ZapSpace
    \expandafter\BitSet@Temp\BitSet@Space\BitSet@Empty
  }%
  \edef\BitSet@Temp{%
    \expandafter\BitSet@KillZeros\BitSet@Temp\BitSet@Empty
  }%
  \ifx\BitSet@Temp\BitSet@Empty
    \expandafter\let\csname BS@#1\endcsname\BitSet@Zero
  \else
    \ifcase\bigintcalcSgn{\BitSet@Temp} %
      \expandafter\let\csname BS@#1\endcsname\BitSet@Zero
    \or
      \ifnum\bigintcalcCmp\BitSet@Temp\BitSet@MaxSize>0 %
        \expandafter\edef\csname BS@#1\endcsname{%
          \expandafter\BitSet@SetDecBig\BitSet@Temp!%
        }%
      \else
        \expandafter\edef\csname BS@#1\endcsname{%
          \expandafter\BitSet@SetDec\BitSet@Temp!%
        }%
      \fi
    \else
      \@PackageError{bitset}{%
        Bit sets cannot be negative%
      }\@ehc
    \fi
  \fi
}
%    \end{macrocode}
%    \end{macro}
%    \begin{macro}{\BitSet@SetDecBig}
%    \begin{macrocode}
\def\BitSet@SetDecBig#1#2#3#4#5#6#7#8#9!{%
  \ifx\\#9\\%
    \BitSet@SetDec#1#2#3#4#5#6#7#8!%
  \else
    \ifcase\BigIntCalcOdd#1#2#4#5#6#7#8#9! %
      0%
    \or
      1%
?   \else\BitSetError:ThisCannotHappen%
    \fi
    \BitSet@AfterFi{%
      \expandafter\expandafter\expandafter\BitSet@SetDecBig
      \BigIntCalcShr#1#2#3#4#5#6#7#8#9!!%
    }%
  \BitSet@Fi
}
%    \end{macrocode}
%    \end{macro}
%    \begin{macro}{\BitSet@SetDec}
%    \begin{macrocode}
\def\BitSet@SetDec#1!{%
  \ifcase#1 %
  \or 1%
  \else
    \ifodd#1 %
      1%
    \else
      0%
    \fi
    \BitSet@AfterFi{%
      \expandafter\expandafter\expandafter\BitSet@SetDec
      \IntCalcShr#1!!%
    }%
  \BitSet@Fi
}
%    \end{macrocode}
%    \end{macro}
%
% \subsection{Export}\label{src:export}
%
% \subsubsection{To binary number}
%
%    \begin{macro}{\bitsetGetBin}
%    \begin{macrocode}
\def\bitsetGetBin#1#2{%
  \romannumeral0%
  \expandafter\expandafter\expandafter\BitSet@@GetBin
  \intcalcNum{#2}!{#1}%
}
%    \end{macrocode}
%    \end{macro}
%    \begin{macro}{\BitSet@@GetBin}
%    \begin{macrocode}
\def\BitSet@@GetBin#1!#2{%
  \BitSet@IfUndefined{#2}{%
    \ifnum#1>1 %
      \BitSet@AfterFi{%
        \expandafter\expandafter\expandafter\BitSet@Fill
        \IntCalcDec#1!!0%
      }%
    \else
      \BitSet@AfterFi{ 0}%
    \BitSet@Fi
  }{%
    \expandafter\expandafter\expandafter\BitSet@NumBinRev
    \expandafter\expandafter\expandafter1%
    \expandafter\expandafter\expandafter!%
    \csname BS@#2\endcsname!!#1!%
  }%
}
%    \end{macrocode}
%    \end{macro}
%    \begin{macro}{\BitSet@Fill}
%    |#1|: number of leading digits |0|\\
%    |#2|: result
%    \begin{macrocode}
\def\BitSet@Fill#1!{%
  \ifnum#1>0 %
    \BitSet@AfterFi{%
      \expandafter\expandafter\expandafter\BitSet@Fill
      \IntCalcDec#1!!0%
    }%
  \else
    \BitSet@AfterFi{ }%
  \BitSet@Fi
}
%    \end{macrocode}
%    \end{macro}
%
%    \begin{macro}{\BitSet@NumBinRev}
%    |#1|: bit counter (including |#2|)\\
%    |#2#3|: reverted number\\
%    |#4|: result\\
%    |#5|: min size
%    \begin{macrocode}
\def\BitSet@NumBinRev#1!#2#3!{%
  \ifx\\#3\\%
    \BitSet@AfterFi{%
      \BitSet@NumBinFill#1!#2%
    }%
  \else
    \BitSet@AfterFi{%
      \expandafter\expandafter\expandafter\BitSet@NumBinRev
      \IntCalcInc#1!!#3!#2%
    }%
  \BitSet@Fi
}
%    \end{macrocode}
%    \end{macro}
%    \begin{macro}{\BitSet@NumBinFill}
%    \begin{macrocode}
\def\BitSet@NumBinFill#1!#2!#3!{%
  \ifnum#3>#1 %
    \BitSet@AfterFi{%
      \expandafter\expandafter\expandafter\BitSet@Fill
      \IntCalcSub#3!#1!!#2%
    }%
  \else
    \BitSet@AfterFi{ #2}%
  \BitSet@Fi
}
%    \end{macrocode}
%    \end{macro}
%
% \subsubsection{To octal/hexadecimal number}
%
%    \begin{macro}{\bitsetGetOct}
%    \begin{macrocode}
\def\bitsetGetOct#1#2{%
  \romannumeral0%
  \bitsetIsEmpty{#1}{%
    \expandafter\expandafter\expandafter\BitSet@@GetOctHex
    \intcalcNum{#2}!3!230%
  }{%
    \expandafter\expandafter\expandafter\BitSet@@GetOct
    \expandafter\expandafter\expandafter1%
    \expandafter\expandafter\expandafter!%
    \expandafter\expandafter\expandafter!%
    \csname BS@#1\endcsname00%
    \BitSet@Empty\BitSet@Empty\BitSet@Empty!{#2}%
  }%
}
%    \end{macrocode}
%    \end{macro}
%    \begin{macro}{\bitsetGetHex}
%    \begin{macrocode}
\def\bitsetGetHex#1#2{%
  \romannumeral0%
  \bitsetIsEmpty{#1}{%
    \expandafter\expandafter\expandafter\BitSet@@GetOctHex
    \intcalcNum{#2}!4!340%
  }{%
    \expandafter\expandafter\expandafter\BitSet@@GetHex
    \expandafter\expandafter\expandafter1%
    \expandafter\expandafter\expandafter!%
    \expandafter\expandafter\expandafter!%
    \csname BS@#1\endcsname000%
    \BitSet@Empty\BitSet@Empty\BitSet@Empty\BitSet@Empty!{#2}%
  }%
}
%    \end{macrocode}
%    \end{macro}
%    \begin{macro}{\BitSet@@GetOct}
%    |#1|: number of digits\\
%    |#2|: result\\
%    |#3#4#5|: bits
%    \begin{macrocode}
\def\BitSet@@GetOct#1!#2!#3#4#5{%
  \ifx#5\BitSet@Empty
    \BitSet@AfterFi{%
      \expandafter\expandafter\expandafter\BitSet@GetOctHex
      \IntCalcDec#1!!#2!23%
    }%
  \else
    \BitSet@AfterFi{%
      \expandafter\expandafter\expandafter\BitSet@@GetOct
      \number\IntCalcInc#1!\expandafter\expandafter\expandafter!%
      \csname BitSet@Oct#5#4#3\endcsname#2!%
    }%
  \BitSet@Fi
}
%    \end{macrocode}
%    \end{macro}
%    \begin{macro}{\BitSet@Oct[000..111]}
%    \begin{macrocode}
\def\BitSet@Temp#1#2#3#4{%
  \expandafter\def\csname BitSet@Oct#1#2#3\endcsname{#4}%
}
\BitSet@Temp0000%
\BitSet@Temp0011%
\BitSet@Temp0102%
\BitSet@Temp0113%
\BitSet@Temp1004%
\BitSet@Temp1015%
\BitSet@Temp1106%
\BitSet@Temp1117%
%    \end{macrocode}
%    \end{macro}
%    \begin{macro}{\BitSet@@GetHex}
%    |#1|: number of digits\\
%    |#2|: result\\
%    |#3#4#5#6|: bits
%    \begin{macrocode}
\def\BitSet@@GetHex#1!#2!#3#4#5#6{%
  \ifx#6\BitSet@Empty
    \BitSet@AfterFi{%
      \expandafter\expandafter\expandafter\BitSet@GetOctHex
      \IntCalcDec#1!!#2!34%
    }%
  \else
    \BitSet@AfterFi{%
      \expandafter\expandafter\expandafter\BitSet@@GetHex
      \number\IntCalcInc#1!\expandafter\expandafter\expandafter!%
      \csname BitSet@Hex#6#5#4#3\endcsname#2!%
    }%
  \BitSet@Fi
}
%    \end{macrocode}
%    \end{macro}
%    \begin{macro}{\BitSet@Hex[0000..1111]}
%    \begin{macrocode}
\def\BitSet@Temp#1#2#3#4#5{%
  \expandafter\def\csname BitSet@Hex#1#2#3#4\endcsname{#5}%
}
\BitSet@Temp00000%
\BitSet@Temp00011%
\BitSet@Temp00102%
\BitSet@Temp00113%
\BitSet@Temp01004%
\BitSet@Temp01015%
\BitSet@Temp01106%
\BitSet@Temp01117%
\BitSet@Temp10008%
\BitSet@Temp10019%
\BitSet@Temp1010A%
\BitSet@Temp1011B%
\BitSet@Temp1100C%
\BitSet@Temp1101D%
\BitSet@Temp1110E%
\BitSet@Temp1111F%
%    \end{macrocode}
%    \end{macro}
%    \begin{macro}{\BitSet@GetOctHex}
%    Leading zeros $ (|#4| - |#1| * 3 + 2)/3 $ if $ |#4| > |#1| * 3$\\
%    |#1|: digit size\\
%    |#2|: result\\
%    |#3|: bits per digit - 1\\
%    |#4|: bits per digit
%    |#5|: garbage\\
%    |#6|: min size
%    \begin{macrocode}
\def\BitSet@GetOctHex#1!#2!#3#4#5!#6{%
  \expandafter\BitSet@@GetOctHex
  \number\intcalcNum{#6}\expandafter\expandafter\expandafter!%
  \IntCalcMul#1!#4!!#3#4#2%
}
%    \end{macrocode}
%    \end{macro}
%    \begin{macro}{\BitSet@@GetOctHex}
%    |#1|: plain min size\\
%    |#2|: digits * (bits per digit)\\
%    |#3|: bits per digit - 1\\
%    |#4|: bits per digit
%    \begin{macrocode}
\def\BitSet@@GetOctHex#1!#2!#3#4{%
  \ifnum#1>#2 %
    \BitSet@AfterFi{%
      \expandafter\expandafter\expandafter\expandafter
      \expandafter\expandafter\expandafter\BitSet@Fill
      \expandafter\IntCalcDiv\number
      \expandafter\expandafter\expandafter\IntCalcAdd
      \IntCalcSub#1!#2!!#3!!#4!!%
    }%
  \else
    \BitSet@AfterFi{ }%
  \BitSet@Fi
}
%    \end{macrocode}
%    \end{macro}
%
% \subsubsection{To decimal number}
%
%    \begin{macro}{\bitsetGetDec}
%    \begin{macrocode}
\def\bitsetGetDec#1{%
  \romannumeral0%
  \BitSet@IfUndefined{#1}{ 0}{%
    \expandafter\expandafter\expandafter\BitSet@GetDec
    \csname BS@#1\endcsname!%
  }%
}
%    \end{macrocode}
%    \end{macro}
%    \begin{macro}{\BitSet@GetDec}
%    \begin{macrocode}
\def\BitSet@GetDec#1#2!{%
  \ifx\\#2\\%
    \BitSet@AfterFi{ #1}%
  \else
    \BitSet@AfterFi{%
      \BitSet@@GetDec2!#1!#2!%
    }%
  \BitSet@Fi
}
%    \end{macrocode}
%    \end{macro}
%    \begin{macro}{\BitSet@@GetDec}
%    |#1|: power of two\\
%    |#2|: result\\
%    |#3#4|: number
%    \begin{macrocode}
\def\BitSet@@GetDec#1!#2!#3#4!{%
  \ifx\\#4\\%
    \ifx#31%
      \BitSet@AfterFiFi{%
        \expandafter\expandafter\expandafter\BitSet@Space
        \IntCalcAdd#1!#2!%
      }%
    \else
      \BitSet@AfterFiFi{ #2}%
    \fi
  \else
    \ifx#31%
      \BitSet@AfterFiFi{%
        \csname BitSet@N#1%
        \expandafter\expandafter\expandafter\endcsname
        \IntCalcAdd#1!#2!!#4!%
      }%
    \else
      \BitSet@AfterFiFi{%
        \csname BitSet@N#1\endcsname#2!#4!%
      }%
    \fi
  \BitSet@Fi
}
%    \end{macrocode}
%    \end{macro}
%    \begin{macro}{\BitSet@N[1,2,4,...]}
%    \begin{macrocode}
\def\BitSet@Temp#1#2{%
  \expandafter\def\csname BitSet@N#1\endcsname{%
    \BitSet@@GetDec#2!%
  }%
}
\BitSet@Temp{1}{2}
\BitSet@Temp{2}{4}
\BitSet@Temp{4}{8}
\BitSet@Temp{8}{16}
\BitSet@Temp{16}{32}
\BitSet@Temp{32}{64}
\BitSet@Temp{64}{128}
\BitSet@Temp{128}{256}
\BitSet@Temp{256}{512}
\BitSet@Temp{512}{1024}
\BitSet@Temp{1024}{2048}
\BitSet@Temp{2048}{4096}
\BitSet@Temp{4096}{8192}
\BitSet@Temp{8192}{16384}
\BitSet@Temp{16384}{32768}
\BitSet@Temp{32768}{65536}
\BitSet@Temp{65536}{131072}
\BitSet@Temp{131072}{262144}
\BitSet@Temp{262144}{524288}
\BitSet@Temp{524288}{1048576}
\BitSet@Temp{1048576}{2097152}
\BitSet@Temp{2097152}{4194304}
\BitSet@Temp{4194304}{8388608}
\BitSet@Temp{8388608}{16777216}
\BitSet@Temp{16777216}{33554432}
\BitSet@Temp{33554432}{67108864}
\BitSet@Temp{67108864}{134217728}
\BitSet@Temp{134217728}{268435456}
\BitSet@Temp{268435456}{536870912}
\BitSet@Temp{536870912}{1073741824}
%    \end{macrocode}
%    \end{macro}
%    \begin{macro}{\BitSet@N1073741824}
%    \begin{macrocode}
\expandafter\def\csname BitSet@N1073741824\endcsname{%
  \BitSet@GetDecBig2147483648!%
}%
%    \end{macrocode}
%    \end{macro}
%
%    \begin{macro}{\BitSet@GetDecBig}
%    |#1|: current power of two\\
%    |#2|: result\\
%    |#3#4|: number
%    \begin{macrocode}
\def\BitSet@GetDecBig#1!#2!#3#4!{%
  \ifx\\#4\\%
    \ifx#31%
      \BitSet@AfterFiFi{%
        \expandafter\expandafter\expandafter\BitSet@Space
        \BigIntCalcAdd#1!#2!%
      }%
    \else
      \BitSet@AfterFiFi{ #2}%
    \fi
  \else
    \ifx#31%
      \BitSet@AfterFiFi{%
        \expandafter\expandafter\expandafter\BitSet@@GetDecBig
        \BigIntCalcAdd#1!#2!!#1!#4!%
      }%
    \else
      \BitSet@AfterFiFi{%
        \expandafter\expandafter\expandafter\BitSet@GetDecBig
        \BigIntCalcShl#1!!#2!#4!%
      }%
    \fi
  \BitSet@Fi
}
%    \end{macrocode}
%    \end{macro}
%    \begin{macro}{\BitSet@@GetDecBig}
%    |#1|: result\\
%    |#2|: power of two\\
%    |#3#4|: number
%    \begin{macrocode}
\def\BitSet@@GetDecBig#1!#2!{%
  \expandafter\expandafter\expandafter\BitSet@GetDecBig
  \BigIntCalcShl#2!!#1!%
}
%    \end{macrocode}
%    \end{macro}
%
% \subsection{Logical operators}\label{src:log}
%
% \subsubsection{\cs{bitsetAnd}}
%
%    \begin{macro}{\bitsetAnd}
%    Decision table for \cs{bitsetAnd}:
%    \begin{quote}
%    \begin{tabular}{l|l|l|l|}
%                         & undef(B)   & empty(B)   & cardinality(B)$>$0\\
%      \hline
%      undef(A)           & A := empty & A := empty & A := empty\\
%      \hline
%      empty(A)           &            &            &\\
%      \hline
%      cardinality(A)$>$0 & A := empty & A := empty & A \&= B\\
%      \hline
%    \end{tabular}
%    \end{quote}
%    \begin{macrocode}
\def\bitsetAnd#1#2{%
  \bitsetIsEmpty{#1}{%
    \bitsetReset{#1}%
  }{%
    \bitsetIsEmpty{#2}{%
      \bitsetReset{#1}%
    }{%
      \expandafter\edef\csname BS@#1\endcsname{%
        \expandafter\expandafter\expandafter\BitSet@And
        \csname BS@#1\expandafter\expandafter\expandafter\endcsname
        \expandafter\expandafter\expandafter!%
        \csname BS@#2\endcsname!!%
      }%
      \expandafter\ifx\csname BS@#1\endcsname\BitSet@Empty
        \bitsetReset{#1}%
      \fi
    }%
  }%
}
%    \end{macrocode}
%    \end{macro}
%    \begin{macro}{\BitSet@And}
%    \begin{macrocode}
\def\BitSet@And#1#2!#3#4!#5!{%
  \ifx\\#2\\%
    \ifnum#1#3=11 #51\fi
  \else
    \ifx\\#4\\%
      \ifnum#1#3=11 #51\fi
    \else
      \ifnum#1#3=11 %
        #51%
        \BitSet@AfterFiFiFi{%
          \BitSet@And#2!#4!!%
        }%
      \else
        \BitSet@AfterFiFiFi{%
          \BitSet@And#2!#4!#50!%
        }%
      \fi
    \fi
  \BitSet@Fi
}
%    \end{macrocode}
%    \end{macro}
%
% \subsubsection{\cs{bitsetAndNot}}
%
%    \begin{macro}{\bitsetAndNot}
%    Decision table for \cs{bitsetAndNot}:
%    \begin{quote}
%    \begin{tabular}{l|l|l|l|}
%                         & undef(B)   & empty(B)   & cardinality(B)$>$0\\
%      \hline
%      undef(A)           & A := empty & A := empty & A := empty\\
%      \hline
%      empty(A)           &            &            &\\
%      \hline
%      cardinality(A)$>$0 &            &            & A \&= !B\\
%      \hline
%    \end{tabular}
%    \end{quote}
%    \begin{macrocode}
\def\bitsetAndNot#1#2{%
  \bitsetIsEmpty{#1}{%
    \bitsetReset{#1}%
  }{%
    \bitsetIsEmpty{#2}{%
    }{%
      \expandafter\edef\csname BS@#1\endcsname{%
        \expandafter\expandafter\expandafter\BitSet@AndNot
        \csname BS@#1\expandafter\expandafter\expandafter\endcsname
        \expandafter\expandafter\expandafter!%
        \csname BS@#2\endcsname!!%
      }%
      \expandafter\ifx\csname BS@#1\endcsname\BitSet@Empty
        \bitsetReset{#1}%
      \fi
    }%
  }%
}
%    \end{macrocode}
%    \end{macro}
%    \begin{macro}{\BitSet@AndNot}
%    \begin{macrocode}
\def\BitSet@AndNot#1#2!#3#4!#5!{%
  \ifx\\#2\\%
    \ifnum#1#3=10 #51\fi
  \else
    \ifx\\#4\\%
      #5%
      \ifnum#1#3=10 1\else 0\fi
      #2%
    \else
      \ifnum#1#3=10 %
        #51%
        \BitSet@AfterFiFiFi{%
          \BitSet@AndNot#2!#4!!%
        }%
      \else
        \BitSet@AfterFiFiFi{%
          \BitSet@AndNot#2!#4!#50!%
        }%
      \fi
    \fi
  \BitSet@Fi
}
%    \end{macrocode}
%    \end{macro}
%
% \subsubsection{\cs{bitsetOr}}
%
%    \begin{macro}{\bitsetOr}
%    Decision table for \cs{bitsetOr}:
%    \begin{quote}
%    \begin{tabular}{l|l|l|l|}
%                         & undef(B)   & empty(B)   & cardinality(B)$>$0\\
%      \hline
%      undef(A)           & A := empty & A := empty & A := B\\
%      \hline
%      empty(A)           &            &            & A := B\\
%      \hline
%      cardinality(A)$>$0 &            &            & A \string|= B\\
%      \hline
%    \end{tabular}
%    \end{quote}
%    \begin{macrocode}
\def\bitsetOr#1#2{%
  \bitsetIsEmpty{#2}{%
    \BitSet@IfUndefined{#1}{\bitsetReset{#1}}{}%
  }{%
    \bitsetIsEmpty{#1}{%
      \expandafter\let\csname BS@#1\expandafter\endcsname
                      \csname BS@#2\endcsname
    }{%
      \expandafter\edef\csname BS@#1\endcsname{%
        \expandafter\expandafter\expandafter\BitSet@Or
        \csname BS@#1\expandafter\expandafter\expandafter\endcsname
        \expandafter\expandafter\expandafter!%
        \csname BS@#2\endcsname!%
      }%
    }%
  }%
}
%    \end{macrocode}
%    \end{macro}
%    \begin{macro}{\BitSet@Or}
%    \begin{macrocode}
\def\BitSet@Or#1#2!#3#4!{%
  \ifnum#1#3>0 1\else 0\fi
  \ifx\\#2\\%
    #4%
  \else
    \ifx\\#4\\%
      #2%
    \else
      \BitSet@AfterFiFi{%
        \BitSet@Or#2!#4!%
      }%
    \fi
  \BitSet@Fi
}
%    \end{macrocode}
%    \end{macro}
%
% \subsubsection{\cs{bitsetXor}}
%
%    \begin{macro}{\bitsetXor}
%    Decision table for \cs{bitsetXor}:
%    \begin{quote}
%    \begin{tabular}{l|l|l|l|}
%                         & undef(B)   & empty(B)   & cardinality(B)$>$0\\
%      \hline
%      undef(A)           & A := empty & A := empty & A := B\\
%      \hline
%      empty(A)           &            &            & A := B\\
%      \hline
%      cardinality(A)$>$0 &            &            & A \string^= B\\
%      \hline
%    \end{tabular}
%    \end{quote}
%    \begin{macrocode}
\def\bitsetXor#1#2{%
  \bitsetIsEmpty{#2}{%
    \BitSet@IfUndefined{#1}{\bitsetReset{#1}}{}%
  }{%
    \bitsetIsEmpty{#1}{%
      \expandafter\let\csname BS@#1\expandafter\endcsname
                      \csname BS@#2\endcsname
    }{%
      \expandafter\edef\csname BS@#1\endcsname{%
        \expandafter\expandafter\expandafter\BitSet@Xor
        \csname BS@#1\expandafter\expandafter\expandafter\endcsname
        \expandafter\expandafter\expandafter!%
        \csname BS@#2\endcsname!!%
      }%
      \expandafter\ifx\csname BS@#1\endcsname\BitSet@Empty
        \bitsetReset{#1}%
      \fi
    }%
  }%
}
%    \end{macrocode}
%    \end{macro}
%    \begin{macro}{\BitSet@Xor}
%    \begin{macrocode}
\def\BitSet@Xor#1#2!#3#4!#5!{%
  \ifx\\#2\\%
    \ifx#1#3%
      \ifx\\#4\\%
      \else
        #50#4%
      \fi
    \else
      #51#4%
    \fi
  \else
    \ifx\\#4\\%
      #5%
      \ifx#1#30\else 1\fi
      #2%
    \else
      \ifx#1#3%
        \BitSet@AfterFiFiFi{%
          \BitSet@Xor#2!#4!#50!%
        }%
      \else
        #51%
        \BitSet@AfterFiFiFi{%
          \BitSet@Xor#2!#4!!%
        }%
      \fi
    \fi
  \BitSet@Fi
}
%    \end{macrocode}
%    \end{macro}
%
% \subsubsection{Shifting}\label{src:shift}
%
% \subsubsection{\cs{bitsetShiftLeft}}
%
%    \begin{macro}{\bitsetShiftLeft}
%    \begin{macrocode}
\def\bitsetShiftLeft#1#2{%
  \BitSet@IfUndefined{#1}{%
    \bitsetReset{#1}%
  }{%
    \bitsetIsEmpty{#1}{%
    }{%
      \expandafter\expandafter\expandafter\BitSet@ShiftLeft
      \intcalcNum{#2}!{#1}%
    }%
  }%
}
%    \end{macrocode}
%    \end{macro}
%    \begin{macro}{\BitSet@ShiftLeft}
%    \begin{macrocode}
\def\BitSet@ShiftLeft#1!#2{%
  \ifcase\intcalcSgn{#1} %
  \or
    \begingroup
      \uccode`m=`0 %
    \uppercase\expandafter{\expandafter\endgroup
      \expandafter\edef\csname BS@#2\expandafter\endcsname
      \expandafter{%
        \romannumeral#1000\expandafter\BitSet@Space
        \csname BS@#2\endcsname
      }%
    }%
  \else
    \expandafter\BitSet@ShiftRight\BitSet@Gobble#1!{#2}%
  \fi
}
%    \end{macrocode}
%    \end{macro}
%
% \subsubsection{\cs{bitsetShiftRight}}
%
%    \begin{macro}{\bitsetShiftRight}
%    \begin{macrocode}
\def\bitsetShiftRight#1#2{%
  \BitSet@IfUndefined{#1}{%
    \bitsetReset{#1}%
  }{%
    \bitsetIsEmpty{#1}{%
    }{%
      \expandafter\expandafter\expandafter\BitSet@ShiftRight
      \intcalcNum{#2}!{#1}%
    }%
  }%
}
%    \end{macrocode}
%    \end{macro}
%    \begin{macro}{\BitSet@ShiftRight}
%    \begin{macrocode}
\def\BitSet@ShiftRight#1!#2{%
  \ifcase\intcalcSgn{#1} %
  \or
    \expandafter\edef\csname BS@#2\endcsname{%
      \expandafter\expandafter\expandafter\BitSet@Kill
      \csname BS@#2\expandafter\endcsname\expandafter\BitSet@Empty
      \expandafter=%
      \expandafter{\expandafter}\expandafter{\expandafter}%
      \romannumeral#1000!%
    }%
  \else
    \expandafter\BitSet@ShiftLeft\BitSet@Gobble#1!{#2}%
  \fi
}
%    \end{macrocode}
%    \end{macro}
%    \begin{macro}{\BitSet@Kill}
%    \begin{macrocode}
\def\BitSet@Kill#1#2=#3#4#5{%
  #3#4%
  \ifx#5!%
    \ifx#1\BitSet@Empty
      0%
    \else
      #1#2%
    \fi
  \else
    \ifx#1\BitSet@Empty
      0%
      \BitSet@AfterFiFi\BitSet@Cleanup
    \else
      \BitSet@Kill#2=%
    \fi
  \BitSet@Fi
}
%    \end{macrocode}
%    \end{macro}
%
% \subsection{Bit manipulation}\label{src:bitman}
%
%    \begin{macro}{\bitsetClear}
%    \begin{macrocode}
\def\bitsetClear{%
  \BitSet@CheckIndex\BitSet@Clear
}
%    \end{macrocode}
%    \end{macro}
%    \begin{macro}{\bitsetSet}
%    \begin{macrocode}
\def\bitsetSet{%
  \BitSet@CheckIndex\BitSet@Set
}
%    \end{macrocode}
%    \end{macro}
%    \begin{macro}{\bitsetFlip}
%    \begin{macrocode}
\def\bitsetFlip{%
  \BitSet@CheckIndex\BitSet@Flip
}
%    \end{macrocode}
%    \end{macro}
%    \begin{macro}{\bitsetSetValue}
%    \begin{macrocode}
\def\bitsetSetValue#1#2#3{%
  \expandafter\expandafter\expandafter\BitSet@SetValue
  \intcalcNum{#3}!{#1}{#2}%
}
%    \end{macrocode}
%    \end{macro}
%    \begin{macro}{\BitSet@SetValue}
%    |#1|: plain value\\
%    |#2|: BitSet\\
%    |#3|: Index
%    \begin{macrocode}
\def\BitSet@SetValue#1!{%
  \BitSet@CheckIndex{%
    \ifcase#1 %
      \expandafter\BitSet@Clear
    \or
      \expandafter\BitSet@Set
    \else
      \BitSet@ErrorInvalidBitValue{#1}%
      \expandafter\expandafter\expandafter\BitSet@Gobble
      \expandafter\BitSet@Gobble
    \fi
  }%
}
%    \end{macrocode}
%    \end{macro}
%    \begin{macro}{\BitSet@ErrorInvalidBitValue}
%    |#1|: Wrong bit value
%    \begin{macrocode}
\def\BitSet@ErrorInvalidBitValue#1{%
  \@PackageError{bitset}{%
    Invalid bit value (#1) not in range 0..1%
  }\@ehc
}
%    \end{macrocode}
%    \end{macro}
%
% \subsubsection{Clear operation}
%
%    \begin{macro}{\BitSet@Clear}
%    |#1|: BitSet\\
%    |#2|: plain and checked index
%    \begin{macrocode}
\def\BitSet@Clear#1#2{%
  \edef\BitSet@Temp{%
    \expandafter\expandafter\expandafter\BitSet@@Clear
    \csname BS@#1\expandafter\endcsname
    \expandafter\BitSet@Empty\expandafter=\expandafter!%
    \romannumeral#2000!%
  }%
  \expandafter\let\csname BS@#1\expandafter\endcsname
  \ifx\BitSet@Temp\BitSet@Empty
    \BitSet@Zero
  \else
    \BitSet@Temp
  \fi
}
%    \end{macrocode}
%    \end{macro}
%    \begin{macro}{\BitSet@@Clear}
%    \begin{macrocode}
\def\BitSet@@Clear#1#2=#3!#4{%
  \ifx#4!%
    \ifx#1\BitSet@Empty
    \else
      \ifx\BitSet@Empty#2%
      \else
        #30#2%
      \fi
    \fi
  \else
    \ifx#1\BitSet@Empty
      \BitSet@AfterFiFi\BitSet@Cleanup
    \else
      \ifx#10%
        \BitSet@AfterFiFiFi{%
          \BitSet@@Clear#2=#30!%
        }%
      \else
        #31%
        \BitSet@AfterFiFiFi{%
          \BitSet@@Clear#2=!%
        }%
      \fi
    \fi
  \BitSet@Fi
}
%    \end{macrocode}
%    \end{macro}
%
% \subsubsection{Set operation}
%
%    \begin{macro}{\BitSet@Set}
%    |#1|: BitSet\\
%    |#2|: plain and checked Index
%    \begin{macrocode}
\def\BitSet@Set#1#2{%
  \expandafter\edef\csname BS@#1\endcsname{%
    \expandafter\expandafter\expandafter\BitSet@@Set
    \csname BS@#1\expandafter\endcsname
    \expandafter\BitSet@Empty\expandafter=%
    \expandafter{\expandafter}\expandafter{\expandafter}%
    \romannumeral#2000!%
  }%
}
%    \end{macrocode}
%    \end{macro}
%    \begin{macro}{\BitSet@@Set}
%    \begin{macrocode}
\def\BitSet@@Set#1#2=#3#4#5{%
  #3#4%
  \ifx#5!%
    1#2%
  \else
    \ifx#1\BitSet@Empty
      0%
      \BitSet@AfterFiFi\BitSet@@@Set
    \else
      #1%
      \BitSet@@Set#2=%
    \fi
  \BitSet@Fi
}
%    \end{macrocode}
%    \end{macro}
%    \begin{macro}{\BitSet@@@Set}
%    \begin{macrocode}
\def\BitSet@@@Set#1{%
  \ifx#1!%
    1%
  \else
    0%
    \expandafter\BitSet@@@Set
  \fi
}
%    \end{macrocode}
%    \end{macro}
%
% \subsubsection{Flip operation}
%
%    \begin{macro}{\BitSet@Flip}
%    |#1|: BitSet\\
%    |#2|: plain and checked Index
%    \begin{macrocode}
\def\BitSet@Flip#1#2{%
  \edef\BitSet@Temp{%
    \expandafter\expandafter\expandafter\BitSet@@Flip
    \csname BS@#1\expandafter\endcsname
    \expandafter\BitSet@Empty\expandafter=\expandafter!%
    \romannumeral#2000!%
  }%
  \expandafter\let\csname BS@#1\expandafter\endcsname
  \ifx\BitSet@Temp\BitSet@Empty
    \BitSet@Zero
  \else
    \BitSet@Temp
  \fi
}
%    \end{macrocode}
%    \end{macro}
%    \begin{macro}{\BitSet@@Flip}
%    \begin{macrocode}
\def\BitSet@@Flip#1#2=#3!#4{%
  \ifx#4!%
    \ifx#11%
      \ifx\BitSet@Empty#2%
      \else
        #30#2%
      \fi
    \else
      #31#2%
    \fi
  \else
    \ifx#1\BitSet@Empty
      #30%
      \BitSet@AfterFiFi\BitSet@@@Set
    \else
      \ifx#10%
        \BitSet@AfterFiFiFi{%
          \BitSet@@Flip#2=#30!%
        }%
      \else
        #31%
        \BitSet@AfterFiFiFi{%
          \BitSet@@Flip#2=!%
        }%
      \fi
    \fi
  \BitSet@Fi
}
%    \end{macrocode}
%    \end{macro}
%
% \subsubsection{Range operators}
%
%    \begin{macro}{\bitsetClearRange}
%    \begin{macrocode}
\def\bitsetClearRange{%
  \BitSet@Range\BitSet@Clear
}
%    \end{macrocode}
%    \end{macro}
%    \begin{macro}{\bitsetSetRange}
%    \begin{macrocode}
\def\bitsetSetRange{%
  \BitSet@Range\BitSet@Set
}
%    \end{macrocode}
%    \end{macro}
%    \begin{macro}{\bitsetFlipRange}
%    \begin{macrocode}
\def\bitsetFlipRange{%
  \BitSet@Range\BitSet@Flip
}
%    \end{macrocode}
%    \end{macro}
%    \begin{macro}{\bitsetSetValueRange}
%    \begin{macrocode}
\def\bitsetSetValueRange#1#2#3#4{%
  \expandafter\expandafter\expandafter\BitSet@SetValueRange
  \intcalcNum{#4}!{#1}{#2}{#3}%
}
%    \end{macrocode}
%    \end{macro}
%    \begin{macro}{\BitSet@SetValueRange}
%    \begin{macrocode}
\def\BitSet@SetValueRange#1!#2#3#4{%
  \ifcase#1 %
    \BitSet@Range\BitSet@Clear{#2}{#3}{#4}%
  \or
    \BitSet@Range\BitSet@Set{#2}{#3}{#4}%
  \else
    \BitSet@ErrorInvalidBitValue{#1}%
  \fi
}
%    \end{macrocode}
%    \end{macro}
%
%    \begin{macro}{\BitSet@Range}
%    |#1|: clear/set/flip macro\\
%    |#2|: BitSet\\
%    |#3|: Index from\\
%    |#4|: Index to
%    \begin{macrocode}
\def\BitSet@Range#1#2#3#4{%
  \edef\BitSet@Temp{%
    \noexpand\BitSet@@Range\noexpand#1{#2}%
    \intcalcNum{#3}!\intcalcNum{#4}!%
  }%
  \BitSet@Temp
}
%    \end{macrocode}
%    \end{macro}
%    \begin{macro}{\BitSet@@Range}
%    |#1|: clear/set/flip macro\\
%    |#2|: BitSet\\
%    |#3|: Index from\\
%    |#4|: Index to
%    \begin{macrocode}
\def\BitSet@@Range#1#2#3!#4!{%
  \ifnum#3<0 %
    \BitSet@NegativeIndex#1{#2}#3!#4!0!#4!%
  \else
    \ifnum#4<0 %
      \BitSet@NegativeIndex#1{#2}#3!#4!#3!0!%
    \else
      \ifcase\intcalcCmp{#3}{#4} %
      \or
        \@PackageError{bitset}{%
          Wrong index numbers in range [#3..#4]\MessageBreak% hash-ok
          for clear/set/flip on bit set `#2'.\MessageBreak
          The lower index exceeds the upper index.\MessageBreak
          Canceling the operation as error recovery%
        }\@ehc
      \else
        \BitSet@@@Range#3!#4!#1{#2}%
      \fi
    \fi
  \fi
}
%    \end{macrocode}
%    \end{macro}
%    \begin{macro}{\BitSet@NegativeIndex}
%    \begin{macrocode}
\def\BitSet@NegativeIndex#1#2#3!#4!#5!#6!{%
  \@PackageError{bitset}{%
    Negative index in range [#3..#4]\MessageBreak % hash-ok
    for \string\bitset
    \ifx#1\BitSet@Clear
      Clear%
    \else
      \ifx#1\BitSet@Set
        Set%
      \else
        Flip%
      \fi
    \fi
    Range on bit set `#2'.\MessageBreak
    Using [#5..#6] as error recovery% hash-ok
  }\@ehc
  \BitSet@@Range#1{#2}#5!#6!%
}
%    \end{macrocode}
%    \end{macro}
%    \begin{macro}{\BitSet@@Range}
%    \begin{macrocode}
\def\BitSet@@@Range#1!#2!#3#4{%
  \ifnum#1<#2 %
    #3{#4}{#1}%
    \BitSet@AfterFi{%
      \expandafter\expandafter\expandafter\BitSet@@@Range
      \IntCalcInc#1!!#2!#3{#4}%
    }%
  \BitSet@Fi
}
%    \end{macrocode}
%    \end{macro}
%
% \subsection{Bit retrieval}\label{src:bitret}
%
% \subsubsection{\cs{bitsetGet}}
%
%    \begin{macro}{\bitsetGet}
%    \begin{macrocode}
\def\bitsetGet#1#2{%
  \number
  \expandafter\expandafter\expandafter\BitSet@Get
  \intcalcNum{#2}!{#1}%
}
%    \end{macrocode}
%    \end{macro}
%    \begin{macro}{\BitSet@Get}
%    |#1|: plain index\\
%    |#2|: BitSet
%    \begin{macrocode}
\def\BitSet@Get#1!#2{%
  \ifnum#1<0 %
    \BitSet@AfterFi{%
      0 \BitSetError:NegativeIndex%
    }%
  \else
    \BitSet@IfUndefined{#2}{0}{%
      \expandafter\expandafter\expandafter\BitSet@@Get
      \csname BS@#2\expandafter\endcsname
      \expandafter!\expandafter=%
      \expandafter{\expandafter}\expandafter{\expandafter}%
      \romannumeral\intcalcNum{#1}000!%
    }%
    \expandafter\BitSet@Space
  \BitSet@Fi
}
%    \end{macrocode}
%    \end{macro}
%    \begin{macro}{\BitSet@@Get}
%    \begin{macrocode}
\def\BitSet@@Get#1#2=#3#4#5{%
  #3#4%
  \ifx#5!%
    \ifx#1!%
      0%
    \else
      #1%
    \fi
  \else
    \ifx#1!%
      0%
      \BitSet@AfterFiFi\BitSet@Cleanup
    \else
      \BitSet@@Get#2=%
    \fi
  \BitSet@Fi
}
%    \end{macrocode}
%    \end{macro}
%
% \subsubsection{\cs{bitsetNextClearBit}, \cs{bitsetNextSetBit}}
%
%    \begin{macro}{\bitsetNextClearBit}
%    \begin{macrocode}
\def\bitsetNextClearBit#1#2{%
  \number
  \expandafter\expandafter\expandafter\BitSet@NextClearBit
  \intcalcNum{#2}!{#1} %
}
%    \end{macrocode}
%    \end{macro}
%    \begin{macro}{\BitSet@NextClearBit}
%    |#1|: Index\\
%    |#2|: BitSet
%    \begin{macrocode}
\def\BitSet@NextClearBit#1!#2{%
  \ifnum#1<0 %
    \BitSet@NextClearBit0!{#2}%
    \BitSet@AfterFi{%
      \expandafter\BitSet@Space
      \expandafter\BitSetError:NegativeIndex\romannumeral0%
    }%
  \else
    \bitsetIsEmpty{#2}{#1}{%
      \expandafter\BitSet@Skip
      \number#1\expandafter\expandafter\expandafter!%
      \csname BS@#2\endcsname!!!!!!!!!=%
      {\BitSet@@NextClearBit#1!}%
    }%
  \BitSet@Fi
}
%    \end{macrocode}
%    \end{macro}
%    \begin{macro}{\BitSet@@NextClearBit}
%    |#1|: index for next bit in |#2|\\
%    |#2|: next bit
%    \begin{macrocode}
\def\BitSet@@NextClearBit#1!#2{%
  \ifx#2!%
    #1%
  \else
    \ifx#20%
      #1%
      \BitSet@AfterFiFi\BitSet@Cleanup
    \else
      \BitSet@AfterFiFi{%
        \expandafter\expandafter\expandafter\BitSet@@NextClearBit
        \IntCalcInc#1!!%
      }%
    \fi
  \BitSet@Fi
}
%    \end{macrocode}
%    \end{macro}
%
%    \begin{macro}{\bitsetNextSetBit}
%    \begin{macrocode}
\def\bitsetNextSetBit#1#2{%
  \number
  \expandafter\expandafter\expandafter\BitSet@NextSetBit
  \intcalcNum{#2}!{#1} %
}
%    \end{macrocode}
%    \end{macro}
%    \begin{macro}{\BitSet@NextSetBit}
%    |#1|: Index\\
%    |#2|: BitSet
%    \begin{macrocode}
\def\BitSet@NextSetBit#1!#2{%
  \ifnum#1<0 %
    \BitSet@NextSetBit0!{#2}%
    \BitSet@AfterFi{%
      \expandafter\BitSet@Space
      \expandafter\BitSetError:NegativeIndex\romannumeral0%
    }%
  \else
    \bitsetIsEmpty{#2}{-1}{%
      \expandafter\BitSet@Skip
      \number#1\expandafter\expandafter\expandafter!%
      \csname BS@#2\endcsname!!!!!!!!!=%
      {\BitSet@@NextSetBit#1!}%
    }%
  \BitSet@Fi
}
%    \end{macrocode}
%    \end{macro}
%    \begin{macro}{\BitSet@@NextSetBit}
%    |#1|: index for next bit in |#2|\\
%    |#2|: next bit
%    \begin{macrocode}
\def\BitSet@@NextSetBit#1!#2{%
  \ifx#2!%
    -1%
  \else
    \ifx#21%
      #1%
      \BitSet@AfterFiFi\BitSet@Cleanup
    \else
      \BitSet@AfterFiFi{%
        \expandafter\expandafter\expandafter\BitSet@@NextSetBit
        \IntCalcInc#1!!%
      }%
    \fi
  \BitSet@Fi
}
%    \end{macrocode}
%    \end{macro}
%    \begin{macro}{\BitSet@Cleanup}
%    \begin{macrocode}
\def\BitSet@Cleanup#1!{}
%    \end{macrocode}
%    \end{macro}
%
%    \begin{macro}{\BitSet@Skip}
%    |#1|: number of bits to skip\\
%    |#2|: bits\\
%    |#3|: continuation code
%    \begin{macrocode}
\def\BitSet@Skip#1!#2{%
  \ifx#2!%
    \BitSet@AfterFi{%
      \BitSet@SkipContinue%
    }%
  \else
    \ifcase#1 %
      \BitSet@AfterFiFi{%
        \BitSet@SkipContinue#2%
      }%
    \or
      \BitSet@AfterFiFi\BitSet@SkipContinue
    \or
      \BitSet@AfterFiFi{%
        \expandafter\BitSet@SkipContinue\BitSet@Gobble
      }%
    \else
      \ifnum#1>8 %
        \BitSet@AfterFiFiFi{%
          \expandafter\BitSet@Skip
          \number\IntCalcSub#1!8!\expandafter!%
          \BitSet@GobbleSeven
        }%
      \else
        \BitSet@AfterFiFiFi{%
          \expandafter\expandafter\expandafter\BitSet@Skip
          \IntCalcDec#1!!%
        }%
      \fi
    \fi
  \BitSet@Fi
}
%    \end{macrocode}
%    \end{macro}
%    \begin{macro}{\BitSet@SkipContinue}
%    |#1|: remaining bits\\
%    |#2|: continuation code
%    \begin{macrocode}
\def\BitSet@SkipContinue#1!#2=#3{%
  #3#1!%
}
%    \end{macrocode}
%    \end{macro}
%    \begin{macro}{\BitSet@GobbleSeven}
%    \begin{macrocode}
\def\BitSet@GobbleSeven#1#2#3#4#5#6#7{}
%    \end{macrocode}
%    \end{macro}
%
% \subsubsection{\cs{bitsetGetSetBitList}}
%
%    \begin{macro}{\bitsetGetSetBitList}
%    It's just a wrapper for \cs{bitsetNextSetBit}.
%    \begin{macrocode}
\def\bitsetGetSetBitList#1{%
  \romannumeral0%
  \bitsetIsEmpty{#1}{ }{%
    \expandafter\BitSet@GetSetBitList
    \number\BitSet@NextSetBit0!{#1}!{#1}{}!%
  }%
}
%    \end{macrocode}
%    \end{macro}
%    \begin{macro}{\BitSet@GetSetBitList}
%    |#1|: found index\\
%    |#2|: BitSet\\
%    |#3|: comma
%    |#4|: result
%    \begin{macrocode}
\def\BitSet@GetSetBitList#1!#2#3#4!{%
  \ifnum#1<0 %
    \BitSet@AfterFi{ #4}%
  \else
    \BitSet@AfterFi{%
      \expandafter\BitSet@GetSetBitList\number
      \expandafter\expandafter\expandafter\BitSet@NextSetBit
      \IntCalcInc#1!!{#2}!{#2},#4#3#1!%
    }%
  \BitSet@Fi
}
%    \end{macrocode}
%    \end{macro}
%
% \subsection{Bit set properties}\label{src:prop}
%
%    \begin{macro}{\bitsetSize}
%    \begin{macrocode}
\def\bitsetSize#1{%
  \number
  \BitSet@IfUndefined{#1}{0 }{%
    \expandafter\expandafter\expandafter\BitSet@Size
    \expandafter\expandafter\expandafter1%
    \expandafter\expandafter\expandafter!%
    \csname BS@#1\endcsname!0!%
  }%
}
%    \end{macrocode}
%    \end{macro}
%    \begin{macro}{\BitSet@Size}
%    |#1|: counter\\
%    |#2#3|: bits\\
%    |#4|: result
%    \begin{macrocode}
\def\BitSet@Size#1!#2#3!#4!{%
  \ifx#21%
    \ifx\\#3\\%
      \BitSet@AfterFiFi{#1 }%
    \else
      \BitSet@AfterFiFi{%
        \expandafter\expandafter\expandafter\BitSet@Size
        \IntCalcInc#1!!#3!#1!%
      }%
    \fi
  \else
    \ifx\\#3\\%
      \BitSet@AfterFiFi{#4 }%
    \else
      \BitSet@AfterFiFi{%
        \expandafter\expandafter\expandafter\BitSet@Size
        \IntCalcInc#1!!#3!#4!%
      }%
    \fi
  \fi
  \BitSet@Fi
}
%    \end{macrocode}
%    \end{macro}
%
%    \begin{macro}{\bitsetCardinality}
%    \begin{macrocode}
\def\bitsetCardinality#1{%
  \number
  \BitSet@IfUndefined{#1}{0 }{%
    \expandafter\expandafter\expandafter\BitSet@Cardinality
    \expandafter\expandafter\expandafter0%
    \expandafter\expandafter\expandafter!%
    \csname BS@#1\endcsname!%
  }%
}
%    \end{macrocode}
%    \end{macro}
%    \begin{macro}{\BitSet@Cardinality}
%    |#1|: result\\
%    |#2#3|: bits
%    \begin{macrocode}
\def\BitSet@Cardinality#1!#2#3!{%
  \ifx#21%
    \ifx\\#3\\%
      \BitSet@AfterFiFi{\IntCalcInc#1! }%
    \else
      \BitSet@AfterFiFi{%
        \expandafter\expandafter\expandafter\BitSet@Cardinality
        \IntCalcInc#1!!#3!%
      }%
    \fi
  \else
    \ifx\\#3\\%
      \BitSet@AfterFiFi{#1 }%
    \else
      \BitSet@AfterFiFi{%
        \BitSet@Cardinality#1!#3!%
      }%
    \fi
  \fi
  \BitSet@Fi
}
%    \end{macrocode}
%    \end{macro}
%
% \subsection{Queries}\label{src:query}
%
%    \begin{macro}{\bitsetIsDefined}
%    \begin{macrocode}
\def\bitsetIsDefined#1{%
  \BitSet@IfUndefined{#1}%
  \BitSet@SecondOfTwo
  \BitSet@FirstOfTwo
}
%    \end{macrocode}
%    \end{macro}
%    \begin{macro}{\bitsetIsEmpty}
%    \begin{macrocode}
\def\bitsetIsEmpty#1{%
  \BitSet@IfUndefined{#1}\BitSet@FirstOfTwo{%
    \expandafter\ifx\csname BS@#1\endcsname\BitSet@Zero
      \expandafter\BitSet@FirstOfTwo
    \else
      \expandafter\BitSet@SecondOfTwo
    \fi
  }%
}
%    \end{macrocode}
%    \end{macro}
%    \begin{macro}{\BitSet@Zero}
%    \begin{macrocode}
\def\BitSet@Zero{0}
%    \end{macrocode}
%    \end{macro}
%
%    \begin{macro}{\bitsetQuery}
%    \begin{macrocode}
\def\bitsetQuery#1#2{%
  \ifnum\bitsetGet{#1}{#2}=1 %
    \expandafter\BitSet@FirstOfTwo
  \else
    \expandafter\BitSet@SecondOfTwo
  \fi
}
%    \end{macrocode}
%    \end{macro}
%
%    \begin{macro}{\bitsetEquals}
%    \begin{macrocode}
\def\bitsetEquals#1#2{%
  \BitSet@IfUndefined{#1}{%
    \BitSet@IfUndefined{#2}\BitSet@FirstOfTwo\BitSet@SecondOfTwo
  }{%
    \BitSet@IfUndefined{#2}\BitSet@SecondOfTwo{%
      \expandafter\ifx\csname BS@#1\expandafter\endcsname
                      \csname BS@#2\endcsname
        \expandafter\BitSet@FirstOfTwo
      \else
        \expandafter\BitSet@SecondOfTwo
      \fi
    }%
  }%
}
%    \end{macrocode}
%    \end{macro}
%
%    \begin{macro}{\bitsetIntersects}
%    \begin{macrocode}
\def\bitsetIntersects#1#2{%
  \bitsetIsEmpty{#1}\BitSet@SecondOfTwo{%
    \bitsetIsEmpty{#2}\BitSet@SecondOfTwo{%
      \expandafter\expandafter\expandafter\BitSet@Intersects
      \csname BS@#1\expandafter\expandafter\expandafter\endcsname
      \expandafter\expandafter\expandafter!%
      \csname BS@#2\endcsname!%
    }%
  }%
}
%    \end{macrocode}
%    \end{macro}
%    \begin{macro}{\BitSet@Intersects}
%    \begin{macrocode}
\def\BitSet@Intersects#1#2!#3#4!{%
  \ifnum#1#3=11 %
    \BitSet@AfterFi\BitSet@FirstOfTwo
  \else
    \ifx\\#2\\%
      \BitSet@AfterFiFi\BitSet@SecondOfTwo
    \else
      \ifx\\#4\\%
        \BitSet@AfterFiFiFi\BitSet@SecondOfTwo
      \else
        \BitSet@AfterFiFiFi{%
          \BitSet@Intersects#2!#4!%
        }%
      \fi
    \fi
  \BitSet@Fi
}
%    \end{macrocode}
%    \end{macro}
%
%    \begin{macrocode}
\BitSet@AtEnd%
%</package>
%    \end{macrocode}
%
% \section{Test}
%
% \subsection{Catcode checks for loading}
%
%    \begin{macrocode}
%<*test1>
%    \end{macrocode}
%    \begin{macrocode}
\catcode`\{=1 %
\catcode`\}=2 %
\catcode`\#=6 %
\catcode`\@=11 %
\expandafter\ifx\csname count@\endcsname\relax
  \countdef\count@=255 %
\fi
\expandafter\ifx\csname @gobble\endcsname\relax
  \long\def\@gobble#1{}%
\fi
\expandafter\ifx\csname @firstofone\endcsname\relax
  \long\def\@firstofone#1{#1}%
\fi
\expandafter\ifx\csname loop\endcsname\relax
  \expandafter\@firstofone
\else
  \expandafter\@gobble
\fi
{%
  \def\loop#1\repeat{%
    \def\body{#1}%
    \iterate
  }%
  \def\iterate{%
    \body
      \let\next\iterate
    \else
      \let\next\relax
    \fi
    \next
  }%
  \let\repeat=\fi
}%
\def\RestoreCatcodes{}
\count@=0 %
\loop
  \edef\RestoreCatcodes{%
    \RestoreCatcodes
    \catcode\the\count@=\the\catcode\count@\relax
  }%
\ifnum\count@<255 %
  \advance\count@ 1 %
\repeat

\def\RangeCatcodeInvalid#1#2{%
  \count@=#1\relax
  \loop
    \catcode\count@=15 %
  \ifnum\count@<#2\relax
    \advance\count@ 1 %
  \repeat
}
\def\RangeCatcodeCheck#1#2#3{%
  \count@=#1\relax
  \loop
    \ifnum#3=\catcode\count@
    \else
      \errmessage{%
        Character \the\count@\space
        with wrong catcode \the\catcode\count@\space
        instead of \number#3%
      }%
    \fi
  \ifnum\count@<#2\relax
    \advance\count@ 1 %
  \repeat
}
\def\space{ }
\expandafter\ifx\csname LoadCommand\endcsname\relax
  \def\LoadCommand{\input bitset.sty\relax}%
\fi
\def\Test{%
  \RangeCatcodeInvalid{0}{47}%
  \RangeCatcodeInvalid{58}{64}%
  \RangeCatcodeInvalid{91}{96}%
  \RangeCatcodeInvalid{123}{255}%
  \catcode`\@=12 %
  \catcode`\\=0 %
  \catcode`\%=14 %
  \LoadCommand
  \RangeCatcodeCheck{0}{36}{15}%
  \RangeCatcodeCheck{37}{37}{14}%
  \RangeCatcodeCheck{38}{47}{15}%
  \RangeCatcodeCheck{48}{57}{12}%
  \RangeCatcodeCheck{58}{63}{15}%
  \RangeCatcodeCheck{64}{64}{12}%
  \RangeCatcodeCheck{65}{90}{11}%
  \RangeCatcodeCheck{91}{91}{15}%
  \RangeCatcodeCheck{92}{92}{0}%
  \RangeCatcodeCheck{93}{96}{15}%
  \RangeCatcodeCheck{97}{122}{11}%
  \RangeCatcodeCheck{123}{255}{15}%
  \RestoreCatcodes
}
\Test
\csname @@end\endcsname
\end
%    \end{macrocode}
%    \begin{macrocode}
%</test1>
%    \end{macrocode}
%
% \subsection{Macro tests}
%
% \subsubsection{Preamble}
%
%    \begin{macrocode}
%<*test2>
\NeedsTeXFormat{LaTeX2e}
\nofiles
\documentclass{article}
\makeatletter
%<*noetex>
\let\SavedNumexpr\numexpr
\let\SavedIfcsname\ifcsname
\let\SavedCurrentgrouplevel\currentgrouplevel
\def\ETeXDisable{%
  \let\ifcsname\@undefined
  \let\numexpr\@undefined
  \let\currentgrouplevel\@undefined
}
\ETeXDisable
%</noetex>
\makeatletter
\chardef\BitSet@TestMode=1 %
\makeatother
\usepackage{bitset}[2011/01/30]
%<*noetex>
\def\ETeXEnable{%
  \let\numexpr\SavedNumexpr
  \let\ifcsname\SavedIfcsname
  \let\currentgrouplevel\SavedCurrentgrouplevel
}
\ETeXEnable
%</noetex>
\usepackage{qstest}
\IncludeTests{*}
\LogTests{log}{*}{*}
\makeatletter
%    \end{macrocode}
%
% \subsubsection{Time}
%
%    \begin{macrocode}
\begingroup\expandafter\expandafter\expandafter\endgroup
\expandafter\ifx\csname pdfresettimer\endcsname\relax
\else
  \newcount\SummaryTime
  \newcount\TestTime
  \SummaryTime=\z@
  \newcommand*{\PrintTime}[2]{%
    \typeout{%
      [Time #1: \strip@pt\dimexpr\number#2sp\relax\space s]%
    }%
  }%
  \newcommand*{\StartTime}[1]{%
    \renewcommand*{\TimeDescription}{#1}%
    \pdfresettimer
  }%
  \newcommand*{\TimeDescription}{}%
  \newcommand*{\StopTime}{%
    \TestTime=\pdfelapsedtime
    \global\advance\SummaryTime\TestTime
    \PrintTime\TimeDescription\TestTime
  }%
  \let\saved@qstest\qstest
  \let\saved@endqstest\endqstest
  \def\qstest#1#2{%
    \saved@qstest{#1}{#2}%
    \StartTime{#1}%
  }%
  \def\endqstest{%
    \StopTime
    \saved@endqstest
  }%
  \AtEndDocument{%
    \PrintTime{summary}\SummaryTime
  }%
\fi
%    \end{macrocode}
%
% \subsubsection{Detection of unwanted space}
%
%    \begin{macrocode}
\let\orig@qstest\qstest
\let\orig@endqstest\endqstest
\def\qstest#1#2{%
  \orig@qstest{#1}{#2}%
  \setbox0\hbox\bgroup\begingroup\ignorespaces
}
\def\endqstest{%
  \endgroup\egroup
  \Expect*{\the\wd0}{0.0pt}%
  \orig@endqstest
}
%    \end{macrocode}
%
% \subsubsection{Test macros}
%
%    \begin{macrocode}
\newcounter{Test}

\def\TestError#1#2{%
  \begingroup
    \setcounter{Test}{0}%
    \sbox0{%
      \def\@PackageError##1##2##3{%
        \stepcounter{Test}%
        \begingroup
          \let\MessageBreak\relax
%<*noetex>
          \ETeXEnable
%</noetex>
          \Expect{##1}{bitset}%
          \Expect*{##2}*{#1}%
        \endgroup
      }%
%<*noetex>
      \ETeXDisable
%</noetex>
      #2%
    }%
    \Expect*{\theTest}{1}%
    \Expect*{\the\wd0}{0.0pt}%
  \endgroup
}

\def\TestErrorNegativeIndex#1#2{%
  \TestError{Invalid negative index (#1)}{#2}%
}

\def\TestGetterUndefined#1{%
  \CheckUndef{dummy}%
  \expandafter\expandafter\expandafter\Expect
  \expandafter\expandafter\expandafter{#1{dummy}}{0}%
}

\def\ExpectBitSet#1#2{%
  \expandafter\expandafter\expandafter\Expect
  \expandafter\expandafter\expandafter
  {\csname BS@#1\endcsname}*{#2}%
}
\def\Check#1#2{%
  \ExpectBitSet{#1}{#2}%
}
\def\CheckUndef#1{%
  \begingroup
    \Expect*{%
      \expandafter
      \ifx\csname BS@#1\endcsname\relax true\else false\fi
    }{true}%
  \endgroup
}
\def\RevCheck#1#2{%
  \ExpectBitSet{#1}{\Reverse#2!!}%
}
\def\Set#1#2{%
  \expandafter\def\csname BS@#1\endcsname{#2}%
}
\def\RevSet#1#2{%
  \expandafter\edef\csname BS@#1\endcsname{%
    \Reverse#2!!%
  }%
}
\def\Reverse#1#2!#3!{%
  \ifx\\#2\\%
    #1#3%
    \expandafter\@gobble
  \else
    \expandafter\@firstofone
  \fi
  {\Reverse#2!#1#3!}%
}
%    \end{macrocode}
%
% \subsubsection{Test sets}
%
%    \begin{macrocode}
\begin{qstest}{Let}{Let}
  \CheckUndef{abc}%
  \CheckUndef{xyz}%
  \bitsetLet{xyz}{abc}%
  \CheckUndef{abc}%
  \Check{xyz}{0}%
  \Set{abc}{1}%
  \Check{abc}{1}%
  \Check{xyz}{0}%
  \bitsetLet{xyz}{abc}%
  \Check{abc}{1}%
  \Check{xyz}{1}%
  \Set{xyz}{11}%
  \Check{abc}{1}%
  \Check{xyz}{11}%
\end{qstest}

\begin{qstest}{Reset}{Reset}
  \bitsetReset{xyz}%
  \Check{xyz}{0}%
  \bitsetReset{abc}%
  \Check{abc}{0}%
  \Set{abc}{10101}%
  \bitsetReset{abc}%
  \Check{abc}{0}%
\end{qstest}

\begin{qstest}{Get/Query}{Get/Query}
  \expandafter\expandafter\expandafter\Expect
  \expandafter\expandafter\expandafter{%
    \bitsetGet{dummy}{0}%
  }{0}%
  \begingroup
    \expandafter\def\csname BitSetError:NegativeIndex\endcsname{}%
    \Set{abc}{1}%
    \Expect*{\bitsetQuery{abc}{-1}{true}{false}}{false}%
  \endgroup
  \def\Test#1#2#3{%
    \Set{abc}{#1}%
    \expandafter\expandafter\expandafter\Expect
    \expandafter\expandafter\expandafter{\bitsetGet{abc}{#2}}{#3}%
    \Expect*{\bitsetQuery{abc}{#2}{true}{false}}%
           *{\ifcase#3 false\or true\else error\fi}%
  }%
  \Test{1}{100}{0}%
  \Test{0}{0}{0}%
  \Test{1}{0}{1}%
  \Test{11}{1}{1}%
  \Test{111}{1}{1}%
  \Test{101}{1}{0}%
  \Test{101}{2}{1}%
  \Test{10100110011}{10}{1}%
\end{qstest}

\begin{qstest}{Size}{Size}
  \TestGetterUndefined\bitsetSize
  \def\Test#1#2{%
    \Set{abc}{#1}%
    \expandafter\expandafter\expandafter\Expect
    \expandafter\expandafter\expandafter{\bitsetSize{abc}}{#2}%
  }%
  \Test{0}{0}%
  \Test{1}{1}%
  \Test{00}{0}%
  \Test{0000000}{0}%
  \Test{10}{1}%
  \Test{01}{2}%
  \Test{11}{2}%
  \Test{010}{2}%
  \Test{011}{3}%
  \Test{100110011}{9}%
  \Test{0000011111000001111100000}{20}%
  \Test{000000000000000000000000011111111111111111111}{45}%
\end{qstest}

\begin{qstest}{Cardinality}{Cardinality}
  \TestGetterUndefined\bitsetCardinality
  \def\Test#1#2{%
    \Set{abc}{#1}%
    \expandafter\expandafter\expandafter\Expect
    \expandafter\expandafter\expandafter{%
      \bitsetCardinality{abc}%
    }{#2}%
  }%
  \Test{0}{0}%
  \Test{1}{1}%
  \Test{00}{0}%
  \Test{0000000}{0}%
  \Test{10}{1}%
  \Test{01}{1}%
  \Test{11}{2}%
  \Test{010}{1}%
  \Test{011}{2}%
  \Test{100110011}{5}%
  \Test{0000011111000001111100000}{10}%
  \Test{000000000000000000000000011111111111111111111}{20}%
\end{qstest}

\begin{qstest}{NextClearBit/NextSetBit}{NextClearBit/NextSetBit}
  \def\Test#1#2{%
    \expandafter\expandafter\expandafter\Expect
    \expandafter\expandafter\expandafter{%
      \TestOp{abc}{#1}%
    }{#2}%
  }%
  \def\Clear{\let\TestOp\bitsetNextClearBit}%
  \def\Set{\let\TestOp\bitsetNextSetBit}%
  \begingroup
    \catcode`\:=11 %
    \bitsetSetBin{abc}{1}%
    \Clear
    \Test{-1}{1\BitSetError:NegativeIndex}%
    \Set
    \Test{-1}{0\BitSetError:NegativeIndex}%
  \endgroup
  \let\BS@abc\@undefined
  \Clear
  \Test{0}{0}%
  \Test{1}{1}%
  \Test{2}{2}%
  \Test{100}{100}%
  \Set
  \Test{0}{-1}%
  \Test{1}{-1}%
  \Test{100}{-1}%
  \bitsetReset{abc}%
  \Clear
  \Test{0}{0}%
  \Test{1}{1}%
  \Test{2}{2}%
  \Test{100}{100}%
  \Set
  \Test{0}{-1}%
  \Test{1}{-1}%
  \Test{100}{-1}%
  \bitsetSetBin{abc}{1}%
  \Clear
  \Test{0}{1}%
  \Test{1}{1}%
  \Test{2}{2}%
  \Test{100}{100}%
  \Set
  \Test{0}{0}%
  \Test{1}{-1}%
  \Test{100}{-1}%
  \bitsetSetBin{abc}{111000111000111000111}%
  \Clear
  \Test{0}{3}%
  \Test{1}{3}%
  \Test{2}{3}%
  \Test{3}{3}%
  \Test{4}{4}%
  \Test{5}{5}%
  \Test{6}{9}%
  \Test{7}{9}%
  \Test{8}{9}%
  \Test{9}{9}%
  \Test{10}{10}%
  \Test{11}{11}%
  \Test{12}{15}%
  \Test{13}{15}%
  \Test{14}{15}%
  \Test{15}{15}%
  \Test{16}{16}%
  \Test{17}{17}%
  \Test{18}{21}%
  \Test{19}{21}%
  \Test{20}{21}%
  \Test{21}{21}%
  \Test{22}{22}%
  \Test{100}{100}%
  \Set
  \Test{0}{0}%
  \Test{1}{1}%
  \Test{2}{2}%
  \Test{3}{6}%
  \Test{4}{6}%
  \Test{5}{6}%
  \Test{6}{6}%
  \Test{7}{7}%
  \Test{8}{8}%
  \Test{9}{12}%
  \Test{10}{12}%
  \Test{11}{12}%
  \Test{12}{12}%
  \Test{13}{13}%
  \Test{14}{14}%
  \Test{15}{18}%
  \Test{16}{18}%
  \Test{17}{18}%
  \Test{18}{18}%
  \Test{19}{19}%
  \Test{20}{20}%
  \Test{21}{-1}%
  \Test{22}{-1}%
  \Test{100}{-1}%
  \bitsetSetBin{abc}{1111111}%
  \Clear
  \Test{6}{7}%
  \Test{7}{7}%
  \Test{8}{8}%
  \Test{100}{100}%
  \Set
  \Test{6}{6}%
  \Test{7}{-1}%
  \Test{8}{-1}%
  \Test{100}{-1}%
  \bitsetSetBin{abc}{11111111}%
  \Clear
  \Test{7}{8}%
  \Test{8}{8}%
  \Test{9}{9}%
  \Test{100}{100}%
  \Set
  \Test{7}{7}%
  \Test{8}{-1}%
  \Test{9}{-1}%
  \Test{100}{-1}%
  \bitsetSetBin{abc}{111111111}%
  \Clear
  \Test{8}{9}%
  \Test{9}{9}%
  \Test{10}{10}%
  \Test{100}{100}%
  \Set
  \Test{8}{8}%
  \Test{9}{-1}%
  \Test{10}{-1}%
  \Test{100}{-1}%
  \bitsetSetBin{abc}{1111111111}%
  \Clear
  \Test{9}{10}%
  \Test{10}{10}%
  \Test{11}{11}%
  \Test{100}{100}%
  \Set
  \Test{9}{9}%
  \Test{10}{-1}%
  \Test{11}{-1}%
  \Test{100}{-1}%
\end{qstest}

\begin{qstest}{GetSetBitList}{GetSetBitList}
  \let\BS@abc\@undefined
  \expandafter\expandafter\expandafter\Expect
  \expandafter\expandafter\expandafter{%
    \bitsetGetSetBitList{abc}%
  }{}%
  \def\Test#1#2{%
    \bitsetSetBin{abc}{#1}%
    \expandafter\expandafter\expandafter\Expect
    \expandafter\expandafter\expandafter{%
      \bitsetGetSetBitList{abc}%
    }{#2}%
  }%
  \Test{0}{}%
  \Test{1}{0}%
  \Test{10}{1}%
  \Test{11}{0,1}%
  \Test{10110100}{2,4,5,7}%
  \Test{101101001010011}{0,1,4,6,9,11,12,14}%
\end{qstest}

\begin{qstest}{GetDec}{GetDec}
  \TestGetterUndefined\bitsetGetDec
  \def\Test#1#2{%
    \RevSet{abc}{#1}%
%<*noetex>
    \begingroup\expandafter\expandafter\expandafter\endgroup
%</noetex>
    \expandafter\expandafter\expandafter\Expect
    \expandafter\expandafter\expandafter{%
      \bitsetGetDec{abc}%
    }{#2}%
  }%
  \Test{0}{0}%
  \Test{1}{1}%
  \Test{10}{2}%
  \Test{11}{3}%
  \Test{100}{4}%
  \Test{101}{5}%
  \Test{110}{6}%
  \Test{111}{7}%
  \Test{1000}{8}%
  \Test{000111}{7}%
  \Test{1111111111111111%
        111111111111111}{2147483647}%
  \Test{0001111111111111111%
        111111111111111}{2147483647}%
  \Test{1000000000000000%
        0000000000000000}{2147483648}%
  \Test{1000000000000000%
        00000000000000000}{4294967296}%
  \Test{0001000000000000000%
        00000000000000000}{4294967296}%
  \Test{1100000000000000%
        00000000000000011}{6442450947}%
\end{qstest}

\begin{qstest}{Clear}{Clear}
  \def\Test#1#2#3{%
    \RevSet{abc}{#1}%
    \bitsetClear{abc}{#2}%
    \Expect*{\BS@abc}*{\Reverse#3!!}%
  }%
  \bitsetClear{abc}{2}%
  \RevCheck{abc}{0}%
  \TestErrorNegativeIndex{-1}{\bitsetClear{abc}{-1}}%
  \RevCheck{abc}{0}%
  \Test{0}{0}{0}%
  \Test{1}{0}{0}%
  \Test{111}{1}{101}%
  \Test{111}{30}{111}%
  \Test{0000111}{5}{0000111}% 111 would also be ok
  \Test{10000111}{5}{10000111}%
  \Test{1001001}{3}{1000001}%
\end{qstest}

\begin{qstest}{Set}{Set}
  \def\Test#1#2#3{%
    \RevSet{abc}{#1}%
    \bitsetSet{abc}{#2}%
    \Expect*{\BS@abc}*{\Reverse#3!!}%
  }%
  \bitsetSet{abc}{2}%
  \RevCheck{abc}{100}%
  \TestErrorNegativeIndex{-1}{\bitsetSet{abc}{-1}}%
  \RevCheck{abc}{100}%
  \Test{0}{0}{1}%
  \Test{1}{0}{1}%
  \Test{100}{1}{110}%
  \Test{111}{1}{111}%
  \Test{11}{1}{11}%
  \Test{11}{2}{111}%
  \Test{11}{3}{1011}%
  \Test{111}{10}{10000000111}%
  \Test{0000111}{5}{0100111}% 100111 would also be ok
  \Test{10000111}{5}{10100111}%
  \Test{1000001}{3}{1001001}%
  \Test{1001001}{3}{1001001}%
\end{qstest}

\begin{qstest}{Flip}{Flip}
  \def\Test#1#2#3{%
    \RevSet{abc}{#1}%
    \bitsetFlip{abc}{#2}%
    \Expect*{\BS@abc}*{\Reverse#3!!}%
  }%
  \bitsetFlip{abc}{2}%
  \RevCheck{abc}{100}%
  \TestErrorNegativeIndex{-1}{\bitsetFlip{abc}{-1}}%
  \RevCheck{abc}{100}%
  \Test{0}{0}{1}%
  \Test{1}{0}{0}%
  \Test{0}{2}{100}%
  \Test{100}{1}{110}%
  \Test{111}{1}{101}%
  \Test{11}{1}{1}%
  \Test{11}{2}{111}%
  \Test{11}{3}{1011}%
  \Test{111}{10}{10000000111}%
  \Test{0000111}{5}{0100111}% 100111 would also be ok
  \Test{10000111}{5}{10100111}%
  \Test{1000001}{3}{1001001}%
  \Test{1001001}{3}{1000001}%
  \Test{11111}{2}{11011}%
\end{qstest}

\begin{qstest}{SetValue}{SetValue}
  \def\Test#1#2{%
    \TestError{Invalid bit value (#2) not in range 0..1}{%
      \bitsetSetValue{abc}{#1}{#2}%
    }%
  }%
  \Test{0}{-1}%
  \Test{0}{2}%
  \Test{0}{10}%
  \def\Test#1#2#3{%
    \let\BS@abc\@undefined
    \bitsetSetValue{abc}{#1}{#2}%
    \bitsetSetBin{result}{#3}%
    \Expect*{\BS@abc}*{\BS@result}%
  }%
  \Test{0}{0}{0}%
  \Test{0}{1}{1}%
  \Test{1}{0}{0}%
  \Test{1}{1}{10}%
  \def\Test#1#2#3#4{%
    \bitsetSetBin{abc}{#1}%
    \bitsetSetBin{result}{#4}%
    \bitsetSetValue{abc}{#2}{#3}%
    \Expect*{\BS@abc}*{\BS@result}%
  }%
  \Test{0}{0}{0}{0}%
  \Test{0}{0}{0}{0}%
  \Test{0}{0}{1}{1}%
  \Test{0}{1}{0}{0}%
  \Test{0}{1}{1}{10}%
  \Test{1010}{2}{1}{1110}%
  \Test{1010}{4}{1}{11010}%
  \Test{1010}{6}{1}{1001010}%
  \Test{1010}{1}{0}{1000}%
  \Test{1010}{2}{0}{1010}%
  \Test{1010}{3}{0}{10}%
  \Test{1010}{4}{0}{1010}%
  \Test{1010}{6}{0}{1010}%
  \Test{1010}{2}{\csname iffalse\endcsname 0\else 1\fi}{1110}%
  \Test{1010}{1}{\csname iffalse\endcsname 1\else 0\fi}{1000}%
\end{qstest}

\begin{qstest}{IsDefined}{IsDefined}
  \let\BS@abc\@undefined
  \Expect*{\bitsetIsDefined{abc}{true}{false}}{false}%
  \bitsetReset{abc}%
  \Expect*{\bitsetIsDefined{abc}{true}{false}}{true}%
\end{qstest}

\begin{qstest}{IsEmpty}{IsEmpty}
  \let\BS@abc\@undefined
  \Expect*{\bitsetIsEmpty{abc}{true}{false}}{true}%
  \bitsetReset{abc}%
  \Expect*{\bitsetIsEmpty{abc}{true}{false}}{true}%
  \bitsetSet{abc}{1}%
  \Expect*{\bitsetIsEmpty{abc}{true}{false}}{false}%
\end{qstest}

\begin{qstest}{Equals}{Equals}
  \def\Test#1#2#3{%
    \Expect*{\bitsetEquals{#1}{#2}{true}{false}}{#3}%
  }%
  \let\BS@abc\@undefined
  \Test{abc}{abc}{true}%
  \Test{abc}{foo}{true}%
  \Test{foo}{abc}{true}%
  \bitsetReset{abc}%
  \Test{abc}{abc}{true}%
  \Test{abc}{foo}{false}%
  \Test{foo}{abc}{false}%
  \bitsetReset{foo}%
  \Test{abc}{foo}{true}%
  \Test{foo}{abc}{true}%
  \bitsetSet{abc}{4}%
  \Test{abc}{foo}{false}%
  \Test{foo}{abc}{false}%
  \bitsetFlip{foo}{4}%
  \Test{abc}{foo}{true}%
  \Test{foo}{abc}{true}%
\end{qstest}

\begin{qstest}{Intersects}{Intersects}
  \def\Test#1{%
    \Expect*{\bitsetIntersects{abc}{foo}{true}{false}}{#1}%
  }%
  \let\BS@abc\@undefined
  \let\BS@foo\@undefined
  \Test{false}%
  \Set{abc}{0}%
  \Test{false}%
  \Set{foo}{0}%
  \Test{false}%
  \let\BS@abc\@undefined
  \Test{false}%
  \Set{foo}{1}%
  \Test{false}%
  \Set{abc}{0}%
  \Test{false}%
  \Set{abc}{1}%
  \Test{true}%
  \let\BS@foo\@undefined
  \Test{false}%
  \Set{foo}{0}%
  \Test{false}%
  \def\Test#1#2#3{%
    \bitsetSetBin{abc}{#1}%
    \bitsetSetBin{foo}{#2}%
    \Expect*{\bitsetIntersects{abc}{foo}{true}{false}}{#3}%
  }%
  \Test{1010}{0101}{false}%
  \Test{0}{10}{false}%
  \Test{1}{11}{true}%
  \Test{11}{1}{true}%
  \Test{10}{1}{false}%
\end{qstest}

\begin{qstest}{And/AndNot/Or/Xor}{And/AndNot/Or/Xor}
  \def\@Test#1#2#3#4#5{%
    \begingroup
      #5%
      \begingroup
        \let\BS@foo\@undefined
        \csname bitset#1\endcsname{abc}{foo}%
        \CheckUndef{foo}%
        \Check{abc}{#2}%
      \endgroup
      \begingroup
        \bitsetReset{foo}%
        \csname bitset#1\endcsname{abc}{foo}%
        \Check{foo}{0}%
        \Check{abc}{#3}%
      \endgroup
      \begingroup
        \def\BS@foo{0101}%
        \csname bitset#1\endcsname{abc}{foo}%
        \Check{foo}{0101}%
        \Check{abc}{#4}%
      \endgroup
    \endgroup
  }%
  \def\Test#1{%
    \def\Op{#1}%
    \Test@
  }%
  \def\Test@#1#2#3#4#5#6#7#8#9{%
    \@Test\Op{#1}{#2}{#3}{%
      \let\BS@abc\@undefined
    }%
    \@Test\Op{#4}{#5}{#6}{%
      \bitsetReset{abc}%
    }%
    \@Test\Op{#7}{#8}{#9}{%
      \def\BS@abc{1001}%
    }%
  }%
  \Test{And}%
       {0}{0}{0}%
       {0}{0}{0}%
       {0}{0}{0001}%
  \Test{AndNot}%
       {0}{0}{0}%
       {0}{0}{0}%
       {1001}{1001}{1}%
  \Test{Or}%
       {0}{0}{0101}%
       {0}{0}{0101}%
       {1001}{1001}{1101}%
  \Test{Xor}%
       {0}{0}{0101}%
       {0}{0}{0101}%
       {1001}{1001}{11}%
  \def\Test#1#2#3{%
    \bitsetSetBin{abc}{#1}%
    \bitsetSetBin{foo}{#2}%
    \csname bitset\Op\endcsname{abc}{foo}%
    \RevCheck{foo}{#2}%
    \RevCheck{abc}{#3}%
  }%
  \def\Op{And}%
  \Test{1}{111}{1}%
  \Test{111}{1}{1}%
  \Test{10}{111}{10}%
  \Test{111}{10}{10}%
  \Test{111}{1000}{0}%
  \Test{1000}{111}{0}%
  \def\Op{AndNot}%
  \Test{1010}{11}{1000}%
  \Test{100}{100}{0}%
  \Test{111}{1111}{0}%
  \Test{100}{111}{0}%
  \def\Op{Or}%
  \Test{0}{0}{0}%
  \Test{1}{0}{1}%
  \Test{0}{1}{1}%
  \Test{1}{1}{1}%
  \Test{1000}{10}{1010}%
  \Test{10}{1000}{1010}%
  \def\Op{Xor}%
  \Test{0}{0}{0}%
  \Test{1}{0}{1}%
  \Test{0}{1}{1}%
  \Test{1}{1}{0}%
  \Test{1000}{10}{1010}%
  \Test{10}{1000}{1010}%
  \Test   {110011001100}%
       {111000111000111}%
       {111110100001011}%
  \Test{111000111000111}%
          {110011001100}%
       {111110100001011}%
\end{qstest}

\begin{qstest}{GetUndef}{GetUndef, GetBin, GetOct, GetHex}
  \def\TestUndef#1#2{%
    \let\BS@abc\@undefined
    \expandafter\expandafter\expandafter\Expect
    \expandafter\expandafter\expandafter{%
      \x{abc}{#1}%
    }{#2}%
  }%
  \let\x\bitsetGetBin
  \TestUndef{-1}{0}%
  \TestUndef{0}{0}%
  \TestUndef{1}{0}%
  \TestUndef{2}{00}%
  \TestUndef{8}{00000000}%
  \let\x\bitsetGetOct
  \TestUndef{-1}{0}%
  \TestUndef{0}{0}%
  \TestUndef{1}{0}%
  \TestUndef{2}{0}%
  \TestUndef{3}{0}%
  \TestUndef{4}{00}%
  \TestUndef{5}{00}%
  \TestUndef{6}{00}%
  \TestUndef{7}{000}%
  \TestUndef{8}{000}%
  \TestUndef{9}{000}%
  \TestUndef{10}{0000}%
  \let\x\bitsetGetHex
  \TestUndef{-1}{0}%
  \TestUndef{0}{0}%
  \TestUndef{1}{0}%
  \TestUndef{2}{0}%
  \TestUndef{3}{0}%
  \TestUndef{4}{0}%
  \TestUndef{5}{00}%
  \TestUndef{6}{00}%
  \TestUndef{7}{00}%
  \TestUndef{8}{00}%
  \TestUndef{9}{000}%
  \TestUndef{10}{000}%
  \TestUndef{12}{000}%
  \TestUndef{13}{0000}%
  \TestUndef{16}{0000}%
  \TestUndef{17}{00000}%
\end{qstest}

\begin{qstest}{SetBin}{SetBin}
  \def\Test#1#2{%
    \let\BS@abc\@undefined
    \bitsetSetBin{abc}{#1}%
    \expandafter\Expect\expandafter{\BS@abc}{#2}%
  }%
  \Test{}{0}%
  \Test{0}{0}%
  \Test{1}{1}%
  \Test{10}{01}%
  \Test{11}{11}%
  \Test{010}{01}%
  \Test{011}{11}%
  \Test{0010}{01}%
  \Test{1010}{0101}%
\end{qstest}

\begin{qstest}{SetOct}{SetOct}
  \def\Test#1#2{%
    \bitsetSetOct{abc}{#1}%
    \expandafter\Expect\expandafter{\BS@abc}{#2}%
  }%
  \Test{}{0}%
  \Test{0}{0}%
  \Test{000}{0}%
  \Test{1}{1}%
  \Test{001}{1}%
  \Test{010}{0001}%
  \Test{020}{00001}%
  \Test{42}{010001}%
  \Test{377}{11111111}%
  \Test{0377}{11111111}%
  \Test{76543210}{000100010110001101011111}%
  \Test{ 0 7 0 7 1 }{100111000111}%
\end{qstest}

\begin{qstest}{SetHex}{SetHex}
  \def\Test#1#2{%
    \bitsetSetHex{abc}{#1}%
    \expandafter\Expect\expandafter{\BS@abc}{#2}%
  }%
  \Test{}{0}%
  \Test{0}{0}%
  \Test{000}{0}%
  \Test{1}{1}%
  \Test{001}{1}%
  \Test{010}{00001}%
  \Test{020}{000001}%
  \Test{42}{0100001}%
  \Test{3F}{111111}%
  \Test{03F}{111111}%
  \Test{43210}{0000100001001100001}%
  \Test{98765}{10100110111000011001}%
  \Test{FEDCBA}{010111010011101101111111}%
  \Test{ 0 F 0 F 1 }{1000111100001111}%
\end{qstest}

\begin{qstest}{SetDec}{SetDec}
  \def\Test#1#2{%
    \bitsetSetDec{abc}{#1}%
    \expandafter\Expect\expandafter{\BS@abc}{#2}%
  }%
  \Test{}{0}%
  \Test{0}{0}%
  \Test{000}{0}%
  \Test{1}{1}%
  \Test{7}{111}%
  \Test{8}{0001}%
  \Test{001}{1}%
  \Test{010}{0101}%
  \Test{020}{00101}%
  \Test{53}{101011}%
  \Test{255}{11111111}%
  \Test{256}{000000001}%
  \Test{999999999}{111111111001001101011001110111}%
  \Test{1000000000}{000000000101001101011001110111}%
  \Test{4210987654}{01100001010010010111111101011111}%
  \Test{2147483647}{1111111111111111111111111111111}%
  \Test{2147483648}{00000000000000000000000000000001}%
\end{qstest}

\begin{qstest}{GetBin}{GetBin}
  \def\TestUndef#1#2{%
    \let\BS@abc\@undefined
    \expandafter\expandafter\expandafter\Expect
    \expandafter\expandafter\expandafter{%
      \bitsetGetBin{abc}{#1}%
    }{#2}%
  }%
  \TestUndef{-1}{0}%
  \TestUndef{0}{0}%
  \TestUndef{1}{0}%
  \TestUndef{2}{00}%
  \TestUndef{8}{00000000}%
  \def\Test#1#2{%
    \bitsetSetBin{abc}{#2}%
    \expandafter\expandafter\expandafter\Expect
    \expandafter\expandafter\expandafter{%
      \bitsetGetBin{abc}{#1}%
    }{#2}%
  }%
  \Test{-1}{0}%
  \Test{0}{0}%
  \Test{1}{0}%
  \Test{1}{1}%
  \Test{2}{01}%
  \Test{2}{10}%
  \Test{3}{010}%
  \Test{2}{00}%
  \Test{2}{01}%
  \Test{8}{00101100}%
  \Test{2}{10101}%
  \Test{-100}{11011}%
\end{qstest}

\begin{qstest}{GetOct}{GetOct}
  \def\Test#1#2#3{%
    \edef\x{\zap@space#1 \@empty}%
    \edef\x{\noexpand\bitsetSetBin{abc}{\x}}%
    \x
    \expandafter\expandafter\expandafter\Expect
    \expandafter\expandafter\expandafter{%
      \bitsetGetOct{abc}{#2}%
    }{#3}%
  }%
  \Test{111 110 101 100 011 010 001 000}{0}{76543210}%
  \Test{000 111}{0}{7}%
  \Test{101 000}{-1}{50}%
  \Test{111}{-1}{7}%
  \Test{111}{0}{7}%
  \Test{111}{1}{7}%
  \Test{111}{3}{7}%
  \Test{111}{4}{07}%
  \Test{111}{6}{07}%
  \Test{111}{7}{007}%
  \Test{111 010}{6}{72}%
  \Test{111 010}{7}{072}%
  \Test{011 111}{0}{37}%
  \Test{011 111}{6}{37}%
  \Test{011 111}{7}{037}%
  \Test{001 111}{0}{17}%
  \Test{001 111}{6}{17}%
  \Test{001 111}{7}{017}%
\end{qstest}

\begin{qstest}{GetHex}{GetHex}
  \def\Test#1#2#3{%
    \bitsetSetBin{abc}{#1}%
    \expandafter\expandafter\expandafter\Expect
    \expandafter\expandafter\expandafter{%
      \bitsetGetHex{abc}{#2}%
    }{#3}%
  }%
  \Test{1111 1110 1101 1100 1011 1010 1001 1000}{0}{FEDCBA98}%
  \Test{0111 0110 0101 0100 0011 0010 0001 0000}{0}{76543210}%
  \Test{0000 1111}{0}{F}%
  \Test{0101 0000}{-1}{50}%
  \Test{1111}{-1}{F}%
  \Test{1111}{0}{F}%
  \Test{1111}{1}{F}%
  \Test{1111}{4}{F}%
  \Test{1111}{5}{0F}%
  \Test{1111}{8}{0F}%
  \Test{1111}{9}{00F}%
  \Test{1111 0010}{8}{F2}%
  \Test{1111 0010}{9}{0F2}%
  \Test{0111 1111}{0}{7F}%
  \Test{0111 1111}{8}{7F}%
  \Test{0111 1111}{9}{07F}%
  \Test{0011 1111}{0}{3F}%
  \Test{0011 1111}{8}{3F}%
  \Test{0011 1111}{9}{03F}%
  \Test{0001 1111}{0}{1F}%
  \Test{0001 1111}{8}{1F}%
  \Test{0001 1111}{9}{01F}%
\end{qstest}

\begin{qstest}{Range}{Range}
  \TestError{%
    Wrong index numbers in range [9..8]\MessageBreak% hash-ok
    for clear/set/flip on bit set `abc'.\MessageBreak
    The lower index exceeds the upper index.\MessageBreak
    Canceling the operation as error recovery%
  }{%
    \bitsetSetRange{abc}{9}{8}%
  }%
  \def\TestErrorNegInd#1#2#3#4#5#6{%
    \TestError{%
      Negative index in range [#2..#3]\MessageBreak % hash-ok
      for \string\bitset #1Range on bit set `abc'.\MessageBreak
      Using [#4..#5] as error recovery% hash-ok
    }{%
      \csname bitset#1Range\endcsname{abc}{#2}{#3}%
      \global\let\BS@global\BS@abc
    }%
    \Check{global}{#6}%
  }%
  \Set{abc}{111}%
  \TestErrorNegInd{Clear}{-1}{0}{0}{0}{111}%
  \TestErrorNegInd{Clear}{0}{-1}{0}{0}{111}%
  \TestErrorNegInd{Clear}{-2}{2}{0}{2}{001}%
  \bitsetReset{abc}%
  \TestErrorNegInd{Set}{-1}{0}{0}{0}{0}%
  \TestErrorNegInd{Set}{0}{-1}{0}{0}{0}%
  \TestErrorNegInd{Set}{-2}{2}{0}{2}{11}%
  \Set{abc}{101}%
  \TestErrorNegInd{Flip}{-1}{0}{0}{0}{101}%
  \TestErrorNegInd{Flip}{0}{-1}{0}{0}{101}%
  \TestErrorNegInd{Flip}{-2}{2}{0}{2}{011}%
  \def\Test#1#2#3#4{%
    \bitsetSetBin{abc}{#1}%
    \csname bitset\TestOp Range\endcsname{abc}{#2}{#3}%
    \Expect*{\bitsetGetBin{abc}{0}}{#4}%
  }%
  \def\TestOp{Clear}%
  \Test{0}{0}{1}{0}%
  \Test{1111}{1}{2}{1101}%
  \Test{1111}{1}{3}{1001}%
  \Test{1111111100000000}{12}{14}{1100111100000000}%
  \def\TestOp{Set}%
  \Test{0}{0}{1}{1}%
  \Test{1000}{1}{2}{1010}%
  \Test{0}{1}{2}{10}%
  \Test{1}{12}{15}{111000000000001}%
  \Test{1111}{1}{3}{1111}%
  \Test{1000000000000000}{12}{14}{1011000000000000}%
  \def\TestOp{Flip}%
  \Test{0}{0}{1}{1}%
  \Test{1}{0}{1}{0}%
  \Test{10101010}{1}{5}{10110100}%
  \def\Test#1#2#3#4#5{%
    \bitsetSetBin{abc}{#1}%
    \bitsetSetValueRange{abc}{#2}{#3}{#4}%
    \Expect*{\bitsetGetBin{abc}{0}}{#5}%
  }%
  \Test{0}{0}{1}{0}{0}%
  \Test{0}{0}{1}{1}{1}%
  \Test{1010}{1}{3}{0}{1000}%
  \Test{1010}{1}{3}{1}{1110}%
\end{qstest}

\begin{qstest}{ShiftLeft/ShiftRight}{ShiftLeft/ShiftRight}
  \def\@Test#1#2{%
    \let\BS@abc\@undefined
    \csname bitsetShift#1\endcsname{abc}{#2}%
    \Expect*{\BS@abc}{0}%
  }%
  \def\Test#1{%
    \@Test{Left}{#1}%
    \@Test{Right}{#1}%
  }%
  \Test{-16}%
  \Test{-1}%
  \Test{0}%
  \Test{1}%
  \Test{16}%
  \def\Test#1#2#3{%
    \bitsetSetBin{abc}{#1}%
    \bitsetSetBin{result}{#3}%
    \csname bitsetShift\Op\endcsname{abc}{#2}%
    \Expect*{\bitsetGetBin{abc}{0}}*{\bitsetGetBin{result}{0}}%
  }%
  \def\Op{Left}%
  \Test{0}{0}{0}%
  \Test{0}{1}{0}%
  \Test{0}{-1}{0}%
  \Test{1}{0}{1}%
  \Test{1}{1}{10}%
  \Test{1}{-1}{0}%
  \Test{10}{1}{100}%
  \Test{10}{-1}{1}%
  \Test{1}{32}{100000000000000000000000000000000}%
  \Test{1}{-100}{0}%
  \def\Op{Right}%
  \Test{0}{0}{0}%
  \Test{0}{1}{0}%
  \Test{0}{-1}{0}%
  \Test{1}{0}{1}%
  \Test{1}{1}{0}%
  \Test{1}{-1}{10}%
  \Test{10}{1}{1}%
  \Test{10}{-1}{100}%
  \Test{1}{-32}{100000000000000000000000000000000}%
  \Test{1}{100}{0}%
  \Test{110110110110110}{10}{11011}%
  \Test{110110110110110}{100}{0}%
  \Test{1}{100000}{0}%
\end{qstest}

\begin{qstest}{Profile: Set}{Profile: Set}
  \bitsetSet{abc}{4095}%
  \global\let\BS@global\BS@abc
\end{qstest}

\begin{qstest}{Profile: Get}{Profile: Get}
  \edef\x{\bitsetGet{global}{4095}}%
\end{qstest}

\begin{document}
\end{document}
%</test2>
%    \end{macrocode}
%
% \section{Installation}
%
% \subsection{Download}
%
% \paragraph{Package.} This package is available on
% CTAN\footnote{\url{ftp://ftp.ctan.org/tex-archive/}}:
% \begin{description}
% \item[\CTAN{macros/latex/contrib/oberdiek/bitset.dtx}] The source file.
% \item[\CTAN{macros/latex/contrib/oberdiek/bitset.pdf}] Documentation.
% \end{description}
%
%
% \paragraph{Bundle.} All the packages of the bundle `oberdiek'
% are also available in a TDS compliant ZIP archive. There
% the packages are already unpacked and the documentation files
% are generated. The files and directories obey the TDS standard.
% \begin{description}
% \item[\CTAN{install/macros/latex/contrib/oberdiek.tds.zip}]
% \end{description}
% \emph{TDS} refers to the standard ``A Directory Structure
% for \TeX\ Files'' (\CTAN{tds/tds.pdf}). Directories
% with \xfile{texmf} in their name are usually organized this way.
%
% \subsection{Bundle installation}
%
% \paragraph{Unpacking.} Unpack the \xfile{oberdiek.tds.zip} in the
% TDS tree (also known as \xfile{texmf} tree) of your choice.
% Example (linux):
% \begin{quote}
%   |unzip oberdiek.tds.zip -d ~/texmf|
% \end{quote}
%
% \paragraph{Script installation.}
% Check the directory \xfile{TDS:scripts/oberdiek/} for
% scripts that need further installation steps.
% Package \xpackage{attachfile2} comes with the Perl script
% \xfile{pdfatfi.pl} that should be installed in such a way
% that it can be called as \texttt{pdfatfi}.
% Example (linux):
% \begin{quote}
%   |chmod +x scripts/oberdiek/pdfatfi.pl|\\
%   |cp scripts/oberdiek/pdfatfi.pl /usr/local/bin/|
% \end{quote}
%
% \subsection{Package installation}
%
% \paragraph{Unpacking.} The \xfile{.dtx} file is a self-extracting
% \docstrip\ archive. The files are extracted by running the
% \xfile{.dtx} through \plainTeX:
% \begin{quote}
%   \verb|tex bitset.dtx|
% \end{quote}
%
% \paragraph{TDS.} Now the different files must be moved into
% the different directories in your installation TDS tree
% (also known as \xfile{texmf} tree):
% \begin{quote}
% \def\t{^^A
% \begin{tabular}{@{}>{\ttfamily}l@{ $\rightarrow$ }>{\ttfamily}l@{}}
%   bitset.sty & tex/generic/oberdiek/bitset.sty\\
%   bitset.pdf & doc/latex/oberdiek/bitset.pdf\\
%   test/bitset-test1.tex & doc/latex/oberdiek/test/bitset-test1.tex\\
%   test/bitset-test2.tex & doc/latex/oberdiek/test/bitset-test2.tex\\
%   test/bitset-test3.tex & doc/latex/oberdiek/test/bitset-test3.tex\\
%   bitset.dtx & source/latex/oberdiek/bitset.dtx\\
% \end{tabular}^^A
% }^^A
% \sbox0{\t}^^A
% \ifdim\wd0>\linewidth
%   \begingroup
%     \advance\linewidth by\leftmargin
%     \advance\linewidth by\rightmargin
%   \edef\x{\endgroup
%     \def\noexpand\lw{\the\linewidth}^^A
%   }\x
%   \def\lwbox{^^A
%     \leavevmode
%     \hbox to \linewidth{^^A
%       \kern-\leftmargin\relax
%       \hss
%       \usebox0
%       \hss
%       \kern-\rightmargin\relax
%     }^^A
%   }^^A
%   \ifdim\wd0>\lw
%     \sbox0{\small\t}^^A
%     \ifdim\wd0>\linewidth
%       \ifdim\wd0>\lw
%         \sbox0{\footnotesize\t}^^A
%         \ifdim\wd0>\linewidth
%           \ifdim\wd0>\lw
%             \sbox0{\scriptsize\t}^^A
%             \ifdim\wd0>\linewidth
%               \ifdim\wd0>\lw
%                 \sbox0{\tiny\t}^^A
%                 \ifdim\wd0>\linewidth
%                   \lwbox
%                 \else
%                   \usebox0
%                 \fi
%               \else
%                 \lwbox
%               \fi
%             \else
%               \usebox0
%             \fi
%           \else
%             \lwbox
%           \fi
%         \else
%           \usebox0
%         \fi
%       \else
%         \lwbox
%       \fi
%     \else
%       \usebox0
%     \fi
%   \else
%     \lwbox
%   \fi
% \else
%   \usebox0
% \fi
% \end{quote}
% If you have a \xfile{docstrip.cfg} that configures and enables \docstrip's
% TDS installing feature, then some files can already be in the right
% place, see the documentation of \docstrip.
%
% \subsection{Refresh file name databases}
%
% If your \TeX~distribution
% (\teTeX, \mikTeX, \dots) relies on file name databases, you must refresh
% these. For example, \teTeX\ users run \verb|texhash| or
% \verb|mktexlsr|.
%
% \subsection{Some details for the interested}
%
% \paragraph{Attached source.}
%
% The PDF documentation on CTAN also includes the
% \xfile{.dtx} source file. It can be extracted by
% AcrobatReader 6 or higher. Another option is \textsf{pdftk},
% e.g. unpack the file into the current directory:
% \begin{quote}
%   \verb|pdftk bitset.pdf unpack_files output .|
% \end{quote}
%
% \paragraph{Unpacking with \LaTeX.}
% The \xfile{.dtx} chooses its action depending on the format:
% \begin{description}
% \item[\plainTeX:] Run \docstrip\ and extract the files.
% \item[\LaTeX:] Generate the documentation.
% \end{description}
% If you insist on using \LaTeX\ for \docstrip\ (really,
% \docstrip\ does not need \LaTeX), then inform the autodetect routine
% about your intention:
% \begin{quote}
%   \verb|latex \let\install=y% \iffalse meta-comment
%
% File: bitset.dtx
% Version: 2011/01/30 v1.1
% Info: Handle bit-vector datatype
%
% Copyright (C) 2007, 2011 by
%    Heiko Oberdiek <heiko.oberdiek at googlemail.com>
%
% This work may be distributed and/or modified under the
% conditions of the LaTeX Project Public License, either
% version 1.3c of this license or (at your option) any later
% version. This version of this license is in
%    http://www.latex-project.org/lppl/lppl-1-3c.txt
% and the latest version of this license is in
%    http://www.latex-project.org/lppl.txt
% and version 1.3 or later is part of all distributions of
% LaTeX version 2005/12/01 or later.
%
% This work has the LPPL maintenance status "maintained".
%
% This Current Maintainer of this work is Heiko Oberdiek.
%
% The Base Interpreter refers to any `TeX-Format',
% because some files are installed in TDS:tex/generic//.
%
% This work consists of the main source file bitset.dtx
% and the derived files
%    bitset.sty, bitset.pdf, bitset.ins, bitset.drv, bitset-test1.tex,
%    bitset-test2.tex, bitset-test3.tex.
%
% Distribution:
%    CTAN:macros/latex/contrib/oberdiek/bitset.dtx
%    CTAN:macros/latex/contrib/oberdiek/bitset.pdf
%
% Unpacking:
%    (a) If bitset.ins is present:
%           tex bitset.ins
%    (b) Without bitset.ins:
%           tex bitset.dtx
%    (c) If you insist on using LaTeX
%           latex \let\install=y\input{bitset.dtx}
%        (quote the arguments according to the demands of your shell)
%
% Documentation:
%    (a) If bitset.drv is present:
%           latex bitset.drv
%    (b) Without bitset.drv:
%           latex bitset.dtx; ...
%    The class ltxdoc loads the configuration file ltxdoc.cfg
%    if available. Here you can specify further options, e.g.
%    use A4 as paper format:
%       \PassOptionsToClass{a4paper}{article}
%
%    Programm calls to get the documentation (example):
%       pdflatex bitset.dtx
%       makeindex -s gind.ist bitset.idx
%       pdflatex bitset.dtx
%       makeindex -s gind.ist bitset.idx
%       pdflatex bitset.dtx
%
% Installation:
%    TDS:tex/generic/oberdiek/bitset.sty
%    TDS:doc/latex/oberdiek/bitset.pdf
%    TDS:doc/latex/oberdiek/test/bitset-test1.tex
%    TDS:doc/latex/oberdiek/test/bitset-test2.tex
%    TDS:doc/latex/oberdiek/test/bitset-test3.tex
%    TDS:source/latex/oberdiek/bitset.dtx
%
%<*ignore>
\begingroup
  \catcode123=1 %
  \catcode125=2 %
  \def\x{LaTeX2e}%
\expandafter\endgroup
\ifcase 0\ifx\install y1\fi\expandafter
         \ifx\csname processbatchFile\endcsname\relax\else1\fi
         \ifx\fmtname\x\else 1\fi\relax
\else\csname fi\endcsname
%</ignore>
%<*install>
\input docstrip.tex
\Msg{************************************************************************}
\Msg{* Installation}
\Msg{* Package: bitset 2011/01/30 v1.1 Handle bit-vector datatype (HO)}
\Msg{************************************************************************}

\keepsilent
\askforoverwritefalse

\let\MetaPrefix\relax
\preamble

This is a generated file.

Project: bitset
Version: 2011/01/30 v1.1

Copyright (C) 2007, 2011 by
   Heiko Oberdiek <heiko.oberdiek at googlemail.com>

This work may be distributed and/or modified under the
conditions of the LaTeX Project Public License, either
version 1.3c of this license or (at your option) any later
version. This version of this license is in
   http://www.latex-project.org/lppl/lppl-1-3c.txt
and the latest version of this license is in
   http://www.latex-project.org/lppl.txt
and version 1.3 or later is part of all distributions of
LaTeX version 2005/12/01 or later.

This work has the LPPL maintenance status "maintained".

This Current Maintainer of this work is Heiko Oberdiek.

The Base Interpreter refers to any `TeX-Format',
because some files are installed in TDS:tex/generic//.

This work consists of the main source file bitset.dtx
and the derived files
   bitset.sty, bitset.pdf, bitset.ins, bitset.drv, bitset-test1.tex,
   bitset-test2.tex, bitset-test3.tex.

\endpreamble
\let\MetaPrefix\DoubleperCent

\generate{%
  \file{bitset.ins}{\from{bitset.dtx}{install}}%
  \file{bitset.drv}{\from{bitset.dtx}{driver}}%
  \usedir{tex/generic/oberdiek}%
  \file{bitset.sty}{\from{bitset.dtx}{package}}%
  \usedir{doc/latex/oberdiek/test}%
  \file{bitset-test1.tex}{\from{bitset.dtx}{test1}}%
  \file{bitset-test2.tex}{\from{bitset.dtx}{test2,etex}}%
  \file{bitset-test3.tex}{\from{bitset.dtx}{test2,noetex}}%
  \nopreamble
  \nopostamble
  \usedir{source/latex/oberdiek/catalogue}%
  \file{bitset.xml}{\from{bitset.dtx}{catalogue}}%
}

\catcode32=13\relax% active space
\let =\space%
\Msg{************************************************************************}
\Msg{*}
\Msg{* To finish the installation you have to move the following}
\Msg{* file into a directory searched by TeX:}
\Msg{*}
\Msg{*     bitset.sty}
\Msg{*}
\Msg{* To produce the documentation run the file `bitset.drv'}
\Msg{* through LaTeX.}
\Msg{*}
\Msg{* Happy TeXing!}
\Msg{*}
\Msg{************************************************************************}

\endbatchfile
%</install>
%<*ignore>
\fi
%</ignore>
%<*driver>
\NeedsTeXFormat{LaTeX2e}
\ProvidesFile{bitset.drv}%
  [2011/01/30 v1.1 Handle bit-vector datatype (HO)]%
\documentclass{ltxdoc}
\usepackage{holtxdoc}[2011/11/22]
\usepackage{amstext}
\usepackage{ifluatex}
\ifluatex
\else
  \usepackage[T1]{fontenc}%
  \usepackage{lmodern}%
\fi
\begin{document}
  \DocInput{bitset.dtx}%
\end{document}
%</driver>
% \fi
%
% \CheckSum{3510}
%
% \CharacterTable
%  {Upper-case    \A\B\C\D\E\F\G\H\I\J\K\L\M\N\O\P\Q\R\S\T\U\V\W\X\Y\Z
%   Lower-case    \a\b\c\d\e\f\g\h\i\j\k\l\m\n\o\p\q\r\s\t\u\v\w\x\y\z
%   Digits        \0\1\2\3\4\5\6\7\8\9
%   Exclamation   \!     Double quote  \"     Hash (number) \#
%   Dollar        \$     Percent       \%     Ampersand     \&
%   Acute accent  \'     Left paren    \(     Right paren   \)
%   Asterisk      \*     Plus          \+     Comma         \,
%   Minus         \-     Point         \.     Solidus       \/
%   Colon         \:     Semicolon     \;     Less than     \<
%   Equals        \=     Greater than  \>     Question mark \?
%   Commercial at \@     Left bracket  \[     Backslash     \\
%   Right bracket \]     Circumflex    \^     Underscore    \_
%   Grave accent  \`     Left brace    \{     Vertical bar  \|
%   Right brace   \}     Tilde         \~}
%
% \GetFileInfo{bitset.drv}
%
% \title{The \xpackage{bitset} package}
% \date{2011/01/30 v1.1}
% \author{Heiko Oberdiek\\\xemail{heiko.oberdiek at googlemail.com}}
%
% \maketitle
%
% \begin{abstract}
% This package defines and implements the data type bit set,
% a vector of bits. The size of the vector may grow dynamically.
% Individual bits can be manipulated.
% \end{abstract}
%
% \tableofcontents
%
% \clearpage
% \section{Documentation}
%
% \subsection{Introduction}
%
% Annotations in the PDF format know entries whose values are
% integers. This numbers are interpreted as set of flags
% specifying properties. For example, annotation dictionaries
% can have a key |/F|. The bits of its integer value
% are interpreted the following way:
% \begin{quote}
% \begin{tabular}{@{}l|l@{}}
%   Bit position & Property name\\
%   \hline
%   1 & Invisible\\
%   2 & Hidden\\
%   3 & Print\\
%   4 & NoZoom\\
%   5 & NoRotate\\
%   6 & NoView\\
%   7 & ReadOnly\\
%   \dots & \dots
% \end{tabular}
% \end{quote}
% Now, let's see how these values are set in package \xpackage{hyperref}
% before it uses this package (before v6.77a):
%\begin{quote}
%\begin{verbatim}
%\ifFld@hidden /F 6\else /F 4\fi
%\end{verbatim}
%\end{quote}
% Where are the other flags? The following example for key |/Ff|
% in a widget annotation
% supports at least three properties:
%\begin{quote}
%\begin{verbatim}
%\ifFld@multiline
%  \ifFld@readonly /Ff 4097\else /Ff 4096\fi
%\else
%  \ifFld@password
%    \ifFld@readonly /Ff 8193\else /Ff 8192\fi
%  \else
%    \ifFld@readonly /Ff 1\fi
%  \fi
%\fi
%\end{verbatim}
%\end{quote}
% But you see the point. It would be a nightmare to continue
% this way in supporting the missing flag settings. This kind
% of integers may have up to 32 bits.
%
% Therefore I wanted a data structure for setting and
% clearing individual bits. Also it should provide
% an export as decimal number. The snipsets above are
% executed in expansion contexts without \TeX's stomach
% commands. It would be convenient to have an expandable
% conversion from the data structure to the integer that
% gets written to the PDF file.
%
% This package \xpackage{bitset} implements such a data
% structure. The interface is quite close to Java's
% class |BitSet| in order not to learn to many interfaces
% for the same kind of data structure.
%
% \subsection{Glossary}
%
% \begin{description}
% \item[Bit set:]
%   A bit set is a vector of bits or flags.
%   The vector size is unlimited and
%   grows dynamically. An undefined bit set is treated as bit
%   set where all bits are cleared.
%
%   Bit sets are addressed by name. A name should consists of
%   letters or digits. Technically it must survive \cs{csname},
%   see \LaTeX's environment names for other names with such
%   a constraint. Package \xpackage{babel}'s shorthands are
%   not supported due to technical reasons. Shorthand support
%   would break expandable operations.
% \item[Size:]
%   A size of a bit set is the number of bits in use.
%   It's the number of the highest index, incremented by one.
%   Sizes are in the range |0| up to |2147483647|, the highest
%   number supported by \TeX.
% \item[Index:]
%   Bit positions in a bit set are addressed by an index number.
%   The bit vector is zero based. The first and least significant
%   bit is addressed by index |0| and the highest possible bit
%   by |2147483646|.
% \item[Bit:]
%   A bit is enoded as |0| for cleared/disabled or |1| for set/enabled.
% \end{description}
%
% \subsection{Design principles}
%
% \begin{description}
% \item[Name conventions:]
%   To avoid conflicts with existing macro names, the operations
%   are prefixed by the package name.
% \item[Zero based indexes:] The first bit is addressed by zero.
%   (Convention of array indexing in C, Java, \dots)
% \item[Unlimited size:]
%   There is no restriction on the size of a bit set other than
%   usual memory limitations. \cs{bitsetSetDec} and \cs{bitsetGetDec}
%   transparently switch to package \xpackage{bigintcalc} if the numbers
%   get too large for \TeX's number limit.
% \item[Expandibility:]
%   Any operation that does not change the bit set is expandable.
%   And all operations that extract or calculate some result
%   do this in exact two expansion steps. For example,
%   a macro \cs{Macro} wants a bit set as decimal number.
%   But the argument must be a plain number without macros.
%   Thus you could prefix \cs{bitsetGetDec} with \cs{number}.
%   However this won't work for bit sets with 31 or more bits
%   because of \TeX's number limit of $2^{31}-1$.
%   then just hit the operator with two \cs{expandafter}:
%   \begin{quote}
%   |\expandafter\expandafter\expandafter|\\
%   |\Macro\bitsetGetDec{foo}|
%   \end{quote}
%   |\bitsetGetDec| is hit first by the third \cs{expandafter}
%   and then by the second one.
% \item[Format independence:] This package is written as \LaTeX\ package,
%   but it does not depend on \LaTeX. It will also work for other formats
%   such as \plainTeX.
% \item[Independence from \TeX\ engines:]
%   Vanilla \TeX\ is all you need. Calculations are delegated to
%   packages \xpackage{intcalc} and \xpackage{bigintcalc}.
%   They don't need any special features, but they will switch
%   to a little more efficient implementation if features such
%   as \cs{numexpr} are available.
% \item[Numeric arguments:] Anything that is accepted by
%   \cs{number}. If \eTeX\ is detected, also expressions
%   for \cs{numexpr} are supported. The only exception so far
%   is the number for \cs{bitsetSetDec}. The number might be too
%   large for \cs{number} or \cs{numexpr}.
% \item[Error messages:]
%   In expandable contexts, only a limited set of \TeX\ primitive
%   commands work as expected. So called stomach commands behave
%   like \cs{relax} and don't get expanded or executed.
%   Unhappily also the error commands belong to this category.
%   The expandable operations will throw an unknown control
%   sequence instead to get \TeX's and user's attention.
%   The name of these control sequences starts with
%   \cs{BitSetError:} with the type of error after the colon.
% \end{description}
%
% \subsection{Operator overview}
%
% \begin{minipage}{\linewidth}
% \begin{description}
% \def\It#1#2{^^A
%   \item[^^A
%     {#1} ^^A
%     \textmd{(\hyperref[{sec:#2}]{section \ref*{sec:#2}})}^^A
%   ]^^A
%   \mbox{}\\^^A
% }
% \def\op#1{^^A
%   \cs{bitset}\texttt{\bfseries#1}^^A
% }
% \def\M#1{\meta{#1}}
% \def\Q{\hspace*{\fill}}
% \It{Miscellaneous}{misc}
%   \op{Reset}\Q \M{BitSet}\\
%   \op{Let}\Q \M{BitSet A} \M{BitSet B}
% \It{Import}{import}
%   \op{SetBin}, \op{SetOct}, \op{SetHex}\Q \M{BitSet} \M{Value}\\
%   \op{SetDec}\Q \M{BitSet} \M{Value}
% \It{Export\textsuperscript{a}}{export}
%   \op{GetBin}, \op{GetOct}, \op{GetHex}\Q \M{BitSet} \M{MinSize}\\
%   \op{GetDec}\Q \M{BitSet}
% \It{Logical operators}{log}
%   \op{And}, \op{AndNot}\Q \M{BitSet A} \M{BitSet B}\\
%   \op{Or}, \op{Xor}\Q \M{BitSet A} \M{BitSet B}
% \It{Shifting}{shift}
%   \op{ShiftLeft}, \op{ShiftRight}\Q \M{BitSet} \M{ShiftAmount}\\
% \It{Bit manipulation}{bitman}
%   \op{Clear}, \op{Set}, \op{Flip}\Q \M{BitSet} \M{Index}\\
%   \op{SetValue}\Q \M{BitSet} \M{Index} \M{Value}\\
%   \op{ClearRange}, \op{SetRange}, \op{FlipRange}\\
%     \Q \M{BitSet} \M{IndexFrom} \M{IndexTo}\\
%   \op{SetValueRange}\Q \M{BitSet} \M{IndexFrom} \M{IndexTo}
% \It{Bit retrieval\textsuperscript{a}}{bitret}
%   \op{Get}\Q \M{BitSet} \M{Index}\\
%   \op{NextClearBit}, \op{NextSetBit}\Q \M{BitSet} \M{Index}\\
%   \op{GetSetBitList}\Q \M{BitSet}
% \It{Bit set properties}{prop}
%   \op{Size}, \op{Cardinality}\Q \M{BitSet}
% \It{Queries\textsuperscript{b}}{query}
%   \op{IsDefined}, \op{IsEmpty}\Q \M{BitSet} \M{Then} \M{Else}\\
%   \op{Equals}, \op{Intersects}\Q \M{BitSet A} \M{BitSet B} \M{Then} \M{Else}\\
%   \op{Query}\Q \M{BitSet} \M{Index} \M{Then} \M{Else}
% \end{description}
% \footnotetext[1]{Macros are expandable, full expansion by two steps.}
% \footnotetext[2]{Macros are expandable.}
% \end{minipage}
%
% \subsection{Package loading}
%
% The package can be used as normal \LaTeX\ package:
%\begin{quote}
%|\usepackage{bitset}|
%\end{quote}
% Also \plainTeX\ is supported:
%\begin{quote}
%|\input bitset.sty\relax|
%\end{quote}
%
% \subsection{Operators}
%
% The following macros work on and with bit sets.
% A bit set \meta{BitSet} is represented by a name.
% The should consist of letters and digits. Technically
% it must survive \cs{csname}. It is the same constraint
% that must be satisfied by label or environment names
% in \LaTeX.
%
% However active characters that are shorthands
% of package \xpackage{babel} are not supported.
% Support for shorthands works by an assignment.
% But many operators
% such as \cs{bitsetGetDec} must be usable in expandable
% contexts. There assignments will not be executed in the
% best case or they will cause errors.
%
% The bits in a bit set are addressed by non-negative
% integers starting from zero. Thus negative index
% numbers cause an error message. Because index numbers
% are \TeX\ numbers. The largest index is 2147483647.
% But in practice memory limits and patience limits
% will be very likely reached much before.
%
% \subsubsection{Miscellaneous}\label{sec:misc}
%
% There isn't a separate operation for bit set creation.
% For simplicity an undefined bit set is treated as bit set
% with all bits cleared.
%
% \begin{declcs}{bitsetReset} \M{BitSet}
% \end{declcs}
% Macro \cs{bitsetReset} clears all bits. The result is an
% empty bit set. It may also be used as replacement for
% an operation ``new'', because an undefined bit set is defined
% afterwards.
%
% \begin{declcs}{bitsetLet} \M{BitSet A} \M{BitSet B}
% \end{declcs}
% Macro \cs{bitsetLet} performs a simple assignment similar
% to \TeX's \cs{let}. After the operation \meta{BitSet A} has
% the same value as \meta{BitSet B}. If \meta{BitSet B} is
% undefined, then \meta{BitSet A} will be the empty bit set.
%
% Note: If \meta{BitSet A} exists, it will be overwritten.
%
% \subsubsection{Import}\label{sec:import}
%
% \begin{declcs}{bitsetSetBin} \M{BitSet} \M{BinaryNumber}\\
%   \cs{bitsetSetOct} \M{BitSet} \M{OctalNumber}\\
%   \cs{bitsetSetHex} \M{BitSet} \M{HexadecimalNumber}
% \end{declcs}
% The numbers are interpreted as bit vectors and the flags in
% the bit \meta{BitSet} set are set accordingly.
% These numeric arguments are the only arguments where
% spaces are allowed. Then the numbers are easier to read.
%
% \begin{declcs}{bitsetSetDec} \M{BitSet} \M{DecimalNumber}
% \end{declcs}
% Macro \cs{bitsetSetDec} uses \meta{DecimalNumber} to
% set the bit set \meta{BitSet}.
% The numeric argument must expand
% to a plain number consisting of decimal digits without
% command tokens or spaces. Internally this argument is
% expanded only. It cannot be passed to \cs{number} or
% \cs{numexpr}, because the number may be too large for them.
% However \cs{number} or |\the\numexpr| may be used
% explicitely. This also helps for unexpandable number
% command tokens or registers (\cs{z@}, \cs{@ne}, \cs{count@}, \dots).
% Also \LaTeX'\ \cs{value} needs prefixing:
% \begin{quote}
% |\bitsetSetDec{foo}{\number\value{bar}}|
% \end{quote}
%
% \subsubsection{Export}\label{sec:export}
%
% \begin{declcs}{bitsetGetBin} \M{BitSet} \M{MinSize}\\
%   \cs{bitsetGetOct} \M{BitSet} \M{MinSize}\\
%   \cs{bitsetGetHex} \M{BitSet} \M{MinSize}
% \end{declcs}
% These macros returns the bit set as binary, octal or hexadecimal
% number. If the bit size is smaller than \meta{MinSize} the gap is
% filled with leading zeros.
% Example:
% \begin{quote}
% |\bitsetReset{abc}|\\
% |\bitsetSet{abc}{2}|\\
% |\bitsetGetBin{abc}{8}| $\rightarrow$ |00000100|\\
% |\bitsetSet{abc}{5}\bitsetSet{abc}{7}|\\
% |\bitsetGetHex{abc}{16}| $\rightarrow$ |00A2|
% \end{quote}
% Macro \cs{bitsetGetHex} uses the uppercase letters |A| to |F|.
% The catcode of the letters is one of 11 (letter) or 12 (other).
%
% \begin{declcs}{bitsetGetDec} \M{BitSet}
% \end{declcs}
% Macro \cs{bitsetGetDec} returns the bit set \meta{BitSet} as
% decimal number. The returned number can be larger than
% \TeX's number limit of $2^{31}-1$.
%
% \subsubsection{Logical operators}\label{sec:log}
%
% \begin{declcs}{bitsetAnd} \M{BitSet A} \M{BitSet B}
% \end{declcs}
% $A_{\text{new}} \mathrel{\mathop{:}}= A_{\text{old}}
% \mathbin{\text{and}} B \qquad (\forall\, \text{bits})$
%
% \begin{declcs}{bitsetAndNot} \M{BitSet A} \M{BitSet B}
% \end{declcs}
% $A_{\text{new}} \mathrel{\mathop{:}}= A_{\text{old}}
% \mathbin{\text{and}} (\text{not}\,B) \qquad (\forall\, \text{bits})$
%
% \begin{declcs}{bitsetOr} \M{BitSet A} \M{BitSet B}
% \end{declcs}
% $A_{\text{new}} \mathrel{\mathop{:}}= A_{\text{old}}
% \mathbin{\text{or}} B \qquad (\forall\, \text{bits})$
%
% \begin{declcs}{bitsetXor} \M{BitSet A} \M{BitSet B}
% \end{declcs}
% $A_{\text{new}} \mathrel{\mathop{:}}= A_{\text{old}}
% \mathbin{\text{xor}} B \qquad (\forall\, \text{bits})$
%
% \subsubsection{Shifting}\label{sec:shift}
%
% \begin{declcs}{bitsetShiftLeft} \M{BitSet} \M{ShiftAmount}\\
%   \cs{bitsetShiftRight} \M{BitSet} \M{ShiftAmount}
% \end{declcs}
% A left shift by one is a multiplication by two, thus left shifting
% moves the flags to higher positions. The new created low positions
% are filled by zeros.
%
% A right shift is the opposite, dividing by two, movint the bits
% to lower positions. The number will become smaller, the lowest
% bits are lost.
%
% If the \meta{ShiftAmount} is negative, it reverts the meaning of
% the shift operation. A left shift becomes a right shift.
% A \meta{ShiftAmount} of zero is ignored.
%
% \subsubsection{Bit manipulation}\label{sec:bitman}
%
% \begin{declcs}{bitsetClear} \M{BitSet} \M{Index}\\
%   \cs{bitsetSet} \M{BitSet} \M{Index}\\
%   \cs{bitsetFlip} \M{BitSet} \M{Index}
% \end{declcs}
% This macros manipulate a single bit in \meta{BitSet} addressed
% by \cs{Index}. Macro \cs{bitsetClear} disables the bit,
% \cs{bitsetSet} enables it and \cs{bitsetFlip} reverts the current
% setting of the bit.
%
% \begin{declcs}{bitsetSetValue} \M{BitSet} \M{Index} \M{Bit}
% \end{declcs}
% Macro \cs{bitsetSetValue} puts bit \meta{Bit} at position \meta{Index}
% in bit set \meta{BitSet}. \meta{Bit} must be a valid \TeX\ number
% equals to zero (disabled/cleared) or one (enabled/set).
%
% \subsubsection{Bit retrieval}\label{sec:bitret}
%
% \begin{declcs}{bitsetGet} \M{BitSet} \M{Index}
% \end{declcs}
% Macro \cs{bitsetGet} extracts the status of the bit at
% position \meta{Index} in bit set \meta{BitSet}.
% Digit |1| is returned if the bit is set/enabled. If the
% bit is cleared/disabled and in cases of an undefined bitset or
% an index number out of range the return value is |0|.
%
% \begin{declcs}{bitsetNextClearBit} \M{BitSet} \M{Index}
% \end{declcs}
% Starting at position \meta{Index} (inclusive) the bits
% are inspected. The first position without a set bit is returned.
% Possible results are decimal numbers: \meta{Index}, \meta{Index} + 1,
% \dots, ($\infty$)
%
% \begin{declcs}{bitsetNextSetBit} \M{BitSet} \M{Index}
% \end{declcs}
% Starting at position \meta{Index} (inclusive) the bits are
% inspected and the index position of the first found set bit
% is returned. If there isn't such a bit, then the result is
% |-1|. In summary possible results are decimal numbers:
% |-1|, \meta{Index}, \meta{Index} + 1, \dots, ($\infty$)
%
% \begin{declcs}{bitsetGetSetBitList} \M{BitSet}
% \end{declcs}
% Macro \cs{bitsetGetSetBitList} is an application for
% \cs{bitsetNextSetBit}. The set bits are iterated and
% returned as comma separated list of index positions
% in increasing order. The list is empty in case of
% an empty bit set.
%
% \subsubsection{Bit set properties}\label{sec:prop}
%
% \begin{declcs}{bitsetSize} \M{BitSet}
% \end{declcs}
% Macro \cs{bitsetSize} returns number of bits in use.
% It is the same as the index number of the highest set/enabled bit
% incremented by one.
%
% \begin{declcs}{bitsetCardinality} \M{BitSet}
% \end{declcs}
% Macro \cs{bitsetCardinality} counts the number of set/enabled bits.
%
% \subsubsection{Queries}\label{sec:query}
%
% Also the query procedures are expandable. They ask for a piece
% of information about a bit set and execute code depending
% on the answer.
%
% \begin{declcs}{bitsetIsDefined} \M{BitSet} \M{Then} \M{Else}
% \end{declcs}
% If the bit set with the name \meta{BitSet} exists the code
% given in \meta{Then} is executed, otherwise \meta{Else} is used.
%
% \begin{declcs}{bitsetIsEmpty} \M{BitSet} \M{Then} \M{Else}
% \end{declcs}
% If the bit set \meta{BitSet} exists and at least one bit
% is set/enabled, the code in \meta{Then} is executed,
% \meta{Else} otherwise.
%
% \begin{declcs}{bitsetEquals} \M{BitSet A} \M{BitSet B} \M{Then} \M{Else}
% \end{declcs}
% Both bit sets are equal if and only if either both are undefined
% or both are defined and represents the same bit values at the
% same positions. Thus this definition is reflexive, symmeric,
% and transitive, enough for an equivalent relation.
%
% \begin{declcs}{bitsetIntersects} \M{BitSet A} \M{BitSet B} \M{Then} \M{Else}
% \end{declcs}
% If and only if \meta{BitSet A} and \meta{BitSet B} have
% at least one bit at the same position that is set, then
% code part \meta{Then} is executed.
%
% \begin{declcs}{bitsetQuery} \M{BitSet} \M{Index} \M{Then} \M{Else}
% \end{declcs}
% It's just a wrapper for \cs{bitsetGet}. If the bit at position
% \meta{Index} is enabled, code \meta{Then} is called.
%
% \StopEventually{
% }
%
% \section{Implementation}
%
% The internal format of a bit set is quite simple, a sequence
% of digits |0| and |1|. The least significant bit is left.
% A bit set without any flag set is encoded by |0|. Also
% undefined bit sets are treated that way. After the
% highest bit that is set there are no further zeroes.
% A regular expression of valid bit sets values:
% \begin{quote}
%  |0|\string||[01]*1|
% \end{quote}
%
%    \begin{macrocode}
%<*package>
%    \end{macrocode}
%
% \subsection{Reload check and package identification}
%    Reload check, especially if the package is not used with \LaTeX.
%    \begin{macrocode}
\begingroup\catcode61\catcode48\catcode32=10\relax%
  \catcode13=5 % ^^M
  \endlinechar=13 %
  \catcode35=6 % #
  \catcode39=12 % '
  \catcode44=12 % ,
  \catcode45=12 % -
  \catcode46=12 % .
  \catcode58=12 % :
  \catcode64=11 % @
  \catcode123=1 % {
  \catcode125=2 % }
  \expandafter\let\expandafter\x\csname ver@bitset.sty\endcsname
  \ifx\x\relax % plain-TeX, first loading
  \else
    \def\empty{}%
    \ifx\x\empty % LaTeX, first loading,
      % variable is initialized, but \ProvidesPackage not yet seen
    \else
      \expandafter\ifx\csname PackageInfo\endcsname\relax
        \def\x#1#2{%
          \immediate\write-1{Package #1 Info: #2.}%
        }%
      \else
        \def\x#1#2{\PackageInfo{#1}{#2, stopped}}%
      \fi
      \x{bitset}{The package is already loaded}%
      \aftergroup\endinput
    \fi
  \fi
\endgroup%
%    \end{macrocode}
%    Package identification:
%    \begin{macrocode}
\begingroup\catcode61\catcode48\catcode32=10\relax%
  \catcode13=5 % ^^M
  \endlinechar=13 %
  \catcode35=6 % #
  \catcode39=12 % '
  \catcode40=12 % (
  \catcode41=12 % )
  \catcode44=12 % ,
  \catcode45=12 % -
  \catcode46=12 % .
  \catcode47=12 % /
  \catcode58=12 % :
  \catcode64=11 % @
  \catcode91=12 % [
  \catcode93=12 % ]
  \catcode123=1 % {
  \catcode125=2 % }
  \expandafter\ifx\csname ProvidesPackage\endcsname\relax
    \def\x#1#2#3[#4]{\endgroup
      \immediate\write-1{Package: #3 #4}%
      \xdef#1{#4}%
    }%
  \else
    \def\x#1#2[#3]{\endgroup
      #2[{#3}]%
      \ifx#1\@undefined
        \xdef#1{#3}%
      \fi
      \ifx#1\relax
        \xdef#1{#3}%
      \fi
    }%
  \fi
\expandafter\x\csname ver@bitset.sty\endcsname
\ProvidesPackage{bitset}%
  [2011/01/30 v1.1 Handle bit-vector datatype (HO)]%
%    \end{macrocode}
%
% \subsection{Catcodes}
%
%    \begin{macrocode}
\begingroup\catcode61\catcode48\catcode32=10\relax%
  \catcode13=5 % ^^M
  \endlinechar=13 %
  \catcode123=1 % {
  \catcode125=2 % }
  \catcode64=11 % @
  \def\x{\endgroup
    \expandafter\edef\csname BitSet@AtEnd\endcsname{%
      \endlinechar=\the\endlinechar\relax
      \catcode13=\the\catcode13\relax
      \catcode32=\the\catcode32\relax
      \catcode35=\the\catcode35\relax
      \catcode61=\the\catcode61\relax
      \catcode64=\the\catcode64\relax
      \catcode123=\the\catcode123\relax
      \catcode125=\the\catcode125\relax
    }%
  }%
\x\catcode61\catcode48\catcode32=10\relax%
\catcode13=5 % ^^M
\endlinechar=13 %
\catcode35=6 % #
\catcode64=11 % @
\catcode123=1 % {
\catcode125=2 % }
\def\TMP@EnsureCode#1#2{%
  \edef\BitSet@AtEnd{%
    \BitSet@AtEnd
    \catcode#1=\the\catcode#1\relax
  }%
  \catcode#1=#2\relax
}
\TMP@EnsureCode{33}{12}% !
\TMP@EnsureCode{39}{12}% '
\TMP@EnsureCode{40}{12}% (
\TMP@EnsureCode{41}{12}% )
\TMP@EnsureCode{42}{12}% *
\TMP@EnsureCode{43}{12}% +
\TMP@EnsureCode{44}{12}% ,
\TMP@EnsureCode{45}{12}% -
\TMP@EnsureCode{46}{12}% .
\TMP@EnsureCode{47}{12}% /
\TMP@EnsureCode{58}{11}% : (letter!)
\TMP@EnsureCode{60}{12}% <
\TMP@EnsureCode{62}{12}% >
\TMP@EnsureCode{63}{14}% ? (comment!)
\TMP@EnsureCode{91}{12}% [
\TMP@EnsureCode{93}{12}% ]
\TMP@EnsureCode{96}{12}% `
\edef\BitSet@AtEnd{\BitSet@AtEnd\noexpand\endinput}
\begingroup\expandafter\expandafter\expandafter\endgroup
\expandafter\ifx\csname BitSet@TestMode\endcsname\relax
\else
  \catcode63=9 % ? (ignore)
\fi
? \let\BitSet@@TestMode\BitSet@TestMode
%    \end{macrocode}
%
% \subsection{Package loading}
%
%    \begin{macrocode}
\begingroup\expandafter\expandafter\expandafter\endgroup
\expandafter\ifx\csname RequirePackage\endcsname\relax
  \def\TMP@RequirePackage#1[#2]{%
    \begingroup\expandafter\expandafter\expandafter\endgroup
    \expandafter\ifx\csname ver@#1.sty\endcsname\relax
      \input #1.sty\relax
    \fi
  }%
  \TMP@RequirePackage{infwarerr}[2007/09/09]%
  \TMP@RequirePackage{intcalc}[2007/09/27]%
  \TMP@RequirePackage{bigintcalc}[2007/09/27]%
\else
  \RequirePackage{infwarerr}[2007/09/09]%
  \RequirePackage{intcalc}[2007/09/27]%
  \RequirePackage{bigintcalc}[2007/09/27]%
\fi
%    \end{macrocode}
%
% \subsection{Help macros}
%
% \subsubsection{Number constant}
%
%    \begin{macro}{\BitSet@MaxSize}
%    \begin{macrocode}
\def\BitSet@MaxSize{2147483647}%
%    \end{macrocode}
%    \end{macro}
%
% \subsubsection{General basic macros}
%
%    \begin{macro}{\BitSet@Empty}
%    \begin{macrocode}
\def\BitSet@Empty{}
%    \end{macrocode}
%    \end{macro}
%    \begin{macro}{\BitSet@FirstOfOne}
%    \begin{macrocode}
\def\BitSet@FirstOfOne#1{#1}
%    \end{macrocode}
%    \end{macro}
%    \begin{macro}{\BitSet@Gobble}
%    \begin{macrocode}
\def\BitSet@Gobble#1{}
%    \end{macrocode}
%    \end{macro}
%    \begin{macro}{\BitSet@FirstOfTwo}
%    \begin{macrocode}
\def\BitSet@FirstOfTwo#1#2{#1}
%    \end{macrocode}
%    \end{macro}
%    \begin{macro}{\BitSet@SecondOfTwo}
%    \begin{macrocode}
\def\BitSet@SecondOfTwo#1#2{#2}
%    \end{macrocode}
%    \end{macro}
%
%    \begin{macro}{\BitSet@Space}
%    \begin{macrocode}
\def\BitSet@Space{ }
%    \end{macrocode}
%    \end{macro}
%    \begin{macro}{\BitSet@ZapSpace}
%    \begin{macrocode}
\def\BitSet@ZapSpace#1 #2{%
  #1%
  \ifx\BitSet@Empty#2%
  \else
    \expandafter\BitSet@ZapSpace
  \fi
  #2%
}
%    \end{macrocode}
%    \end{macro}
%
% \subsubsection{Tail recursion}
%
%    \begin{macro}{\BitSet@Fi}
%    \begin{macrocode}
\let\BitSet@Fi\fi
%    \end{macrocode}
%    \end{macro}
%    \begin{macro}{\BitSet@AfterFi}
%    \begin{macrocode}
\def\BitSet@AfterFi#1#2\BitSet@Fi{\fi#1}
%    \end{macrocode}
%    \end{macro}
%    \begin{macro}{\BitSet@AfterFiFi}
%    \begin{macrocode}
\def\BitSet@AfterFiFi#1#2\BitSet@Fi{\fi\fi#1}%
%    \end{macrocode}
%    \end{macro}
%    \begin{macro}{\BitSet@AfterFiFiFi}
%    \begin{macrocode}
\def\BitSet@AfterFiFiFi#1#2\BitSet@Fi{\fi\fi\fi#1}%
%    \end{macrocode}
%    \end{macro}
%
% \subsubsection{Check macros}
%
%    \begin{macro}{\BitSet@IfUndefined}
%    \begin{macrocode}
\def\BitSet@IfUndefined#1{%
  \expandafter\ifx\csname BS@#1\endcsname\relax
    \expandafter\BitSet@FirstOfTwo
  \else
    \expandafter\BitSet@SecondOfTwo
  \fi
}
%    \end{macrocode}
%    \end{macro}
%
%    \begin{macro}{\BitSet@CheckIndex}
%    |#1|: continuation code\\
%    |#2|: BitSet\\
%    |#3|: Index
%    \begin{macrocode}
\def\BitSet@CheckIndex#1#2#3{%
  \BitSet@IfUndefined{#2}{\bitsetReset{#2}}{}%
  \expandafter\expandafter\expandafter\BitSet@@CheckIndex
  \intcalcNum{#3}!%
  {#2}{#1}%
}
%    \end{macrocode}
%    \end{macro}
%    \begin{macro}{\BitSet@@CheckIndex}
%    |#1|: plain Index\\
%    |#2|: BitSet\\
%    |#3|: continuation code
%    \begin{macrocode}
\def\BitSet@@CheckIndex#1!#2#3{%
  \ifnum#1<0 %
    \BitSet@AfterFi{%
      \@PackageError{bitset}{%
        Invalid negative index (#1)%
      }\@ehc
    }%
  \else
    \BitSet@AfterFi{%
      #3{#2}{#1}%
    }%
  \BitSet@Fi
}
%    \end{macrocode}
%    \end{macro}
%
% \subsection{Miscellaneous}\label{src:misc}
%
%    \begin{macro}{\bitsetReset}
%    \begin{macrocode}
\def\bitsetReset#1{%
  \expandafter\def\csname BS@#1\endcsname{0}%
}
%    \end{macrocode}
%    \end{macro}
%
%    \begin{macro}{\bitsetLet}
%    \begin{macrocode}
\def\bitsetLet#1#2{%
  \BitSet@IfUndefined{#2}{%
    \bitsetReset{#1}%
  }{%
    \expandafter\let\csname BS@#1\expandafter\endcsname
                    \csname BS@#2\endcsname
  }%
}
%    \end{macrocode}
%    \end{macro}
%
% \subsection{Import}\label{src:import}
%
% \subsubsection{From binary number}
%
%    \begin{macro}{\bitsetSetBin}
%    \begin{macrocode}
\def\bitsetSetBin#1#2{%
  \edef\BitSet@Temp{#2}%
  \edef\BitSet@Temp{%
    \expandafter\expandafter\expandafter\BitSet@ZapSpace
    \expandafter\BitSet@Temp\BitSet@Space\BitSet@Empty
  }%
  \edef\BitSet@Temp{%
    \expandafter\BitSet@KillZeros\BitSet@Temp\BitSet@Empty
  }%
  \ifx\BitSet@Temp\BitSet@Empty
    \expandafter\let\csname BS@#1\endcsname\BitSet@Zero
  \else
    \expandafter\edef\csname BS@#1\endcsname{%
      \expandafter\BitSet@Reverse\BitSet@Temp!%
    }%
  \fi
}
%    \end{macrocode}
%    \end{macro}
%    \begin{macro}{\BitSet@KillZeros}
%    \begin{macrocode}
\def\BitSet@KillZeros#1{%
  \ifx#10%
    \expandafter\BitSet@KillZeros
  \else
    #1%
  \fi
}
%    \end{macrocode}
%    \end{macro}
%    \begin{macro}{\BitSet@Reverse}
%    \begin{macrocode}
\def\BitSet@Reverse#1#2!{%
  \ifx\\#2\\%
    #1%
  \else
    \BitSet@AfterFi{%
      \BitSet@Reverse#2!#1%
    }%
  \BitSet@Fi
}
%    \end{macrocode}
%    \end{macro}
%
% \subsubsection{From octal/hex number}
%
%    \begin{macro}{\bitsetSetOct}
%    \begin{macrocode}
\def\bitsetSetOct{%
  \BitSet@SetOctHex\BitSet@FromFirstOct
}
%    \end{macrocode}
%    \end{macro}
%    \begin{macro}{\bitsetSetHex}
%    \begin{macrocode}
\def\bitsetSetHex{%
  \BitSet@SetOctHex\BitSet@FromFirstHex
}
%    \end{macrocode}
%    \end{macro}
%    \begin{macro}{\BitSet@SetOctHex}
%    \begin{macrocode}
\def\BitSet@SetOctHex#1#2#3{%
  \edef\BitSet@Temp{#3}%
  \edef\BitSet@Temp{%
    \expandafter\expandafter\expandafter\BitSet@ZapSpace
    \expandafter\BitSet@Temp\BitSet@Space\BitSet@Empty
  }%
  \edef\BitSet@Temp{%
    \expandafter\BitSet@KillZeros\BitSet@Temp\BitSet@Empty
  }%
  \ifx\BitSet@Temp\BitSet@Empty
    \expandafter\let\csname BS@#2\endcsname\BitSet@Zero
  \else
    \edef\BitSet@Temp{%
      \expandafter#1\BitSet@Temp!%
    }%
    \ifx\BitSet@Temp\BitSet@Empty
      \expandafter\let\csname BS@#2\endcsname\BitSet@Zero
    \else
      \expandafter\edef\csname BS@#2\endcsname{%
        \expandafter\BitSet@Reverse\BitSet@Temp!%
      }%
    \fi
  \fi
}
%    \end{macrocode}
%    \end{macro}
%    \begin{macro}{\BitSet@FromFirstOct}
%    \begin{macrocode}
\def\BitSet@FromFirstOct#1{%
  \ifx#1!%
  \else
    \ifcase#1 \BitSet@AfterFiFi\BitSet@FromFirstOct
    \or 1%
    \or 10%
    \or 11%
    \or 100%
    \or 101%
    \or 110%
    \or 111%
    \else \BitSetError:WrongOctalDigit%
    \fi
    \expandafter\BitSet@FromOct
  \BitSet@Fi
}
%    \end{macrocode}
%    \end{macro}
%    \begin{macro}{\BitSet@FromOct}
%    \begin{macrocode}
\def\BitSet@FromOct#1{%
  \ifx#1!%
  \else
    \ifcase#1 000%
    \or 001%
    \or 010%
    \or 011%
    \or 100%
    \or 101%
    \or 110%
    \or 111%
    \else \BitSetError:WrongOctalDigit%
    \fi
    \expandafter\BitSet@FromOct
  \fi
}
%    \end{macrocode}
%    \end{macro}
%
%    \begin{macro}{\BitSet@FromFirstHex}
%    \begin{macrocode}
\def\BitSet@FromFirstHex#1{%
  \ifx#1!%
  \else
    \ifx#10%
      \BitSet@AfterFiFi\BitSet@FromFirstHex
    \fi
    \expandafter\ifx\csname BitSet@Hex#1\endcsname\relax
      \BitSetError:InvalidHexDigit%
    \else
      \expandafter\expandafter\expandafter\BitSet@KillZeros
      \csname BitSet@Hex#1\endcsname
    \fi
    \expandafter\BitSet@FromHex
  \BitSet@Fi
}
%    \end{macrocode}
%    \end{macro}
%    \begin{macro}{\BitSet@FromHex}
%    \begin{macrocode}
\def\BitSet@FromHex#1{%
  \ifx#1!%
  \else
    \expandafter\ifx\csname BitSet@Hex#1\endcsname\relax
      \BitSetError:InvalidHexDigit%
    \else
      \csname BitSet@Hex#1\endcsname
    \fi
    \expandafter\BitSet@FromHex
  \fi
}
%    \end{macrocode}
%    \end{macro}
%    \begin{macro}{\BitSet@Hex[0..F]}
%    \begin{macrocode}
\def\BitSet@Temp#1{%
  \expandafter\def\csname BitSet@Hex#1\endcsname
}
\BitSet@Temp 0{0000}%
\BitSet@Temp 1{0001}%
\BitSet@Temp 2{0010}%
\BitSet@Temp 3{0011}%
\BitSet@Temp 4{0100}%
\BitSet@Temp 5{0101}%
\BitSet@Temp 6{0110}%
\BitSet@Temp 7{0111}%
\BitSet@Temp 8{1000}%
\BitSet@Temp 9{1001}%
\BitSet@Temp A{1010}%
\BitSet@Temp B{1011}%
\BitSet@Temp C{1100}%
\BitSet@Temp D{1101}%
\BitSet@Temp E{1110}%
\BitSet@Temp F{1111}%
\BitSet@Temp a{1010}%
\BitSet@Temp b{1011}%
\BitSet@Temp c{1100}%
\BitSet@Temp d{1101}%
\BitSet@Temp e{1110}%
\BitSet@Temp f{1111}%
%    \end{macrocode}
%    \end{macro}
%
% \subsubsection{From decimal number}
%
%    \begin{macro}{\bitsetSetDec}
%    \begin{macrocode}
\def\bitsetSetDec#1#2{%
  \edef\BitSet@Temp{#2}%
  \edef\BitSet@Temp{%
    \expandafter\expandafter\expandafter\BitSet@ZapSpace
    \expandafter\BitSet@Temp\BitSet@Space\BitSet@Empty
  }%
  \edef\BitSet@Temp{%
    \expandafter\BitSet@KillZeros\BitSet@Temp\BitSet@Empty
  }%
  \ifx\BitSet@Temp\BitSet@Empty
    \expandafter\let\csname BS@#1\endcsname\BitSet@Zero
  \else
    \ifcase\bigintcalcSgn{\BitSet@Temp} %
      \expandafter\let\csname BS@#1\endcsname\BitSet@Zero
    \or
      \ifnum\bigintcalcCmp\BitSet@Temp\BitSet@MaxSize>0 %
        \expandafter\edef\csname BS@#1\endcsname{%
          \expandafter\BitSet@SetDecBig\BitSet@Temp!%
        }%
      \else
        \expandafter\edef\csname BS@#1\endcsname{%
          \expandafter\BitSet@SetDec\BitSet@Temp!%
        }%
      \fi
    \else
      \@PackageError{bitset}{%
        Bit sets cannot be negative%
      }\@ehc
    \fi
  \fi
}
%    \end{macrocode}
%    \end{macro}
%    \begin{macro}{\BitSet@SetDecBig}
%    \begin{macrocode}
\def\BitSet@SetDecBig#1#2#3#4#5#6#7#8#9!{%
  \ifx\\#9\\%
    \BitSet@SetDec#1#2#3#4#5#6#7#8!%
  \else
    \ifcase\BigIntCalcOdd#1#2#4#5#6#7#8#9! %
      0%
    \or
      1%
?   \else\BitSetError:ThisCannotHappen%
    \fi
    \BitSet@AfterFi{%
      \expandafter\expandafter\expandafter\BitSet@SetDecBig
      \BigIntCalcShr#1#2#3#4#5#6#7#8#9!!%
    }%
  \BitSet@Fi
}
%    \end{macrocode}
%    \end{macro}
%    \begin{macro}{\BitSet@SetDec}
%    \begin{macrocode}
\def\BitSet@SetDec#1!{%
  \ifcase#1 %
  \or 1%
  \else
    \ifodd#1 %
      1%
    \else
      0%
    \fi
    \BitSet@AfterFi{%
      \expandafter\expandafter\expandafter\BitSet@SetDec
      \IntCalcShr#1!!%
    }%
  \BitSet@Fi
}
%    \end{macrocode}
%    \end{macro}
%
% \subsection{Export}\label{src:export}
%
% \subsubsection{To binary number}
%
%    \begin{macro}{\bitsetGetBin}
%    \begin{macrocode}
\def\bitsetGetBin#1#2{%
  \romannumeral0%
  \expandafter\expandafter\expandafter\BitSet@@GetBin
  \intcalcNum{#2}!{#1}%
}
%    \end{macrocode}
%    \end{macro}
%    \begin{macro}{\BitSet@@GetBin}
%    \begin{macrocode}
\def\BitSet@@GetBin#1!#2{%
  \BitSet@IfUndefined{#2}{%
    \ifnum#1>1 %
      \BitSet@AfterFi{%
        \expandafter\expandafter\expandafter\BitSet@Fill
        \IntCalcDec#1!!0%
      }%
    \else
      \BitSet@AfterFi{ 0}%
    \BitSet@Fi
  }{%
    \expandafter\expandafter\expandafter\BitSet@NumBinRev
    \expandafter\expandafter\expandafter1%
    \expandafter\expandafter\expandafter!%
    \csname BS@#2\endcsname!!#1!%
  }%
}
%    \end{macrocode}
%    \end{macro}
%    \begin{macro}{\BitSet@Fill}
%    |#1|: number of leading digits |0|\\
%    |#2|: result
%    \begin{macrocode}
\def\BitSet@Fill#1!{%
  \ifnum#1>0 %
    \BitSet@AfterFi{%
      \expandafter\expandafter\expandafter\BitSet@Fill
      \IntCalcDec#1!!0%
    }%
  \else
    \BitSet@AfterFi{ }%
  \BitSet@Fi
}
%    \end{macrocode}
%    \end{macro}
%
%    \begin{macro}{\BitSet@NumBinRev}
%    |#1|: bit counter (including |#2|)\\
%    |#2#3|: reverted number\\
%    |#4|: result\\
%    |#5|: min size
%    \begin{macrocode}
\def\BitSet@NumBinRev#1!#2#3!{%
  \ifx\\#3\\%
    \BitSet@AfterFi{%
      \BitSet@NumBinFill#1!#2%
    }%
  \else
    \BitSet@AfterFi{%
      \expandafter\expandafter\expandafter\BitSet@NumBinRev
      \IntCalcInc#1!!#3!#2%
    }%
  \BitSet@Fi
}
%    \end{macrocode}
%    \end{macro}
%    \begin{macro}{\BitSet@NumBinFill}
%    \begin{macrocode}
\def\BitSet@NumBinFill#1!#2!#3!{%
  \ifnum#3>#1 %
    \BitSet@AfterFi{%
      \expandafter\expandafter\expandafter\BitSet@Fill
      \IntCalcSub#3!#1!!#2%
    }%
  \else
    \BitSet@AfterFi{ #2}%
  \BitSet@Fi
}
%    \end{macrocode}
%    \end{macro}
%
% \subsubsection{To octal/hexadecimal number}
%
%    \begin{macro}{\bitsetGetOct}
%    \begin{macrocode}
\def\bitsetGetOct#1#2{%
  \romannumeral0%
  \bitsetIsEmpty{#1}{%
    \expandafter\expandafter\expandafter\BitSet@@GetOctHex
    \intcalcNum{#2}!3!230%
  }{%
    \expandafter\expandafter\expandafter\BitSet@@GetOct
    \expandafter\expandafter\expandafter1%
    \expandafter\expandafter\expandafter!%
    \expandafter\expandafter\expandafter!%
    \csname BS@#1\endcsname00%
    \BitSet@Empty\BitSet@Empty\BitSet@Empty!{#2}%
  }%
}
%    \end{macrocode}
%    \end{macro}
%    \begin{macro}{\bitsetGetHex}
%    \begin{macrocode}
\def\bitsetGetHex#1#2{%
  \romannumeral0%
  \bitsetIsEmpty{#1}{%
    \expandafter\expandafter\expandafter\BitSet@@GetOctHex
    \intcalcNum{#2}!4!340%
  }{%
    \expandafter\expandafter\expandafter\BitSet@@GetHex
    \expandafter\expandafter\expandafter1%
    \expandafter\expandafter\expandafter!%
    \expandafter\expandafter\expandafter!%
    \csname BS@#1\endcsname000%
    \BitSet@Empty\BitSet@Empty\BitSet@Empty\BitSet@Empty!{#2}%
  }%
}
%    \end{macrocode}
%    \end{macro}
%    \begin{macro}{\BitSet@@GetOct}
%    |#1|: number of digits\\
%    |#2|: result\\
%    |#3#4#5|: bits
%    \begin{macrocode}
\def\BitSet@@GetOct#1!#2!#3#4#5{%
  \ifx#5\BitSet@Empty
    \BitSet@AfterFi{%
      \expandafter\expandafter\expandafter\BitSet@GetOctHex
      \IntCalcDec#1!!#2!23%
    }%
  \else
    \BitSet@AfterFi{%
      \expandafter\expandafter\expandafter\BitSet@@GetOct
      \number\IntCalcInc#1!\expandafter\expandafter\expandafter!%
      \csname BitSet@Oct#5#4#3\endcsname#2!%
    }%
  \BitSet@Fi
}
%    \end{macrocode}
%    \end{macro}
%    \begin{macro}{\BitSet@Oct[000..111]}
%    \begin{macrocode}
\def\BitSet@Temp#1#2#3#4{%
  \expandafter\def\csname BitSet@Oct#1#2#3\endcsname{#4}%
}
\BitSet@Temp0000%
\BitSet@Temp0011%
\BitSet@Temp0102%
\BitSet@Temp0113%
\BitSet@Temp1004%
\BitSet@Temp1015%
\BitSet@Temp1106%
\BitSet@Temp1117%
%    \end{macrocode}
%    \end{macro}
%    \begin{macro}{\BitSet@@GetHex}
%    |#1|: number of digits\\
%    |#2|: result\\
%    |#3#4#5#6|: bits
%    \begin{macrocode}
\def\BitSet@@GetHex#1!#2!#3#4#5#6{%
  \ifx#6\BitSet@Empty
    \BitSet@AfterFi{%
      \expandafter\expandafter\expandafter\BitSet@GetOctHex
      \IntCalcDec#1!!#2!34%
    }%
  \else
    \BitSet@AfterFi{%
      \expandafter\expandafter\expandafter\BitSet@@GetHex
      \number\IntCalcInc#1!\expandafter\expandafter\expandafter!%
      \csname BitSet@Hex#6#5#4#3\endcsname#2!%
    }%
  \BitSet@Fi
}
%    \end{macrocode}
%    \end{macro}
%    \begin{macro}{\BitSet@Hex[0000..1111]}
%    \begin{macrocode}
\def\BitSet@Temp#1#2#3#4#5{%
  \expandafter\def\csname BitSet@Hex#1#2#3#4\endcsname{#5}%
}
\BitSet@Temp00000%
\BitSet@Temp00011%
\BitSet@Temp00102%
\BitSet@Temp00113%
\BitSet@Temp01004%
\BitSet@Temp01015%
\BitSet@Temp01106%
\BitSet@Temp01117%
\BitSet@Temp10008%
\BitSet@Temp10019%
\BitSet@Temp1010A%
\BitSet@Temp1011B%
\BitSet@Temp1100C%
\BitSet@Temp1101D%
\BitSet@Temp1110E%
\BitSet@Temp1111F%
%    \end{macrocode}
%    \end{macro}
%    \begin{macro}{\BitSet@GetOctHex}
%    Leading zeros $ (|#4| - |#1| * 3 + 2)/3 $ if $ |#4| > |#1| * 3$\\
%    |#1|: digit size\\
%    |#2|: result\\
%    |#3|: bits per digit - 1\\
%    |#4|: bits per digit
%    |#5|: garbage\\
%    |#6|: min size
%    \begin{macrocode}
\def\BitSet@GetOctHex#1!#2!#3#4#5!#6{%
  \expandafter\BitSet@@GetOctHex
  \number\intcalcNum{#6}\expandafter\expandafter\expandafter!%
  \IntCalcMul#1!#4!!#3#4#2%
}
%    \end{macrocode}
%    \end{macro}
%    \begin{macro}{\BitSet@@GetOctHex}
%    |#1|: plain min size\\
%    |#2|: digits * (bits per digit)\\
%    |#3|: bits per digit - 1\\
%    |#4|: bits per digit
%    \begin{macrocode}
\def\BitSet@@GetOctHex#1!#2!#3#4{%
  \ifnum#1>#2 %
    \BitSet@AfterFi{%
      \expandafter\expandafter\expandafter\expandafter
      \expandafter\expandafter\expandafter\BitSet@Fill
      \expandafter\IntCalcDiv\number
      \expandafter\expandafter\expandafter\IntCalcAdd
      \IntCalcSub#1!#2!!#3!!#4!!%
    }%
  \else
    \BitSet@AfterFi{ }%
  \BitSet@Fi
}
%    \end{macrocode}
%    \end{macro}
%
% \subsubsection{To decimal number}
%
%    \begin{macro}{\bitsetGetDec}
%    \begin{macrocode}
\def\bitsetGetDec#1{%
  \romannumeral0%
  \BitSet@IfUndefined{#1}{ 0}{%
    \expandafter\expandafter\expandafter\BitSet@GetDec
    \csname BS@#1\endcsname!%
  }%
}
%    \end{macrocode}
%    \end{macro}
%    \begin{macro}{\BitSet@GetDec}
%    \begin{macrocode}
\def\BitSet@GetDec#1#2!{%
  \ifx\\#2\\%
    \BitSet@AfterFi{ #1}%
  \else
    \BitSet@AfterFi{%
      \BitSet@@GetDec2!#1!#2!%
    }%
  \BitSet@Fi
}
%    \end{macrocode}
%    \end{macro}
%    \begin{macro}{\BitSet@@GetDec}
%    |#1|: power of two\\
%    |#2|: result\\
%    |#3#4|: number
%    \begin{macrocode}
\def\BitSet@@GetDec#1!#2!#3#4!{%
  \ifx\\#4\\%
    \ifx#31%
      \BitSet@AfterFiFi{%
        \expandafter\expandafter\expandafter\BitSet@Space
        \IntCalcAdd#1!#2!%
      }%
    \else
      \BitSet@AfterFiFi{ #2}%
    \fi
  \else
    \ifx#31%
      \BitSet@AfterFiFi{%
        \csname BitSet@N#1%
        \expandafter\expandafter\expandafter\endcsname
        \IntCalcAdd#1!#2!!#4!%
      }%
    \else
      \BitSet@AfterFiFi{%
        \csname BitSet@N#1\endcsname#2!#4!%
      }%
    \fi
  \BitSet@Fi
}
%    \end{macrocode}
%    \end{macro}
%    \begin{macro}{\BitSet@N[1,2,4,...]}
%    \begin{macrocode}
\def\BitSet@Temp#1#2{%
  \expandafter\def\csname BitSet@N#1\endcsname{%
    \BitSet@@GetDec#2!%
  }%
}
\BitSet@Temp{1}{2}
\BitSet@Temp{2}{4}
\BitSet@Temp{4}{8}
\BitSet@Temp{8}{16}
\BitSet@Temp{16}{32}
\BitSet@Temp{32}{64}
\BitSet@Temp{64}{128}
\BitSet@Temp{128}{256}
\BitSet@Temp{256}{512}
\BitSet@Temp{512}{1024}
\BitSet@Temp{1024}{2048}
\BitSet@Temp{2048}{4096}
\BitSet@Temp{4096}{8192}
\BitSet@Temp{8192}{16384}
\BitSet@Temp{16384}{32768}
\BitSet@Temp{32768}{65536}
\BitSet@Temp{65536}{131072}
\BitSet@Temp{131072}{262144}
\BitSet@Temp{262144}{524288}
\BitSet@Temp{524288}{1048576}
\BitSet@Temp{1048576}{2097152}
\BitSet@Temp{2097152}{4194304}
\BitSet@Temp{4194304}{8388608}
\BitSet@Temp{8388608}{16777216}
\BitSet@Temp{16777216}{33554432}
\BitSet@Temp{33554432}{67108864}
\BitSet@Temp{67108864}{134217728}
\BitSet@Temp{134217728}{268435456}
\BitSet@Temp{268435456}{536870912}
\BitSet@Temp{536870912}{1073741824}
%    \end{macrocode}
%    \end{macro}
%    \begin{macro}{\BitSet@N1073741824}
%    \begin{macrocode}
\expandafter\def\csname BitSet@N1073741824\endcsname{%
  \BitSet@GetDecBig2147483648!%
}%
%    \end{macrocode}
%    \end{macro}
%
%    \begin{macro}{\BitSet@GetDecBig}
%    |#1|: current power of two\\
%    |#2|: result\\
%    |#3#4|: number
%    \begin{macrocode}
\def\BitSet@GetDecBig#1!#2!#3#4!{%
  \ifx\\#4\\%
    \ifx#31%
      \BitSet@AfterFiFi{%
        \expandafter\expandafter\expandafter\BitSet@Space
        \BigIntCalcAdd#1!#2!%
      }%
    \else
      \BitSet@AfterFiFi{ #2}%
    \fi
  \else
    \ifx#31%
      \BitSet@AfterFiFi{%
        \expandafter\expandafter\expandafter\BitSet@@GetDecBig
        \BigIntCalcAdd#1!#2!!#1!#4!%
      }%
    \else
      \BitSet@AfterFiFi{%
        \expandafter\expandafter\expandafter\BitSet@GetDecBig
        \BigIntCalcShl#1!!#2!#4!%
      }%
    \fi
  \BitSet@Fi
}
%    \end{macrocode}
%    \end{macro}
%    \begin{macro}{\BitSet@@GetDecBig}
%    |#1|: result\\
%    |#2|: power of two\\
%    |#3#4|: number
%    \begin{macrocode}
\def\BitSet@@GetDecBig#1!#2!{%
  \expandafter\expandafter\expandafter\BitSet@GetDecBig
  \BigIntCalcShl#2!!#1!%
}
%    \end{macrocode}
%    \end{macro}
%
% \subsection{Logical operators}\label{src:log}
%
% \subsubsection{\cs{bitsetAnd}}
%
%    \begin{macro}{\bitsetAnd}
%    Decision table for \cs{bitsetAnd}:
%    \begin{quote}
%    \begin{tabular}{l|l|l|l|}
%                         & undef(B)   & empty(B)   & cardinality(B)$>$0\\
%      \hline
%      undef(A)           & A := empty & A := empty & A := empty\\
%      \hline
%      empty(A)           &            &            &\\
%      \hline
%      cardinality(A)$>$0 & A := empty & A := empty & A \&= B\\
%      \hline
%    \end{tabular}
%    \end{quote}
%    \begin{macrocode}
\def\bitsetAnd#1#2{%
  \bitsetIsEmpty{#1}{%
    \bitsetReset{#1}%
  }{%
    \bitsetIsEmpty{#2}{%
      \bitsetReset{#1}%
    }{%
      \expandafter\edef\csname BS@#1\endcsname{%
        \expandafter\expandafter\expandafter\BitSet@And
        \csname BS@#1\expandafter\expandafter\expandafter\endcsname
        \expandafter\expandafter\expandafter!%
        \csname BS@#2\endcsname!!%
      }%
      \expandafter\ifx\csname BS@#1\endcsname\BitSet@Empty
        \bitsetReset{#1}%
      \fi
    }%
  }%
}
%    \end{macrocode}
%    \end{macro}
%    \begin{macro}{\BitSet@And}
%    \begin{macrocode}
\def\BitSet@And#1#2!#3#4!#5!{%
  \ifx\\#2\\%
    \ifnum#1#3=11 #51\fi
  \else
    \ifx\\#4\\%
      \ifnum#1#3=11 #51\fi
    \else
      \ifnum#1#3=11 %
        #51%
        \BitSet@AfterFiFiFi{%
          \BitSet@And#2!#4!!%
        }%
      \else
        \BitSet@AfterFiFiFi{%
          \BitSet@And#2!#4!#50!%
        }%
      \fi
    \fi
  \BitSet@Fi
}
%    \end{macrocode}
%    \end{macro}
%
% \subsubsection{\cs{bitsetAndNot}}
%
%    \begin{macro}{\bitsetAndNot}
%    Decision table for \cs{bitsetAndNot}:
%    \begin{quote}
%    \begin{tabular}{l|l|l|l|}
%                         & undef(B)   & empty(B)   & cardinality(B)$>$0\\
%      \hline
%      undef(A)           & A := empty & A := empty & A := empty\\
%      \hline
%      empty(A)           &            &            &\\
%      \hline
%      cardinality(A)$>$0 &            &            & A \&= !B\\
%      \hline
%    \end{tabular}
%    \end{quote}
%    \begin{macrocode}
\def\bitsetAndNot#1#2{%
  \bitsetIsEmpty{#1}{%
    \bitsetReset{#1}%
  }{%
    \bitsetIsEmpty{#2}{%
    }{%
      \expandafter\edef\csname BS@#1\endcsname{%
        \expandafter\expandafter\expandafter\BitSet@AndNot
        \csname BS@#1\expandafter\expandafter\expandafter\endcsname
        \expandafter\expandafter\expandafter!%
        \csname BS@#2\endcsname!!%
      }%
      \expandafter\ifx\csname BS@#1\endcsname\BitSet@Empty
        \bitsetReset{#1}%
      \fi
    }%
  }%
}
%    \end{macrocode}
%    \end{macro}
%    \begin{macro}{\BitSet@AndNot}
%    \begin{macrocode}
\def\BitSet@AndNot#1#2!#3#4!#5!{%
  \ifx\\#2\\%
    \ifnum#1#3=10 #51\fi
  \else
    \ifx\\#4\\%
      #5%
      \ifnum#1#3=10 1\else 0\fi
      #2%
    \else
      \ifnum#1#3=10 %
        #51%
        \BitSet@AfterFiFiFi{%
          \BitSet@AndNot#2!#4!!%
        }%
      \else
        \BitSet@AfterFiFiFi{%
          \BitSet@AndNot#2!#4!#50!%
        }%
      \fi
    \fi
  \BitSet@Fi
}
%    \end{macrocode}
%    \end{macro}
%
% \subsubsection{\cs{bitsetOr}}
%
%    \begin{macro}{\bitsetOr}
%    Decision table for \cs{bitsetOr}:
%    \begin{quote}
%    \begin{tabular}{l|l|l|l|}
%                         & undef(B)   & empty(B)   & cardinality(B)$>$0\\
%      \hline
%      undef(A)           & A := empty & A := empty & A := B\\
%      \hline
%      empty(A)           &            &            & A := B\\
%      \hline
%      cardinality(A)$>$0 &            &            & A \string|= B\\
%      \hline
%    \end{tabular}
%    \end{quote}
%    \begin{macrocode}
\def\bitsetOr#1#2{%
  \bitsetIsEmpty{#2}{%
    \BitSet@IfUndefined{#1}{\bitsetReset{#1}}{}%
  }{%
    \bitsetIsEmpty{#1}{%
      \expandafter\let\csname BS@#1\expandafter\endcsname
                      \csname BS@#2\endcsname
    }{%
      \expandafter\edef\csname BS@#1\endcsname{%
        \expandafter\expandafter\expandafter\BitSet@Or
        \csname BS@#1\expandafter\expandafter\expandafter\endcsname
        \expandafter\expandafter\expandafter!%
        \csname BS@#2\endcsname!%
      }%
    }%
  }%
}
%    \end{macrocode}
%    \end{macro}
%    \begin{macro}{\BitSet@Or}
%    \begin{macrocode}
\def\BitSet@Or#1#2!#3#4!{%
  \ifnum#1#3>0 1\else 0\fi
  \ifx\\#2\\%
    #4%
  \else
    \ifx\\#4\\%
      #2%
    \else
      \BitSet@AfterFiFi{%
        \BitSet@Or#2!#4!%
      }%
    \fi
  \BitSet@Fi
}
%    \end{macrocode}
%    \end{macro}
%
% \subsubsection{\cs{bitsetXor}}
%
%    \begin{macro}{\bitsetXor}
%    Decision table for \cs{bitsetXor}:
%    \begin{quote}
%    \begin{tabular}{l|l|l|l|}
%                         & undef(B)   & empty(B)   & cardinality(B)$>$0\\
%      \hline
%      undef(A)           & A := empty & A := empty & A := B\\
%      \hline
%      empty(A)           &            &            & A := B\\
%      \hline
%      cardinality(A)$>$0 &            &            & A \string^= B\\
%      \hline
%    \end{tabular}
%    \end{quote}
%    \begin{macrocode}
\def\bitsetXor#1#2{%
  \bitsetIsEmpty{#2}{%
    \BitSet@IfUndefined{#1}{\bitsetReset{#1}}{}%
  }{%
    \bitsetIsEmpty{#1}{%
      \expandafter\let\csname BS@#1\expandafter\endcsname
                      \csname BS@#2\endcsname
    }{%
      \expandafter\edef\csname BS@#1\endcsname{%
        \expandafter\expandafter\expandafter\BitSet@Xor
        \csname BS@#1\expandafter\expandafter\expandafter\endcsname
        \expandafter\expandafter\expandafter!%
        \csname BS@#2\endcsname!!%
      }%
      \expandafter\ifx\csname BS@#1\endcsname\BitSet@Empty
        \bitsetReset{#1}%
      \fi
    }%
  }%
}
%    \end{macrocode}
%    \end{macro}
%    \begin{macro}{\BitSet@Xor}
%    \begin{macrocode}
\def\BitSet@Xor#1#2!#3#4!#5!{%
  \ifx\\#2\\%
    \ifx#1#3%
      \ifx\\#4\\%
      \else
        #50#4%
      \fi
    \else
      #51#4%
    \fi
  \else
    \ifx\\#4\\%
      #5%
      \ifx#1#30\else 1\fi
      #2%
    \else
      \ifx#1#3%
        \BitSet@AfterFiFiFi{%
          \BitSet@Xor#2!#4!#50!%
        }%
      \else
        #51%
        \BitSet@AfterFiFiFi{%
          \BitSet@Xor#2!#4!!%
        }%
      \fi
    \fi
  \BitSet@Fi
}
%    \end{macrocode}
%    \end{macro}
%
% \subsubsection{Shifting}\label{src:shift}
%
% \subsubsection{\cs{bitsetShiftLeft}}
%
%    \begin{macro}{\bitsetShiftLeft}
%    \begin{macrocode}
\def\bitsetShiftLeft#1#2{%
  \BitSet@IfUndefined{#1}{%
    \bitsetReset{#1}%
  }{%
    \bitsetIsEmpty{#1}{%
    }{%
      \expandafter\expandafter\expandafter\BitSet@ShiftLeft
      \intcalcNum{#2}!{#1}%
    }%
  }%
}
%    \end{macrocode}
%    \end{macro}
%    \begin{macro}{\BitSet@ShiftLeft}
%    \begin{macrocode}
\def\BitSet@ShiftLeft#1!#2{%
  \ifcase\intcalcSgn{#1} %
  \or
    \begingroup
      \uccode`m=`0 %
    \uppercase\expandafter{\expandafter\endgroup
      \expandafter\edef\csname BS@#2\expandafter\endcsname
      \expandafter{%
        \romannumeral#1000\expandafter\BitSet@Space
        \csname BS@#2\endcsname
      }%
    }%
  \else
    \expandafter\BitSet@ShiftRight\BitSet@Gobble#1!{#2}%
  \fi
}
%    \end{macrocode}
%    \end{macro}
%
% \subsubsection{\cs{bitsetShiftRight}}
%
%    \begin{macro}{\bitsetShiftRight}
%    \begin{macrocode}
\def\bitsetShiftRight#1#2{%
  \BitSet@IfUndefined{#1}{%
    \bitsetReset{#1}%
  }{%
    \bitsetIsEmpty{#1}{%
    }{%
      \expandafter\expandafter\expandafter\BitSet@ShiftRight
      \intcalcNum{#2}!{#1}%
    }%
  }%
}
%    \end{macrocode}
%    \end{macro}
%    \begin{macro}{\BitSet@ShiftRight}
%    \begin{macrocode}
\def\BitSet@ShiftRight#1!#2{%
  \ifcase\intcalcSgn{#1} %
  \or
    \expandafter\edef\csname BS@#2\endcsname{%
      \expandafter\expandafter\expandafter\BitSet@Kill
      \csname BS@#2\expandafter\endcsname\expandafter\BitSet@Empty
      \expandafter=%
      \expandafter{\expandafter}\expandafter{\expandafter}%
      \romannumeral#1000!%
    }%
  \else
    \expandafter\BitSet@ShiftLeft\BitSet@Gobble#1!{#2}%
  \fi
}
%    \end{macrocode}
%    \end{macro}
%    \begin{macro}{\BitSet@Kill}
%    \begin{macrocode}
\def\BitSet@Kill#1#2=#3#4#5{%
  #3#4%
  \ifx#5!%
    \ifx#1\BitSet@Empty
      0%
    \else
      #1#2%
    \fi
  \else
    \ifx#1\BitSet@Empty
      0%
      \BitSet@AfterFiFi\BitSet@Cleanup
    \else
      \BitSet@Kill#2=%
    \fi
  \BitSet@Fi
}
%    \end{macrocode}
%    \end{macro}
%
% \subsection{Bit manipulation}\label{src:bitman}
%
%    \begin{macro}{\bitsetClear}
%    \begin{macrocode}
\def\bitsetClear{%
  \BitSet@CheckIndex\BitSet@Clear
}
%    \end{macrocode}
%    \end{macro}
%    \begin{macro}{\bitsetSet}
%    \begin{macrocode}
\def\bitsetSet{%
  \BitSet@CheckIndex\BitSet@Set
}
%    \end{macrocode}
%    \end{macro}
%    \begin{macro}{\bitsetFlip}
%    \begin{macrocode}
\def\bitsetFlip{%
  \BitSet@CheckIndex\BitSet@Flip
}
%    \end{macrocode}
%    \end{macro}
%    \begin{macro}{\bitsetSetValue}
%    \begin{macrocode}
\def\bitsetSetValue#1#2#3{%
  \expandafter\expandafter\expandafter\BitSet@SetValue
  \intcalcNum{#3}!{#1}{#2}%
}
%    \end{macrocode}
%    \end{macro}
%    \begin{macro}{\BitSet@SetValue}
%    |#1|: plain value\\
%    |#2|: BitSet\\
%    |#3|: Index
%    \begin{macrocode}
\def\BitSet@SetValue#1!{%
  \BitSet@CheckIndex{%
    \ifcase#1 %
      \expandafter\BitSet@Clear
    \or
      \expandafter\BitSet@Set
    \else
      \BitSet@ErrorInvalidBitValue{#1}%
      \expandafter\expandafter\expandafter\BitSet@Gobble
      \expandafter\BitSet@Gobble
    \fi
  }%
}
%    \end{macrocode}
%    \end{macro}
%    \begin{macro}{\BitSet@ErrorInvalidBitValue}
%    |#1|: Wrong bit value
%    \begin{macrocode}
\def\BitSet@ErrorInvalidBitValue#1{%
  \@PackageError{bitset}{%
    Invalid bit value (#1) not in range 0..1%
  }\@ehc
}
%    \end{macrocode}
%    \end{macro}
%
% \subsubsection{Clear operation}
%
%    \begin{macro}{\BitSet@Clear}
%    |#1|: BitSet\\
%    |#2|: plain and checked index
%    \begin{macrocode}
\def\BitSet@Clear#1#2{%
  \edef\BitSet@Temp{%
    \expandafter\expandafter\expandafter\BitSet@@Clear
    \csname BS@#1\expandafter\endcsname
    \expandafter\BitSet@Empty\expandafter=\expandafter!%
    \romannumeral#2000!%
  }%
  \expandafter\let\csname BS@#1\expandafter\endcsname
  \ifx\BitSet@Temp\BitSet@Empty
    \BitSet@Zero
  \else
    \BitSet@Temp
  \fi
}
%    \end{macrocode}
%    \end{macro}
%    \begin{macro}{\BitSet@@Clear}
%    \begin{macrocode}
\def\BitSet@@Clear#1#2=#3!#4{%
  \ifx#4!%
    \ifx#1\BitSet@Empty
    \else
      \ifx\BitSet@Empty#2%
      \else
        #30#2%
      \fi
    \fi
  \else
    \ifx#1\BitSet@Empty
      \BitSet@AfterFiFi\BitSet@Cleanup
    \else
      \ifx#10%
        \BitSet@AfterFiFiFi{%
          \BitSet@@Clear#2=#30!%
        }%
      \else
        #31%
        \BitSet@AfterFiFiFi{%
          \BitSet@@Clear#2=!%
        }%
      \fi
    \fi
  \BitSet@Fi
}
%    \end{macrocode}
%    \end{macro}
%
% \subsubsection{Set operation}
%
%    \begin{macro}{\BitSet@Set}
%    |#1|: BitSet\\
%    |#2|: plain and checked Index
%    \begin{macrocode}
\def\BitSet@Set#1#2{%
  \expandafter\edef\csname BS@#1\endcsname{%
    \expandafter\expandafter\expandafter\BitSet@@Set
    \csname BS@#1\expandafter\endcsname
    \expandafter\BitSet@Empty\expandafter=%
    \expandafter{\expandafter}\expandafter{\expandafter}%
    \romannumeral#2000!%
  }%
}
%    \end{macrocode}
%    \end{macro}
%    \begin{macro}{\BitSet@@Set}
%    \begin{macrocode}
\def\BitSet@@Set#1#2=#3#4#5{%
  #3#4%
  \ifx#5!%
    1#2%
  \else
    \ifx#1\BitSet@Empty
      0%
      \BitSet@AfterFiFi\BitSet@@@Set
    \else
      #1%
      \BitSet@@Set#2=%
    \fi
  \BitSet@Fi
}
%    \end{macrocode}
%    \end{macro}
%    \begin{macro}{\BitSet@@@Set}
%    \begin{macrocode}
\def\BitSet@@@Set#1{%
  \ifx#1!%
    1%
  \else
    0%
    \expandafter\BitSet@@@Set
  \fi
}
%    \end{macrocode}
%    \end{macro}
%
% \subsubsection{Flip operation}
%
%    \begin{macro}{\BitSet@Flip}
%    |#1|: BitSet\\
%    |#2|: plain and checked Index
%    \begin{macrocode}
\def\BitSet@Flip#1#2{%
  \edef\BitSet@Temp{%
    \expandafter\expandafter\expandafter\BitSet@@Flip
    \csname BS@#1\expandafter\endcsname
    \expandafter\BitSet@Empty\expandafter=\expandafter!%
    \romannumeral#2000!%
  }%
  \expandafter\let\csname BS@#1\expandafter\endcsname
  \ifx\BitSet@Temp\BitSet@Empty
    \BitSet@Zero
  \else
    \BitSet@Temp
  \fi
}
%    \end{macrocode}
%    \end{macro}
%    \begin{macro}{\BitSet@@Flip}
%    \begin{macrocode}
\def\BitSet@@Flip#1#2=#3!#4{%
  \ifx#4!%
    \ifx#11%
      \ifx\BitSet@Empty#2%
      \else
        #30#2%
      \fi
    \else
      #31#2%
    \fi
  \else
    \ifx#1\BitSet@Empty
      #30%
      \BitSet@AfterFiFi\BitSet@@@Set
    \else
      \ifx#10%
        \BitSet@AfterFiFiFi{%
          \BitSet@@Flip#2=#30!%
        }%
      \else
        #31%
        \BitSet@AfterFiFiFi{%
          \BitSet@@Flip#2=!%
        }%
      \fi
    \fi
  \BitSet@Fi
}
%    \end{macrocode}
%    \end{macro}
%
% \subsubsection{Range operators}
%
%    \begin{macro}{\bitsetClearRange}
%    \begin{macrocode}
\def\bitsetClearRange{%
  \BitSet@Range\BitSet@Clear
}
%    \end{macrocode}
%    \end{macro}
%    \begin{macro}{\bitsetSetRange}
%    \begin{macrocode}
\def\bitsetSetRange{%
  \BitSet@Range\BitSet@Set
}
%    \end{macrocode}
%    \end{macro}
%    \begin{macro}{\bitsetFlipRange}
%    \begin{macrocode}
\def\bitsetFlipRange{%
  \BitSet@Range\BitSet@Flip
}
%    \end{macrocode}
%    \end{macro}
%    \begin{macro}{\bitsetSetValueRange}
%    \begin{macrocode}
\def\bitsetSetValueRange#1#2#3#4{%
  \expandafter\expandafter\expandafter\BitSet@SetValueRange
  \intcalcNum{#4}!{#1}{#2}{#3}%
}
%    \end{macrocode}
%    \end{macro}
%    \begin{macro}{\BitSet@SetValueRange}
%    \begin{macrocode}
\def\BitSet@SetValueRange#1!#2#3#4{%
  \ifcase#1 %
    \BitSet@Range\BitSet@Clear{#2}{#3}{#4}%
  \or
    \BitSet@Range\BitSet@Set{#2}{#3}{#4}%
  \else
    \BitSet@ErrorInvalidBitValue{#1}%
  \fi
}
%    \end{macrocode}
%    \end{macro}
%
%    \begin{macro}{\BitSet@Range}
%    |#1|: clear/set/flip macro\\
%    |#2|: BitSet\\
%    |#3|: Index from\\
%    |#4|: Index to
%    \begin{macrocode}
\def\BitSet@Range#1#2#3#4{%
  \edef\BitSet@Temp{%
    \noexpand\BitSet@@Range\noexpand#1{#2}%
    \intcalcNum{#3}!\intcalcNum{#4}!%
  }%
  \BitSet@Temp
}
%    \end{macrocode}
%    \end{macro}
%    \begin{macro}{\BitSet@@Range}
%    |#1|: clear/set/flip macro\\
%    |#2|: BitSet\\
%    |#3|: Index from\\
%    |#4|: Index to
%    \begin{macrocode}
\def\BitSet@@Range#1#2#3!#4!{%
  \ifnum#3<0 %
    \BitSet@NegativeIndex#1{#2}#3!#4!0!#4!%
  \else
    \ifnum#4<0 %
      \BitSet@NegativeIndex#1{#2}#3!#4!#3!0!%
    \else
      \ifcase\intcalcCmp{#3}{#4} %
      \or
        \@PackageError{bitset}{%
          Wrong index numbers in range [#3..#4]\MessageBreak% hash-ok
          for clear/set/flip on bit set `#2'.\MessageBreak
          The lower index exceeds the upper index.\MessageBreak
          Canceling the operation as error recovery%
        }\@ehc
      \else
        \BitSet@@@Range#3!#4!#1{#2}%
      \fi
    \fi
  \fi
}
%    \end{macrocode}
%    \end{macro}
%    \begin{macro}{\BitSet@NegativeIndex}
%    \begin{macrocode}
\def\BitSet@NegativeIndex#1#2#3!#4!#5!#6!{%
  \@PackageError{bitset}{%
    Negative index in range [#3..#4]\MessageBreak % hash-ok
    for \string\bitset
    \ifx#1\BitSet@Clear
      Clear%
    \else
      \ifx#1\BitSet@Set
        Set%
      \else
        Flip%
      \fi
    \fi
    Range on bit set `#2'.\MessageBreak
    Using [#5..#6] as error recovery% hash-ok
  }\@ehc
  \BitSet@@Range#1{#2}#5!#6!%
}
%    \end{macrocode}
%    \end{macro}
%    \begin{macro}{\BitSet@@Range}
%    \begin{macrocode}
\def\BitSet@@@Range#1!#2!#3#4{%
  \ifnum#1<#2 %
    #3{#4}{#1}%
    \BitSet@AfterFi{%
      \expandafter\expandafter\expandafter\BitSet@@@Range
      \IntCalcInc#1!!#2!#3{#4}%
    }%
  \BitSet@Fi
}
%    \end{macrocode}
%    \end{macro}
%
% \subsection{Bit retrieval}\label{src:bitret}
%
% \subsubsection{\cs{bitsetGet}}
%
%    \begin{macro}{\bitsetGet}
%    \begin{macrocode}
\def\bitsetGet#1#2{%
  \number
  \expandafter\expandafter\expandafter\BitSet@Get
  \intcalcNum{#2}!{#1}%
}
%    \end{macrocode}
%    \end{macro}
%    \begin{macro}{\BitSet@Get}
%    |#1|: plain index\\
%    |#2|: BitSet
%    \begin{macrocode}
\def\BitSet@Get#1!#2{%
  \ifnum#1<0 %
    \BitSet@AfterFi{%
      0 \BitSetError:NegativeIndex%
    }%
  \else
    \BitSet@IfUndefined{#2}{0}{%
      \expandafter\expandafter\expandafter\BitSet@@Get
      \csname BS@#2\expandafter\endcsname
      \expandafter!\expandafter=%
      \expandafter{\expandafter}\expandafter{\expandafter}%
      \romannumeral\intcalcNum{#1}000!%
    }%
    \expandafter\BitSet@Space
  \BitSet@Fi
}
%    \end{macrocode}
%    \end{macro}
%    \begin{macro}{\BitSet@@Get}
%    \begin{macrocode}
\def\BitSet@@Get#1#2=#3#4#5{%
  #3#4%
  \ifx#5!%
    \ifx#1!%
      0%
    \else
      #1%
    \fi
  \else
    \ifx#1!%
      0%
      \BitSet@AfterFiFi\BitSet@Cleanup
    \else
      \BitSet@@Get#2=%
    \fi
  \BitSet@Fi
}
%    \end{macrocode}
%    \end{macro}
%
% \subsubsection{\cs{bitsetNextClearBit}, \cs{bitsetNextSetBit}}
%
%    \begin{macro}{\bitsetNextClearBit}
%    \begin{macrocode}
\def\bitsetNextClearBit#1#2{%
  \number
  \expandafter\expandafter\expandafter\BitSet@NextClearBit
  \intcalcNum{#2}!{#1} %
}
%    \end{macrocode}
%    \end{macro}
%    \begin{macro}{\BitSet@NextClearBit}
%    |#1|: Index\\
%    |#2|: BitSet
%    \begin{macrocode}
\def\BitSet@NextClearBit#1!#2{%
  \ifnum#1<0 %
    \BitSet@NextClearBit0!{#2}%
    \BitSet@AfterFi{%
      \expandafter\BitSet@Space
      \expandafter\BitSetError:NegativeIndex\romannumeral0%
    }%
  \else
    \bitsetIsEmpty{#2}{#1}{%
      \expandafter\BitSet@Skip
      \number#1\expandafter\expandafter\expandafter!%
      \csname BS@#2\endcsname!!!!!!!!!=%
      {\BitSet@@NextClearBit#1!}%
    }%
  \BitSet@Fi
}
%    \end{macrocode}
%    \end{macro}
%    \begin{macro}{\BitSet@@NextClearBit}
%    |#1|: index for next bit in |#2|\\
%    |#2|: next bit
%    \begin{macrocode}
\def\BitSet@@NextClearBit#1!#2{%
  \ifx#2!%
    #1%
  \else
    \ifx#20%
      #1%
      \BitSet@AfterFiFi\BitSet@Cleanup
    \else
      \BitSet@AfterFiFi{%
        \expandafter\expandafter\expandafter\BitSet@@NextClearBit
        \IntCalcInc#1!!%
      }%
    \fi
  \BitSet@Fi
}
%    \end{macrocode}
%    \end{macro}
%
%    \begin{macro}{\bitsetNextSetBit}
%    \begin{macrocode}
\def\bitsetNextSetBit#1#2{%
  \number
  \expandafter\expandafter\expandafter\BitSet@NextSetBit
  \intcalcNum{#2}!{#1} %
}
%    \end{macrocode}
%    \end{macro}
%    \begin{macro}{\BitSet@NextSetBit}
%    |#1|: Index\\
%    |#2|: BitSet
%    \begin{macrocode}
\def\BitSet@NextSetBit#1!#2{%
  \ifnum#1<0 %
    \BitSet@NextSetBit0!{#2}%
    \BitSet@AfterFi{%
      \expandafter\BitSet@Space
      \expandafter\BitSetError:NegativeIndex\romannumeral0%
    }%
  \else
    \bitsetIsEmpty{#2}{-1}{%
      \expandafter\BitSet@Skip
      \number#1\expandafter\expandafter\expandafter!%
      \csname BS@#2\endcsname!!!!!!!!!=%
      {\BitSet@@NextSetBit#1!}%
    }%
  \BitSet@Fi
}
%    \end{macrocode}
%    \end{macro}
%    \begin{macro}{\BitSet@@NextSetBit}
%    |#1|: index for next bit in |#2|\\
%    |#2|: next bit
%    \begin{macrocode}
\def\BitSet@@NextSetBit#1!#2{%
  \ifx#2!%
    -1%
  \else
    \ifx#21%
      #1%
      \BitSet@AfterFiFi\BitSet@Cleanup
    \else
      \BitSet@AfterFiFi{%
        \expandafter\expandafter\expandafter\BitSet@@NextSetBit
        \IntCalcInc#1!!%
      }%
    \fi
  \BitSet@Fi
}
%    \end{macrocode}
%    \end{macro}
%    \begin{macro}{\BitSet@Cleanup}
%    \begin{macrocode}
\def\BitSet@Cleanup#1!{}
%    \end{macrocode}
%    \end{macro}
%
%    \begin{macro}{\BitSet@Skip}
%    |#1|: number of bits to skip\\
%    |#2|: bits\\
%    |#3|: continuation code
%    \begin{macrocode}
\def\BitSet@Skip#1!#2{%
  \ifx#2!%
    \BitSet@AfterFi{%
      \BitSet@SkipContinue%
    }%
  \else
    \ifcase#1 %
      \BitSet@AfterFiFi{%
        \BitSet@SkipContinue#2%
      }%
    \or
      \BitSet@AfterFiFi\BitSet@SkipContinue
    \or
      \BitSet@AfterFiFi{%
        \expandafter\BitSet@SkipContinue\BitSet@Gobble
      }%
    \else
      \ifnum#1>8 %
        \BitSet@AfterFiFiFi{%
          \expandafter\BitSet@Skip
          \number\IntCalcSub#1!8!\expandafter!%
          \BitSet@GobbleSeven
        }%
      \else
        \BitSet@AfterFiFiFi{%
          \expandafter\expandafter\expandafter\BitSet@Skip
          \IntCalcDec#1!!%
        }%
      \fi
    \fi
  \BitSet@Fi
}
%    \end{macrocode}
%    \end{macro}
%    \begin{macro}{\BitSet@SkipContinue}
%    |#1|: remaining bits\\
%    |#2|: continuation code
%    \begin{macrocode}
\def\BitSet@SkipContinue#1!#2=#3{%
  #3#1!%
}
%    \end{macrocode}
%    \end{macro}
%    \begin{macro}{\BitSet@GobbleSeven}
%    \begin{macrocode}
\def\BitSet@GobbleSeven#1#2#3#4#5#6#7{}
%    \end{macrocode}
%    \end{macro}
%
% \subsubsection{\cs{bitsetGetSetBitList}}
%
%    \begin{macro}{\bitsetGetSetBitList}
%    It's just a wrapper for \cs{bitsetNextSetBit}.
%    \begin{macrocode}
\def\bitsetGetSetBitList#1{%
  \romannumeral0%
  \bitsetIsEmpty{#1}{ }{%
    \expandafter\BitSet@GetSetBitList
    \number\BitSet@NextSetBit0!{#1}!{#1}{}!%
  }%
}
%    \end{macrocode}
%    \end{macro}
%    \begin{macro}{\BitSet@GetSetBitList}
%    |#1|: found index\\
%    |#2|: BitSet\\
%    |#3|: comma
%    |#4|: result
%    \begin{macrocode}
\def\BitSet@GetSetBitList#1!#2#3#4!{%
  \ifnum#1<0 %
    \BitSet@AfterFi{ #4}%
  \else
    \BitSet@AfterFi{%
      \expandafter\BitSet@GetSetBitList\number
      \expandafter\expandafter\expandafter\BitSet@NextSetBit
      \IntCalcInc#1!!{#2}!{#2},#4#3#1!%
    }%
  \BitSet@Fi
}
%    \end{macrocode}
%    \end{macro}
%
% \subsection{Bit set properties}\label{src:prop}
%
%    \begin{macro}{\bitsetSize}
%    \begin{macrocode}
\def\bitsetSize#1{%
  \number
  \BitSet@IfUndefined{#1}{0 }{%
    \expandafter\expandafter\expandafter\BitSet@Size
    \expandafter\expandafter\expandafter1%
    \expandafter\expandafter\expandafter!%
    \csname BS@#1\endcsname!0!%
  }%
}
%    \end{macrocode}
%    \end{macro}
%    \begin{macro}{\BitSet@Size}
%    |#1|: counter\\
%    |#2#3|: bits\\
%    |#4|: result
%    \begin{macrocode}
\def\BitSet@Size#1!#2#3!#4!{%
  \ifx#21%
    \ifx\\#3\\%
      \BitSet@AfterFiFi{#1 }%
    \else
      \BitSet@AfterFiFi{%
        \expandafter\expandafter\expandafter\BitSet@Size
        \IntCalcInc#1!!#3!#1!%
      }%
    \fi
  \else
    \ifx\\#3\\%
      \BitSet@AfterFiFi{#4 }%
    \else
      \BitSet@AfterFiFi{%
        \expandafter\expandafter\expandafter\BitSet@Size
        \IntCalcInc#1!!#3!#4!%
      }%
    \fi
  \fi
  \BitSet@Fi
}
%    \end{macrocode}
%    \end{macro}
%
%    \begin{macro}{\bitsetCardinality}
%    \begin{macrocode}
\def\bitsetCardinality#1{%
  \number
  \BitSet@IfUndefined{#1}{0 }{%
    \expandafter\expandafter\expandafter\BitSet@Cardinality
    \expandafter\expandafter\expandafter0%
    \expandafter\expandafter\expandafter!%
    \csname BS@#1\endcsname!%
  }%
}
%    \end{macrocode}
%    \end{macro}
%    \begin{macro}{\BitSet@Cardinality}
%    |#1|: result\\
%    |#2#3|: bits
%    \begin{macrocode}
\def\BitSet@Cardinality#1!#2#3!{%
  \ifx#21%
    \ifx\\#3\\%
      \BitSet@AfterFiFi{\IntCalcInc#1! }%
    \else
      \BitSet@AfterFiFi{%
        \expandafter\expandafter\expandafter\BitSet@Cardinality
        \IntCalcInc#1!!#3!%
      }%
    \fi
  \else
    \ifx\\#3\\%
      \BitSet@AfterFiFi{#1 }%
    \else
      \BitSet@AfterFiFi{%
        \BitSet@Cardinality#1!#3!%
      }%
    \fi
  \fi
  \BitSet@Fi
}
%    \end{macrocode}
%    \end{macro}
%
% \subsection{Queries}\label{src:query}
%
%    \begin{macro}{\bitsetIsDefined}
%    \begin{macrocode}
\def\bitsetIsDefined#1{%
  \BitSet@IfUndefined{#1}%
  \BitSet@SecondOfTwo
  \BitSet@FirstOfTwo
}
%    \end{macrocode}
%    \end{macro}
%    \begin{macro}{\bitsetIsEmpty}
%    \begin{macrocode}
\def\bitsetIsEmpty#1{%
  \BitSet@IfUndefined{#1}\BitSet@FirstOfTwo{%
    \expandafter\ifx\csname BS@#1\endcsname\BitSet@Zero
      \expandafter\BitSet@FirstOfTwo
    \else
      \expandafter\BitSet@SecondOfTwo
    \fi
  }%
}
%    \end{macrocode}
%    \end{macro}
%    \begin{macro}{\BitSet@Zero}
%    \begin{macrocode}
\def\BitSet@Zero{0}
%    \end{macrocode}
%    \end{macro}
%
%    \begin{macro}{\bitsetQuery}
%    \begin{macrocode}
\def\bitsetQuery#1#2{%
  \ifnum\bitsetGet{#1}{#2}=1 %
    \expandafter\BitSet@FirstOfTwo
  \else
    \expandafter\BitSet@SecondOfTwo
  \fi
}
%    \end{macrocode}
%    \end{macro}
%
%    \begin{macro}{\bitsetEquals}
%    \begin{macrocode}
\def\bitsetEquals#1#2{%
  \BitSet@IfUndefined{#1}{%
    \BitSet@IfUndefined{#2}\BitSet@FirstOfTwo\BitSet@SecondOfTwo
  }{%
    \BitSet@IfUndefined{#2}\BitSet@SecondOfTwo{%
      \expandafter\ifx\csname BS@#1\expandafter\endcsname
                      \csname BS@#2\endcsname
        \expandafter\BitSet@FirstOfTwo
      \else
        \expandafter\BitSet@SecondOfTwo
      \fi
    }%
  }%
}
%    \end{macrocode}
%    \end{macro}
%
%    \begin{macro}{\bitsetIntersects}
%    \begin{macrocode}
\def\bitsetIntersects#1#2{%
  \bitsetIsEmpty{#1}\BitSet@SecondOfTwo{%
    \bitsetIsEmpty{#2}\BitSet@SecondOfTwo{%
      \expandafter\expandafter\expandafter\BitSet@Intersects
      \csname BS@#1\expandafter\expandafter\expandafter\endcsname
      \expandafter\expandafter\expandafter!%
      \csname BS@#2\endcsname!%
    }%
  }%
}
%    \end{macrocode}
%    \end{macro}
%    \begin{macro}{\BitSet@Intersects}
%    \begin{macrocode}
\def\BitSet@Intersects#1#2!#3#4!{%
  \ifnum#1#3=11 %
    \BitSet@AfterFi\BitSet@FirstOfTwo
  \else
    \ifx\\#2\\%
      \BitSet@AfterFiFi\BitSet@SecondOfTwo
    \else
      \ifx\\#4\\%
        \BitSet@AfterFiFiFi\BitSet@SecondOfTwo
      \else
        \BitSet@AfterFiFiFi{%
          \BitSet@Intersects#2!#4!%
        }%
      \fi
    \fi
  \BitSet@Fi
}
%    \end{macrocode}
%    \end{macro}
%
%    \begin{macrocode}
\BitSet@AtEnd%
%</package>
%    \end{macrocode}
%
% \section{Test}
%
% \subsection{Catcode checks for loading}
%
%    \begin{macrocode}
%<*test1>
%    \end{macrocode}
%    \begin{macrocode}
\catcode`\{=1 %
\catcode`\}=2 %
\catcode`\#=6 %
\catcode`\@=11 %
\expandafter\ifx\csname count@\endcsname\relax
  \countdef\count@=255 %
\fi
\expandafter\ifx\csname @gobble\endcsname\relax
  \long\def\@gobble#1{}%
\fi
\expandafter\ifx\csname @firstofone\endcsname\relax
  \long\def\@firstofone#1{#1}%
\fi
\expandafter\ifx\csname loop\endcsname\relax
  \expandafter\@firstofone
\else
  \expandafter\@gobble
\fi
{%
  \def\loop#1\repeat{%
    \def\body{#1}%
    \iterate
  }%
  \def\iterate{%
    \body
      \let\next\iterate
    \else
      \let\next\relax
    \fi
    \next
  }%
  \let\repeat=\fi
}%
\def\RestoreCatcodes{}
\count@=0 %
\loop
  \edef\RestoreCatcodes{%
    \RestoreCatcodes
    \catcode\the\count@=\the\catcode\count@\relax
  }%
\ifnum\count@<255 %
  \advance\count@ 1 %
\repeat

\def\RangeCatcodeInvalid#1#2{%
  \count@=#1\relax
  \loop
    \catcode\count@=15 %
  \ifnum\count@<#2\relax
    \advance\count@ 1 %
  \repeat
}
\def\RangeCatcodeCheck#1#2#3{%
  \count@=#1\relax
  \loop
    \ifnum#3=\catcode\count@
    \else
      \errmessage{%
        Character \the\count@\space
        with wrong catcode \the\catcode\count@\space
        instead of \number#3%
      }%
    \fi
  \ifnum\count@<#2\relax
    \advance\count@ 1 %
  \repeat
}
\def\space{ }
\expandafter\ifx\csname LoadCommand\endcsname\relax
  \def\LoadCommand{\input bitset.sty\relax}%
\fi
\def\Test{%
  \RangeCatcodeInvalid{0}{47}%
  \RangeCatcodeInvalid{58}{64}%
  \RangeCatcodeInvalid{91}{96}%
  \RangeCatcodeInvalid{123}{255}%
  \catcode`\@=12 %
  \catcode`\\=0 %
  \catcode`\%=14 %
  \LoadCommand
  \RangeCatcodeCheck{0}{36}{15}%
  \RangeCatcodeCheck{37}{37}{14}%
  \RangeCatcodeCheck{38}{47}{15}%
  \RangeCatcodeCheck{48}{57}{12}%
  \RangeCatcodeCheck{58}{63}{15}%
  \RangeCatcodeCheck{64}{64}{12}%
  \RangeCatcodeCheck{65}{90}{11}%
  \RangeCatcodeCheck{91}{91}{15}%
  \RangeCatcodeCheck{92}{92}{0}%
  \RangeCatcodeCheck{93}{96}{15}%
  \RangeCatcodeCheck{97}{122}{11}%
  \RangeCatcodeCheck{123}{255}{15}%
  \RestoreCatcodes
}
\Test
\csname @@end\endcsname
\end
%    \end{macrocode}
%    \begin{macrocode}
%</test1>
%    \end{macrocode}
%
% \subsection{Macro tests}
%
% \subsubsection{Preamble}
%
%    \begin{macrocode}
%<*test2>
\NeedsTeXFormat{LaTeX2e}
\nofiles
\documentclass{article}
\makeatletter
%<*noetex>
\let\SavedNumexpr\numexpr
\let\SavedIfcsname\ifcsname
\let\SavedCurrentgrouplevel\currentgrouplevel
\def\ETeXDisable{%
  \let\ifcsname\@undefined
  \let\numexpr\@undefined
  \let\currentgrouplevel\@undefined
}
\ETeXDisable
%</noetex>
\makeatletter
\chardef\BitSet@TestMode=1 %
\makeatother
\usepackage{bitset}[2011/01/30]
%<*noetex>
\def\ETeXEnable{%
  \let\numexpr\SavedNumexpr
  \let\ifcsname\SavedIfcsname
  \let\currentgrouplevel\SavedCurrentgrouplevel
}
\ETeXEnable
%</noetex>
\usepackage{qstest}
\IncludeTests{*}
\LogTests{log}{*}{*}
\makeatletter
%    \end{macrocode}
%
% \subsubsection{Time}
%
%    \begin{macrocode}
\begingroup\expandafter\expandafter\expandafter\endgroup
\expandafter\ifx\csname pdfresettimer\endcsname\relax
\else
  \newcount\SummaryTime
  \newcount\TestTime
  \SummaryTime=\z@
  \newcommand*{\PrintTime}[2]{%
    \typeout{%
      [Time #1: \strip@pt\dimexpr\number#2sp\relax\space s]%
    }%
  }%
  \newcommand*{\StartTime}[1]{%
    \renewcommand*{\TimeDescription}{#1}%
    \pdfresettimer
  }%
  \newcommand*{\TimeDescription}{}%
  \newcommand*{\StopTime}{%
    \TestTime=\pdfelapsedtime
    \global\advance\SummaryTime\TestTime
    \PrintTime\TimeDescription\TestTime
  }%
  \let\saved@qstest\qstest
  \let\saved@endqstest\endqstest
  \def\qstest#1#2{%
    \saved@qstest{#1}{#2}%
    \StartTime{#1}%
  }%
  \def\endqstest{%
    \StopTime
    \saved@endqstest
  }%
  \AtEndDocument{%
    \PrintTime{summary}\SummaryTime
  }%
\fi
%    \end{macrocode}
%
% \subsubsection{Detection of unwanted space}
%
%    \begin{macrocode}
\let\orig@qstest\qstest
\let\orig@endqstest\endqstest
\def\qstest#1#2{%
  \orig@qstest{#1}{#2}%
  \setbox0\hbox\bgroup\begingroup\ignorespaces
}
\def\endqstest{%
  \endgroup\egroup
  \Expect*{\the\wd0}{0.0pt}%
  \orig@endqstest
}
%    \end{macrocode}
%
% \subsubsection{Test macros}
%
%    \begin{macrocode}
\newcounter{Test}

\def\TestError#1#2{%
  \begingroup
    \setcounter{Test}{0}%
    \sbox0{%
      \def\@PackageError##1##2##3{%
        \stepcounter{Test}%
        \begingroup
          \let\MessageBreak\relax
%<*noetex>
          \ETeXEnable
%</noetex>
          \Expect{##1}{bitset}%
          \Expect*{##2}*{#1}%
        \endgroup
      }%
%<*noetex>
      \ETeXDisable
%</noetex>
      #2%
    }%
    \Expect*{\theTest}{1}%
    \Expect*{\the\wd0}{0.0pt}%
  \endgroup
}

\def\TestErrorNegativeIndex#1#2{%
  \TestError{Invalid negative index (#1)}{#2}%
}

\def\TestGetterUndefined#1{%
  \CheckUndef{dummy}%
  \expandafter\expandafter\expandafter\Expect
  \expandafter\expandafter\expandafter{#1{dummy}}{0}%
}

\def\ExpectBitSet#1#2{%
  \expandafter\expandafter\expandafter\Expect
  \expandafter\expandafter\expandafter
  {\csname BS@#1\endcsname}*{#2}%
}
\def\Check#1#2{%
  \ExpectBitSet{#1}{#2}%
}
\def\CheckUndef#1{%
  \begingroup
    \Expect*{%
      \expandafter
      \ifx\csname BS@#1\endcsname\relax true\else false\fi
    }{true}%
  \endgroup
}
\def\RevCheck#1#2{%
  \ExpectBitSet{#1}{\Reverse#2!!}%
}
\def\Set#1#2{%
  \expandafter\def\csname BS@#1\endcsname{#2}%
}
\def\RevSet#1#2{%
  \expandafter\edef\csname BS@#1\endcsname{%
    \Reverse#2!!%
  }%
}
\def\Reverse#1#2!#3!{%
  \ifx\\#2\\%
    #1#3%
    \expandafter\@gobble
  \else
    \expandafter\@firstofone
  \fi
  {\Reverse#2!#1#3!}%
}
%    \end{macrocode}
%
% \subsubsection{Test sets}
%
%    \begin{macrocode}
\begin{qstest}{Let}{Let}
  \CheckUndef{abc}%
  \CheckUndef{xyz}%
  \bitsetLet{xyz}{abc}%
  \CheckUndef{abc}%
  \Check{xyz}{0}%
  \Set{abc}{1}%
  \Check{abc}{1}%
  \Check{xyz}{0}%
  \bitsetLet{xyz}{abc}%
  \Check{abc}{1}%
  \Check{xyz}{1}%
  \Set{xyz}{11}%
  \Check{abc}{1}%
  \Check{xyz}{11}%
\end{qstest}

\begin{qstest}{Reset}{Reset}
  \bitsetReset{xyz}%
  \Check{xyz}{0}%
  \bitsetReset{abc}%
  \Check{abc}{0}%
  \Set{abc}{10101}%
  \bitsetReset{abc}%
  \Check{abc}{0}%
\end{qstest}

\begin{qstest}{Get/Query}{Get/Query}
  \expandafter\expandafter\expandafter\Expect
  \expandafter\expandafter\expandafter{%
    \bitsetGet{dummy}{0}%
  }{0}%
  \begingroup
    \expandafter\def\csname BitSetError:NegativeIndex\endcsname{}%
    \Set{abc}{1}%
    \Expect*{\bitsetQuery{abc}{-1}{true}{false}}{false}%
  \endgroup
  \def\Test#1#2#3{%
    \Set{abc}{#1}%
    \expandafter\expandafter\expandafter\Expect
    \expandafter\expandafter\expandafter{\bitsetGet{abc}{#2}}{#3}%
    \Expect*{\bitsetQuery{abc}{#2}{true}{false}}%
           *{\ifcase#3 false\or true\else error\fi}%
  }%
  \Test{1}{100}{0}%
  \Test{0}{0}{0}%
  \Test{1}{0}{1}%
  \Test{11}{1}{1}%
  \Test{111}{1}{1}%
  \Test{101}{1}{0}%
  \Test{101}{2}{1}%
  \Test{10100110011}{10}{1}%
\end{qstest}

\begin{qstest}{Size}{Size}
  \TestGetterUndefined\bitsetSize
  \def\Test#1#2{%
    \Set{abc}{#1}%
    \expandafter\expandafter\expandafter\Expect
    \expandafter\expandafter\expandafter{\bitsetSize{abc}}{#2}%
  }%
  \Test{0}{0}%
  \Test{1}{1}%
  \Test{00}{0}%
  \Test{0000000}{0}%
  \Test{10}{1}%
  \Test{01}{2}%
  \Test{11}{2}%
  \Test{010}{2}%
  \Test{011}{3}%
  \Test{100110011}{9}%
  \Test{0000011111000001111100000}{20}%
  \Test{000000000000000000000000011111111111111111111}{45}%
\end{qstest}

\begin{qstest}{Cardinality}{Cardinality}
  \TestGetterUndefined\bitsetCardinality
  \def\Test#1#2{%
    \Set{abc}{#1}%
    \expandafter\expandafter\expandafter\Expect
    \expandafter\expandafter\expandafter{%
      \bitsetCardinality{abc}%
    }{#2}%
  }%
  \Test{0}{0}%
  \Test{1}{1}%
  \Test{00}{0}%
  \Test{0000000}{0}%
  \Test{10}{1}%
  \Test{01}{1}%
  \Test{11}{2}%
  \Test{010}{1}%
  \Test{011}{2}%
  \Test{100110011}{5}%
  \Test{0000011111000001111100000}{10}%
  \Test{000000000000000000000000011111111111111111111}{20}%
\end{qstest}

\begin{qstest}{NextClearBit/NextSetBit}{NextClearBit/NextSetBit}
  \def\Test#1#2{%
    \expandafter\expandafter\expandafter\Expect
    \expandafter\expandafter\expandafter{%
      \TestOp{abc}{#1}%
    }{#2}%
  }%
  \def\Clear{\let\TestOp\bitsetNextClearBit}%
  \def\Set{\let\TestOp\bitsetNextSetBit}%
  \begingroup
    \catcode`\:=11 %
    \bitsetSetBin{abc}{1}%
    \Clear
    \Test{-1}{1\BitSetError:NegativeIndex}%
    \Set
    \Test{-1}{0\BitSetError:NegativeIndex}%
  \endgroup
  \let\BS@abc\@undefined
  \Clear
  \Test{0}{0}%
  \Test{1}{1}%
  \Test{2}{2}%
  \Test{100}{100}%
  \Set
  \Test{0}{-1}%
  \Test{1}{-1}%
  \Test{100}{-1}%
  \bitsetReset{abc}%
  \Clear
  \Test{0}{0}%
  \Test{1}{1}%
  \Test{2}{2}%
  \Test{100}{100}%
  \Set
  \Test{0}{-1}%
  \Test{1}{-1}%
  \Test{100}{-1}%
  \bitsetSetBin{abc}{1}%
  \Clear
  \Test{0}{1}%
  \Test{1}{1}%
  \Test{2}{2}%
  \Test{100}{100}%
  \Set
  \Test{0}{0}%
  \Test{1}{-1}%
  \Test{100}{-1}%
  \bitsetSetBin{abc}{111000111000111000111}%
  \Clear
  \Test{0}{3}%
  \Test{1}{3}%
  \Test{2}{3}%
  \Test{3}{3}%
  \Test{4}{4}%
  \Test{5}{5}%
  \Test{6}{9}%
  \Test{7}{9}%
  \Test{8}{9}%
  \Test{9}{9}%
  \Test{10}{10}%
  \Test{11}{11}%
  \Test{12}{15}%
  \Test{13}{15}%
  \Test{14}{15}%
  \Test{15}{15}%
  \Test{16}{16}%
  \Test{17}{17}%
  \Test{18}{21}%
  \Test{19}{21}%
  \Test{20}{21}%
  \Test{21}{21}%
  \Test{22}{22}%
  \Test{100}{100}%
  \Set
  \Test{0}{0}%
  \Test{1}{1}%
  \Test{2}{2}%
  \Test{3}{6}%
  \Test{4}{6}%
  \Test{5}{6}%
  \Test{6}{6}%
  \Test{7}{7}%
  \Test{8}{8}%
  \Test{9}{12}%
  \Test{10}{12}%
  \Test{11}{12}%
  \Test{12}{12}%
  \Test{13}{13}%
  \Test{14}{14}%
  \Test{15}{18}%
  \Test{16}{18}%
  \Test{17}{18}%
  \Test{18}{18}%
  \Test{19}{19}%
  \Test{20}{20}%
  \Test{21}{-1}%
  \Test{22}{-1}%
  \Test{100}{-1}%
  \bitsetSetBin{abc}{1111111}%
  \Clear
  \Test{6}{7}%
  \Test{7}{7}%
  \Test{8}{8}%
  \Test{100}{100}%
  \Set
  \Test{6}{6}%
  \Test{7}{-1}%
  \Test{8}{-1}%
  \Test{100}{-1}%
  \bitsetSetBin{abc}{11111111}%
  \Clear
  \Test{7}{8}%
  \Test{8}{8}%
  \Test{9}{9}%
  \Test{100}{100}%
  \Set
  \Test{7}{7}%
  \Test{8}{-1}%
  \Test{9}{-1}%
  \Test{100}{-1}%
  \bitsetSetBin{abc}{111111111}%
  \Clear
  \Test{8}{9}%
  \Test{9}{9}%
  \Test{10}{10}%
  \Test{100}{100}%
  \Set
  \Test{8}{8}%
  \Test{9}{-1}%
  \Test{10}{-1}%
  \Test{100}{-1}%
  \bitsetSetBin{abc}{1111111111}%
  \Clear
  \Test{9}{10}%
  \Test{10}{10}%
  \Test{11}{11}%
  \Test{100}{100}%
  \Set
  \Test{9}{9}%
  \Test{10}{-1}%
  \Test{11}{-1}%
  \Test{100}{-1}%
\end{qstest}

\begin{qstest}{GetSetBitList}{GetSetBitList}
  \let\BS@abc\@undefined
  \expandafter\expandafter\expandafter\Expect
  \expandafter\expandafter\expandafter{%
    \bitsetGetSetBitList{abc}%
  }{}%
  \def\Test#1#2{%
    \bitsetSetBin{abc}{#1}%
    \expandafter\expandafter\expandafter\Expect
    \expandafter\expandafter\expandafter{%
      \bitsetGetSetBitList{abc}%
    }{#2}%
  }%
  \Test{0}{}%
  \Test{1}{0}%
  \Test{10}{1}%
  \Test{11}{0,1}%
  \Test{10110100}{2,4,5,7}%
  \Test{101101001010011}{0,1,4,6,9,11,12,14}%
\end{qstest}

\begin{qstest}{GetDec}{GetDec}
  \TestGetterUndefined\bitsetGetDec
  \def\Test#1#2{%
    \RevSet{abc}{#1}%
%<*noetex>
    \begingroup\expandafter\expandafter\expandafter\endgroup
%</noetex>
    \expandafter\expandafter\expandafter\Expect
    \expandafter\expandafter\expandafter{%
      \bitsetGetDec{abc}%
    }{#2}%
  }%
  \Test{0}{0}%
  \Test{1}{1}%
  \Test{10}{2}%
  \Test{11}{3}%
  \Test{100}{4}%
  \Test{101}{5}%
  \Test{110}{6}%
  \Test{111}{7}%
  \Test{1000}{8}%
  \Test{000111}{7}%
  \Test{1111111111111111%
        111111111111111}{2147483647}%
  \Test{0001111111111111111%
        111111111111111}{2147483647}%
  \Test{1000000000000000%
        0000000000000000}{2147483648}%
  \Test{1000000000000000%
        00000000000000000}{4294967296}%
  \Test{0001000000000000000%
        00000000000000000}{4294967296}%
  \Test{1100000000000000%
        00000000000000011}{6442450947}%
\end{qstest}

\begin{qstest}{Clear}{Clear}
  \def\Test#1#2#3{%
    \RevSet{abc}{#1}%
    \bitsetClear{abc}{#2}%
    \Expect*{\BS@abc}*{\Reverse#3!!}%
  }%
  \bitsetClear{abc}{2}%
  \RevCheck{abc}{0}%
  \TestErrorNegativeIndex{-1}{\bitsetClear{abc}{-1}}%
  \RevCheck{abc}{0}%
  \Test{0}{0}{0}%
  \Test{1}{0}{0}%
  \Test{111}{1}{101}%
  \Test{111}{30}{111}%
  \Test{0000111}{5}{0000111}% 111 would also be ok
  \Test{10000111}{5}{10000111}%
  \Test{1001001}{3}{1000001}%
\end{qstest}

\begin{qstest}{Set}{Set}
  \def\Test#1#2#3{%
    \RevSet{abc}{#1}%
    \bitsetSet{abc}{#2}%
    \Expect*{\BS@abc}*{\Reverse#3!!}%
  }%
  \bitsetSet{abc}{2}%
  \RevCheck{abc}{100}%
  \TestErrorNegativeIndex{-1}{\bitsetSet{abc}{-1}}%
  \RevCheck{abc}{100}%
  \Test{0}{0}{1}%
  \Test{1}{0}{1}%
  \Test{100}{1}{110}%
  \Test{111}{1}{111}%
  \Test{11}{1}{11}%
  \Test{11}{2}{111}%
  \Test{11}{3}{1011}%
  \Test{111}{10}{10000000111}%
  \Test{0000111}{5}{0100111}% 100111 would also be ok
  \Test{10000111}{5}{10100111}%
  \Test{1000001}{3}{1001001}%
  \Test{1001001}{3}{1001001}%
\end{qstest}

\begin{qstest}{Flip}{Flip}
  \def\Test#1#2#3{%
    \RevSet{abc}{#1}%
    \bitsetFlip{abc}{#2}%
    \Expect*{\BS@abc}*{\Reverse#3!!}%
  }%
  \bitsetFlip{abc}{2}%
  \RevCheck{abc}{100}%
  \TestErrorNegativeIndex{-1}{\bitsetFlip{abc}{-1}}%
  \RevCheck{abc}{100}%
  \Test{0}{0}{1}%
  \Test{1}{0}{0}%
  \Test{0}{2}{100}%
  \Test{100}{1}{110}%
  \Test{111}{1}{101}%
  \Test{11}{1}{1}%
  \Test{11}{2}{111}%
  \Test{11}{3}{1011}%
  \Test{111}{10}{10000000111}%
  \Test{0000111}{5}{0100111}% 100111 would also be ok
  \Test{10000111}{5}{10100111}%
  \Test{1000001}{3}{1001001}%
  \Test{1001001}{3}{1000001}%
  \Test{11111}{2}{11011}%
\end{qstest}

\begin{qstest}{SetValue}{SetValue}
  \def\Test#1#2{%
    \TestError{Invalid bit value (#2) not in range 0..1}{%
      \bitsetSetValue{abc}{#1}{#2}%
    }%
  }%
  \Test{0}{-1}%
  \Test{0}{2}%
  \Test{0}{10}%
  \def\Test#1#2#3{%
    \let\BS@abc\@undefined
    \bitsetSetValue{abc}{#1}{#2}%
    \bitsetSetBin{result}{#3}%
    \Expect*{\BS@abc}*{\BS@result}%
  }%
  \Test{0}{0}{0}%
  \Test{0}{1}{1}%
  \Test{1}{0}{0}%
  \Test{1}{1}{10}%
  \def\Test#1#2#3#4{%
    \bitsetSetBin{abc}{#1}%
    \bitsetSetBin{result}{#4}%
    \bitsetSetValue{abc}{#2}{#3}%
    \Expect*{\BS@abc}*{\BS@result}%
  }%
  \Test{0}{0}{0}{0}%
  \Test{0}{0}{0}{0}%
  \Test{0}{0}{1}{1}%
  \Test{0}{1}{0}{0}%
  \Test{0}{1}{1}{10}%
  \Test{1010}{2}{1}{1110}%
  \Test{1010}{4}{1}{11010}%
  \Test{1010}{6}{1}{1001010}%
  \Test{1010}{1}{0}{1000}%
  \Test{1010}{2}{0}{1010}%
  \Test{1010}{3}{0}{10}%
  \Test{1010}{4}{0}{1010}%
  \Test{1010}{6}{0}{1010}%
  \Test{1010}{2}{\csname iffalse\endcsname 0\else 1\fi}{1110}%
  \Test{1010}{1}{\csname iffalse\endcsname 1\else 0\fi}{1000}%
\end{qstest}

\begin{qstest}{IsDefined}{IsDefined}
  \let\BS@abc\@undefined
  \Expect*{\bitsetIsDefined{abc}{true}{false}}{false}%
  \bitsetReset{abc}%
  \Expect*{\bitsetIsDefined{abc}{true}{false}}{true}%
\end{qstest}

\begin{qstest}{IsEmpty}{IsEmpty}
  \let\BS@abc\@undefined
  \Expect*{\bitsetIsEmpty{abc}{true}{false}}{true}%
  \bitsetReset{abc}%
  \Expect*{\bitsetIsEmpty{abc}{true}{false}}{true}%
  \bitsetSet{abc}{1}%
  \Expect*{\bitsetIsEmpty{abc}{true}{false}}{false}%
\end{qstest}

\begin{qstest}{Equals}{Equals}
  \def\Test#1#2#3{%
    \Expect*{\bitsetEquals{#1}{#2}{true}{false}}{#3}%
  }%
  \let\BS@abc\@undefined
  \Test{abc}{abc}{true}%
  \Test{abc}{foo}{true}%
  \Test{foo}{abc}{true}%
  \bitsetReset{abc}%
  \Test{abc}{abc}{true}%
  \Test{abc}{foo}{false}%
  \Test{foo}{abc}{false}%
  \bitsetReset{foo}%
  \Test{abc}{foo}{true}%
  \Test{foo}{abc}{true}%
  \bitsetSet{abc}{4}%
  \Test{abc}{foo}{false}%
  \Test{foo}{abc}{false}%
  \bitsetFlip{foo}{4}%
  \Test{abc}{foo}{true}%
  \Test{foo}{abc}{true}%
\end{qstest}

\begin{qstest}{Intersects}{Intersects}
  \def\Test#1{%
    \Expect*{\bitsetIntersects{abc}{foo}{true}{false}}{#1}%
  }%
  \let\BS@abc\@undefined
  \let\BS@foo\@undefined
  \Test{false}%
  \Set{abc}{0}%
  \Test{false}%
  \Set{foo}{0}%
  \Test{false}%
  \let\BS@abc\@undefined
  \Test{false}%
  \Set{foo}{1}%
  \Test{false}%
  \Set{abc}{0}%
  \Test{false}%
  \Set{abc}{1}%
  \Test{true}%
  \let\BS@foo\@undefined
  \Test{false}%
  \Set{foo}{0}%
  \Test{false}%
  \def\Test#1#2#3{%
    \bitsetSetBin{abc}{#1}%
    \bitsetSetBin{foo}{#2}%
    \Expect*{\bitsetIntersects{abc}{foo}{true}{false}}{#3}%
  }%
  \Test{1010}{0101}{false}%
  \Test{0}{10}{false}%
  \Test{1}{11}{true}%
  \Test{11}{1}{true}%
  \Test{10}{1}{false}%
\end{qstest}

\begin{qstest}{And/AndNot/Or/Xor}{And/AndNot/Or/Xor}
  \def\@Test#1#2#3#4#5{%
    \begingroup
      #5%
      \begingroup
        \let\BS@foo\@undefined
        \csname bitset#1\endcsname{abc}{foo}%
        \CheckUndef{foo}%
        \Check{abc}{#2}%
      \endgroup
      \begingroup
        \bitsetReset{foo}%
        \csname bitset#1\endcsname{abc}{foo}%
        \Check{foo}{0}%
        \Check{abc}{#3}%
      \endgroup
      \begingroup
        \def\BS@foo{0101}%
        \csname bitset#1\endcsname{abc}{foo}%
        \Check{foo}{0101}%
        \Check{abc}{#4}%
      \endgroup
    \endgroup
  }%
  \def\Test#1{%
    \def\Op{#1}%
    \Test@
  }%
  \def\Test@#1#2#3#4#5#6#7#8#9{%
    \@Test\Op{#1}{#2}{#3}{%
      \let\BS@abc\@undefined
    }%
    \@Test\Op{#4}{#5}{#6}{%
      \bitsetReset{abc}%
    }%
    \@Test\Op{#7}{#8}{#9}{%
      \def\BS@abc{1001}%
    }%
  }%
  \Test{And}%
       {0}{0}{0}%
       {0}{0}{0}%
       {0}{0}{0001}%
  \Test{AndNot}%
       {0}{0}{0}%
       {0}{0}{0}%
       {1001}{1001}{1}%
  \Test{Or}%
       {0}{0}{0101}%
       {0}{0}{0101}%
       {1001}{1001}{1101}%
  \Test{Xor}%
       {0}{0}{0101}%
       {0}{0}{0101}%
       {1001}{1001}{11}%
  \def\Test#1#2#3{%
    \bitsetSetBin{abc}{#1}%
    \bitsetSetBin{foo}{#2}%
    \csname bitset\Op\endcsname{abc}{foo}%
    \RevCheck{foo}{#2}%
    \RevCheck{abc}{#3}%
  }%
  \def\Op{And}%
  \Test{1}{111}{1}%
  \Test{111}{1}{1}%
  \Test{10}{111}{10}%
  \Test{111}{10}{10}%
  \Test{111}{1000}{0}%
  \Test{1000}{111}{0}%
  \def\Op{AndNot}%
  \Test{1010}{11}{1000}%
  \Test{100}{100}{0}%
  \Test{111}{1111}{0}%
  \Test{100}{111}{0}%
  \def\Op{Or}%
  \Test{0}{0}{0}%
  \Test{1}{0}{1}%
  \Test{0}{1}{1}%
  \Test{1}{1}{1}%
  \Test{1000}{10}{1010}%
  \Test{10}{1000}{1010}%
  \def\Op{Xor}%
  \Test{0}{0}{0}%
  \Test{1}{0}{1}%
  \Test{0}{1}{1}%
  \Test{1}{1}{0}%
  \Test{1000}{10}{1010}%
  \Test{10}{1000}{1010}%
  \Test   {110011001100}%
       {111000111000111}%
       {111110100001011}%
  \Test{111000111000111}%
          {110011001100}%
       {111110100001011}%
\end{qstest}

\begin{qstest}{GetUndef}{GetUndef, GetBin, GetOct, GetHex}
  \def\TestUndef#1#2{%
    \let\BS@abc\@undefined
    \expandafter\expandafter\expandafter\Expect
    \expandafter\expandafter\expandafter{%
      \x{abc}{#1}%
    }{#2}%
  }%
  \let\x\bitsetGetBin
  \TestUndef{-1}{0}%
  \TestUndef{0}{0}%
  \TestUndef{1}{0}%
  \TestUndef{2}{00}%
  \TestUndef{8}{00000000}%
  \let\x\bitsetGetOct
  \TestUndef{-1}{0}%
  \TestUndef{0}{0}%
  \TestUndef{1}{0}%
  \TestUndef{2}{0}%
  \TestUndef{3}{0}%
  \TestUndef{4}{00}%
  \TestUndef{5}{00}%
  \TestUndef{6}{00}%
  \TestUndef{7}{000}%
  \TestUndef{8}{000}%
  \TestUndef{9}{000}%
  \TestUndef{10}{0000}%
  \let\x\bitsetGetHex
  \TestUndef{-1}{0}%
  \TestUndef{0}{0}%
  \TestUndef{1}{0}%
  \TestUndef{2}{0}%
  \TestUndef{3}{0}%
  \TestUndef{4}{0}%
  \TestUndef{5}{00}%
  \TestUndef{6}{00}%
  \TestUndef{7}{00}%
  \TestUndef{8}{00}%
  \TestUndef{9}{000}%
  \TestUndef{10}{000}%
  \TestUndef{12}{000}%
  \TestUndef{13}{0000}%
  \TestUndef{16}{0000}%
  \TestUndef{17}{00000}%
\end{qstest}

\begin{qstest}{SetBin}{SetBin}
  \def\Test#1#2{%
    \let\BS@abc\@undefined
    \bitsetSetBin{abc}{#1}%
    \expandafter\Expect\expandafter{\BS@abc}{#2}%
  }%
  \Test{}{0}%
  \Test{0}{0}%
  \Test{1}{1}%
  \Test{10}{01}%
  \Test{11}{11}%
  \Test{010}{01}%
  \Test{011}{11}%
  \Test{0010}{01}%
  \Test{1010}{0101}%
\end{qstest}

\begin{qstest}{SetOct}{SetOct}
  \def\Test#1#2{%
    \bitsetSetOct{abc}{#1}%
    \expandafter\Expect\expandafter{\BS@abc}{#2}%
  }%
  \Test{}{0}%
  \Test{0}{0}%
  \Test{000}{0}%
  \Test{1}{1}%
  \Test{001}{1}%
  \Test{010}{0001}%
  \Test{020}{00001}%
  \Test{42}{010001}%
  \Test{377}{11111111}%
  \Test{0377}{11111111}%
  \Test{76543210}{000100010110001101011111}%
  \Test{ 0 7 0 7 1 }{100111000111}%
\end{qstest}

\begin{qstest}{SetHex}{SetHex}
  \def\Test#1#2{%
    \bitsetSetHex{abc}{#1}%
    \expandafter\Expect\expandafter{\BS@abc}{#2}%
  }%
  \Test{}{0}%
  \Test{0}{0}%
  \Test{000}{0}%
  \Test{1}{1}%
  \Test{001}{1}%
  \Test{010}{00001}%
  \Test{020}{000001}%
  \Test{42}{0100001}%
  \Test{3F}{111111}%
  \Test{03F}{111111}%
  \Test{43210}{0000100001001100001}%
  \Test{98765}{10100110111000011001}%
  \Test{FEDCBA}{010111010011101101111111}%
  \Test{ 0 F 0 F 1 }{1000111100001111}%
\end{qstest}

\begin{qstest}{SetDec}{SetDec}
  \def\Test#1#2{%
    \bitsetSetDec{abc}{#1}%
    \expandafter\Expect\expandafter{\BS@abc}{#2}%
  }%
  \Test{}{0}%
  \Test{0}{0}%
  \Test{000}{0}%
  \Test{1}{1}%
  \Test{7}{111}%
  \Test{8}{0001}%
  \Test{001}{1}%
  \Test{010}{0101}%
  \Test{020}{00101}%
  \Test{53}{101011}%
  \Test{255}{11111111}%
  \Test{256}{000000001}%
  \Test{999999999}{111111111001001101011001110111}%
  \Test{1000000000}{000000000101001101011001110111}%
  \Test{4210987654}{01100001010010010111111101011111}%
  \Test{2147483647}{1111111111111111111111111111111}%
  \Test{2147483648}{00000000000000000000000000000001}%
\end{qstest}

\begin{qstest}{GetBin}{GetBin}
  \def\TestUndef#1#2{%
    \let\BS@abc\@undefined
    \expandafter\expandafter\expandafter\Expect
    \expandafter\expandafter\expandafter{%
      \bitsetGetBin{abc}{#1}%
    }{#2}%
  }%
  \TestUndef{-1}{0}%
  \TestUndef{0}{0}%
  \TestUndef{1}{0}%
  \TestUndef{2}{00}%
  \TestUndef{8}{00000000}%
  \def\Test#1#2{%
    \bitsetSetBin{abc}{#2}%
    \expandafter\expandafter\expandafter\Expect
    \expandafter\expandafter\expandafter{%
      \bitsetGetBin{abc}{#1}%
    }{#2}%
  }%
  \Test{-1}{0}%
  \Test{0}{0}%
  \Test{1}{0}%
  \Test{1}{1}%
  \Test{2}{01}%
  \Test{2}{10}%
  \Test{3}{010}%
  \Test{2}{00}%
  \Test{2}{01}%
  \Test{8}{00101100}%
  \Test{2}{10101}%
  \Test{-100}{11011}%
\end{qstest}

\begin{qstest}{GetOct}{GetOct}
  \def\Test#1#2#3{%
    \edef\x{\zap@space#1 \@empty}%
    \edef\x{\noexpand\bitsetSetBin{abc}{\x}}%
    \x
    \expandafter\expandafter\expandafter\Expect
    \expandafter\expandafter\expandafter{%
      \bitsetGetOct{abc}{#2}%
    }{#3}%
  }%
  \Test{111 110 101 100 011 010 001 000}{0}{76543210}%
  \Test{000 111}{0}{7}%
  \Test{101 000}{-1}{50}%
  \Test{111}{-1}{7}%
  \Test{111}{0}{7}%
  \Test{111}{1}{7}%
  \Test{111}{3}{7}%
  \Test{111}{4}{07}%
  \Test{111}{6}{07}%
  \Test{111}{7}{007}%
  \Test{111 010}{6}{72}%
  \Test{111 010}{7}{072}%
  \Test{011 111}{0}{37}%
  \Test{011 111}{6}{37}%
  \Test{011 111}{7}{037}%
  \Test{001 111}{0}{17}%
  \Test{001 111}{6}{17}%
  \Test{001 111}{7}{017}%
\end{qstest}

\begin{qstest}{GetHex}{GetHex}
  \def\Test#1#2#3{%
    \bitsetSetBin{abc}{#1}%
    \expandafter\expandafter\expandafter\Expect
    \expandafter\expandafter\expandafter{%
      \bitsetGetHex{abc}{#2}%
    }{#3}%
  }%
  \Test{1111 1110 1101 1100 1011 1010 1001 1000}{0}{FEDCBA98}%
  \Test{0111 0110 0101 0100 0011 0010 0001 0000}{0}{76543210}%
  \Test{0000 1111}{0}{F}%
  \Test{0101 0000}{-1}{50}%
  \Test{1111}{-1}{F}%
  \Test{1111}{0}{F}%
  \Test{1111}{1}{F}%
  \Test{1111}{4}{F}%
  \Test{1111}{5}{0F}%
  \Test{1111}{8}{0F}%
  \Test{1111}{9}{00F}%
  \Test{1111 0010}{8}{F2}%
  \Test{1111 0010}{9}{0F2}%
  \Test{0111 1111}{0}{7F}%
  \Test{0111 1111}{8}{7F}%
  \Test{0111 1111}{9}{07F}%
  \Test{0011 1111}{0}{3F}%
  \Test{0011 1111}{8}{3F}%
  \Test{0011 1111}{9}{03F}%
  \Test{0001 1111}{0}{1F}%
  \Test{0001 1111}{8}{1F}%
  \Test{0001 1111}{9}{01F}%
\end{qstest}

\begin{qstest}{Range}{Range}
  \TestError{%
    Wrong index numbers in range [9..8]\MessageBreak% hash-ok
    for clear/set/flip on bit set `abc'.\MessageBreak
    The lower index exceeds the upper index.\MessageBreak
    Canceling the operation as error recovery%
  }{%
    \bitsetSetRange{abc}{9}{8}%
  }%
  \def\TestErrorNegInd#1#2#3#4#5#6{%
    \TestError{%
      Negative index in range [#2..#3]\MessageBreak % hash-ok
      for \string\bitset #1Range on bit set `abc'.\MessageBreak
      Using [#4..#5] as error recovery% hash-ok
    }{%
      \csname bitset#1Range\endcsname{abc}{#2}{#3}%
      \global\let\BS@global\BS@abc
    }%
    \Check{global}{#6}%
  }%
  \Set{abc}{111}%
  \TestErrorNegInd{Clear}{-1}{0}{0}{0}{111}%
  \TestErrorNegInd{Clear}{0}{-1}{0}{0}{111}%
  \TestErrorNegInd{Clear}{-2}{2}{0}{2}{001}%
  \bitsetReset{abc}%
  \TestErrorNegInd{Set}{-1}{0}{0}{0}{0}%
  \TestErrorNegInd{Set}{0}{-1}{0}{0}{0}%
  \TestErrorNegInd{Set}{-2}{2}{0}{2}{11}%
  \Set{abc}{101}%
  \TestErrorNegInd{Flip}{-1}{0}{0}{0}{101}%
  \TestErrorNegInd{Flip}{0}{-1}{0}{0}{101}%
  \TestErrorNegInd{Flip}{-2}{2}{0}{2}{011}%
  \def\Test#1#2#3#4{%
    \bitsetSetBin{abc}{#1}%
    \csname bitset\TestOp Range\endcsname{abc}{#2}{#3}%
    \Expect*{\bitsetGetBin{abc}{0}}{#4}%
  }%
  \def\TestOp{Clear}%
  \Test{0}{0}{1}{0}%
  \Test{1111}{1}{2}{1101}%
  \Test{1111}{1}{3}{1001}%
  \Test{1111111100000000}{12}{14}{1100111100000000}%
  \def\TestOp{Set}%
  \Test{0}{0}{1}{1}%
  \Test{1000}{1}{2}{1010}%
  \Test{0}{1}{2}{10}%
  \Test{1}{12}{15}{111000000000001}%
  \Test{1111}{1}{3}{1111}%
  \Test{1000000000000000}{12}{14}{1011000000000000}%
  \def\TestOp{Flip}%
  \Test{0}{0}{1}{1}%
  \Test{1}{0}{1}{0}%
  \Test{10101010}{1}{5}{10110100}%
  \def\Test#1#2#3#4#5{%
    \bitsetSetBin{abc}{#1}%
    \bitsetSetValueRange{abc}{#2}{#3}{#4}%
    \Expect*{\bitsetGetBin{abc}{0}}{#5}%
  }%
  \Test{0}{0}{1}{0}{0}%
  \Test{0}{0}{1}{1}{1}%
  \Test{1010}{1}{3}{0}{1000}%
  \Test{1010}{1}{3}{1}{1110}%
\end{qstest}

\begin{qstest}{ShiftLeft/ShiftRight}{ShiftLeft/ShiftRight}
  \def\@Test#1#2{%
    \let\BS@abc\@undefined
    \csname bitsetShift#1\endcsname{abc}{#2}%
    \Expect*{\BS@abc}{0}%
  }%
  \def\Test#1{%
    \@Test{Left}{#1}%
    \@Test{Right}{#1}%
  }%
  \Test{-16}%
  \Test{-1}%
  \Test{0}%
  \Test{1}%
  \Test{16}%
  \def\Test#1#2#3{%
    \bitsetSetBin{abc}{#1}%
    \bitsetSetBin{result}{#3}%
    \csname bitsetShift\Op\endcsname{abc}{#2}%
    \Expect*{\bitsetGetBin{abc}{0}}*{\bitsetGetBin{result}{0}}%
  }%
  \def\Op{Left}%
  \Test{0}{0}{0}%
  \Test{0}{1}{0}%
  \Test{0}{-1}{0}%
  \Test{1}{0}{1}%
  \Test{1}{1}{10}%
  \Test{1}{-1}{0}%
  \Test{10}{1}{100}%
  \Test{10}{-1}{1}%
  \Test{1}{32}{100000000000000000000000000000000}%
  \Test{1}{-100}{0}%
  \def\Op{Right}%
  \Test{0}{0}{0}%
  \Test{0}{1}{0}%
  \Test{0}{-1}{0}%
  \Test{1}{0}{1}%
  \Test{1}{1}{0}%
  \Test{1}{-1}{10}%
  \Test{10}{1}{1}%
  \Test{10}{-1}{100}%
  \Test{1}{-32}{100000000000000000000000000000000}%
  \Test{1}{100}{0}%
  \Test{110110110110110}{10}{11011}%
  \Test{110110110110110}{100}{0}%
  \Test{1}{100000}{0}%
\end{qstest}

\begin{qstest}{Profile: Set}{Profile: Set}
  \bitsetSet{abc}{4095}%
  \global\let\BS@global\BS@abc
\end{qstest}

\begin{qstest}{Profile: Get}{Profile: Get}
  \edef\x{\bitsetGet{global}{4095}}%
\end{qstest}

\begin{document}
\end{document}
%</test2>
%    \end{macrocode}
%
% \section{Installation}
%
% \subsection{Download}
%
% \paragraph{Package.} This package is available on
% CTAN\footnote{\url{ftp://ftp.ctan.org/tex-archive/}}:
% \begin{description}
% \item[\CTAN{macros/latex/contrib/oberdiek/bitset.dtx}] The source file.
% \item[\CTAN{macros/latex/contrib/oberdiek/bitset.pdf}] Documentation.
% \end{description}
%
%
% \paragraph{Bundle.} All the packages of the bundle `oberdiek'
% are also available in a TDS compliant ZIP archive. There
% the packages are already unpacked and the documentation files
% are generated. The files and directories obey the TDS standard.
% \begin{description}
% \item[\CTAN{install/macros/latex/contrib/oberdiek.tds.zip}]
% \end{description}
% \emph{TDS} refers to the standard ``A Directory Structure
% for \TeX\ Files'' (\CTAN{tds/tds.pdf}). Directories
% with \xfile{texmf} in their name are usually organized this way.
%
% \subsection{Bundle installation}
%
% \paragraph{Unpacking.} Unpack the \xfile{oberdiek.tds.zip} in the
% TDS tree (also known as \xfile{texmf} tree) of your choice.
% Example (linux):
% \begin{quote}
%   |unzip oberdiek.tds.zip -d ~/texmf|
% \end{quote}
%
% \paragraph{Script installation.}
% Check the directory \xfile{TDS:scripts/oberdiek/} for
% scripts that need further installation steps.
% Package \xpackage{attachfile2} comes with the Perl script
% \xfile{pdfatfi.pl} that should be installed in such a way
% that it can be called as \texttt{pdfatfi}.
% Example (linux):
% \begin{quote}
%   |chmod +x scripts/oberdiek/pdfatfi.pl|\\
%   |cp scripts/oberdiek/pdfatfi.pl /usr/local/bin/|
% \end{quote}
%
% \subsection{Package installation}
%
% \paragraph{Unpacking.} The \xfile{.dtx} file is a self-extracting
% \docstrip\ archive. The files are extracted by running the
% \xfile{.dtx} through \plainTeX:
% \begin{quote}
%   \verb|tex bitset.dtx|
% \end{quote}
%
% \paragraph{TDS.} Now the different files must be moved into
% the different directories in your installation TDS tree
% (also known as \xfile{texmf} tree):
% \begin{quote}
% \def\t{^^A
% \begin{tabular}{@{}>{\ttfamily}l@{ $\rightarrow$ }>{\ttfamily}l@{}}
%   bitset.sty & tex/generic/oberdiek/bitset.sty\\
%   bitset.pdf & doc/latex/oberdiek/bitset.pdf\\
%   test/bitset-test1.tex & doc/latex/oberdiek/test/bitset-test1.tex\\
%   test/bitset-test2.tex & doc/latex/oberdiek/test/bitset-test2.tex\\
%   test/bitset-test3.tex & doc/latex/oberdiek/test/bitset-test3.tex\\
%   bitset.dtx & source/latex/oberdiek/bitset.dtx\\
% \end{tabular}^^A
% }^^A
% \sbox0{\t}^^A
% \ifdim\wd0>\linewidth
%   \begingroup
%     \advance\linewidth by\leftmargin
%     \advance\linewidth by\rightmargin
%   \edef\x{\endgroup
%     \def\noexpand\lw{\the\linewidth}^^A
%   }\x
%   \def\lwbox{^^A
%     \leavevmode
%     \hbox to \linewidth{^^A
%       \kern-\leftmargin\relax
%       \hss
%       \usebox0
%       \hss
%       \kern-\rightmargin\relax
%     }^^A
%   }^^A
%   \ifdim\wd0>\lw
%     \sbox0{\small\t}^^A
%     \ifdim\wd0>\linewidth
%       \ifdim\wd0>\lw
%         \sbox0{\footnotesize\t}^^A
%         \ifdim\wd0>\linewidth
%           \ifdim\wd0>\lw
%             \sbox0{\scriptsize\t}^^A
%             \ifdim\wd0>\linewidth
%               \ifdim\wd0>\lw
%                 \sbox0{\tiny\t}^^A
%                 \ifdim\wd0>\linewidth
%                   \lwbox
%                 \else
%                   \usebox0
%                 \fi
%               \else
%                 \lwbox
%               \fi
%             \else
%               \usebox0
%             \fi
%           \else
%             \lwbox
%           \fi
%         \else
%           \usebox0
%         \fi
%       \else
%         \lwbox
%       \fi
%     \else
%       \usebox0
%     \fi
%   \else
%     \lwbox
%   \fi
% \else
%   \usebox0
% \fi
% \end{quote}
% If you have a \xfile{docstrip.cfg} that configures and enables \docstrip's
% TDS installing feature, then some files can already be in the right
% place, see the documentation of \docstrip.
%
% \subsection{Refresh file name databases}
%
% If your \TeX~distribution
% (\teTeX, \mikTeX, \dots) relies on file name databases, you must refresh
% these. For example, \teTeX\ users run \verb|texhash| or
% \verb|mktexlsr|.
%
% \subsection{Some details for the interested}
%
% \paragraph{Attached source.}
%
% The PDF documentation on CTAN also includes the
% \xfile{.dtx} source file. It can be extracted by
% AcrobatReader 6 or higher. Another option is \textsf{pdftk},
% e.g. unpack the file into the current directory:
% \begin{quote}
%   \verb|pdftk bitset.pdf unpack_files output .|
% \end{quote}
%
% \paragraph{Unpacking with \LaTeX.}
% The \xfile{.dtx} chooses its action depending on the format:
% \begin{description}
% \item[\plainTeX:] Run \docstrip\ and extract the files.
% \item[\LaTeX:] Generate the documentation.
% \end{description}
% If you insist on using \LaTeX\ for \docstrip\ (really,
% \docstrip\ does not need \LaTeX), then inform the autodetect routine
% about your intention:
% \begin{quote}
%   \verb|latex \let\install=y\input{bitset.dtx}|
% \end{quote}
% Do not forget to quote the argument according to the demands
% of your shell.
%
% \paragraph{Generating the documentation.}
% You can use both the \xfile{.dtx} or the \xfile{.drv} to generate
% the documentation. The process can be configured by the
% configuration file \xfile{ltxdoc.cfg}. For instance, put this
% line into this file, if you want to have A4 as paper format:
% \begin{quote}
%   \verb|\PassOptionsToClass{a4paper}{article}|
% \end{quote}
% An example follows how to generate the
% documentation with pdf\LaTeX:
% \begin{quote}
%\begin{verbatim}
%pdflatex bitset.dtx
%makeindex -s gind.ist bitset.idx
%pdflatex bitset.dtx
%makeindex -s gind.ist bitset.idx
%pdflatex bitset.dtx
%\end{verbatim}
% \end{quote}
%
% \section{Catalogue}
%
% The following XML file can be used as source for the
% \href{http://mirror.ctan.org/help/Catalogue/catalogue.html}{\TeX\ Catalogue}.
% The elements \texttt{caption} and \texttt{description} are imported
% from the original XML file from the Catalogue.
% The name of the XML file in the Catalogue is \xfile{bitset.xml}.
%    \begin{macrocode}
%<*catalogue>
<?xml version='1.0' encoding='us-ascii'?>
<!DOCTYPE entry SYSTEM 'catalogue.dtd'>
<entry datestamp='$Date$' modifier='$Author$' id='bitset'>
  <name>bitset</name>
  <caption>Handle bit-vector datatype.</caption>
  <authorref id='auth:oberdiek'/>
  <copyright owner='Heiko Oberdiek' year='2007,2011'/>
  <license type='lppl1.3'/>
  <version number='1.1'/>
  <description>
    This package defines and implements the data type bit set,
    a vector of bits. The size of the vector may grow dynamically.
    Individual bits can be manipulated.
    <p/>
    The package is part of the <xref refid='oberdiek'>oberdiek</xref> bundle.
  </description>
  <documentation details='Package documentation'
      href='ctan:/macros/latex/contrib/oberdiek/bitset.pdf'/>
  <ctan file='true' path='/macros/latex/contrib/oberdiek/bitset.dtx'/>
  <miktex location='oberdiek'/>
  <texlive location='oberdiek'/>
  <install path='/macros/latex/contrib/oberdiek/oberdiek.tds.zip'/>
</entry>
%</catalogue>
%    \end{macrocode}
%
% \begin{History}
%   \begin{Version}{2007/09/28 v1.0}
%   \item
%     First version.
%   \end{Version}
%   \begin{Version}{2011/01/30 v1.1}
%   \item
%     Already loaded package files are not input in \hologo{plainTeX}.
%   \end{Version}
% \end{History}
%
% \PrintIndex
%
% \Finale
\endinput
|
% \end{quote}
% Do not forget to quote the argument according to the demands
% of your shell.
%
% \paragraph{Generating the documentation.}
% You can use both the \xfile{.dtx} or the \xfile{.drv} to generate
% the documentation. The process can be configured by the
% configuration file \xfile{ltxdoc.cfg}. For instance, put this
% line into this file, if you want to have A4 as paper format:
% \begin{quote}
%   \verb|\PassOptionsToClass{a4paper}{article}|
% \end{quote}
% An example follows how to generate the
% documentation with pdf\LaTeX:
% \begin{quote}
%\begin{verbatim}
%pdflatex bitset.dtx
%makeindex -s gind.ist bitset.idx
%pdflatex bitset.dtx
%makeindex -s gind.ist bitset.idx
%pdflatex bitset.dtx
%\end{verbatim}
% \end{quote}
%
% \section{Catalogue}
%
% The following XML file can be used as source for the
% \href{http://mirror.ctan.org/help/Catalogue/catalogue.html}{\TeX\ Catalogue}.
% The elements \texttt{caption} and \texttt{description} are imported
% from the original XML file from the Catalogue.
% The name of the XML file in the Catalogue is \xfile{bitset.xml}.
%    \begin{macrocode}
%<*catalogue>
<?xml version='1.0' encoding='us-ascii'?>
<!DOCTYPE entry SYSTEM 'catalogue.dtd'>
<entry datestamp='$Date$' modifier='$Author$' id='bitset'>
  <name>bitset</name>
  <caption>Handle bit-vector datatype.</caption>
  <authorref id='auth:oberdiek'/>
  <copyright owner='Heiko Oberdiek' year='2007,2011'/>
  <license type='lppl1.3'/>
  <version number='1.1'/>
  <description>
    This package defines and implements the data type bit set,
    a vector of bits. The size of the vector may grow dynamically.
    Individual bits can be manipulated.
    <p/>
    The package is part of the <xref refid='oberdiek'>oberdiek</xref> bundle.
  </description>
  <documentation details='Package documentation'
      href='ctan:/macros/latex/contrib/oberdiek/bitset.pdf'/>
  <ctan file='true' path='/macros/latex/contrib/oberdiek/bitset.dtx'/>
  <miktex location='oberdiek'/>
  <texlive location='oberdiek'/>
  <install path='/macros/latex/contrib/oberdiek/oberdiek.tds.zip'/>
</entry>
%</catalogue>
%    \end{macrocode}
%
% \begin{History}
%   \begin{Version}{2007/09/28 v1.0}
%   \item
%     First version.
%   \end{Version}
%   \begin{Version}{2011/01/30 v1.1}
%   \item
%     Already loaded package files are not input in \hologo{plainTeX}.
%   \end{Version}
% \end{History}
%
% \PrintIndex
%
% \Finale
\endinput

%        (quote the arguments according to the demands of your shell)
%
% Documentation:
%    (a) If bitset.drv is present:
%           latex bitset.drv
%    (b) Without bitset.drv:
%           latex bitset.dtx; ...
%    The class ltxdoc loads the configuration file ltxdoc.cfg
%    if available. Here you can specify further options, e.g.
%    use A4 as paper format:
%       \PassOptionsToClass{a4paper}{article}
%
%    Programm calls to get the documentation (example):
%       pdflatex bitset.dtx
%       makeindex -s gind.ist bitset.idx
%       pdflatex bitset.dtx
%       makeindex -s gind.ist bitset.idx
%       pdflatex bitset.dtx
%
% Installation:
%    TDS:tex/generic/oberdiek/bitset.sty
%    TDS:doc/latex/oberdiek/bitset.pdf
%    TDS:doc/latex/oberdiek/test/bitset-test1.tex
%    TDS:doc/latex/oberdiek/test/bitset-test2.tex
%    TDS:doc/latex/oberdiek/test/bitset-test3.tex
%    TDS:source/latex/oberdiek/bitset.dtx
%
%<*ignore>
\begingroup
  \catcode123=1 %
  \catcode125=2 %
  \def\x{LaTeX2e}%
\expandafter\endgroup
\ifcase 0\ifx\install y1\fi\expandafter
         \ifx\csname processbatchFile\endcsname\relax\else1\fi
         \ifx\fmtname\x\else 1\fi\relax
\else\csname fi\endcsname
%</ignore>
%<*install>
\input docstrip.tex
\Msg{************************************************************************}
\Msg{* Installation}
\Msg{* Package: bitset 2011/01/30 v1.1 Handle bit-vector datatype (HO)}
\Msg{************************************************************************}

\keepsilent
\askforoverwritefalse

\let\MetaPrefix\relax
\preamble

This is a generated file.

Project: bitset
Version: 2011/01/30 v1.1

Copyright (C) 2007, 2011 by
   Heiko Oberdiek <heiko.oberdiek at googlemail.com>

This work may be distributed and/or modified under the
conditions of the LaTeX Project Public License, either
version 1.3c of this license or (at your option) any later
version. This version of this license is in
   http://www.latex-project.org/lppl/lppl-1-3c.txt
and the latest version of this license is in
   http://www.latex-project.org/lppl.txt
and version 1.3 or later is part of all distributions of
LaTeX version 2005/12/01 or later.

This work has the LPPL maintenance status "maintained".

This Current Maintainer of this work is Heiko Oberdiek.

The Base Interpreter refers to any `TeX-Format',
because some files are installed in TDS:tex/generic//.

This work consists of the main source file bitset.dtx
and the derived files
   bitset.sty, bitset.pdf, bitset.ins, bitset.drv, bitset-test1.tex,
   bitset-test2.tex, bitset-test3.tex.

\endpreamble
\let\MetaPrefix\DoubleperCent

\generate{%
  \file{bitset.ins}{\from{bitset.dtx}{install}}%
  \file{bitset.drv}{\from{bitset.dtx}{driver}}%
  \usedir{tex/generic/oberdiek}%
  \file{bitset.sty}{\from{bitset.dtx}{package}}%
  \usedir{doc/latex/oberdiek/test}%
  \file{bitset-test1.tex}{\from{bitset.dtx}{test1}}%
  \file{bitset-test2.tex}{\from{bitset.dtx}{test2,etex}}%
  \file{bitset-test3.tex}{\from{bitset.dtx}{test2,noetex}}%
  \nopreamble
  \nopostamble
  \usedir{source/latex/oberdiek/catalogue}%
  \file{bitset.xml}{\from{bitset.dtx}{catalogue}}%
}

\catcode32=13\relax% active space
\let =\space%
\Msg{************************************************************************}
\Msg{*}
\Msg{* To finish the installation you have to move the following}
\Msg{* file into a directory searched by TeX:}
\Msg{*}
\Msg{*     bitset.sty}
\Msg{*}
\Msg{* To produce the documentation run the file `bitset.drv'}
\Msg{* through LaTeX.}
\Msg{*}
\Msg{* Happy TeXing!}
\Msg{*}
\Msg{************************************************************************}

\endbatchfile
%</install>
%<*ignore>
\fi
%</ignore>
%<*driver>
\NeedsTeXFormat{LaTeX2e}
\ProvidesFile{bitset.drv}%
  [2011/01/30 v1.1 Handle bit-vector datatype (HO)]%
\documentclass{ltxdoc}
\usepackage{holtxdoc}[2011/11/22]
\usepackage{amstext}
\usepackage{ifluatex}
\ifluatex
\else
  \usepackage[T1]{fontenc}%
  \usepackage{lmodern}%
\fi
\begin{document}
  \DocInput{bitset.dtx}%
\end{document}
%</driver>
% \fi
%
% \CheckSum{3510}
%
% \CharacterTable
%  {Upper-case    \A\B\C\D\E\F\G\H\I\J\K\L\M\N\O\P\Q\R\S\T\U\V\W\X\Y\Z
%   Lower-case    \a\b\c\d\e\f\g\h\i\j\k\l\m\n\o\p\q\r\s\t\u\v\w\x\y\z
%   Digits        \0\1\2\3\4\5\6\7\8\9
%   Exclamation   \!     Double quote  \"     Hash (number) \#
%   Dollar        \$     Percent       \%     Ampersand     \&
%   Acute accent  \'     Left paren    \(     Right paren   \)
%   Asterisk      \*     Plus          \+     Comma         \,
%   Minus         \-     Point         \.     Solidus       \/
%   Colon         \:     Semicolon     \;     Less than     \<
%   Equals        \=     Greater than  \>     Question mark \?
%   Commercial at \@     Left bracket  \[     Backslash     \\
%   Right bracket \]     Circumflex    \^     Underscore    \_
%   Grave accent  \`     Left brace    \{     Vertical bar  \|
%   Right brace   \}     Tilde         \~}
%
% \GetFileInfo{bitset.drv}
%
% \title{The \xpackage{bitset} package}
% \date{2011/01/30 v1.1}
% \author{Heiko Oberdiek\\\xemail{heiko.oberdiek at googlemail.com}}
%
% \maketitle
%
% \begin{abstract}
% This package defines and implements the data type bit set,
% a vector of bits. The size of the vector may grow dynamically.
% Individual bits can be manipulated.
% \end{abstract}
%
% \tableofcontents
%
% \clearpage
% \section{Documentation}
%
% \subsection{Introduction}
%
% Annotations in the PDF format know entries whose values are
% integers. This numbers are interpreted as set of flags
% specifying properties. For example, annotation dictionaries
% can have a key |/F|. The bits of its integer value
% are interpreted the following way:
% \begin{quote}
% \begin{tabular}{@{}l|l@{}}
%   Bit position & Property name\\
%   \hline
%   1 & Invisible\\
%   2 & Hidden\\
%   3 & Print\\
%   4 & NoZoom\\
%   5 & NoRotate\\
%   6 & NoView\\
%   7 & ReadOnly\\
%   \dots & \dots
% \end{tabular}
% \end{quote}
% Now, let's see how these values are set in package \xpackage{hyperref}
% before it uses this package (before v6.77a):
%\begin{quote}
%\begin{verbatim}
%\ifFld@hidden /F 6\else /F 4\fi
%\end{verbatim}
%\end{quote}
% Where are the other flags? The following example for key |/Ff|
% in a widget annotation
% supports at least three properties:
%\begin{quote}
%\begin{verbatim}
%\ifFld@multiline
%  \ifFld@readonly /Ff 4097\else /Ff 4096\fi
%\else
%  \ifFld@password
%    \ifFld@readonly /Ff 8193\else /Ff 8192\fi
%  \else
%    \ifFld@readonly /Ff 1\fi
%  \fi
%\fi
%\end{verbatim}
%\end{quote}
% But you see the point. It would be a nightmare to continue
% this way in supporting the missing flag settings. This kind
% of integers may have up to 32 bits.
%
% Therefore I wanted a data structure for setting and
% clearing individual bits. Also it should provide
% an export as decimal number. The snipsets above are
% executed in expansion contexts without \TeX's stomach
% commands. It would be convenient to have an expandable
% conversion from the data structure to the integer that
% gets written to the PDF file.
%
% This package \xpackage{bitset} implements such a data
% structure. The interface is quite close to Java's
% class |BitSet| in order not to learn to many interfaces
% for the same kind of data structure.
%
% \subsection{Glossary}
%
% \begin{description}
% \item[Bit set:]
%   A bit set is a vector of bits or flags.
%   The vector size is unlimited and
%   grows dynamically. An undefined bit set is treated as bit
%   set where all bits are cleared.
%
%   Bit sets are addressed by name. A name should consists of
%   letters or digits. Technically it must survive \cs{csname},
%   see \LaTeX's environment names for other names with such
%   a constraint. Package \xpackage{babel}'s shorthands are
%   not supported due to technical reasons. Shorthand support
%   would break expandable operations.
% \item[Size:]
%   A size of a bit set is the number of bits in use.
%   It's the number of the highest index, incremented by one.
%   Sizes are in the range |0| up to |2147483647|, the highest
%   number supported by \TeX.
% \item[Index:]
%   Bit positions in a bit set are addressed by an index number.
%   The bit vector is zero based. The first and least significant
%   bit is addressed by index |0| and the highest possible bit
%   by |2147483646|.
% \item[Bit:]
%   A bit is enoded as |0| for cleared/disabled or |1| for set/enabled.
% \end{description}
%
% \subsection{Design principles}
%
% \begin{description}
% \item[Name conventions:]
%   To avoid conflicts with existing macro names, the operations
%   are prefixed by the package name.
% \item[Zero based indexes:] The first bit is addressed by zero.
%   (Convention of array indexing in C, Java, \dots)
% \item[Unlimited size:]
%   There is no restriction on the size of a bit set other than
%   usual memory limitations. \cs{bitsetSetDec} and \cs{bitsetGetDec}
%   transparently switch to package \xpackage{bigintcalc} if the numbers
%   get too large for \TeX's number limit.
% \item[Expandibility:]
%   Any operation that does not change the bit set is expandable.
%   And all operations that extract or calculate some result
%   do this in exact two expansion steps. For example,
%   a macro \cs{Macro} wants a bit set as decimal number.
%   But the argument must be a plain number without macros.
%   Thus you could prefix \cs{bitsetGetDec} with \cs{number}.
%   However this won't work for bit sets with 31 or more bits
%   because of \TeX's number limit of $2^{31}-1$.
%   then just hit the operator with two \cs{expandafter}:
%   \begin{quote}
%   |\expandafter\expandafter\expandafter|\\
%   |\Macro\bitsetGetDec{foo}|
%   \end{quote}
%   |\bitsetGetDec| is hit first by the third \cs{expandafter}
%   and then by the second one.
% \item[Format independence:] This package is written as \LaTeX\ package,
%   but it does not depend on \LaTeX. It will also work for other formats
%   such as \plainTeX.
% \item[Independence from \TeX\ engines:]
%   Vanilla \TeX\ is all you need. Calculations are delegated to
%   packages \xpackage{intcalc} and \xpackage{bigintcalc}.
%   They don't need any special features, but they will switch
%   to a little more efficient implementation if features such
%   as \cs{numexpr} are available.
% \item[Numeric arguments:] Anything that is accepted by
%   \cs{number}. If \eTeX\ is detected, also expressions
%   for \cs{numexpr} are supported. The only exception so far
%   is the number for \cs{bitsetSetDec}. The number might be too
%   large for \cs{number} or \cs{numexpr}.
% \item[Error messages:]
%   In expandable contexts, only a limited set of \TeX\ primitive
%   commands work as expected. So called stomach commands behave
%   like \cs{relax} and don't get expanded or executed.
%   Unhappily also the error commands belong to this category.
%   The expandable operations will throw an unknown control
%   sequence instead to get \TeX's and user's attention.
%   The name of these control sequences starts with
%   \cs{BitSetError:} with the type of error after the colon.
% \end{description}
%
% \subsection{Operator overview}
%
% \begin{minipage}{\linewidth}
% \begin{description}
% \def\It#1#2{^^A
%   \item[^^A
%     {#1} ^^A
%     \textmd{(\hyperref[{sec:#2}]{section \ref*{sec:#2}})}^^A
%   ]^^A
%   \mbox{}\\^^A
% }
% \def\op#1{^^A
%   \cs{bitset}\texttt{\bfseries#1}^^A
% }
% \def\M#1{\meta{#1}}
% \def\Q{\hspace*{\fill}}
% \It{Miscellaneous}{misc}
%   \op{Reset}\Q \M{BitSet}\\
%   \op{Let}\Q \M{BitSet A} \M{BitSet B}
% \It{Import}{import}
%   \op{SetBin}, \op{SetOct}, \op{SetHex}\Q \M{BitSet} \M{Value}\\
%   \op{SetDec}\Q \M{BitSet} \M{Value}
% \It{Export\textsuperscript{a}}{export}
%   \op{GetBin}, \op{GetOct}, \op{GetHex}\Q \M{BitSet} \M{MinSize}\\
%   \op{GetDec}\Q \M{BitSet}
% \It{Logical operators}{log}
%   \op{And}, \op{AndNot}\Q \M{BitSet A} \M{BitSet B}\\
%   \op{Or}, \op{Xor}\Q \M{BitSet A} \M{BitSet B}
% \It{Shifting}{shift}
%   \op{ShiftLeft}, \op{ShiftRight}\Q \M{BitSet} \M{ShiftAmount}\\
% \It{Bit manipulation}{bitman}
%   \op{Clear}, \op{Set}, \op{Flip}\Q \M{BitSet} \M{Index}\\
%   \op{SetValue}\Q \M{BitSet} \M{Index} \M{Value}\\
%   \op{ClearRange}, \op{SetRange}, \op{FlipRange}\\
%     \Q \M{BitSet} \M{IndexFrom} \M{IndexTo}\\
%   \op{SetValueRange}\Q \M{BitSet} \M{IndexFrom} \M{IndexTo}
% \It{Bit retrieval\textsuperscript{a}}{bitret}
%   \op{Get}\Q \M{BitSet} \M{Index}\\
%   \op{NextClearBit}, \op{NextSetBit}\Q \M{BitSet} \M{Index}\\
%   \op{GetSetBitList}\Q \M{BitSet}
% \It{Bit set properties}{prop}
%   \op{Size}, \op{Cardinality}\Q \M{BitSet}
% \It{Queries\textsuperscript{b}}{query}
%   \op{IsDefined}, \op{IsEmpty}\Q \M{BitSet} \M{Then} \M{Else}\\
%   \op{Equals}, \op{Intersects}\Q \M{BitSet A} \M{BitSet B} \M{Then} \M{Else}\\
%   \op{Query}\Q \M{BitSet} \M{Index} \M{Then} \M{Else}
% \end{description}
% \footnotetext[1]{Macros are expandable, full expansion by two steps.}
% \footnotetext[2]{Macros are expandable.}
% \end{minipage}
%
% \subsection{Package loading}
%
% The package can be used as normal \LaTeX\ package:
%\begin{quote}
%|\usepackage{bitset}|
%\end{quote}
% Also \plainTeX\ is supported:
%\begin{quote}
%|\input bitset.sty\relax|
%\end{quote}
%
% \subsection{Operators}
%
% The following macros work on and with bit sets.
% A bit set \meta{BitSet} is represented by a name.
% The should consist of letters and digits. Technically
% it must survive \cs{csname}. It is the same constraint
% that must be satisfied by label or environment names
% in \LaTeX.
%
% However active characters that are shorthands
% of package \xpackage{babel} are not supported.
% Support for shorthands works by an assignment.
% But many operators
% such as \cs{bitsetGetDec} must be usable in expandable
% contexts. There assignments will not be executed in the
% best case or they will cause errors.
%
% The bits in a bit set are addressed by non-negative
% integers starting from zero. Thus negative index
% numbers cause an error message. Because index numbers
% are \TeX\ numbers. The largest index is 2147483647.
% But in practice memory limits and patience limits
% will be very likely reached much before.
%
% \subsubsection{Miscellaneous}\label{sec:misc}
%
% There isn't a separate operation for bit set creation.
% For simplicity an undefined bit set is treated as bit set
% with all bits cleared.
%
% \begin{declcs}{bitsetReset} \M{BitSet}
% \end{declcs}
% Macro \cs{bitsetReset} clears all bits. The result is an
% empty bit set. It may also be used as replacement for
% an operation ``new'', because an undefined bit set is defined
% afterwards.
%
% \begin{declcs}{bitsetLet} \M{BitSet A} \M{BitSet B}
% \end{declcs}
% Macro \cs{bitsetLet} performs a simple assignment similar
% to \TeX's \cs{let}. After the operation \meta{BitSet A} has
% the same value as \meta{BitSet B}. If \meta{BitSet B} is
% undefined, then \meta{BitSet A} will be the empty bit set.
%
% Note: If \meta{BitSet A} exists, it will be overwritten.
%
% \subsubsection{Import}\label{sec:import}
%
% \begin{declcs}{bitsetSetBin} \M{BitSet} \M{BinaryNumber}\\
%   \cs{bitsetSetOct} \M{BitSet} \M{OctalNumber}\\
%   \cs{bitsetSetHex} \M{BitSet} \M{HexadecimalNumber}
% \end{declcs}
% The numbers are interpreted as bit vectors and the flags in
% the bit \meta{BitSet} set are set accordingly.
% These numeric arguments are the only arguments where
% spaces are allowed. Then the numbers are easier to read.
%
% \begin{declcs}{bitsetSetDec} \M{BitSet} \M{DecimalNumber}
% \end{declcs}
% Macro \cs{bitsetSetDec} uses \meta{DecimalNumber} to
% set the bit set \meta{BitSet}.
% The numeric argument must expand
% to a plain number consisting of decimal digits without
% command tokens or spaces. Internally this argument is
% expanded only. It cannot be passed to \cs{number} or
% \cs{numexpr}, because the number may be too large for them.
% However \cs{number} or |\the\numexpr| may be used
% explicitely. This also helps for unexpandable number
% command tokens or registers (\cs{z@}, \cs{@ne}, \cs{count@}, \dots).
% Also \LaTeX'\ \cs{value} needs prefixing:
% \begin{quote}
% |\bitsetSetDec{foo}{\number\value{bar}}|
% \end{quote}
%
% \subsubsection{Export}\label{sec:export}
%
% \begin{declcs}{bitsetGetBin} \M{BitSet} \M{MinSize}\\
%   \cs{bitsetGetOct} \M{BitSet} \M{MinSize}\\
%   \cs{bitsetGetHex} \M{BitSet} \M{MinSize}
% \end{declcs}
% These macros returns the bit set as binary, octal or hexadecimal
% number. If the bit size is smaller than \meta{MinSize} the gap is
% filled with leading zeros.
% Example:
% \begin{quote}
% |\bitsetReset{abc}|\\
% |\bitsetSet{abc}{2}|\\
% |\bitsetGetBin{abc}{8}| $\rightarrow$ |00000100|\\
% |\bitsetSet{abc}{5}\bitsetSet{abc}{7}|\\
% |\bitsetGetHex{abc}{16}| $\rightarrow$ |00A2|
% \end{quote}
% Macro \cs{bitsetGetHex} uses the uppercase letters |A| to |F|.
% The catcode of the letters is one of 11 (letter) or 12 (other).
%
% \begin{declcs}{bitsetGetDec} \M{BitSet}
% \end{declcs}
% Macro \cs{bitsetGetDec} returns the bit set \meta{BitSet} as
% decimal number. The returned number can be larger than
% \TeX's number limit of $2^{31}-1$.
%
% \subsubsection{Logical operators}\label{sec:log}
%
% \begin{declcs}{bitsetAnd} \M{BitSet A} \M{BitSet B}
% \end{declcs}
% $A_{\text{new}} \mathrel{\mathop{:}}= A_{\text{old}}
% \mathbin{\text{and}} B \qquad (\forall\, \text{bits})$
%
% \begin{declcs}{bitsetAndNot} \M{BitSet A} \M{BitSet B}
% \end{declcs}
% $A_{\text{new}} \mathrel{\mathop{:}}= A_{\text{old}}
% \mathbin{\text{and}} (\text{not}\,B) \qquad (\forall\, \text{bits})$
%
% \begin{declcs}{bitsetOr} \M{BitSet A} \M{BitSet B}
% \end{declcs}
% $A_{\text{new}} \mathrel{\mathop{:}}= A_{\text{old}}
% \mathbin{\text{or}} B \qquad (\forall\, \text{bits})$
%
% \begin{declcs}{bitsetXor} \M{BitSet A} \M{BitSet B}
% \end{declcs}
% $A_{\text{new}} \mathrel{\mathop{:}}= A_{\text{old}}
% \mathbin{\text{xor}} B \qquad (\forall\, \text{bits})$
%
% \subsubsection{Shifting}\label{sec:shift}
%
% \begin{declcs}{bitsetShiftLeft} \M{BitSet} \M{ShiftAmount}\\
%   \cs{bitsetShiftRight} \M{BitSet} \M{ShiftAmount}
% \end{declcs}
% A left shift by one is a multiplication by two, thus left shifting
% moves the flags to higher positions. The new created low positions
% are filled by zeros.
%
% A right shift is the opposite, dividing by two, movint the bits
% to lower positions. The number will become smaller, the lowest
% bits are lost.
%
% If the \meta{ShiftAmount} is negative, it reverts the meaning of
% the shift operation. A left shift becomes a right shift.
% A \meta{ShiftAmount} of zero is ignored.
%
% \subsubsection{Bit manipulation}\label{sec:bitman}
%
% \begin{declcs}{bitsetClear} \M{BitSet} \M{Index}\\
%   \cs{bitsetSet} \M{BitSet} \M{Index}\\
%   \cs{bitsetFlip} \M{BitSet} \M{Index}
% \end{declcs}
% This macros manipulate a single bit in \meta{BitSet} addressed
% by \cs{Index}. Macro \cs{bitsetClear} disables the bit,
% \cs{bitsetSet} enables it and \cs{bitsetFlip} reverts the current
% setting of the bit.
%
% \begin{declcs}{bitsetSetValue} \M{BitSet} \M{Index} \M{Bit}
% \end{declcs}
% Macro \cs{bitsetSetValue} puts bit \meta{Bit} at position \meta{Index}
% in bit set \meta{BitSet}. \meta{Bit} must be a valid \TeX\ number
% equals to zero (disabled/cleared) or one (enabled/set).
%
% \subsubsection{Bit retrieval}\label{sec:bitret}
%
% \begin{declcs}{bitsetGet} \M{BitSet} \M{Index}
% \end{declcs}
% Macro \cs{bitsetGet} extracts the status of the bit at
% position \meta{Index} in bit set \meta{BitSet}.
% Digit |1| is returned if the bit is set/enabled. If the
% bit is cleared/disabled and in cases of an undefined bitset or
% an index number out of range the return value is |0|.
%
% \begin{declcs}{bitsetNextClearBit} \M{BitSet} \M{Index}
% \end{declcs}
% Starting at position \meta{Index} (inclusive) the bits
% are inspected. The first position without a set bit is returned.
% Possible results are decimal numbers: \meta{Index}, \meta{Index} + 1,
% \dots, ($\infty$)
%
% \begin{declcs}{bitsetNextSetBit} \M{BitSet} \M{Index}
% \end{declcs}
% Starting at position \meta{Index} (inclusive) the bits are
% inspected and the index position of the first found set bit
% is returned. If there isn't such a bit, then the result is
% |-1|. In summary possible results are decimal numbers:
% |-1|, \meta{Index}, \meta{Index} + 1, \dots, ($\infty$)
%
% \begin{declcs}{bitsetGetSetBitList} \M{BitSet}
% \end{declcs}
% Macro \cs{bitsetGetSetBitList} is an application for
% \cs{bitsetNextSetBit}. The set bits are iterated and
% returned as comma separated list of index positions
% in increasing order. The list is empty in case of
% an empty bit set.
%
% \subsubsection{Bit set properties}\label{sec:prop}
%
% \begin{declcs}{bitsetSize} \M{BitSet}
% \end{declcs}
% Macro \cs{bitsetSize} returns number of bits in use.
% It is the same as the index number of the highest set/enabled bit
% incremented by one.
%
% \begin{declcs}{bitsetCardinality} \M{BitSet}
% \end{declcs}
% Macro \cs{bitsetCardinality} counts the number of set/enabled bits.
%
% \subsubsection{Queries}\label{sec:query}
%
% Also the query procedures are expandable. They ask for a piece
% of information about a bit set and execute code depending
% on the answer.
%
% \begin{declcs}{bitsetIsDefined} \M{BitSet} \M{Then} \M{Else}
% \end{declcs}
% If the bit set with the name \meta{BitSet} exists the code
% given in \meta{Then} is executed, otherwise \meta{Else} is used.
%
% \begin{declcs}{bitsetIsEmpty} \M{BitSet} \M{Then} \M{Else}
% \end{declcs}
% If the bit set \meta{BitSet} exists and at least one bit
% is set/enabled, the code in \meta{Then} is executed,
% \meta{Else} otherwise.
%
% \begin{declcs}{bitsetEquals} \M{BitSet A} \M{BitSet B} \M{Then} \M{Else}
% \end{declcs}
% Both bit sets are equal if and only if either both are undefined
% or both are defined and represents the same bit values at the
% same positions. Thus this definition is reflexive, symmeric,
% and transitive, enough for an equivalent relation.
%
% \begin{declcs}{bitsetIntersects} \M{BitSet A} \M{BitSet B} \M{Then} \M{Else}
% \end{declcs}
% If and only if \meta{BitSet A} and \meta{BitSet B} have
% at least one bit at the same position that is set, then
% code part \meta{Then} is executed.
%
% \begin{declcs}{bitsetQuery} \M{BitSet} \M{Index} \M{Then} \M{Else}
% \end{declcs}
% It's just a wrapper for \cs{bitsetGet}. If the bit at position
% \meta{Index} is enabled, code \meta{Then} is called.
%
% \StopEventually{
% }
%
% \section{Implementation}
%
% The internal format of a bit set is quite simple, a sequence
% of digits |0| and |1|. The least significant bit is left.
% A bit set without any flag set is encoded by |0|. Also
% undefined bit sets are treated that way. After the
% highest bit that is set there are no further zeroes.
% A regular expression of valid bit sets values:
% \begin{quote}
%  |0|\string||[01]*1|
% \end{quote}
%
%    \begin{macrocode}
%<*package>
%    \end{macrocode}
%
% \subsection{Reload check and package identification}
%    Reload check, especially if the package is not used with \LaTeX.
%    \begin{macrocode}
\begingroup\catcode61\catcode48\catcode32=10\relax%
  \catcode13=5 % ^^M
  \endlinechar=13 %
  \catcode35=6 % #
  \catcode39=12 % '
  \catcode44=12 % ,
  \catcode45=12 % -
  \catcode46=12 % .
  \catcode58=12 % :
  \catcode64=11 % @
  \catcode123=1 % {
  \catcode125=2 % }
  \expandafter\let\expandafter\x\csname ver@bitset.sty\endcsname
  \ifx\x\relax % plain-TeX, first loading
  \else
    \def\empty{}%
    \ifx\x\empty % LaTeX, first loading,
      % variable is initialized, but \ProvidesPackage not yet seen
    \else
      \expandafter\ifx\csname PackageInfo\endcsname\relax
        \def\x#1#2{%
          \immediate\write-1{Package #1 Info: #2.}%
        }%
      \else
        \def\x#1#2{\PackageInfo{#1}{#2, stopped}}%
      \fi
      \x{bitset}{The package is already loaded}%
      \aftergroup\endinput
    \fi
  \fi
\endgroup%
%    \end{macrocode}
%    Package identification:
%    \begin{macrocode}
\begingroup\catcode61\catcode48\catcode32=10\relax%
  \catcode13=5 % ^^M
  \endlinechar=13 %
  \catcode35=6 % #
  \catcode39=12 % '
  \catcode40=12 % (
  \catcode41=12 % )
  \catcode44=12 % ,
  \catcode45=12 % -
  \catcode46=12 % .
  \catcode47=12 % /
  \catcode58=12 % :
  \catcode64=11 % @
  \catcode91=12 % [
  \catcode93=12 % ]
  \catcode123=1 % {
  \catcode125=2 % }
  \expandafter\ifx\csname ProvidesPackage\endcsname\relax
    \def\x#1#2#3[#4]{\endgroup
      \immediate\write-1{Package: #3 #4}%
      \xdef#1{#4}%
    }%
  \else
    \def\x#1#2[#3]{\endgroup
      #2[{#3}]%
      \ifx#1\@undefined
        \xdef#1{#3}%
      \fi
      \ifx#1\relax
        \xdef#1{#3}%
      \fi
    }%
  \fi
\expandafter\x\csname ver@bitset.sty\endcsname
\ProvidesPackage{bitset}%
  [2011/01/30 v1.1 Handle bit-vector datatype (HO)]%
%    \end{macrocode}
%
% \subsection{Catcodes}
%
%    \begin{macrocode}
\begingroup\catcode61\catcode48\catcode32=10\relax%
  \catcode13=5 % ^^M
  \endlinechar=13 %
  \catcode123=1 % {
  \catcode125=2 % }
  \catcode64=11 % @
  \def\x{\endgroup
    \expandafter\edef\csname BitSet@AtEnd\endcsname{%
      \endlinechar=\the\endlinechar\relax
      \catcode13=\the\catcode13\relax
      \catcode32=\the\catcode32\relax
      \catcode35=\the\catcode35\relax
      \catcode61=\the\catcode61\relax
      \catcode64=\the\catcode64\relax
      \catcode123=\the\catcode123\relax
      \catcode125=\the\catcode125\relax
    }%
  }%
\x\catcode61\catcode48\catcode32=10\relax%
\catcode13=5 % ^^M
\endlinechar=13 %
\catcode35=6 % #
\catcode64=11 % @
\catcode123=1 % {
\catcode125=2 % }
\def\TMP@EnsureCode#1#2{%
  \edef\BitSet@AtEnd{%
    \BitSet@AtEnd
    \catcode#1=\the\catcode#1\relax
  }%
  \catcode#1=#2\relax
}
\TMP@EnsureCode{33}{12}% !
\TMP@EnsureCode{39}{12}% '
\TMP@EnsureCode{40}{12}% (
\TMP@EnsureCode{41}{12}% )
\TMP@EnsureCode{42}{12}% *
\TMP@EnsureCode{43}{12}% +
\TMP@EnsureCode{44}{12}% ,
\TMP@EnsureCode{45}{12}% -
\TMP@EnsureCode{46}{12}% .
\TMP@EnsureCode{47}{12}% /
\TMP@EnsureCode{58}{11}% : (letter!)
\TMP@EnsureCode{60}{12}% <
\TMP@EnsureCode{62}{12}% >
\TMP@EnsureCode{63}{14}% ? (comment!)
\TMP@EnsureCode{91}{12}% [
\TMP@EnsureCode{93}{12}% ]
\TMP@EnsureCode{96}{12}% `
\edef\BitSet@AtEnd{\BitSet@AtEnd\noexpand\endinput}
\begingroup\expandafter\expandafter\expandafter\endgroup
\expandafter\ifx\csname BitSet@TestMode\endcsname\relax
\else
  \catcode63=9 % ? (ignore)
\fi
? \let\BitSet@@TestMode\BitSet@TestMode
%    \end{macrocode}
%
% \subsection{Package loading}
%
%    \begin{macrocode}
\begingroup\expandafter\expandafter\expandafter\endgroup
\expandafter\ifx\csname RequirePackage\endcsname\relax
  \def\TMP@RequirePackage#1[#2]{%
    \begingroup\expandafter\expandafter\expandafter\endgroup
    \expandafter\ifx\csname ver@#1.sty\endcsname\relax
      \input #1.sty\relax
    \fi
  }%
  \TMP@RequirePackage{infwarerr}[2007/09/09]%
  \TMP@RequirePackage{intcalc}[2007/09/27]%
  \TMP@RequirePackage{bigintcalc}[2007/09/27]%
\else
  \RequirePackage{infwarerr}[2007/09/09]%
  \RequirePackage{intcalc}[2007/09/27]%
  \RequirePackage{bigintcalc}[2007/09/27]%
\fi
%    \end{macrocode}
%
% \subsection{Help macros}
%
% \subsubsection{Number constant}
%
%    \begin{macro}{\BitSet@MaxSize}
%    \begin{macrocode}
\def\BitSet@MaxSize{2147483647}%
%    \end{macrocode}
%    \end{macro}
%
% \subsubsection{General basic macros}
%
%    \begin{macro}{\BitSet@Empty}
%    \begin{macrocode}
\def\BitSet@Empty{}
%    \end{macrocode}
%    \end{macro}
%    \begin{macro}{\BitSet@FirstOfOne}
%    \begin{macrocode}
\def\BitSet@FirstOfOne#1{#1}
%    \end{macrocode}
%    \end{macro}
%    \begin{macro}{\BitSet@Gobble}
%    \begin{macrocode}
\def\BitSet@Gobble#1{}
%    \end{macrocode}
%    \end{macro}
%    \begin{macro}{\BitSet@FirstOfTwo}
%    \begin{macrocode}
\def\BitSet@FirstOfTwo#1#2{#1}
%    \end{macrocode}
%    \end{macro}
%    \begin{macro}{\BitSet@SecondOfTwo}
%    \begin{macrocode}
\def\BitSet@SecondOfTwo#1#2{#2}
%    \end{macrocode}
%    \end{macro}
%
%    \begin{macro}{\BitSet@Space}
%    \begin{macrocode}
\def\BitSet@Space{ }
%    \end{macrocode}
%    \end{macro}
%    \begin{macro}{\BitSet@ZapSpace}
%    \begin{macrocode}
\def\BitSet@ZapSpace#1 #2{%
  #1%
  \ifx\BitSet@Empty#2%
  \else
    \expandafter\BitSet@ZapSpace
  \fi
  #2%
}
%    \end{macrocode}
%    \end{macro}
%
% \subsubsection{Tail recursion}
%
%    \begin{macro}{\BitSet@Fi}
%    \begin{macrocode}
\let\BitSet@Fi\fi
%    \end{macrocode}
%    \end{macro}
%    \begin{macro}{\BitSet@AfterFi}
%    \begin{macrocode}
\def\BitSet@AfterFi#1#2\BitSet@Fi{\fi#1}
%    \end{macrocode}
%    \end{macro}
%    \begin{macro}{\BitSet@AfterFiFi}
%    \begin{macrocode}
\def\BitSet@AfterFiFi#1#2\BitSet@Fi{\fi\fi#1}%
%    \end{macrocode}
%    \end{macro}
%    \begin{macro}{\BitSet@AfterFiFiFi}
%    \begin{macrocode}
\def\BitSet@AfterFiFiFi#1#2\BitSet@Fi{\fi\fi\fi#1}%
%    \end{macrocode}
%    \end{macro}
%
% \subsubsection{Check macros}
%
%    \begin{macro}{\BitSet@IfUndefined}
%    \begin{macrocode}
\def\BitSet@IfUndefined#1{%
  \expandafter\ifx\csname BS@#1\endcsname\relax
    \expandafter\BitSet@FirstOfTwo
  \else
    \expandafter\BitSet@SecondOfTwo
  \fi
}
%    \end{macrocode}
%    \end{macro}
%
%    \begin{macro}{\BitSet@CheckIndex}
%    |#1|: continuation code\\
%    |#2|: BitSet\\
%    |#3|: Index
%    \begin{macrocode}
\def\BitSet@CheckIndex#1#2#3{%
  \BitSet@IfUndefined{#2}{\bitsetReset{#2}}{}%
  \expandafter\expandafter\expandafter\BitSet@@CheckIndex
  \intcalcNum{#3}!%
  {#2}{#1}%
}
%    \end{macrocode}
%    \end{macro}
%    \begin{macro}{\BitSet@@CheckIndex}
%    |#1|: plain Index\\
%    |#2|: BitSet\\
%    |#3|: continuation code
%    \begin{macrocode}
\def\BitSet@@CheckIndex#1!#2#3{%
  \ifnum#1<0 %
    \BitSet@AfterFi{%
      \@PackageError{bitset}{%
        Invalid negative index (#1)%
      }\@ehc
    }%
  \else
    \BitSet@AfterFi{%
      #3{#2}{#1}%
    }%
  \BitSet@Fi
}
%    \end{macrocode}
%    \end{macro}
%
% \subsection{Miscellaneous}\label{src:misc}
%
%    \begin{macro}{\bitsetReset}
%    \begin{macrocode}
\def\bitsetReset#1{%
  \expandafter\def\csname BS@#1\endcsname{0}%
}
%    \end{macrocode}
%    \end{macro}
%
%    \begin{macro}{\bitsetLet}
%    \begin{macrocode}
\def\bitsetLet#1#2{%
  \BitSet@IfUndefined{#2}{%
    \bitsetReset{#1}%
  }{%
    \expandafter\let\csname BS@#1\expandafter\endcsname
                    \csname BS@#2\endcsname
  }%
}
%    \end{macrocode}
%    \end{macro}
%
% \subsection{Import}\label{src:import}
%
% \subsubsection{From binary number}
%
%    \begin{macro}{\bitsetSetBin}
%    \begin{macrocode}
\def\bitsetSetBin#1#2{%
  \edef\BitSet@Temp{#2}%
  \edef\BitSet@Temp{%
    \expandafter\expandafter\expandafter\BitSet@ZapSpace
    \expandafter\BitSet@Temp\BitSet@Space\BitSet@Empty
  }%
  \edef\BitSet@Temp{%
    \expandafter\BitSet@KillZeros\BitSet@Temp\BitSet@Empty
  }%
  \ifx\BitSet@Temp\BitSet@Empty
    \expandafter\let\csname BS@#1\endcsname\BitSet@Zero
  \else
    \expandafter\edef\csname BS@#1\endcsname{%
      \expandafter\BitSet@Reverse\BitSet@Temp!%
    }%
  \fi
}
%    \end{macrocode}
%    \end{macro}
%    \begin{macro}{\BitSet@KillZeros}
%    \begin{macrocode}
\def\BitSet@KillZeros#1{%
  \ifx#10%
    \expandafter\BitSet@KillZeros
  \else
    #1%
  \fi
}
%    \end{macrocode}
%    \end{macro}
%    \begin{macro}{\BitSet@Reverse}
%    \begin{macrocode}
\def\BitSet@Reverse#1#2!{%
  \ifx\\#2\\%
    #1%
  \else
    \BitSet@AfterFi{%
      \BitSet@Reverse#2!#1%
    }%
  \BitSet@Fi
}
%    \end{macrocode}
%    \end{macro}
%
% \subsubsection{From octal/hex number}
%
%    \begin{macro}{\bitsetSetOct}
%    \begin{macrocode}
\def\bitsetSetOct{%
  \BitSet@SetOctHex\BitSet@FromFirstOct
}
%    \end{macrocode}
%    \end{macro}
%    \begin{macro}{\bitsetSetHex}
%    \begin{macrocode}
\def\bitsetSetHex{%
  \BitSet@SetOctHex\BitSet@FromFirstHex
}
%    \end{macrocode}
%    \end{macro}
%    \begin{macro}{\BitSet@SetOctHex}
%    \begin{macrocode}
\def\BitSet@SetOctHex#1#2#3{%
  \edef\BitSet@Temp{#3}%
  \edef\BitSet@Temp{%
    \expandafter\expandafter\expandafter\BitSet@ZapSpace
    \expandafter\BitSet@Temp\BitSet@Space\BitSet@Empty
  }%
  \edef\BitSet@Temp{%
    \expandafter\BitSet@KillZeros\BitSet@Temp\BitSet@Empty
  }%
  \ifx\BitSet@Temp\BitSet@Empty
    \expandafter\let\csname BS@#2\endcsname\BitSet@Zero
  \else
    \edef\BitSet@Temp{%
      \expandafter#1\BitSet@Temp!%
    }%
    \ifx\BitSet@Temp\BitSet@Empty
      \expandafter\let\csname BS@#2\endcsname\BitSet@Zero
    \else
      \expandafter\edef\csname BS@#2\endcsname{%
        \expandafter\BitSet@Reverse\BitSet@Temp!%
      }%
    \fi
  \fi
}
%    \end{macrocode}
%    \end{macro}
%    \begin{macro}{\BitSet@FromFirstOct}
%    \begin{macrocode}
\def\BitSet@FromFirstOct#1{%
  \ifx#1!%
  \else
    \ifcase#1 \BitSet@AfterFiFi\BitSet@FromFirstOct
    \or 1%
    \or 10%
    \or 11%
    \or 100%
    \or 101%
    \or 110%
    \or 111%
    \else \BitSetError:WrongOctalDigit%
    \fi
    \expandafter\BitSet@FromOct
  \BitSet@Fi
}
%    \end{macrocode}
%    \end{macro}
%    \begin{macro}{\BitSet@FromOct}
%    \begin{macrocode}
\def\BitSet@FromOct#1{%
  \ifx#1!%
  \else
    \ifcase#1 000%
    \or 001%
    \or 010%
    \or 011%
    \or 100%
    \or 101%
    \or 110%
    \or 111%
    \else \BitSetError:WrongOctalDigit%
    \fi
    \expandafter\BitSet@FromOct
  \fi
}
%    \end{macrocode}
%    \end{macro}
%
%    \begin{macro}{\BitSet@FromFirstHex}
%    \begin{macrocode}
\def\BitSet@FromFirstHex#1{%
  \ifx#1!%
  \else
    \ifx#10%
      \BitSet@AfterFiFi\BitSet@FromFirstHex
    \fi
    \expandafter\ifx\csname BitSet@Hex#1\endcsname\relax
      \BitSetError:InvalidHexDigit%
    \else
      \expandafter\expandafter\expandafter\BitSet@KillZeros
      \csname BitSet@Hex#1\endcsname
    \fi
    \expandafter\BitSet@FromHex
  \BitSet@Fi
}
%    \end{macrocode}
%    \end{macro}
%    \begin{macro}{\BitSet@FromHex}
%    \begin{macrocode}
\def\BitSet@FromHex#1{%
  \ifx#1!%
  \else
    \expandafter\ifx\csname BitSet@Hex#1\endcsname\relax
      \BitSetError:InvalidHexDigit%
    \else
      \csname BitSet@Hex#1\endcsname
    \fi
    \expandafter\BitSet@FromHex
  \fi
}
%    \end{macrocode}
%    \end{macro}
%    \begin{macro}{\BitSet@Hex[0..F]}
%    \begin{macrocode}
\def\BitSet@Temp#1{%
  \expandafter\def\csname BitSet@Hex#1\endcsname
}
\BitSet@Temp 0{0000}%
\BitSet@Temp 1{0001}%
\BitSet@Temp 2{0010}%
\BitSet@Temp 3{0011}%
\BitSet@Temp 4{0100}%
\BitSet@Temp 5{0101}%
\BitSet@Temp 6{0110}%
\BitSet@Temp 7{0111}%
\BitSet@Temp 8{1000}%
\BitSet@Temp 9{1001}%
\BitSet@Temp A{1010}%
\BitSet@Temp B{1011}%
\BitSet@Temp C{1100}%
\BitSet@Temp D{1101}%
\BitSet@Temp E{1110}%
\BitSet@Temp F{1111}%
\BitSet@Temp a{1010}%
\BitSet@Temp b{1011}%
\BitSet@Temp c{1100}%
\BitSet@Temp d{1101}%
\BitSet@Temp e{1110}%
\BitSet@Temp f{1111}%
%    \end{macrocode}
%    \end{macro}
%
% \subsubsection{From decimal number}
%
%    \begin{macro}{\bitsetSetDec}
%    \begin{macrocode}
\def\bitsetSetDec#1#2{%
  \edef\BitSet@Temp{#2}%
  \edef\BitSet@Temp{%
    \expandafter\expandafter\expandafter\BitSet@ZapSpace
    \expandafter\BitSet@Temp\BitSet@Space\BitSet@Empty
  }%
  \edef\BitSet@Temp{%
    \expandafter\BitSet@KillZeros\BitSet@Temp\BitSet@Empty
  }%
  \ifx\BitSet@Temp\BitSet@Empty
    \expandafter\let\csname BS@#1\endcsname\BitSet@Zero
  \else
    \ifcase\bigintcalcSgn{\BitSet@Temp} %
      \expandafter\let\csname BS@#1\endcsname\BitSet@Zero
    \or
      \ifnum\bigintcalcCmp\BitSet@Temp\BitSet@MaxSize>0 %
        \expandafter\edef\csname BS@#1\endcsname{%
          \expandafter\BitSet@SetDecBig\BitSet@Temp!%
        }%
      \else
        \expandafter\edef\csname BS@#1\endcsname{%
          \expandafter\BitSet@SetDec\BitSet@Temp!%
        }%
      \fi
    \else
      \@PackageError{bitset}{%
        Bit sets cannot be negative%
      }\@ehc
    \fi
  \fi
}
%    \end{macrocode}
%    \end{macro}
%    \begin{macro}{\BitSet@SetDecBig}
%    \begin{macrocode}
\def\BitSet@SetDecBig#1#2#3#4#5#6#7#8#9!{%
  \ifx\\#9\\%
    \BitSet@SetDec#1#2#3#4#5#6#7#8!%
  \else
    \ifcase\BigIntCalcOdd#1#2#4#5#6#7#8#9! %
      0%
    \or
      1%
?   \else\BitSetError:ThisCannotHappen%
    \fi
    \BitSet@AfterFi{%
      \expandafter\expandafter\expandafter\BitSet@SetDecBig
      \BigIntCalcShr#1#2#3#4#5#6#7#8#9!!%
    }%
  \BitSet@Fi
}
%    \end{macrocode}
%    \end{macro}
%    \begin{macro}{\BitSet@SetDec}
%    \begin{macrocode}
\def\BitSet@SetDec#1!{%
  \ifcase#1 %
  \or 1%
  \else
    \ifodd#1 %
      1%
    \else
      0%
    \fi
    \BitSet@AfterFi{%
      \expandafter\expandafter\expandafter\BitSet@SetDec
      \IntCalcShr#1!!%
    }%
  \BitSet@Fi
}
%    \end{macrocode}
%    \end{macro}
%
% \subsection{Export}\label{src:export}
%
% \subsubsection{To binary number}
%
%    \begin{macro}{\bitsetGetBin}
%    \begin{macrocode}
\def\bitsetGetBin#1#2{%
  \romannumeral0%
  \expandafter\expandafter\expandafter\BitSet@@GetBin
  \intcalcNum{#2}!{#1}%
}
%    \end{macrocode}
%    \end{macro}
%    \begin{macro}{\BitSet@@GetBin}
%    \begin{macrocode}
\def\BitSet@@GetBin#1!#2{%
  \BitSet@IfUndefined{#2}{%
    \ifnum#1>1 %
      \BitSet@AfterFi{%
        \expandafter\expandafter\expandafter\BitSet@Fill
        \IntCalcDec#1!!0%
      }%
    \else
      \BitSet@AfterFi{ 0}%
    \BitSet@Fi
  }{%
    \expandafter\expandafter\expandafter\BitSet@NumBinRev
    \expandafter\expandafter\expandafter1%
    \expandafter\expandafter\expandafter!%
    \csname BS@#2\endcsname!!#1!%
  }%
}
%    \end{macrocode}
%    \end{macro}
%    \begin{macro}{\BitSet@Fill}
%    |#1|: number of leading digits |0|\\
%    |#2|: result
%    \begin{macrocode}
\def\BitSet@Fill#1!{%
  \ifnum#1>0 %
    \BitSet@AfterFi{%
      \expandafter\expandafter\expandafter\BitSet@Fill
      \IntCalcDec#1!!0%
    }%
  \else
    \BitSet@AfterFi{ }%
  \BitSet@Fi
}
%    \end{macrocode}
%    \end{macro}
%
%    \begin{macro}{\BitSet@NumBinRev}
%    |#1|: bit counter (including |#2|)\\
%    |#2#3|: reverted number\\
%    |#4|: result\\
%    |#5|: min size
%    \begin{macrocode}
\def\BitSet@NumBinRev#1!#2#3!{%
  \ifx\\#3\\%
    \BitSet@AfterFi{%
      \BitSet@NumBinFill#1!#2%
    }%
  \else
    \BitSet@AfterFi{%
      \expandafter\expandafter\expandafter\BitSet@NumBinRev
      \IntCalcInc#1!!#3!#2%
    }%
  \BitSet@Fi
}
%    \end{macrocode}
%    \end{macro}
%    \begin{macro}{\BitSet@NumBinFill}
%    \begin{macrocode}
\def\BitSet@NumBinFill#1!#2!#3!{%
  \ifnum#3>#1 %
    \BitSet@AfterFi{%
      \expandafter\expandafter\expandafter\BitSet@Fill
      \IntCalcSub#3!#1!!#2%
    }%
  \else
    \BitSet@AfterFi{ #2}%
  \BitSet@Fi
}
%    \end{macrocode}
%    \end{macro}
%
% \subsubsection{To octal/hexadecimal number}
%
%    \begin{macro}{\bitsetGetOct}
%    \begin{macrocode}
\def\bitsetGetOct#1#2{%
  \romannumeral0%
  \bitsetIsEmpty{#1}{%
    \expandafter\expandafter\expandafter\BitSet@@GetOctHex
    \intcalcNum{#2}!3!230%
  }{%
    \expandafter\expandafter\expandafter\BitSet@@GetOct
    \expandafter\expandafter\expandafter1%
    \expandafter\expandafter\expandafter!%
    \expandafter\expandafter\expandafter!%
    \csname BS@#1\endcsname00%
    \BitSet@Empty\BitSet@Empty\BitSet@Empty!{#2}%
  }%
}
%    \end{macrocode}
%    \end{macro}
%    \begin{macro}{\bitsetGetHex}
%    \begin{macrocode}
\def\bitsetGetHex#1#2{%
  \romannumeral0%
  \bitsetIsEmpty{#1}{%
    \expandafter\expandafter\expandafter\BitSet@@GetOctHex
    \intcalcNum{#2}!4!340%
  }{%
    \expandafter\expandafter\expandafter\BitSet@@GetHex
    \expandafter\expandafter\expandafter1%
    \expandafter\expandafter\expandafter!%
    \expandafter\expandafter\expandafter!%
    \csname BS@#1\endcsname000%
    \BitSet@Empty\BitSet@Empty\BitSet@Empty\BitSet@Empty!{#2}%
  }%
}
%    \end{macrocode}
%    \end{macro}
%    \begin{macro}{\BitSet@@GetOct}
%    |#1|: number of digits\\
%    |#2|: result\\
%    |#3#4#5|: bits
%    \begin{macrocode}
\def\BitSet@@GetOct#1!#2!#3#4#5{%
  \ifx#5\BitSet@Empty
    \BitSet@AfterFi{%
      \expandafter\expandafter\expandafter\BitSet@GetOctHex
      \IntCalcDec#1!!#2!23%
    }%
  \else
    \BitSet@AfterFi{%
      \expandafter\expandafter\expandafter\BitSet@@GetOct
      \number\IntCalcInc#1!\expandafter\expandafter\expandafter!%
      \csname BitSet@Oct#5#4#3\endcsname#2!%
    }%
  \BitSet@Fi
}
%    \end{macrocode}
%    \end{macro}
%    \begin{macro}{\BitSet@Oct[000..111]}
%    \begin{macrocode}
\def\BitSet@Temp#1#2#3#4{%
  \expandafter\def\csname BitSet@Oct#1#2#3\endcsname{#4}%
}
\BitSet@Temp0000%
\BitSet@Temp0011%
\BitSet@Temp0102%
\BitSet@Temp0113%
\BitSet@Temp1004%
\BitSet@Temp1015%
\BitSet@Temp1106%
\BitSet@Temp1117%
%    \end{macrocode}
%    \end{macro}
%    \begin{macro}{\BitSet@@GetHex}
%    |#1|: number of digits\\
%    |#2|: result\\
%    |#3#4#5#6|: bits
%    \begin{macrocode}
\def\BitSet@@GetHex#1!#2!#3#4#5#6{%
  \ifx#6\BitSet@Empty
    \BitSet@AfterFi{%
      \expandafter\expandafter\expandafter\BitSet@GetOctHex
      \IntCalcDec#1!!#2!34%
    }%
  \else
    \BitSet@AfterFi{%
      \expandafter\expandafter\expandafter\BitSet@@GetHex
      \number\IntCalcInc#1!\expandafter\expandafter\expandafter!%
      \csname BitSet@Hex#6#5#4#3\endcsname#2!%
    }%
  \BitSet@Fi
}
%    \end{macrocode}
%    \end{macro}
%    \begin{macro}{\BitSet@Hex[0000..1111]}
%    \begin{macrocode}
\def\BitSet@Temp#1#2#3#4#5{%
  \expandafter\def\csname BitSet@Hex#1#2#3#4\endcsname{#5}%
}
\BitSet@Temp00000%
\BitSet@Temp00011%
\BitSet@Temp00102%
\BitSet@Temp00113%
\BitSet@Temp01004%
\BitSet@Temp01015%
\BitSet@Temp01106%
\BitSet@Temp01117%
\BitSet@Temp10008%
\BitSet@Temp10019%
\BitSet@Temp1010A%
\BitSet@Temp1011B%
\BitSet@Temp1100C%
\BitSet@Temp1101D%
\BitSet@Temp1110E%
\BitSet@Temp1111F%
%    \end{macrocode}
%    \end{macro}
%    \begin{macro}{\BitSet@GetOctHex}
%    Leading zeros $ (|#4| - |#1| * 3 + 2)/3 $ if $ |#4| > |#1| * 3$\\
%    |#1|: digit size\\
%    |#2|: result\\
%    |#3|: bits per digit - 1\\
%    |#4|: bits per digit
%    |#5|: garbage\\
%    |#6|: min size
%    \begin{macrocode}
\def\BitSet@GetOctHex#1!#2!#3#4#5!#6{%
  \expandafter\BitSet@@GetOctHex
  \number\intcalcNum{#6}\expandafter\expandafter\expandafter!%
  \IntCalcMul#1!#4!!#3#4#2%
}
%    \end{macrocode}
%    \end{macro}
%    \begin{macro}{\BitSet@@GetOctHex}
%    |#1|: plain min size\\
%    |#2|: digits * (bits per digit)\\
%    |#3|: bits per digit - 1\\
%    |#4|: bits per digit
%    \begin{macrocode}
\def\BitSet@@GetOctHex#1!#2!#3#4{%
  \ifnum#1>#2 %
    \BitSet@AfterFi{%
      \expandafter\expandafter\expandafter\expandafter
      \expandafter\expandafter\expandafter\BitSet@Fill
      \expandafter\IntCalcDiv\number
      \expandafter\expandafter\expandafter\IntCalcAdd
      \IntCalcSub#1!#2!!#3!!#4!!%
    }%
  \else
    \BitSet@AfterFi{ }%
  \BitSet@Fi
}
%    \end{macrocode}
%    \end{macro}
%
% \subsubsection{To decimal number}
%
%    \begin{macro}{\bitsetGetDec}
%    \begin{macrocode}
\def\bitsetGetDec#1{%
  \romannumeral0%
  \BitSet@IfUndefined{#1}{ 0}{%
    \expandafter\expandafter\expandafter\BitSet@GetDec
    \csname BS@#1\endcsname!%
  }%
}
%    \end{macrocode}
%    \end{macro}
%    \begin{macro}{\BitSet@GetDec}
%    \begin{macrocode}
\def\BitSet@GetDec#1#2!{%
  \ifx\\#2\\%
    \BitSet@AfterFi{ #1}%
  \else
    \BitSet@AfterFi{%
      \BitSet@@GetDec2!#1!#2!%
    }%
  \BitSet@Fi
}
%    \end{macrocode}
%    \end{macro}
%    \begin{macro}{\BitSet@@GetDec}
%    |#1|: power of two\\
%    |#2|: result\\
%    |#3#4|: number
%    \begin{macrocode}
\def\BitSet@@GetDec#1!#2!#3#4!{%
  \ifx\\#4\\%
    \ifx#31%
      \BitSet@AfterFiFi{%
        \expandafter\expandafter\expandafter\BitSet@Space
        \IntCalcAdd#1!#2!%
      }%
    \else
      \BitSet@AfterFiFi{ #2}%
    \fi
  \else
    \ifx#31%
      \BitSet@AfterFiFi{%
        \csname BitSet@N#1%
        \expandafter\expandafter\expandafter\endcsname
        \IntCalcAdd#1!#2!!#4!%
      }%
    \else
      \BitSet@AfterFiFi{%
        \csname BitSet@N#1\endcsname#2!#4!%
      }%
    \fi
  \BitSet@Fi
}
%    \end{macrocode}
%    \end{macro}
%    \begin{macro}{\BitSet@N[1,2,4,...]}
%    \begin{macrocode}
\def\BitSet@Temp#1#2{%
  \expandafter\def\csname BitSet@N#1\endcsname{%
    \BitSet@@GetDec#2!%
  }%
}
\BitSet@Temp{1}{2}
\BitSet@Temp{2}{4}
\BitSet@Temp{4}{8}
\BitSet@Temp{8}{16}
\BitSet@Temp{16}{32}
\BitSet@Temp{32}{64}
\BitSet@Temp{64}{128}
\BitSet@Temp{128}{256}
\BitSet@Temp{256}{512}
\BitSet@Temp{512}{1024}
\BitSet@Temp{1024}{2048}
\BitSet@Temp{2048}{4096}
\BitSet@Temp{4096}{8192}
\BitSet@Temp{8192}{16384}
\BitSet@Temp{16384}{32768}
\BitSet@Temp{32768}{65536}
\BitSet@Temp{65536}{131072}
\BitSet@Temp{131072}{262144}
\BitSet@Temp{262144}{524288}
\BitSet@Temp{524288}{1048576}
\BitSet@Temp{1048576}{2097152}
\BitSet@Temp{2097152}{4194304}
\BitSet@Temp{4194304}{8388608}
\BitSet@Temp{8388608}{16777216}
\BitSet@Temp{16777216}{33554432}
\BitSet@Temp{33554432}{67108864}
\BitSet@Temp{67108864}{134217728}
\BitSet@Temp{134217728}{268435456}
\BitSet@Temp{268435456}{536870912}
\BitSet@Temp{536870912}{1073741824}
%    \end{macrocode}
%    \end{macro}
%    \begin{macro}{\BitSet@N1073741824}
%    \begin{macrocode}
\expandafter\def\csname BitSet@N1073741824\endcsname{%
  \BitSet@GetDecBig2147483648!%
}%
%    \end{macrocode}
%    \end{macro}
%
%    \begin{macro}{\BitSet@GetDecBig}
%    |#1|: current power of two\\
%    |#2|: result\\
%    |#3#4|: number
%    \begin{macrocode}
\def\BitSet@GetDecBig#1!#2!#3#4!{%
  \ifx\\#4\\%
    \ifx#31%
      \BitSet@AfterFiFi{%
        \expandafter\expandafter\expandafter\BitSet@Space
        \BigIntCalcAdd#1!#2!%
      }%
    \else
      \BitSet@AfterFiFi{ #2}%
    \fi
  \else
    \ifx#31%
      \BitSet@AfterFiFi{%
        \expandafter\expandafter\expandafter\BitSet@@GetDecBig
        \BigIntCalcAdd#1!#2!!#1!#4!%
      }%
    \else
      \BitSet@AfterFiFi{%
        \expandafter\expandafter\expandafter\BitSet@GetDecBig
        \BigIntCalcShl#1!!#2!#4!%
      }%
    \fi
  \BitSet@Fi
}
%    \end{macrocode}
%    \end{macro}
%    \begin{macro}{\BitSet@@GetDecBig}
%    |#1|: result\\
%    |#2|: power of two\\
%    |#3#4|: number
%    \begin{macrocode}
\def\BitSet@@GetDecBig#1!#2!{%
  \expandafter\expandafter\expandafter\BitSet@GetDecBig
  \BigIntCalcShl#2!!#1!%
}
%    \end{macrocode}
%    \end{macro}
%
% \subsection{Logical operators}\label{src:log}
%
% \subsubsection{\cs{bitsetAnd}}
%
%    \begin{macro}{\bitsetAnd}
%    Decision table for \cs{bitsetAnd}:
%    \begin{quote}
%    \begin{tabular}{l|l|l|l|}
%                         & undef(B)   & empty(B)   & cardinality(B)$>$0\\
%      \hline
%      undef(A)           & A := empty & A := empty & A := empty\\
%      \hline
%      empty(A)           &            &            &\\
%      \hline
%      cardinality(A)$>$0 & A := empty & A := empty & A \&= B\\
%      \hline
%    \end{tabular}
%    \end{quote}
%    \begin{macrocode}
\def\bitsetAnd#1#2{%
  \bitsetIsEmpty{#1}{%
    \bitsetReset{#1}%
  }{%
    \bitsetIsEmpty{#2}{%
      \bitsetReset{#1}%
    }{%
      \expandafter\edef\csname BS@#1\endcsname{%
        \expandafter\expandafter\expandafter\BitSet@And
        \csname BS@#1\expandafter\expandafter\expandafter\endcsname
        \expandafter\expandafter\expandafter!%
        \csname BS@#2\endcsname!!%
      }%
      \expandafter\ifx\csname BS@#1\endcsname\BitSet@Empty
        \bitsetReset{#1}%
      \fi
    }%
  }%
}
%    \end{macrocode}
%    \end{macro}
%    \begin{macro}{\BitSet@And}
%    \begin{macrocode}
\def\BitSet@And#1#2!#3#4!#5!{%
  \ifx\\#2\\%
    \ifnum#1#3=11 #51\fi
  \else
    \ifx\\#4\\%
      \ifnum#1#3=11 #51\fi
    \else
      \ifnum#1#3=11 %
        #51%
        \BitSet@AfterFiFiFi{%
          \BitSet@And#2!#4!!%
        }%
      \else
        \BitSet@AfterFiFiFi{%
          \BitSet@And#2!#4!#50!%
        }%
      \fi
    \fi
  \BitSet@Fi
}
%    \end{macrocode}
%    \end{macro}
%
% \subsubsection{\cs{bitsetAndNot}}
%
%    \begin{macro}{\bitsetAndNot}
%    Decision table for \cs{bitsetAndNot}:
%    \begin{quote}
%    \begin{tabular}{l|l|l|l|}
%                         & undef(B)   & empty(B)   & cardinality(B)$>$0\\
%      \hline
%      undef(A)           & A := empty & A := empty & A := empty\\
%      \hline
%      empty(A)           &            &            &\\
%      \hline
%      cardinality(A)$>$0 &            &            & A \&= !B\\
%      \hline
%    \end{tabular}
%    \end{quote}
%    \begin{macrocode}
\def\bitsetAndNot#1#2{%
  \bitsetIsEmpty{#1}{%
    \bitsetReset{#1}%
  }{%
    \bitsetIsEmpty{#2}{%
    }{%
      \expandafter\edef\csname BS@#1\endcsname{%
        \expandafter\expandafter\expandafter\BitSet@AndNot
        \csname BS@#1\expandafter\expandafter\expandafter\endcsname
        \expandafter\expandafter\expandafter!%
        \csname BS@#2\endcsname!!%
      }%
      \expandafter\ifx\csname BS@#1\endcsname\BitSet@Empty
        \bitsetReset{#1}%
      \fi
    }%
  }%
}
%    \end{macrocode}
%    \end{macro}
%    \begin{macro}{\BitSet@AndNot}
%    \begin{macrocode}
\def\BitSet@AndNot#1#2!#3#4!#5!{%
  \ifx\\#2\\%
    \ifnum#1#3=10 #51\fi
  \else
    \ifx\\#4\\%
      #5%
      \ifnum#1#3=10 1\else 0\fi
      #2%
    \else
      \ifnum#1#3=10 %
        #51%
        \BitSet@AfterFiFiFi{%
          \BitSet@AndNot#2!#4!!%
        }%
      \else
        \BitSet@AfterFiFiFi{%
          \BitSet@AndNot#2!#4!#50!%
        }%
      \fi
    \fi
  \BitSet@Fi
}
%    \end{macrocode}
%    \end{macro}
%
% \subsubsection{\cs{bitsetOr}}
%
%    \begin{macro}{\bitsetOr}
%    Decision table for \cs{bitsetOr}:
%    \begin{quote}
%    \begin{tabular}{l|l|l|l|}
%                         & undef(B)   & empty(B)   & cardinality(B)$>$0\\
%      \hline
%      undef(A)           & A := empty & A := empty & A := B\\
%      \hline
%      empty(A)           &            &            & A := B\\
%      \hline
%      cardinality(A)$>$0 &            &            & A \string|= B\\
%      \hline
%    \end{tabular}
%    \end{quote}
%    \begin{macrocode}
\def\bitsetOr#1#2{%
  \bitsetIsEmpty{#2}{%
    \BitSet@IfUndefined{#1}{\bitsetReset{#1}}{}%
  }{%
    \bitsetIsEmpty{#1}{%
      \expandafter\let\csname BS@#1\expandafter\endcsname
                      \csname BS@#2\endcsname
    }{%
      \expandafter\edef\csname BS@#1\endcsname{%
        \expandafter\expandafter\expandafter\BitSet@Or
        \csname BS@#1\expandafter\expandafter\expandafter\endcsname
        \expandafter\expandafter\expandafter!%
        \csname BS@#2\endcsname!%
      }%
    }%
  }%
}
%    \end{macrocode}
%    \end{macro}
%    \begin{macro}{\BitSet@Or}
%    \begin{macrocode}
\def\BitSet@Or#1#2!#3#4!{%
  \ifnum#1#3>0 1\else 0\fi
  \ifx\\#2\\%
    #4%
  \else
    \ifx\\#4\\%
      #2%
    \else
      \BitSet@AfterFiFi{%
        \BitSet@Or#2!#4!%
      }%
    \fi
  \BitSet@Fi
}
%    \end{macrocode}
%    \end{macro}
%
% \subsubsection{\cs{bitsetXor}}
%
%    \begin{macro}{\bitsetXor}
%    Decision table for \cs{bitsetXor}:
%    \begin{quote}
%    \begin{tabular}{l|l|l|l|}
%                         & undef(B)   & empty(B)   & cardinality(B)$>$0\\
%      \hline
%      undef(A)           & A := empty & A := empty & A := B\\
%      \hline
%      empty(A)           &            &            & A := B\\
%      \hline
%      cardinality(A)$>$0 &            &            & A \string^= B\\
%      \hline
%    \end{tabular}
%    \end{quote}
%    \begin{macrocode}
\def\bitsetXor#1#2{%
  \bitsetIsEmpty{#2}{%
    \BitSet@IfUndefined{#1}{\bitsetReset{#1}}{}%
  }{%
    \bitsetIsEmpty{#1}{%
      \expandafter\let\csname BS@#1\expandafter\endcsname
                      \csname BS@#2\endcsname
    }{%
      \expandafter\edef\csname BS@#1\endcsname{%
        \expandafter\expandafter\expandafter\BitSet@Xor
        \csname BS@#1\expandafter\expandafter\expandafter\endcsname
        \expandafter\expandafter\expandafter!%
        \csname BS@#2\endcsname!!%
      }%
      \expandafter\ifx\csname BS@#1\endcsname\BitSet@Empty
        \bitsetReset{#1}%
      \fi
    }%
  }%
}
%    \end{macrocode}
%    \end{macro}
%    \begin{macro}{\BitSet@Xor}
%    \begin{macrocode}
\def\BitSet@Xor#1#2!#3#4!#5!{%
  \ifx\\#2\\%
    \ifx#1#3%
      \ifx\\#4\\%
      \else
        #50#4%
      \fi
    \else
      #51#4%
    \fi
  \else
    \ifx\\#4\\%
      #5%
      \ifx#1#30\else 1\fi
      #2%
    \else
      \ifx#1#3%
        \BitSet@AfterFiFiFi{%
          \BitSet@Xor#2!#4!#50!%
        }%
      \else
        #51%
        \BitSet@AfterFiFiFi{%
          \BitSet@Xor#2!#4!!%
        }%
      \fi
    \fi
  \BitSet@Fi
}
%    \end{macrocode}
%    \end{macro}
%
% \subsubsection{Shifting}\label{src:shift}
%
% \subsubsection{\cs{bitsetShiftLeft}}
%
%    \begin{macro}{\bitsetShiftLeft}
%    \begin{macrocode}
\def\bitsetShiftLeft#1#2{%
  \BitSet@IfUndefined{#1}{%
    \bitsetReset{#1}%
  }{%
    \bitsetIsEmpty{#1}{%
    }{%
      \expandafter\expandafter\expandafter\BitSet@ShiftLeft
      \intcalcNum{#2}!{#1}%
    }%
  }%
}
%    \end{macrocode}
%    \end{macro}
%    \begin{macro}{\BitSet@ShiftLeft}
%    \begin{macrocode}
\def\BitSet@ShiftLeft#1!#2{%
  \ifcase\intcalcSgn{#1} %
  \or
    \begingroup
      \uccode`m=`0 %
    \uppercase\expandafter{\expandafter\endgroup
      \expandafter\edef\csname BS@#2\expandafter\endcsname
      \expandafter{%
        \romannumeral#1000\expandafter\BitSet@Space
        \csname BS@#2\endcsname
      }%
    }%
  \else
    \expandafter\BitSet@ShiftRight\BitSet@Gobble#1!{#2}%
  \fi
}
%    \end{macrocode}
%    \end{macro}
%
% \subsubsection{\cs{bitsetShiftRight}}
%
%    \begin{macro}{\bitsetShiftRight}
%    \begin{macrocode}
\def\bitsetShiftRight#1#2{%
  \BitSet@IfUndefined{#1}{%
    \bitsetReset{#1}%
  }{%
    \bitsetIsEmpty{#1}{%
    }{%
      \expandafter\expandafter\expandafter\BitSet@ShiftRight
      \intcalcNum{#2}!{#1}%
    }%
  }%
}
%    \end{macrocode}
%    \end{macro}
%    \begin{macro}{\BitSet@ShiftRight}
%    \begin{macrocode}
\def\BitSet@ShiftRight#1!#2{%
  \ifcase\intcalcSgn{#1} %
  \or
    \expandafter\edef\csname BS@#2\endcsname{%
      \expandafter\expandafter\expandafter\BitSet@Kill
      \csname BS@#2\expandafter\endcsname\expandafter\BitSet@Empty
      \expandafter=%
      \expandafter{\expandafter}\expandafter{\expandafter}%
      \romannumeral#1000!%
    }%
  \else
    \expandafter\BitSet@ShiftLeft\BitSet@Gobble#1!{#2}%
  \fi
}
%    \end{macrocode}
%    \end{macro}
%    \begin{macro}{\BitSet@Kill}
%    \begin{macrocode}
\def\BitSet@Kill#1#2=#3#4#5{%
  #3#4%
  \ifx#5!%
    \ifx#1\BitSet@Empty
      0%
    \else
      #1#2%
    \fi
  \else
    \ifx#1\BitSet@Empty
      0%
      \BitSet@AfterFiFi\BitSet@Cleanup
    \else
      \BitSet@Kill#2=%
    \fi
  \BitSet@Fi
}
%    \end{macrocode}
%    \end{macro}
%
% \subsection{Bit manipulation}\label{src:bitman}
%
%    \begin{macro}{\bitsetClear}
%    \begin{macrocode}
\def\bitsetClear{%
  \BitSet@CheckIndex\BitSet@Clear
}
%    \end{macrocode}
%    \end{macro}
%    \begin{macro}{\bitsetSet}
%    \begin{macrocode}
\def\bitsetSet{%
  \BitSet@CheckIndex\BitSet@Set
}
%    \end{macrocode}
%    \end{macro}
%    \begin{macro}{\bitsetFlip}
%    \begin{macrocode}
\def\bitsetFlip{%
  \BitSet@CheckIndex\BitSet@Flip
}
%    \end{macrocode}
%    \end{macro}
%    \begin{macro}{\bitsetSetValue}
%    \begin{macrocode}
\def\bitsetSetValue#1#2#3{%
  \expandafter\expandafter\expandafter\BitSet@SetValue
  \intcalcNum{#3}!{#1}{#2}%
}
%    \end{macrocode}
%    \end{macro}
%    \begin{macro}{\BitSet@SetValue}
%    |#1|: plain value\\
%    |#2|: BitSet\\
%    |#3|: Index
%    \begin{macrocode}
\def\BitSet@SetValue#1!{%
  \BitSet@CheckIndex{%
    \ifcase#1 %
      \expandafter\BitSet@Clear
    \or
      \expandafter\BitSet@Set
    \else
      \BitSet@ErrorInvalidBitValue{#1}%
      \expandafter\expandafter\expandafter\BitSet@Gobble
      \expandafter\BitSet@Gobble
    \fi
  }%
}
%    \end{macrocode}
%    \end{macro}
%    \begin{macro}{\BitSet@ErrorInvalidBitValue}
%    |#1|: Wrong bit value
%    \begin{macrocode}
\def\BitSet@ErrorInvalidBitValue#1{%
  \@PackageError{bitset}{%
    Invalid bit value (#1) not in range 0..1%
  }\@ehc
}
%    \end{macrocode}
%    \end{macro}
%
% \subsubsection{Clear operation}
%
%    \begin{macro}{\BitSet@Clear}
%    |#1|: BitSet\\
%    |#2|: plain and checked index
%    \begin{macrocode}
\def\BitSet@Clear#1#2{%
  \edef\BitSet@Temp{%
    \expandafter\expandafter\expandafter\BitSet@@Clear
    \csname BS@#1\expandafter\endcsname
    \expandafter\BitSet@Empty\expandafter=\expandafter!%
    \romannumeral#2000!%
  }%
  \expandafter\let\csname BS@#1\expandafter\endcsname
  \ifx\BitSet@Temp\BitSet@Empty
    \BitSet@Zero
  \else
    \BitSet@Temp
  \fi
}
%    \end{macrocode}
%    \end{macro}
%    \begin{macro}{\BitSet@@Clear}
%    \begin{macrocode}
\def\BitSet@@Clear#1#2=#3!#4{%
  \ifx#4!%
    \ifx#1\BitSet@Empty
    \else
      \ifx\BitSet@Empty#2%
      \else
        #30#2%
      \fi
    \fi
  \else
    \ifx#1\BitSet@Empty
      \BitSet@AfterFiFi\BitSet@Cleanup
    \else
      \ifx#10%
        \BitSet@AfterFiFiFi{%
          \BitSet@@Clear#2=#30!%
        }%
      \else
        #31%
        \BitSet@AfterFiFiFi{%
          \BitSet@@Clear#2=!%
        }%
      \fi
    \fi
  \BitSet@Fi
}
%    \end{macrocode}
%    \end{macro}
%
% \subsubsection{Set operation}
%
%    \begin{macro}{\BitSet@Set}
%    |#1|: BitSet\\
%    |#2|: plain and checked Index
%    \begin{macrocode}
\def\BitSet@Set#1#2{%
  \expandafter\edef\csname BS@#1\endcsname{%
    \expandafter\expandafter\expandafter\BitSet@@Set
    \csname BS@#1\expandafter\endcsname
    \expandafter\BitSet@Empty\expandafter=%
    \expandafter{\expandafter}\expandafter{\expandafter}%
    \romannumeral#2000!%
  }%
}
%    \end{macrocode}
%    \end{macro}
%    \begin{macro}{\BitSet@@Set}
%    \begin{macrocode}
\def\BitSet@@Set#1#2=#3#4#5{%
  #3#4%
  \ifx#5!%
    1#2%
  \else
    \ifx#1\BitSet@Empty
      0%
      \BitSet@AfterFiFi\BitSet@@@Set
    \else
      #1%
      \BitSet@@Set#2=%
    \fi
  \BitSet@Fi
}
%    \end{macrocode}
%    \end{macro}
%    \begin{macro}{\BitSet@@@Set}
%    \begin{macrocode}
\def\BitSet@@@Set#1{%
  \ifx#1!%
    1%
  \else
    0%
    \expandafter\BitSet@@@Set
  \fi
}
%    \end{macrocode}
%    \end{macro}
%
% \subsubsection{Flip operation}
%
%    \begin{macro}{\BitSet@Flip}
%    |#1|: BitSet\\
%    |#2|: plain and checked Index
%    \begin{macrocode}
\def\BitSet@Flip#1#2{%
  \edef\BitSet@Temp{%
    \expandafter\expandafter\expandafter\BitSet@@Flip
    \csname BS@#1\expandafter\endcsname
    \expandafter\BitSet@Empty\expandafter=\expandafter!%
    \romannumeral#2000!%
  }%
  \expandafter\let\csname BS@#1\expandafter\endcsname
  \ifx\BitSet@Temp\BitSet@Empty
    \BitSet@Zero
  \else
    \BitSet@Temp
  \fi
}
%    \end{macrocode}
%    \end{macro}
%    \begin{macro}{\BitSet@@Flip}
%    \begin{macrocode}
\def\BitSet@@Flip#1#2=#3!#4{%
  \ifx#4!%
    \ifx#11%
      \ifx\BitSet@Empty#2%
      \else
        #30#2%
      \fi
    \else
      #31#2%
    \fi
  \else
    \ifx#1\BitSet@Empty
      #30%
      \BitSet@AfterFiFi\BitSet@@@Set
    \else
      \ifx#10%
        \BitSet@AfterFiFiFi{%
          \BitSet@@Flip#2=#30!%
        }%
      \else
        #31%
        \BitSet@AfterFiFiFi{%
          \BitSet@@Flip#2=!%
        }%
      \fi
    \fi
  \BitSet@Fi
}
%    \end{macrocode}
%    \end{macro}
%
% \subsubsection{Range operators}
%
%    \begin{macro}{\bitsetClearRange}
%    \begin{macrocode}
\def\bitsetClearRange{%
  \BitSet@Range\BitSet@Clear
}
%    \end{macrocode}
%    \end{macro}
%    \begin{macro}{\bitsetSetRange}
%    \begin{macrocode}
\def\bitsetSetRange{%
  \BitSet@Range\BitSet@Set
}
%    \end{macrocode}
%    \end{macro}
%    \begin{macro}{\bitsetFlipRange}
%    \begin{macrocode}
\def\bitsetFlipRange{%
  \BitSet@Range\BitSet@Flip
}
%    \end{macrocode}
%    \end{macro}
%    \begin{macro}{\bitsetSetValueRange}
%    \begin{macrocode}
\def\bitsetSetValueRange#1#2#3#4{%
  \expandafter\expandafter\expandafter\BitSet@SetValueRange
  \intcalcNum{#4}!{#1}{#2}{#3}%
}
%    \end{macrocode}
%    \end{macro}
%    \begin{macro}{\BitSet@SetValueRange}
%    \begin{macrocode}
\def\BitSet@SetValueRange#1!#2#3#4{%
  \ifcase#1 %
    \BitSet@Range\BitSet@Clear{#2}{#3}{#4}%
  \or
    \BitSet@Range\BitSet@Set{#2}{#3}{#4}%
  \else
    \BitSet@ErrorInvalidBitValue{#1}%
  \fi
}
%    \end{macrocode}
%    \end{macro}
%
%    \begin{macro}{\BitSet@Range}
%    |#1|: clear/set/flip macro\\
%    |#2|: BitSet\\
%    |#3|: Index from\\
%    |#4|: Index to
%    \begin{macrocode}
\def\BitSet@Range#1#2#3#4{%
  \edef\BitSet@Temp{%
    \noexpand\BitSet@@Range\noexpand#1{#2}%
    \intcalcNum{#3}!\intcalcNum{#4}!%
  }%
  \BitSet@Temp
}
%    \end{macrocode}
%    \end{macro}
%    \begin{macro}{\BitSet@@Range}
%    |#1|: clear/set/flip macro\\
%    |#2|: BitSet\\
%    |#3|: Index from\\
%    |#4|: Index to
%    \begin{macrocode}
\def\BitSet@@Range#1#2#3!#4!{%
  \ifnum#3<0 %
    \BitSet@NegativeIndex#1{#2}#3!#4!0!#4!%
  \else
    \ifnum#4<0 %
      \BitSet@NegativeIndex#1{#2}#3!#4!#3!0!%
    \else
      \ifcase\intcalcCmp{#3}{#4} %
      \or
        \@PackageError{bitset}{%
          Wrong index numbers in range [#3..#4]\MessageBreak% hash-ok
          for clear/set/flip on bit set `#2'.\MessageBreak
          The lower index exceeds the upper index.\MessageBreak
          Canceling the operation as error recovery%
        }\@ehc
      \else
        \BitSet@@@Range#3!#4!#1{#2}%
      \fi
    \fi
  \fi
}
%    \end{macrocode}
%    \end{macro}
%    \begin{macro}{\BitSet@NegativeIndex}
%    \begin{macrocode}
\def\BitSet@NegativeIndex#1#2#3!#4!#5!#6!{%
  \@PackageError{bitset}{%
    Negative index in range [#3..#4]\MessageBreak % hash-ok
    for \string\bitset
    \ifx#1\BitSet@Clear
      Clear%
    \else
      \ifx#1\BitSet@Set
        Set%
      \else
        Flip%
      \fi
    \fi
    Range on bit set `#2'.\MessageBreak
    Using [#5..#6] as error recovery% hash-ok
  }\@ehc
  \BitSet@@Range#1{#2}#5!#6!%
}
%    \end{macrocode}
%    \end{macro}
%    \begin{macro}{\BitSet@@Range}
%    \begin{macrocode}
\def\BitSet@@@Range#1!#2!#3#4{%
  \ifnum#1<#2 %
    #3{#4}{#1}%
    \BitSet@AfterFi{%
      \expandafter\expandafter\expandafter\BitSet@@@Range
      \IntCalcInc#1!!#2!#3{#4}%
    }%
  \BitSet@Fi
}
%    \end{macrocode}
%    \end{macro}
%
% \subsection{Bit retrieval}\label{src:bitret}
%
% \subsubsection{\cs{bitsetGet}}
%
%    \begin{macro}{\bitsetGet}
%    \begin{macrocode}
\def\bitsetGet#1#2{%
  \number
  \expandafter\expandafter\expandafter\BitSet@Get
  \intcalcNum{#2}!{#1}%
}
%    \end{macrocode}
%    \end{macro}
%    \begin{macro}{\BitSet@Get}
%    |#1|: plain index\\
%    |#2|: BitSet
%    \begin{macrocode}
\def\BitSet@Get#1!#2{%
  \ifnum#1<0 %
    \BitSet@AfterFi{%
      0 \BitSetError:NegativeIndex%
    }%
  \else
    \BitSet@IfUndefined{#2}{0}{%
      \expandafter\expandafter\expandafter\BitSet@@Get
      \csname BS@#2\expandafter\endcsname
      \expandafter!\expandafter=%
      \expandafter{\expandafter}\expandafter{\expandafter}%
      \romannumeral\intcalcNum{#1}000!%
    }%
    \expandafter\BitSet@Space
  \BitSet@Fi
}
%    \end{macrocode}
%    \end{macro}
%    \begin{macro}{\BitSet@@Get}
%    \begin{macrocode}
\def\BitSet@@Get#1#2=#3#4#5{%
  #3#4%
  \ifx#5!%
    \ifx#1!%
      0%
    \else
      #1%
    \fi
  \else
    \ifx#1!%
      0%
      \BitSet@AfterFiFi\BitSet@Cleanup
    \else
      \BitSet@@Get#2=%
    \fi
  \BitSet@Fi
}
%    \end{macrocode}
%    \end{macro}
%
% \subsubsection{\cs{bitsetNextClearBit}, \cs{bitsetNextSetBit}}
%
%    \begin{macro}{\bitsetNextClearBit}
%    \begin{macrocode}
\def\bitsetNextClearBit#1#2{%
  \number
  \expandafter\expandafter\expandafter\BitSet@NextClearBit
  \intcalcNum{#2}!{#1} %
}
%    \end{macrocode}
%    \end{macro}
%    \begin{macro}{\BitSet@NextClearBit}
%    |#1|: Index\\
%    |#2|: BitSet
%    \begin{macrocode}
\def\BitSet@NextClearBit#1!#2{%
  \ifnum#1<0 %
    \BitSet@NextClearBit0!{#2}%
    \BitSet@AfterFi{%
      \expandafter\BitSet@Space
      \expandafter\BitSetError:NegativeIndex\romannumeral0%
    }%
  \else
    \bitsetIsEmpty{#2}{#1}{%
      \expandafter\BitSet@Skip
      \number#1\expandafter\expandafter\expandafter!%
      \csname BS@#2\endcsname!!!!!!!!!=%
      {\BitSet@@NextClearBit#1!}%
    }%
  \BitSet@Fi
}
%    \end{macrocode}
%    \end{macro}
%    \begin{macro}{\BitSet@@NextClearBit}
%    |#1|: index for next bit in |#2|\\
%    |#2|: next bit
%    \begin{macrocode}
\def\BitSet@@NextClearBit#1!#2{%
  \ifx#2!%
    #1%
  \else
    \ifx#20%
      #1%
      \BitSet@AfterFiFi\BitSet@Cleanup
    \else
      \BitSet@AfterFiFi{%
        \expandafter\expandafter\expandafter\BitSet@@NextClearBit
        \IntCalcInc#1!!%
      }%
    \fi
  \BitSet@Fi
}
%    \end{macrocode}
%    \end{macro}
%
%    \begin{macro}{\bitsetNextSetBit}
%    \begin{macrocode}
\def\bitsetNextSetBit#1#2{%
  \number
  \expandafter\expandafter\expandafter\BitSet@NextSetBit
  \intcalcNum{#2}!{#1} %
}
%    \end{macrocode}
%    \end{macro}
%    \begin{macro}{\BitSet@NextSetBit}
%    |#1|: Index\\
%    |#2|: BitSet
%    \begin{macrocode}
\def\BitSet@NextSetBit#1!#2{%
  \ifnum#1<0 %
    \BitSet@NextSetBit0!{#2}%
    \BitSet@AfterFi{%
      \expandafter\BitSet@Space
      \expandafter\BitSetError:NegativeIndex\romannumeral0%
    }%
  \else
    \bitsetIsEmpty{#2}{-1}{%
      \expandafter\BitSet@Skip
      \number#1\expandafter\expandafter\expandafter!%
      \csname BS@#2\endcsname!!!!!!!!!=%
      {\BitSet@@NextSetBit#1!}%
    }%
  \BitSet@Fi
}
%    \end{macrocode}
%    \end{macro}
%    \begin{macro}{\BitSet@@NextSetBit}
%    |#1|: index for next bit in |#2|\\
%    |#2|: next bit
%    \begin{macrocode}
\def\BitSet@@NextSetBit#1!#2{%
  \ifx#2!%
    -1%
  \else
    \ifx#21%
      #1%
      \BitSet@AfterFiFi\BitSet@Cleanup
    \else
      \BitSet@AfterFiFi{%
        \expandafter\expandafter\expandafter\BitSet@@NextSetBit
        \IntCalcInc#1!!%
      }%
    \fi
  \BitSet@Fi
}
%    \end{macrocode}
%    \end{macro}
%    \begin{macro}{\BitSet@Cleanup}
%    \begin{macrocode}
\def\BitSet@Cleanup#1!{}
%    \end{macrocode}
%    \end{macro}
%
%    \begin{macro}{\BitSet@Skip}
%    |#1|: number of bits to skip\\
%    |#2|: bits\\
%    |#3|: continuation code
%    \begin{macrocode}
\def\BitSet@Skip#1!#2{%
  \ifx#2!%
    \BitSet@AfterFi{%
      \BitSet@SkipContinue%
    }%
  \else
    \ifcase#1 %
      \BitSet@AfterFiFi{%
        \BitSet@SkipContinue#2%
      }%
    \or
      \BitSet@AfterFiFi\BitSet@SkipContinue
    \or
      \BitSet@AfterFiFi{%
        \expandafter\BitSet@SkipContinue\BitSet@Gobble
      }%
    \else
      \ifnum#1>8 %
        \BitSet@AfterFiFiFi{%
          \expandafter\BitSet@Skip
          \number\IntCalcSub#1!8!\expandafter!%
          \BitSet@GobbleSeven
        }%
      \else
        \BitSet@AfterFiFiFi{%
          \expandafter\expandafter\expandafter\BitSet@Skip
          \IntCalcDec#1!!%
        }%
      \fi
    \fi
  \BitSet@Fi
}
%    \end{macrocode}
%    \end{macro}
%    \begin{macro}{\BitSet@SkipContinue}
%    |#1|: remaining bits\\
%    |#2|: continuation code
%    \begin{macrocode}
\def\BitSet@SkipContinue#1!#2=#3{%
  #3#1!%
}
%    \end{macrocode}
%    \end{macro}
%    \begin{macro}{\BitSet@GobbleSeven}
%    \begin{macrocode}
\def\BitSet@GobbleSeven#1#2#3#4#5#6#7{}
%    \end{macrocode}
%    \end{macro}
%
% \subsubsection{\cs{bitsetGetSetBitList}}
%
%    \begin{macro}{\bitsetGetSetBitList}
%    It's just a wrapper for \cs{bitsetNextSetBit}.
%    \begin{macrocode}
\def\bitsetGetSetBitList#1{%
  \romannumeral0%
  \bitsetIsEmpty{#1}{ }{%
    \expandafter\BitSet@GetSetBitList
    \number\BitSet@NextSetBit0!{#1}!{#1}{}!%
  }%
}
%    \end{macrocode}
%    \end{macro}
%    \begin{macro}{\BitSet@GetSetBitList}
%    |#1|: found index\\
%    |#2|: BitSet\\
%    |#3|: comma
%    |#4|: result
%    \begin{macrocode}
\def\BitSet@GetSetBitList#1!#2#3#4!{%
  \ifnum#1<0 %
    \BitSet@AfterFi{ #4}%
  \else
    \BitSet@AfterFi{%
      \expandafter\BitSet@GetSetBitList\number
      \expandafter\expandafter\expandafter\BitSet@NextSetBit
      \IntCalcInc#1!!{#2}!{#2},#4#3#1!%
    }%
  \BitSet@Fi
}
%    \end{macrocode}
%    \end{macro}
%
% \subsection{Bit set properties}\label{src:prop}
%
%    \begin{macro}{\bitsetSize}
%    \begin{macrocode}
\def\bitsetSize#1{%
  \number
  \BitSet@IfUndefined{#1}{0 }{%
    \expandafter\expandafter\expandafter\BitSet@Size
    \expandafter\expandafter\expandafter1%
    \expandafter\expandafter\expandafter!%
    \csname BS@#1\endcsname!0!%
  }%
}
%    \end{macrocode}
%    \end{macro}
%    \begin{macro}{\BitSet@Size}
%    |#1|: counter\\
%    |#2#3|: bits\\
%    |#4|: result
%    \begin{macrocode}
\def\BitSet@Size#1!#2#3!#4!{%
  \ifx#21%
    \ifx\\#3\\%
      \BitSet@AfterFiFi{#1 }%
    \else
      \BitSet@AfterFiFi{%
        \expandafter\expandafter\expandafter\BitSet@Size
        \IntCalcInc#1!!#3!#1!%
      }%
    \fi
  \else
    \ifx\\#3\\%
      \BitSet@AfterFiFi{#4 }%
    \else
      \BitSet@AfterFiFi{%
        \expandafter\expandafter\expandafter\BitSet@Size
        \IntCalcInc#1!!#3!#4!%
      }%
    \fi
  \fi
  \BitSet@Fi
}
%    \end{macrocode}
%    \end{macro}
%
%    \begin{macro}{\bitsetCardinality}
%    \begin{macrocode}
\def\bitsetCardinality#1{%
  \number
  \BitSet@IfUndefined{#1}{0 }{%
    \expandafter\expandafter\expandafter\BitSet@Cardinality
    \expandafter\expandafter\expandafter0%
    \expandafter\expandafter\expandafter!%
    \csname BS@#1\endcsname!%
  }%
}
%    \end{macrocode}
%    \end{macro}
%    \begin{macro}{\BitSet@Cardinality}
%    |#1|: result\\
%    |#2#3|: bits
%    \begin{macrocode}
\def\BitSet@Cardinality#1!#2#3!{%
  \ifx#21%
    \ifx\\#3\\%
      \BitSet@AfterFiFi{\IntCalcInc#1! }%
    \else
      \BitSet@AfterFiFi{%
        \expandafter\expandafter\expandafter\BitSet@Cardinality
        \IntCalcInc#1!!#3!%
      }%
    \fi
  \else
    \ifx\\#3\\%
      \BitSet@AfterFiFi{#1 }%
    \else
      \BitSet@AfterFiFi{%
        \BitSet@Cardinality#1!#3!%
      }%
    \fi
  \fi
  \BitSet@Fi
}
%    \end{macrocode}
%    \end{macro}
%
% \subsection{Queries}\label{src:query}
%
%    \begin{macro}{\bitsetIsDefined}
%    \begin{macrocode}
\def\bitsetIsDefined#1{%
  \BitSet@IfUndefined{#1}%
  \BitSet@SecondOfTwo
  \BitSet@FirstOfTwo
}
%    \end{macrocode}
%    \end{macro}
%    \begin{macro}{\bitsetIsEmpty}
%    \begin{macrocode}
\def\bitsetIsEmpty#1{%
  \BitSet@IfUndefined{#1}\BitSet@FirstOfTwo{%
    \expandafter\ifx\csname BS@#1\endcsname\BitSet@Zero
      \expandafter\BitSet@FirstOfTwo
    \else
      \expandafter\BitSet@SecondOfTwo
    \fi
  }%
}
%    \end{macrocode}
%    \end{macro}
%    \begin{macro}{\BitSet@Zero}
%    \begin{macrocode}
\def\BitSet@Zero{0}
%    \end{macrocode}
%    \end{macro}
%
%    \begin{macro}{\bitsetQuery}
%    \begin{macrocode}
\def\bitsetQuery#1#2{%
  \ifnum\bitsetGet{#1}{#2}=1 %
    \expandafter\BitSet@FirstOfTwo
  \else
    \expandafter\BitSet@SecondOfTwo
  \fi
}
%    \end{macrocode}
%    \end{macro}
%
%    \begin{macro}{\bitsetEquals}
%    \begin{macrocode}
\def\bitsetEquals#1#2{%
  \BitSet@IfUndefined{#1}{%
    \BitSet@IfUndefined{#2}\BitSet@FirstOfTwo\BitSet@SecondOfTwo
  }{%
    \BitSet@IfUndefined{#2}\BitSet@SecondOfTwo{%
      \expandafter\ifx\csname BS@#1\expandafter\endcsname
                      \csname BS@#2\endcsname
        \expandafter\BitSet@FirstOfTwo
      \else
        \expandafter\BitSet@SecondOfTwo
      \fi
    }%
  }%
}
%    \end{macrocode}
%    \end{macro}
%
%    \begin{macro}{\bitsetIntersects}
%    \begin{macrocode}
\def\bitsetIntersects#1#2{%
  \bitsetIsEmpty{#1}\BitSet@SecondOfTwo{%
    \bitsetIsEmpty{#2}\BitSet@SecondOfTwo{%
      \expandafter\expandafter\expandafter\BitSet@Intersects
      \csname BS@#1\expandafter\expandafter\expandafter\endcsname
      \expandafter\expandafter\expandafter!%
      \csname BS@#2\endcsname!%
    }%
  }%
}
%    \end{macrocode}
%    \end{macro}
%    \begin{macro}{\BitSet@Intersects}
%    \begin{macrocode}
\def\BitSet@Intersects#1#2!#3#4!{%
  \ifnum#1#3=11 %
    \BitSet@AfterFi\BitSet@FirstOfTwo
  \else
    \ifx\\#2\\%
      \BitSet@AfterFiFi\BitSet@SecondOfTwo
    \else
      \ifx\\#4\\%
        \BitSet@AfterFiFiFi\BitSet@SecondOfTwo
      \else
        \BitSet@AfterFiFiFi{%
          \BitSet@Intersects#2!#4!%
        }%
      \fi
    \fi
  \BitSet@Fi
}
%    \end{macrocode}
%    \end{macro}
%
%    \begin{macrocode}
\BitSet@AtEnd%
%</package>
%    \end{macrocode}
%
% \section{Test}
%
% \subsection{Catcode checks for loading}
%
%    \begin{macrocode}
%<*test1>
%    \end{macrocode}
%    \begin{macrocode}
\catcode`\{=1 %
\catcode`\}=2 %
\catcode`\#=6 %
\catcode`\@=11 %
\expandafter\ifx\csname count@\endcsname\relax
  \countdef\count@=255 %
\fi
\expandafter\ifx\csname @gobble\endcsname\relax
  \long\def\@gobble#1{}%
\fi
\expandafter\ifx\csname @firstofone\endcsname\relax
  \long\def\@firstofone#1{#1}%
\fi
\expandafter\ifx\csname loop\endcsname\relax
  \expandafter\@firstofone
\else
  \expandafter\@gobble
\fi
{%
  \def\loop#1\repeat{%
    \def\body{#1}%
    \iterate
  }%
  \def\iterate{%
    \body
      \let\next\iterate
    \else
      \let\next\relax
    \fi
    \next
  }%
  \let\repeat=\fi
}%
\def\RestoreCatcodes{}
\count@=0 %
\loop
  \edef\RestoreCatcodes{%
    \RestoreCatcodes
    \catcode\the\count@=\the\catcode\count@\relax
  }%
\ifnum\count@<255 %
  \advance\count@ 1 %
\repeat

\def\RangeCatcodeInvalid#1#2{%
  \count@=#1\relax
  \loop
    \catcode\count@=15 %
  \ifnum\count@<#2\relax
    \advance\count@ 1 %
  \repeat
}
\def\RangeCatcodeCheck#1#2#3{%
  \count@=#1\relax
  \loop
    \ifnum#3=\catcode\count@
    \else
      \errmessage{%
        Character \the\count@\space
        with wrong catcode \the\catcode\count@\space
        instead of \number#3%
      }%
    \fi
  \ifnum\count@<#2\relax
    \advance\count@ 1 %
  \repeat
}
\def\space{ }
\expandafter\ifx\csname LoadCommand\endcsname\relax
  \def\LoadCommand{\input bitset.sty\relax}%
\fi
\def\Test{%
  \RangeCatcodeInvalid{0}{47}%
  \RangeCatcodeInvalid{58}{64}%
  \RangeCatcodeInvalid{91}{96}%
  \RangeCatcodeInvalid{123}{255}%
  \catcode`\@=12 %
  \catcode`\\=0 %
  \catcode`\%=14 %
  \LoadCommand
  \RangeCatcodeCheck{0}{36}{15}%
  \RangeCatcodeCheck{37}{37}{14}%
  \RangeCatcodeCheck{38}{47}{15}%
  \RangeCatcodeCheck{48}{57}{12}%
  \RangeCatcodeCheck{58}{63}{15}%
  \RangeCatcodeCheck{64}{64}{12}%
  \RangeCatcodeCheck{65}{90}{11}%
  \RangeCatcodeCheck{91}{91}{15}%
  \RangeCatcodeCheck{92}{92}{0}%
  \RangeCatcodeCheck{93}{96}{15}%
  \RangeCatcodeCheck{97}{122}{11}%
  \RangeCatcodeCheck{123}{255}{15}%
  \RestoreCatcodes
}
\Test
\csname @@end\endcsname
\end
%    \end{macrocode}
%    \begin{macrocode}
%</test1>
%    \end{macrocode}
%
% \subsection{Macro tests}
%
% \subsubsection{Preamble}
%
%    \begin{macrocode}
%<*test2>
\NeedsTeXFormat{LaTeX2e}
\nofiles
\documentclass{article}
\makeatletter
%<*noetex>
\let\SavedNumexpr\numexpr
\let\SavedIfcsname\ifcsname
\let\SavedCurrentgrouplevel\currentgrouplevel
\def\ETeXDisable{%
  \let\ifcsname\@undefined
  \let\numexpr\@undefined
  \let\currentgrouplevel\@undefined
}
\ETeXDisable
%</noetex>
\makeatletter
\chardef\BitSet@TestMode=1 %
\makeatother
\usepackage{bitset}[2011/01/30]
%<*noetex>
\def\ETeXEnable{%
  \let\numexpr\SavedNumexpr
  \let\ifcsname\SavedIfcsname
  \let\currentgrouplevel\SavedCurrentgrouplevel
}
\ETeXEnable
%</noetex>
\usepackage{qstest}
\IncludeTests{*}
\LogTests{log}{*}{*}
\makeatletter
%    \end{macrocode}
%
% \subsubsection{Time}
%
%    \begin{macrocode}
\begingroup\expandafter\expandafter\expandafter\endgroup
\expandafter\ifx\csname pdfresettimer\endcsname\relax
\else
  \newcount\SummaryTime
  \newcount\TestTime
  \SummaryTime=\z@
  \newcommand*{\PrintTime}[2]{%
    \typeout{%
      [Time #1: \strip@pt\dimexpr\number#2sp\relax\space s]%
    }%
  }%
  \newcommand*{\StartTime}[1]{%
    \renewcommand*{\TimeDescription}{#1}%
    \pdfresettimer
  }%
  \newcommand*{\TimeDescription}{}%
  \newcommand*{\StopTime}{%
    \TestTime=\pdfelapsedtime
    \global\advance\SummaryTime\TestTime
    \PrintTime\TimeDescription\TestTime
  }%
  \let\saved@qstest\qstest
  \let\saved@endqstest\endqstest
  \def\qstest#1#2{%
    \saved@qstest{#1}{#2}%
    \StartTime{#1}%
  }%
  \def\endqstest{%
    \StopTime
    \saved@endqstest
  }%
  \AtEndDocument{%
    \PrintTime{summary}\SummaryTime
  }%
\fi
%    \end{macrocode}
%
% \subsubsection{Detection of unwanted space}
%
%    \begin{macrocode}
\let\orig@qstest\qstest
\let\orig@endqstest\endqstest
\def\qstest#1#2{%
  \orig@qstest{#1}{#2}%
  \setbox0\hbox\bgroup\begingroup\ignorespaces
}
\def\endqstest{%
  \endgroup\egroup
  \Expect*{\the\wd0}{0.0pt}%
  \orig@endqstest
}
%    \end{macrocode}
%
% \subsubsection{Test macros}
%
%    \begin{macrocode}
\newcounter{Test}

\def\TestError#1#2{%
  \begingroup
    \setcounter{Test}{0}%
    \sbox0{%
      \def\@PackageError##1##2##3{%
        \stepcounter{Test}%
        \begingroup
          \let\MessageBreak\relax
%<*noetex>
          \ETeXEnable
%</noetex>
          \Expect{##1}{bitset}%
          \Expect*{##2}*{#1}%
        \endgroup
      }%
%<*noetex>
      \ETeXDisable
%</noetex>
      #2%
    }%
    \Expect*{\theTest}{1}%
    \Expect*{\the\wd0}{0.0pt}%
  \endgroup
}

\def\TestErrorNegativeIndex#1#2{%
  \TestError{Invalid negative index (#1)}{#2}%
}

\def\TestGetterUndefined#1{%
  \CheckUndef{dummy}%
  \expandafter\expandafter\expandafter\Expect
  \expandafter\expandafter\expandafter{#1{dummy}}{0}%
}

\def\ExpectBitSet#1#2{%
  \expandafter\expandafter\expandafter\Expect
  \expandafter\expandafter\expandafter
  {\csname BS@#1\endcsname}*{#2}%
}
\def\Check#1#2{%
  \ExpectBitSet{#1}{#2}%
}
\def\CheckUndef#1{%
  \begingroup
    \Expect*{%
      \expandafter
      \ifx\csname BS@#1\endcsname\relax true\else false\fi
    }{true}%
  \endgroup
}
\def\RevCheck#1#2{%
  \ExpectBitSet{#1}{\Reverse#2!!}%
}
\def\Set#1#2{%
  \expandafter\def\csname BS@#1\endcsname{#2}%
}
\def\RevSet#1#2{%
  \expandafter\edef\csname BS@#1\endcsname{%
    \Reverse#2!!%
  }%
}
\def\Reverse#1#2!#3!{%
  \ifx\\#2\\%
    #1#3%
    \expandafter\@gobble
  \else
    \expandafter\@firstofone
  \fi
  {\Reverse#2!#1#3!}%
}
%    \end{macrocode}
%
% \subsubsection{Test sets}
%
%    \begin{macrocode}
\begin{qstest}{Let}{Let}
  \CheckUndef{abc}%
  \CheckUndef{xyz}%
  \bitsetLet{xyz}{abc}%
  \CheckUndef{abc}%
  \Check{xyz}{0}%
  \Set{abc}{1}%
  \Check{abc}{1}%
  \Check{xyz}{0}%
  \bitsetLet{xyz}{abc}%
  \Check{abc}{1}%
  \Check{xyz}{1}%
  \Set{xyz}{11}%
  \Check{abc}{1}%
  \Check{xyz}{11}%
\end{qstest}

\begin{qstest}{Reset}{Reset}
  \bitsetReset{xyz}%
  \Check{xyz}{0}%
  \bitsetReset{abc}%
  \Check{abc}{0}%
  \Set{abc}{10101}%
  \bitsetReset{abc}%
  \Check{abc}{0}%
\end{qstest}

\begin{qstest}{Get/Query}{Get/Query}
  \expandafter\expandafter\expandafter\Expect
  \expandafter\expandafter\expandafter{%
    \bitsetGet{dummy}{0}%
  }{0}%
  \begingroup
    \expandafter\def\csname BitSetError:NegativeIndex\endcsname{}%
    \Set{abc}{1}%
    \Expect*{\bitsetQuery{abc}{-1}{true}{false}}{false}%
  \endgroup
  \def\Test#1#2#3{%
    \Set{abc}{#1}%
    \expandafter\expandafter\expandafter\Expect
    \expandafter\expandafter\expandafter{\bitsetGet{abc}{#2}}{#3}%
    \Expect*{\bitsetQuery{abc}{#2}{true}{false}}%
           *{\ifcase#3 false\or true\else error\fi}%
  }%
  \Test{1}{100}{0}%
  \Test{0}{0}{0}%
  \Test{1}{0}{1}%
  \Test{11}{1}{1}%
  \Test{111}{1}{1}%
  \Test{101}{1}{0}%
  \Test{101}{2}{1}%
  \Test{10100110011}{10}{1}%
\end{qstest}

\begin{qstest}{Size}{Size}
  \TestGetterUndefined\bitsetSize
  \def\Test#1#2{%
    \Set{abc}{#1}%
    \expandafter\expandafter\expandafter\Expect
    \expandafter\expandafter\expandafter{\bitsetSize{abc}}{#2}%
  }%
  \Test{0}{0}%
  \Test{1}{1}%
  \Test{00}{0}%
  \Test{0000000}{0}%
  \Test{10}{1}%
  \Test{01}{2}%
  \Test{11}{2}%
  \Test{010}{2}%
  \Test{011}{3}%
  \Test{100110011}{9}%
  \Test{0000011111000001111100000}{20}%
  \Test{000000000000000000000000011111111111111111111}{45}%
\end{qstest}

\begin{qstest}{Cardinality}{Cardinality}
  \TestGetterUndefined\bitsetCardinality
  \def\Test#1#2{%
    \Set{abc}{#1}%
    \expandafter\expandafter\expandafter\Expect
    \expandafter\expandafter\expandafter{%
      \bitsetCardinality{abc}%
    }{#2}%
  }%
  \Test{0}{0}%
  \Test{1}{1}%
  \Test{00}{0}%
  \Test{0000000}{0}%
  \Test{10}{1}%
  \Test{01}{1}%
  \Test{11}{2}%
  \Test{010}{1}%
  \Test{011}{2}%
  \Test{100110011}{5}%
  \Test{0000011111000001111100000}{10}%
  \Test{000000000000000000000000011111111111111111111}{20}%
\end{qstest}

\begin{qstest}{NextClearBit/NextSetBit}{NextClearBit/NextSetBit}
  \def\Test#1#2{%
    \expandafter\expandafter\expandafter\Expect
    \expandafter\expandafter\expandafter{%
      \TestOp{abc}{#1}%
    }{#2}%
  }%
  \def\Clear{\let\TestOp\bitsetNextClearBit}%
  \def\Set{\let\TestOp\bitsetNextSetBit}%
  \begingroup
    \catcode`\:=11 %
    \bitsetSetBin{abc}{1}%
    \Clear
    \Test{-1}{1\BitSetError:NegativeIndex}%
    \Set
    \Test{-1}{0\BitSetError:NegativeIndex}%
  \endgroup
  \let\BS@abc\@undefined
  \Clear
  \Test{0}{0}%
  \Test{1}{1}%
  \Test{2}{2}%
  \Test{100}{100}%
  \Set
  \Test{0}{-1}%
  \Test{1}{-1}%
  \Test{100}{-1}%
  \bitsetReset{abc}%
  \Clear
  \Test{0}{0}%
  \Test{1}{1}%
  \Test{2}{2}%
  \Test{100}{100}%
  \Set
  \Test{0}{-1}%
  \Test{1}{-1}%
  \Test{100}{-1}%
  \bitsetSetBin{abc}{1}%
  \Clear
  \Test{0}{1}%
  \Test{1}{1}%
  \Test{2}{2}%
  \Test{100}{100}%
  \Set
  \Test{0}{0}%
  \Test{1}{-1}%
  \Test{100}{-1}%
  \bitsetSetBin{abc}{111000111000111000111}%
  \Clear
  \Test{0}{3}%
  \Test{1}{3}%
  \Test{2}{3}%
  \Test{3}{3}%
  \Test{4}{4}%
  \Test{5}{5}%
  \Test{6}{9}%
  \Test{7}{9}%
  \Test{8}{9}%
  \Test{9}{9}%
  \Test{10}{10}%
  \Test{11}{11}%
  \Test{12}{15}%
  \Test{13}{15}%
  \Test{14}{15}%
  \Test{15}{15}%
  \Test{16}{16}%
  \Test{17}{17}%
  \Test{18}{21}%
  \Test{19}{21}%
  \Test{20}{21}%
  \Test{21}{21}%
  \Test{22}{22}%
  \Test{100}{100}%
  \Set
  \Test{0}{0}%
  \Test{1}{1}%
  \Test{2}{2}%
  \Test{3}{6}%
  \Test{4}{6}%
  \Test{5}{6}%
  \Test{6}{6}%
  \Test{7}{7}%
  \Test{8}{8}%
  \Test{9}{12}%
  \Test{10}{12}%
  \Test{11}{12}%
  \Test{12}{12}%
  \Test{13}{13}%
  \Test{14}{14}%
  \Test{15}{18}%
  \Test{16}{18}%
  \Test{17}{18}%
  \Test{18}{18}%
  \Test{19}{19}%
  \Test{20}{20}%
  \Test{21}{-1}%
  \Test{22}{-1}%
  \Test{100}{-1}%
  \bitsetSetBin{abc}{1111111}%
  \Clear
  \Test{6}{7}%
  \Test{7}{7}%
  \Test{8}{8}%
  \Test{100}{100}%
  \Set
  \Test{6}{6}%
  \Test{7}{-1}%
  \Test{8}{-1}%
  \Test{100}{-1}%
  \bitsetSetBin{abc}{11111111}%
  \Clear
  \Test{7}{8}%
  \Test{8}{8}%
  \Test{9}{9}%
  \Test{100}{100}%
  \Set
  \Test{7}{7}%
  \Test{8}{-1}%
  \Test{9}{-1}%
  \Test{100}{-1}%
  \bitsetSetBin{abc}{111111111}%
  \Clear
  \Test{8}{9}%
  \Test{9}{9}%
  \Test{10}{10}%
  \Test{100}{100}%
  \Set
  \Test{8}{8}%
  \Test{9}{-1}%
  \Test{10}{-1}%
  \Test{100}{-1}%
  \bitsetSetBin{abc}{1111111111}%
  \Clear
  \Test{9}{10}%
  \Test{10}{10}%
  \Test{11}{11}%
  \Test{100}{100}%
  \Set
  \Test{9}{9}%
  \Test{10}{-1}%
  \Test{11}{-1}%
  \Test{100}{-1}%
\end{qstest}

\begin{qstest}{GetSetBitList}{GetSetBitList}
  \let\BS@abc\@undefined
  \expandafter\expandafter\expandafter\Expect
  \expandafter\expandafter\expandafter{%
    \bitsetGetSetBitList{abc}%
  }{}%
  \def\Test#1#2{%
    \bitsetSetBin{abc}{#1}%
    \expandafter\expandafter\expandafter\Expect
    \expandafter\expandafter\expandafter{%
      \bitsetGetSetBitList{abc}%
    }{#2}%
  }%
  \Test{0}{}%
  \Test{1}{0}%
  \Test{10}{1}%
  \Test{11}{0,1}%
  \Test{10110100}{2,4,5,7}%
  \Test{101101001010011}{0,1,4,6,9,11,12,14}%
\end{qstest}

\begin{qstest}{GetDec}{GetDec}
  \TestGetterUndefined\bitsetGetDec
  \def\Test#1#2{%
    \RevSet{abc}{#1}%
%<*noetex>
    \begingroup\expandafter\expandafter\expandafter\endgroup
%</noetex>
    \expandafter\expandafter\expandafter\Expect
    \expandafter\expandafter\expandafter{%
      \bitsetGetDec{abc}%
    }{#2}%
  }%
  \Test{0}{0}%
  \Test{1}{1}%
  \Test{10}{2}%
  \Test{11}{3}%
  \Test{100}{4}%
  \Test{101}{5}%
  \Test{110}{6}%
  \Test{111}{7}%
  \Test{1000}{8}%
  \Test{000111}{7}%
  \Test{1111111111111111%
        111111111111111}{2147483647}%
  \Test{0001111111111111111%
        111111111111111}{2147483647}%
  \Test{1000000000000000%
        0000000000000000}{2147483648}%
  \Test{1000000000000000%
        00000000000000000}{4294967296}%
  \Test{0001000000000000000%
        00000000000000000}{4294967296}%
  \Test{1100000000000000%
        00000000000000011}{6442450947}%
\end{qstest}

\begin{qstest}{Clear}{Clear}
  \def\Test#1#2#3{%
    \RevSet{abc}{#1}%
    \bitsetClear{abc}{#2}%
    \Expect*{\BS@abc}*{\Reverse#3!!}%
  }%
  \bitsetClear{abc}{2}%
  \RevCheck{abc}{0}%
  \TestErrorNegativeIndex{-1}{\bitsetClear{abc}{-1}}%
  \RevCheck{abc}{0}%
  \Test{0}{0}{0}%
  \Test{1}{0}{0}%
  \Test{111}{1}{101}%
  \Test{111}{30}{111}%
  \Test{0000111}{5}{0000111}% 111 would also be ok
  \Test{10000111}{5}{10000111}%
  \Test{1001001}{3}{1000001}%
\end{qstest}

\begin{qstest}{Set}{Set}
  \def\Test#1#2#3{%
    \RevSet{abc}{#1}%
    \bitsetSet{abc}{#2}%
    \Expect*{\BS@abc}*{\Reverse#3!!}%
  }%
  \bitsetSet{abc}{2}%
  \RevCheck{abc}{100}%
  \TestErrorNegativeIndex{-1}{\bitsetSet{abc}{-1}}%
  \RevCheck{abc}{100}%
  \Test{0}{0}{1}%
  \Test{1}{0}{1}%
  \Test{100}{1}{110}%
  \Test{111}{1}{111}%
  \Test{11}{1}{11}%
  \Test{11}{2}{111}%
  \Test{11}{3}{1011}%
  \Test{111}{10}{10000000111}%
  \Test{0000111}{5}{0100111}% 100111 would also be ok
  \Test{10000111}{5}{10100111}%
  \Test{1000001}{3}{1001001}%
  \Test{1001001}{3}{1001001}%
\end{qstest}

\begin{qstest}{Flip}{Flip}
  \def\Test#1#2#3{%
    \RevSet{abc}{#1}%
    \bitsetFlip{abc}{#2}%
    \Expect*{\BS@abc}*{\Reverse#3!!}%
  }%
  \bitsetFlip{abc}{2}%
  \RevCheck{abc}{100}%
  \TestErrorNegativeIndex{-1}{\bitsetFlip{abc}{-1}}%
  \RevCheck{abc}{100}%
  \Test{0}{0}{1}%
  \Test{1}{0}{0}%
  \Test{0}{2}{100}%
  \Test{100}{1}{110}%
  \Test{111}{1}{101}%
  \Test{11}{1}{1}%
  \Test{11}{2}{111}%
  \Test{11}{3}{1011}%
  \Test{111}{10}{10000000111}%
  \Test{0000111}{5}{0100111}% 100111 would also be ok
  \Test{10000111}{5}{10100111}%
  \Test{1000001}{3}{1001001}%
  \Test{1001001}{3}{1000001}%
  \Test{11111}{2}{11011}%
\end{qstest}

\begin{qstest}{SetValue}{SetValue}
  \def\Test#1#2{%
    \TestError{Invalid bit value (#2) not in range 0..1}{%
      \bitsetSetValue{abc}{#1}{#2}%
    }%
  }%
  \Test{0}{-1}%
  \Test{0}{2}%
  \Test{0}{10}%
  \def\Test#1#2#3{%
    \let\BS@abc\@undefined
    \bitsetSetValue{abc}{#1}{#2}%
    \bitsetSetBin{result}{#3}%
    \Expect*{\BS@abc}*{\BS@result}%
  }%
  \Test{0}{0}{0}%
  \Test{0}{1}{1}%
  \Test{1}{0}{0}%
  \Test{1}{1}{10}%
  \def\Test#1#2#3#4{%
    \bitsetSetBin{abc}{#1}%
    \bitsetSetBin{result}{#4}%
    \bitsetSetValue{abc}{#2}{#3}%
    \Expect*{\BS@abc}*{\BS@result}%
  }%
  \Test{0}{0}{0}{0}%
  \Test{0}{0}{0}{0}%
  \Test{0}{0}{1}{1}%
  \Test{0}{1}{0}{0}%
  \Test{0}{1}{1}{10}%
  \Test{1010}{2}{1}{1110}%
  \Test{1010}{4}{1}{11010}%
  \Test{1010}{6}{1}{1001010}%
  \Test{1010}{1}{0}{1000}%
  \Test{1010}{2}{0}{1010}%
  \Test{1010}{3}{0}{10}%
  \Test{1010}{4}{0}{1010}%
  \Test{1010}{6}{0}{1010}%
  \Test{1010}{2}{\csname iffalse\endcsname 0\else 1\fi}{1110}%
  \Test{1010}{1}{\csname iffalse\endcsname 1\else 0\fi}{1000}%
\end{qstest}

\begin{qstest}{IsDefined}{IsDefined}
  \let\BS@abc\@undefined
  \Expect*{\bitsetIsDefined{abc}{true}{false}}{false}%
  \bitsetReset{abc}%
  \Expect*{\bitsetIsDefined{abc}{true}{false}}{true}%
\end{qstest}

\begin{qstest}{IsEmpty}{IsEmpty}
  \let\BS@abc\@undefined
  \Expect*{\bitsetIsEmpty{abc}{true}{false}}{true}%
  \bitsetReset{abc}%
  \Expect*{\bitsetIsEmpty{abc}{true}{false}}{true}%
  \bitsetSet{abc}{1}%
  \Expect*{\bitsetIsEmpty{abc}{true}{false}}{false}%
\end{qstest}

\begin{qstest}{Equals}{Equals}
  \def\Test#1#2#3{%
    \Expect*{\bitsetEquals{#1}{#2}{true}{false}}{#3}%
  }%
  \let\BS@abc\@undefined
  \Test{abc}{abc}{true}%
  \Test{abc}{foo}{true}%
  \Test{foo}{abc}{true}%
  \bitsetReset{abc}%
  \Test{abc}{abc}{true}%
  \Test{abc}{foo}{false}%
  \Test{foo}{abc}{false}%
  \bitsetReset{foo}%
  \Test{abc}{foo}{true}%
  \Test{foo}{abc}{true}%
  \bitsetSet{abc}{4}%
  \Test{abc}{foo}{false}%
  \Test{foo}{abc}{false}%
  \bitsetFlip{foo}{4}%
  \Test{abc}{foo}{true}%
  \Test{foo}{abc}{true}%
\end{qstest}

\begin{qstest}{Intersects}{Intersects}
  \def\Test#1{%
    \Expect*{\bitsetIntersects{abc}{foo}{true}{false}}{#1}%
  }%
  \let\BS@abc\@undefined
  \let\BS@foo\@undefined
  \Test{false}%
  \Set{abc}{0}%
  \Test{false}%
  \Set{foo}{0}%
  \Test{false}%
  \let\BS@abc\@undefined
  \Test{false}%
  \Set{foo}{1}%
  \Test{false}%
  \Set{abc}{0}%
  \Test{false}%
  \Set{abc}{1}%
  \Test{true}%
  \let\BS@foo\@undefined
  \Test{false}%
  \Set{foo}{0}%
  \Test{false}%
  \def\Test#1#2#3{%
    \bitsetSetBin{abc}{#1}%
    \bitsetSetBin{foo}{#2}%
    \Expect*{\bitsetIntersects{abc}{foo}{true}{false}}{#3}%
  }%
  \Test{1010}{0101}{false}%
  \Test{0}{10}{false}%
  \Test{1}{11}{true}%
  \Test{11}{1}{true}%
  \Test{10}{1}{false}%
\end{qstest}

\begin{qstest}{And/AndNot/Or/Xor}{And/AndNot/Or/Xor}
  \def\@Test#1#2#3#4#5{%
    \begingroup
      #5%
      \begingroup
        \let\BS@foo\@undefined
        \csname bitset#1\endcsname{abc}{foo}%
        \CheckUndef{foo}%
        \Check{abc}{#2}%
      \endgroup
      \begingroup
        \bitsetReset{foo}%
        \csname bitset#1\endcsname{abc}{foo}%
        \Check{foo}{0}%
        \Check{abc}{#3}%
      \endgroup
      \begingroup
        \def\BS@foo{0101}%
        \csname bitset#1\endcsname{abc}{foo}%
        \Check{foo}{0101}%
        \Check{abc}{#4}%
      \endgroup
    \endgroup
  }%
  \def\Test#1{%
    \def\Op{#1}%
    \Test@
  }%
  \def\Test@#1#2#3#4#5#6#7#8#9{%
    \@Test\Op{#1}{#2}{#3}{%
      \let\BS@abc\@undefined
    }%
    \@Test\Op{#4}{#5}{#6}{%
      \bitsetReset{abc}%
    }%
    \@Test\Op{#7}{#8}{#9}{%
      \def\BS@abc{1001}%
    }%
  }%
  \Test{And}%
       {0}{0}{0}%
       {0}{0}{0}%
       {0}{0}{0001}%
  \Test{AndNot}%
       {0}{0}{0}%
       {0}{0}{0}%
       {1001}{1001}{1}%
  \Test{Or}%
       {0}{0}{0101}%
       {0}{0}{0101}%
       {1001}{1001}{1101}%
  \Test{Xor}%
       {0}{0}{0101}%
       {0}{0}{0101}%
       {1001}{1001}{11}%
  \def\Test#1#2#3{%
    \bitsetSetBin{abc}{#1}%
    \bitsetSetBin{foo}{#2}%
    \csname bitset\Op\endcsname{abc}{foo}%
    \RevCheck{foo}{#2}%
    \RevCheck{abc}{#3}%
  }%
  \def\Op{And}%
  \Test{1}{111}{1}%
  \Test{111}{1}{1}%
  \Test{10}{111}{10}%
  \Test{111}{10}{10}%
  \Test{111}{1000}{0}%
  \Test{1000}{111}{0}%
  \def\Op{AndNot}%
  \Test{1010}{11}{1000}%
  \Test{100}{100}{0}%
  \Test{111}{1111}{0}%
  \Test{100}{111}{0}%
  \def\Op{Or}%
  \Test{0}{0}{0}%
  \Test{1}{0}{1}%
  \Test{0}{1}{1}%
  \Test{1}{1}{1}%
  \Test{1000}{10}{1010}%
  \Test{10}{1000}{1010}%
  \def\Op{Xor}%
  \Test{0}{0}{0}%
  \Test{1}{0}{1}%
  \Test{0}{1}{1}%
  \Test{1}{1}{0}%
  \Test{1000}{10}{1010}%
  \Test{10}{1000}{1010}%
  \Test   {110011001100}%
       {111000111000111}%
       {111110100001011}%
  \Test{111000111000111}%
          {110011001100}%
       {111110100001011}%
\end{qstest}

\begin{qstest}{GetUndef}{GetUndef, GetBin, GetOct, GetHex}
  \def\TestUndef#1#2{%
    \let\BS@abc\@undefined
    \expandafter\expandafter\expandafter\Expect
    \expandafter\expandafter\expandafter{%
      \x{abc}{#1}%
    }{#2}%
  }%
  \let\x\bitsetGetBin
  \TestUndef{-1}{0}%
  \TestUndef{0}{0}%
  \TestUndef{1}{0}%
  \TestUndef{2}{00}%
  \TestUndef{8}{00000000}%
  \let\x\bitsetGetOct
  \TestUndef{-1}{0}%
  \TestUndef{0}{0}%
  \TestUndef{1}{0}%
  \TestUndef{2}{0}%
  \TestUndef{3}{0}%
  \TestUndef{4}{00}%
  \TestUndef{5}{00}%
  \TestUndef{6}{00}%
  \TestUndef{7}{000}%
  \TestUndef{8}{000}%
  \TestUndef{9}{000}%
  \TestUndef{10}{0000}%
  \let\x\bitsetGetHex
  \TestUndef{-1}{0}%
  \TestUndef{0}{0}%
  \TestUndef{1}{0}%
  \TestUndef{2}{0}%
  \TestUndef{3}{0}%
  \TestUndef{4}{0}%
  \TestUndef{5}{00}%
  \TestUndef{6}{00}%
  \TestUndef{7}{00}%
  \TestUndef{8}{00}%
  \TestUndef{9}{000}%
  \TestUndef{10}{000}%
  \TestUndef{12}{000}%
  \TestUndef{13}{0000}%
  \TestUndef{16}{0000}%
  \TestUndef{17}{00000}%
\end{qstest}

\begin{qstest}{SetBin}{SetBin}
  \def\Test#1#2{%
    \let\BS@abc\@undefined
    \bitsetSetBin{abc}{#1}%
    \expandafter\Expect\expandafter{\BS@abc}{#2}%
  }%
  \Test{}{0}%
  \Test{0}{0}%
  \Test{1}{1}%
  \Test{10}{01}%
  \Test{11}{11}%
  \Test{010}{01}%
  \Test{011}{11}%
  \Test{0010}{01}%
  \Test{1010}{0101}%
\end{qstest}

\begin{qstest}{SetOct}{SetOct}
  \def\Test#1#2{%
    \bitsetSetOct{abc}{#1}%
    \expandafter\Expect\expandafter{\BS@abc}{#2}%
  }%
  \Test{}{0}%
  \Test{0}{0}%
  \Test{000}{0}%
  \Test{1}{1}%
  \Test{001}{1}%
  \Test{010}{0001}%
  \Test{020}{00001}%
  \Test{42}{010001}%
  \Test{377}{11111111}%
  \Test{0377}{11111111}%
  \Test{76543210}{000100010110001101011111}%
  \Test{ 0 7 0 7 1 }{100111000111}%
\end{qstest}

\begin{qstest}{SetHex}{SetHex}
  \def\Test#1#2{%
    \bitsetSetHex{abc}{#1}%
    \expandafter\Expect\expandafter{\BS@abc}{#2}%
  }%
  \Test{}{0}%
  \Test{0}{0}%
  \Test{000}{0}%
  \Test{1}{1}%
  \Test{001}{1}%
  \Test{010}{00001}%
  \Test{020}{000001}%
  \Test{42}{0100001}%
  \Test{3F}{111111}%
  \Test{03F}{111111}%
  \Test{43210}{0000100001001100001}%
  \Test{98765}{10100110111000011001}%
  \Test{FEDCBA}{010111010011101101111111}%
  \Test{ 0 F 0 F 1 }{1000111100001111}%
\end{qstest}

\begin{qstest}{SetDec}{SetDec}
  \def\Test#1#2{%
    \bitsetSetDec{abc}{#1}%
    \expandafter\Expect\expandafter{\BS@abc}{#2}%
  }%
  \Test{}{0}%
  \Test{0}{0}%
  \Test{000}{0}%
  \Test{1}{1}%
  \Test{7}{111}%
  \Test{8}{0001}%
  \Test{001}{1}%
  \Test{010}{0101}%
  \Test{020}{00101}%
  \Test{53}{101011}%
  \Test{255}{11111111}%
  \Test{256}{000000001}%
  \Test{999999999}{111111111001001101011001110111}%
  \Test{1000000000}{000000000101001101011001110111}%
  \Test{4210987654}{01100001010010010111111101011111}%
  \Test{2147483647}{1111111111111111111111111111111}%
  \Test{2147483648}{00000000000000000000000000000001}%
\end{qstest}

\begin{qstest}{GetBin}{GetBin}
  \def\TestUndef#1#2{%
    \let\BS@abc\@undefined
    \expandafter\expandafter\expandafter\Expect
    \expandafter\expandafter\expandafter{%
      \bitsetGetBin{abc}{#1}%
    }{#2}%
  }%
  \TestUndef{-1}{0}%
  \TestUndef{0}{0}%
  \TestUndef{1}{0}%
  \TestUndef{2}{00}%
  \TestUndef{8}{00000000}%
  \def\Test#1#2{%
    \bitsetSetBin{abc}{#2}%
    \expandafter\expandafter\expandafter\Expect
    \expandafter\expandafter\expandafter{%
      \bitsetGetBin{abc}{#1}%
    }{#2}%
  }%
  \Test{-1}{0}%
  \Test{0}{0}%
  \Test{1}{0}%
  \Test{1}{1}%
  \Test{2}{01}%
  \Test{2}{10}%
  \Test{3}{010}%
  \Test{2}{00}%
  \Test{2}{01}%
  \Test{8}{00101100}%
  \Test{2}{10101}%
  \Test{-100}{11011}%
\end{qstest}

\begin{qstest}{GetOct}{GetOct}
  \def\Test#1#2#3{%
    \edef\x{\zap@space#1 \@empty}%
    \edef\x{\noexpand\bitsetSetBin{abc}{\x}}%
    \x
    \expandafter\expandafter\expandafter\Expect
    \expandafter\expandafter\expandafter{%
      \bitsetGetOct{abc}{#2}%
    }{#3}%
  }%
  \Test{111 110 101 100 011 010 001 000}{0}{76543210}%
  \Test{000 111}{0}{7}%
  \Test{101 000}{-1}{50}%
  \Test{111}{-1}{7}%
  \Test{111}{0}{7}%
  \Test{111}{1}{7}%
  \Test{111}{3}{7}%
  \Test{111}{4}{07}%
  \Test{111}{6}{07}%
  \Test{111}{7}{007}%
  \Test{111 010}{6}{72}%
  \Test{111 010}{7}{072}%
  \Test{011 111}{0}{37}%
  \Test{011 111}{6}{37}%
  \Test{011 111}{7}{037}%
  \Test{001 111}{0}{17}%
  \Test{001 111}{6}{17}%
  \Test{001 111}{7}{017}%
\end{qstest}

\begin{qstest}{GetHex}{GetHex}
  \def\Test#1#2#3{%
    \bitsetSetBin{abc}{#1}%
    \expandafter\expandafter\expandafter\Expect
    \expandafter\expandafter\expandafter{%
      \bitsetGetHex{abc}{#2}%
    }{#3}%
  }%
  \Test{1111 1110 1101 1100 1011 1010 1001 1000}{0}{FEDCBA98}%
  \Test{0111 0110 0101 0100 0011 0010 0001 0000}{0}{76543210}%
  \Test{0000 1111}{0}{F}%
  \Test{0101 0000}{-1}{50}%
  \Test{1111}{-1}{F}%
  \Test{1111}{0}{F}%
  \Test{1111}{1}{F}%
  \Test{1111}{4}{F}%
  \Test{1111}{5}{0F}%
  \Test{1111}{8}{0F}%
  \Test{1111}{9}{00F}%
  \Test{1111 0010}{8}{F2}%
  \Test{1111 0010}{9}{0F2}%
  \Test{0111 1111}{0}{7F}%
  \Test{0111 1111}{8}{7F}%
  \Test{0111 1111}{9}{07F}%
  \Test{0011 1111}{0}{3F}%
  \Test{0011 1111}{8}{3F}%
  \Test{0011 1111}{9}{03F}%
  \Test{0001 1111}{0}{1F}%
  \Test{0001 1111}{8}{1F}%
  \Test{0001 1111}{9}{01F}%
\end{qstest}

\begin{qstest}{Range}{Range}
  \TestError{%
    Wrong index numbers in range [9..8]\MessageBreak% hash-ok
    for clear/set/flip on bit set `abc'.\MessageBreak
    The lower index exceeds the upper index.\MessageBreak
    Canceling the operation as error recovery%
  }{%
    \bitsetSetRange{abc}{9}{8}%
  }%
  \def\TestErrorNegInd#1#2#3#4#5#6{%
    \TestError{%
      Negative index in range [#2..#3]\MessageBreak % hash-ok
      for \string\bitset #1Range on bit set `abc'.\MessageBreak
      Using [#4..#5] as error recovery% hash-ok
    }{%
      \csname bitset#1Range\endcsname{abc}{#2}{#3}%
      \global\let\BS@global\BS@abc
    }%
    \Check{global}{#6}%
  }%
  \Set{abc}{111}%
  \TestErrorNegInd{Clear}{-1}{0}{0}{0}{111}%
  \TestErrorNegInd{Clear}{0}{-1}{0}{0}{111}%
  \TestErrorNegInd{Clear}{-2}{2}{0}{2}{001}%
  \bitsetReset{abc}%
  \TestErrorNegInd{Set}{-1}{0}{0}{0}{0}%
  \TestErrorNegInd{Set}{0}{-1}{0}{0}{0}%
  \TestErrorNegInd{Set}{-2}{2}{0}{2}{11}%
  \Set{abc}{101}%
  \TestErrorNegInd{Flip}{-1}{0}{0}{0}{101}%
  \TestErrorNegInd{Flip}{0}{-1}{0}{0}{101}%
  \TestErrorNegInd{Flip}{-2}{2}{0}{2}{011}%
  \def\Test#1#2#3#4{%
    \bitsetSetBin{abc}{#1}%
    \csname bitset\TestOp Range\endcsname{abc}{#2}{#3}%
    \Expect*{\bitsetGetBin{abc}{0}}{#4}%
  }%
  \def\TestOp{Clear}%
  \Test{0}{0}{1}{0}%
  \Test{1111}{1}{2}{1101}%
  \Test{1111}{1}{3}{1001}%
  \Test{1111111100000000}{12}{14}{1100111100000000}%
  \def\TestOp{Set}%
  \Test{0}{0}{1}{1}%
  \Test{1000}{1}{2}{1010}%
  \Test{0}{1}{2}{10}%
  \Test{1}{12}{15}{111000000000001}%
  \Test{1111}{1}{3}{1111}%
  \Test{1000000000000000}{12}{14}{1011000000000000}%
  \def\TestOp{Flip}%
  \Test{0}{0}{1}{1}%
  \Test{1}{0}{1}{0}%
  \Test{10101010}{1}{5}{10110100}%
  \def\Test#1#2#3#4#5{%
    \bitsetSetBin{abc}{#1}%
    \bitsetSetValueRange{abc}{#2}{#3}{#4}%
    \Expect*{\bitsetGetBin{abc}{0}}{#5}%
  }%
  \Test{0}{0}{1}{0}{0}%
  \Test{0}{0}{1}{1}{1}%
  \Test{1010}{1}{3}{0}{1000}%
  \Test{1010}{1}{3}{1}{1110}%
\end{qstest}

\begin{qstest}{ShiftLeft/ShiftRight}{ShiftLeft/ShiftRight}
  \def\@Test#1#2{%
    \let\BS@abc\@undefined
    \csname bitsetShift#1\endcsname{abc}{#2}%
    \Expect*{\BS@abc}{0}%
  }%
  \def\Test#1{%
    \@Test{Left}{#1}%
    \@Test{Right}{#1}%
  }%
  \Test{-16}%
  \Test{-1}%
  \Test{0}%
  \Test{1}%
  \Test{16}%
  \def\Test#1#2#3{%
    \bitsetSetBin{abc}{#1}%
    \bitsetSetBin{result}{#3}%
    \csname bitsetShift\Op\endcsname{abc}{#2}%
    \Expect*{\bitsetGetBin{abc}{0}}*{\bitsetGetBin{result}{0}}%
  }%
  \def\Op{Left}%
  \Test{0}{0}{0}%
  \Test{0}{1}{0}%
  \Test{0}{-1}{0}%
  \Test{1}{0}{1}%
  \Test{1}{1}{10}%
  \Test{1}{-1}{0}%
  \Test{10}{1}{100}%
  \Test{10}{-1}{1}%
  \Test{1}{32}{100000000000000000000000000000000}%
  \Test{1}{-100}{0}%
  \def\Op{Right}%
  \Test{0}{0}{0}%
  \Test{0}{1}{0}%
  \Test{0}{-1}{0}%
  \Test{1}{0}{1}%
  \Test{1}{1}{0}%
  \Test{1}{-1}{10}%
  \Test{10}{1}{1}%
  \Test{10}{-1}{100}%
  \Test{1}{-32}{100000000000000000000000000000000}%
  \Test{1}{100}{0}%
  \Test{110110110110110}{10}{11011}%
  \Test{110110110110110}{100}{0}%
  \Test{1}{100000}{0}%
\end{qstest}

\begin{qstest}{Profile: Set}{Profile: Set}
  \bitsetSet{abc}{4095}%
  \global\let\BS@global\BS@abc
\end{qstest}

\begin{qstest}{Profile: Get}{Profile: Get}
  \edef\x{\bitsetGet{global}{4095}}%
\end{qstest}

\begin{document}
\end{document}
%</test2>
%    \end{macrocode}
%
% \section{Installation}
%
% \subsection{Download}
%
% \paragraph{Package.} This package is available on
% CTAN\footnote{\url{ftp://ftp.ctan.org/tex-archive/}}:
% \begin{description}
% \item[\CTAN{macros/latex/contrib/oberdiek/bitset.dtx}] The source file.
% \item[\CTAN{macros/latex/contrib/oberdiek/bitset.pdf}] Documentation.
% \end{description}
%
%
% \paragraph{Bundle.} All the packages of the bundle `oberdiek'
% are also available in a TDS compliant ZIP archive. There
% the packages are already unpacked and the documentation files
% are generated. The files and directories obey the TDS standard.
% \begin{description}
% \item[\CTAN{install/macros/latex/contrib/oberdiek.tds.zip}]
% \end{description}
% \emph{TDS} refers to the standard ``A Directory Structure
% for \TeX\ Files'' (\CTAN{tds/tds.pdf}). Directories
% with \xfile{texmf} in their name are usually organized this way.
%
% \subsection{Bundle installation}
%
% \paragraph{Unpacking.} Unpack the \xfile{oberdiek.tds.zip} in the
% TDS tree (also known as \xfile{texmf} tree) of your choice.
% Example (linux):
% \begin{quote}
%   |unzip oberdiek.tds.zip -d ~/texmf|
% \end{quote}
%
% \paragraph{Script installation.}
% Check the directory \xfile{TDS:scripts/oberdiek/} for
% scripts that need further installation steps.
% Package \xpackage{attachfile2} comes with the Perl script
% \xfile{pdfatfi.pl} that should be installed in such a way
% that it can be called as \texttt{pdfatfi}.
% Example (linux):
% \begin{quote}
%   |chmod +x scripts/oberdiek/pdfatfi.pl|\\
%   |cp scripts/oberdiek/pdfatfi.pl /usr/local/bin/|
% \end{quote}
%
% \subsection{Package installation}
%
% \paragraph{Unpacking.} The \xfile{.dtx} file is a self-extracting
% \docstrip\ archive. The files are extracted by running the
% \xfile{.dtx} through \plainTeX:
% \begin{quote}
%   \verb|tex bitset.dtx|
% \end{quote}
%
% \paragraph{TDS.} Now the different files must be moved into
% the different directories in your installation TDS tree
% (also known as \xfile{texmf} tree):
% \begin{quote}
% \def\t{^^A
% \begin{tabular}{@{}>{\ttfamily}l@{ $\rightarrow$ }>{\ttfamily}l@{}}
%   bitset.sty & tex/generic/oberdiek/bitset.sty\\
%   bitset.pdf & doc/latex/oberdiek/bitset.pdf\\
%   test/bitset-test1.tex & doc/latex/oberdiek/test/bitset-test1.tex\\
%   test/bitset-test2.tex & doc/latex/oberdiek/test/bitset-test2.tex\\
%   test/bitset-test3.tex & doc/latex/oberdiek/test/bitset-test3.tex\\
%   bitset.dtx & source/latex/oberdiek/bitset.dtx\\
% \end{tabular}^^A
% }^^A
% \sbox0{\t}^^A
% \ifdim\wd0>\linewidth
%   \begingroup
%     \advance\linewidth by\leftmargin
%     \advance\linewidth by\rightmargin
%   \edef\x{\endgroup
%     \def\noexpand\lw{\the\linewidth}^^A
%   }\x
%   \def\lwbox{^^A
%     \leavevmode
%     \hbox to \linewidth{^^A
%       \kern-\leftmargin\relax
%       \hss
%       \usebox0
%       \hss
%       \kern-\rightmargin\relax
%     }^^A
%   }^^A
%   \ifdim\wd0>\lw
%     \sbox0{\small\t}^^A
%     \ifdim\wd0>\linewidth
%       \ifdim\wd0>\lw
%         \sbox0{\footnotesize\t}^^A
%         \ifdim\wd0>\linewidth
%           \ifdim\wd0>\lw
%             \sbox0{\scriptsize\t}^^A
%             \ifdim\wd0>\linewidth
%               \ifdim\wd0>\lw
%                 \sbox0{\tiny\t}^^A
%                 \ifdim\wd0>\linewidth
%                   \lwbox
%                 \else
%                   \usebox0
%                 \fi
%               \else
%                 \lwbox
%               \fi
%             \else
%               \usebox0
%             \fi
%           \else
%             \lwbox
%           \fi
%         \else
%           \usebox0
%         \fi
%       \else
%         \lwbox
%       \fi
%     \else
%       \usebox0
%     \fi
%   \else
%     \lwbox
%   \fi
% \else
%   \usebox0
% \fi
% \end{quote}
% If you have a \xfile{docstrip.cfg} that configures and enables \docstrip's
% TDS installing feature, then some files can already be in the right
% place, see the documentation of \docstrip.
%
% \subsection{Refresh file name databases}
%
% If your \TeX~distribution
% (\teTeX, \mikTeX, \dots) relies on file name databases, you must refresh
% these. For example, \teTeX\ users run \verb|texhash| or
% \verb|mktexlsr|.
%
% \subsection{Some details for the interested}
%
% \paragraph{Attached source.}
%
% The PDF documentation on CTAN also includes the
% \xfile{.dtx} source file. It can be extracted by
% AcrobatReader 6 or higher. Another option is \textsf{pdftk},
% e.g. unpack the file into the current directory:
% \begin{quote}
%   \verb|pdftk bitset.pdf unpack_files output .|
% \end{quote}
%
% \paragraph{Unpacking with \LaTeX.}
% The \xfile{.dtx} chooses its action depending on the format:
% \begin{description}
% \item[\plainTeX:] Run \docstrip\ and extract the files.
% \item[\LaTeX:] Generate the documentation.
% \end{description}
% If you insist on using \LaTeX\ for \docstrip\ (really,
% \docstrip\ does not need \LaTeX), then inform the autodetect routine
% about your intention:
% \begin{quote}
%   \verb|latex \let\install=y% \iffalse meta-comment
%
% File: bitset.dtx
% Version: 2011/01/30 v1.1
% Info: Handle bit-vector datatype
%
% Copyright (C) 2007, 2011 by
%    Heiko Oberdiek <heiko.oberdiek at googlemail.com>
%
% This work may be distributed and/or modified under the
% conditions of the LaTeX Project Public License, either
% version 1.3c of this license or (at your option) any later
% version. This version of this license is in
%    http://www.latex-project.org/lppl/lppl-1-3c.txt
% and the latest version of this license is in
%    http://www.latex-project.org/lppl.txt
% and version 1.3 or later is part of all distributions of
% LaTeX version 2005/12/01 or later.
%
% This work has the LPPL maintenance status "maintained".
%
% This Current Maintainer of this work is Heiko Oberdiek.
%
% The Base Interpreter refers to any `TeX-Format',
% because some files are installed in TDS:tex/generic//.
%
% This work consists of the main source file bitset.dtx
% and the derived files
%    bitset.sty, bitset.pdf, bitset.ins, bitset.drv, bitset-test1.tex,
%    bitset-test2.tex, bitset-test3.tex.
%
% Distribution:
%    CTAN:macros/latex/contrib/oberdiek/bitset.dtx
%    CTAN:macros/latex/contrib/oberdiek/bitset.pdf
%
% Unpacking:
%    (a) If bitset.ins is present:
%           tex bitset.ins
%    (b) Without bitset.ins:
%           tex bitset.dtx
%    (c) If you insist on using LaTeX
%           latex \let\install=y% \iffalse meta-comment
%
% File: bitset.dtx
% Version: 2011/01/30 v1.1
% Info: Handle bit-vector datatype
%
% Copyright (C) 2007, 2011 by
%    Heiko Oberdiek <heiko.oberdiek at googlemail.com>
%
% This work may be distributed and/or modified under the
% conditions of the LaTeX Project Public License, either
% version 1.3c of this license or (at your option) any later
% version. This version of this license is in
%    http://www.latex-project.org/lppl/lppl-1-3c.txt
% and the latest version of this license is in
%    http://www.latex-project.org/lppl.txt
% and version 1.3 or later is part of all distributions of
% LaTeX version 2005/12/01 or later.
%
% This work has the LPPL maintenance status "maintained".
%
% This Current Maintainer of this work is Heiko Oberdiek.
%
% The Base Interpreter refers to any `TeX-Format',
% because some files are installed in TDS:tex/generic//.
%
% This work consists of the main source file bitset.dtx
% and the derived files
%    bitset.sty, bitset.pdf, bitset.ins, bitset.drv, bitset-test1.tex,
%    bitset-test2.tex, bitset-test3.tex.
%
% Distribution:
%    CTAN:macros/latex/contrib/oberdiek/bitset.dtx
%    CTAN:macros/latex/contrib/oberdiek/bitset.pdf
%
% Unpacking:
%    (a) If bitset.ins is present:
%           tex bitset.ins
%    (b) Without bitset.ins:
%           tex bitset.dtx
%    (c) If you insist on using LaTeX
%           latex \let\install=y\input{bitset.dtx}
%        (quote the arguments according to the demands of your shell)
%
% Documentation:
%    (a) If bitset.drv is present:
%           latex bitset.drv
%    (b) Without bitset.drv:
%           latex bitset.dtx; ...
%    The class ltxdoc loads the configuration file ltxdoc.cfg
%    if available. Here you can specify further options, e.g.
%    use A4 as paper format:
%       \PassOptionsToClass{a4paper}{article}
%
%    Programm calls to get the documentation (example):
%       pdflatex bitset.dtx
%       makeindex -s gind.ist bitset.idx
%       pdflatex bitset.dtx
%       makeindex -s gind.ist bitset.idx
%       pdflatex bitset.dtx
%
% Installation:
%    TDS:tex/generic/oberdiek/bitset.sty
%    TDS:doc/latex/oberdiek/bitset.pdf
%    TDS:doc/latex/oberdiek/test/bitset-test1.tex
%    TDS:doc/latex/oberdiek/test/bitset-test2.tex
%    TDS:doc/latex/oberdiek/test/bitset-test3.tex
%    TDS:source/latex/oberdiek/bitset.dtx
%
%<*ignore>
\begingroup
  \catcode123=1 %
  \catcode125=2 %
  \def\x{LaTeX2e}%
\expandafter\endgroup
\ifcase 0\ifx\install y1\fi\expandafter
         \ifx\csname processbatchFile\endcsname\relax\else1\fi
         \ifx\fmtname\x\else 1\fi\relax
\else\csname fi\endcsname
%</ignore>
%<*install>
\input docstrip.tex
\Msg{************************************************************************}
\Msg{* Installation}
\Msg{* Package: bitset 2011/01/30 v1.1 Handle bit-vector datatype (HO)}
\Msg{************************************************************************}

\keepsilent
\askforoverwritefalse

\let\MetaPrefix\relax
\preamble

This is a generated file.

Project: bitset
Version: 2011/01/30 v1.1

Copyright (C) 2007, 2011 by
   Heiko Oberdiek <heiko.oberdiek at googlemail.com>

This work may be distributed and/or modified under the
conditions of the LaTeX Project Public License, either
version 1.3c of this license or (at your option) any later
version. This version of this license is in
   http://www.latex-project.org/lppl/lppl-1-3c.txt
and the latest version of this license is in
   http://www.latex-project.org/lppl.txt
and version 1.3 or later is part of all distributions of
LaTeX version 2005/12/01 or later.

This work has the LPPL maintenance status "maintained".

This Current Maintainer of this work is Heiko Oberdiek.

The Base Interpreter refers to any `TeX-Format',
because some files are installed in TDS:tex/generic//.

This work consists of the main source file bitset.dtx
and the derived files
   bitset.sty, bitset.pdf, bitset.ins, bitset.drv, bitset-test1.tex,
   bitset-test2.tex, bitset-test3.tex.

\endpreamble
\let\MetaPrefix\DoubleperCent

\generate{%
  \file{bitset.ins}{\from{bitset.dtx}{install}}%
  \file{bitset.drv}{\from{bitset.dtx}{driver}}%
  \usedir{tex/generic/oberdiek}%
  \file{bitset.sty}{\from{bitset.dtx}{package}}%
  \usedir{doc/latex/oberdiek/test}%
  \file{bitset-test1.tex}{\from{bitset.dtx}{test1}}%
  \file{bitset-test2.tex}{\from{bitset.dtx}{test2,etex}}%
  \file{bitset-test3.tex}{\from{bitset.dtx}{test2,noetex}}%
  \nopreamble
  \nopostamble
  \usedir{source/latex/oberdiek/catalogue}%
  \file{bitset.xml}{\from{bitset.dtx}{catalogue}}%
}

\catcode32=13\relax% active space
\let =\space%
\Msg{************************************************************************}
\Msg{*}
\Msg{* To finish the installation you have to move the following}
\Msg{* file into a directory searched by TeX:}
\Msg{*}
\Msg{*     bitset.sty}
\Msg{*}
\Msg{* To produce the documentation run the file `bitset.drv'}
\Msg{* through LaTeX.}
\Msg{*}
\Msg{* Happy TeXing!}
\Msg{*}
\Msg{************************************************************************}

\endbatchfile
%</install>
%<*ignore>
\fi
%</ignore>
%<*driver>
\NeedsTeXFormat{LaTeX2e}
\ProvidesFile{bitset.drv}%
  [2011/01/30 v1.1 Handle bit-vector datatype (HO)]%
\documentclass{ltxdoc}
\usepackage{holtxdoc}[2011/11/22]
\usepackage{amstext}
\usepackage{ifluatex}
\ifluatex
\else
  \usepackage[T1]{fontenc}%
  \usepackage{lmodern}%
\fi
\begin{document}
  \DocInput{bitset.dtx}%
\end{document}
%</driver>
% \fi
%
% \CheckSum{3510}
%
% \CharacterTable
%  {Upper-case    \A\B\C\D\E\F\G\H\I\J\K\L\M\N\O\P\Q\R\S\T\U\V\W\X\Y\Z
%   Lower-case    \a\b\c\d\e\f\g\h\i\j\k\l\m\n\o\p\q\r\s\t\u\v\w\x\y\z
%   Digits        \0\1\2\3\4\5\6\7\8\9
%   Exclamation   \!     Double quote  \"     Hash (number) \#
%   Dollar        \$     Percent       \%     Ampersand     \&
%   Acute accent  \'     Left paren    \(     Right paren   \)
%   Asterisk      \*     Plus          \+     Comma         \,
%   Minus         \-     Point         \.     Solidus       \/
%   Colon         \:     Semicolon     \;     Less than     \<
%   Equals        \=     Greater than  \>     Question mark \?
%   Commercial at \@     Left bracket  \[     Backslash     \\
%   Right bracket \]     Circumflex    \^     Underscore    \_
%   Grave accent  \`     Left brace    \{     Vertical bar  \|
%   Right brace   \}     Tilde         \~}
%
% \GetFileInfo{bitset.drv}
%
% \title{The \xpackage{bitset} package}
% \date{2011/01/30 v1.1}
% \author{Heiko Oberdiek\\\xemail{heiko.oberdiek at googlemail.com}}
%
% \maketitle
%
% \begin{abstract}
% This package defines and implements the data type bit set,
% a vector of bits. The size of the vector may grow dynamically.
% Individual bits can be manipulated.
% \end{abstract}
%
% \tableofcontents
%
% \clearpage
% \section{Documentation}
%
% \subsection{Introduction}
%
% Annotations in the PDF format know entries whose values are
% integers. This numbers are interpreted as set of flags
% specifying properties. For example, annotation dictionaries
% can have a key |/F|. The bits of its integer value
% are interpreted the following way:
% \begin{quote}
% \begin{tabular}{@{}l|l@{}}
%   Bit position & Property name\\
%   \hline
%   1 & Invisible\\
%   2 & Hidden\\
%   3 & Print\\
%   4 & NoZoom\\
%   5 & NoRotate\\
%   6 & NoView\\
%   7 & ReadOnly\\
%   \dots & \dots
% \end{tabular}
% \end{quote}
% Now, let's see how these values are set in package \xpackage{hyperref}
% before it uses this package (before v6.77a):
%\begin{quote}
%\begin{verbatim}
%\ifFld@hidden /F 6\else /F 4\fi
%\end{verbatim}
%\end{quote}
% Where are the other flags? The following example for key |/Ff|
% in a widget annotation
% supports at least three properties:
%\begin{quote}
%\begin{verbatim}
%\ifFld@multiline
%  \ifFld@readonly /Ff 4097\else /Ff 4096\fi
%\else
%  \ifFld@password
%    \ifFld@readonly /Ff 8193\else /Ff 8192\fi
%  \else
%    \ifFld@readonly /Ff 1\fi
%  \fi
%\fi
%\end{verbatim}
%\end{quote}
% But you see the point. It would be a nightmare to continue
% this way in supporting the missing flag settings. This kind
% of integers may have up to 32 bits.
%
% Therefore I wanted a data structure for setting and
% clearing individual bits. Also it should provide
% an export as decimal number. The snipsets above are
% executed in expansion contexts without \TeX's stomach
% commands. It would be convenient to have an expandable
% conversion from the data structure to the integer that
% gets written to the PDF file.
%
% This package \xpackage{bitset} implements such a data
% structure. The interface is quite close to Java's
% class |BitSet| in order not to learn to many interfaces
% for the same kind of data structure.
%
% \subsection{Glossary}
%
% \begin{description}
% \item[Bit set:]
%   A bit set is a vector of bits or flags.
%   The vector size is unlimited and
%   grows dynamically. An undefined bit set is treated as bit
%   set where all bits are cleared.
%
%   Bit sets are addressed by name. A name should consists of
%   letters or digits. Technically it must survive \cs{csname},
%   see \LaTeX's environment names for other names with such
%   a constraint. Package \xpackage{babel}'s shorthands are
%   not supported due to technical reasons. Shorthand support
%   would break expandable operations.
% \item[Size:]
%   A size of a bit set is the number of bits in use.
%   It's the number of the highest index, incremented by one.
%   Sizes are in the range |0| up to |2147483647|, the highest
%   number supported by \TeX.
% \item[Index:]
%   Bit positions in a bit set are addressed by an index number.
%   The bit vector is zero based. The first and least significant
%   bit is addressed by index |0| and the highest possible bit
%   by |2147483646|.
% \item[Bit:]
%   A bit is enoded as |0| for cleared/disabled or |1| for set/enabled.
% \end{description}
%
% \subsection{Design principles}
%
% \begin{description}
% \item[Name conventions:]
%   To avoid conflicts with existing macro names, the operations
%   are prefixed by the package name.
% \item[Zero based indexes:] The first bit is addressed by zero.
%   (Convention of array indexing in C, Java, \dots)
% \item[Unlimited size:]
%   There is no restriction on the size of a bit set other than
%   usual memory limitations. \cs{bitsetSetDec} and \cs{bitsetGetDec}
%   transparently switch to package \xpackage{bigintcalc} if the numbers
%   get too large for \TeX's number limit.
% \item[Expandibility:]
%   Any operation that does not change the bit set is expandable.
%   And all operations that extract or calculate some result
%   do this in exact two expansion steps. For example,
%   a macro \cs{Macro} wants a bit set as decimal number.
%   But the argument must be a plain number without macros.
%   Thus you could prefix \cs{bitsetGetDec} with \cs{number}.
%   However this won't work for bit sets with 31 or more bits
%   because of \TeX's number limit of $2^{31}-1$.
%   then just hit the operator with two \cs{expandafter}:
%   \begin{quote}
%   |\expandafter\expandafter\expandafter|\\
%   |\Macro\bitsetGetDec{foo}|
%   \end{quote}
%   |\bitsetGetDec| is hit first by the third \cs{expandafter}
%   and then by the second one.
% \item[Format independence:] This package is written as \LaTeX\ package,
%   but it does not depend on \LaTeX. It will also work for other formats
%   such as \plainTeX.
% \item[Independence from \TeX\ engines:]
%   Vanilla \TeX\ is all you need. Calculations are delegated to
%   packages \xpackage{intcalc} and \xpackage{bigintcalc}.
%   They don't need any special features, but they will switch
%   to a little more efficient implementation if features such
%   as \cs{numexpr} are available.
% \item[Numeric arguments:] Anything that is accepted by
%   \cs{number}. If \eTeX\ is detected, also expressions
%   for \cs{numexpr} are supported. The only exception so far
%   is the number for \cs{bitsetSetDec}. The number might be too
%   large for \cs{number} or \cs{numexpr}.
% \item[Error messages:]
%   In expandable contexts, only a limited set of \TeX\ primitive
%   commands work as expected. So called stomach commands behave
%   like \cs{relax} and don't get expanded or executed.
%   Unhappily also the error commands belong to this category.
%   The expandable operations will throw an unknown control
%   sequence instead to get \TeX's and user's attention.
%   The name of these control sequences starts with
%   \cs{BitSetError:} with the type of error after the colon.
% \end{description}
%
% \subsection{Operator overview}
%
% \begin{minipage}{\linewidth}
% \begin{description}
% \def\It#1#2{^^A
%   \item[^^A
%     {#1} ^^A
%     \textmd{(\hyperref[{sec:#2}]{section \ref*{sec:#2}})}^^A
%   ]^^A
%   \mbox{}\\^^A
% }
% \def\op#1{^^A
%   \cs{bitset}\texttt{\bfseries#1}^^A
% }
% \def\M#1{\meta{#1}}
% \def\Q{\hspace*{\fill}}
% \It{Miscellaneous}{misc}
%   \op{Reset}\Q \M{BitSet}\\
%   \op{Let}\Q \M{BitSet A} \M{BitSet B}
% \It{Import}{import}
%   \op{SetBin}, \op{SetOct}, \op{SetHex}\Q \M{BitSet} \M{Value}\\
%   \op{SetDec}\Q \M{BitSet} \M{Value}
% \It{Export\textsuperscript{a}}{export}
%   \op{GetBin}, \op{GetOct}, \op{GetHex}\Q \M{BitSet} \M{MinSize}\\
%   \op{GetDec}\Q \M{BitSet}
% \It{Logical operators}{log}
%   \op{And}, \op{AndNot}\Q \M{BitSet A} \M{BitSet B}\\
%   \op{Or}, \op{Xor}\Q \M{BitSet A} \M{BitSet B}
% \It{Shifting}{shift}
%   \op{ShiftLeft}, \op{ShiftRight}\Q \M{BitSet} \M{ShiftAmount}\\
% \It{Bit manipulation}{bitman}
%   \op{Clear}, \op{Set}, \op{Flip}\Q \M{BitSet} \M{Index}\\
%   \op{SetValue}\Q \M{BitSet} \M{Index} \M{Value}\\
%   \op{ClearRange}, \op{SetRange}, \op{FlipRange}\\
%     \Q \M{BitSet} \M{IndexFrom} \M{IndexTo}\\
%   \op{SetValueRange}\Q \M{BitSet} \M{IndexFrom} \M{IndexTo}
% \It{Bit retrieval\textsuperscript{a}}{bitret}
%   \op{Get}\Q \M{BitSet} \M{Index}\\
%   \op{NextClearBit}, \op{NextSetBit}\Q \M{BitSet} \M{Index}\\
%   \op{GetSetBitList}\Q \M{BitSet}
% \It{Bit set properties}{prop}
%   \op{Size}, \op{Cardinality}\Q \M{BitSet}
% \It{Queries\textsuperscript{b}}{query}
%   \op{IsDefined}, \op{IsEmpty}\Q \M{BitSet} \M{Then} \M{Else}\\
%   \op{Equals}, \op{Intersects}\Q \M{BitSet A} \M{BitSet B} \M{Then} \M{Else}\\
%   \op{Query}\Q \M{BitSet} \M{Index} \M{Then} \M{Else}
% \end{description}
% \footnotetext[1]{Macros are expandable, full expansion by two steps.}
% \footnotetext[2]{Macros are expandable.}
% \end{minipage}
%
% \subsection{Package loading}
%
% The package can be used as normal \LaTeX\ package:
%\begin{quote}
%|\usepackage{bitset}|
%\end{quote}
% Also \plainTeX\ is supported:
%\begin{quote}
%|\input bitset.sty\relax|
%\end{quote}
%
% \subsection{Operators}
%
% The following macros work on and with bit sets.
% A bit set \meta{BitSet} is represented by a name.
% The should consist of letters and digits. Technically
% it must survive \cs{csname}. It is the same constraint
% that must be satisfied by label or environment names
% in \LaTeX.
%
% However active characters that are shorthands
% of package \xpackage{babel} are not supported.
% Support for shorthands works by an assignment.
% But many operators
% such as \cs{bitsetGetDec} must be usable in expandable
% contexts. There assignments will not be executed in the
% best case or they will cause errors.
%
% The bits in a bit set are addressed by non-negative
% integers starting from zero. Thus negative index
% numbers cause an error message. Because index numbers
% are \TeX\ numbers. The largest index is 2147483647.
% But in practice memory limits and patience limits
% will be very likely reached much before.
%
% \subsubsection{Miscellaneous}\label{sec:misc}
%
% There isn't a separate operation for bit set creation.
% For simplicity an undefined bit set is treated as bit set
% with all bits cleared.
%
% \begin{declcs}{bitsetReset} \M{BitSet}
% \end{declcs}
% Macro \cs{bitsetReset} clears all bits. The result is an
% empty bit set. It may also be used as replacement for
% an operation ``new'', because an undefined bit set is defined
% afterwards.
%
% \begin{declcs}{bitsetLet} \M{BitSet A} \M{BitSet B}
% \end{declcs}
% Macro \cs{bitsetLet} performs a simple assignment similar
% to \TeX's \cs{let}. After the operation \meta{BitSet A} has
% the same value as \meta{BitSet B}. If \meta{BitSet B} is
% undefined, then \meta{BitSet A} will be the empty bit set.
%
% Note: If \meta{BitSet A} exists, it will be overwritten.
%
% \subsubsection{Import}\label{sec:import}
%
% \begin{declcs}{bitsetSetBin} \M{BitSet} \M{BinaryNumber}\\
%   \cs{bitsetSetOct} \M{BitSet} \M{OctalNumber}\\
%   \cs{bitsetSetHex} \M{BitSet} \M{HexadecimalNumber}
% \end{declcs}
% The numbers are interpreted as bit vectors and the flags in
% the bit \meta{BitSet} set are set accordingly.
% These numeric arguments are the only arguments where
% spaces are allowed. Then the numbers are easier to read.
%
% \begin{declcs}{bitsetSetDec} \M{BitSet} \M{DecimalNumber}
% \end{declcs}
% Macro \cs{bitsetSetDec} uses \meta{DecimalNumber} to
% set the bit set \meta{BitSet}.
% The numeric argument must expand
% to a plain number consisting of decimal digits without
% command tokens or spaces. Internally this argument is
% expanded only. It cannot be passed to \cs{number} or
% \cs{numexpr}, because the number may be too large for them.
% However \cs{number} or |\the\numexpr| may be used
% explicitely. This also helps for unexpandable number
% command tokens or registers (\cs{z@}, \cs{@ne}, \cs{count@}, \dots).
% Also \LaTeX'\ \cs{value} needs prefixing:
% \begin{quote}
% |\bitsetSetDec{foo}{\number\value{bar}}|
% \end{quote}
%
% \subsubsection{Export}\label{sec:export}
%
% \begin{declcs}{bitsetGetBin} \M{BitSet} \M{MinSize}\\
%   \cs{bitsetGetOct} \M{BitSet} \M{MinSize}\\
%   \cs{bitsetGetHex} \M{BitSet} \M{MinSize}
% \end{declcs}
% These macros returns the bit set as binary, octal or hexadecimal
% number. If the bit size is smaller than \meta{MinSize} the gap is
% filled with leading zeros.
% Example:
% \begin{quote}
% |\bitsetReset{abc}|\\
% |\bitsetSet{abc}{2}|\\
% |\bitsetGetBin{abc}{8}| $\rightarrow$ |00000100|\\
% |\bitsetSet{abc}{5}\bitsetSet{abc}{7}|\\
% |\bitsetGetHex{abc}{16}| $\rightarrow$ |00A2|
% \end{quote}
% Macro \cs{bitsetGetHex} uses the uppercase letters |A| to |F|.
% The catcode of the letters is one of 11 (letter) or 12 (other).
%
% \begin{declcs}{bitsetGetDec} \M{BitSet}
% \end{declcs}
% Macro \cs{bitsetGetDec} returns the bit set \meta{BitSet} as
% decimal number. The returned number can be larger than
% \TeX's number limit of $2^{31}-1$.
%
% \subsubsection{Logical operators}\label{sec:log}
%
% \begin{declcs}{bitsetAnd} \M{BitSet A} \M{BitSet B}
% \end{declcs}
% $A_{\text{new}} \mathrel{\mathop{:}}= A_{\text{old}}
% \mathbin{\text{and}} B \qquad (\forall\, \text{bits})$
%
% \begin{declcs}{bitsetAndNot} \M{BitSet A} \M{BitSet B}
% \end{declcs}
% $A_{\text{new}} \mathrel{\mathop{:}}= A_{\text{old}}
% \mathbin{\text{and}} (\text{not}\,B) \qquad (\forall\, \text{bits})$
%
% \begin{declcs}{bitsetOr} \M{BitSet A} \M{BitSet B}
% \end{declcs}
% $A_{\text{new}} \mathrel{\mathop{:}}= A_{\text{old}}
% \mathbin{\text{or}} B \qquad (\forall\, \text{bits})$
%
% \begin{declcs}{bitsetXor} \M{BitSet A} \M{BitSet B}
% \end{declcs}
% $A_{\text{new}} \mathrel{\mathop{:}}= A_{\text{old}}
% \mathbin{\text{xor}} B \qquad (\forall\, \text{bits})$
%
% \subsubsection{Shifting}\label{sec:shift}
%
% \begin{declcs}{bitsetShiftLeft} \M{BitSet} \M{ShiftAmount}\\
%   \cs{bitsetShiftRight} \M{BitSet} \M{ShiftAmount}
% \end{declcs}
% A left shift by one is a multiplication by two, thus left shifting
% moves the flags to higher positions. The new created low positions
% are filled by zeros.
%
% A right shift is the opposite, dividing by two, movint the bits
% to lower positions. The number will become smaller, the lowest
% bits are lost.
%
% If the \meta{ShiftAmount} is negative, it reverts the meaning of
% the shift operation. A left shift becomes a right shift.
% A \meta{ShiftAmount} of zero is ignored.
%
% \subsubsection{Bit manipulation}\label{sec:bitman}
%
% \begin{declcs}{bitsetClear} \M{BitSet} \M{Index}\\
%   \cs{bitsetSet} \M{BitSet} \M{Index}\\
%   \cs{bitsetFlip} \M{BitSet} \M{Index}
% \end{declcs}
% This macros manipulate a single bit in \meta{BitSet} addressed
% by \cs{Index}. Macro \cs{bitsetClear} disables the bit,
% \cs{bitsetSet} enables it and \cs{bitsetFlip} reverts the current
% setting of the bit.
%
% \begin{declcs}{bitsetSetValue} \M{BitSet} \M{Index} \M{Bit}
% \end{declcs}
% Macro \cs{bitsetSetValue} puts bit \meta{Bit} at position \meta{Index}
% in bit set \meta{BitSet}. \meta{Bit} must be a valid \TeX\ number
% equals to zero (disabled/cleared) or one (enabled/set).
%
% \subsubsection{Bit retrieval}\label{sec:bitret}
%
% \begin{declcs}{bitsetGet} \M{BitSet} \M{Index}
% \end{declcs}
% Macro \cs{bitsetGet} extracts the status of the bit at
% position \meta{Index} in bit set \meta{BitSet}.
% Digit |1| is returned if the bit is set/enabled. If the
% bit is cleared/disabled and in cases of an undefined bitset or
% an index number out of range the return value is |0|.
%
% \begin{declcs}{bitsetNextClearBit} \M{BitSet} \M{Index}
% \end{declcs}
% Starting at position \meta{Index} (inclusive) the bits
% are inspected. The first position without a set bit is returned.
% Possible results are decimal numbers: \meta{Index}, \meta{Index} + 1,
% \dots, ($\infty$)
%
% \begin{declcs}{bitsetNextSetBit} \M{BitSet} \M{Index}
% \end{declcs}
% Starting at position \meta{Index} (inclusive) the bits are
% inspected and the index position of the first found set bit
% is returned. If there isn't such a bit, then the result is
% |-1|. In summary possible results are decimal numbers:
% |-1|, \meta{Index}, \meta{Index} + 1, \dots, ($\infty$)
%
% \begin{declcs}{bitsetGetSetBitList} \M{BitSet}
% \end{declcs}
% Macro \cs{bitsetGetSetBitList} is an application for
% \cs{bitsetNextSetBit}. The set bits are iterated and
% returned as comma separated list of index positions
% in increasing order. The list is empty in case of
% an empty bit set.
%
% \subsubsection{Bit set properties}\label{sec:prop}
%
% \begin{declcs}{bitsetSize} \M{BitSet}
% \end{declcs}
% Macro \cs{bitsetSize} returns number of bits in use.
% It is the same as the index number of the highest set/enabled bit
% incremented by one.
%
% \begin{declcs}{bitsetCardinality} \M{BitSet}
% \end{declcs}
% Macro \cs{bitsetCardinality} counts the number of set/enabled bits.
%
% \subsubsection{Queries}\label{sec:query}
%
% Also the query procedures are expandable. They ask for a piece
% of information about a bit set and execute code depending
% on the answer.
%
% \begin{declcs}{bitsetIsDefined} \M{BitSet} \M{Then} \M{Else}
% \end{declcs}
% If the bit set with the name \meta{BitSet} exists the code
% given in \meta{Then} is executed, otherwise \meta{Else} is used.
%
% \begin{declcs}{bitsetIsEmpty} \M{BitSet} \M{Then} \M{Else}
% \end{declcs}
% If the bit set \meta{BitSet} exists and at least one bit
% is set/enabled, the code in \meta{Then} is executed,
% \meta{Else} otherwise.
%
% \begin{declcs}{bitsetEquals} \M{BitSet A} \M{BitSet B} \M{Then} \M{Else}
% \end{declcs}
% Both bit sets are equal if and only if either both are undefined
% or both are defined and represents the same bit values at the
% same positions. Thus this definition is reflexive, symmeric,
% and transitive, enough for an equivalent relation.
%
% \begin{declcs}{bitsetIntersects} \M{BitSet A} \M{BitSet B} \M{Then} \M{Else}
% \end{declcs}
% If and only if \meta{BitSet A} and \meta{BitSet B} have
% at least one bit at the same position that is set, then
% code part \meta{Then} is executed.
%
% \begin{declcs}{bitsetQuery} \M{BitSet} \M{Index} \M{Then} \M{Else}
% \end{declcs}
% It's just a wrapper for \cs{bitsetGet}. If the bit at position
% \meta{Index} is enabled, code \meta{Then} is called.
%
% \StopEventually{
% }
%
% \section{Implementation}
%
% The internal format of a bit set is quite simple, a sequence
% of digits |0| and |1|. The least significant bit is left.
% A bit set without any flag set is encoded by |0|. Also
% undefined bit sets are treated that way. After the
% highest bit that is set there are no further zeroes.
% A regular expression of valid bit sets values:
% \begin{quote}
%  |0|\string||[01]*1|
% \end{quote}
%
%    \begin{macrocode}
%<*package>
%    \end{macrocode}
%
% \subsection{Reload check and package identification}
%    Reload check, especially if the package is not used with \LaTeX.
%    \begin{macrocode}
\begingroup\catcode61\catcode48\catcode32=10\relax%
  \catcode13=5 % ^^M
  \endlinechar=13 %
  \catcode35=6 % #
  \catcode39=12 % '
  \catcode44=12 % ,
  \catcode45=12 % -
  \catcode46=12 % .
  \catcode58=12 % :
  \catcode64=11 % @
  \catcode123=1 % {
  \catcode125=2 % }
  \expandafter\let\expandafter\x\csname ver@bitset.sty\endcsname
  \ifx\x\relax % plain-TeX, first loading
  \else
    \def\empty{}%
    \ifx\x\empty % LaTeX, first loading,
      % variable is initialized, but \ProvidesPackage not yet seen
    \else
      \expandafter\ifx\csname PackageInfo\endcsname\relax
        \def\x#1#2{%
          \immediate\write-1{Package #1 Info: #2.}%
        }%
      \else
        \def\x#1#2{\PackageInfo{#1}{#2, stopped}}%
      \fi
      \x{bitset}{The package is already loaded}%
      \aftergroup\endinput
    \fi
  \fi
\endgroup%
%    \end{macrocode}
%    Package identification:
%    \begin{macrocode}
\begingroup\catcode61\catcode48\catcode32=10\relax%
  \catcode13=5 % ^^M
  \endlinechar=13 %
  \catcode35=6 % #
  \catcode39=12 % '
  \catcode40=12 % (
  \catcode41=12 % )
  \catcode44=12 % ,
  \catcode45=12 % -
  \catcode46=12 % .
  \catcode47=12 % /
  \catcode58=12 % :
  \catcode64=11 % @
  \catcode91=12 % [
  \catcode93=12 % ]
  \catcode123=1 % {
  \catcode125=2 % }
  \expandafter\ifx\csname ProvidesPackage\endcsname\relax
    \def\x#1#2#3[#4]{\endgroup
      \immediate\write-1{Package: #3 #4}%
      \xdef#1{#4}%
    }%
  \else
    \def\x#1#2[#3]{\endgroup
      #2[{#3}]%
      \ifx#1\@undefined
        \xdef#1{#3}%
      \fi
      \ifx#1\relax
        \xdef#1{#3}%
      \fi
    }%
  \fi
\expandafter\x\csname ver@bitset.sty\endcsname
\ProvidesPackage{bitset}%
  [2011/01/30 v1.1 Handle bit-vector datatype (HO)]%
%    \end{macrocode}
%
% \subsection{Catcodes}
%
%    \begin{macrocode}
\begingroup\catcode61\catcode48\catcode32=10\relax%
  \catcode13=5 % ^^M
  \endlinechar=13 %
  \catcode123=1 % {
  \catcode125=2 % }
  \catcode64=11 % @
  \def\x{\endgroup
    \expandafter\edef\csname BitSet@AtEnd\endcsname{%
      \endlinechar=\the\endlinechar\relax
      \catcode13=\the\catcode13\relax
      \catcode32=\the\catcode32\relax
      \catcode35=\the\catcode35\relax
      \catcode61=\the\catcode61\relax
      \catcode64=\the\catcode64\relax
      \catcode123=\the\catcode123\relax
      \catcode125=\the\catcode125\relax
    }%
  }%
\x\catcode61\catcode48\catcode32=10\relax%
\catcode13=5 % ^^M
\endlinechar=13 %
\catcode35=6 % #
\catcode64=11 % @
\catcode123=1 % {
\catcode125=2 % }
\def\TMP@EnsureCode#1#2{%
  \edef\BitSet@AtEnd{%
    \BitSet@AtEnd
    \catcode#1=\the\catcode#1\relax
  }%
  \catcode#1=#2\relax
}
\TMP@EnsureCode{33}{12}% !
\TMP@EnsureCode{39}{12}% '
\TMP@EnsureCode{40}{12}% (
\TMP@EnsureCode{41}{12}% )
\TMP@EnsureCode{42}{12}% *
\TMP@EnsureCode{43}{12}% +
\TMP@EnsureCode{44}{12}% ,
\TMP@EnsureCode{45}{12}% -
\TMP@EnsureCode{46}{12}% .
\TMP@EnsureCode{47}{12}% /
\TMP@EnsureCode{58}{11}% : (letter!)
\TMP@EnsureCode{60}{12}% <
\TMP@EnsureCode{62}{12}% >
\TMP@EnsureCode{63}{14}% ? (comment!)
\TMP@EnsureCode{91}{12}% [
\TMP@EnsureCode{93}{12}% ]
\TMP@EnsureCode{96}{12}% `
\edef\BitSet@AtEnd{\BitSet@AtEnd\noexpand\endinput}
\begingroup\expandafter\expandafter\expandafter\endgroup
\expandafter\ifx\csname BitSet@TestMode\endcsname\relax
\else
  \catcode63=9 % ? (ignore)
\fi
? \let\BitSet@@TestMode\BitSet@TestMode
%    \end{macrocode}
%
% \subsection{Package loading}
%
%    \begin{macrocode}
\begingroup\expandafter\expandafter\expandafter\endgroup
\expandafter\ifx\csname RequirePackage\endcsname\relax
  \def\TMP@RequirePackage#1[#2]{%
    \begingroup\expandafter\expandafter\expandafter\endgroup
    \expandafter\ifx\csname ver@#1.sty\endcsname\relax
      \input #1.sty\relax
    \fi
  }%
  \TMP@RequirePackage{infwarerr}[2007/09/09]%
  \TMP@RequirePackage{intcalc}[2007/09/27]%
  \TMP@RequirePackage{bigintcalc}[2007/09/27]%
\else
  \RequirePackage{infwarerr}[2007/09/09]%
  \RequirePackage{intcalc}[2007/09/27]%
  \RequirePackage{bigintcalc}[2007/09/27]%
\fi
%    \end{macrocode}
%
% \subsection{Help macros}
%
% \subsubsection{Number constant}
%
%    \begin{macro}{\BitSet@MaxSize}
%    \begin{macrocode}
\def\BitSet@MaxSize{2147483647}%
%    \end{macrocode}
%    \end{macro}
%
% \subsubsection{General basic macros}
%
%    \begin{macro}{\BitSet@Empty}
%    \begin{macrocode}
\def\BitSet@Empty{}
%    \end{macrocode}
%    \end{macro}
%    \begin{macro}{\BitSet@FirstOfOne}
%    \begin{macrocode}
\def\BitSet@FirstOfOne#1{#1}
%    \end{macrocode}
%    \end{macro}
%    \begin{macro}{\BitSet@Gobble}
%    \begin{macrocode}
\def\BitSet@Gobble#1{}
%    \end{macrocode}
%    \end{macro}
%    \begin{macro}{\BitSet@FirstOfTwo}
%    \begin{macrocode}
\def\BitSet@FirstOfTwo#1#2{#1}
%    \end{macrocode}
%    \end{macro}
%    \begin{macro}{\BitSet@SecondOfTwo}
%    \begin{macrocode}
\def\BitSet@SecondOfTwo#1#2{#2}
%    \end{macrocode}
%    \end{macro}
%
%    \begin{macro}{\BitSet@Space}
%    \begin{macrocode}
\def\BitSet@Space{ }
%    \end{macrocode}
%    \end{macro}
%    \begin{macro}{\BitSet@ZapSpace}
%    \begin{macrocode}
\def\BitSet@ZapSpace#1 #2{%
  #1%
  \ifx\BitSet@Empty#2%
  \else
    \expandafter\BitSet@ZapSpace
  \fi
  #2%
}
%    \end{macrocode}
%    \end{macro}
%
% \subsubsection{Tail recursion}
%
%    \begin{macro}{\BitSet@Fi}
%    \begin{macrocode}
\let\BitSet@Fi\fi
%    \end{macrocode}
%    \end{macro}
%    \begin{macro}{\BitSet@AfterFi}
%    \begin{macrocode}
\def\BitSet@AfterFi#1#2\BitSet@Fi{\fi#1}
%    \end{macrocode}
%    \end{macro}
%    \begin{macro}{\BitSet@AfterFiFi}
%    \begin{macrocode}
\def\BitSet@AfterFiFi#1#2\BitSet@Fi{\fi\fi#1}%
%    \end{macrocode}
%    \end{macro}
%    \begin{macro}{\BitSet@AfterFiFiFi}
%    \begin{macrocode}
\def\BitSet@AfterFiFiFi#1#2\BitSet@Fi{\fi\fi\fi#1}%
%    \end{macrocode}
%    \end{macro}
%
% \subsubsection{Check macros}
%
%    \begin{macro}{\BitSet@IfUndefined}
%    \begin{macrocode}
\def\BitSet@IfUndefined#1{%
  \expandafter\ifx\csname BS@#1\endcsname\relax
    \expandafter\BitSet@FirstOfTwo
  \else
    \expandafter\BitSet@SecondOfTwo
  \fi
}
%    \end{macrocode}
%    \end{macro}
%
%    \begin{macro}{\BitSet@CheckIndex}
%    |#1|: continuation code\\
%    |#2|: BitSet\\
%    |#3|: Index
%    \begin{macrocode}
\def\BitSet@CheckIndex#1#2#3{%
  \BitSet@IfUndefined{#2}{\bitsetReset{#2}}{}%
  \expandafter\expandafter\expandafter\BitSet@@CheckIndex
  \intcalcNum{#3}!%
  {#2}{#1}%
}
%    \end{macrocode}
%    \end{macro}
%    \begin{macro}{\BitSet@@CheckIndex}
%    |#1|: plain Index\\
%    |#2|: BitSet\\
%    |#3|: continuation code
%    \begin{macrocode}
\def\BitSet@@CheckIndex#1!#2#3{%
  \ifnum#1<0 %
    \BitSet@AfterFi{%
      \@PackageError{bitset}{%
        Invalid negative index (#1)%
      }\@ehc
    }%
  \else
    \BitSet@AfterFi{%
      #3{#2}{#1}%
    }%
  \BitSet@Fi
}
%    \end{macrocode}
%    \end{macro}
%
% \subsection{Miscellaneous}\label{src:misc}
%
%    \begin{macro}{\bitsetReset}
%    \begin{macrocode}
\def\bitsetReset#1{%
  \expandafter\def\csname BS@#1\endcsname{0}%
}
%    \end{macrocode}
%    \end{macro}
%
%    \begin{macro}{\bitsetLet}
%    \begin{macrocode}
\def\bitsetLet#1#2{%
  \BitSet@IfUndefined{#2}{%
    \bitsetReset{#1}%
  }{%
    \expandafter\let\csname BS@#1\expandafter\endcsname
                    \csname BS@#2\endcsname
  }%
}
%    \end{macrocode}
%    \end{macro}
%
% \subsection{Import}\label{src:import}
%
% \subsubsection{From binary number}
%
%    \begin{macro}{\bitsetSetBin}
%    \begin{macrocode}
\def\bitsetSetBin#1#2{%
  \edef\BitSet@Temp{#2}%
  \edef\BitSet@Temp{%
    \expandafter\expandafter\expandafter\BitSet@ZapSpace
    \expandafter\BitSet@Temp\BitSet@Space\BitSet@Empty
  }%
  \edef\BitSet@Temp{%
    \expandafter\BitSet@KillZeros\BitSet@Temp\BitSet@Empty
  }%
  \ifx\BitSet@Temp\BitSet@Empty
    \expandafter\let\csname BS@#1\endcsname\BitSet@Zero
  \else
    \expandafter\edef\csname BS@#1\endcsname{%
      \expandafter\BitSet@Reverse\BitSet@Temp!%
    }%
  \fi
}
%    \end{macrocode}
%    \end{macro}
%    \begin{macro}{\BitSet@KillZeros}
%    \begin{macrocode}
\def\BitSet@KillZeros#1{%
  \ifx#10%
    \expandafter\BitSet@KillZeros
  \else
    #1%
  \fi
}
%    \end{macrocode}
%    \end{macro}
%    \begin{macro}{\BitSet@Reverse}
%    \begin{macrocode}
\def\BitSet@Reverse#1#2!{%
  \ifx\\#2\\%
    #1%
  \else
    \BitSet@AfterFi{%
      \BitSet@Reverse#2!#1%
    }%
  \BitSet@Fi
}
%    \end{macrocode}
%    \end{macro}
%
% \subsubsection{From octal/hex number}
%
%    \begin{macro}{\bitsetSetOct}
%    \begin{macrocode}
\def\bitsetSetOct{%
  \BitSet@SetOctHex\BitSet@FromFirstOct
}
%    \end{macrocode}
%    \end{macro}
%    \begin{macro}{\bitsetSetHex}
%    \begin{macrocode}
\def\bitsetSetHex{%
  \BitSet@SetOctHex\BitSet@FromFirstHex
}
%    \end{macrocode}
%    \end{macro}
%    \begin{macro}{\BitSet@SetOctHex}
%    \begin{macrocode}
\def\BitSet@SetOctHex#1#2#3{%
  \edef\BitSet@Temp{#3}%
  \edef\BitSet@Temp{%
    \expandafter\expandafter\expandafter\BitSet@ZapSpace
    \expandafter\BitSet@Temp\BitSet@Space\BitSet@Empty
  }%
  \edef\BitSet@Temp{%
    \expandafter\BitSet@KillZeros\BitSet@Temp\BitSet@Empty
  }%
  \ifx\BitSet@Temp\BitSet@Empty
    \expandafter\let\csname BS@#2\endcsname\BitSet@Zero
  \else
    \edef\BitSet@Temp{%
      \expandafter#1\BitSet@Temp!%
    }%
    \ifx\BitSet@Temp\BitSet@Empty
      \expandafter\let\csname BS@#2\endcsname\BitSet@Zero
    \else
      \expandafter\edef\csname BS@#2\endcsname{%
        \expandafter\BitSet@Reverse\BitSet@Temp!%
      }%
    \fi
  \fi
}
%    \end{macrocode}
%    \end{macro}
%    \begin{macro}{\BitSet@FromFirstOct}
%    \begin{macrocode}
\def\BitSet@FromFirstOct#1{%
  \ifx#1!%
  \else
    \ifcase#1 \BitSet@AfterFiFi\BitSet@FromFirstOct
    \or 1%
    \or 10%
    \or 11%
    \or 100%
    \or 101%
    \or 110%
    \or 111%
    \else \BitSetError:WrongOctalDigit%
    \fi
    \expandafter\BitSet@FromOct
  \BitSet@Fi
}
%    \end{macrocode}
%    \end{macro}
%    \begin{macro}{\BitSet@FromOct}
%    \begin{macrocode}
\def\BitSet@FromOct#1{%
  \ifx#1!%
  \else
    \ifcase#1 000%
    \or 001%
    \or 010%
    \or 011%
    \or 100%
    \or 101%
    \or 110%
    \or 111%
    \else \BitSetError:WrongOctalDigit%
    \fi
    \expandafter\BitSet@FromOct
  \fi
}
%    \end{macrocode}
%    \end{macro}
%
%    \begin{macro}{\BitSet@FromFirstHex}
%    \begin{macrocode}
\def\BitSet@FromFirstHex#1{%
  \ifx#1!%
  \else
    \ifx#10%
      \BitSet@AfterFiFi\BitSet@FromFirstHex
    \fi
    \expandafter\ifx\csname BitSet@Hex#1\endcsname\relax
      \BitSetError:InvalidHexDigit%
    \else
      \expandafter\expandafter\expandafter\BitSet@KillZeros
      \csname BitSet@Hex#1\endcsname
    \fi
    \expandafter\BitSet@FromHex
  \BitSet@Fi
}
%    \end{macrocode}
%    \end{macro}
%    \begin{macro}{\BitSet@FromHex}
%    \begin{macrocode}
\def\BitSet@FromHex#1{%
  \ifx#1!%
  \else
    \expandafter\ifx\csname BitSet@Hex#1\endcsname\relax
      \BitSetError:InvalidHexDigit%
    \else
      \csname BitSet@Hex#1\endcsname
    \fi
    \expandafter\BitSet@FromHex
  \fi
}
%    \end{macrocode}
%    \end{macro}
%    \begin{macro}{\BitSet@Hex[0..F]}
%    \begin{macrocode}
\def\BitSet@Temp#1{%
  \expandafter\def\csname BitSet@Hex#1\endcsname
}
\BitSet@Temp 0{0000}%
\BitSet@Temp 1{0001}%
\BitSet@Temp 2{0010}%
\BitSet@Temp 3{0011}%
\BitSet@Temp 4{0100}%
\BitSet@Temp 5{0101}%
\BitSet@Temp 6{0110}%
\BitSet@Temp 7{0111}%
\BitSet@Temp 8{1000}%
\BitSet@Temp 9{1001}%
\BitSet@Temp A{1010}%
\BitSet@Temp B{1011}%
\BitSet@Temp C{1100}%
\BitSet@Temp D{1101}%
\BitSet@Temp E{1110}%
\BitSet@Temp F{1111}%
\BitSet@Temp a{1010}%
\BitSet@Temp b{1011}%
\BitSet@Temp c{1100}%
\BitSet@Temp d{1101}%
\BitSet@Temp e{1110}%
\BitSet@Temp f{1111}%
%    \end{macrocode}
%    \end{macro}
%
% \subsubsection{From decimal number}
%
%    \begin{macro}{\bitsetSetDec}
%    \begin{macrocode}
\def\bitsetSetDec#1#2{%
  \edef\BitSet@Temp{#2}%
  \edef\BitSet@Temp{%
    \expandafter\expandafter\expandafter\BitSet@ZapSpace
    \expandafter\BitSet@Temp\BitSet@Space\BitSet@Empty
  }%
  \edef\BitSet@Temp{%
    \expandafter\BitSet@KillZeros\BitSet@Temp\BitSet@Empty
  }%
  \ifx\BitSet@Temp\BitSet@Empty
    \expandafter\let\csname BS@#1\endcsname\BitSet@Zero
  \else
    \ifcase\bigintcalcSgn{\BitSet@Temp} %
      \expandafter\let\csname BS@#1\endcsname\BitSet@Zero
    \or
      \ifnum\bigintcalcCmp\BitSet@Temp\BitSet@MaxSize>0 %
        \expandafter\edef\csname BS@#1\endcsname{%
          \expandafter\BitSet@SetDecBig\BitSet@Temp!%
        }%
      \else
        \expandafter\edef\csname BS@#1\endcsname{%
          \expandafter\BitSet@SetDec\BitSet@Temp!%
        }%
      \fi
    \else
      \@PackageError{bitset}{%
        Bit sets cannot be negative%
      }\@ehc
    \fi
  \fi
}
%    \end{macrocode}
%    \end{macro}
%    \begin{macro}{\BitSet@SetDecBig}
%    \begin{macrocode}
\def\BitSet@SetDecBig#1#2#3#4#5#6#7#8#9!{%
  \ifx\\#9\\%
    \BitSet@SetDec#1#2#3#4#5#6#7#8!%
  \else
    \ifcase\BigIntCalcOdd#1#2#4#5#6#7#8#9! %
      0%
    \or
      1%
?   \else\BitSetError:ThisCannotHappen%
    \fi
    \BitSet@AfterFi{%
      \expandafter\expandafter\expandafter\BitSet@SetDecBig
      \BigIntCalcShr#1#2#3#4#5#6#7#8#9!!%
    }%
  \BitSet@Fi
}
%    \end{macrocode}
%    \end{macro}
%    \begin{macro}{\BitSet@SetDec}
%    \begin{macrocode}
\def\BitSet@SetDec#1!{%
  \ifcase#1 %
  \or 1%
  \else
    \ifodd#1 %
      1%
    \else
      0%
    \fi
    \BitSet@AfterFi{%
      \expandafter\expandafter\expandafter\BitSet@SetDec
      \IntCalcShr#1!!%
    }%
  \BitSet@Fi
}
%    \end{macrocode}
%    \end{macro}
%
% \subsection{Export}\label{src:export}
%
% \subsubsection{To binary number}
%
%    \begin{macro}{\bitsetGetBin}
%    \begin{macrocode}
\def\bitsetGetBin#1#2{%
  \romannumeral0%
  \expandafter\expandafter\expandafter\BitSet@@GetBin
  \intcalcNum{#2}!{#1}%
}
%    \end{macrocode}
%    \end{macro}
%    \begin{macro}{\BitSet@@GetBin}
%    \begin{macrocode}
\def\BitSet@@GetBin#1!#2{%
  \BitSet@IfUndefined{#2}{%
    \ifnum#1>1 %
      \BitSet@AfterFi{%
        \expandafter\expandafter\expandafter\BitSet@Fill
        \IntCalcDec#1!!0%
      }%
    \else
      \BitSet@AfterFi{ 0}%
    \BitSet@Fi
  }{%
    \expandafter\expandafter\expandafter\BitSet@NumBinRev
    \expandafter\expandafter\expandafter1%
    \expandafter\expandafter\expandafter!%
    \csname BS@#2\endcsname!!#1!%
  }%
}
%    \end{macrocode}
%    \end{macro}
%    \begin{macro}{\BitSet@Fill}
%    |#1|: number of leading digits |0|\\
%    |#2|: result
%    \begin{macrocode}
\def\BitSet@Fill#1!{%
  \ifnum#1>0 %
    \BitSet@AfterFi{%
      \expandafter\expandafter\expandafter\BitSet@Fill
      \IntCalcDec#1!!0%
    }%
  \else
    \BitSet@AfterFi{ }%
  \BitSet@Fi
}
%    \end{macrocode}
%    \end{macro}
%
%    \begin{macro}{\BitSet@NumBinRev}
%    |#1|: bit counter (including |#2|)\\
%    |#2#3|: reverted number\\
%    |#4|: result\\
%    |#5|: min size
%    \begin{macrocode}
\def\BitSet@NumBinRev#1!#2#3!{%
  \ifx\\#3\\%
    \BitSet@AfterFi{%
      \BitSet@NumBinFill#1!#2%
    }%
  \else
    \BitSet@AfterFi{%
      \expandafter\expandafter\expandafter\BitSet@NumBinRev
      \IntCalcInc#1!!#3!#2%
    }%
  \BitSet@Fi
}
%    \end{macrocode}
%    \end{macro}
%    \begin{macro}{\BitSet@NumBinFill}
%    \begin{macrocode}
\def\BitSet@NumBinFill#1!#2!#3!{%
  \ifnum#3>#1 %
    \BitSet@AfterFi{%
      \expandafter\expandafter\expandafter\BitSet@Fill
      \IntCalcSub#3!#1!!#2%
    }%
  \else
    \BitSet@AfterFi{ #2}%
  \BitSet@Fi
}
%    \end{macrocode}
%    \end{macro}
%
% \subsubsection{To octal/hexadecimal number}
%
%    \begin{macro}{\bitsetGetOct}
%    \begin{macrocode}
\def\bitsetGetOct#1#2{%
  \romannumeral0%
  \bitsetIsEmpty{#1}{%
    \expandafter\expandafter\expandafter\BitSet@@GetOctHex
    \intcalcNum{#2}!3!230%
  }{%
    \expandafter\expandafter\expandafter\BitSet@@GetOct
    \expandafter\expandafter\expandafter1%
    \expandafter\expandafter\expandafter!%
    \expandafter\expandafter\expandafter!%
    \csname BS@#1\endcsname00%
    \BitSet@Empty\BitSet@Empty\BitSet@Empty!{#2}%
  }%
}
%    \end{macrocode}
%    \end{macro}
%    \begin{macro}{\bitsetGetHex}
%    \begin{macrocode}
\def\bitsetGetHex#1#2{%
  \romannumeral0%
  \bitsetIsEmpty{#1}{%
    \expandafter\expandafter\expandafter\BitSet@@GetOctHex
    \intcalcNum{#2}!4!340%
  }{%
    \expandafter\expandafter\expandafter\BitSet@@GetHex
    \expandafter\expandafter\expandafter1%
    \expandafter\expandafter\expandafter!%
    \expandafter\expandafter\expandafter!%
    \csname BS@#1\endcsname000%
    \BitSet@Empty\BitSet@Empty\BitSet@Empty\BitSet@Empty!{#2}%
  }%
}
%    \end{macrocode}
%    \end{macro}
%    \begin{macro}{\BitSet@@GetOct}
%    |#1|: number of digits\\
%    |#2|: result\\
%    |#3#4#5|: bits
%    \begin{macrocode}
\def\BitSet@@GetOct#1!#2!#3#4#5{%
  \ifx#5\BitSet@Empty
    \BitSet@AfterFi{%
      \expandafter\expandafter\expandafter\BitSet@GetOctHex
      \IntCalcDec#1!!#2!23%
    }%
  \else
    \BitSet@AfterFi{%
      \expandafter\expandafter\expandafter\BitSet@@GetOct
      \number\IntCalcInc#1!\expandafter\expandafter\expandafter!%
      \csname BitSet@Oct#5#4#3\endcsname#2!%
    }%
  \BitSet@Fi
}
%    \end{macrocode}
%    \end{macro}
%    \begin{macro}{\BitSet@Oct[000..111]}
%    \begin{macrocode}
\def\BitSet@Temp#1#2#3#4{%
  \expandafter\def\csname BitSet@Oct#1#2#3\endcsname{#4}%
}
\BitSet@Temp0000%
\BitSet@Temp0011%
\BitSet@Temp0102%
\BitSet@Temp0113%
\BitSet@Temp1004%
\BitSet@Temp1015%
\BitSet@Temp1106%
\BitSet@Temp1117%
%    \end{macrocode}
%    \end{macro}
%    \begin{macro}{\BitSet@@GetHex}
%    |#1|: number of digits\\
%    |#2|: result\\
%    |#3#4#5#6|: bits
%    \begin{macrocode}
\def\BitSet@@GetHex#1!#2!#3#4#5#6{%
  \ifx#6\BitSet@Empty
    \BitSet@AfterFi{%
      \expandafter\expandafter\expandafter\BitSet@GetOctHex
      \IntCalcDec#1!!#2!34%
    }%
  \else
    \BitSet@AfterFi{%
      \expandafter\expandafter\expandafter\BitSet@@GetHex
      \number\IntCalcInc#1!\expandafter\expandafter\expandafter!%
      \csname BitSet@Hex#6#5#4#3\endcsname#2!%
    }%
  \BitSet@Fi
}
%    \end{macrocode}
%    \end{macro}
%    \begin{macro}{\BitSet@Hex[0000..1111]}
%    \begin{macrocode}
\def\BitSet@Temp#1#2#3#4#5{%
  \expandafter\def\csname BitSet@Hex#1#2#3#4\endcsname{#5}%
}
\BitSet@Temp00000%
\BitSet@Temp00011%
\BitSet@Temp00102%
\BitSet@Temp00113%
\BitSet@Temp01004%
\BitSet@Temp01015%
\BitSet@Temp01106%
\BitSet@Temp01117%
\BitSet@Temp10008%
\BitSet@Temp10019%
\BitSet@Temp1010A%
\BitSet@Temp1011B%
\BitSet@Temp1100C%
\BitSet@Temp1101D%
\BitSet@Temp1110E%
\BitSet@Temp1111F%
%    \end{macrocode}
%    \end{macro}
%    \begin{macro}{\BitSet@GetOctHex}
%    Leading zeros $ (|#4| - |#1| * 3 + 2)/3 $ if $ |#4| > |#1| * 3$\\
%    |#1|: digit size\\
%    |#2|: result\\
%    |#3|: bits per digit - 1\\
%    |#4|: bits per digit
%    |#5|: garbage\\
%    |#6|: min size
%    \begin{macrocode}
\def\BitSet@GetOctHex#1!#2!#3#4#5!#6{%
  \expandafter\BitSet@@GetOctHex
  \number\intcalcNum{#6}\expandafter\expandafter\expandafter!%
  \IntCalcMul#1!#4!!#3#4#2%
}
%    \end{macrocode}
%    \end{macro}
%    \begin{macro}{\BitSet@@GetOctHex}
%    |#1|: plain min size\\
%    |#2|: digits * (bits per digit)\\
%    |#3|: bits per digit - 1\\
%    |#4|: bits per digit
%    \begin{macrocode}
\def\BitSet@@GetOctHex#1!#2!#3#4{%
  \ifnum#1>#2 %
    \BitSet@AfterFi{%
      \expandafter\expandafter\expandafter\expandafter
      \expandafter\expandafter\expandafter\BitSet@Fill
      \expandafter\IntCalcDiv\number
      \expandafter\expandafter\expandafter\IntCalcAdd
      \IntCalcSub#1!#2!!#3!!#4!!%
    }%
  \else
    \BitSet@AfterFi{ }%
  \BitSet@Fi
}
%    \end{macrocode}
%    \end{macro}
%
% \subsubsection{To decimal number}
%
%    \begin{macro}{\bitsetGetDec}
%    \begin{macrocode}
\def\bitsetGetDec#1{%
  \romannumeral0%
  \BitSet@IfUndefined{#1}{ 0}{%
    \expandafter\expandafter\expandafter\BitSet@GetDec
    \csname BS@#1\endcsname!%
  }%
}
%    \end{macrocode}
%    \end{macro}
%    \begin{macro}{\BitSet@GetDec}
%    \begin{macrocode}
\def\BitSet@GetDec#1#2!{%
  \ifx\\#2\\%
    \BitSet@AfterFi{ #1}%
  \else
    \BitSet@AfterFi{%
      \BitSet@@GetDec2!#1!#2!%
    }%
  \BitSet@Fi
}
%    \end{macrocode}
%    \end{macro}
%    \begin{macro}{\BitSet@@GetDec}
%    |#1|: power of two\\
%    |#2|: result\\
%    |#3#4|: number
%    \begin{macrocode}
\def\BitSet@@GetDec#1!#2!#3#4!{%
  \ifx\\#4\\%
    \ifx#31%
      \BitSet@AfterFiFi{%
        \expandafter\expandafter\expandafter\BitSet@Space
        \IntCalcAdd#1!#2!%
      }%
    \else
      \BitSet@AfterFiFi{ #2}%
    \fi
  \else
    \ifx#31%
      \BitSet@AfterFiFi{%
        \csname BitSet@N#1%
        \expandafter\expandafter\expandafter\endcsname
        \IntCalcAdd#1!#2!!#4!%
      }%
    \else
      \BitSet@AfterFiFi{%
        \csname BitSet@N#1\endcsname#2!#4!%
      }%
    \fi
  \BitSet@Fi
}
%    \end{macrocode}
%    \end{macro}
%    \begin{macro}{\BitSet@N[1,2,4,...]}
%    \begin{macrocode}
\def\BitSet@Temp#1#2{%
  \expandafter\def\csname BitSet@N#1\endcsname{%
    \BitSet@@GetDec#2!%
  }%
}
\BitSet@Temp{1}{2}
\BitSet@Temp{2}{4}
\BitSet@Temp{4}{8}
\BitSet@Temp{8}{16}
\BitSet@Temp{16}{32}
\BitSet@Temp{32}{64}
\BitSet@Temp{64}{128}
\BitSet@Temp{128}{256}
\BitSet@Temp{256}{512}
\BitSet@Temp{512}{1024}
\BitSet@Temp{1024}{2048}
\BitSet@Temp{2048}{4096}
\BitSet@Temp{4096}{8192}
\BitSet@Temp{8192}{16384}
\BitSet@Temp{16384}{32768}
\BitSet@Temp{32768}{65536}
\BitSet@Temp{65536}{131072}
\BitSet@Temp{131072}{262144}
\BitSet@Temp{262144}{524288}
\BitSet@Temp{524288}{1048576}
\BitSet@Temp{1048576}{2097152}
\BitSet@Temp{2097152}{4194304}
\BitSet@Temp{4194304}{8388608}
\BitSet@Temp{8388608}{16777216}
\BitSet@Temp{16777216}{33554432}
\BitSet@Temp{33554432}{67108864}
\BitSet@Temp{67108864}{134217728}
\BitSet@Temp{134217728}{268435456}
\BitSet@Temp{268435456}{536870912}
\BitSet@Temp{536870912}{1073741824}
%    \end{macrocode}
%    \end{macro}
%    \begin{macro}{\BitSet@N1073741824}
%    \begin{macrocode}
\expandafter\def\csname BitSet@N1073741824\endcsname{%
  \BitSet@GetDecBig2147483648!%
}%
%    \end{macrocode}
%    \end{macro}
%
%    \begin{macro}{\BitSet@GetDecBig}
%    |#1|: current power of two\\
%    |#2|: result\\
%    |#3#4|: number
%    \begin{macrocode}
\def\BitSet@GetDecBig#1!#2!#3#4!{%
  \ifx\\#4\\%
    \ifx#31%
      \BitSet@AfterFiFi{%
        \expandafter\expandafter\expandafter\BitSet@Space
        \BigIntCalcAdd#1!#2!%
      }%
    \else
      \BitSet@AfterFiFi{ #2}%
    \fi
  \else
    \ifx#31%
      \BitSet@AfterFiFi{%
        \expandafter\expandafter\expandafter\BitSet@@GetDecBig
        \BigIntCalcAdd#1!#2!!#1!#4!%
      }%
    \else
      \BitSet@AfterFiFi{%
        \expandafter\expandafter\expandafter\BitSet@GetDecBig
        \BigIntCalcShl#1!!#2!#4!%
      }%
    \fi
  \BitSet@Fi
}
%    \end{macrocode}
%    \end{macro}
%    \begin{macro}{\BitSet@@GetDecBig}
%    |#1|: result\\
%    |#2|: power of two\\
%    |#3#4|: number
%    \begin{macrocode}
\def\BitSet@@GetDecBig#1!#2!{%
  \expandafter\expandafter\expandafter\BitSet@GetDecBig
  \BigIntCalcShl#2!!#1!%
}
%    \end{macrocode}
%    \end{macro}
%
% \subsection{Logical operators}\label{src:log}
%
% \subsubsection{\cs{bitsetAnd}}
%
%    \begin{macro}{\bitsetAnd}
%    Decision table for \cs{bitsetAnd}:
%    \begin{quote}
%    \begin{tabular}{l|l|l|l|}
%                         & undef(B)   & empty(B)   & cardinality(B)$>$0\\
%      \hline
%      undef(A)           & A := empty & A := empty & A := empty\\
%      \hline
%      empty(A)           &            &            &\\
%      \hline
%      cardinality(A)$>$0 & A := empty & A := empty & A \&= B\\
%      \hline
%    \end{tabular}
%    \end{quote}
%    \begin{macrocode}
\def\bitsetAnd#1#2{%
  \bitsetIsEmpty{#1}{%
    \bitsetReset{#1}%
  }{%
    \bitsetIsEmpty{#2}{%
      \bitsetReset{#1}%
    }{%
      \expandafter\edef\csname BS@#1\endcsname{%
        \expandafter\expandafter\expandafter\BitSet@And
        \csname BS@#1\expandafter\expandafter\expandafter\endcsname
        \expandafter\expandafter\expandafter!%
        \csname BS@#2\endcsname!!%
      }%
      \expandafter\ifx\csname BS@#1\endcsname\BitSet@Empty
        \bitsetReset{#1}%
      \fi
    }%
  }%
}
%    \end{macrocode}
%    \end{macro}
%    \begin{macro}{\BitSet@And}
%    \begin{macrocode}
\def\BitSet@And#1#2!#3#4!#5!{%
  \ifx\\#2\\%
    \ifnum#1#3=11 #51\fi
  \else
    \ifx\\#4\\%
      \ifnum#1#3=11 #51\fi
    \else
      \ifnum#1#3=11 %
        #51%
        \BitSet@AfterFiFiFi{%
          \BitSet@And#2!#4!!%
        }%
      \else
        \BitSet@AfterFiFiFi{%
          \BitSet@And#2!#4!#50!%
        }%
      \fi
    \fi
  \BitSet@Fi
}
%    \end{macrocode}
%    \end{macro}
%
% \subsubsection{\cs{bitsetAndNot}}
%
%    \begin{macro}{\bitsetAndNot}
%    Decision table for \cs{bitsetAndNot}:
%    \begin{quote}
%    \begin{tabular}{l|l|l|l|}
%                         & undef(B)   & empty(B)   & cardinality(B)$>$0\\
%      \hline
%      undef(A)           & A := empty & A := empty & A := empty\\
%      \hline
%      empty(A)           &            &            &\\
%      \hline
%      cardinality(A)$>$0 &            &            & A \&= !B\\
%      \hline
%    \end{tabular}
%    \end{quote}
%    \begin{macrocode}
\def\bitsetAndNot#1#2{%
  \bitsetIsEmpty{#1}{%
    \bitsetReset{#1}%
  }{%
    \bitsetIsEmpty{#2}{%
    }{%
      \expandafter\edef\csname BS@#1\endcsname{%
        \expandafter\expandafter\expandafter\BitSet@AndNot
        \csname BS@#1\expandafter\expandafter\expandafter\endcsname
        \expandafter\expandafter\expandafter!%
        \csname BS@#2\endcsname!!%
      }%
      \expandafter\ifx\csname BS@#1\endcsname\BitSet@Empty
        \bitsetReset{#1}%
      \fi
    }%
  }%
}
%    \end{macrocode}
%    \end{macro}
%    \begin{macro}{\BitSet@AndNot}
%    \begin{macrocode}
\def\BitSet@AndNot#1#2!#3#4!#5!{%
  \ifx\\#2\\%
    \ifnum#1#3=10 #51\fi
  \else
    \ifx\\#4\\%
      #5%
      \ifnum#1#3=10 1\else 0\fi
      #2%
    \else
      \ifnum#1#3=10 %
        #51%
        \BitSet@AfterFiFiFi{%
          \BitSet@AndNot#2!#4!!%
        }%
      \else
        \BitSet@AfterFiFiFi{%
          \BitSet@AndNot#2!#4!#50!%
        }%
      \fi
    \fi
  \BitSet@Fi
}
%    \end{macrocode}
%    \end{macro}
%
% \subsubsection{\cs{bitsetOr}}
%
%    \begin{macro}{\bitsetOr}
%    Decision table for \cs{bitsetOr}:
%    \begin{quote}
%    \begin{tabular}{l|l|l|l|}
%                         & undef(B)   & empty(B)   & cardinality(B)$>$0\\
%      \hline
%      undef(A)           & A := empty & A := empty & A := B\\
%      \hline
%      empty(A)           &            &            & A := B\\
%      \hline
%      cardinality(A)$>$0 &            &            & A \string|= B\\
%      \hline
%    \end{tabular}
%    \end{quote}
%    \begin{macrocode}
\def\bitsetOr#1#2{%
  \bitsetIsEmpty{#2}{%
    \BitSet@IfUndefined{#1}{\bitsetReset{#1}}{}%
  }{%
    \bitsetIsEmpty{#1}{%
      \expandafter\let\csname BS@#1\expandafter\endcsname
                      \csname BS@#2\endcsname
    }{%
      \expandafter\edef\csname BS@#1\endcsname{%
        \expandafter\expandafter\expandafter\BitSet@Or
        \csname BS@#1\expandafter\expandafter\expandafter\endcsname
        \expandafter\expandafter\expandafter!%
        \csname BS@#2\endcsname!%
      }%
    }%
  }%
}
%    \end{macrocode}
%    \end{macro}
%    \begin{macro}{\BitSet@Or}
%    \begin{macrocode}
\def\BitSet@Or#1#2!#3#4!{%
  \ifnum#1#3>0 1\else 0\fi
  \ifx\\#2\\%
    #4%
  \else
    \ifx\\#4\\%
      #2%
    \else
      \BitSet@AfterFiFi{%
        \BitSet@Or#2!#4!%
      }%
    \fi
  \BitSet@Fi
}
%    \end{macrocode}
%    \end{macro}
%
% \subsubsection{\cs{bitsetXor}}
%
%    \begin{macro}{\bitsetXor}
%    Decision table for \cs{bitsetXor}:
%    \begin{quote}
%    \begin{tabular}{l|l|l|l|}
%                         & undef(B)   & empty(B)   & cardinality(B)$>$0\\
%      \hline
%      undef(A)           & A := empty & A := empty & A := B\\
%      \hline
%      empty(A)           &            &            & A := B\\
%      \hline
%      cardinality(A)$>$0 &            &            & A \string^= B\\
%      \hline
%    \end{tabular}
%    \end{quote}
%    \begin{macrocode}
\def\bitsetXor#1#2{%
  \bitsetIsEmpty{#2}{%
    \BitSet@IfUndefined{#1}{\bitsetReset{#1}}{}%
  }{%
    \bitsetIsEmpty{#1}{%
      \expandafter\let\csname BS@#1\expandafter\endcsname
                      \csname BS@#2\endcsname
    }{%
      \expandafter\edef\csname BS@#1\endcsname{%
        \expandafter\expandafter\expandafter\BitSet@Xor
        \csname BS@#1\expandafter\expandafter\expandafter\endcsname
        \expandafter\expandafter\expandafter!%
        \csname BS@#2\endcsname!!%
      }%
      \expandafter\ifx\csname BS@#1\endcsname\BitSet@Empty
        \bitsetReset{#1}%
      \fi
    }%
  }%
}
%    \end{macrocode}
%    \end{macro}
%    \begin{macro}{\BitSet@Xor}
%    \begin{macrocode}
\def\BitSet@Xor#1#2!#3#4!#5!{%
  \ifx\\#2\\%
    \ifx#1#3%
      \ifx\\#4\\%
      \else
        #50#4%
      \fi
    \else
      #51#4%
    \fi
  \else
    \ifx\\#4\\%
      #5%
      \ifx#1#30\else 1\fi
      #2%
    \else
      \ifx#1#3%
        \BitSet@AfterFiFiFi{%
          \BitSet@Xor#2!#4!#50!%
        }%
      \else
        #51%
        \BitSet@AfterFiFiFi{%
          \BitSet@Xor#2!#4!!%
        }%
      \fi
    \fi
  \BitSet@Fi
}
%    \end{macrocode}
%    \end{macro}
%
% \subsubsection{Shifting}\label{src:shift}
%
% \subsubsection{\cs{bitsetShiftLeft}}
%
%    \begin{macro}{\bitsetShiftLeft}
%    \begin{macrocode}
\def\bitsetShiftLeft#1#2{%
  \BitSet@IfUndefined{#1}{%
    \bitsetReset{#1}%
  }{%
    \bitsetIsEmpty{#1}{%
    }{%
      \expandafter\expandafter\expandafter\BitSet@ShiftLeft
      \intcalcNum{#2}!{#1}%
    }%
  }%
}
%    \end{macrocode}
%    \end{macro}
%    \begin{macro}{\BitSet@ShiftLeft}
%    \begin{macrocode}
\def\BitSet@ShiftLeft#1!#2{%
  \ifcase\intcalcSgn{#1} %
  \or
    \begingroup
      \uccode`m=`0 %
    \uppercase\expandafter{\expandafter\endgroup
      \expandafter\edef\csname BS@#2\expandafter\endcsname
      \expandafter{%
        \romannumeral#1000\expandafter\BitSet@Space
        \csname BS@#2\endcsname
      }%
    }%
  \else
    \expandafter\BitSet@ShiftRight\BitSet@Gobble#1!{#2}%
  \fi
}
%    \end{macrocode}
%    \end{macro}
%
% \subsubsection{\cs{bitsetShiftRight}}
%
%    \begin{macro}{\bitsetShiftRight}
%    \begin{macrocode}
\def\bitsetShiftRight#1#2{%
  \BitSet@IfUndefined{#1}{%
    \bitsetReset{#1}%
  }{%
    \bitsetIsEmpty{#1}{%
    }{%
      \expandafter\expandafter\expandafter\BitSet@ShiftRight
      \intcalcNum{#2}!{#1}%
    }%
  }%
}
%    \end{macrocode}
%    \end{macro}
%    \begin{macro}{\BitSet@ShiftRight}
%    \begin{macrocode}
\def\BitSet@ShiftRight#1!#2{%
  \ifcase\intcalcSgn{#1} %
  \or
    \expandafter\edef\csname BS@#2\endcsname{%
      \expandafter\expandafter\expandafter\BitSet@Kill
      \csname BS@#2\expandafter\endcsname\expandafter\BitSet@Empty
      \expandafter=%
      \expandafter{\expandafter}\expandafter{\expandafter}%
      \romannumeral#1000!%
    }%
  \else
    \expandafter\BitSet@ShiftLeft\BitSet@Gobble#1!{#2}%
  \fi
}
%    \end{macrocode}
%    \end{macro}
%    \begin{macro}{\BitSet@Kill}
%    \begin{macrocode}
\def\BitSet@Kill#1#2=#3#4#5{%
  #3#4%
  \ifx#5!%
    \ifx#1\BitSet@Empty
      0%
    \else
      #1#2%
    \fi
  \else
    \ifx#1\BitSet@Empty
      0%
      \BitSet@AfterFiFi\BitSet@Cleanup
    \else
      \BitSet@Kill#2=%
    \fi
  \BitSet@Fi
}
%    \end{macrocode}
%    \end{macro}
%
% \subsection{Bit manipulation}\label{src:bitman}
%
%    \begin{macro}{\bitsetClear}
%    \begin{macrocode}
\def\bitsetClear{%
  \BitSet@CheckIndex\BitSet@Clear
}
%    \end{macrocode}
%    \end{macro}
%    \begin{macro}{\bitsetSet}
%    \begin{macrocode}
\def\bitsetSet{%
  \BitSet@CheckIndex\BitSet@Set
}
%    \end{macrocode}
%    \end{macro}
%    \begin{macro}{\bitsetFlip}
%    \begin{macrocode}
\def\bitsetFlip{%
  \BitSet@CheckIndex\BitSet@Flip
}
%    \end{macrocode}
%    \end{macro}
%    \begin{macro}{\bitsetSetValue}
%    \begin{macrocode}
\def\bitsetSetValue#1#2#3{%
  \expandafter\expandafter\expandafter\BitSet@SetValue
  \intcalcNum{#3}!{#1}{#2}%
}
%    \end{macrocode}
%    \end{macro}
%    \begin{macro}{\BitSet@SetValue}
%    |#1|: plain value\\
%    |#2|: BitSet\\
%    |#3|: Index
%    \begin{macrocode}
\def\BitSet@SetValue#1!{%
  \BitSet@CheckIndex{%
    \ifcase#1 %
      \expandafter\BitSet@Clear
    \or
      \expandafter\BitSet@Set
    \else
      \BitSet@ErrorInvalidBitValue{#1}%
      \expandafter\expandafter\expandafter\BitSet@Gobble
      \expandafter\BitSet@Gobble
    \fi
  }%
}
%    \end{macrocode}
%    \end{macro}
%    \begin{macro}{\BitSet@ErrorInvalidBitValue}
%    |#1|: Wrong bit value
%    \begin{macrocode}
\def\BitSet@ErrorInvalidBitValue#1{%
  \@PackageError{bitset}{%
    Invalid bit value (#1) not in range 0..1%
  }\@ehc
}
%    \end{macrocode}
%    \end{macro}
%
% \subsubsection{Clear operation}
%
%    \begin{macro}{\BitSet@Clear}
%    |#1|: BitSet\\
%    |#2|: plain and checked index
%    \begin{macrocode}
\def\BitSet@Clear#1#2{%
  \edef\BitSet@Temp{%
    \expandafter\expandafter\expandafter\BitSet@@Clear
    \csname BS@#1\expandafter\endcsname
    \expandafter\BitSet@Empty\expandafter=\expandafter!%
    \romannumeral#2000!%
  }%
  \expandafter\let\csname BS@#1\expandafter\endcsname
  \ifx\BitSet@Temp\BitSet@Empty
    \BitSet@Zero
  \else
    \BitSet@Temp
  \fi
}
%    \end{macrocode}
%    \end{macro}
%    \begin{macro}{\BitSet@@Clear}
%    \begin{macrocode}
\def\BitSet@@Clear#1#2=#3!#4{%
  \ifx#4!%
    \ifx#1\BitSet@Empty
    \else
      \ifx\BitSet@Empty#2%
      \else
        #30#2%
      \fi
    \fi
  \else
    \ifx#1\BitSet@Empty
      \BitSet@AfterFiFi\BitSet@Cleanup
    \else
      \ifx#10%
        \BitSet@AfterFiFiFi{%
          \BitSet@@Clear#2=#30!%
        }%
      \else
        #31%
        \BitSet@AfterFiFiFi{%
          \BitSet@@Clear#2=!%
        }%
      \fi
    \fi
  \BitSet@Fi
}
%    \end{macrocode}
%    \end{macro}
%
% \subsubsection{Set operation}
%
%    \begin{macro}{\BitSet@Set}
%    |#1|: BitSet\\
%    |#2|: plain and checked Index
%    \begin{macrocode}
\def\BitSet@Set#1#2{%
  \expandafter\edef\csname BS@#1\endcsname{%
    \expandafter\expandafter\expandafter\BitSet@@Set
    \csname BS@#1\expandafter\endcsname
    \expandafter\BitSet@Empty\expandafter=%
    \expandafter{\expandafter}\expandafter{\expandafter}%
    \romannumeral#2000!%
  }%
}
%    \end{macrocode}
%    \end{macro}
%    \begin{macro}{\BitSet@@Set}
%    \begin{macrocode}
\def\BitSet@@Set#1#2=#3#4#5{%
  #3#4%
  \ifx#5!%
    1#2%
  \else
    \ifx#1\BitSet@Empty
      0%
      \BitSet@AfterFiFi\BitSet@@@Set
    \else
      #1%
      \BitSet@@Set#2=%
    \fi
  \BitSet@Fi
}
%    \end{macrocode}
%    \end{macro}
%    \begin{macro}{\BitSet@@@Set}
%    \begin{macrocode}
\def\BitSet@@@Set#1{%
  \ifx#1!%
    1%
  \else
    0%
    \expandafter\BitSet@@@Set
  \fi
}
%    \end{macrocode}
%    \end{macro}
%
% \subsubsection{Flip operation}
%
%    \begin{macro}{\BitSet@Flip}
%    |#1|: BitSet\\
%    |#2|: plain and checked Index
%    \begin{macrocode}
\def\BitSet@Flip#1#2{%
  \edef\BitSet@Temp{%
    \expandafter\expandafter\expandafter\BitSet@@Flip
    \csname BS@#1\expandafter\endcsname
    \expandafter\BitSet@Empty\expandafter=\expandafter!%
    \romannumeral#2000!%
  }%
  \expandafter\let\csname BS@#1\expandafter\endcsname
  \ifx\BitSet@Temp\BitSet@Empty
    \BitSet@Zero
  \else
    \BitSet@Temp
  \fi
}
%    \end{macrocode}
%    \end{macro}
%    \begin{macro}{\BitSet@@Flip}
%    \begin{macrocode}
\def\BitSet@@Flip#1#2=#3!#4{%
  \ifx#4!%
    \ifx#11%
      \ifx\BitSet@Empty#2%
      \else
        #30#2%
      \fi
    \else
      #31#2%
    \fi
  \else
    \ifx#1\BitSet@Empty
      #30%
      \BitSet@AfterFiFi\BitSet@@@Set
    \else
      \ifx#10%
        \BitSet@AfterFiFiFi{%
          \BitSet@@Flip#2=#30!%
        }%
      \else
        #31%
        \BitSet@AfterFiFiFi{%
          \BitSet@@Flip#2=!%
        }%
      \fi
    \fi
  \BitSet@Fi
}
%    \end{macrocode}
%    \end{macro}
%
% \subsubsection{Range operators}
%
%    \begin{macro}{\bitsetClearRange}
%    \begin{macrocode}
\def\bitsetClearRange{%
  \BitSet@Range\BitSet@Clear
}
%    \end{macrocode}
%    \end{macro}
%    \begin{macro}{\bitsetSetRange}
%    \begin{macrocode}
\def\bitsetSetRange{%
  \BitSet@Range\BitSet@Set
}
%    \end{macrocode}
%    \end{macro}
%    \begin{macro}{\bitsetFlipRange}
%    \begin{macrocode}
\def\bitsetFlipRange{%
  \BitSet@Range\BitSet@Flip
}
%    \end{macrocode}
%    \end{macro}
%    \begin{macro}{\bitsetSetValueRange}
%    \begin{macrocode}
\def\bitsetSetValueRange#1#2#3#4{%
  \expandafter\expandafter\expandafter\BitSet@SetValueRange
  \intcalcNum{#4}!{#1}{#2}{#3}%
}
%    \end{macrocode}
%    \end{macro}
%    \begin{macro}{\BitSet@SetValueRange}
%    \begin{macrocode}
\def\BitSet@SetValueRange#1!#2#3#4{%
  \ifcase#1 %
    \BitSet@Range\BitSet@Clear{#2}{#3}{#4}%
  \or
    \BitSet@Range\BitSet@Set{#2}{#3}{#4}%
  \else
    \BitSet@ErrorInvalidBitValue{#1}%
  \fi
}
%    \end{macrocode}
%    \end{macro}
%
%    \begin{macro}{\BitSet@Range}
%    |#1|: clear/set/flip macro\\
%    |#2|: BitSet\\
%    |#3|: Index from\\
%    |#4|: Index to
%    \begin{macrocode}
\def\BitSet@Range#1#2#3#4{%
  \edef\BitSet@Temp{%
    \noexpand\BitSet@@Range\noexpand#1{#2}%
    \intcalcNum{#3}!\intcalcNum{#4}!%
  }%
  \BitSet@Temp
}
%    \end{macrocode}
%    \end{macro}
%    \begin{macro}{\BitSet@@Range}
%    |#1|: clear/set/flip macro\\
%    |#2|: BitSet\\
%    |#3|: Index from\\
%    |#4|: Index to
%    \begin{macrocode}
\def\BitSet@@Range#1#2#3!#4!{%
  \ifnum#3<0 %
    \BitSet@NegativeIndex#1{#2}#3!#4!0!#4!%
  \else
    \ifnum#4<0 %
      \BitSet@NegativeIndex#1{#2}#3!#4!#3!0!%
    \else
      \ifcase\intcalcCmp{#3}{#4} %
      \or
        \@PackageError{bitset}{%
          Wrong index numbers in range [#3..#4]\MessageBreak% hash-ok
          for clear/set/flip on bit set `#2'.\MessageBreak
          The lower index exceeds the upper index.\MessageBreak
          Canceling the operation as error recovery%
        }\@ehc
      \else
        \BitSet@@@Range#3!#4!#1{#2}%
      \fi
    \fi
  \fi
}
%    \end{macrocode}
%    \end{macro}
%    \begin{macro}{\BitSet@NegativeIndex}
%    \begin{macrocode}
\def\BitSet@NegativeIndex#1#2#3!#4!#5!#6!{%
  \@PackageError{bitset}{%
    Negative index in range [#3..#4]\MessageBreak % hash-ok
    for \string\bitset
    \ifx#1\BitSet@Clear
      Clear%
    \else
      \ifx#1\BitSet@Set
        Set%
      \else
        Flip%
      \fi
    \fi
    Range on bit set `#2'.\MessageBreak
    Using [#5..#6] as error recovery% hash-ok
  }\@ehc
  \BitSet@@Range#1{#2}#5!#6!%
}
%    \end{macrocode}
%    \end{macro}
%    \begin{macro}{\BitSet@@Range}
%    \begin{macrocode}
\def\BitSet@@@Range#1!#2!#3#4{%
  \ifnum#1<#2 %
    #3{#4}{#1}%
    \BitSet@AfterFi{%
      \expandafter\expandafter\expandafter\BitSet@@@Range
      \IntCalcInc#1!!#2!#3{#4}%
    }%
  \BitSet@Fi
}
%    \end{macrocode}
%    \end{macro}
%
% \subsection{Bit retrieval}\label{src:bitret}
%
% \subsubsection{\cs{bitsetGet}}
%
%    \begin{macro}{\bitsetGet}
%    \begin{macrocode}
\def\bitsetGet#1#2{%
  \number
  \expandafter\expandafter\expandafter\BitSet@Get
  \intcalcNum{#2}!{#1}%
}
%    \end{macrocode}
%    \end{macro}
%    \begin{macro}{\BitSet@Get}
%    |#1|: plain index\\
%    |#2|: BitSet
%    \begin{macrocode}
\def\BitSet@Get#1!#2{%
  \ifnum#1<0 %
    \BitSet@AfterFi{%
      0 \BitSetError:NegativeIndex%
    }%
  \else
    \BitSet@IfUndefined{#2}{0}{%
      \expandafter\expandafter\expandafter\BitSet@@Get
      \csname BS@#2\expandafter\endcsname
      \expandafter!\expandafter=%
      \expandafter{\expandafter}\expandafter{\expandafter}%
      \romannumeral\intcalcNum{#1}000!%
    }%
    \expandafter\BitSet@Space
  \BitSet@Fi
}
%    \end{macrocode}
%    \end{macro}
%    \begin{macro}{\BitSet@@Get}
%    \begin{macrocode}
\def\BitSet@@Get#1#2=#3#4#5{%
  #3#4%
  \ifx#5!%
    \ifx#1!%
      0%
    \else
      #1%
    \fi
  \else
    \ifx#1!%
      0%
      \BitSet@AfterFiFi\BitSet@Cleanup
    \else
      \BitSet@@Get#2=%
    \fi
  \BitSet@Fi
}
%    \end{macrocode}
%    \end{macro}
%
% \subsubsection{\cs{bitsetNextClearBit}, \cs{bitsetNextSetBit}}
%
%    \begin{macro}{\bitsetNextClearBit}
%    \begin{macrocode}
\def\bitsetNextClearBit#1#2{%
  \number
  \expandafter\expandafter\expandafter\BitSet@NextClearBit
  \intcalcNum{#2}!{#1} %
}
%    \end{macrocode}
%    \end{macro}
%    \begin{macro}{\BitSet@NextClearBit}
%    |#1|: Index\\
%    |#2|: BitSet
%    \begin{macrocode}
\def\BitSet@NextClearBit#1!#2{%
  \ifnum#1<0 %
    \BitSet@NextClearBit0!{#2}%
    \BitSet@AfterFi{%
      \expandafter\BitSet@Space
      \expandafter\BitSetError:NegativeIndex\romannumeral0%
    }%
  \else
    \bitsetIsEmpty{#2}{#1}{%
      \expandafter\BitSet@Skip
      \number#1\expandafter\expandafter\expandafter!%
      \csname BS@#2\endcsname!!!!!!!!!=%
      {\BitSet@@NextClearBit#1!}%
    }%
  \BitSet@Fi
}
%    \end{macrocode}
%    \end{macro}
%    \begin{macro}{\BitSet@@NextClearBit}
%    |#1|: index for next bit in |#2|\\
%    |#2|: next bit
%    \begin{macrocode}
\def\BitSet@@NextClearBit#1!#2{%
  \ifx#2!%
    #1%
  \else
    \ifx#20%
      #1%
      \BitSet@AfterFiFi\BitSet@Cleanup
    \else
      \BitSet@AfterFiFi{%
        \expandafter\expandafter\expandafter\BitSet@@NextClearBit
        \IntCalcInc#1!!%
      }%
    \fi
  \BitSet@Fi
}
%    \end{macrocode}
%    \end{macro}
%
%    \begin{macro}{\bitsetNextSetBit}
%    \begin{macrocode}
\def\bitsetNextSetBit#1#2{%
  \number
  \expandafter\expandafter\expandafter\BitSet@NextSetBit
  \intcalcNum{#2}!{#1} %
}
%    \end{macrocode}
%    \end{macro}
%    \begin{macro}{\BitSet@NextSetBit}
%    |#1|: Index\\
%    |#2|: BitSet
%    \begin{macrocode}
\def\BitSet@NextSetBit#1!#2{%
  \ifnum#1<0 %
    \BitSet@NextSetBit0!{#2}%
    \BitSet@AfterFi{%
      \expandafter\BitSet@Space
      \expandafter\BitSetError:NegativeIndex\romannumeral0%
    }%
  \else
    \bitsetIsEmpty{#2}{-1}{%
      \expandafter\BitSet@Skip
      \number#1\expandafter\expandafter\expandafter!%
      \csname BS@#2\endcsname!!!!!!!!!=%
      {\BitSet@@NextSetBit#1!}%
    }%
  \BitSet@Fi
}
%    \end{macrocode}
%    \end{macro}
%    \begin{macro}{\BitSet@@NextSetBit}
%    |#1|: index for next bit in |#2|\\
%    |#2|: next bit
%    \begin{macrocode}
\def\BitSet@@NextSetBit#1!#2{%
  \ifx#2!%
    -1%
  \else
    \ifx#21%
      #1%
      \BitSet@AfterFiFi\BitSet@Cleanup
    \else
      \BitSet@AfterFiFi{%
        \expandafter\expandafter\expandafter\BitSet@@NextSetBit
        \IntCalcInc#1!!%
      }%
    \fi
  \BitSet@Fi
}
%    \end{macrocode}
%    \end{macro}
%    \begin{macro}{\BitSet@Cleanup}
%    \begin{macrocode}
\def\BitSet@Cleanup#1!{}
%    \end{macrocode}
%    \end{macro}
%
%    \begin{macro}{\BitSet@Skip}
%    |#1|: number of bits to skip\\
%    |#2|: bits\\
%    |#3|: continuation code
%    \begin{macrocode}
\def\BitSet@Skip#1!#2{%
  \ifx#2!%
    \BitSet@AfterFi{%
      \BitSet@SkipContinue%
    }%
  \else
    \ifcase#1 %
      \BitSet@AfterFiFi{%
        \BitSet@SkipContinue#2%
      }%
    \or
      \BitSet@AfterFiFi\BitSet@SkipContinue
    \or
      \BitSet@AfterFiFi{%
        \expandafter\BitSet@SkipContinue\BitSet@Gobble
      }%
    \else
      \ifnum#1>8 %
        \BitSet@AfterFiFiFi{%
          \expandafter\BitSet@Skip
          \number\IntCalcSub#1!8!\expandafter!%
          \BitSet@GobbleSeven
        }%
      \else
        \BitSet@AfterFiFiFi{%
          \expandafter\expandafter\expandafter\BitSet@Skip
          \IntCalcDec#1!!%
        }%
      \fi
    \fi
  \BitSet@Fi
}
%    \end{macrocode}
%    \end{macro}
%    \begin{macro}{\BitSet@SkipContinue}
%    |#1|: remaining bits\\
%    |#2|: continuation code
%    \begin{macrocode}
\def\BitSet@SkipContinue#1!#2=#3{%
  #3#1!%
}
%    \end{macrocode}
%    \end{macro}
%    \begin{macro}{\BitSet@GobbleSeven}
%    \begin{macrocode}
\def\BitSet@GobbleSeven#1#2#3#4#5#6#7{}
%    \end{macrocode}
%    \end{macro}
%
% \subsubsection{\cs{bitsetGetSetBitList}}
%
%    \begin{macro}{\bitsetGetSetBitList}
%    It's just a wrapper for \cs{bitsetNextSetBit}.
%    \begin{macrocode}
\def\bitsetGetSetBitList#1{%
  \romannumeral0%
  \bitsetIsEmpty{#1}{ }{%
    \expandafter\BitSet@GetSetBitList
    \number\BitSet@NextSetBit0!{#1}!{#1}{}!%
  }%
}
%    \end{macrocode}
%    \end{macro}
%    \begin{macro}{\BitSet@GetSetBitList}
%    |#1|: found index\\
%    |#2|: BitSet\\
%    |#3|: comma
%    |#4|: result
%    \begin{macrocode}
\def\BitSet@GetSetBitList#1!#2#3#4!{%
  \ifnum#1<0 %
    \BitSet@AfterFi{ #4}%
  \else
    \BitSet@AfterFi{%
      \expandafter\BitSet@GetSetBitList\number
      \expandafter\expandafter\expandafter\BitSet@NextSetBit
      \IntCalcInc#1!!{#2}!{#2},#4#3#1!%
    }%
  \BitSet@Fi
}
%    \end{macrocode}
%    \end{macro}
%
% \subsection{Bit set properties}\label{src:prop}
%
%    \begin{macro}{\bitsetSize}
%    \begin{macrocode}
\def\bitsetSize#1{%
  \number
  \BitSet@IfUndefined{#1}{0 }{%
    \expandafter\expandafter\expandafter\BitSet@Size
    \expandafter\expandafter\expandafter1%
    \expandafter\expandafter\expandafter!%
    \csname BS@#1\endcsname!0!%
  }%
}
%    \end{macrocode}
%    \end{macro}
%    \begin{macro}{\BitSet@Size}
%    |#1|: counter\\
%    |#2#3|: bits\\
%    |#4|: result
%    \begin{macrocode}
\def\BitSet@Size#1!#2#3!#4!{%
  \ifx#21%
    \ifx\\#3\\%
      \BitSet@AfterFiFi{#1 }%
    \else
      \BitSet@AfterFiFi{%
        \expandafter\expandafter\expandafter\BitSet@Size
        \IntCalcInc#1!!#3!#1!%
      }%
    \fi
  \else
    \ifx\\#3\\%
      \BitSet@AfterFiFi{#4 }%
    \else
      \BitSet@AfterFiFi{%
        \expandafter\expandafter\expandafter\BitSet@Size
        \IntCalcInc#1!!#3!#4!%
      }%
    \fi
  \fi
  \BitSet@Fi
}
%    \end{macrocode}
%    \end{macro}
%
%    \begin{macro}{\bitsetCardinality}
%    \begin{macrocode}
\def\bitsetCardinality#1{%
  \number
  \BitSet@IfUndefined{#1}{0 }{%
    \expandafter\expandafter\expandafter\BitSet@Cardinality
    \expandafter\expandafter\expandafter0%
    \expandafter\expandafter\expandafter!%
    \csname BS@#1\endcsname!%
  }%
}
%    \end{macrocode}
%    \end{macro}
%    \begin{macro}{\BitSet@Cardinality}
%    |#1|: result\\
%    |#2#3|: bits
%    \begin{macrocode}
\def\BitSet@Cardinality#1!#2#3!{%
  \ifx#21%
    \ifx\\#3\\%
      \BitSet@AfterFiFi{\IntCalcInc#1! }%
    \else
      \BitSet@AfterFiFi{%
        \expandafter\expandafter\expandafter\BitSet@Cardinality
        \IntCalcInc#1!!#3!%
      }%
    \fi
  \else
    \ifx\\#3\\%
      \BitSet@AfterFiFi{#1 }%
    \else
      \BitSet@AfterFiFi{%
        \BitSet@Cardinality#1!#3!%
      }%
    \fi
  \fi
  \BitSet@Fi
}
%    \end{macrocode}
%    \end{macro}
%
% \subsection{Queries}\label{src:query}
%
%    \begin{macro}{\bitsetIsDefined}
%    \begin{macrocode}
\def\bitsetIsDefined#1{%
  \BitSet@IfUndefined{#1}%
  \BitSet@SecondOfTwo
  \BitSet@FirstOfTwo
}
%    \end{macrocode}
%    \end{macro}
%    \begin{macro}{\bitsetIsEmpty}
%    \begin{macrocode}
\def\bitsetIsEmpty#1{%
  \BitSet@IfUndefined{#1}\BitSet@FirstOfTwo{%
    \expandafter\ifx\csname BS@#1\endcsname\BitSet@Zero
      \expandafter\BitSet@FirstOfTwo
    \else
      \expandafter\BitSet@SecondOfTwo
    \fi
  }%
}
%    \end{macrocode}
%    \end{macro}
%    \begin{macro}{\BitSet@Zero}
%    \begin{macrocode}
\def\BitSet@Zero{0}
%    \end{macrocode}
%    \end{macro}
%
%    \begin{macro}{\bitsetQuery}
%    \begin{macrocode}
\def\bitsetQuery#1#2{%
  \ifnum\bitsetGet{#1}{#2}=1 %
    \expandafter\BitSet@FirstOfTwo
  \else
    \expandafter\BitSet@SecondOfTwo
  \fi
}
%    \end{macrocode}
%    \end{macro}
%
%    \begin{macro}{\bitsetEquals}
%    \begin{macrocode}
\def\bitsetEquals#1#2{%
  \BitSet@IfUndefined{#1}{%
    \BitSet@IfUndefined{#2}\BitSet@FirstOfTwo\BitSet@SecondOfTwo
  }{%
    \BitSet@IfUndefined{#2}\BitSet@SecondOfTwo{%
      \expandafter\ifx\csname BS@#1\expandafter\endcsname
                      \csname BS@#2\endcsname
        \expandafter\BitSet@FirstOfTwo
      \else
        \expandafter\BitSet@SecondOfTwo
      \fi
    }%
  }%
}
%    \end{macrocode}
%    \end{macro}
%
%    \begin{macro}{\bitsetIntersects}
%    \begin{macrocode}
\def\bitsetIntersects#1#2{%
  \bitsetIsEmpty{#1}\BitSet@SecondOfTwo{%
    \bitsetIsEmpty{#2}\BitSet@SecondOfTwo{%
      \expandafter\expandafter\expandafter\BitSet@Intersects
      \csname BS@#1\expandafter\expandafter\expandafter\endcsname
      \expandafter\expandafter\expandafter!%
      \csname BS@#2\endcsname!%
    }%
  }%
}
%    \end{macrocode}
%    \end{macro}
%    \begin{macro}{\BitSet@Intersects}
%    \begin{macrocode}
\def\BitSet@Intersects#1#2!#3#4!{%
  \ifnum#1#3=11 %
    \BitSet@AfterFi\BitSet@FirstOfTwo
  \else
    \ifx\\#2\\%
      \BitSet@AfterFiFi\BitSet@SecondOfTwo
    \else
      \ifx\\#4\\%
        \BitSet@AfterFiFiFi\BitSet@SecondOfTwo
      \else
        \BitSet@AfterFiFiFi{%
          \BitSet@Intersects#2!#4!%
        }%
      \fi
    \fi
  \BitSet@Fi
}
%    \end{macrocode}
%    \end{macro}
%
%    \begin{macrocode}
\BitSet@AtEnd%
%</package>
%    \end{macrocode}
%
% \section{Test}
%
% \subsection{Catcode checks for loading}
%
%    \begin{macrocode}
%<*test1>
%    \end{macrocode}
%    \begin{macrocode}
\catcode`\{=1 %
\catcode`\}=2 %
\catcode`\#=6 %
\catcode`\@=11 %
\expandafter\ifx\csname count@\endcsname\relax
  \countdef\count@=255 %
\fi
\expandafter\ifx\csname @gobble\endcsname\relax
  \long\def\@gobble#1{}%
\fi
\expandafter\ifx\csname @firstofone\endcsname\relax
  \long\def\@firstofone#1{#1}%
\fi
\expandafter\ifx\csname loop\endcsname\relax
  \expandafter\@firstofone
\else
  \expandafter\@gobble
\fi
{%
  \def\loop#1\repeat{%
    \def\body{#1}%
    \iterate
  }%
  \def\iterate{%
    \body
      \let\next\iterate
    \else
      \let\next\relax
    \fi
    \next
  }%
  \let\repeat=\fi
}%
\def\RestoreCatcodes{}
\count@=0 %
\loop
  \edef\RestoreCatcodes{%
    \RestoreCatcodes
    \catcode\the\count@=\the\catcode\count@\relax
  }%
\ifnum\count@<255 %
  \advance\count@ 1 %
\repeat

\def\RangeCatcodeInvalid#1#2{%
  \count@=#1\relax
  \loop
    \catcode\count@=15 %
  \ifnum\count@<#2\relax
    \advance\count@ 1 %
  \repeat
}
\def\RangeCatcodeCheck#1#2#3{%
  \count@=#1\relax
  \loop
    \ifnum#3=\catcode\count@
    \else
      \errmessage{%
        Character \the\count@\space
        with wrong catcode \the\catcode\count@\space
        instead of \number#3%
      }%
    \fi
  \ifnum\count@<#2\relax
    \advance\count@ 1 %
  \repeat
}
\def\space{ }
\expandafter\ifx\csname LoadCommand\endcsname\relax
  \def\LoadCommand{\input bitset.sty\relax}%
\fi
\def\Test{%
  \RangeCatcodeInvalid{0}{47}%
  \RangeCatcodeInvalid{58}{64}%
  \RangeCatcodeInvalid{91}{96}%
  \RangeCatcodeInvalid{123}{255}%
  \catcode`\@=12 %
  \catcode`\\=0 %
  \catcode`\%=14 %
  \LoadCommand
  \RangeCatcodeCheck{0}{36}{15}%
  \RangeCatcodeCheck{37}{37}{14}%
  \RangeCatcodeCheck{38}{47}{15}%
  \RangeCatcodeCheck{48}{57}{12}%
  \RangeCatcodeCheck{58}{63}{15}%
  \RangeCatcodeCheck{64}{64}{12}%
  \RangeCatcodeCheck{65}{90}{11}%
  \RangeCatcodeCheck{91}{91}{15}%
  \RangeCatcodeCheck{92}{92}{0}%
  \RangeCatcodeCheck{93}{96}{15}%
  \RangeCatcodeCheck{97}{122}{11}%
  \RangeCatcodeCheck{123}{255}{15}%
  \RestoreCatcodes
}
\Test
\csname @@end\endcsname
\end
%    \end{macrocode}
%    \begin{macrocode}
%</test1>
%    \end{macrocode}
%
% \subsection{Macro tests}
%
% \subsubsection{Preamble}
%
%    \begin{macrocode}
%<*test2>
\NeedsTeXFormat{LaTeX2e}
\nofiles
\documentclass{article}
\makeatletter
%<*noetex>
\let\SavedNumexpr\numexpr
\let\SavedIfcsname\ifcsname
\let\SavedCurrentgrouplevel\currentgrouplevel
\def\ETeXDisable{%
  \let\ifcsname\@undefined
  \let\numexpr\@undefined
  \let\currentgrouplevel\@undefined
}
\ETeXDisable
%</noetex>
\makeatletter
\chardef\BitSet@TestMode=1 %
\makeatother
\usepackage{bitset}[2011/01/30]
%<*noetex>
\def\ETeXEnable{%
  \let\numexpr\SavedNumexpr
  \let\ifcsname\SavedIfcsname
  \let\currentgrouplevel\SavedCurrentgrouplevel
}
\ETeXEnable
%</noetex>
\usepackage{qstest}
\IncludeTests{*}
\LogTests{log}{*}{*}
\makeatletter
%    \end{macrocode}
%
% \subsubsection{Time}
%
%    \begin{macrocode}
\begingroup\expandafter\expandafter\expandafter\endgroup
\expandafter\ifx\csname pdfresettimer\endcsname\relax
\else
  \newcount\SummaryTime
  \newcount\TestTime
  \SummaryTime=\z@
  \newcommand*{\PrintTime}[2]{%
    \typeout{%
      [Time #1: \strip@pt\dimexpr\number#2sp\relax\space s]%
    }%
  }%
  \newcommand*{\StartTime}[1]{%
    \renewcommand*{\TimeDescription}{#1}%
    \pdfresettimer
  }%
  \newcommand*{\TimeDescription}{}%
  \newcommand*{\StopTime}{%
    \TestTime=\pdfelapsedtime
    \global\advance\SummaryTime\TestTime
    \PrintTime\TimeDescription\TestTime
  }%
  \let\saved@qstest\qstest
  \let\saved@endqstest\endqstest
  \def\qstest#1#2{%
    \saved@qstest{#1}{#2}%
    \StartTime{#1}%
  }%
  \def\endqstest{%
    \StopTime
    \saved@endqstest
  }%
  \AtEndDocument{%
    \PrintTime{summary}\SummaryTime
  }%
\fi
%    \end{macrocode}
%
% \subsubsection{Detection of unwanted space}
%
%    \begin{macrocode}
\let\orig@qstest\qstest
\let\orig@endqstest\endqstest
\def\qstest#1#2{%
  \orig@qstest{#1}{#2}%
  \setbox0\hbox\bgroup\begingroup\ignorespaces
}
\def\endqstest{%
  \endgroup\egroup
  \Expect*{\the\wd0}{0.0pt}%
  \orig@endqstest
}
%    \end{macrocode}
%
% \subsubsection{Test macros}
%
%    \begin{macrocode}
\newcounter{Test}

\def\TestError#1#2{%
  \begingroup
    \setcounter{Test}{0}%
    \sbox0{%
      \def\@PackageError##1##2##3{%
        \stepcounter{Test}%
        \begingroup
          \let\MessageBreak\relax
%<*noetex>
          \ETeXEnable
%</noetex>
          \Expect{##1}{bitset}%
          \Expect*{##2}*{#1}%
        \endgroup
      }%
%<*noetex>
      \ETeXDisable
%</noetex>
      #2%
    }%
    \Expect*{\theTest}{1}%
    \Expect*{\the\wd0}{0.0pt}%
  \endgroup
}

\def\TestErrorNegativeIndex#1#2{%
  \TestError{Invalid negative index (#1)}{#2}%
}

\def\TestGetterUndefined#1{%
  \CheckUndef{dummy}%
  \expandafter\expandafter\expandafter\Expect
  \expandafter\expandafter\expandafter{#1{dummy}}{0}%
}

\def\ExpectBitSet#1#2{%
  \expandafter\expandafter\expandafter\Expect
  \expandafter\expandafter\expandafter
  {\csname BS@#1\endcsname}*{#2}%
}
\def\Check#1#2{%
  \ExpectBitSet{#1}{#2}%
}
\def\CheckUndef#1{%
  \begingroup
    \Expect*{%
      \expandafter
      \ifx\csname BS@#1\endcsname\relax true\else false\fi
    }{true}%
  \endgroup
}
\def\RevCheck#1#2{%
  \ExpectBitSet{#1}{\Reverse#2!!}%
}
\def\Set#1#2{%
  \expandafter\def\csname BS@#1\endcsname{#2}%
}
\def\RevSet#1#2{%
  \expandafter\edef\csname BS@#1\endcsname{%
    \Reverse#2!!%
  }%
}
\def\Reverse#1#2!#3!{%
  \ifx\\#2\\%
    #1#3%
    \expandafter\@gobble
  \else
    \expandafter\@firstofone
  \fi
  {\Reverse#2!#1#3!}%
}
%    \end{macrocode}
%
% \subsubsection{Test sets}
%
%    \begin{macrocode}
\begin{qstest}{Let}{Let}
  \CheckUndef{abc}%
  \CheckUndef{xyz}%
  \bitsetLet{xyz}{abc}%
  \CheckUndef{abc}%
  \Check{xyz}{0}%
  \Set{abc}{1}%
  \Check{abc}{1}%
  \Check{xyz}{0}%
  \bitsetLet{xyz}{abc}%
  \Check{abc}{1}%
  \Check{xyz}{1}%
  \Set{xyz}{11}%
  \Check{abc}{1}%
  \Check{xyz}{11}%
\end{qstest}

\begin{qstest}{Reset}{Reset}
  \bitsetReset{xyz}%
  \Check{xyz}{0}%
  \bitsetReset{abc}%
  \Check{abc}{0}%
  \Set{abc}{10101}%
  \bitsetReset{abc}%
  \Check{abc}{0}%
\end{qstest}

\begin{qstest}{Get/Query}{Get/Query}
  \expandafter\expandafter\expandafter\Expect
  \expandafter\expandafter\expandafter{%
    \bitsetGet{dummy}{0}%
  }{0}%
  \begingroup
    \expandafter\def\csname BitSetError:NegativeIndex\endcsname{}%
    \Set{abc}{1}%
    \Expect*{\bitsetQuery{abc}{-1}{true}{false}}{false}%
  \endgroup
  \def\Test#1#2#3{%
    \Set{abc}{#1}%
    \expandafter\expandafter\expandafter\Expect
    \expandafter\expandafter\expandafter{\bitsetGet{abc}{#2}}{#3}%
    \Expect*{\bitsetQuery{abc}{#2}{true}{false}}%
           *{\ifcase#3 false\or true\else error\fi}%
  }%
  \Test{1}{100}{0}%
  \Test{0}{0}{0}%
  \Test{1}{0}{1}%
  \Test{11}{1}{1}%
  \Test{111}{1}{1}%
  \Test{101}{1}{0}%
  \Test{101}{2}{1}%
  \Test{10100110011}{10}{1}%
\end{qstest}

\begin{qstest}{Size}{Size}
  \TestGetterUndefined\bitsetSize
  \def\Test#1#2{%
    \Set{abc}{#1}%
    \expandafter\expandafter\expandafter\Expect
    \expandafter\expandafter\expandafter{\bitsetSize{abc}}{#2}%
  }%
  \Test{0}{0}%
  \Test{1}{1}%
  \Test{00}{0}%
  \Test{0000000}{0}%
  \Test{10}{1}%
  \Test{01}{2}%
  \Test{11}{2}%
  \Test{010}{2}%
  \Test{011}{3}%
  \Test{100110011}{9}%
  \Test{0000011111000001111100000}{20}%
  \Test{000000000000000000000000011111111111111111111}{45}%
\end{qstest}

\begin{qstest}{Cardinality}{Cardinality}
  \TestGetterUndefined\bitsetCardinality
  \def\Test#1#2{%
    \Set{abc}{#1}%
    \expandafter\expandafter\expandafter\Expect
    \expandafter\expandafter\expandafter{%
      \bitsetCardinality{abc}%
    }{#2}%
  }%
  \Test{0}{0}%
  \Test{1}{1}%
  \Test{00}{0}%
  \Test{0000000}{0}%
  \Test{10}{1}%
  \Test{01}{1}%
  \Test{11}{2}%
  \Test{010}{1}%
  \Test{011}{2}%
  \Test{100110011}{5}%
  \Test{0000011111000001111100000}{10}%
  \Test{000000000000000000000000011111111111111111111}{20}%
\end{qstest}

\begin{qstest}{NextClearBit/NextSetBit}{NextClearBit/NextSetBit}
  \def\Test#1#2{%
    \expandafter\expandafter\expandafter\Expect
    \expandafter\expandafter\expandafter{%
      \TestOp{abc}{#1}%
    }{#2}%
  }%
  \def\Clear{\let\TestOp\bitsetNextClearBit}%
  \def\Set{\let\TestOp\bitsetNextSetBit}%
  \begingroup
    \catcode`\:=11 %
    \bitsetSetBin{abc}{1}%
    \Clear
    \Test{-1}{1\BitSetError:NegativeIndex}%
    \Set
    \Test{-1}{0\BitSetError:NegativeIndex}%
  \endgroup
  \let\BS@abc\@undefined
  \Clear
  \Test{0}{0}%
  \Test{1}{1}%
  \Test{2}{2}%
  \Test{100}{100}%
  \Set
  \Test{0}{-1}%
  \Test{1}{-1}%
  \Test{100}{-1}%
  \bitsetReset{abc}%
  \Clear
  \Test{0}{0}%
  \Test{1}{1}%
  \Test{2}{2}%
  \Test{100}{100}%
  \Set
  \Test{0}{-1}%
  \Test{1}{-1}%
  \Test{100}{-1}%
  \bitsetSetBin{abc}{1}%
  \Clear
  \Test{0}{1}%
  \Test{1}{1}%
  \Test{2}{2}%
  \Test{100}{100}%
  \Set
  \Test{0}{0}%
  \Test{1}{-1}%
  \Test{100}{-1}%
  \bitsetSetBin{abc}{111000111000111000111}%
  \Clear
  \Test{0}{3}%
  \Test{1}{3}%
  \Test{2}{3}%
  \Test{3}{3}%
  \Test{4}{4}%
  \Test{5}{5}%
  \Test{6}{9}%
  \Test{7}{9}%
  \Test{8}{9}%
  \Test{9}{9}%
  \Test{10}{10}%
  \Test{11}{11}%
  \Test{12}{15}%
  \Test{13}{15}%
  \Test{14}{15}%
  \Test{15}{15}%
  \Test{16}{16}%
  \Test{17}{17}%
  \Test{18}{21}%
  \Test{19}{21}%
  \Test{20}{21}%
  \Test{21}{21}%
  \Test{22}{22}%
  \Test{100}{100}%
  \Set
  \Test{0}{0}%
  \Test{1}{1}%
  \Test{2}{2}%
  \Test{3}{6}%
  \Test{4}{6}%
  \Test{5}{6}%
  \Test{6}{6}%
  \Test{7}{7}%
  \Test{8}{8}%
  \Test{9}{12}%
  \Test{10}{12}%
  \Test{11}{12}%
  \Test{12}{12}%
  \Test{13}{13}%
  \Test{14}{14}%
  \Test{15}{18}%
  \Test{16}{18}%
  \Test{17}{18}%
  \Test{18}{18}%
  \Test{19}{19}%
  \Test{20}{20}%
  \Test{21}{-1}%
  \Test{22}{-1}%
  \Test{100}{-1}%
  \bitsetSetBin{abc}{1111111}%
  \Clear
  \Test{6}{7}%
  \Test{7}{7}%
  \Test{8}{8}%
  \Test{100}{100}%
  \Set
  \Test{6}{6}%
  \Test{7}{-1}%
  \Test{8}{-1}%
  \Test{100}{-1}%
  \bitsetSetBin{abc}{11111111}%
  \Clear
  \Test{7}{8}%
  \Test{8}{8}%
  \Test{9}{9}%
  \Test{100}{100}%
  \Set
  \Test{7}{7}%
  \Test{8}{-1}%
  \Test{9}{-1}%
  \Test{100}{-1}%
  \bitsetSetBin{abc}{111111111}%
  \Clear
  \Test{8}{9}%
  \Test{9}{9}%
  \Test{10}{10}%
  \Test{100}{100}%
  \Set
  \Test{8}{8}%
  \Test{9}{-1}%
  \Test{10}{-1}%
  \Test{100}{-1}%
  \bitsetSetBin{abc}{1111111111}%
  \Clear
  \Test{9}{10}%
  \Test{10}{10}%
  \Test{11}{11}%
  \Test{100}{100}%
  \Set
  \Test{9}{9}%
  \Test{10}{-1}%
  \Test{11}{-1}%
  \Test{100}{-1}%
\end{qstest}

\begin{qstest}{GetSetBitList}{GetSetBitList}
  \let\BS@abc\@undefined
  \expandafter\expandafter\expandafter\Expect
  \expandafter\expandafter\expandafter{%
    \bitsetGetSetBitList{abc}%
  }{}%
  \def\Test#1#2{%
    \bitsetSetBin{abc}{#1}%
    \expandafter\expandafter\expandafter\Expect
    \expandafter\expandafter\expandafter{%
      \bitsetGetSetBitList{abc}%
    }{#2}%
  }%
  \Test{0}{}%
  \Test{1}{0}%
  \Test{10}{1}%
  \Test{11}{0,1}%
  \Test{10110100}{2,4,5,7}%
  \Test{101101001010011}{0,1,4,6,9,11,12,14}%
\end{qstest}

\begin{qstest}{GetDec}{GetDec}
  \TestGetterUndefined\bitsetGetDec
  \def\Test#1#2{%
    \RevSet{abc}{#1}%
%<*noetex>
    \begingroup\expandafter\expandafter\expandafter\endgroup
%</noetex>
    \expandafter\expandafter\expandafter\Expect
    \expandafter\expandafter\expandafter{%
      \bitsetGetDec{abc}%
    }{#2}%
  }%
  \Test{0}{0}%
  \Test{1}{1}%
  \Test{10}{2}%
  \Test{11}{3}%
  \Test{100}{4}%
  \Test{101}{5}%
  \Test{110}{6}%
  \Test{111}{7}%
  \Test{1000}{8}%
  \Test{000111}{7}%
  \Test{1111111111111111%
        111111111111111}{2147483647}%
  \Test{0001111111111111111%
        111111111111111}{2147483647}%
  \Test{1000000000000000%
        0000000000000000}{2147483648}%
  \Test{1000000000000000%
        00000000000000000}{4294967296}%
  \Test{0001000000000000000%
        00000000000000000}{4294967296}%
  \Test{1100000000000000%
        00000000000000011}{6442450947}%
\end{qstest}

\begin{qstest}{Clear}{Clear}
  \def\Test#1#2#3{%
    \RevSet{abc}{#1}%
    \bitsetClear{abc}{#2}%
    \Expect*{\BS@abc}*{\Reverse#3!!}%
  }%
  \bitsetClear{abc}{2}%
  \RevCheck{abc}{0}%
  \TestErrorNegativeIndex{-1}{\bitsetClear{abc}{-1}}%
  \RevCheck{abc}{0}%
  \Test{0}{0}{0}%
  \Test{1}{0}{0}%
  \Test{111}{1}{101}%
  \Test{111}{30}{111}%
  \Test{0000111}{5}{0000111}% 111 would also be ok
  \Test{10000111}{5}{10000111}%
  \Test{1001001}{3}{1000001}%
\end{qstest}

\begin{qstest}{Set}{Set}
  \def\Test#1#2#3{%
    \RevSet{abc}{#1}%
    \bitsetSet{abc}{#2}%
    \Expect*{\BS@abc}*{\Reverse#3!!}%
  }%
  \bitsetSet{abc}{2}%
  \RevCheck{abc}{100}%
  \TestErrorNegativeIndex{-1}{\bitsetSet{abc}{-1}}%
  \RevCheck{abc}{100}%
  \Test{0}{0}{1}%
  \Test{1}{0}{1}%
  \Test{100}{1}{110}%
  \Test{111}{1}{111}%
  \Test{11}{1}{11}%
  \Test{11}{2}{111}%
  \Test{11}{3}{1011}%
  \Test{111}{10}{10000000111}%
  \Test{0000111}{5}{0100111}% 100111 would also be ok
  \Test{10000111}{5}{10100111}%
  \Test{1000001}{3}{1001001}%
  \Test{1001001}{3}{1001001}%
\end{qstest}

\begin{qstest}{Flip}{Flip}
  \def\Test#1#2#3{%
    \RevSet{abc}{#1}%
    \bitsetFlip{abc}{#2}%
    \Expect*{\BS@abc}*{\Reverse#3!!}%
  }%
  \bitsetFlip{abc}{2}%
  \RevCheck{abc}{100}%
  \TestErrorNegativeIndex{-1}{\bitsetFlip{abc}{-1}}%
  \RevCheck{abc}{100}%
  \Test{0}{0}{1}%
  \Test{1}{0}{0}%
  \Test{0}{2}{100}%
  \Test{100}{1}{110}%
  \Test{111}{1}{101}%
  \Test{11}{1}{1}%
  \Test{11}{2}{111}%
  \Test{11}{3}{1011}%
  \Test{111}{10}{10000000111}%
  \Test{0000111}{5}{0100111}% 100111 would also be ok
  \Test{10000111}{5}{10100111}%
  \Test{1000001}{3}{1001001}%
  \Test{1001001}{3}{1000001}%
  \Test{11111}{2}{11011}%
\end{qstest}

\begin{qstest}{SetValue}{SetValue}
  \def\Test#1#2{%
    \TestError{Invalid bit value (#2) not in range 0..1}{%
      \bitsetSetValue{abc}{#1}{#2}%
    }%
  }%
  \Test{0}{-1}%
  \Test{0}{2}%
  \Test{0}{10}%
  \def\Test#1#2#3{%
    \let\BS@abc\@undefined
    \bitsetSetValue{abc}{#1}{#2}%
    \bitsetSetBin{result}{#3}%
    \Expect*{\BS@abc}*{\BS@result}%
  }%
  \Test{0}{0}{0}%
  \Test{0}{1}{1}%
  \Test{1}{0}{0}%
  \Test{1}{1}{10}%
  \def\Test#1#2#3#4{%
    \bitsetSetBin{abc}{#1}%
    \bitsetSetBin{result}{#4}%
    \bitsetSetValue{abc}{#2}{#3}%
    \Expect*{\BS@abc}*{\BS@result}%
  }%
  \Test{0}{0}{0}{0}%
  \Test{0}{0}{0}{0}%
  \Test{0}{0}{1}{1}%
  \Test{0}{1}{0}{0}%
  \Test{0}{1}{1}{10}%
  \Test{1010}{2}{1}{1110}%
  \Test{1010}{4}{1}{11010}%
  \Test{1010}{6}{1}{1001010}%
  \Test{1010}{1}{0}{1000}%
  \Test{1010}{2}{0}{1010}%
  \Test{1010}{3}{0}{10}%
  \Test{1010}{4}{0}{1010}%
  \Test{1010}{6}{0}{1010}%
  \Test{1010}{2}{\csname iffalse\endcsname 0\else 1\fi}{1110}%
  \Test{1010}{1}{\csname iffalse\endcsname 1\else 0\fi}{1000}%
\end{qstest}

\begin{qstest}{IsDefined}{IsDefined}
  \let\BS@abc\@undefined
  \Expect*{\bitsetIsDefined{abc}{true}{false}}{false}%
  \bitsetReset{abc}%
  \Expect*{\bitsetIsDefined{abc}{true}{false}}{true}%
\end{qstest}

\begin{qstest}{IsEmpty}{IsEmpty}
  \let\BS@abc\@undefined
  \Expect*{\bitsetIsEmpty{abc}{true}{false}}{true}%
  \bitsetReset{abc}%
  \Expect*{\bitsetIsEmpty{abc}{true}{false}}{true}%
  \bitsetSet{abc}{1}%
  \Expect*{\bitsetIsEmpty{abc}{true}{false}}{false}%
\end{qstest}

\begin{qstest}{Equals}{Equals}
  \def\Test#1#2#3{%
    \Expect*{\bitsetEquals{#1}{#2}{true}{false}}{#3}%
  }%
  \let\BS@abc\@undefined
  \Test{abc}{abc}{true}%
  \Test{abc}{foo}{true}%
  \Test{foo}{abc}{true}%
  \bitsetReset{abc}%
  \Test{abc}{abc}{true}%
  \Test{abc}{foo}{false}%
  \Test{foo}{abc}{false}%
  \bitsetReset{foo}%
  \Test{abc}{foo}{true}%
  \Test{foo}{abc}{true}%
  \bitsetSet{abc}{4}%
  \Test{abc}{foo}{false}%
  \Test{foo}{abc}{false}%
  \bitsetFlip{foo}{4}%
  \Test{abc}{foo}{true}%
  \Test{foo}{abc}{true}%
\end{qstest}

\begin{qstest}{Intersects}{Intersects}
  \def\Test#1{%
    \Expect*{\bitsetIntersects{abc}{foo}{true}{false}}{#1}%
  }%
  \let\BS@abc\@undefined
  \let\BS@foo\@undefined
  \Test{false}%
  \Set{abc}{0}%
  \Test{false}%
  \Set{foo}{0}%
  \Test{false}%
  \let\BS@abc\@undefined
  \Test{false}%
  \Set{foo}{1}%
  \Test{false}%
  \Set{abc}{0}%
  \Test{false}%
  \Set{abc}{1}%
  \Test{true}%
  \let\BS@foo\@undefined
  \Test{false}%
  \Set{foo}{0}%
  \Test{false}%
  \def\Test#1#2#3{%
    \bitsetSetBin{abc}{#1}%
    \bitsetSetBin{foo}{#2}%
    \Expect*{\bitsetIntersects{abc}{foo}{true}{false}}{#3}%
  }%
  \Test{1010}{0101}{false}%
  \Test{0}{10}{false}%
  \Test{1}{11}{true}%
  \Test{11}{1}{true}%
  \Test{10}{1}{false}%
\end{qstest}

\begin{qstest}{And/AndNot/Or/Xor}{And/AndNot/Or/Xor}
  \def\@Test#1#2#3#4#5{%
    \begingroup
      #5%
      \begingroup
        \let\BS@foo\@undefined
        \csname bitset#1\endcsname{abc}{foo}%
        \CheckUndef{foo}%
        \Check{abc}{#2}%
      \endgroup
      \begingroup
        \bitsetReset{foo}%
        \csname bitset#1\endcsname{abc}{foo}%
        \Check{foo}{0}%
        \Check{abc}{#3}%
      \endgroup
      \begingroup
        \def\BS@foo{0101}%
        \csname bitset#1\endcsname{abc}{foo}%
        \Check{foo}{0101}%
        \Check{abc}{#4}%
      \endgroup
    \endgroup
  }%
  \def\Test#1{%
    \def\Op{#1}%
    \Test@
  }%
  \def\Test@#1#2#3#4#5#6#7#8#9{%
    \@Test\Op{#1}{#2}{#3}{%
      \let\BS@abc\@undefined
    }%
    \@Test\Op{#4}{#5}{#6}{%
      \bitsetReset{abc}%
    }%
    \@Test\Op{#7}{#8}{#9}{%
      \def\BS@abc{1001}%
    }%
  }%
  \Test{And}%
       {0}{0}{0}%
       {0}{0}{0}%
       {0}{0}{0001}%
  \Test{AndNot}%
       {0}{0}{0}%
       {0}{0}{0}%
       {1001}{1001}{1}%
  \Test{Or}%
       {0}{0}{0101}%
       {0}{0}{0101}%
       {1001}{1001}{1101}%
  \Test{Xor}%
       {0}{0}{0101}%
       {0}{0}{0101}%
       {1001}{1001}{11}%
  \def\Test#1#2#3{%
    \bitsetSetBin{abc}{#1}%
    \bitsetSetBin{foo}{#2}%
    \csname bitset\Op\endcsname{abc}{foo}%
    \RevCheck{foo}{#2}%
    \RevCheck{abc}{#3}%
  }%
  \def\Op{And}%
  \Test{1}{111}{1}%
  \Test{111}{1}{1}%
  \Test{10}{111}{10}%
  \Test{111}{10}{10}%
  \Test{111}{1000}{0}%
  \Test{1000}{111}{0}%
  \def\Op{AndNot}%
  \Test{1010}{11}{1000}%
  \Test{100}{100}{0}%
  \Test{111}{1111}{0}%
  \Test{100}{111}{0}%
  \def\Op{Or}%
  \Test{0}{0}{0}%
  \Test{1}{0}{1}%
  \Test{0}{1}{1}%
  \Test{1}{1}{1}%
  \Test{1000}{10}{1010}%
  \Test{10}{1000}{1010}%
  \def\Op{Xor}%
  \Test{0}{0}{0}%
  \Test{1}{0}{1}%
  \Test{0}{1}{1}%
  \Test{1}{1}{0}%
  \Test{1000}{10}{1010}%
  \Test{10}{1000}{1010}%
  \Test   {110011001100}%
       {111000111000111}%
       {111110100001011}%
  \Test{111000111000111}%
          {110011001100}%
       {111110100001011}%
\end{qstest}

\begin{qstest}{GetUndef}{GetUndef, GetBin, GetOct, GetHex}
  \def\TestUndef#1#2{%
    \let\BS@abc\@undefined
    \expandafter\expandafter\expandafter\Expect
    \expandafter\expandafter\expandafter{%
      \x{abc}{#1}%
    }{#2}%
  }%
  \let\x\bitsetGetBin
  \TestUndef{-1}{0}%
  \TestUndef{0}{0}%
  \TestUndef{1}{0}%
  \TestUndef{2}{00}%
  \TestUndef{8}{00000000}%
  \let\x\bitsetGetOct
  \TestUndef{-1}{0}%
  \TestUndef{0}{0}%
  \TestUndef{1}{0}%
  \TestUndef{2}{0}%
  \TestUndef{3}{0}%
  \TestUndef{4}{00}%
  \TestUndef{5}{00}%
  \TestUndef{6}{00}%
  \TestUndef{7}{000}%
  \TestUndef{8}{000}%
  \TestUndef{9}{000}%
  \TestUndef{10}{0000}%
  \let\x\bitsetGetHex
  \TestUndef{-1}{0}%
  \TestUndef{0}{0}%
  \TestUndef{1}{0}%
  \TestUndef{2}{0}%
  \TestUndef{3}{0}%
  \TestUndef{4}{0}%
  \TestUndef{5}{00}%
  \TestUndef{6}{00}%
  \TestUndef{7}{00}%
  \TestUndef{8}{00}%
  \TestUndef{9}{000}%
  \TestUndef{10}{000}%
  \TestUndef{12}{000}%
  \TestUndef{13}{0000}%
  \TestUndef{16}{0000}%
  \TestUndef{17}{00000}%
\end{qstest}

\begin{qstest}{SetBin}{SetBin}
  \def\Test#1#2{%
    \let\BS@abc\@undefined
    \bitsetSetBin{abc}{#1}%
    \expandafter\Expect\expandafter{\BS@abc}{#2}%
  }%
  \Test{}{0}%
  \Test{0}{0}%
  \Test{1}{1}%
  \Test{10}{01}%
  \Test{11}{11}%
  \Test{010}{01}%
  \Test{011}{11}%
  \Test{0010}{01}%
  \Test{1010}{0101}%
\end{qstest}

\begin{qstest}{SetOct}{SetOct}
  \def\Test#1#2{%
    \bitsetSetOct{abc}{#1}%
    \expandafter\Expect\expandafter{\BS@abc}{#2}%
  }%
  \Test{}{0}%
  \Test{0}{0}%
  \Test{000}{0}%
  \Test{1}{1}%
  \Test{001}{1}%
  \Test{010}{0001}%
  \Test{020}{00001}%
  \Test{42}{010001}%
  \Test{377}{11111111}%
  \Test{0377}{11111111}%
  \Test{76543210}{000100010110001101011111}%
  \Test{ 0 7 0 7 1 }{100111000111}%
\end{qstest}

\begin{qstest}{SetHex}{SetHex}
  \def\Test#1#2{%
    \bitsetSetHex{abc}{#1}%
    \expandafter\Expect\expandafter{\BS@abc}{#2}%
  }%
  \Test{}{0}%
  \Test{0}{0}%
  \Test{000}{0}%
  \Test{1}{1}%
  \Test{001}{1}%
  \Test{010}{00001}%
  \Test{020}{000001}%
  \Test{42}{0100001}%
  \Test{3F}{111111}%
  \Test{03F}{111111}%
  \Test{43210}{0000100001001100001}%
  \Test{98765}{10100110111000011001}%
  \Test{FEDCBA}{010111010011101101111111}%
  \Test{ 0 F 0 F 1 }{1000111100001111}%
\end{qstest}

\begin{qstest}{SetDec}{SetDec}
  \def\Test#1#2{%
    \bitsetSetDec{abc}{#1}%
    \expandafter\Expect\expandafter{\BS@abc}{#2}%
  }%
  \Test{}{0}%
  \Test{0}{0}%
  \Test{000}{0}%
  \Test{1}{1}%
  \Test{7}{111}%
  \Test{8}{0001}%
  \Test{001}{1}%
  \Test{010}{0101}%
  \Test{020}{00101}%
  \Test{53}{101011}%
  \Test{255}{11111111}%
  \Test{256}{000000001}%
  \Test{999999999}{111111111001001101011001110111}%
  \Test{1000000000}{000000000101001101011001110111}%
  \Test{4210987654}{01100001010010010111111101011111}%
  \Test{2147483647}{1111111111111111111111111111111}%
  \Test{2147483648}{00000000000000000000000000000001}%
\end{qstest}

\begin{qstest}{GetBin}{GetBin}
  \def\TestUndef#1#2{%
    \let\BS@abc\@undefined
    \expandafter\expandafter\expandafter\Expect
    \expandafter\expandafter\expandafter{%
      \bitsetGetBin{abc}{#1}%
    }{#2}%
  }%
  \TestUndef{-1}{0}%
  \TestUndef{0}{0}%
  \TestUndef{1}{0}%
  \TestUndef{2}{00}%
  \TestUndef{8}{00000000}%
  \def\Test#1#2{%
    \bitsetSetBin{abc}{#2}%
    \expandafter\expandafter\expandafter\Expect
    \expandafter\expandafter\expandafter{%
      \bitsetGetBin{abc}{#1}%
    }{#2}%
  }%
  \Test{-1}{0}%
  \Test{0}{0}%
  \Test{1}{0}%
  \Test{1}{1}%
  \Test{2}{01}%
  \Test{2}{10}%
  \Test{3}{010}%
  \Test{2}{00}%
  \Test{2}{01}%
  \Test{8}{00101100}%
  \Test{2}{10101}%
  \Test{-100}{11011}%
\end{qstest}

\begin{qstest}{GetOct}{GetOct}
  \def\Test#1#2#3{%
    \edef\x{\zap@space#1 \@empty}%
    \edef\x{\noexpand\bitsetSetBin{abc}{\x}}%
    \x
    \expandafter\expandafter\expandafter\Expect
    \expandafter\expandafter\expandafter{%
      \bitsetGetOct{abc}{#2}%
    }{#3}%
  }%
  \Test{111 110 101 100 011 010 001 000}{0}{76543210}%
  \Test{000 111}{0}{7}%
  \Test{101 000}{-1}{50}%
  \Test{111}{-1}{7}%
  \Test{111}{0}{7}%
  \Test{111}{1}{7}%
  \Test{111}{3}{7}%
  \Test{111}{4}{07}%
  \Test{111}{6}{07}%
  \Test{111}{7}{007}%
  \Test{111 010}{6}{72}%
  \Test{111 010}{7}{072}%
  \Test{011 111}{0}{37}%
  \Test{011 111}{6}{37}%
  \Test{011 111}{7}{037}%
  \Test{001 111}{0}{17}%
  \Test{001 111}{6}{17}%
  \Test{001 111}{7}{017}%
\end{qstest}

\begin{qstest}{GetHex}{GetHex}
  \def\Test#1#2#3{%
    \bitsetSetBin{abc}{#1}%
    \expandafter\expandafter\expandafter\Expect
    \expandafter\expandafter\expandafter{%
      \bitsetGetHex{abc}{#2}%
    }{#3}%
  }%
  \Test{1111 1110 1101 1100 1011 1010 1001 1000}{0}{FEDCBA98}%
  \Test{0111 0110 0101 0100 0011 0010 0001 0000}{0}{76543210}%
  \Test{0000 1111}{0}{F}%
  \Test{0101 0000}{-1}{50}%
  \Test{1111}{-1}{F}%
  \Test{1111}{0}{F}%
  \Test{1111}{1}{F}%
  \Test{1111}{4}{F}%
  \Test{1111}{5}{0F}%
  \Test{1111}{8}{0F}%
  \Test{1111}{9}{00F}%
  \Test{1111 0010}{8}{F2}%
  \Test{1111 0010}{9}{0F2}%
  \Test{0111 1111}{0}{7F}%
  \Test{0111 1111}{8}{7F}%
  \Test{0111 1111}{9}{07F}%
  \Test{0011 1111}{0}{3F}%
  \Test{0011 1111}{8}{3F}%
  \Test{0011 1111}{9}{03F}%
  \Test{0001 1111}{0}{1F}%
  \Test{0001 1111}{8}{1F}%
  \Test{0001 1111}{9}{01F}%
\end{qstest}

\begin{qstest}{Range}{Range}
  \TestError{%
    Wrong index numbers in range [9..8]\MessageBreak% hash-ok
    for clear/set/flip on bit set `abc'.\MessageBreak
    The lower index exceeds the upper index.\MessageBreak
    Canceling the operation as error recovery%
  }{%
    \bitsetSetRange{abc}{9}{8}%
  }%
  \def\TestErrorNegInd#1#2#3#4#5#6{%
    \TestError{%
      Negative index in range [#2..#3]\MessageBreak % hash-ok
      for \string\bitset #1Range on bit set `abc'.\MessageBreak
      Using [#4..#5] as error recovery% hash-ok
    }{%
      \csname bitset#1Range\endcsname{abc}{#2}{#3}%
      \global\let\BS@global\BS@abc
    }%
    \Check{global}{#6}%
  }%
  \Set{abc}{111}%
  \TestErrorNegInd{Clear}{-1}{0}{0}{0}{111}%
  \TestErrorNegInd{Clear}{0}{-1}{0}{0}{111}%
  \TestErrorNegInd{Clear}{-2}{2}{0}{2}{001}%
  \bitsetReset{abc}%
  \TestErrorNegInd{Set}{-1}{0}{0}{0}{0}%
  \TestErrorNegInd{Set}{0}{-1}{0}{0}{0}%
  \TestErrorNegInd{Set}{-2}{2}{0}{2}{11}%
  \Set{abc}{101}%
  \TestErrorNegInd{Flip}{-1}{0}{0}{0}{101}%
  \TestErrorNegInd{Flip}{0}{-1}{0}{0}{101}%
  \TestErrorNegInd{Flip}{-2}{2}{0}{2}{011}%
  \def\Test#1#2#3#4{%
    \bitsetSetBin{abc}{#1}%
    \csname bitset\TestOp Range\endcsname{abc}{#2}{#3}%
    \Expect*{\bitsetGetBin{abc}{0}}{#4}%
  }%
  \def\TestOp{Clear}%
  \Test{0}{0}{1}{0}%
  \Test{1111}{1}{2}{1101}%
  \Test{1111}{1}{3}{1001}%
  \Test{1111111100000000}{12}{14}{1100111100000000}%
  \def\TestOp{Set}%
  \Test{0}{0}{1}{1}%
  \Test{1000}{1}{2}{1010}%
  \Test{0}{1}{2}{10}%
  \Test{1}{12}{15}{111000000000001}%
  \Test{1111}{1}{3}{1111}%
  \Test{1000000000000000}{12}{14}{1011000000000000}%
  \def\TestOp{Flip}%
  \Test{0}{0}{1}{1}%
  \Test{1}{0}{1}{0}%
  \Test{10101010}{1}{5}{10110100}%
  \def\Test#1#2#3#4#5{%
    \bitsetSetBin{abc}{#1}%
    \bitsetSetValueRange{abc}{#2}{#3}{#4}%
    \Expect*{\bitsetGetBin{abc}{0}}{#5}%
  }%
  \Test{0}{0}{1}{0}{0}%
  \Test{0}{0}{1}{1}{1}%
  \Test{1010}{1}{3}{0}{1000}%
  \Test{1010}{1}{3}{1}{1110}%
\end{qstest}

\begin{qstest}{ShiftLeft/ShiftRight}{ShiftLeft/ShiftRight}
  \def\@Test#1#2{%
    \let\BS@abc\@undefined
    \csname bitsetShift#1\endcsname{abc}{#2}%
    \Expect*{\BS@abc}{0}%
  }%
  \def\Test#1{%
    \@Test{Left}{#1}%
    \@Test{Right}{#1}%
  }%
  \Test{-16}%
  \Test{-1}%
  \Test{0}%
  \Test{1}%
  \Test{16}%
  \def\Test#1#2#3{%
    \bitsetSetBin{abc}{#1}%
    \bitsetSetBin{result}{#3}%
    \csname bitsetShift\Op\endcsname{abc}{#2}%
    \Expect*{\bitsetGetBin{abc}{0}}*{\bitsetGetBin{result}{0}}%
  }%
  \def\Op{Left}%
  \Test{0}{0}{0}%
  \Test{0}{1}{0}%
  \Test{0}{-1}{0}%
  \Test{1}{0}{1}%
  \Test{1}{1}{10}%
  \Test{1}{-1}{0}%
  \Test{10}{1}{100}%
  \Test{10}{-1}{1}%
  \Test{1}{32}{100000000000000000000000000000000}%
  \Test{1}{-100}{0}%
  \def\Op{Right}%
  \Test{0}{0}{0}%
  \Test{0}{1}{0}%
  \Test{0}{-1}{0}%
  \Test{1}{0}{1}%
  \Test{1}{1}{0}%
  \Test{1}{-1}{10}%
  \Test{10}{1}{1}%
  \Test{10}{-1}{100}%
  \Test{1}{-32}{100000000000000000000000000000000}%
  \Test{1}{100}{0}%
  \Test{110110110110110}{10}{11011}%
  \Test{110110110110110}{100}{0}%
  \Test{1}{100000}{0}%
\end{qstest}

\begin{qstest}{Profile: Set}{Profile: Set}
  \bitsetSet{abc}{4095}%
  \global\let\BS@global\BS@abc
\end{qstest}

\begin{qstest}{Profile: Get}{Profile: Get}
  \edef\x{\bitsetGet{global}{4095}}%
\end{qstest}

\begin{document}
\end{document}
%</test2>
%    \end{macrocode}
%
% \section{Installation}
%
% \subsection{Download}
%
% \paragraph{Package.} This package is available on
% CTAN\footnote{\url{ftp://ftp.ctan.org/tex-archive/}}:
% \begin{description}
% \item[\CTAN{macros/latex/contrib/oberdiek/bitset.dtx}] The source file.
% \item[\CTAN{macros/latex/contrib/oberdiek/bitset.pdf}] Documentation.
% \end{description}
%
%
% \paragraph{Bundle.} All the packages of the bundle `oberdiek'
% are also available in a TDS compliant ZIP archive. There
% the packages are already unpacked and the documentation files
% are generated. The files and directories obey the TDS standard.
% \begin{description}
% \item[\CTAN{install/macros/latex/contrib/oberdiek.tds.zip}]
% \end{description}
% \emph{TDS} refers to the standard ``A Directory Structure
% for \TeX\ Files'' (\CTAN{tds/tds.pdf}). Directories
% with \xfile{texmf} in their name are usually organized this way.
%
% \subsection{Bundle installation}
%
% \paragraph{Unpacking.} Unpack the \xfile{oberdiek.tds.zip} in the
% TDS tree (also known as \xfile{texmf} tree) of your choice.
% Example (linux):
% \begin{quote}
%   |unzip oberdiek.tds.zip -d ~/texmf|
% \end{quote}
%
% \paragraph{Script installation.}
% Check the directory \xfile{TDS:scripts/oberdiek/} for
% scripts that need further installation steps.
% Package \xpackage{attachfile2} comes with the Perl script
% \xfile{pdfatfi.pl} that should be installed in such a way
% that it can be called as \texttt{pdfatfi}.
% Example (linux):
% \begin{quote}
%   |chmod +x scripts/oberdiek/pdfatfi.pl|\\
%   |cp scripts/oberdiek/pdfatfi.pl /usr/local/bin/|
% \end{quote}
%
% \subsection{Package installation}
%
% \paragraph{Unpacking.} The \xfile{.dtx} file is a self-extracting
% \docstrip\ archive. The files are extracted by running the
% \xfile{.dtx} through \plainTeX:
% \begin{quote}
%   \verb|tex bitset.dtx|
% \end{quote}
%
% \paragraph{TDS.} Now the different files must be moved into
% the different directories in your installation TDS tree
% (also known as \xfile{texmf} tree):
% \begin{quote}
% \def\t{^^A
% \begin{tabular}{@{}>{\ttfamily}l@{ $\rightarrow$ }>{\ttfamily}l@{}}
%   bitset.sty & tex/generic/oberdiek/bitset.sty\\
%   bitset.pdf & doc/latex/oberdiek/bitset.pdf\\
%   test/bitset-test1.tex & doc/latex/oberdiek/test/bitset-test1.tex\\
%   test/bitset-test2.tex & doc/latex/oberdiek/test/bitset-test2.tex\\
%   test/bitset-test3.tex & doc/latex/oberdiek/test/bitset-test3.tex\\
%   bitset.dtx & source/latex/oberdiek/bitset.dtx\\
% \end{tabular}^^A
% }^^A
% \sbox0{\t}^^A
% \ifdim\wd0>\linewidth
%   \begingroup
%     \advance\linewidth by\leftmargin
%     \advance\linewidth by\rightmargin
%   \edef\x{\endgroup
%     \def\noexpand\lw{\the\linewidth}^^A
%   }\x
%   \def\lwbox{^^A
%     \leavevmode
%     \hbox to \linewidth{^^A
%       \kern-\leftmargin\relax
%       \hss
%       \usebox0
%       \hss
%       \kern-\rightmargin\relax
%     }^^A
%   }^^A
%   \ifdim\wd0>\lw
%     \sbox0{\small\t}^^A
%     \ifdim\wd0>\linewidth
%       \ifdim\wd0>\lw
%         \sbox0{\footnotesize\t}^^A
%         \ifdim\wd0>\linewidth
%           \ifdim\wd0>\lw
%             \sbox0{\scriptsize\t}^^A
%             \ifdim\wd0>\linewidth
%               \ifdim\wd0>\lw
%                 \sbox0{\tiny\t}^^A
%                 \ifdim\wd0>\linewidth
%                   \lwbox
%                 \else
%                   \usebox0
%                 \fi
%               \else
%                 \lwbox
%               \fi
%             \else
%               \usebox0
%             \fi
%           \else
%             \lwbox
%           \fi
%         \else
%           \usebox0
%         \fi
%       \else
%         \lwbox
%       \fi
%     \else
%       \usebox0
%     \fi
%   \else
%     \lwbox
%   \fi
% \else
%   \usebox0
% \fi
% \end{quote}
% If you have a \xfile{docstrip.cfg} that configures and enables \docstrip's
% TDS installing feature, then some files can already be in the right
% place, see the documentation of \docstrip.
%
% \subsection{Refresh file name databases}
%
% If your \TeX~distribution
% (\teTeX, \mikTeX, \dots) relies on file name databases, you must refresh
% these. For example, \teTeX\ users run \verb|texhash| or
% \verb|mktexlsr|.
%
% \subsection{Some details for the interested}
%
% \paragraph{Attached source.}
%
% The PDF documentation on CTAN also includes the
% \xfile{.dtx} source file. It can be extracted by
% AcrobatReader 6 or higher. Another option is \textsf{pdftk},
% e.g. unpack the file into the current directory:
% \begin{quote}
%   \verb|pdftk bitset.pdf unpack_files output .|
% \end{quote}
%
% \paragraph{Unpacking with \LaTeX.}
% The \xfile{.dtx} chooses its action depending on the format:
% \begin{description}
% \item[\plainTeX:] Run \docstrip\ and extract the files.
% \item[\LaTeX:] Generate the documentation.
% \end{description}
% If you insist on using \LaTeX\ for \docstrip\ (really,
% \docstrip\ does not need \LaTeX), then inform the autodetect routine
% about your intention:
% \begin{quote}
%   \verb|latex \let\install=y\input{bitset.dtx}|
% \end{quote}
% Do not forget to quote the argument according to the demands
% of your shell.
%
% \paragraph{Generating the documentation.}
% You can use both the \xfile{.dtx} or the \xfile{.drv} to generate
% the documentation. The process can be configured by the
% configuration file \xfile{ltxdoc.cfg}. For instance, put this
% line into this file, if you want to have A4 as paper format:
% \begin{quote}
%   \verb|\PassOptionsToClass{a4paper}{article}|
% \end{quote}
% An example follows how to generate the
% documentation with pdf\LaTeX:
% \begin{quote}
%\begin{verbatim}
%pdflatex bitset.dtx
%makeindex -s gind.ist bitset.idx
%pdflatex bitset.dtx
%makeindex -s gind.ist bitset.idx
%pdflatex bitset.dtx
%\end{verbatim}
% \end{quote}
%
% \section{Catalogue}
%
% The following XML file can be used as source for the
% \href{http://mirror.ctan.org/help/Catalogue/catalogue.html}{\TeX\ Catalogue}.
% The elements \texttt{caption} and \texttt{description} are imported
% from the original XML file from the Catalogue.
% The name of the XML file in the Catalogue is \xfile{bitset.xml}.
%    \begin{macrocode}
%<*catalogue>
<?xml version='1.0' encoding='us-ascii'?>
<!DOCTYPE entry SYSTEM 'catalogue.dtd'>
<entry datestamp='$Date$' modifier='$Author$' id='bitset'>
  <name>bitset</name>
  <caption>Handle bit-vector datatype.</caption>
  <authorref id='auth:oberdiek'/>
  <copyright owner='Heiko Oberdiek' year='2007,2011'/>
  <license type='lppl1.3'/>
  <version number='1.1'/>
  <description>
    This package defines and implements the data type bit set,
    a vector of bits. The size of the vector may grow dynamically.
    Individual bits can be manipulated.
    <p/>
    The package is part of the <xref refid='oberdiek'>oberdiek</xref> bundle.
  </description>
  <documentation details='Package documentation'
      href='ctan:/macros/latex/contrib/oberdiek/bitset.pdf'/>
  <ctan file='true' path='/macros/latex/contrib/oberdiek/bitset.dtx'/>
  <miktex location='oberdiek'/>
  <texlive location='oberdiek'/>
  <install path='/macros/latex/contrib/oberdiek/oberdiek.tds.zip'/>
</entry>
%</catalogue>
%    \end{macrocode}
%
% \begin{History}
%   \begin{Version}{2007/09/28 v1.0}
%   \item
%     First version.
%   \end{Version}
%   \begin{Version}{2011/01/30 v1.1}
%   \item
%     Already loaded package files are not input in \hologo{plainTeX}.
%   \end{Version}
% \end{History}
%
% \PrintIndex
%
% \Finale
\endinput

%        (quote the arguments according to the demands of your shell)
%
% Documentation:
%    (a) If bitset.drv is present:
%           latex bitset.drv
%    (b) Without bitset.drv:
%           latex bitset.dtx; ...
%    The class ltxdoc loads the configuration file ltxdoc.cfg
%    if available. Here you can specify further options, e.g.
%    use A4 as paper format:
%       \PassOptionsToClass{a4paper}{article}
%
%    Programm calls to get the documentation (example):
%       pdflatex bitset.dtx
%       makeindex -s gind.ist bitset.idx
%       pdflatex bitset.dtx
%       makeindex -s gind.ist bitset.idx
%       pdflatex bitset.dtx
%
% Installation:
%    TDS:tex/generic/oberdiek/bitset.sty
%    TDS:doc/latex/oberdiek/bitset.pdf
%    TDS:doc/latex/oberdiek/test/bitset-test1.tex
%    TDS:doc/latex/oberdiek/test/bitset-test2.tex
%    TDS:doc/latex/oberdiek/test/bitset-test3.tex
%    TDS:source/latex/oberdiek/bitset.dtx
%
%<*ignore>
\begingroup
  \catcode123=1 %
  \catcode125=2 %
  \def\x{LaTeX2e}%
\expandafter\endgroup
\ifcase 0\ifx\install y1\fi\expandafter
         \ifx\csname processbatchFile\endcsname\relax\else1\fi
         \ifx\fmtname\x\else 1\fi\relax
\else\csname fi\endcsname
%</ignore>
%<*install>
\input docstrip.tex
\Msg{************************************************************************}
\Msg{* Installation}
\Msg{* Package: bitset 2011/01/30 v1.1 Handle bit-vector datatype (HO)}
\Msg{************************************************************************}

\keepsilent
\askforoverwritefalse

\let\MetaPrefix\relax
\preamble

This is a generated file.

Project: bitset
Version: 2011/01/30 v1.1

Copyright (C) 2007, 2011 by
   Heiko Oberdiek <heiko.oberdiek at googlemail.com>

This work may be distributed and/or modified under the
conditions of the LaTeX Project Public License, either
version 1.3c of this license or (at your option) any later
version. This version of this license is in
   http://www.latex-project.org/lppl/lppl-1-3c.txt
and the latest version of this license is in
   http://www.latex-project.org/lppl.txt
and version 1.3 or later is part of all distributions of
LaTeX version 2005/12/01 or later.

This work has the LPPL maintenance status "maintained".

This Current Maintainer of this work is Heiko Oberdiek.

The Base Interpreter refers to any `TeX-Format',
because some files are installed in TDS:tex/generic//.

This work consists of the main source file bitset.dtx
and the derived files
   bitset.sty, bitset.pdf, bitset.ins, bitset.drv, bitset-test1.tex,
   bitset-test2.tex, bitset-test3.tex.

\endpreamble
\let\MetaPrefix\DoubleperCent

\generate{%
  \file{bitset.ins}{\from{bitset.dtx}{install}}%
  \file{bitset.drv}{\from{bitset.dtx}{driver}}%
  \usedir{tex/generic/oberdiek}%
  \file{bitset.sty}{\from{bitset.dtx}{package}}%
  \usedir{doc/latex/oberdiek/test}%
  \file{bitset-test1.tex}{\from{bitset.dtx}{test1}}%
  \file{bitset-test2.tex}{\from{bitset.dtx}{test2,etex}}%
  \file{bitset-test3.tex}{\from{bitset.dtx}{test2,noetex}}%
  \nopreamble
  \nopostamble
  \usedir{source/latex/oberdiek/catalogue}%
  \file{bitset.xml}{\from{bitset.dtx}{catalogue}}%
}

\catcode32=13\relax% active space
\let =\space%
\Msg{************************************************************************}
\Msg{*}
\Msg{* To finish the installation you have to move the following}
\Msg{* file into a directory searched by TeX:}
\Msg{*}
\Msg{*     bitset.sty}
\Msg{*}
\Msg{* To produce the documentation run the file `bitset.drv'}
\Msg{* through LaTeX.}
\Msg{*}
\Msg{* Happy TeXing!}
\Msg{*}
\Msg{************************************************************************}

\endbatchfile
%</install>
%<*ignore>
\fi
%</ignore>
%<*driver>
\NeedsTeXFormat{LaTeX2e}
\ProvidesFile{bitset.drv}%
  [2011/01/30 v1.1 Handle bit-vector datatype (HO)]%
\documentclass{ltxdoc}
\usepackage{holtxdoc}[2011/11/22]
\usepackage{amstext}
\usepackage{ifluatex}
\ifluatex
\else
  \usepackage[T1]{fontenc}%
  \usepackage{lmodern}%
\fi
\begin{document}
  \DocInput{bitset.dtx}%
\end{document}
%</driver>
% \fi
%
% \CheckSum{3510}
%
% \CharacterTable
%  {Upper-case    \A\B\C\D\E\F\G\H\I\J\K\L\M\N\O\P\Q\R\S\T\U\V\W\X\Y\Z
%   Lower-case    \a\b\c\d\e\f\g\h\i\j\k\l\m\n\o\p\q\r\s\t\u\v\w\x\y\z
%   Digits        \0\1\2\3\4\5\6\7\8\9
%   Exclamation   \!     Double quote  \"     Hash (number) \#
%   Dollar        \$     Percent       \%     Ampersand     \&
%   Acute accent  \'     Left paren    \(     Right paren   \)
%   Asterisk      \*     Plus          \+     Comma         \,
%   Minus         \-     Point         \.     Solidus       \/
%   Colon         \:     Semicolon     \;     Less than     \<
%   Equals        \=     Greater than  \>     Question mark \?
%   Commercial at \@     Left bracket  \[     Backslash     \\
%   Right bracket \]     Circumflex    \^     Underscore    \_
%   Grave accent  \`     Left brace    \{     Vertical bar  \|
%   Right brace   \}     Tilde         \~}
%
% \GetFileInfo{bitset.drv}
%
% \title{The \xpackage{bitset} package}
% \date{2011/01/30 v1.1}
% \author{Heiko Oberdiek\\\xemail{heiko.oberdiek at googlemail.com}}
%
% \maketitle
%
% \begin{abstract}
% This package defines and implements the data type bit set,
% a vector of bits. The size of the vector may grow dynamically.
% Individual bits can be manipulated.
% \end{abstract}
%
% \tableofcontents
%
% \clearpage
% \section{Documentation}
%
% \subsection{Introduction}
%
% Annotations in the PDF format know entries whose values are
% integers. This numbers are interpreted as set of flags
% specifying properties. For example, annotation dictionaries
% can have a key |/F|. The bits of its integer value
% are interpreted the following way:
% \begin{quote}
% \begin{tabular}{@{}l|l@{}}
%   Bit position & Property name\\
%   \hline
%   1 & Invisible\\
%   2 & Hidden\\
%   3 & Print\\
%   4 & NoZoom\\
%   5 & NoRotate\\
%   6 & NoView\\
%   7 & ReadOnly\\
%   \dots & \dots
% \end{tabular}
% \end{quote}
% Now, let's see how these values are set in package \xpackage{hyperref}
% before it uses this package (before v6.77a):
%\begin{quote}
%\begin{verbatim}
%\ifFld@hidden /F 6\else /F 4\fi
%\end{verbatim}
%\end{quote}
% Where are the other flags? The following example for key |/Ff|
% in a widget annotation
% supports at least three properties:
%\begin{quote}
%\begin{verbatim}
%\ifFld@multiline
%  \ifFld@readonly /Ff 4097\else /Ff 4096\fi
%\else
%  \ifFld@password
%    \ifFld@readonly /Ff 8193\else /Ff 8192\fi
%  \else
%    \ifFld@readonly /Ff 1\fi
%  \fi
%\fi
%\end{verbatim}
%\end{quote}
% But you see the point. It would be a nightmare to continue
% this way in supporting the missing flag settings. This kind
% of integers may have up to 32 bits.
%
% Therefore I wanted a data structure for setting and
% clearing individual bits. Also it should provide
% an export as decimal number. The snipsets above are
% executed in expansion contexts without \TeX's stomach
% commands. It would be convenient to have an expandable
% conversion from the data structure to the integer that
% gets written to the PDF file.
%
% This package \xpackage{bitset} implements such a data
% structure. The interface is quite close to Java's
% class |BitSet| in order not to learn to many interfaces
% for the same kind of data structure.
%
% \subsection{Glossary}
%
% \begin{description}
% \item[Bit set:]
%   A bit set is a vector of bits or flags.
%   The vector size is unlimited and
%   grows dynamically. An undefined bit set is treated as bit
%   set where all bits are cleared.
%
%   Bit sets are addressed by name. A name should consists of
%   letters or digits. Technically it must survive \cs{csname},
%   see \LaTeX's environment names for other names with such
%   a constraint. Package \xpackage{babel}'s shorthands are
%   not supported due to technical reasons. Shorthand support
%   would break expandable operations.
% \item[Size:]
%   A size of a bit set is the number of bits in use.
%   It's the number of the highest index, incremented by one.
%   Sizes are in the range |0| up to |2147483647|, the highest
%   number supported by \TeX.
% \item[Index:]
%   Bit positions in a bit set are addressed by an index number.
%   The bit vector is zero based. The first and least significant
%   bit is addressed by index |0| and the highest possible bit
%   by |2147483646|.
% \item[Bit:]
%   A bit is enoded as |0| for cleared/disabled or |1| for set/enabled.
% \end{description}
%
% \subsection{Design principles}
%
% \begin{description}
% \item[Name conventions:]
%   To avoid conflicts with existing macro names, the operations
%   are prefixed by the package name.
% \item[Zero based indexes:] The first bit is addressed by zero.
%   (Convention of array indexing in C, Java, \dots)
% \item[Unlimited size:]
%   There is no restriction on the size of a bit set other than
%   usual memory limitations. \cs{bitsetSetDec} and \cs{bitsetGetDec}
%   transparently switch to package \xpackage{bigintcalc} if the numbers
%   get too large for \TeX's number limit.
% \item[Expandibility:]
%   Any operation that does not change the bit set is expandable.
%   And all operations that extract or calculate some result
%   do this in exact two expansion steps. For example,
%   a macro \cs{Macro} wants a bit set as decimal number.
%   But the argument must be a plain number without macros.
%   Thus you could prefix \cs{bitsetGetDec} with \cs{number}.
%   However this won't work for bit sets with 31 or more bits
%   because of \TeX's number limit of $2^{31}-1$.
%   then just hit the operator with two \cs{expandafter}:
%   \begin{quote}
%   |\expandafter\expandafter\expandafter|\\
%   |\Macro\bitsetGetDec{foo}|
%   \end{quote}
%   |\bitsetGetDec| is hit first by the third \cs{expandafter}
%   and then by the second one.
% \item[Format independence:] This package is written as \LaTeX\ package,
%   but it does not depend on \LaTeX. It will also work for other formats
%   such as \plainTeX.
% \item[Independence from \TeX\ engines:]
%   Vanilla \TeX\ is all you need. Calculations are delegated to
%   packages \xpackage{intcalc} and \xpackage{bigintcalc}.
%   They don't need any special features, but they will switch
%   to a little more efficient implementation if features such
%   as \cs{numexpr} are available.
% \item[Numeric arguments:] Anything that is accepted by
%   \cs{number}. If \eTeX\ is detected, also expressions
%   for \cs{numexpr} are supported. The only exception so far
%   is the number for \cs{bitsetSetDec}. The number might be too
%   large for \cs{number} or \cs{numexpr}.
% \item[Error messages:]
%   In expandable contexts, only a limited set of \TeX\ primitive
%   commands work as expected. So called stomach commands behave
%   like \cs{relax} and don't get expanded or executed.
%   Unhappily also the error commands belong to this category.
%   The expandable operations will throw an unknown control
%   sequence instead to get \TeX's and user's attention.
%   The name of these control sequences starts with
%   \cs{BitSetError:} with the type of error after the colon.
% \end{description}
%
% \subsection{Operator overview}
%
% \begin{minipage}{\linewidth}
% \begin{description}
% \def\It#1#2{^^A
%   \item[^^A
%     {#1} ^^A
%     \textmd{(\hyperref[{sec:#2}]{section \ref*{sec:#2}})}^^A
%   ]^^A
%   \mbox{}\\^^A
% }
% \def\op#1{^^A
%   \cs{bitset}\texttt{\bfseries#1}^^A
% }
% \def\M#1{\meta{#1}}
% \def\Q{\hspace*{\fill}}
% \It{Miscellaneous}{misc}
%   \op{Reset}\Q \M{BitSet}\\
%   \op{Let}\Q \M{BitSet A} \M{BitSet B}
% \It{Import}{import}
%   \op{SetBin}, \op{SetOct}, \op{SetHex}\Q \M{BitSet} \M{Value}\\
%   \op{SetDec}\Q \M{BitSet} \M{Value}
% \It{Export\textsuperscript{a}}{export}
%   \op{GetBin}, \op{GetOct}, \op{GetHex}\Q \M{BitSet} \M{MinSize}\\
%   \op{GetDec}\Q \M{BitSet}
% \It{Logical operators}{log}
%   \op{And}, \op{AndNot}\Q \M{BitSet A} \M{BitSet B}\\
%   \op{Or}, \op{Xor}\Q \M{BitSet A} \M{BitSet B}
% \It{Shifting}{shift}
%   \op{ShiftLeft}, \op{ShiftRight}\Q \M{BitSet} \M{ShiftAmount}\\
% \It{Bit manipulation}{bitman}
%   \op{Clear}, \op{Set}, \op{Flip}\Q \M{BitSet} \M{Index}\\
%   \op{SetValue}\Q \M{BitSet} \M{Index} \M{Value}\\
%   \op{ClearRange}, \op{SetRange}, \op{FlipRange}\\
%     \Q \M{BitSet} \M{IndexFrom} \M{IndexTo}\\
%   \op{SetValueRange}\Q \M{BitSet} \M{IndexFrom} \M{IndexTo}
% \It{Bit retrieval\textsuperscript{a}}{bitret}
%   \op{Get}\Q \M{BitSet} \M{Index}\\
%   \op{NextClearBit}, \op{NextSetBit}\Q \M{BitSet} \M{Index}\\
%   \op{GetSetBitList}\Q \M{BitSet}
% \It{Bit set properties}{prop}
%   \op{Size}, \op{Cardinality}\Q \M{BitSet}
% \It{Queries\textsuperscript{b}}{query}
%   \op{IsDefined}, \op{IsEmpty}\Q \M{BitSet} \M{Then} \M{Else}\\
%   \op{Equals}, \op{Intersects}\Q \M{BitSet A} \M{BitSet B} \M{Then} \M{Else}\\
%   \op{Query}\Q \M{BitSet} \M{Index} \M{Then} \M{Else}
% \end{description}
% \footnotetext[1]{Macros are expandable, full expansion by two steps.}
% \footnotetext[2]{Macros are expandable.}
% \end{minipage}
%
% \subsection{Package loading}
%
% The package can be used as normal \LaTeX\ package:
%\begin{quote}
%|\usepackage{bitset}|
%\end{quote}
% Also \plainTeX\ is supported:
%\begin{quote}
%|\input bitset.sty\relax|
%\end{quote}
%
% \subsection{Operators}
%
% The following macros work on and with bit sets.
% A bit set \meta{BitSet} is represented by a name.
% The should consist of letters and digits. Technically
% it must survive \cs{csname}. It is the same constraint
% that must be satisfied by label or environment names
% in \LaTeX.
%
% However active characters that are shorthands
% of package \xpackage{babel} are not supported.
% Support for shorthands works by an assignment.
% But many operators
% such as \cs{bitsetGetDec} must be usable in expandable
% contexts. There assignments will not be executed in the
% best case or they will cause errors.
%
% The bits in a bit set are addressed by non-negative
% integers starting from zero. Thus negative index
% numbers cause an error message. Because index numbers
% are \TeX\ numbers. The largest index is 2147483647.
% But in practice memory limits and patience limits
% will be very likely reached much before.
%
% \subsubsection{Miscellaneous}\label{sec:misc}
%
% There isn't a separate operation for bit set creation.
% For simplicity an undefined bit set is treated as bit set
% with all bits cleared.
%
% \begin{declcs}{bitsetReset} \M{BitSet}
% \end{declcs}
% Macro \cs{bitsetReset} clears all bits. The result is an
% empty bit set. It may also be used as replacement for
% an operation ``new'', because an undefined bit set is defined
% afterwards.
%
% \begin{declcs}{bitsetLet} \M{BitSet A} \M{BitSet B}
% \end{declcs}
% Macro \cs{bitsetLet} performs a simple assignment similar
% to \TeX's \cs{let}. After the operation \meta{BitSet A} has
% the same value as \meta{BitSet B}. If \meta{BitSet B} is
% undefined, then \meta{BitSet A} will be the empty bit set.
%
% Note: If \meta{BitSet A} exists, it will be overwritten.
%
% \subsubsection{Import}\label{sec:import}
%
% \begin{declcs}{bitsetSetBin} \M{BitSet} \M{BinaryNumber}\\
%   \cs{bitsetSetOct} \M{BitSet} \M{OctalNumber}\\
%   \cs{bitsetSetHex} \M{BitSet} \M{HexadecimalNumber}
% \end{declcs}
% The numbers are interpreted as bit vectors and the flags in
% the bit \meta{BitSet} set are set accordingly.
% These numeric arguments are the only arguments where
% spaces are allowed. Then the numbers are easier to read.
%
% \begin{declcs}{bitsetSetDec} \M{BitSet} \M{DecimalNumber}
% \end{declcs}
% Macro \cs{bitsetSetDec} uses \meta{DecimalNumber} to
% set the bit set \meta{BitSet}.
% The numeric argument must expand
% to a plain number consisting of decimal digits without
% command tokens or spaces. Internally this argument is
% expanded only. It cannot be passed to \cs{number} or
% \cs{numexpr}, because the number may be too large for them.
% However \cs{number} or |\the\numexpr| may be used
% explicitely. This also helps for unexpandable number
% command tokens or registers (\cs{z@}, \cs{@ne}, \cs{count@}, \dots).
% Also \LaTeX'\ \cs{value} needs prefixing:
% \begin{quote}
% |\bitsetSetDec{foo}{\number\value{bar}}|
% \end{quote}
%
% \subsubsection{Export}\label{sec:export}
%
% \begin{declcs}{bitsetGetBin} \M{BitSet} \M{MinSize}\\
%   \cs{bitsetGetOct} \M{BitSet} \M{MinSize}\\
%   \cs{bitsetGetHex} \M{BitSet} \M{MinSize}
% \end{declcs}
% These macros returns the bit set as binary, octal or hexadecimal
% number. If the bit size is smaller than \meta{MinSize} the gap is
% filled with leading zeros.
% Example:
% \begin{quote}
% |\bitsetReset{abc}|\\
% |\bitsetSet{abc}{2}|\\
% |\bitsetGetBin{abc}{8}| $\rightarrow$ |00000100|\\
% |\bitsetSet{abc}{5}\bitsetSet{abc}{7}|\\
% |\bitsetGetHex{abc}{16}| $\rightarrow$ |00A2|
% \end{quote}
% Macro \cs{bitsetGetHex} uses the uppercase letters |A| to |F|.
% The catcode of the letters is one of 11 (letter) or 12 (other).
%
% \begin{declcs}{bitsetGetDec} \M{BitSet}
% \end{declcs}
% Macro \cs{bitsetGetDec} returns the bit set \meta{BitSet} as
% decimal number. The returned number can be larger than
% \TeX's number limit of $2^{31}-1$.
%
% \subsubsection{Logical operators}\label{sec:log}
%
% \begin{declcs}{bitsetAnd} \M{BitSet A} \M{BitSet B}
% \end{declcs}
% $A_{\text{new}} \mathrel{\mathop{:}}= A_{\text{old}}
% \mathbin{\text{and}} B \qquad (\forall\, \text{bits})$
%
% \begin{declcs}{bitsetAndNot} \M{BitSet A} \M{BitSet B}
% \end{declcs}
% $A_{\text{new}} \mathrel{\mathop{:}}= A_{\text{old}}
% \mathbin{\text{and}} (\text{not}\,B) \qquad (\forall\, \text{bits})$
%
% \begin{declcs}{bitsetOr} \M{BitSet A} \M{BitSet B}
% \end{declcs}
% $A_{\text{new}} \mathrel{\mathop{:}}= A_{\text{old}}
% \mathbin{\text{or}} B \qquad (\forall\, \text{bits})$
%
% \begin{declcs}{bitsetXor} \M{BitSet A} \M{BitSet B}
% \end{declcs}
% $A_{\text{new}} \mathrel{\mathop{:}}= A_{\text{old}}
% \mathbin{\text{xor}} B \qquad (\forall\, \text{bits})$
%
% \subsubsection{Shifting}\label{sec:shift}
%
% \begin{declcs}{bitsetShiftLeft} \M{BitSet} \M{ShiftAmount}\\
%   \cs{bitsetShiftRight} \M{BitSet} \M{ShiftAmount}
% \end{declcs}
% A left shift by one is a multiplication by two, thus left shifting
% moves the flags to higher positions. The new created low positions
% are filled by zeros.
%
% A right shift is the opposite, dividing by two, movint the bits
% to lower positions. The number will become smaller, the lowest
% bits are lost.
%
% If the \meta{ShiftAmount} is negative, it reverts the meaning of
% the shift operation. A left shift becomes a right shift.
% A \meta{ShiftAmount} of zero is ignored.
%
% \subsubsection{Bit manipulation}\label{sec:bitman}
%
% \begin{declcs}{bitsetClear} \M{BitSet} \M{Index}\\
%   \cs{bitsetSet} \M{BitSet} \M{Index}\\
%   \cs{bitsetFlip} \M{BitSet} \M{Index}
% \end{declcs}
% This macros manipulate a single bit in \meta{BitSet} addressed
% by \cs{Index}. Macro \cs{bitsetClear} disables the bit,
% \cs{bitsetSet} enables it and \cs{bitsetFlip} reverts the current
% setting of the bit.
%
% \begin{declcs}{bitsetSetValue} \M{BitSet} \M{Index} \M{Bit}
% \end{declcs}
% Macro \cs{bitsetSetValue} puts bit \meta{Bit} at position \meta{Index}
% in bit set \meta{BitSet}. \meta{Bit} must be a valid \TeX\ number
% equals to zero (disabled/cleared) or one (enabled/set).
%
% \subsubsection{Bit retrieval}\label{sec:bitret}
%
% \begin{declcs}{bitsetGet} \M{BitSet} \M{Index}
% \end{declcs}
% Macro \cs{bitsetGet} extracts the status of the bit at
% position \meta{Index} in bit set \meta{BitSet}.
% Digit |1| is returned if the bit is set/enabled. If the
% bit is cleared/disabled and in cases of an undefined bitset or
% an index number out of range the return value is |0|.
%
% \begin{declcs}{bitsetNextClearBit} \M{BitSet} \M{Index}
% \end{declcs}
% Starting at position \meta{Index} (inclusive) the bits
% are inspected. The first position without a set bit is returned.
% Possible results are decimal numbers: \meta{Index}, \meta{Index} + 1,
% \dots, ($\infty$)
%
% \begin{declcs}{bitsetNextSetBit} \M{BitSet} \M{Index}
% \end{declcs}
% Starting at position \meta{Index} (inclusive) the bits are
% inspected and the index position of the first found set bit
% is returned. If there isn't such a bit, then the result is
% |-1|. In summary possible results are decimal numbers:
% |-1|, \meta{Index}, \meta{Index} + 1, \dots, ($\infty$)
%
% \begin{declcs}{bitsetGetSetBitList} \M{BitSet}
% \end{declcs}
% Macro \cs{bitsetGetSetBitList} is an application for
% \cs{bitsetNextSetBit}. The set bits are iterated and
% returned as comma separated list of index positions
% in increasing order. The list is empty in case of
% an empty bit set.
%
% \subsubsection{Bit set properties}\label{sec:prop}
%
% \begin{declcs}{bitsetSize} \M{BitSet}
% \end{declcs}
% Macro \cs{bitsetSize} returns number of bits in use.
% It is the same as the index number of the highest set/enabled bit
% incremented by one.
%
% \begin{declcs}{bitsetCardinality} \M{BitSet}
% \end{declcs}
% Macro \cs{bitsetCardinality} counts the number of set/enabled bits.
%
% \subsubsection{Queries}\label{sec:query}
%
% Also the query procedures are expandable. They ask for a piece
% of information about a bit set and execute code depending
% on the answer.
%
% \begin{declcs}{bitsetIsDefined} \M{BitSet} \M{Then} \M{Else}
% \end{declcs}
% If the bit set with the name \meta{BitSet} exists the code
% given in \meta{Then} is executed, otherwise \meta{Else} is used.
%
% \begin{declcs}{bitsetIsEmpty} \M{BitSet} \M{Then} \M{Else}
% \end{declcs}
% If the bit set \meta{BitSet} exists and at least one bit
% is set/enabled, the code in \meta{Then} is executed,
% \meta{Else} otherwise.
%
% \begin{declcs}{bitsetEquals} \M{BitSet A} \M{BitSet B} \M{Then} \M{Else}
% \end{declcs}
% Both bit sets are equal if and only if either both are undefined
% or both are defined and represents the same bit values at the
% same positions. Thus this definition is reflexive, symmeric,
% and transitive, enough for an equivalent relation.
%
% \begin{declcs}{bitsetIntersects} \M{BitSet A} \M{BitSet B} \M{Then} \M{Else}
% \end{declcs}
% If and only if \meta{BitSet A} and \meta{BitSet B} have
% at least one bit at the same position that is set, then
% code part \meta{Then} is executed.
%
% \begin{declcs}{bitsetQuery} \M{BitSet} \M{Index} \M{Then} \M{Else}
% \end{declcs}
% It's just a wrapper for \cs{bitsetGet}. If the bit at position
% \meta{Index} is enabled, code \meta{Then} is called.
%
% \StopEventually{
% }
%
% \section{Implementation}
%
% The internal format of a bit set is quite simple, a sequence
% of digits |0| and |1|. The least significant bit is left.
% A bit set without any flag set is encoded by |0|. Also
% undefined bit sets are treated that way. After the
% highest bit that is set there are no further zeroes.
% A regular expression of valid bit sets values:
% \begin{quote}
%  |0|\string||[01]*1|
% \end{quote}
%
%    \begin{macrocode}
%<*package>
%    \end{macrocode}
%
% \subsection{Reload check and package identification}
%    Reload check, especially if the package is not used with \LaTeX.
%    \begin{macrocode}
\begingroup\catcode61\catcode48\catcode32=10\relax%
  \catcode13=5 % ^^M
  \endlinechar=13 %
  \catcode35=6 % #
  \catcode39=12 % '
  \catcode44=12 % ,
  \catcode45=12 % -
  \catcode46=12 % .
  \catcode58=12 % :
  \catcode64=11 % @
  \catcode123=1 % {
  \catcode125=2 % }
  \expandafter\let\expandafter\x\csname ver@bitset.sty\endcsname
  \ifx\x\relax % plain-TeX, first loading
  \else
    \def\empty{}%
    \ifx\x\empty % LaTeX, first loading,
      % variable is initialized, but \ProvidesPackage not yet seen
    \else
      \expandafter\ifx\csname PackageInfo\endcsname\relax
        \def\x#1#2{%
          \immediate\write-1{Package #1 Info: #2.}%
        }%
      \else
        \def\x#1#2{\PackageInfo{#1}{#2, stopped}}%
      \fi
      \x{bitset}{The package is already loaded}%
      \aftergroup\endinput
    \fi
  \fi
\endgroup%
%    \end{macrocode}
%    Package identification:
%    \begin{macrocode}
\begingroup\catcode61\catcode48\catcode32=10\relax%
  \catcode13=5 % ^^M
  \endlinechar=13 %
  \catcode35=6 % #
  \catcode39=12 % '
  \catcode40=12 % (
  \catcode41=12 % )
  \catcode44=12 % ,
  \catcode45=12 % -
  \catcode46=12 % .
  \catcode47=12 % /
  \catcode58=12 % :
  \catcode64=11 % @
  \catcode91=12 % [
  \catcode93=12 % ]
  \catcode123=1 % {
  \catcode125=2 % }
  \expandafter\ifx\csname ProvidesPackage\endcsname\relax
    \def\x#1#2#3[#4]{\endgroup
      \immediate\write-1{Package: #3 #4}%
      \xdef#1{#4}%
    }%
  \else
    \def\x#1#2[#3]{\endgroup
      #2[{#3}]%
      \ifx#1\@undefined
        \xdef#1{#3}%
      \fi
      \ifx#1\relax
        \xdef#1{#3}%
      \fi
    }%
  \fi
\expandafter\x\csname ver@bitset.sty\endcsname
\ProvidesPackage{bitset}%
  [2011/01/30 v1.1 Handle bit-vector datatype (HO)]%
%    \end{macrocode}
%
% \subsection{Catcodes}
%
%    \begin{macrocode}
\begingroup\catcode61\catcode48\catcode32=10\relax%
  \catcode13=5 % ^^M
  \endlinechar=13 %
  \catcode123=1 % {
  \catcode125=2 % }
  \catcode64=11 % @
  \def\x{\endgroup
    \expandafter\edef\csname BitSet@AtEnd\endcsname{%
      \endlinechar=\the\endlinechar\relax
      \catcode13=\the\catcode13\relax
      \catcode32=\the\catcode32\relax
      \catcode35=\the\catcode35\relax
      \catcode61=\the\catcode61\relax
      \catcode64=\the\catcode64\relax
      \catcode123=\the\catcode123\relax
      \catcode125=\the\catcode125\relax
    }%
  }%
\x\catcode61\catcode48\catcode32=10\relax%
\catcode13=5 % ^^M
\endlinechar=13 %
\catcode35=6 % #
\catcode64=11 % @
\catcode123=1 % {
\catcode125=2 % }
\def\TMP@EnsureCode#1#2{%
  \edef\BitSet@AtEnd{%
    \BitSet@AtEnd
    \catcode#1=\the\catcode#1\relax
  }%
  \catcode#1=#2\relax
}
\TMP@EnsureCode{33}{12}% !
\TMP@EnsureCode{39}{12}% '
\TMP@EnsureCode{40}{12}% (
\TMP@EnsureCode{41}{12}% )
\TMP@EnsureCode{42}{12}% *
\TMP@EnsureCode{43}{12}% +
\TMP@EnsureCode{44}{12}% ,
\TMP@EnsureCode{45}{12}% -
\TMP@EnsureCode{46}{12}% .
\TMP@EnsureCode{47}{12}% /
\TMP@EnsureCode{58}{11}% : (letter!)
\TMP@EnsureCode{60}{12}% <
\TMP@EnsureCode{62}{12}% >
\TMP@EnsureCode{63}{14}% ? (comment!)
\TMP@EnsureCode{91}{12}% [
\TMP@EnsureCode{93}{12}% ]
\TMP@EnsureCode{96}{12}% `
\edef\BitSet@AtEnd{\BitSet@AtEnd\noexpand\endinput}
\begingroup\expandafter\expandafter\expandafter\endgroup
\expandafter\ifx\csname BitSet@TestMode\endcsname\relax
\else
  \catcode63=9 % ? (ignore)
\fi
? \let\BitSet@@TestMode\BitSet@TestMode
%    \end{macrocode}
%
% \subsection{Package loading}
%
%    \begin{macrocode}
\begingroup\expandafter\expandafter\expandafter\endgroup
\expandafter\ifx\csname RequirePackage\endcsname\relax
  \def\TMP@RequirePackage#1[#2]{%
    \begingroup\expandafter\expandafter\expandafter\endgroup
    \expandafter\ifx\csname ver@#1.sty\endcsname\relax
      \input #1.sty\relax
    \fi
  }%
  \TMP@RequirePackage{infwarerr}[2007/09/09]%
  \TMP@RequirePackage{intcalc}[2007/09/27]%
  \TMP@RequirePackage{bigintcalc}[2007/09/27]%
\else
  \RequirePackage{infwarerr}[2007/09/09]%
  \RequirePackage{intcalc}[2007/09/27]%
  \RequirePackage{bigintcalc}[2007/09/27]%
\fi
%    \end{macrocode}
%
% \subsection{Help macros}
%
% \subsubsection{Number constant}
%
%    \begin{macro}{\BitSet@MaxSize}
%    \begin{macrocode}
\def\BitSet@MaxSize{2147483647}%
%    \end{macrocode}
%    \end{macro}
%
% \subsubsection{General basic macros}
%
%    \begin{macro}{\BitSet@Empty}
%    \begin{macrocode}
\def\BitSet@Empty{}
%    \end{macrocode}
%    \end{macro}
%    \begin{macro}{\BitSet@FirstOfOne}
%    \begin{macrocode}
\def\BitSet@FirstOfOne#1{#1}
%    \end{macrocode}
%    \end{macro}
%    \begin{macro}{\BitSet@Gobble}
%    \begin{macrocode}
\def\BitSet@Gobble#1{}
%    \end{macrocode}
%    \end{macro}
%    \begin{macro}{\BitSet@FirstOfTwo}
%    \begin{macrocode}
\def\BitSet@FirstOfTwo#1#2{#1}
%    \end{macrocode}
%    \end{macro}
%    \begin{macro}{\BitSet@SecondOfTwo}
%    \begin{macrocode}
\def\BitSet@SecondOfTwo#1#2{#2}
%    \end{macrocode}
%    \end{macro}
%
%    \begin{macro}{\BitSet@Space}
%    \begin{macrocode}
\def\BitSet@Space{ }
%    \end{macrocode}
%    \end{macro}
%    \begin{macro}{\BitSet@ZapSpace}
%    \begin{macrocode}
\def\BitSet@ZapSpace#1 #2{%
  #1%
  \ifx\BitSet@Empty#2%
  \else
    \expandafter\BitSet@ZapSpace
  \fi
  #2%
}
%    \end{macrocode}
%    \end{macro}
%
% \subsubsection{Tail recursion}
%
%    \begin{macro}{\BitSet@Fi}
%    \begin{macrocode}
\let\BitSet@Fi\fi
%    \end{macrocode}
%    \end{macro}
%    \begin{macro}{\BitSet@AfterFi}
%    \begin{macrocode}
\def\BitSet@AfterFi#1#2\BitSet@Fi{\fi#1}
%    \end{macrocode}
%    \end{macro}
%    \begin{macro}{\BitSet@AfterFiFi}
%    \begin{macrocode}
\def\BitSet@AfterFiFi#1#2\BitSet@Fi{\fi\fi#1}%
%    \end{macrocode}
%    \end{macro}
%    \begin{macro}{\BitSet@AfterFiFiFi}
%    \begin{macrocode}
\def\BitSet@AfterFiFiFi#1#2\BitSet@Fi{\fi\fi\fi#1}%
%    \end{macrocode}
%    \end{macro}
%
% \subsubsection{Check macros}
%
%    \begin{macro}{\BitSet@IfUndefined}
%    \begin{macrocode}
\def\BitSet@IfUndefined#1{%
  \expandafter\ifx\csname BS@#1\endcsname\relax
    \expandafter\BitSet@FirstOfTwo
  \else
    \expandafter\BitSet@SecondOfTwo
  \fi
}
%    \end{macrocode}
%    \end{macro}
%
%    \begin{macro}{\BitSet@CheckIndex}
%    |#1|: continuation code\\
%    |#2|: BitSet\\
%    |#3|: Index
%    \begin{macrocode}
\def\BitSet@CheckIndex#1#2#3{%
  \BitSet@IfUndefined{#2}{\bitsetReset{#2}}{}%
  \expandafter\expandafter\expandafter\BitSet@@CheckIndex
  \intcalcNum{#3}!%
  {#2}{#1}%
}
%    \end{macrocode}
%    \end{macro}
%    \begin{macro}{\BitSet@@CheckIndex}
%    |#1|: plain Index\\
%    |#2|: BitSet\\
%    |#3|: continuation code
%    \begin{macrocode}
\def\BitSet@@CheckIndex#1!#2#3{%
  \ifnum#1<0 %
    \BitSet@AfterFi{%
      \@PackageError{bitset}{%
        Invalid negative index (#1)%
      }\@ehc
    }%
  \else
    \BitSet@AfterFi{%
      #3{#2}{#1}%
    }%
  \BitSet@Fi
}
%    \end{macrocode}
%    \end{macro}
%
% \subsection{Miscellaneous}\label{src:misc}
%
%    \begin{macro}{\bitsetReset}
%    \begin{macrocode}
\def\bitsetReset#1{%
  \expandafter\def\csname BS@#1\endcsname{0}%
}
%    \end{macrocode}
%    \end{macro}
%
%    \begin{macro}{\bitsetLet}
%    \begin{macrocode}
\def\bitsetLet#1#2{%
  \BitSet@IfUndefined{#2}{%
    \bitsetReset{#1}%
  }{%
    \expandafter\let\csname BS@#1\expandafter\endcsname
                    \csname BS@#2\endcsname
  }%
}
%    \end{macrocode}
%    \end{macro}
%
% \subsection{Import}\label{src:import}
%
% \subsubsection{From binary number}
%
%    \begin{macro}{\bitsetSetBin}
%    \begin{macrocode}
\def\bitsetSetBin#1#2{%
  \edef\BitSet@Temp{#2}%
  \edef\BitSet@Temp{%
    \expandafter\expandafter\expandafter\BitSet@ZapSpace
    \expandafter\BitSet@Temp\BitSet@Space\BitSet@Empty
  }%
  \edef\BitSet@Temp{%
    \expandafter\BitSet@KillZeros\BitSet@Temp\BitSet@Empty
  }%
  \ifx\BitSet@Temp\BitSet@Empty
    \expandafter\let\csname BS@#1\endcsname\BitSet@Zero
  \else
    \expandafter\edef\csname BS@#1\endcsname{%
      \expandafter\BitSet@Reverse\BitSet@Temp!%
    }%
  \fi
}
%    \end{macrocode}
%    \end{macro}
%    \begin{macro}{\BitSet@KillZeros}
%    \begin{macrocode}
\def\BitSet@KillZeros#1{%
  \ifx#10%
    \expandafter\BitSet@KillZeros
  \else
    #1%
  \fi
}
%    \end{macrocode}
%    \end{macro}
%    \begin{macro}{\BitSet@Reverse}
%    \begin{macrocode}
\def\BitSet@Reverse#1#2!{%
  \ifx\\#2\\%
    #1%
  \else
    \BitSet@AfterFi{%
      \BitSet@Reverse#2!#1%
    }%
  \BitSet@Fi
}
%    \end{macrocode}
%    \end{macro}
%
% \subsubsection{From octal/hex number}
%
%    \begin{macro}{\bitsetSetOct}
%    \begin{macrocode}
\def\bitsetSetOct{%
  \BitSet@SetOctHex\BitSet@FromFirstOct
}
%    \end{macrocode}
%    \end{macro}
%    \begin{macro}{\bitsetSetHex}
%    \begin{macrocode}
\def\bitsetSetHex{%
  \BitSet@SetOctHex\BitSet@FromFirstHex
}
%    \end{macrocode}
%    \end{macro}
%    \begin{macro}{\BitSet@SetOctHex}
%    \begin{macrocode}
\def\BitSet@SetOctHex#1#2#3{%
  \edef\BitSet@Temp{#3}%
  \edef\BitSet@Temp{%
    \expandafter\expandafter\expandafter\BitSet@ZapSpace
    \expandafter\BitSet@Temp\BitSet@Space\BitSet@Empty
  }%
  \edef\BitSet@Temp{%
    \expandafter\BitSet@KillZeros\BitSet@Temp\BitSet@Empty
  }%
  \ifx\BitSet@Temp\BitSet@Empty
    \expandafter\let\csname BS@#2\endcsname\BitSet@Zero
  \else
    \edef\BitSet@Temp{%
      \expandafter#1\BitSet@Temp!%
    }%
    \ifx\BitSet@Temp\BitSet@Empty
      \expandafter\let\csname BS@#2\endcsname\BitSet@Zero
    \else
      \expandafter\edef\csname BS@#2\endcsname{%
        \expandafter\BitSet@Reverse\BitSet@Temp!%
      }%
    \fi
  \fi
}
%    \end{macrocode}
%    \end{macro}
%    \begin{macro}{\BitSet@FromFirstOct}
%    \begin{macrocode}
\def\BitSet@FromFirstOct#1{%
  \ifx#1!%
  \else
    \ifcase#1 \BitSet@AfterFiFi\BitSet@FromFirstOct
    \or 1%
    \or 10%
    \or 11%
    \or 100%
    \or 101%
    \or 110%
    \or 111%
    \else \BitSetError:WrongOctalDigit%
    \fi
    \expandafter\BitSet@FromOct
  \BitSet@Fi
}
%    \end{macrocode}
%    \end{macro}
%    \begin{macro}{\BitSet@FromOct}
%    \begin{macrocode}
\def\BitSet@FromOct#1{%
  \ifx#1!%
  \else
    \ifcase#1 000%
    \or 001%
    \or 010%
    \or 011%
    \or 100%
    \or 101%
    \or 110%
    \or 111%
    \else \BitSetError:WrongOctalDigit%
    \fi
    \expandafter\BitSet@FromOct
  \fi
}
%    \end{macrocode}
%    \end{macro}
%
%    \begin{macro}{\BitSet@FromFirstHex}
%    \begin{macrocode}
\def\BitSet@FromFirstHex#1{%
  \ifx#1!%
  \else
    \ifx#10%
      \BitSet@AfterFiFi\BitSet@FromFirstHex
    \fi
    \expandafter\ifx\csname BitSet@Hex#1\endcsname\relax
      \BitSetError:InvalidHexDigit%
    \else
      \expandafter\expandafter\expandafter\BitSet@KillZeros
      \csname BitSet@Hex#1\endcsname
    \fi
    \expandafter\BitSet@FromHex
  \BitSet@Fi
}
%    \end{macrocode}
%    \end{macro}
%    \begin{macro}{\BitSet@FromHex}
%    \begin{macrocode}
\def\BitSet@FromHex#1{%
  \ifx#1!%
  \else
    \expandafter\ifx\csname BitSet@Hex#1\endcsname\relax
      \BitSetError:InvalidHexDigit%
    \else
      \csname BitSet@Hex#1\endcsname
    \fi
    \expandafter\BitSet@FromHex
  \fi
}
%    \end{macrocode}
%    \end{macro}
%    \begin{macro}{\BitSet@Hex[0..F]}
%    \begin{macrocode}
\def\BitSet@Temp#1{%
  \expandafter\def\csname BitSet@Hex#1\endcsname
}
\BitSet@Temp 0{0000}%
\BitSet@Temp 1{0001}%
\BitSet@Temp 2{0010}%
\BitSet@Temp 3{0011}%
\BitSet@Temp 4{0100}%
\BitSet@Temp 5{0101}%
\BitSet@Temp 6{0110}%
\BitSet@Temp 7{0111}%
\BitSet@Temp 8{1000}%
\BitSet@Temp 9{1001}%
\BitSet@Temp A{1010}%
\BitSet@Temp B{1011}%
\BitSet@Temp C{1100}%
\BitSet@Temp D{1101}%
\BitSet@Temp E{1110}%
\BitSet@Temp F{1111}%
\BitSet@Temp a{1010}%
\BitSet@Temp b{1011}%
\BitSet@Temp c{1100}%
\BitSet@Temp d{1101}%
\BitSet@Temp e{1110}%
\BitSet@Temp f{1111}%
%    \end{macrocode}
%    \end{macro}
%
% \subsubsection{From decimal number}
%
%    \begin{macro}{\bitsetSetDec}
%    \begin{macrocode}
\def\bitsetSetDec#1#2{%
  \edef\BitSet@Temp{#2}%
  \edef\BitSet@Temp{%
    \expandafter\expandafter\expandafter\BitSet@ZapSpace
    \expandafter\BitSet@Temp\BitSet@Space\BitSet@Empty
  }%
  \edef\BitSet@Temp{%
    \expandafter\BitSet@KillZeros\BitSet@Temp\BitSet@Empty
  }%
  \ifx\BitSet@Temp\BitSet@Empty
    \expandafter\let\csname BS@#1\endcsname\BitSet@Zero
  \else
    \ifcase\bigintcalcSgn{\BitSet@Temp} %
      \expandafter\let\csname BS@#1\endcsname\BitSet@Zero
    \or
      \ifnum\bigintcalcCmp\BitSet@Temp\BitSet@MaxSize>0 %
        \expandafter\edef\csname BS@#1\endcsname{%
          \expandafter\BitSet@SetDecBig\BitSet@Temp!%
        }%
      \else
        \expandafter\edef\csname BS@#1\endcsname{%
          \expandafter\BitSet@SetDec\BitSet@Temp!%
        }%
      \fi
    \else
      \@PackageError{bitset}{%
        Bit sets cannot be negative%
      }\@ehc
    \fi
  \fi
}
%    \end{macrocode}
%    \end{macro}
%    \begin{macro}{\BitSet@SetDecBig}
%    \begin{macrocode}
\def\BitSet@SetDecBig#1#2#3#4#5#6#7#8#9!{%
  \ifx\\#9\\%
    \BitSet@SetDec#1#2#3#4#5#6#7#8!%
  \else
    \ifcase\BigIntCalcOdd#1#2#4#5#6#7#8#9! %
      0%
    \or
      1%
?   \else\BitSetError:ThisCannotHappen%
    \fi
    \BitSet@AfterFi{%
      \expandafter\expandafter\expandafter\BitSet@SetDecBig
      \BigIntCalcShr#1#2#3#4#5#6#7#8#9!!%
    }%
  \BitSet@Fi
}
%    \end{macrocode}
%    \end{macro}
%    \begin{macro}{\BitSet@SetDec}
%    \begin{macrocode}
\def\BitSet@SetDec#1!{%
  \ifcase#1 %
  \or 1%
  \else
    \ifodd#1 %
      1%
    \else
      0%
    \fi
    \BitSet@AfterFi{%
      \expandafter\expandafter\expandafter\BitSet@SetDec
      \IntCalcShr#1!!%
    }%
  \BitSet@Fi
}
%    \end{macrocode}
%    \end{macro}
%
% \subsection{Export}\label{src:export}
%
% \subsubsection{To binary number}
%
%    \begin{macro}{\bitsetGetBin}
%    \begin{macrocode}
\def\bitsetGetBin#1#2{%
  \romannumeral0%
  \expandafter\expandafter\expandafter\BitSet@@GetBin
  \intcalcNum{#2}!{#1}%
}
%    \end{macrocode}
%    \end{macro}
%    \begin{macro}{\BitSet@@GetBin}
%    \begin{macrocode}
\def\BitSet@@GetBin#1!#2{%
  \BitSet@IfUndefined{#2}{%
    \ifnum#1>1 %
      \BitSet@AfterFi{%
        \expandafter\expandafter\expandafter\BitSet@Fill
        \IntCalcDec#1!!0%
      }%
    \else
      \BitSet@AfterFi{ 0}%
    \BitSet@Fi
  }{%
    \expandafter\expandafter\expandafter\BitSet@NumBinRev
    \expandafter\expandafter\expandafter1%
    \expandafter\expandafter\expandafter!%
    \csname BS@#2\endcsname!!#1!%
  }%
}
%    \end{macrocode}
%    \end{macro}
%    \begin{macro}{\BitSet@Fill}
%    |#1|: number of leading digits |0|\\
%    |#2|: result
%    \begin{macrocode}
\def\BitSet@Fill#1!{%
  \ifnum#1>0 %
    \BitSet@AfterFi{%
      \expandafter\expandafter\expandafter\BitSet@Fill
      \IntCalcDec#1!!0%
    }%
  \else
    \BitSet@AfterFi{ }%
  \BitSet@Fi
}
%    \end{macrocode}
%    \end{macro}
%
%    \begin{macro}{\BitSet@NumBinRev}
%    |#1|: bit counter (including |#2|)\\
%    |#2#3|: reverted number\\
%    |#4|: result\\
%    |#5|: min size
%    \begin{macrocode}
\def\BitSet@NumBinRev#1!#2#3!{%
  \ifx\\#3\\%
    \BitSet@AfterFi{%
      \BitSet@NumBinFill#1!#2%
    }%
  \else
    \BitSet@AfterFi{%
      \expandafter\expandafter\expandafter\BitSet@NumBinRev
      \IntCalcInc#1!!#3!#2%
    }%
  \BitSet@Fi
}
%    \end{macrocode}
%    \end{macro}
%    \begin{macro}{\BitSet@NumBinFill}
%    \begin{macrocode}
\def\BitSet@NumBinFill#1!#2!#3!{%
  \ifnum#3>#1 %
    \BitSet@AfterFi{%
      \expandafter\expandafter\expandafter\BitSet@Fill
      \IntCalcSub#3!#1!!#2%
    }%
  \else
    \BitSet@AfterFi{ #2}%
  \BitSet@Fi
}
%    \end{macrocode}
%    \end{macro}
%
% \subsubsection{To octal/hexadecimal number}
%
%    \begin{macro}{\bitsetGetOct}
%    \begin{macrocode}
\def\bitsetGetOct#1#2{%
  \romannumeral0%
  \bitsetIsEmpty{#1}{%
    \expandafter\expandafter\expandafter\BitSet@@GetOctHex
    \intcalcNum{#2}!3!230%
  }{%
    \expandafter\expandafter\expandafter\BitSet@@GetOct
    \expandafter\expandafter\expandafter1%
    \expandafter\expandafter\expandafter!%
    \expandafter\expandafter\expandafter!%
    \csname BS@#1\endcsname00%
    \BitSet@Empty\BitSet@Empty\BitSet@Empty!{#2}%
  }%
}
%    \end{macrocode}
%    \end{macro}
%    \begin{macro}{\bitsetGetHex}
%    \begin{macrocode}
\def\bitsetGetHex#1#2{%
  \romannumeral0%
  \bitsetIsEmpty{#1}{%
    \expandafter\expandafter\expandafter\BitSet@@GetOctHex
    \intcalcNum{#2}!4!340%
  }{%
    \expandafter\expandafter\expandafter\BitSet@@GetHex
    \expandafter\expandafter\expandafter1%
    \expandafter\expandafter\expandafter!%
    \expandafter\expandafter\expandafter!%
    \csname BS@#1\endcsname000%
    \BitSet@Empty\BitSet@Empty\BitSet@Empty\BitSet@Empty!{#2}%
  }%
}
%    \end{macrocode}
%    \end{macro}
%    \begin{macro}{\BitSet@@GetOct}
%    |#1|: number of digits\\
%    |#2|: result\\
%    |#3#4#5|: bits
%    \begin{macrocode}
\def\BitSet@@GetOct#1!#2!#3#4#5{%
  \ifx#5\BitSet@Empty
    \BitSet@AfterFi{%
      \expandafter\expandafter\expandafter\BitSet@GetOctHex
      \IntCalcDec#1!!#2!23%
    }%
  \else
    \BitSet@AfterFi{%
      \expandafter\expandafter\expandafter\BitSet@@GetOct
      \number\IntCalcInc#1!\expandafter\expandafter\expandafter!%
      \csname BitSet@Oct#5#4#3\endcsname#2!%
    }%
  \BitSet@Fi
}
%    \end{macrocode}
%    \end{macro}
%    \begin{macro}{\BitSet@Oct[000..111]}
%    \begin{macrocode}
\def\BitSet@Temp#1#2#3#4{%
  \expandafter\def\csname BitSet@Oct#1#2#3\endcsname{#4}%
}
\BitSet@Temp0000%
\BitSet@Temp0011%
\BitSet@Temp0102%
\BitSet@Temp0113%
\BitSet@Temp1004%
\BitSet@Temp1015%
\BitSet@Temp1106%
\BitSet@Temp1117%
%    \end{macrocode}
%    \end{macro}
%    \begin{macro}{\BitSet@@GetHex}
%    |#1|: number of digits\\
%    |#2|: result\\
%    |#3#4#5#6|: bits
%    \begin{macrocode}
\def\BitSet@@GetHex#1!#2!#3#4#5#6{%
  \ifx#6\BitSet@Empty
    \BitSet@AfterFi{%
      \expandafter\expandafter\expandafter\BitSet@GetOctHex
      \IntCalcDec#1!!#2!34%
    }%
  \else
    \BitSet@AfterFi{%
      \expandafter\expandafter\expandafter\BitSet@@GetHex
      \number\IntCalcInc#1!\expandafter\expandafter\expandafter!%
      \csname BitSet@Hex#6#5#4#3\endcsname#2!%
    }%
  \BitSet@Fi
}
%    \end{macrocode}
%    \end{macro}
%    \begin{macro}{\BitSet@Hex[0000..1111]}
%    \begin{macrocode}
\def\BitSet@Temp#1#2#3#4#5{%
  \expandafter\def\csname BitSet@Hex#1#2#3#4\endcsname{#5}%
}
\BitSet@Temp00000%
\BitSet@Temp00011%
\BitSet@Temp00102%
\BitSet@Temp00113%
\BitSet@Temp01004%
\BitSet@Temp01015%
\BitSet@Temp01106%
\BitSet@Temp01117%
\BitSet@Temp10008%
\BitSet@Temp10019%
\BitSet@Temp1010A%
\BitSet@Temp1011B%
\BitSet@Temp1100C%
\BitSet@Temp1101D%
\BitSet@Temp1110E%
\BitSet@Temp1111F%
%    \end{macrocode}
%    \end{macro}
%    \begin{macro}{\BitSet@GetOctHex}
%    Leading zeros $ (|#4| - |#1| * 3 + 2)/3 $ if $ |#4| > |#1| * 3$\\
%    |#1|: digit size\\
%    |#2|: result\\
%    |#3|: bits per digit - 1\\
%    |#4|: bits per digit
%    |#5|: garbage\\
%    |#6|: min size
%    \begin{macrocode}
\def\BitSet@GetOctHex#1!#2!#3#4#5!#6{%
  \expandafter\BitSet@@GetOctHex
  \number\intcalcNum{#6}\expandafter\expandafter\expandafter!%
  \IntCalcMul#1!#4!!#3#4#2%
}
%    \end{macrocode}
%    \end{macro}
%    \begin{macro}{\BitSet@@GetOctHex}
%    |#1|: plain min size\\
%    |#2|: digits * (bits per digit)\\
%    |#3|: bits per digit - 1\\
%    |#4|: bits per digit
%    \begin{macrocode}
\def\BitSet@@GetOctHex#1!#2!#3#4{%
  \ifnum#1>#2 %
    \BitSet@AfterFi{%
      \expandafter\expandafter\expandafter\expandafter
      \expandafter\expandafter\expandafter\BitSet@Fill
      \expandafter\IntCalcDiv\number
      \expandafter\expandafter\expandafter\IntCalcAdd
      \IntCalcSub#1!#2!!#3!!#4!!%
    }%
  \else
    \BitSet@AfterFi{ }%
  \BitSet@Fi
}
%    \end{macrocode}
%    \end{macro}
%
% \subsubsection{To decimal number}
%
%    \begin{macro}{\bitsetGetDec}
%    \begin{macrocode}
\def\bitsetGetDec#1{%
  \romannumeral0%
  \BitSet@IfUndefined{#1}{ 0}{%
    \expandafter\expandafter\expandafter\BitSet@GetDec
    \csname BS@#1\endcsname!%
  }%
}
%    \end{macrocode}
%    \end{macro}
%    \begin{macro}{\BitSet@GetDec}
%    \begin{macrocode}
\def\BitSet@GetDec#1#2!{%
  \ifx\\#2\\%
    \BitSet@AfterFi{ #1}%
  \else
    \BitSet@AfterFi{%
      \BitSet@@GetDec2!#1!#2!%
    }%
  \BitSet@Fi
}
%    \end{macrocode}
%    \end{macro}
%    \begin{macro}{\BitSet@@GetDec}
%    |#1|: power of two\\
%    |#2|: result\\
%    |#3#4|: number
%    \begin{macrocode}
\def\BitSet@@GetDec#1!#2!#3#4!{%
  \ifx\\#4\\%
    \ifx#31%
      \BitSet@AfterFiFi{%
        \expandafter\expandafter\expandafter\BitSet@Space
        \IntCalcAdd#1!#2!%
      }%
    \else
      \BitSet@AfterFiFi{ #2}%
    \fi
  \else
    \ifx#31%
      \BitSet@AfterFiFi{%
        \csname BitSet@N#1%
        \expandafter\expandafter\expandafter\endcsname
        \IntCalcAdd#1!#2!!#4!%
      }%
    \else
      \BitSet@AfterFiFi{%
        \csname BitSet@N#1\endcsname#2!#4!%
      }%
    \fi
  \BitSet@Fi
}
%    \end{macrocode}
%    \end{macro}
%    \begin{macro}{\BitSet@N[1,2,4,...]}
%    \begin{macrocode}
\def\BitSet@Temp#1#2{%
  \expandafter\def\csname BitSet@N#1\endcsname{%
    \BitSet@@GetDec#2!%
  }%
}
\BitSet@Temp{1}{2}
\BitSet@Temp{2}{4}
\BitSet@Temp{4}{8}
\BitSet@Temp{8}{16}
\BitSet@Temp{16}{32}
\BitSet@Temp{32}{64}
\BitSet@Temp{64}{128}
\BitSet@Temp{128}{256}
\BitSet@Temp{256}{512}
\BitSet@Temp{512}{1024}
\BitSet@Temp{1024}{2048}
\BitSet@Temp{2048}{4096}
\BitSet@Temp{4096}{8192}
\BitSet@Temp{8192}{16384}
\BitSet@Temp{16384}{32768}
\BitSet@Temp{32768}{65536}
\BitSet@Temp{65536}{131072}
\BitSet@Temp{131072}{262144}
\BitSet@Temp{262144}{524288}
\BitSet@Temp{524288}{1048576}
\BitSet@Temp{1048576}{2097152}
\BitSet@Temp{2097152}{4194304}
\BitSet@Temp{4194304}{8388608}
\BitSet@Temp{8388608}{16777216}
\BitSet@Temp{16777216}{33554432}
\BitSet@Temp{33554432}{67108864}
\BitSet@Temp{67108864}{134217728}
\BitSet@Temp{134217728}{268435456}
\BitSet@Temp{268435456}{536870912}
\BitSet@Temp{536870912}{1073741824}
%    \end{macrocode}
%    \end{macro}
%    \begin{macro}{\BitSet@N1073741824}
%    \begin{macrocode}
\expandafter\def\csname BitSet@N1073741824\endcsname{%
  \BitSet@GetDecBig2147483648!%
}%
%    \end{macrocode}
%    \end{macro}
%
%    \begin{macro}{\BitSet@GetDecBig}
%    |#1|: current power of two\\
%    |#2|: result\\
%    |#3#4|: number
%    \begin{macrocode}
\def\BitSet@GetDecBig#1!#2!#3#4!{%
  \ifx\\#4\\%
    \ifx#31%
      \BitSet@AfterFiFi{%
        \expandafter\expandafter\expandafter\BitSet@Space
        \BigIntCalcAdd#1!#2!%
      }%
    \else
      \BitSet@AfterFiFi{ #2}%
    \fi
  \else
    \ifx#31%
      \BitSet@AfterFiFi{%
        \expandafter\expandafter\expandafter\BitSet@@GetDecBig
        \BigIntCalcAdd#1!#2!!#1!#4!%
      }%
    \else
      \BitSet@AfterFiFi{%
        \expandafter\expandafter\expandafter\BitSet@GetDecBig
        \BigIntCalcShl#1!!#2!#4!%
      }%
    \fi
  \BitSet@Fi
}
%    \end{macrocode}
%    \end{macro}
%    \begin{macro}{\BitSet@@GetDecBig}
%    |#1|: result\\
%    |#2|: power of two\\
%    |#3#4|: number
%    \begin{macrocode}
\def\BitSet@@GetDecBig#1!#2!{%
  \expandafter\expandafter\expandafter\BitSet@GetDecBig
  \BigIntCalcShl#2!!#1!%
}
%    \end{macrocode}
%    \end{macro}
%
% \subsection{Logical operators}\label{src:log}
%
% \subsubsection{\cs{bitsetAnd}}
%
%    \begin{macro}{\bitsetAnd}
%    Decision table for \cs{bitsetAnd}:
%    \begin{quote}
%    \begin{tabular}{l|l|l|l|}
%                         & undef(B)   & empty(B)   & cardinality(B)$>$0\\
%      \hline
%      undef(A)           & A := empty & A := empty & A := empty\\
%      \hline
%      empty(A)           &            &            &\\
%      \hline
%      cardinality(A)$>$0 & A := empty & A := empty & A \&= B\\
%      \hline
%    \end{tabular}
%    \end{quote}
%    \begin{macrocode}
\def\bitsetAnd#1#2{%
  \bitsetIsEmpty{#1}{%
    \bitsetReset{#1}%
  }{%
    \bitsetIsEmpty{#2}{%
      \bitsetReset{#1}%
    }{%
      \expandafter\edef\csname BS@#1\endcsname{%
        \expandafter\expandafter\expandafter\BitSet@And
        \csname BS@#1\expandafter\expandafter\expandafter\endcsname
        \expandafter\expandafter\expandafter!%
        \csname BS@#2\endcsname!!%
      }%
      \expandafter\ifx\csname BS@#1\endcsname\BitSet@Empty
        \bitsetReset{#1}%
      \fi
    }%
  }%
}
%    \end{macrocode}
%    \end{macro}
%    \begin{macro}{\BitSet@And}
%    \begin{macrocode}
\def\BitSet@And#1#2!#3#4!#5!{%
  \ifx\\#2\\%
    \ifnum#1#3=11 #51\fi
  \else
    \ifx\\#4\\%
      \ifnum#1#3=11 #51\fi
    \else
      \ifnum#1#3=11 %
        #51%
        \BitSet@AfterFiFiFi{%
          \BitSet@And#2!#4!!%
        }%
      \else
        \BitSet@AfterFiFiFi{%
          \BitSet@And#2!#4!#50!%
        }%
      \fi
    \fi
  \BitSet@Fi
}
%    \end{macrocode}
%    \end{macro}
%
% \subsubsection{\cs{bitsetAndNot}}
%
%    \begin{macro}{\bitsetAndNot}
%    Decision table for \cs{bitsetAndNot}:
%    \begin{quote}
%    \begin{tabular}{l|l|l|l|}
%                         & undef(B)   & empty(B)   & cardinality(B)$>$0\\
%      \hline
%      undef(A)           & A := empty & A := empty & A := empty\\
%      \hline
%      empty(A)           &            &            &\\
%      \hline
%      cardinality(A)$>$0 &            &            & A \&= !B\\
%      \hline
%    \end{tabular}
%    \end{quote}
%    \begin{macrocode}
\def\bitsetAndNot#1#2{%
  \bitsetIsEmpty{#1}{%
    \bitsetReset{#1}%
  }{%
    \bitsetIsEmpty{#2}{%
    }{%
      \expandafter\edef\csname BS@#1\endcsname{%
        \expandafter\expandafter\expandafter\BitSet@AndNot
        \csname BS@#1\expandafter\expandafter\expandafter\endcsname
        \expandafter\expandafter\expandafter!%
        \csname BS@#2\endcsname!!%
      }%
      \expandafter\ifx\csname BS@#1\endcsname\BitSet@Empty
        \bitsetReset{#1}%
      \fi
    }%
  }%
}
%    \end{macrocode}
%    \end{macro}
%    \begin{macro}{\BitSet@AndNot}
%    \begin{macrocode}
\def\BitSet@AndNot#1#2!#3#4!#5!{%
  \ifx\\#2\\%
    \ifnum#1#3=10 #51\fi
  \else
    \ifx\\#4\\%
      #5%
      \ifnum#1#3=10 1\else 0\fi
      #2%
    \else
      \ifnum#1#3=10 %
        #51%
        \BitSet@AfterFiFiFi{%
          \BitSet@AndNot#2!#4!!%
        }%
      \else
        \BitSet@AfterFiFiFi{%
          \BitSet@AndNot#2!#4!#50!%
        }%
      \fi
    \fi
  \BitSet@Fi
}
%    \end{macrocode}
%    \end{macro}
%
% \subsubsection{\cs{bitsetOr}}
%
%    \begin{macro}{\bitsetOr}
%    Decision table for \cs{bitsetOr}:
%    \begin{quote}
%    \begin{tabular}{l|l|l|l|}
%                         & undef(B)   & empty(B)   & cardinality(B)$>$0\\
%      \hline
%      undef(A)           & A := empty & A := empty & A := B\\
%      \hline
%      empty(A)           &            &            & A := B\\
%      \hline
%      cardinality(A)$>$0 &            &            & A \string|= B\\
%      \hline
%    \end{tabular}
%    \end{quote}
%    \begin{macrocode}
\def\bitsetOr#1#2{%
  \bitsetIsEmpty{#2}{%
    \BitSet@IfUndefined{#1}{\bitsetReset{#1}}{}%
  }{%
    \bitsetIsEmpty{#1}{%
      \expandafter\let\csname BS@#1\expandafter\endcsname
                      \csname BS@#2\endcsname
    }{%
      \expandafter\edef\csname BS@#1\endcsname{%
        \expandafter\expandafter\expandafter\BitSet@Or
        \csname BS@#1\expandafter\expandafter\expandafter\endcsname
        \expandafter\expandafter\expandafter!%
        \csname BS@#2\endcsname!%
      }%
    }%
  }%
}
%    \end{macrocode}
%    \end{macro}
%    \begin{macro}{\BitSet@Or}
%    \begin{macrocode}
\def\BitSet@Or#1#2!#3#4!{%
  \ifnum#1#3>0 1\else 0\fi
  \ifx\\#2\\%
    #4%
  \else
    \ifx\\#4\\%
      #2%
    \else
      \BitSet@AfterFiFi{%
        \BitSet@Or#2!#4!%
      }%
    \fi
  \BitSet@Fi
}
%    \end{macrocode}
%    \end{macro}
%
% \subsubsection{\cs{bitsetXor}}
%
%    \begin{macro}{\bitsetXor}
%    Decision table for \cs{bitsetXor}:
%    \begin{quote}
%    \begin{tabular}{l|l|l|l|}
%                         & undef(B)   & empty(B)   & cardinality(B)$>$0\\
%      \hline
%      undef(A)           & A := empty & A := empty & A := B\\
%      \hline
%      empty(A)           &            &            & A := B\\
%      \hline
%      cardinality(A)$>$0 &            &            & A \string^= B\\
%      \hline
%    \end{tabular}
%    \end{quote}
%    \begin{macrocode}
\def\bitsetXor#1#2{%
  \bitsetIsEmpty{#2}{%
    \BitSet@IfUndefined{#1}{\bitsetReset{#1}}{}%
  }{%
    \bitsetIsEmpty{#1}{%
      \expandafter\let\csname BS@#1\expandafter\endcsname
                      \csname BS@#2\endcsname
    }{%
      \expandafter\edef\csname BS@#1\endcsname{%
        \expandafter\expandafter\expandafter\BitSet@Xor
        \csname BS@#1\expandafter\expandafter\expandafter\endcsname
        \expandafter\expandafter\expandafter!%
        \csname BS@#2\endcsname!!%
      }%
      \expandafter\ifx\csname BS@#1\endcsname\BitSet@Empty
        \bitsetReset{#1}%
      \fi
    }%
  }%
}
%    \end{macrocode}
%    \end{macro}
%    \begin{macro}{\BitSet@Xor}
%    \begin{macrocode}
\def\BitSet@Xor#1#2!#3#4!#5!{%
  \ifx\\#2\\%
    \ifx#1#3%
      \ifx\\#4\\%
      \else
        #50#4%
      \fi
    \else
      #51#4%
    \fi
  \else
    \ifx\\#4\\%
      #5%
      \ifx#1#30\else 1\fi
      #2%
    \else
      \ifx#1#3%
        \BitSet@AfterFiFiFi{%
          \BitSet@Xor#2!#4!#50!%
        }%
      \else
        #51%
        \BitSet@AfterFiFiFi{%
          \BitSet@Xor#2!#4!!%
        }%
      \fi
    \fi
  \BitSet@Fi
}
%    \end{macrocode}
%    \end{macro}
%
% \subsubsection{Shifting}\label{src:shift}
%
% \subsubsection{\cs{bitsetShiftLeft}}
%
%    \begin{macro}{\bitsetShiftLeft}
%    \begin{macrocode}
\def\bitsetShiftLeft#1#2{%
  \BitSet@IfUndefined{#1}{%
    \bitsetReset{#1}%
  }{%
    \bitsetIsEmpty{#1}{%
    }{%
      \expandafter\expandafter\expandafter\BitSet@ShiftLeft
      \intcalcNum{#2}!{#1}%
    }%
  }%
}
%    \end{macrocode}
%    \end{macro}
%    \begin{macro}{\BitSet@ShiftLeft}
%    \begin{macrocode}
\def\BitSet@ShiftLeft#1!#2{%
  \ifcase\intcalcSgn{#1} %
  \or
    \begingroup
      \uccode`m=`0 %
    \uppercase\expandafter{\expandafter\endgroup
      \expandafter\edef\csname BS@#2\expandafter\endcsname
      \expandafter{%
        \romannumeral#1000\expandafter\BitSet@Space
        \csname BS@#2\endcsname
      }%
    }%
  \else
    \expandafter\BitSet@ShiftRight\BitSet@Gobble#1!{#2}%
  \fi
}
%    \end{macrocode}
%    \end{macro}
%
% \subsubsection{\cs{bitsetShiftRight}}
%
%    \begin{macro}{\bitsetShiftRight}
%    \begin{macrocode}
\def\bitsetShiftRight#1#2{%
  \BitSet@IfUndefined{#1}{%
    \bitsetReset{#1}%
  }{%
    \bitsetIsEmpty{#1}{%
    }{%
      \expandafter\expandafter\expandafter\BitSet@ShiftRight
      \intcalcNum{#2}!{#1}%
    }%
  }%
}
%    \end{macrocode}
%    \end{macro}
%    \begin{macro}{\BitSet@ShiftRight}
%    \begin{macrocode}
\def\BitSet@ShiftRight#1!#2{%
  \ifcase\intcalcSgn{#1} %
  \or
    \expandafter\edef\csname BS@#2\endcsname{%
      \expandafter\expandafter\expandafter\BitSet@Kill
      \csname BS@#2\expandafter\endcsname\expandafter\BitSet@Empty
      \expandafter=%
      \expandafter{\expandafter}\expandafter{\expandafter}%
      \romannumeral#1000!%
    }%
  \else
    \expandafter\BitSet@ShiftLeft\BitSet@Gobble#1!{#2}%
  \fi
}
%    \end{macrocode}
%    \end{macro}
%    \begin{macro}{\BitSet@Kill}
%    \begin{macrocode}
\def\BitSet@Kill#1#2=#3#4#5{%
  #3#4%
  \ifx#5!%
    \ifx#1\BitSet@Empty
      0%
    \else
      #1#2%
    \fi
  \else
    \ifx#1\BitSet@Empty
      0%
      \BitSet@AfterFiFi\BitSet@Cleanup
    \else
      \BitSet@Kill#2=%
    \fi
  \BitSet@Fi
}
%    \end{macrocode}
%    \end{macro}
%
% \subsection{Bit manipulation}\label{src:bitman}
%
%    \begin{macro}{\bitsetClear}
%    \begin{macrocode}
\def\bitsetClear{%
  \BitSet@CheckIndex\BitSet@Clear
}
%    \end{macrocode}
%    \end{macro}
%    \begin{macro}{\bitsetSet}
%    \begin{macrocode}
\def\bitsetSet{%
  \BitSet@CheckIndex\BitSet@Set
}
%    \end{macrocode}
%    \end{macro}
%    \begin{macro}{\bitsetFlip}
%    \begin{macrocode}
\def\bitsetFlip{%
  \BitSet@CheckIndex\BitSet@Flip
}
%    \end{macrocode}
%    \end{macro}
%    \begin{macro}{\bitsetSetValue}
%    \begin{macrocode}
\def\bitsetSetValue#1#2#3{%
  \expandafter\expandafter\expandafter\BitSet@SetValue
  \intcalcNum{#3}!{#1}{#2}%
}
%    \end{macrocode}
%    \end{macro}
%    \begin{macro}{\BitSet@SetValue}
%    |#1|: plain value\\
%    |#2|: BitSet\\
%    |#3|: Index
%    \begin{macrocode}
\def\BitSet@SetValue#1!{%
  \BitSet@CheckIndex{%
    \ifcase#1 %
      \expandafter\BitSet@Clear
    \or
      \expandafter\BitSet@Set
    \else
      \BitSet@ErrorInvalidBitValue{#1}%
      \expandafter\expandafter\expandafter\BitSet@Gobble
      \expandafter\BitSet@Gobble
    \fi
  }%
}
%    \end{macrocode}
%    \end{macro}
%    \begin{macro}{\BitSet@ErrorInvalidBitValue}
%    |#1|: Wrong bit value
%    \begin{macrocode}
\def\BitSet@ErrorInvalidBitValue#1{%
  \@PackageError{bitset}{%
    Invalid bit value (#1) not in range 0..1%
  }\@ehc
}
%    \end{macrocode}
%    \end{macro}
%
% \subsubsection{Clear operation}
%
%    \begin{macro}{\BitSet@Clear}
%    |#1|: BitSet\\
%    |#2|: plain and checked index
%    \begin{macrocode}
\def\BitSet@Clear#1#2{%
  \edef\BitSet@Temp{%
    \expandafter\expandafter\expandafter\BitSet@@Clear
    \csname BS@#1\expandafter\endcsname
    \expandafter\BitSet@Empty\expandafter=\expandafter!%
    \romannumeral#2000!%
  }%
  \expandafter\let\csname BS@#1\expandafter\endcsname
  \ifx\BitSet@Temp\BitSet@Empty
    \BitSet@Zero
  \else
    \BitSet@Temp
  \fi
}
%    \end{macrocode}
%    \end{macro}
%    \begin{macro}{\BitSet@@Clear}
%    \begin{macrocode}
\def\BitSet@@Clear#1#2=#3!#4{%
  \ifx#4!%
    \ifx#1\BitSet@Empty
    \else
      \ifx\BitSet@Empty#2%
      \else
        #30#2%
      \fi
    \fi
  \else
    \ifx#1\BitSet@Empty
      \BitSet@AfterFiFi\BitSet@Cleanup
    \else
      \ifx#10%
        \BitSet@AfterFiFiFi{%
          \BitSet@@Clear#2=#30!%
        }%
      \else
        #31%
        \BitSet@AfterFiFiFi{%
          \BitSet@@Clear#2=!%
        }%
      \fi
    \fi
  \BitSet@Fi
}
%    \end{macrocode}
%    \end{macro}
%
% \subsubsection{Set operation}
%
%    \begin{macro}{\BitSet@Set}
%    |#1|: BitSet\\
%    |#2|: plain and checked Index
%    \begin{macrocode}
\def\BitSet@Set#1#2{%
  \expandafter\edef\csname BS@#1\endcsname{%
    \expandafter\expandafter\expandafter\BitSet@@Set
    \csname BS@#1\expandafter\endcsname
    \expandafter\BitSet@Empty\expandafter=%
    \expandafter{\expandafter}\expandafter{\expandafter}%
    \romannumeral#2000!%
  }%
}
%    \end{macrocode}
%    \end{macro}
%    \begin{macro}{\BitSet@@Set}
%    \begin{macrocode}
\def\BitSet@@Set#1#2=#3#4#5{%
  #3#4%
  \ifx#5!%
    1#2%
  \else
    \ifx#1\BitSet@Empty
      0%
      \BitSet@AfterFiFi\BitSet@@@Set
    \else
      #1%
      \BitSet@@Set#2=%
    \fi
  \BitSet@Fi
}
%    \end{macrocode}
%    \end{macro}
%    \begin{macro}{\BitSet@@@Set}
%    \begin{macrocode}
\def\BitSet@@@Set#1{%
  \ifx#1!%
    1%
  \else
    0%
    \expandafter\BitSet@@@Set
  \fi
}
%    \end{macrocode}
%    \end{macro}
%
% \subsubsection{Flip operation}
%
%    \begin{macro}{\BitSet@Flip}
%    |#1|: BitSet\\
%    |#2|: plain and checked Index
%    \begin{macrocode}
\def\BitSet@Flip#1#2{%
  \edef\BitSet@Temp{%
    \expandafter\expandafter\expandafter\BitSet@@Flip
    \csname BS@#1\expandafter\endcsname
    \expandafter\BitSet@Empty\expandafter=\expandafter!%
    \romannumeral#2000!%
  }%
  \expandafter\let\csname BS@#1\expandafter\endcsname
  \ifx\BitSet@Temp\BitSet@Empty
    \BitSet@Zero
  \else
    \BitSet@Temp
  \fi
}
%    \end{macrocode}
%    \end{macro}
%    \begin{macro}{\BitSet@@Flip}
%    \begin{macrocode}
\def\BitSet@@Flip#1#2=#3!#4{%
  \ifx#4!%
    \ifx#11%
      \ifx\BitSet@Empty#2%
      \else
        #30#2%
      \fi
    \else
      #31#2%
    \fi
  \else
    \ifx#1\BitSet@Empty
      #30%
      \BitSet@AfterFiFi\BitSet@@@Set
    \else
      \ifx#10%
        \BitSet@AfterFiFiFi{%
          \BitSet@@Flip#2=#30!%
        }%
      \else
        #31%
        \BitSet@AfterFiFiFi{%
          \BitSet@@Flip#2=!%
        }%
      \fi
    \fi
  \BitSet@Fi
}
%    \end{macrocode}
%    \end{macro}
%
% \subsubsection{Range operators}
%
%    \begin{macro}{\bitsetClearRange}
%    \begin{macrocode}
\def\bitsetClearRange{%
  \BitSet@Range\BitSet@Clear
}
%    \end{macrocode}
%    \end{macro}
%    \begin{macro}{\bitsetSetRange}
%    \begin{macrocode}
\def\bitsetSetRange{%
  \BitSet@Range\BitSet@Set
}
%    \end{macrocode}
%    \end{macro}
%    \begin{macro}{\bitsetFlipRange}
%    \begin{macrocode}
\def\bitsetFlipRange{%
  \BitSet@Range\BitSet@Flip
}
%    \end{macrocode}
%    \end{macro}
%    \begin{macro}{\bitsetSetValueRange}
%    \begin{macrocode}
\def\bitsetSetValueRange#1#2#3#4{%
  \expandafter\expandafter\expandafter\BitSet@SetValueRange
  \intcalcNum{#4}!{#1}{#2}{#3}%
}
%    \end{macrocode}
%    \end{macro}
%    \begin{macro}{\BitSet@SetValueRange}
%    \begin{macrocode}
\def\BitSet@SetValueRange#1!#2#3#4{%
  \ifcase#1 %
    \BitSet@Range\BitSet@Clear{#2}{#3}{#4}%
  \or
    \BitSet@Range\BitSet@Set{#2}{#3}{#4}%
  \else
    \BitSet@ErrorInvalidBitValue{#1}%
  \fi
}
%    \end{macrocode}
%    \end{macro}
%
%    \begin{macro}{\BitSet@Range}
%    |#1|: clear/set/flip macro\\
%    |#2|: BitSet\\
%    |#3|: Index from\\
%    |#4|: Index to
%    \begin{macrocode}
\def\BitSet@Range#1#2#3#4{%
  \edef\BitSet@Temp{%
    \noexpand\BitSet@@Range\noexpand#1{#2}%
    \intcalcNum{#3}!\intcalcNum{#4}!%
  }%
  \BitSet@Temp
}
%    \end{macrocode}
%    \end{macro}
%    \begin{macro}{\BitSet@@Range}
%    |#1|: clear/set/flip macro\\
%    |#2|: BitSet\\
%    |#3|: Index from\\
%    |#4|: Index to
%    \begin{macrocode}
\def\BitSet@@Range#1#2#3!#4!{%
  \ifnum#3<0 %
    \BitSet@NegativeIndex#1{#2}#3!#4!0!#4!%
  \else
    \ifnum#4<0 %
      \BitSet@NegativeIndex#1{#2}#3!#4!#3!0!%
    \else
      \ifcase\intcalcCmp{#3}{#4} %
      \or
        \@PackageError{bitset}{%
          Wrong index numbers in range [#3..#4]\MessageBreak% hash-ok
          for clear/set/flip on bit set `#2'.\MessageBreak
          The lower index exceeds the upper index.\MessageBreak
          Canceling the operation as error recovery%
        }\@ehc
      \else
        \BitSet@@@Range#3!#4!#1{#2}%
      \fi
    \fi
  \fi
}
%    \end{macrocode}
%    \end{macro}
%    \begin{macro}{\BitSet@NegativeIndex}
%    \begin{macrocode}
\def\BitSet@NegativeIndex#1#2#3!#4!#5!#6!{%
  \@PackageError{bitset}{%
    Negative index in range [#3..#4]\MessageBreak % hash-ok
    for \string\bitset
    \ifx#1\BitSet@Clear
      Clear%
    \else
      \ifx#1\BitSet@Set
        Set%
      \else
        Flip%
      \fi
    \fi
    Range on bit set `#2'.\MessageBreak
    Using [#5..#6] as error recovery% hash-ok
  }\@ehc
  \BitSet@@Range#1{#2}#5!#6!%
}
%    \end{macrocode}
%    \end{macro}
%    \begin{macro}{\BitSet@@Range}
%    \begin{macrocode}
\def\BitSet@@@Range#1!#2!#3#4{%
  \ifnum#1<#2 %
    #3{#4}{#1}%
    \BitSet@AfterFi{%
      \expandafter\expandafter\expandafter\BitSet@@@Range
      \IntCalcInc#1!!#2!#3{#4}%
    }%
  \BitSet@Fi
}
%    \end{macrocode}
%    \end{macro}
%
% \subsection{Bit retrieval}\label{src:bitret}
%
% \subsubsection{\cs{bitsetGet}}
%
%    \begin{macro}{\bitsetGet}
%    \begin{macrocode}
\def\bitsetGet#1#2{%
  \number
  \expandafter\expandafter\expandafter\BitSet@Get
  \intcalcNum{#2}!{#1}%
}
%    \end{macrocode}
%    \end{macro}
%    \begin{macro}{\BitSet@Get}
%    |#1|: plain index\\
%    |#2|: BitSet
%    \begin{macrocode}
\def\BitSet@Get#1!#2{%
  \ifnum#1<0 %
    \BitSet@AfterFi{%
      0 \BitSetError:NegativeIndex%
    }%
  \else
    \BitSet@IfUndefined{#2}{0}{%
      \expandafter\expandafter\expandafter\BitSet@@Get
      \csname BS@#2\expandafter\endcsname
      \expandafter!\expandafter=%
      \expandafter{\expandafter}\expandafter{\expandafter}%
      \romannumeral\intcalcNum{#1}000!%
    }%
    \expandafter\BitSet@Space
  \BitSet@Fi
}
%    \end{macrocode}
%    \end{macro}
%    \begin{macro}{\BitSet@@Get}
%    \begin{macrocode}
\def\BitSet@@Get#1#2=#3#4#5{%
  #3#4%
  \ifx#5!%
    \ifx#1!%
      0%
    \else
      #1%
    \fi
  \else
    \ifx#1!%
      0%
      \BitSet@AfterFiFi\BitSet@Cleanup
    \else
      \BitSet@@Get#2=%
    \fi
  \BitSet@Fi
}
%    \end{macrocode}
%    \end{macro}
%
% \subsubsection{\cs{bitsetNextClearBit}, \cs{bitsetNextSetBit}}
%
%    \begin{macro}{\bitsetNextClearBit}
%    \begin{macrocode}
\def\bitsetNextClearBit#1#2{%
  \number
  \expandafter\expandafter\expandafter\BitSet@NextClearBit
  \intcalcNum{#2}!{#1} %
}
%    \end{macrocode}
%    \end{macro}
%    \begin{macro}{\BitSet@NextClearBit}
%    |#1|: Index\\
%    |#2|: BitSet
%    \begin{macrocode}
\def\BitSet@NextClearBit#1!#2{%
  \ifnum#1<0 %
    \BitSet@NextClearBit0!{#2}%
    \BitSet@AfterFi{%
      \expandafter\BitSet@Space
      \expandafter\BitSetError:NegativeIndex\romannumeral0%
    }%
  \else
    \bitsetIsEmpty{#2}{#1}{%
      \expandafter\BitSet@Skip
      \number#1\expandafter\expandafter\expandafter!%
      \csname BS@#2\endcsname!!!!!!!!!=%
      {\BitSet@@NextClearBit#1!}%
    }%
  \BitSet@Fi
}
%    \end{macrocode}
%    \end{macro}
%    \begin{macro}{\BitSet@@NextClearBit}
%    |#1|: index for next bit in |#2|\\
%    |#2|: next bit
%    \begin{macrocode}
\def\BitSet@@NextClearBit#1!#2{%
  \ifx#2!%
    #1%
  \else
    \ifx#20%
      #1%
      \BitSet@AfterFiFi\BitSet@Cleanup
    \else
      \BitSet@AfterFiFi{%
        \expandafter\expandafter\expandafter\BitSet@@NextClearBit
        \IntCalcInc#1!!%
      }%
    \fi
  \BitSet@Fi
}
%    \end{macrocode}
%    \end{macro}
%
%    \begin{macro}{\bitsetNextSetBit}
%    \begin{macrocode}
\def\bitsetNextSetBit#1#2{%
  \number
  \expandafter\expandafter\expandafter\BitSet@NextSetBit
  \intcalcNum{#2}!{#1} %
}
%    \end{macrocode}
%    \end{macro}
%    \begin{macro}{\BitSet@NextSetBit}
%    |#1|: Index\\
%    |#2|: BitSet
%    \begin{macrocode}
\def\BitSet@NextSetBit#1!#2{%
  \ifnum#1<0 %
    \BitSet@NextSetBit0!{#2}%
    \BitSet@AfterFi{%
      \expandafter\BitSet@Space
      \expandafter\BitSetError:NegativeIndex\romannumeral0%
    }%
  \else
    \bitsetIsEmpty{#2}{-1}{%
      \expandafter\BitSet@Skip
      \number#1\expandafter\expandafter\expandafter!%
      \csname BS@#2\endcsname!!!!!!!!!=%
      {\BitSet@@NextSetBit#1!}%
    }%
  \BitSet@Fi
}
%    \end{macrocode}
%    \end{macro}
%    \begin{macro}{\BitSet@@NextSetBit}
%    |#1|: index for next bit in |#2|\\
%    |#2|: next bit
%    \begin{macrocode}
\def\BitSet@@NextSetBit#1!#2{%
  \ifx#2!%
    -1%
  \else
    \ifx#21%
      #1%
      \BitSet@AfterFiFi\BitSet@Cleanup
    \else
      \BitSet@AfterFiFi{%
        \expandafter\expandafter\expandafter\BitSet@@NextSetBit
        \IntCalcInc#1!!%
      }%
    \fi
  \BitSet@Fi
}
%    \end{macrocode}
%    \end{macro}
%    \begin{macro}{\BitSet@Cleanup}
%    \begin{macrocode}
\def\BitSet@Cleanup#1!{}
%    \end{macrocode}
%    \end{macro}
%
%    \begin{macro}{\BitSet@Skip}
%    |#1|: number of bits to skip\\
%    |#2|: bits\\
%    |#3|: continuation code
%    \begin{macrocode}
\def\BitSet@Skip#1!#2{%
  \ifx#2!%
    \BitSet@AfterFi{%
      \BitSet@SkipContinue%
    }%
  \else
    \ifcase#1 %
      \BitSet@AfterFiFi{%
        \BitSet@SkipContinue#2%
      }%
    \or
      \BitSet@AfterFiFi\BitSet@SkipContinue
    \or
      \BitSet@AfterFiFi{%
        \expandafter\BitSet@SkipContinue\BitSet@Gobble
      }%
    \else
      \ifnum#1>8 %
        \BitSet@AfterFiFiFi{%
          \expandafter\BitSet@Skip
          \number\IntCalcSub#1!8!\expandafter!%
          \BitSet@GobbleSeven
        }%
      \else
        \BitSet@AfterFiFiFi{%
          \expandafter\expandafter\expandafter\BitSet@Skip
          \IntCalcDec#1!!%
        }%
      \fi
    \fi
  \BitSet@Fi
}
%    \end{macrocode}
%    \end{macro}
%    \begin{macro}{\BitSet@SkipContinue}
%    |#1|: remaining bits\\
%    |#2|: continuation code
%    \begin{macrocode}
\def\BitSet@SkipContinue#1!#2=#3{%
  #3#1!%
}
%    \end{macrocode}
%    \end{macro}
%    \begin{macro}{\BitSet@GobbleSeven}
%    \begin{macrocode}
\def\BitSet@GobbleSeven#1#2#3#4#5#6#7{}
%    \end{macrocode}
%    \end{macro}
%
% \subsubsection{\cs{bitsetGetSetBitList}}
%
%    \begin{macro}{\bitsetGetSetBitList}
%    It's just a wrapper for \cs{bitsetNextSetBit}.
%    \begin{macrocode}
\def\bitsetGetSetBitList#1{%
  \romannumeral0%
  \bitsetIsEmpty{#1}{ }{%
    \expandafter\BitSet@GetSetBitList
    \number\BitSet@NextSetBit0!{#1}!{#1}{}!%
  }%
}
%    \end{macrocode}
%    \end{macro}
%    \begin{macro}{\BitSet@GetSetBitList}
%    |#1|: found index\\
%    |#2|: BitSet\\
%    |#3|: comma
%    |#4|: result
%    \begin{macrocode}
\def\BitSet@GetSetBitList#1!#2#3#4!{%
  \ifnum#1<0 %
    \BitSet@AfterFi{ #4}%
  \else
    \BitSet@AfterFi{%
      \expandafter\BitSet@GetSetBitList\number
      \expandafter\expandafter\expandafter\BitSet@NextSetBit
      \IntCalcInc#1!!{#2}!{#2},#4#3#1!%
    }%
  \BitSet@Fi
}
%    \end{macrocode}
%    \end{macro}
%
% \subsection{Bit set properties}\label{src:prop}
%
%    \begin{macro}{\bitsetSize}
%    \begin{macrocode}
\def\bitsetSize#1{%
  \number
  \BitSet@IfUndefined{#1}{0 }{%
    \expandafter\expandafter\expandafter\BitSet@Size
    \expandafter\expandafter\expandafter1%
    \expandafter\expandafter\expandafter!%
    \csname BS@#1\endcsname!0!%
  }%
}
%    \end{macrocode}
%    \end{macro}
%    \begin{macro}{\BitSet@Size}
%    |#1|: counter\\
%    |#2#3|: bits\\
%    |#4|: result
%    \begin{macrocode}
\def\BitSet@Size#1!#2#3!#4!{%
  \ifx#21%
    \ifx\\#3\\%
      \BitSet@AfterFiFi{#1 }%
    \else
      \BitSet@AfterFiFi{%
        \expandafter\expandafter\expandafter\BitSet@Size
        \IntCalcInc#1!!#3!#1!%
      }%
    \fi
  \else
    \ifx\\#3\\%
      \BitSet@AfterFiFi{#4 }%
    \else
      \BitSet@AfterFiFi{%
        \expandafter\expandafter\expandafter\BitSet@Size
        \IntCalcInc#1!!#3!#4!%
      }%
    \fi
  \fi
  \BitSet@Fi
}
%    \end{macrocode}
%    \end{macro}
%
%    \begin{macro}{\bitsetCardinality}
%    \begin{macrocode}
\def\bitsetCardinality#1{%
  \number
  \BitSet@IfUndefined{#1}{0 }{%
    \expandafter\expandafter\expandafter\BitSet@Cardinality
    \expandafter\expandafter\expandafter0%
    \expandafter\expandafter\expandafter!%
    \csname BS@#1\endcsname!%
  }%
}
%    \end{macrocode}
%    \end{macro}
%    \begin{macro}{\BitSet@Cardinality}
%    |#1|: result\\
%    |#2#3|: bits
%    \begin{macrocode}
\def\BitSet@Cardinality#1!#2#3!{%
  \ifx#21%
    \ifx\\#3\\%
      \BitSet@AfterFiFi{\IntCalcInc#1! }%
    \else
      \BitSet@AfterFiFi{%
        \expandafter\expandafter\expandafter\BitSet@Cardinality
        \IntCalcInc#1!!#3!%
      }%
    \fi
  \else
    \ifx\\#3\\%
      \BitSet@AfterFiFi{#1 }%
    \else
      \BitSet@AfterFiFi{%
        \BitSet@Cardinality#1!#3!%
      }%
    \fi
  \fi
  \BitSet@Fi
}
%    \end{macrocode}
%    \end{macro}
%
% \subsection{Queries}\label{src:query}
%
%    \begin{macro}{\bitsetIsDefined}
%    \begin{macrocode}
\def\bitsetIsDefined#1{%
  \BitSet@IfUndefined{#1}%
  \BitSet@SecondOfTwo
  \BitSet@FirstOfTwo
}
%    \end{macrocode}
%    \end{macro}
%    \begin{macro}{\bitsetIsEmpty}
%    \begin{macrocode}
\def\bitsetIsEmpty#1{%
  \BitSet@IfUndefined{#1}\BitSet@FirstOfTwo{%
    \expandafter\ifx\csname BS@#1\endcsname\BitSet@Zero
      \expandafter\BitSet@FirstOfTwo
    \else
      \expandafter\BitSet@SecondOfTwo
    \fi
  }%
}
%    \end{macrocode}
%    \end{macro}
%    \begin{macro}{\BitSet@Zero}
%    \begin{macrocode}
\def\BitSet@Zero{0}
%    \end{macrocode}
%    \end{macro}
%
%    \begin{macro}{\bitsetQuery}
%    \begin{macrocode}
\def\bitsetQuery#1#2{%
  \ifnum\bitsetGet{#1}{#2}=1 %
    \expandafter\BitSet@FirstOfTwo
  \else
    \expandafter\BitSet@SecondOfTwo
  \fi
}
%    \end{macrocode}
%    \end{macro}
%
%    \begin{macro}{\bitsetEquals}
%    \begin{macrocode}
\def\bitsetEquals#1#2{%
  \BitSet@IfUndefined{#1}{%
    \BitSet@IfUndefined{#2}\BitSet@FirstOfTwo\BitSet@SecondOfTwo
  }{%
    \BitSet@IfUndefined{#2}\BitSet@SecondOfTwo{%
      \expandafter\ifx\csname BS@#1\expandafter\endcsname
                      \csname BS@#2\endcsname
        \expandafter\BitSet@FirstOfTwo
      \else
        \expandafter\BitSet@SecondOfTwo
      \fi
    }%
  }%
}
%    \end{macrocode}
%    \end{macro}
%
%    \begin{macro}{\bitsetIntersects}
%    \begin{macrocode}
\def\bitsetIntersects#1#2{%
  \bitsetIsEmpty{#1}\BitSet@SecondOfTwo{%
    \bitsetIsEmpty{#2}\BitSet@SecondOfTwo{%
      \expandafter\expandafter\expandafter\BitSet@Intersects
      \csname BS@#1\expandafter\expandafter\expandafter\endcsname
      \expandafter\expandafter\expandafter!%
      \csname BS@#2\endcsname!%
    }%
  }%
}
%    \end{macrocode}
%    \end{macro}
%    \begin{macro}{\BitSet@Intersects}
%    \begin{macrocode}
\def\BitSet@Intersects#1#2!#3#4!{%
  \ifnum#1#3=11 %
    \BitSet@AfterFi\BitSet@FirstOfTwo
  \else
    \ifx\\#2\\%
      \BitSet@AfterFiFi\BitSet@SecondOfTwo
    \else
      \ifx\\#4\\%
        \BitSet@AfterFiFiFi\BitSet@SecondOfTwo
      \else
        \BitSet@AfterFiFiFi{%
          \BitSet@Intersects#2!#4!%
        }%
      \fi
    \fi
  \BitSet@Fi
}
%    \end{macrocode}
%    \end{macro}
%
%    \begin{macrocode}
\BitSet@AtEnd%
%</package>
%    \end{macrocode}
%
% \section{Test}
%
% \subsection{Catcode checks for loading}
%
%    \begin{macrocode}
%<*test1>
%    \end{macrocode}
%    \begin{macrocode}
\catcode`\{=1 %
\catcode`\}=2 %
\catcode`\#=6 %
\catcode`\@=11 %
\expandafter\ifx\csname count@\endcsname\relax
  \countdef\count@=255 %
\fi
\expandafter\ifx\csname @gobble\endcsname\relax
  \long\def\@gobble#1{}%
\fi
\expandafter\ifx\csname @firstofone\endcsname\relax
  \long\def\@firstofone#1{#1}%
\fi
\expandafter\ifx\csname loop\endcsname\relax
  \expandafter\@firstofone
\else
  \expandafter\@gobble
\fi
{%
  \def\loop#1\repeat{%
    \def\body{#1}%
    \iterate
  }%
  \def\iterate{%
    \body
      \let\next\iterate
    \else
      \let\next\relax
    \fi
    \next
  }%
  \let\repeat=\fi
}%
\def\RestoreCatcodes{}
\count@=0 %
\loop
  \edef\RestoreCatcodes{%
    \RestoreCatcodes
    \catcode\the\count@=\the\catcode\count@\relax
  }%
\ifnum\count@<255 %
  \advance\count@ 1 %
\repeat

\def\RangeCatcodeInvalid#1#2{%
  \count@=#1\relax
  \loop
    \catcode\count@=15 %
  \ifnum\count@<#2\relax
    \advance\count@ 1 %
  \repeat
}
\def\RangeCatcodeCheck#1#2#3{%
  \count@=#1\relax
  \loop
    \ifnum#3=\catcode\count@
    \else
      \errmessage{%
        Character \the\count@\space
        with wrong catcode \the\catcode\count@\space
        instead of \number#3%
      }%
    \fi
  \ifnum\count@<#2\relax
    \advance\count@ 1 %
  \repeat
}
\def\space{ }
\expandafter\ifx\csname LoadCommand\endcsname\relax
  \def\LoadCommand{\input bitset.sty\relax}%
\fi
\def\Test{%
  \RangeCatcodeInvalid{0}{47}%
  \RangeCatcodeInvalid{58}{64}%
  \RangeCatcodeInvalid{91}{96}%
  \RangeCatcodeInvalid{123}{255}%
  \catcode`\@=12 %
  \catcode`\\=0 %
  \catcode`\%=14 %
  \LoadCommand
  \RangeCatcodeCheck{0}{36}{15}%
  \RangeCatcodeCheck{37}{37}{14}%
  \RangeCatcodeCheck{38}{47}{15}%
  \RangeCatcodeCheck{48}{57}{12}%
  \RangeCatcodeCheck{58}{63}{15}%
  \RangeCatcodeCheck{64}{64}{12}%
  \RangeCatcodeCheck{65}{90}{11}%
  \RangeCatcodeCheck{91}{91}{15}%
  \RangeCatcodeCheck{92}{92}{0}%
  \RangeCatcodeCheck{93}{96}{15}%
  \RangeCatcodeCheck{97}{122}{11}%
  \RangeCatcodeCheck{123}{255}{15}%
  \RestoreCatcodes
}
\Test
\csname @@end\endcsname
\end
%    \end{macrocode}
%    \begin{macrocode}
%</test1>
%    \end{macrocode}
%
% \subsection{Macro tests}
%
% \subsubsection{Preamble}
%
%    \begin{macrocode}
%<*test2>
\NeedsTeXFormat{LaTeX2e}
\nofiles
\documentclass{article}
\makeatletter
%<*noetex>
\let\SavedNumexpr\numexpr
\let\SavedIfcsname\ifcsname
\let\SavedCurrentgrouplevel\currentgrouplevel
\def\ETeXDisable{%
  \let\ifcsname\@undefined
  \let\numexpr\@undefined
  \let\currentgrouplevel\@undefined
}
\ETeXDisable
%</noetex>
\makeatletter
\chardef\BitSet@TestMode=1 %
\makeatother
\usepackage{bitset}[2011/01/30]
%<*noetex>
\def\ETeXEnable{%
  \let\numexpr\SavedNumexpr
  \let\ifcsname\SavedIfcsname
  \let\currentgrouplevel\SavedCurrentgrouplevel
}
\ETeXEnable
%</noetex>
\usepackage{qstest}
\IncludeTests{*}
\LogTests{log}{*}{*}
\makeatletter
%    \end{macrocode}
%
% \subsubsection{Time}
%
%    \begin{macrocode}
\begingroup\expandafter\expandafter\expandafter\endgroup
\expandafter\ifx\csname pdfresettimer\endcsname\relax
\else
  \newcount\SummaryTime
  \newcount\TestTime
  \SummaryTime=\z@
  \newcommand*{\PrintTime}[2]{%
    \typeout{%
      [Time #1: \strip@pt\dimexpr\number#2sp\relax\space s]%
    }%
  }%
  \newcommand*{\StartTime}[1]{%
    \renewcommand*{\TimeDescription}{#1}%
    \pdfresettimer
  }%
  \newcommand*{\TimeDescription}{}%
  \newcommand*{\StopTime}{%
    \TestTime=\pdfelapsedtime
    \global\advance\SummaryTime\TestTime
    \PrintTime\TimeDescription\TestTime
  }%
  \let\saved@qstest\qstest
  \let\saved@endqstest\endqstest
  \def\qstest#1#2{%
    \saved@qstest{#1}{#2}%
    \StartTime{#1}%
  }%
  \def\endqstest{%
    \StopTime
    \saved@endqstest
  }%
  \AtEndDocument{%
    \PrintTime{summary}\SummaryTime
  }%
\fi
%    \end{macrocode}
%
% \subsubsection{Detection of unwanted space}
%
%    \begin{macrocode}
\let\orig@qstest\qstest
\let\orig@endqstest\endqstest
\def\qstest#1#2{%
  \orig@qstest{#1}{#2}%
  \setbox0\hbox\bgroup\begingroup\ignorespaces
}
\def\endqstest{%
  \endgroup\egroup
  \Expect*{\the\wd0}{0.0pt}%
  \orig@endqstest
}
%    \end{macrocode}
%
% \subsubsection{Test macros}
%
%    \begin{macrocode}
\newcounter{Test}

\def\TestError#1#2{%
  \begingroup
    \setcounter{Test}{0}%
    \sbox0{%
      \def\@PackageError##1##2##3{%
        \stepcounter{Test}%
        \begingroup
          \let\MessageBreak\relax
%<*noetex>
          \ETeXEnable
%</noetex>
          \Expect{##1}{bitset}%
          \Expect*{##2}*{#1}%
        \endgroup
      }%
%<*noetex>
      \ETeXDisable
%</noetex>
      #2%
    }%
    \Expect*{\theTest}{1}%
    \Expect*{\the\wd0}{0.0pt}%
  \endgroup
}

\def\TestErrorNegativeIndex#1#2{%
  \TestError{Invalid negative index (#1)}{#2}%
}

\def\TestGetterUndefined#1{%
  \CheckUndef{dummy}%
  \expandafter\expandafter\expandafter\Expect
  \expandafter\expandafter\expandafter{#1{dummy}}{0}%
}

\def\ExpectBitSet#1#2{%
  \expandafter\expandafter\expandafter\Expect
  \expandafter\expandafter\expandafter
  {\csname BS@#1\endcsname}*{#2}%
}
\def\Check#1#2{%
  \ExpectBitSet{#1}{#2}%
}
\def\CheckUndef#1{%
  \begingroup
    \Expect*{%
      \expandafter
      \ifx\csname BS@#1\endcsname\relax true\else false\fi
    }{true}%
  \endgroup
}
\def\RevCheck#1#2{%
  \ExpectBitSet{#1}{\Reverse#2!!}%
}
\def\Set#1#2{%
  \expandafter\def\csname BS@#1\endcsname{#2}%
}
\def\RevSet#1#2{%
  \expandafter\edef\csname BS@#1\endcsname{%
    \Reverse#2!!%
  }%
}
\def\Reverse#1#2!#3!{%
  \ifx\\#2\\%
    #1#3%
    \expandafter\@gobble
  \else
    \expandafter\@firstofone
  \fi
  {\Reverse#2!#1#3!}%
}
%    \end{macrocode}
%
% \subsubsection{Test sets}
%
%    \begin{macrocode}
\begin{qstest}{Let}{Let}
  \CheckUndef{abc}%
  \CheckUndef{xyz}%
  \bitsetLet{xyz}{abc}%
  \CheckUndef{abc}%
  \Check{xyz}{0}%
  \Set{abc}{1}%
  \Check{abc}{1}%
  \Check{xyz}{0}%
  \bitsetLet{xyz}{abc}%
  \Check{abc}{1}%
  \Check{xyz}{1}%
  \Set{xyz}{11}%
  \Check{abc}{1}%
  \Check{xyz}{11}%
\end{qstest}

\begin{qstest}{Reset}{Reset}
  \bitsetReset{xyz}%
  \Check{xyz}{0}%
  \bitsetReset{abc}%
  \Check{abc}{0}%
  \Set{abc}{10101}%
  \bitsetReset{abc}%
  \Check{abc}{0}%
\end{qstest}

\begin{qstest}{Get/Query}{Get/Query}
  \expandafter\expandafter\expandafter\Expect
  \expandafter\expandafter\expandafter{%
    \bitsetGet{dummy}{0}%
  }{0}%
  \begingroup
    \expandafter\def\csname BitSetError:NegativeIndex\endcsname{}%
    \Set{abc}{1}%
    \Expect*{\bitsetQuery{abc}{-1}{true}{false}}{false}%
  \endgroup
  \def\Test#1#2#3{%
    \Set{abc}{#1}%
    \expandafter\expandafter\expandafter\Expect
    \expandafter\expandafter\expandafter{\bitsetGet{abc}{#2}}{#3}%
    \Expect*{\bitsetQuery{abc}{#2}{true}{false}}%
           *{\ifcase#3 false\or true\else error\fi}%
  }%
  \Test{1}{100}{0}%
  \Test{0}{0}{0}%
  \Test{1}{0}{1}%
  \Test{11}{1}{1}%
  \Test{111}{1}{1}%
  \Test{101}{1}{0}%
  \Test{101}{2}{1}%
  \Test{10100110011}{10}{1}%
\end{qstest}

\begin{qstest}{Size}{Size}
  \TestGetterUndefined\bitsetSize
  \def\Test#1#2{%
    \Set{abc}{#1}%
    \expandafter\expandafter\expandafter\Expect
    \expandafter\expandafter\expandafter{\bitsetSize{abc}}{#2}%
  }%
  \Test{0}{0}%
  \Test{1}{1}%
  \Test{00}{0}%
  \Test{0000000}{0}%
  \Test{10}{1}%
  \Test{01}{2}%
  \Test{11}{2}%
  \Test{010}{2}%
  \Test{011}{3}%
  \Test{100110011}{9}%
  \Test{0000011111000001111100000}{20}%
  \Test{000000000000000000000000011111111111111111111}{45}%
\end{qstest}

\begin{qstest}{Cardinality}{Cardinality}
  \TestGetterUndefined\bitsetCardinality
  \def\Test#1#2{%
    \Set{abc}{#1}%
    \expandafter\expandafter\expandafter\Expect
    \expandafter\expandafter\expandafter{%
      \bitsetCardinality{abc}%
    }{#2}%
  }%
  \Test{0}{0}%
  \Test{1}{1}%
  \Test{00}{0}%
  \Test{0000000}{0}%
  \Test{10}{1}%
  \Test{01}{1}%
  \Test{11}{2}%
  \Test{010}{1}%
  \Test{011}{2}%
  \Test{100110011}{5}%
  \Test{0000011111000001111100000}{10}%
  \Test{000000000000000000000000011111111111111111111}{20}%
\end{qstest}

\begin{qstest}{NextClearBit/NextSetBit}{NextClearBit/NextSetBit}
  \def\Test#1#2{%
    \expandafter\expandafter\expandafter\Expect
    \expandafter\expandafter\expandafter{%
      \TestOp{abc}{#1}%
    }{#2}%
  }%
  \def\Clear{\let\TestOp\bitsetNextClearBit}%
  \def\Set{\let\TestOp\bitsetNextSetBit}%
  \begingroup
    \catcode`\:=11 %
    \bitsetSetBin{abc}{1}%
    \Clear
    \Test{-1}{1\BitSetError:NegativeIndex}%
    \Set
    \Test{-1}{0\BitSetError:NegativeIndex}%
  \endgroup
  \let\BS@abc\@undefined
  \Clear
  \Test{0}{0}%
  \Test{1}{1}%
  \Test{2}{2}%
  \Test{100}{100}%
  \Set
  \Test{0}{-1}%
  \Test{1}{-1}%
  \Test{100}{-1}%
  \bitsetReset{abc}%
  \Clear
  \Test{0}{0}%
  \Test{1}{1}%
  \Test{2}{2}%
  \Test{100}{100}%
  \Set
  \Test{0}{-1}%
  \Test{1}{-1}%
  \Test{100}{-1}%
  \bitsetSetBin{abc}{1}%
  \Clear
  \Test{0}{1}%
  \Test{1}{1}%
  \Test{2}{2}%
  \Test{100}{100}%
  \Set
  \Test{0}{0}%
  \Test{1}{-1}%
  \Test{100}{-1}%
  \bitsetSetBin{abc}{111000111000111000111}%
  \Clear
  \Test{0}{3}%
  \Test{1}{3}%
  \Test{2}{3}%
  \Test{3}{3}%
  \Test{4}{4}%
  \Test{5}{5}%
  \Test{6}{9}%
  \Test{7}{9}%
  \Test{8}{9}%
  \Test{9}{9}%
  \Test{10}{10}%
  \Test{11}{11}%
  \Test{12}{15}%
  \Test{13}{15}%
  \Test{14}{15}%
  \Test{15}{15}%
  \Test{16}{16}%
  \Test{17}{17}%
  \Test{18}{21}%
  \Test{19}{21}%
  \Test{20}{21}%
  \Test{21}{21}%
  \Test{22}{22}%
  \Test{100}{100}%
  \Set
  \Test{0}{0}%
  \Test{1}{1}%
  \Test{2}{2}%
  \Test{3}{6}%
  \Test{4}{6}%
  \Test{5}{6}%
  \Test{6}{6}%
  \Test{7}{7}%
  \Test{8}{8}%
  \Test{9}{12}%
  \Test{10}{12}%
  \Test{11}{12}%
  \Test{12}{12}%
  \Test{13}{13}%
  \Test{14}{14}%
  \Test{15}{18}%
  \Test{16}{18}%
  \Test{17}{18}%
  \Test{18}{18}%
  \Test{19}{19}%
  \Test{20}{20}%
  \Test{21}{-1}%
  \Test{22}{-1}%
  \Test{100}{-1}%
  \bitsetSetBin{abc}{1111111}%
  \Clear
  \Test{6}{7}%
  \Test{7}{7}%
  \Test{8}{8}%
  \Test{100}{100}%
  \Set
  \Test{6}{6}%
  \Test{7}{-1}%
  \Test{8}{-1}%
  \Test{100}{-1}%
  \bitsetSetBin{abc}{11111111}%
  \Clear
  \Test{7}{8}%
  \Test{8}{8}%
  \Test{9}{9}%
  \Test{100}{100}%
  \Set
  \Test{7}{7}%
  \Test{8}{-1}%
  \Test{9}{-1}%
  \Test{100}{-1}%
  \bitsetSetBin{abc}{111111111}%
  \Clear
  \Test{8}{9}%
  \Test{9}{9}%
  \Test{10}{10}%
  \Test{100}{100}%
  \Set
  \Test{8}{8}%
  \Test{9}{-1}%
  \Test{10}{-1}%
  \Test{100}{-1}%
  \bitsetSetBin{abc}{1111111111}%
  \Clear
  \Test{9}{10}%
  \Test{10}{10}%
  \Test{11}{11}%
  \Test{100}{100}%
  \Set
  \Test{9}{9}%
  \Test{10}{-1}%
  \Test{11}{-1}%
  \Test{100}{-1}%
\end{qstest}

\begin{qstest}{GetSetBitList}{GetSetBitList}
  \let\BS@abc\@undefined
  \expandafter\expandafter\expandafter\Expect
  \expandafter\expandafter\expandafter{%
    \bitsetGetSetBitList{abc}%
  }{}%
  \def\Test#1#2{%
    \bitsetSetBin{abc}{#1}%
    \expandafter\expandafter\expandafter\Expect
    \expandafter\expandafter\expandafter{%
      \bitsetGetSetBitList{abc}%
    }{#2}%
  }%
  \Test{0}{}%
  \Test{1}{0}%
  \Test{10}{1}%
  \Test{11}{0,1}%
  \Test{10110100}{2,4,5,7}%
  \Test{101101001010011}{0,1,4,6,9,11,12,14}%
\end{qstest}

\begin{qstest}{GetDec}{GetDec}
  \TestGetterUndefined\bitsetGetDec
  \def\Test#1#2{%
    \RevSet{abc}{#1}%
%<*noetex>
    \begingroup\expandafter\expandafter\expandafter\endgroup
%</noetex>
    \expandafter\expandafter\expandafter\Expect
    \expandafter\expandafter\expandafter{%
      \bitsetGetDec{abc}%
    }{#2}%
  }%
  \Test{0}{0}%
  \Test{1}{1}%
  \Test{10}{2}%
  \Test{11}{3}%
  \Test{100}{4}%
  \Test{101}{5}%
  \Test{110}{6}%
  \Test{111}{7}%
  \Test{1000}{8}%
  \Test{000111}{7}%
  \Test{1111111111111111%
        111111111111111}{2147483647}%
  \Test{0001111111111111111%
        111111111111111}{2147483647}%
  \Test{1000000000000000%
        0000000000000000}{2147483648}%
  \Test{1000000000000000%
        00000000000000000}{4294967296}%
  \Test{0001000000000000000%
        00000000000000000}{4294967296}%
  \Test{1100000000000000%
        00000000000000011}{6442450947}%
\end{qstest}

\begin{qstest}{Clear}{Clear}
  \def\Test#1#2#3{%
    \RevSet{abc}{#1}%
    \bitsetClear{abc}{#2}%
    \Expect*{\BS@abc}*{\Reverse#3!!}%
  }%
  \bitsetClear{abc}{2}%
  \RevCheck{abc}{0}%
  \TestErrorNegativeIndex{-1}{\bitsetClear{abc}{-1}}%
  \RevCheck{abc}{0}%
  \Test{0}{0}{0}%
  \Test{1}{0}{0}%
  \Test{111}{1}{101}%
  \Test{111}{30}{111}%
  \Test{0000111}{5}{0000111}% 111 would also be ok
  \Test{10000111}{5}{10000111}%
  \Test{1001001}{3}{1000001}%
\end{qstest}

\begin{qstest}{Set}{Set}
  \def\Test#1#2#3{%
    \RevSet{abc}{#1}%
    \bitsetSet{abc}{#2}%
    \Expect*{\BS@abc}*{\Reverse#3!!}%
  }%
  \bitsetSet{abc}{2}%
  \RevCheck{abc}{100}%
  \TestErrorNegativeIndex{-1}{\bitsetSet{abc}{-1}}%
  \RevCheck{abc}{100}%
  \Test{0}{0}{1}%
  \Test{1}{0}{1}%
  \Test{100}{1}{110}%
  \Test{111}{1}{111}%
  \Test{11}{1}{11}%
  \Test{11}{2}{111}%
  \Test{11}{3}{1011}%
  \Test{111}{10}{10000000111}%
  \Test{0000111}{5}{0100111}% 100111 would also be ok
  \Test{10000111}{5}{10100111}%
  \Test{1000001}{3}{1001001}%
  \Test{1001001}{3}{1001001}%
\end{qstest}

\begin{qstest}{Flip}{Flip}
  \def\Test#1#2#3{%
    \RevSet{abc}{#1}%
    \bitsetFlip{abc}{#2}%
    \Expect*{\BS@abc}*{\Reverse#3!!}%
  }%
  \bitsetFlip{abc}{2}%
  \RevCheck{abc}{100}%
  \TestErrorNegativeIndex{-1}{\bitsetFlip{abc}{-1}}%
  \RevCheck{abc}{100}%
  \Test{0}{0}{1}%
  \Test{1}{0}{0}%
  \Test{0}{2}{100}%
  \Test{100}{1}{110}%
  \Test{111}{1}{101}%
  \Test{11}{1}{1}%
  \Test{11}{2}{111}%
  \Test{11}{3}{1011}%
  \Test{111}{10}{10000000111}%
  \Test{0000111}{5}{0100111}% 100111 would also be ok
  \Test{10000111}{5}{10100111}%
  \Test{1000001}{3}{1001001}%
  \Test{1001001}{3}{1000001}%
  \Test{11111}{2}{11011}%
\end{qstest}

\begin{qstest}{SetValue}{SetValue}
  \def\Test#1#2{%
    \TestError{Invalid bit value (#2) not in range 0..1}{%
      \bitsetSetValue{abc}{#1}{#2}%
    }%
  }%
  \Test{0}{-1}%
  \Test{0}{2}%
  \Test{0}{10}%
  \def\Test#1#2#3{%
    \let\BS@abc\@undefined
    \bitsetSetValue{abc}{#1}{#2}%
    \bitsetSetBin{result}{#3}%
    \Expect*{\BS@abc}*{\BS@result}%
  }%
  \Test{0}{0}{0}%
  \Test{0}{1}{1}%
  \Test{1}{0}{0}%
  \Test{1}{1}{10}%
  \def\Test#1#2#3#4{%
    \bitsetSetBin{abc}{#1}%
    \bitsetSetBin{result}{#4}%
    \bitsetSetValue{abc}{#2}{#3}%
    \Expect*{\BS@abc}*{\BS@result}%
  }%
  \Test{0}{0}{0}{0}%
  \Test{0}{0}{0}{0}%
  \Test{0}{0}{1}{1}%
  \Test{0}{1}{0}{0}%
  \Test{0}{1}{1}{10}%
  \Test{1010}{2}{1}{1110}%
  \Test{1010}{4}{1}{11010}%
  \Test{1010}{6}{1}{1001010}%
  \Test{1010}{1}{0}{1000}%
  \Test{1010}{2}{0}{1010}%
  \Test{1010}{3}{0}{10}%
  \Test{1010}{4}{0}{1010}%
  \Test{1010}{6}{0}{1010}%
  \Test{1010}{2}{\csname iffalse\endcsname 0\else 1\fi}{1110}%
  \Test{1010}{1}{\csname iffalse\endcsname 1\else 0\fi}{1000}%
\end{qstest}

\begin{qstest}{IsDefined}{IsDefined}
  \let\BS@abc\@undefined
  \Expect*{\bitsetIsDefined{abc}{true}{false}}{false}%
  \bitsetReset{abc}%
  \Expect*{\bitsetIsDefined{abc}{true}{false}}{true}%
\end{qstest}

\begin{qstest}{IsEmpty}{IsEmpty}
  \let\BS@abc\@undefined
  \Expect*{\bitsetIsEmpty{abc}{true}{false}}{true}%
  \bitsetReset{abc}%
  \Expect*{\bitsetIsEmpty{abc}{true}{false}}{true}%
  \bitsetSet{abc}{1}%
  \Expect*{\bitsetIsEmpty{abc}{true}{false}}{false}%
\end{qstest}

\begin{qstest}{Equals}{Equals}
  \def\Test#1#2#3{%
    \Expect*{\bitsetEquals{#1}{#2}{true}{false}}{#3}%
  }%
  \let\BS@abc\@undefined
  \Test{abc}{abc}{true}%
  \Test{abc}{foo}{true}%
  \Test{foo}{abc}{true}%
  \bitsetReset{abc}%
  \Test{abc}{abc}{true}%
  \Test{abc}{foo}{false}%
  \Test{foo}{abc}{false}%
  \bitsetReset{foo}%
  \Test{abc}{foo}{true}%
  \Test{foo}{abc}{true}%
  \bitsetSet{abc}{4}%
  \Test{abc}{foo}{false}%
  \Test{foo}{abc}{false}%
  \bitsetFlip{foo}{4}%
  \Test{abc}{foo}{true}%
  \Test{foo}{abc}{true}%
\end{qstest}

\begin{qstest}{Intersects}{Intersects}
  \def\Test#1{%
    \Expect*{\bitsetIntersects{abc}{foo}{true}{false}}{#1}%
  }%
  \let\BS@abc\@undefined
  \let\BS@foo\@undefined
  \Test{false}%
  \Set{abc}{0}%
  \Test{false}%
  \Set{foo}{0}%
  \Test{false}%
  \let\BS@abc\@undefined
  \Test{false}%
  \Set{foo}{1}%
  \Test{false}%
  \Set{abc}{0}%
  \Test{false}%
  \Set{abc}{1}%
  \Test{true}%
  \let\BS@foo\@undefined
  \Test{false}%
  \Set{foo}{0}%
  \Test{false}%
  \def\Test#1#2#3{%
    \bitsetSetBin{abc}{#1}%
    \bitsetSetBin{foo}{#2}%
    \Expect*{\bitsetIntersects{abc}{foo}{true}{false}}{#3}%
  }%
  \Test{1010}{0101}{false}%
  \Test{0}{10}{false}%
  \Test{1}{11}{true}%
  \Test{11}{1}{true}%
  \Test{10}{1}{false}%
\end{qstest}

\begin{qstest}{And/AndNot/Or/Xor}{And/AndNot/Or/Xor}
  \def\@Test#1#2#3#4#5{%
    \begingroup
      #5%
      \begingroup
        \let\BS@foo\@undefined
        \csname bitset#1\endcsname{abc}{foo}%
        \CheckUndef{foo}%
        \Check{abc}{#2}%
      \endgroup
      \begingroup
        \bitsetReset{foo}%
        \csname bitset#1\endcsname{abc}{foo}%
        \Check{foo}{0}%
        \Check{abc}{#3}%
      \endgroup
      \begingroup
        \def\BS@foo{0101}%
        \csname bitset#1\endcsname{abc}{foo}%
        \Check{foo}{0101}%
        \Check{abc}{#4}%
      \endgroup
    \endgroup
  }%
  \def\Test#1{%
    \def\Op{#1}%
    \Test@
  }%
  \def\Test@#1#2#3#4#5#6#7#8#9{%
    \@Test\Op{#1}{#2}{#3}{%
      \let\BS@abc\@undefined
    }%
    \@Test\Op{#4}{#5}{#6}{%
      \bitsetReset{abc}%
    }%
    \@Test\Op{#7}{#8}{#9}{%
      \def\BS@abc{1001}%
    }%
  }%
  \Test{And}%
       {0}{0}{0}%
       {0}{0}{0}%
       {0}{0}{0001}%
  \Test{AndNot}%
       {0}{0}{0}%
       {0}{0}{0}%
       {1001}{1001}{1}%
  \Test{Or}%
       {0}{0}{0101}%
       {0}{0}{0101}%
       {1001}{1001}{1101}%
  \Test{Xor}%
       {0}{0}{0101}%
       {0}{0}{0101}%
       {1001}{1001}{11}%
  \def\Test#1#2#3{%
    \bitsetSetBin{abc}{#1}%
    \bitsetSetBin{foo}{#2}%
    \csname bitset\Op\endcsname{abc}{foo}%
    \RevCheck{foo}{#2}%
    \RevCheck{abc}{#3}%
  }%
  \def\Op{And}%
  \Test{1}{111}{1}%
  \Test{111}{1}{1}%
  \Test{10}{111}{10}%
  \Test{111}{10}{10}%
  \Test{111}{1000}{0}%
  \Test{1000}{111}{0}%
  \def\Op{AndNot}%
  \Test{1010}{11}{1000}%
  \Test{100}{100}{0}%
  \Test{111}{1111}{0}%
  \Test{100}{111}{0}%
  \def\Op{Or}%
  \Test{0}{0}{0}%
  \Test{1}{0}{1}%
  \Test{0}{1}{1}%
  \Test{1}{1}{1}%
  \Test{1000}{10}{1010}%
  \Test{10}{1000}{1010}%
  \def\Op{Xor}%
  \Test{0}{0}{0}%
  \Test{1}{0}{1}%
  \Test{0}{1}{1}%
  \Test{1}{1}{0}%
  \Test{1000}{10}{1010}%
  \Test{10}{1000}{1010}%
  \Test   {110011001100}%
       {111000111000111}%
       {111110100001011}%
  \Test{111000111000111}%
          {110011001100}%
       {111110100001011}%
\end{qstest}

\begin{qstest}{GetUndef}{GetUndef, GetBin, GetOct, GetHex}
  \def\TestUndef#1#2{%
    \let\BS@abc\@undefined
    \expandafter\expandafter\expandafter\Expect
    \expandafter\expandafter\expandafter{%
      \x{abc}{#1}%
    }{#2}%
  }%
  \let\x\bitsetGetBin
  \TestUndef{-1}{0}%
  \TestUndef{0}{0}%
  \TestUndef{1}{0}%
  \TestUndef{2}{00}%
  \TestUndef{8}{00000000}%
  \let\x\bitsetGetOct
  \TestUndef{-1}{0}%
  \TestUndef{0}{0}%
  \TestUndef{1}{0}%
  \TestUndef{2}{0}%
  \TestUndef{3}{0}%
  \TestUndef{4}{00}%
  \TestUndef{5}{00}%
  \TestUndef{6}{00}%
  \TestUndef{7}{000}%
  \TestUndef{8}{000}%
  \TestUndef{9}{000}%
  \TestUndef{10}{0000}%
  \let\x\bitsetGetHex
  \TestUndef{-1}{0}%
  \TestUndef{0}{0}%
  \TestUndef{1}{0}%
  \TestUndef{2}{0}%
  \TestUndef{3}{0}%
  \TestUndef{4}{0}%
  \TestUndef{5}{00}%
  \TestUndef{6}{00}%
  \TestUndef{7}{00}%
  \TestUndef{8}{00}%
  \TestUndef{9}{000}%
  \TestUndef{10}{000}%
  \TestUndef{12}{000}%
  \TestUndef{13}{0000}%
  \TestUndef{16}{0000}%
  \TestUndef{17}{00000}%
\end{qstest}

\begin{qstest}{SetBin}{SetBin}
  \def\Test#1#2{%
    \let\BS@abc\@undefined
    \bitsetSetBin{abc}{#1}%
    \expandafter\Expect\expandafter{\BS@abc}{#2}%
  }%
  \Test{}{0}%
  \Test{0}{0}%
  \Test{1}{1}%
  \Test{10}{01}%
  \Test{11}{11}%
  \Test{010}{01}%
  \Test{011}{11}%
  \Test{0010}{01}%
  \Test{1010}{0101}%
\end{qstest}

\begin{qstest}{SetOct}{SetOct}
  \def\Test#1#2{%
    \bitsetSetOct{abc}{#1}%
    \expandafter\Expect\expandafter{\BS@abc}{#2}%
  }%
  \Test{}{0}%
  \Test{0}{0}%
  \Test{000}{0}%
  \Test{1}{1}%
  \Test{001}{1}%
  \Test{010}{0001}%
  \Test{020}{00001}%
  \Test{42}{010001}%
  \Test{377}{11111111}%
  \Test{0377}{11111111}%
  \Test{76543210}{000100010110001101011111}%
  \Test{ 0 7 0 7 1 }{100111000111}%
\end{qstest}

\begin{qstest}{SetHex}{SetHex}
  \def\Test#1#2{%
    \bitsetSetHex{abc}{#1}%
    \expandafter\Expect\expandafter{\BS@abc}{#2}%
  }%
  \Test{}{0}%
  \Test{0}{0}%
  \Test{000}{0}%
  \Test{1}{1}%
  \Test{001}{1}%
  \Test{010}{00001}%
  \Test{020}{000001}%
  \Test{42}{0100001}%
  \Test{3F}{111111}%
  \Test{03F}{111111}%
  \Test{43210}{0000100001001100001}%
  \Test{98765}{10100110111000011001}%
  \Test{FEDCBA}{010111010011101101111111}%
  \Test{ 0 F 0 F 1 }{1000111100001111}%
\end{qstest}

\begin{qstest}{SetDec}{SetDec}
  \def\Test#1#2{%
    \bitsetSetDec{abc}{#1}%
    \expandafter\Expect\expandafter{\BS@abc}{#2}%
  }%
  \Test{}{0}%
  \Test{0}{0}%
  \Test{000}{0}%
  \Test{1}{1}%
  \Test{7}{111}%
  \Test{8}{0001}%
  \Test{001}{1}%
  \Test{010}{0101}%
  \Test{020}{00101}%
  \Test{53}{101011}%
  \Test{255}{11111111}%
  \Test{256}{000000001}%
  \Test{999999999}{111111111001001101011001110111}%
  \Test{1000000000}{000000000101001101011001110111}%
  \Test{4210987654}{01100001010010010111111101011111}%
  \Test{2147483647}{1111111111111111111111111111111}%
  \Test{2147483648}{00000000000000000000000000000001}%
\end{qstest}

\begin{qstest}{GetBin}{GetBin}
  \def\TestUndef#1#2{%
    \let\BS@abc\@undefined
    \expandafter\expandafter\expandafter\Expect
    \expandafter\expandafter\expandafter{%
      \bitsetGetBin{abc}{#1}%
    }{#2}%
  }%
  \TestUndef{-1}{0}%
  \TestUndef{0}{0}%
  \TestUndef{1}{0}%
  \TestUndef{2}{00}%
  \TestUndef{8}{00000000}%
  \def\Test#1#2{%
    \bitsetSetBin{abc}{#2}%
    \expandafter\expandafter\expandafter\Expect
    \expandafter\expandafter\expandafter{%
      \bitsetGetBin{abc}{#1}%
    }{#2}%
  }%
  \Test{-1}{0}%
  \Test{0}{0}%
  \Test{1}{0}%
  \Test{1}{1}%
  \Test{2}{01}%
  \Test{2}{10}%
  \Test{3}{010}%
  \Test{2}{00}%
  \Test{2}{01}%
  \Test{8}{00101100}%
  \Test{2}{10101}%
  \Test{-100}{11011}%
\end{qstest}

\begin{qstest}{GetOct}{GetOct}
  \def\Test#1#2#3{%
    \edef\x{\zap@space#1 \@empty}%
    \edef\x{\noexpand\bitsetSetBin{abc}{\x}}%
    \x
    \expandafter\expandafter\expandafter\Expect
    \expandafter\expandafter\expandafter{%
      \bitsetGetOct{abc}{#2}%
    }{#3}%
  }%
  \Test{111 110 101 100 011 010 001 000}{0}{76543210}%
  \Test{000 111}{0}{7}%
  \Test{101 000}{-1}{50}%
  \Test{111}{-1}{7}%
  \Test{111}{0}{7}%
  \Test{111}{1}{7}%
  \Test{111}{3}{7}%
  \Test{111}{4}{07}%
  \Test{111}{6}{07}%
  \Test{111}{7}{007}%
  \Test{111 010}{6}{72}%
  \Test{111 010}{7}{072}%
  \Test{011 111}{0}{37}%
  \Test{011 111}{6}{37}%
  \Test{011 111}{7}{037}%
  \Test{001 111}{0}{17}%
  \Test{001 111}{6}{17}%
  \Test{001 111}{7}{017}%
\end{qstest}

\begin{qstest}{GetHex}{GetHex}
  \def\Test#1#2#3{%
    \bitsetSetBin{abc}{#1}%
    \expandafter\expandafter\expandafter\Expect
    \expandafter\expandafter\expandafter{%
      \bitsetGetHex{abc}{#2}%
    }{#3}%
  }%
  \Test{1111 1110 1101 1100 1011 1010 1001 1000}{0}{FEDCBA98}%
  \Test{0111 0110 0101 0100 0011 0010 0001 0000}{0}{76543210}%
  \Test{0000 1111}{0}{F}%
  \Test{0101 0000}{-1}{50}%
  \Test{1111}{-1}{F}%
  \Test{1111}{0}{F}%
  \Test{1111}{1}{F}%
  \Test{1111}{4}{F}%
  \Test{1111}{5}{0F}%
  \Test{1111}{8}{0F}%
  \Test{1111}{9}{00F}%
  \Test{1111 0010}{8}{F2}%
  \Test{1111 0010}{9}{0F2}%
  \Test{0111 1111}{0}{7F}%
  \Test{0111 1111}{8}{7F}%
  \Test{0111 1111}{9}{07F}%
  \Test{0011 1111}{0}{3F}%
  \Test{0011 1111}{8}{3F}%
  \Test{0011 1111}{9}{03F}%
  \Test{0001 1111}{0}{1F}%
  \Test{0001 1111}{8}{1F}%
  \Test{0001 1111}{9}{01F}%
\end{qstest}

\begin{qstest}{Range}{Range}
  \TestError{%
    Wrong index numbers in range [9..8]\MessageBreak% hash-ok
    for clear/set/flip on bit set `abc'.\MessageBreak
    The lower index exceeds the upper index.\MessageBreak
    Canceling the operation as error recovery%
  }{%
    \bitsetSetRange{abc}{9}{8}%
  }%
  \def\TestErrorNegInd#1#2#3#4#5#6{%
    \TestError{%
      Negative index in range [#2..#3]\MessageBreak % hash-ok
      for \string\bitset #1Range on bit set `abc'.\MessageBreak
      Using [#4..#5] as error recovery% hash-ok
    }{%
      \csname bitset#1Range\endcsname{abc}{#2}{#3}%
      \global\let\BS@global\BS@abc
    }%
    \Check{global}{#6}%
  }%
  \Set{abc}{111}%
  \TestErrorNegInd{Clear}{-1}{0}{0}{0}{111}%
  \TestErrorNegInd{Clear}{0}{-1}{0}{0}{111}%
  \TestErrorNegInd{Clear}{-2}{2}{0}{2}{001}%
  \bitsetReset{abc}%
  \TestErrorNegInd{Set}{-1}{0}{0}{0}{0}%
  \TestErrorNegInd{Set}{0}{-1}{0}{0}{0}%
  \TestErrorNegInd{Set}{-2}{2}{0}{2}{11}%
  \Set{abc}{101}%
  \TestErrorNegInd{Flip}{-1}{0}{0}{0}{101}%
  \TestErrorNegInd{Flip}{0}{-1}{0}{0}{101}%
  \TestErrorNegInd{Flip}{-2}{2}{0}{2}{011}%
  \def\Test#1#2#3#4{%
    \bitsetSetBin{abc}{#1}%
    \csname bitset\TestOp Range\endcsname{abc}{#2}{#3}%
    \Expect*{\bitsetGetBin{abc}{0}}{#4}%
  }%
  \def\TestOp{Clear}%
  \Test{0}{0}{1}{0}%
  \Test{1111}{1}{2}{1101}%
  \Test{1111}{1}{3}{1001}%
  \Test{1111111100000000}{12}{14}{1100111100000000}%
  \def\TestOp{Set}%
  \Test{0}{0}{1}{1}%
  \Test{1000}{1}{2}{1010}%
  \Test{0}{1}{2}{10}%
  \Test{1}{12}{15}{111000000000001}%
  \Test{1111}{1}{3}{1111}%
  \Test{1000000000000000}{12}{14}{1011000000000000}%
  \def\TestOp{Flip}%
  \Test{0}{0}{1}{1}%
  \Test{1}{0}{1}{0}%
  \Test{10101010}{1}{5}{10110100}%
  \def\Test#1#2#3#4#5{%
    \bitsetSetBin{abc}{#1}%
    \bitsetSetValueRange{abc}{#2}{#3}{#4}%
    \Expect*{\bitsetGetBin{abc}{0}}{#5}%
  }%
  \Test{0}{0}{1}{0}{0}%
  \Test{0}{0}{1}{1}{1}%
  \Test{1010}{1}{3}{0}{1000}%
  \Test{1010}{1}{3}{1}{1110}%
\end{qstest}

\begin{qstest}{ShiftLeft/ShiftRight}{ShiftLeft/ShiftRight}
  \def\@Test#1#2{%
    \let\BS@abc\@undefined
    \csname bitsetShift#1\endcsname{abc}{#2}%
    \Expect*{\BS@abc}{0}%
  }%
  \def\Test#1{%
    \@Test{Left}{#1}%
    \@Test{Right}{#1}%
  }%
  \Test{-16}%
  \Test{-1}%
  \Test{0}%
  \Test{1}%
  \Test{16}%
  \def\Test#1#2#3{%
    \bitsetSetBin{abc}{#1}%
    \bitsetSetBin{result}{#3}%
    \csname bitsetShift\Op\endcsname{abc}{#2}%
    \Expect*{\bitsetGetBin{abc}{0}}*{\bitsetGetBin{result}{0}}%
  }%
  \def\Op{Left}%
  \Test{0}{0}{0}%
  \Test{0}{1}{0}%
  \Test{0}{-1}{0}%
  \Test{1}{0}{1}%
  \Test{1}{1}{10}%
  \Test{1}{-1}{0}%
  \Test{10}{1}{100}%
  \Test{10}{-1}{1}%
  \Test{1}{32}{100000000000000000000000000000000}%
  \Test{1}{-100}{0}%
  \def\Op{Right}%
  \Test{0}{0}{0}%
  \Test{0}{1}{0}%
  \Test{0}{-1}{0}%
  \Test{1}{0}{1}%
  \Test{1}{1}{0}%
  \Test{1}{-1}{10}%
  \Test{10}{1}{1}%
  \Test{10}{-1}{100}%
  \Test{1}{-32}{100000000000000000000000000000000}%
  \Test{1}{100}{0}%
  \Test{110110110110110}{10}{11011}%
  \Test{110110110110110}{100}{0}%
  \Test{1}{100000}{0}%
\end{qstest}

\begin{qstest}{Profile: Set}{Profile: Set}
  \bitsetSet{abc}{4095}%
  \global\let\BS@global\BS@abc
\end{qstest}

\begin{qstest}{Profile: Get}{Profile: Get}
  \edef\x{\bitsetGet{global}{4095}}%
\end{qstest}

\begin{document}
\end{document}
%</test2>
%    \end{macrocode}
%
% \section{Installation}
%
% \subsection{Download}
%
% \paragraph{Package.} This package is available on
% CTAN\footnote{\url{ftp://ftp.ctan.org/tex-archive/}}:
% \begin{description}
% \item[\CTAN{macros/latex/contrib/oberdiek/bitset.dtx}] The source file.
% \item[\CTAN{macros/latex/contrib/oberdiek/bitset.pdf}] Documentation.
% \end{description}
%
%
% \paragraph{Bundle.} All the packages of the bundle `oberdiek'
% are also available in a TDS compliant ZIP archive. There
% the packages are already unpacked and the documentation files
% are generated. The files and directories obey the TDS standard.
% \begin{description}
% \item[\CTAN{install/macros/latex/contrib/oberdiek.tds.zip}]
% \end{description}
% \emph{TDS} refers to the standard ``A Directory Structure
% for \TeX\ Files'' (\CTAN{tds/tds.pdf}). Directories
% with \xfile{texmf} in their name are usually organized this way.
%
% \subsection{Bundle installation}
%
% \paragraph{Unpacking.} Unpack the \xfile{oberdiek.tds.zip} in the
% TDS tree (also known as \xfile{texmf} tree) of your choice.
% Example (linux):
% \begin{quote}
%   |unzip oberdiek.tds.zip -d ~/texmf|
% \end{quote}
%
% \paragraph{Script installation.}
% Check the directory \xfile{TDS:scripts/oberdiek/} for
% scripts that need further installation steps.
% Package \xpackage{attachfile2} comes with the Perl script
% \xfile{pdfatfi.pl} that should be installed in such a way
% that it can be called as \texttt{pdfatfi}.
% Example (linux):
% \begin{quote}
%   |chmod +x scripts/oberdiek/pdfatfi.pl|\\
%   |cp scripts/oberdiek/pdfatfi.pl /usr/local/bin/|
% \end{quote}
%
% \subsection{Package installation}
%
% \paragraph{Unpacking.} The \xfile{.dtx} file is a self-extracting
% \docstrip\ archive. The files are extracted by running the
% \xfile{.dtx} through \plainTeX:
% \begin{quote}
%   \verb|tex bitset.dtx|
% \end{quote}
%
% \paragraph{TDS.} Now the different files must be moved into
% the different directories in your installation TDS tree
% (also known as \xfile{texmf} tree):
% \begin{quote}
% \def\t{^^A
% \begin{tabular}{@{}>{\ttfamily}l@{ $\rightarrow$ }>{\ttfamily}l@{}}
%   bitset.sty & tex/generic/oberdiek/bitset.sty\\
%   bitset.pdf & doc/latex/oberdiek/bitset.pdf\\
%   test/bitset-test1.tex & doc/latex/oberdiek/test/bitset-test1.tex\\
%   test/bitset-test2.tex & doc/latex/oberdiek/test/bitset-test2.tex\\
%   test/bitset-test3.tex & doc/latex/oberdiek/test/bitset-test3.tex\\
%   bitset.dtx & source/latex/oberdiek/bitset.dtx\\
% \end{tabular}^^A
% }^^A
% \sbox0{\t}^^A
% \ifdim\wd0>\linewidth
%   \begingroup
%     \advance\linewidth by\leftmargin
%     \advance\linewidth by\rightmargin
%   \edef\x{\endgroup
%     \def\noexpand\lw{\the\linewidth}^^A
%   }\x
%   \def\lwbox{^^A
%     \leavevmode
%     \hbox to \linewidth{^^A
%       \kern-\leftmargin\relax
%       \hss
%       \usebox0
%       \hss
%       \kern-\rightmargin\relax
%     }^^A
%   }^^A
%   \ifdim\wd0>\lw
%     \sbox0{\small\t}^^A
%     \ifdim\wd0>\linewidth
%       \ifdim\wd0>\lw
%         \sbox0{\footnotesize\t}^^A
%         \ifdim\wd0>\linewidth
%           \ifdim\wd0>\lw
%             \sbox0{\scriptsize\t}^^A
%             \ifdim\wd0>\linewidth
%               \ifdim\wd0>\lw
%                 \sbox0{\tiny\t}^^A
%                 \ifdim\wd0>\linewidth
%                   \lwbox
%                 \else
%                   \usebox0
%                 \fi
%               \else
%                 \lwbox
%               \fi
%             \else
%               \usebox0
%             \fi
%           \else
%             \lwbox
%           \fi
%         \else
%           \usebox0
%         \fi
%       \else
%         \lwbox
%       \fi
%     \else
%       \usebox0
%     \fi
%   \else
%     \lwbox
%   \fi
% \else
%   \usebox0
% \fi
% \end{quote}
% If you have a \xfile{docstrip.cfg} that configures and enables \docstrip's
% TDS installing feature, then some files can already be in the right
% place, see the documentation of \docstrip.
%
% \subsection{Refresh file name databases}
%
% If your \TeX~distribution
% (\teTeX, \mikTeX, \dots) relies on file name databases, you must refresh
% these. For example, \teTeX\ users run \verb|texhash| or
% \verb|mktexlsr|.
%
% \subsection{Some details for the interested}
%
% \paragraph{Attached source.}
%
% The PDF documentation on CTAN also includes the
% \xfile{.dtx} source file. It can be extracted by
% AcrobatReader 6 or higher. Another option is \textsf{pdftk},
% e.g. unpack the file into the current directory:
% \begin{quote}
%   \verb|pdftk bitset.pdf unpack_files output .|
% \end{quote}
%
% \paragraph{Unpacking with \LaTeX.}
% The \xfile{.dtx} chooses its action depending on the format:
% \begin{description}
% \item[\plainTeX:] Run \docstrip\ and extract the files.
% \item[\LaTeX:] Generate the documentation.
% \end{description}
% If you insist on using \LaTeX\ for \docstrip\ (really,
% \docstrip\ does not need \LaTeX), then inform the autodetect routine
% about your intention:
% \begin{quote}
%   \verb|latex \let\install=y% \iffalse meta-comment
%
% File: bitset.dtx
% Version: 2011/01/30 v1.1
% Info: Handle bit-vector datatype
%
% Copyright (C) 2007, 2011 by
%    Heiko Oberdiek <heiko.oberdiek at googlemail.com>
%
% This work may be distributed and/or modified under the
% conditions of the LaTeX Project Public License, either
% version 1.3c of this license or (at your option) any later
% version. This version of this license is in
%    http://www.latex-project.org/lppl/lppl-1-3c.txt
% and the latest version of this license is in
%    http://www.latex-project.org/lppl.txt
% and version 1.3 or later is part of all distributions of
% LaTeX version 2005/12/01 or later.
%
% This work has the LPPL maintenance status "maintained".
%
% This Current Maintainer of this work is Heiko Oberdiek.
%
% The Base Interpreter refers to any `TeX-Format',
% because some files are installed in TDS:tex/generic//.
%
% This work consists of the main source file bitset.dtx
% and the derived files
%    bitset.sty, bitset.pdf, bitset.ins, bitset.drv, bitset-test1.tex,
%    bitset-test2.tex, bitset-test3.tex.
%
% Distribution:
%    CTAN:macros/latex/contrib/oberdiek/bitset.dtx
%    CTAN:macros/latex/contrib/oberdiek/bitset.pdf
%
% Unpacking:
%    (a) If bitset.ins is present:
%           tex bitset.ins
%    (b) Without bitset.ins:
%           tex bitset.dtx
%    (c) If you insist on using LaTeX
%           latex \let\install=y\input{bitset.dtx}
%        (quote the arguments according to the demands of your shell)
%
% Documentation:
%    (a) If bitset.drv is present:
%           latex bitset.drv
%    (b) Without bitset.drv:
%           latex bitset.dtx; ...
%    The class ltxdoc loads the configuration file ltxdoc.cfg
%    if available. Here you can specify further options, e.g.
%    use A4 as paper format:
%       \PassOptionsToClass{a4paper}{article}
%
%    Programm calls to get the documentation (example):
%       pdflatex bitset.dtx
%       makeindex -s gind.ist bitset.idx
%       pdflatex bitset.dtx
%       makeindex -s gind.ist bitset.idx
%       pdflatex bitset.dtx
%
% Installation:
%    TDS:tex/generic/oberdiek/bitset.sty
%    TDS:doc/latex/oberdiek/bitset.pdf
%    TDS:doc/latex/oberdiek/test/bitset-test1.tex
%    TDS:doc/latex/oberdiek/test/bitset-test2.tex
%    TDS:doc/latex/oberdiek/test/bitset-test3.tex
%    TDS:source/latex/oberdiek/bitset.dtx
%
%<*ignore>
\begingroup
  \catcode123=1 %
  \catcode125=2 %
  \def\x{LaTeX2e}%
\expandafter\endgroup
\ifcase 0\ifx\install y1\fi\expandafter
         \ifx\csname processbatchFile\endcsname\relax\else1\fi
         \ifx\fmtname\x\else 1\fi\relax
\else\csname fi\endcsname
%</ignore>
%<*install>
\input docstrip.tex
\Msg{************************************************************************}
\Msg{* Installation}
\Msg{* Package: bitset 2011/01/30 v1.1 Handle bit-vector datatype (HO)}
\Msg{************************************************************************}

\keepsilent
\askforoverwritefalse

\let\MetaPrefix\relax
\preamble

This is a generated file.

Project: bitset
Version: 2011/01/30 v1.1

Copyright (C) 2007, 2011 by
   Heiko Oberdiek <heiko.oberdiek at googlemail.com>

This work may be distributed and/or modified under the
conditions of the LaTeX Project Public License, either
version 1.3c of this license or (at your option) any later
version. This version of this license is in
   http://www.latex-project.org/lppl/lppl-1-3c.txt
and the latest version of this license is in
   http://www.latex-project.org/lppl.txt
and version 1.3 or later is part of all distributions of
LaTeX version 2005/12/01 or later.

This work has the LPPL maintenance status "maintained".

This Current Maintainer of this work is Heiko Oberdiek.

The Base Interpreter refers to any `TeX-Format',
because some files are installed in TDS:tex/generic//.

This work consists of the main source file bitset.dtx
and the derived files
   bitset.sty, bitset.pdf, bitset.ins, bitset.drv, bitset-test1.tex,
   bitset-test2.tex, bitset-test3.tex.

\endpreamble
\let\MetaPrefix\DoubleperCent

\generate{%
  \file{bitset.ins}{\from{bitset.dtx}{install}}%
  \file{bitset.drv}{\from{bitset.dtx}{driver}}%
  \usedir{tex/generic/oberdiek}%
  \file{bitset.sty}{\from{bitset.dtx}{package}}%
  \usedir{doc/latex/oberdiek/test}%
  \file{bitset-test1.tex}{\from{bitset.dtx}{test1}}%
  \file{bitset-test2.tex}{\from{bitset.dtx}{test2,etex}}%
  \file{bitset-test3.tex}{\from{bitset.dtx}{test2,noetex}}%
  \nopreamble
  \nopostamble
  \usedir{source/latex/oberdiek/catalogue}%
  \file{bitset.xml}{\from{bitset.dtx}{catalogue}}%
}

\catcode32=13\relax% active space
\let =\space%
\Msg{************************************************************************}
\Msg{*}
\Msg{* To finish the installation you have to move the following}
\Msg{* file into a directory searched by TeX:}
\Msg{*}
\Msg{*     bitset.sty}
\Msg{*}
\Msg{* To produce the documentation run the file `bitset.drv'}
\Msg{* through LaTeX.}
\Msg{*}
\Msg{* Happy TeXing!}
\Msg{*}
\Msg{************************************************************************}

\endbatchfile
%</install>
%<*ignore>
\fi
%</ignore>
%<*driver>
\NeedsTeXFormat{LaTeX2e}
\ProvidesFile{bitset.drv}%
  [2011/01/30 v1.1 Handle bit-vector datatype (HO)]%
\documentclass{ltxdoc}
\usepackage{holtxdoc}[2011/11/22]
\usepackage{amstext}
\usepackage{ifluatex}
\ifluatex
\else
  \usepackage[T1]{fontenc}%
  \usepackage{lmodern}%
\fi
\begin{document}
  \DocInput{bitset.dtx}%
\end{document}
%</driver>
% \fi
%
% \CheckSum{3510}
%
% \CharacterTable
%  {Upper-case    \A\B\C\D\E\F\G\H\I\J\K\L\M\N\O\P\Q\R\S\T\U\V\W\X\Y\Z
%   Lower-case    \a\b\c\d\e\f\g\h\i\j\k\l\m\n\o\p\q\r\s\t\u\v\w\x\y\z
%   Digits        \0\1\2\3\4\5\6\7\8\9
%   Exclamation   \!     Double quote  \"     Hash (number) \#
%   Dollar        \$     Percent       \%     Ampersand     \&
%   Acute accent  \'     Left paren    \(     Right paren   \)
%   Asterisk      \*     Plus          \+     Comma         \,
%   Minus         \-     Point         \.     Solidus       \/
%   Colon         \:     Semicolon     \;     Less than     \<
%   Equals        \=     Greater than  \>     Question mark \?
%   Commercial at \@     Left bracket  \[     Backslash     \\
%   Right bracket \]     Circumflex    \^     Underscore    \_
%   Grave accent  \`     Left brace    \{     Vertical bar  \|
%   Right brace   \}     Tilde         \~}
%
% \GetFileInfo{bitset.drv}
%
% \title{The \xpackage{bitset} package}
% \date{2011/01/30 v1.1}
% \author{Heiko Oberdiek\\\xemail{heiko.oberdiek at googlemail.com}}
%
% \maketitle
%
% \begin{abstract}
% This package defines and implements the data type bit set,
% a vector of bits. The size of the vector may grow dynamically.
% Individual bits can be manipulated.
% \end{abstract}
%
% \tableofcontents
%
% \clearpage
% \section{Documentation}
%
% \subsection{Introduction}
%
% Annotations in the PDF format know entries whose values are
% integers. This numbers are interpreted as set of flags
% specifying properties. For example, annotation dictionaries
% can have a key |/F|. The bits of its integer value
% are interpreted the following way:
% \begin{quote}
% \begin{tabular}{@{}l|l@{}}
%   Bit position & Property name\\
%   \hline
%   1 & Invisible\\
%   2 & Hidden\\
%   3 & Print\\
%   4 & NoZoom\\
%   5 & NoRotate\\
%   6 & NoView\\
%   7 & ReadOnly\\
%   \dots & \dots
% \end{tabular}
% \end{quote}
% Now, let's see how these values are set in package \xpackage{hyperref}
% before it uses this package (before v6.77a):
%\begin{quote}
%\begin{verbatim}
%\ifFld@hidden /F 6\else /F 4\fi
%\end{verbatim}
%\end{quote}
% Where are the other flags? The following example for key |/Ff|
% in a widget annotation
% supports at least three properties:
%\begin{quote}
%\begin{verbatim}
%\ifFld@multiline
%  \ifFld@readonly /Ff 4097\else /Ff 4096\fi
%\else
%  \ifFld@password
%    \ifFld@readonly /Ff 8193\else /Ff 8192\fi
%  \else
%    \ifFld@readonly /Ff 1\fi
%  \fi
%\fi
%\end{verbatim}
%\end{quote}
% But you see the point. It would be a nightmare to continue
% this way in supporting the missing flag settings. This kind
% of integers may have up to 32 bits.
%
% Therefore I wanted a data structure for setting and
% clearing individual bits. Also it should provide
% an export as decimal number. The snipsets above are
% executed in expansion contexts without \TeX's stomach
% commands. It would be convenient to have an expandable
% conversion from the data structure to the integer that
% gets written to the PDF file.
%
% This package \xpackage{bitset} implements such a data
% structure. The interface is quite close to Java's
% class |BitSet| in order not to learn to many interfaces
% for the same kind of data structure.
%
% \subsection{Glossary}
%
% \begin{description}
% \item[Bit set:]
%   A bit set is a vector of bits or flags.
%   The vector size is unlimited and
%   grows dynamically. An undefined bit set is treated as bit
%   set where all bits are cleared.
%
%   Bit sets are addressed by name. A name should consists of
%   letters or digits. Technically it must survive \cs{csname},
%   see \LaTeX's environment names for other names with such
%   a constraint. Package \xpackage{babel}'s shorthands are
%   not supported due to technical reasons. Shorthand support
%   would break expandable operations.
% \item[Size:]
%   A size of a bit set is the number of bits in use.
%   It's the number of the highest index, incremented by one.
%   Sizes are in the range |0| up to |2147483647|, the highest
%   number supported by \TeX.
% \item[Index:]
%   Bit positions in a bit set are addressed by an index number.
%   The bit vector is zero based. The first and least significant
%   bit is addressed by index |0| and the highest possible bit
%   by |2147483646|.
% \item[Bit:]
%   A bit is enoded as |0| for cleared/disabled or |1| for set/enabled.
% \end{description}
%
% \subsection{Design principles}
%
% \begin{description}
% \item[Name conventions:]
%   To avoid conflicts with existing macro names, the operations
%   are prefixed by the package name.
% \item[Zero based indexes:] The first bit is addressed by zero.
%   (Convention of array indexing in C, Java, \dots)
% \item[Unlimited size:]
%   There is no restriction on the size of a bit set other than
%   usual memory limitations. \cs{bitsetSetDec} and \cs{bitsetGetDec}
%   transparently switch to package \xpackage{bigintcalc} if the numbers
%   get too large for \TeX's number limit.
% \item[Expandibility:]
%   Any operation that does not change the bit set is expandable.
%   And all operations that extract or calculate some result
%   do this in exact two expansion steps. For example,
%   a macro \cs{Macro} wants a bit set as decimal number.
%   But the argument must be a plain number without macros.
%   Thus you could prefix \cs{bitsetGetDec} with \cs{number}.
%   However this won't work for bit sets with 31 or more bits
%   because of \TeX's number limit of $2^{31}-1$.
%   then just hit the operator with two \cs{expandafter}:
%   \begin{quote}
%   |\expandafter\expandafter\expandafter|\\
%   |\Macro\bitsetGetDec{foo}|
%   \end{quote}
%   |\bitsetGetDec| is hit first by the third \cs{expandafter}
%   and then by the second one.
% \item[Format independence:] This package is written as \LaTeX\ package,
%   but it does not depend on \LaTeX. It will also work for other formats
%   such as \plainTeX.
% \item[Independence from \TeX\ engines:]
%   Vanilla \TeX\ is all you need. Calculations are delegated to
%   packages \xpackage{intcalc} and \xpackage{bigintcalc}.
%   They don't need any special features, but they will switch
%   to a little more efficient implementation if features such
%   as \cs{numexpr} are available.
% \item[Numeric arguments:] Anything that is accepted by
%   \cs{number}. If \eTeX\ is detected, also expressions
%   for \cs{numexpr} are supported. The only exception so far
%   is the number for \cs{bitsetSetDec}. The number might be too
%   large for \cs{number} or \cs{numexpr}.
% \item[Error messages:]
%   In expandable contexts, only a limited set of \TeX\ primitive
%   commands work as expected. So called stomach commands behave
%   like \cs{relax} and don't get expanded or executed.
%   Unhappily also the error commands belong to this category.
%   The expandable operations will throw an unknown control
%   sequence instead to get \TeX's and user's attention.
%   The name of these control sequences starts with
%   \cs{BitSetError:} with the type of error after the colon.
% \end{description}
%
% \subsection{Operator overview}
%
% \begin{minipage}{\linewidth}
% \begin{description}
% \def\It#1#2{^^A
%   \item[^^A
%     {#1} ^^A
%     \textmd{(\hyperref[{sec:#2}]{section \ref*{sec:#2}})}^^A
%   ]^^A
%   \mbox{}\\^^A
% }
% \def\op#1{^^A
%   \cs{bitset}\texttt{\bfseries#1}^^A
% }
% \def\M#1{\meta{#1}}
% \def\Q{\hspace*{\fill}}
% \It{Miscellaneous}{misc}
%   \op{Reset}\Q \M{BitSet}\\
%   \op{Let}\Q \M{BitSet A} \M{BitSet B}
% \It{Import}{import}
%   \op{SetBin}, \op{SetOct}, \op{SetHex}\Q \M{BitSet} \M{Value}\\
%   \op{SetDec}\Q \M{BitSet} \M{Value}
% \It{Export\textsuperscript{a}}{export}
%   \op{GetBin}, \op{GetOct}, \op{GetHex}\Q \M{BitSet} \M{MinSize}\\
%   \op{GetDec}\Q \M{BitSet}
% \It{Logical operators}{log}
%   \op{And}, \op{AndNot}\Q \M{BitSet A} \M{BitSet B}\\
%   \op{Or}, \op{Xor}\Q \M{BitSet A} \M{BitSet B}
% \It{Shifting}{shift}
%   \op{ShiftLeft}, \op{ShiftRight}\Q \M{BitSet} \M{ShiftAmount}\\
% \It{Bit manipulation}{bitman}
%   \op{Clear}, \op{Set}, \op{Flip}\Q \M{BitSet} \M{Index}\\
%   \op{SetValue}\Q \M{BitSet} \M{Index} \M{Value}\\
%   \op{ClearRange}, \op{SetRange}, \op{FlipRange}\\
%     \Q \M{BitSet} \M{IndexFrom} \M{IndexTo}\\
%   \op{SetValueRange}\Q \M{BitSet} \M{IndexFrom} \M{IndexTo}
% \It{Bit retrieval\textsuperscript{a}}{bitret}
%   \op{Get}\Q \M{BitSet} \M{Index}\\
%   \op{NextClearBit}, \op{NextSetBit}\Q \M{BitSet} \M{Index}\\
%   \op{GetSetBitList}\Q \M{BitSet}
% \It{Bit set properties}{prop}
%   \op{Size}, \op{Cardinality}\Q \M{BitSet}
% \It{Queries\textsuperscript{b}}{query}
%   \op{IsDefined}, \op{IsEmpty}\Q \M{BitSet} \M{Then} \M{Else}\\
%   \op{Equals}, \op{Intersects}\Q \M{BitSet A} \M{BitSet B} \M{Then} \M{Else}\\
%   \op{Query}\Q \M{BitSet} \M{Index} \M{Then} \M{Else}
% \end{description}
% \footnotetext[1]{Macros are expandable, full expansion by two steps.}
% \footnotetext[2]{Macros are expandable.}
% \end{minipage}
%
% \subsection{Package loading}
%
% The package can be used as normal \LaTeX\ package:
%\begin{quote}
%|\usepackage{bitset}|
%\end{quote}
% Also \plainTeX\ is supported:
%\begin{quote}
%|\input bitset.sty\relax|
%\end{quote}
%
% \subsection{Operators}
%
% The following macros work on and with bit sets.
% A bit set \meta{BitSet} is represented by a name.
% The should consist of letters and digits. Technically
% it must survive \cs{csname}. It is the same constraint
% that must be satisfied by label or environment names
% in \LaTeX.
%
% However active characters that are shorthands
% of package \xpackage{babel} are not supported.
% Support for shorthands works by an assignment.
% But many operators
% such as \cs{bitsetGetDec} must be usable in expandable
% contexts. There assignments will not be executed in the
% best case or they will cause errors.
%
% The bits in a bit set are addressed by non-negative
% integers starting from zero. Thus negative index
% numbers cause an error message. Because index numbers
% are \TeX\ numbers. The largest index is 2147483647.
% But in practice memory limits and patience limits
% will be very likely reached much before.
%
% \subsubsection{Miscellaneous}\label{sec:misc}
%
% There isn't a separate operation for bit set creation.
% For simplicity an undefined bit set is treated as bit set
% with all bits cleared.
%
% \begin{declcs}{bitsetReset} \M{BitSet}
% \end{declcs}
% Macro \cs{bitsetReset} clears all bits. The result is an
% empty bit set. It may also be used as replacement for
% an operation ``new'', because an undefined bit set is defined
% afterwards.
%
% \begin{declcs}{bitsetLet} \M{BitSet A} \M{BitSet B}
% \end{declcs}
% Macro \cs{bitsetLet} performs a simple assignment similar
% to \TeX's \cs{let}. After the operation \meta{BitSet A} has
% the same value as \meta{BitSet B}. If \meta{BitSet B} is
% undefined, then \meta{BitSet A} will be the empty bit set.
%
% Note: If \meta{BitSet A} exists, it will be overwritten.
%
% \subsubsection{Import}\label{sec:import}
%
% \begin{declcs}{bitsetSetBin} \M{BitSet} \M{BinaryNumber}\\
%   \cs{bitsetSetOct} \M{BitSet} \M{OctalNumber}\\
%   \cs{bitsetSetHex} \M{BitSet} \M{HexadecimalNumber}
% \end{declcs}
% The numbers are interpreted as bit vectors and the flags in
% the bit \meta{BitSet} set are set accordingly.
% These numeric arguments are the only arguments where
% spaces are allowed. Then the numbers are easier to read.
%
% \begin{declcs}{bitsetSetDec} \M{BitSet} \M{DecimalNumber}
% \end{declcs}
% Macro \cs{bitsetSetDec} uses \meta{DecimalNumber} to
% set the bit set \meta{BitSet}.
% The numeric argument must expand
% to a plain number consisting of decimal digits without
% command tokens or spaces. Internally this argument is
% expanded only. It cannot be passed to \cs{number} or
% \cs{numexpr}, because the number may be too large for them.
% However \cs{number} or |\the\numexpr| may be used
% explicitely. This also helps for unexpandable number
% command tokens or registers (\cs{z@}, \cs{@ne}, \cs{count@}, \dots).
% Also \LaTeX'\ \cs{value} needs prefixing:
% \begin{quote}
% |\bitsetSetDec{foo}{\number\value{bar}}|
% \end{quote}
%
% \subsubsection{Export}\label{sec:export}
%
% \begin{declcs}{bitsetGetBin} \M{BitSet} \M{MinSize}\\
%   \cs{bitsetGetOct} \M{BitSet} \M{MinSize}\\
%   \cs{bitsetGetHex} \M{BitSet} \M{MinSize}
% \end{declcs}
% These macros returns the bit set as binary, octal or hexadecimal
% number. If the bit size is smaller than \meta{MinSize} the gap is
% filled with leading zeros.
% Example:
% \begin{quote}
% |\bitsetReset{abc}|\\
% |\bitsetSet{abc}{2}|\\
% |\bitsetGetBin{abc}{8}| $\rightarrow$ |00000100|\\
% |\bitsetSet{abc}{5}\bitsetSet{abc}{7}|\\
% |\bitsetGetHex{abc}{16}| $\rightarrow$ |00A2|
% \end{quote}
% Macro \cs{bitsetGetHex} uses the uppercase letters |A| to |F|.
% The catcode of the letters is one of 11 (letter) or 12 (other).
%
% \begin{declcs}{bitsetGetDec} \M{BitSet}
% \end{declcs}
% Macro \cs{bitsetGetDec} returns the bit set \meta{BitSet} as
% decimal number. The returned number can be larger than
% \TeX's number limit of $2^{31}-1$.
%
% \subsubsection{Logical operators}\label{sec:log}
%
% \begin{declcs}{bitsetAnd} \M{BitSet A} \M{BitSet B}
% \end{declcs}
% $A_{\text{new}} \mathrel{\mathop{:}}= A_{\text{old}}
% \mathbin{\text{and}} B \qquad (\forall\, \text{bits})$
%
% \begin{declcs}{bitsetAndNot} \M{BitSet A} \M{BitSet B}
% \end{declcs}
% $A_{\text{new}} \mathrel{\mathop{:}}= A_{\text{old}}
% \mathbin{\text{and}} (\text{not}\,B) \qquad (\forall\, \text{bits})$
%
% \begin{declcs}{bitsetOr} \M{BitSet A} \M{BitSet B}
% \end{declcs}
% $A_{\text{new}} \mathrel{\mathop{:}}= A_{\text{old}}
% \mathbin{\text{or}} B \qquad (\forall\, \text{bits})$
%
% \begin{declcs}{bitsetXor} \M{BitSet A} \M{BitSet B}
% \end{declcs}
% $A_{\text{new}} \mathrel{\mathop{:}}= A_{\text{old}}
% \mathbin{\text{xor}} B \qquad (\forall\, \text{bits})$
%
% \subsubsection{Shifting}\label{sec:shift}
%
% \begin{declcs}{bitsetShiftLeft} \M{BitSet} \M{ShiftAmount}\\
%   \cs{bitsetShiftRight} \M{BitSet} \M{ShiftAmount}
% \end{declcs}
% A left shift by one is a multiplication by two, thus left shifting
% moves the flags to higher positions. The new created low positions
% are filled by zeros.
%
% A right shift is the opposite, dividing by two, movint the bits
% to lower positions. The number will become smaller, the lowest
% bits are lost.
%
% If the \meta{ShiftAmount} is negative, it reverts the meaning of
% the shift operation. A left shift becomes a right shift.
% A \meta{ShiftAmount} of zero is ignored.
%
% \subsubsection{Bit manipulation}\label{sec:bitman}
%
% \begin{declcs}{bitsetClear} \M{BitSet} \M{Index}\\
%   \cs{bitsetSet} \M{BitSet} \M{Index}\\
%   \cs{bitsetFlip} \M{BitSet} \M{Index}
% \end{declcs}
% This macros manipulate a single bit in \meta{BitSet} addressed
% by \cs{Index}. Macro \cs{bitsetClear} disables the bit,
% \cs{bitsetSet} enables it and \cs{bitsetFlip} reverts the current
% setting of the bit.
%
% \begin{declcs}{bitsetSetValue} \M{BitSet} \M{Index} \M{Bit}
% \end{declcs}
% Macro \cs{bitsetSetValue} puts bit \meta{Bit} at position \meta{Index}
% in bit set \meta{BitSet}. \meta{Bit} must be a valid \TeX\ number
% equals to zero (disabled/cleared) or one (enabled/set).
%
% \subsubsection{Bit retrieval}\label{sec:bitret}
%
% \begin{declcs}{bitsetGet} \M{BitSet} \M{Index}
% \end{declcs}
% Macro \cs{bitsetGet} extracts the status of the bit at
% position \meta{Index} in bit set \meta{BitSet}.
% Digit |1| is returned if the bit is set/enabled. If the
% bit is cleared/disabled and in cases of an undefined bitset or
% an index number out of range the return value is |0|.
%
% \begin{declcs}{bitsetNextClearBit} \M{BitSet} \M{Index}
% \end{declcs}
% Starting at position \meta{Index} (inclusive) the bits
% are inspected. The first position without a set bit is returned.
% Possible results are decimal numbers: \meta{Index}, \meta{Index} + 1,
% \dots, ($\infty$)
%
% \begin{declcs}{bitsetNextSetBit} \M{BitSet} \M{Index}
% \end{declcs}
% Starting at position \meta{Index} (inclusive) the bits are
% inspected and the index position of the first found set bit
% is returned. If there isn't such a bit, then the result is
% |-1|. In summary possible results are decimal numbers:
% |-1|, \meta{Index}, \meta{Index} + 1, \dots, ($\infty$)
%
% \begin{declcs}{bitsetGetSetBitList} \M{BitSet}
% \end{declcs}
% Macro \cs{bitsetGetSetBitList} is an application for
% \cs{bitsetNextSetBit}. The set bits are iterated and
% returned as comma separated list of index positions
% in increasing order. The list is empty in case of
% an empty bit set.
%
% \subsubsection{Bit set properties}\label{sec:prop}
%
% \begin{declcs}{bitsetSize} \M{BitSet}
% \end{declcs}
% Macro \cs{bitsetSize} returns number of bits in use.
% It is the same as the index number of the highest set/enabled bit
% incremented by one.
%
% \begin{declcs}{bitsetCardinality} \M{BitSet}
% \end{declcs}
% Macro \cs{bitsetCardinality} counts the number of set/enabled bits.
%
% \subsubsection{Queries}\label{sec:query}
%
% Also the query procedures are expandable. They ask for a piece
% of information about a bit set and execute code depending
% on the answer.
%
% \begin{declcs}{bitsetIsDefined} \M{BitSet} \M{Then} \M{Else}
% \end{declcs}
% If the bit set with the name \meta{BitSet} exists the code
% given in \meta{Then} is executed, otherwise \meta{Else} is used.
%
% \begin{declcs}{bitsetIsEmpty} \M{BitSet} \M{Then} \M{Else}
% \end{declcs}
% If the bit set \meta{BitSet} exists and at least one bit
% is set/enabled, the code in \meta{Then} is executed,
% \meta{Else} otherwise.
%
% \begin{declcs}{bitsetEquals} \M{BitSet A} \M{BitSet B} \M{Then} \M{Else}
% \end{declcs}
% Both bit sets are equal if and only if either both are undefined
% or both are defined and represents the same bit values at the
% same positions. Thus this definition is reflexive, symmeric,
% and transitive, enough for an equivalent relation.
%
% \begin{declcs}{bitsetIntersects} \M{BitSet A} \M{BitSet B} \M{Then} \M{Else}
% \end{declcs}
% If and only if \meta{BitSet A} and \meta{BitSet B} have
% at least one bit at the same position that is set, then
% code part \meta{Then} is executed.
%
% \begin{declcs}{bitsetQuery} \M{BitSet} \M{Index} \M{Then} \M{Else}
% \end{declcs}
% It's just a wrapper for \cs{bitsetGet}. If the bit at position
% \meta{Index} is enabled, code \meta{Then} is called.
%
% \StopEventually{
% }
%
% \section{Implementation}
%
% The internal format of a bit set is quite simple, a sequence
% of digits |0| and |1|. The least significant bit is left.
% A bit set without any flag set is encoded by |0|. Also
% undefined bit sets are treated that way. After the
% highest bit that is set there are no further zeroes.
% A regular expression of valid bit sets values:
% \begin{quote}
%  |0|\string||[01]*1|
% \end{quote}
%
%    \begin{macrocode}
%<*package>
%    \end{macrocode}
%
% \subsection{Reload check and package identification}
%    Reload check, especially if the package is not used with \LaTeX.
%    \begin{macrocode}
\begingroup\catcode61\catcode48\catcode32=10\relax%
  \catcode13=5 % ^^M
  \endlinechar=13 %
  \catcode35=6 % #
  \catcode39=12 % '
  \catcode44=12 % ,
  \catcode45=12 % -
  \catcode46=12 % .
  \catcode58=12 % :
  \catcode64=11 % @
  \catcode123=1 % {
  \catcode125=2 % }
  \expandafter\let\expandafter\x\csname ver@bitset.sty\endcsname
  \ifx\x\relax % plain-TeX, first loading
  \else
    \def\empty{}%
    \ifx\x\empty % LaTeX, first loading,
      % variable is initialized, but \ProvidesPackage not yet seen
    \else
      \expandafter\ifx\csname PackageInfo\endcsname\relax
        \def\x#1#2{%
          \immediate\write-1{Package #1 Info: #2.}%
        }%
      \else
        \def\x#1#2{\PackageInfo{#1}{#2, stopped}}%
      \fi
      \x{bitset}{The package is already loaded}%
      \aftergroup\endinput
    \fi
  \fi
\endgroup%
%    \end{macrocode}
%    Package identification:
%    \begin{macrocode}
\begingroup\catcode61\catcode48\catcode32=10\relax%
  \catcode13=5 % ^^M
  \endlinechar=13 %
  \catcode35=6 % #
  \catcode39=12 % '
  \catcode40=12 % (
  \catcode41=12 % )
  \catcode44=12 % ,
  \catcode45=12 % -
  \catcode46=12 % .
  \catcode47=12 % /
  \catcode58=12 % :
  \catcode64=11 % @
  \catcode91=12 % [
  \catcode93=12 % ]
  \catcode123=1 % {
  \catcode125=2 % }
  \expandafter\ifx\csname ProvidesPackage\endcsname\relax
    \def\x#1#2#3[#4]{\endgroup
      \immediate\write-1{Package: #3 #4}%
      \xdef#1{#4}%
    }%
  \else
    \def\x#1#2[#3]{\endgroup
      #2[{#3}]%
      \ifx#1\@undefined
        \xdef#1{#3}%
      \fi
      \ifx#1\relax
        \xdef#1{#3}%
      \fi
    }%
  \fi
\expandafter\x\csname ver@bitset.sty\endcsname
\ProvidesPackage{bitset}%
  [2011/01/30 v1.1 Handle bit-vector datatype (HO)]%
%    \end{macrocode}
%
% \subsection{Catcodes}
%
%    \begin{macrocode}
\begingroup\catcode61\catcode48\catcode32=10\relax%
  \catcode13=5 % ^^M
  \endlinechar=13 %
  \catcode123=1 % {
  \catcode125=2 % }
  \catcode64=11 % @
  \def\x{\endgroup
    \expandafter\edef\csname BitSet@AtEnd\endcsname{%
      \endlinechar=\the\endlinechar\relax
      \catcode13=\the\catcode13\relax
      \catcode32=\the\catcode32\relax
      \catcode35=\the\catcode35\relax
      \catcode61=\the\catcode61\relax
      \catcode64=\the\catcode64\relax
      \catcode123=\the\catcode123\relax
      \catcode125=\the\catcode125\relax
    }%
  }%
\x\catcode61\catcode48\catcode32=10\relax%
\catcode13=5 % ^^M
\endlinechar=13 %
\catcode35=6 % #
\catcode64=11 % @
\catcode123=1 % {
\catcode125=2 % }
\def\TMP@EnsureCode#1#2{%
  \edef\BitSet@AtEnd{%
    \BitSet@AtEnd
    \catcode#1=\the\catcode#1\relax
  }%
  \catcode#1=#2\relax
}
\TMP@EnsureCode{33}{12}% !
\TMP@EnsureCode{39}{12}% '
\TMP@EnsureCode{40}{12}% (
\TMP@EnsureCode{41}{12}% )
\TMP@EnsureCode{42}{12}% *
\TMP@EnsureCode{43}{12}% +
\TMP@EnsureCode{44}{12}% ,
\TMP@EnsureCode{45}{12}% -
\TMP@EnsureCode{46}{12}% .
\TMP@EnsureCode{47}{12}% /
\TMP@EnsureCode{58}{11}% : (letter!)
\TMP@EnsureCode{60}{12}% <
\TMP@EnsureCode{62}{12}% >
\TMP@EnsureCode{63}{14}% ? (comment!)
\TMP@EnsureCode{91}{12}% [
\TMP@EnsureCode{93}{12}% ]
\TMP@EnsureCode{96}{12}% `
\edef\BitSet@AtEnd{\BitSet@AtEnd\noexpand\endinput}
\begingroup\expandafter\expandafter\expandafter\endgroup
\expandafter\ifx\csname BitSet@TestMode\endcsname\relax
\else
  \catcode63=9 % ? (ignore)
\fi
? \let\BitSet@@TestMode\BitSet@TestMode
%    \end{macrocode}
%
% \subsection{Package loading}
%
%    \begin{macrocode}
\begingroup\expandafter\expandafter\expandafter\endgroup
\expandafter\ifx\csname RequirePackage\endcsname\relax
  \def\TMP@RequirePackage#1[#2]{%
    \begingroup\expandafter\expandafter\expandafter\endgroup
    \expandafter\ifx\csname ver@#1.sty\endcsname\relax
      \input #1.sty\relax
    \fi
  }%
  \TMP@RequirePackage{infwarerr}[2007/09/09]%
  \TMP@RequirePackage{intcalc}[2007/09/27]%
  \TMP@RequirePackage{bigintcalc}[2007/09/27]%
\else
  \RequirePackage{infwarerr}[2007/09/09]%
  \RequirePackage{intcalc}[2007/09/27]%
  \RequirePackage{bigintcalc}[2007/09/27]%
\fi
%    \end{macrocode}
%
% \subsection{Help macros}
%
% \subsubsection{Number constant}
%
%    \begin{macro}{\BitSet@MaxSize}
%    \begin{macrocode}
\def\BitSet@MaxSize{2147483647}%
%    \end{macrocode}
%    \end{macro}
%
% \subsubsection{General basic macros}
%
%    \begin{macro}{\BitSet@Empty}
%    \begin{macrocode}
\def\BitSet@Empty{}
%    \end{macrocode}
%    \end{macro}
%    \begin{macro}{\BitSet@FirstOfOne}
%    \begin{macrocode}
\def\BitSet@FirstOfOne#1{#1}
%    \end{macrocode}
%    \end{macro}
%    \begin{macro}{\BitSet@Gobble}
%    \begin{macrocode}
\def\BitSet@Gobble#1{}
%    \end{macrocode}
%    \end{macro}
%    \begin{macro}{\BitSet@FirstOfTwo}
%    \begin{macrocode}
\def\BitSet@FirstOfTwo#1#2{#1}
%    \end{macrocode}
%    \end{macro}
%    \begin{macro}{\BitSet@SecondOfTwo}
%    \begin{macrocode}
\def\BitSet@SecondOfTwo#1#2{#2}
%    \end{macrocode}
%    \end{macro}
%
%    \begin{macro}{\BitSet@Space}
%    \begin{macrocode}
\def\BitSet@Space{ }
%    \end{macrocode}
%    \end{macro}
%    \begin{macro}{\BitSet@ZapSpace}
%    \begin{macrocode}
\def\BitSet@ZapSpace#1 #2{%
  #1%
  \ifx\BitSet@Empty#2%
  \else
    \expandafter\BitSet@ZapSpace
  \fi
  #2%
}
%    \end{macrocode}
%    \end{macro}
%
% \subsubsection{Tail recursion}
%
%    \begin{macro}{\BitSet@Fi}
%    \begin{macrocode}
\let\BitSet@Fi\fi
%    \end{macrocode}
%    \end{macro}
%    \begin{macro}{\BitSet@AfterFi}
%    \begin{macrocode}
\def\BitSet@AfterFi#1#2\BitSet@Fi{\fi#1}
%    \end{macrocode}
%    \end{macro}
%    \begin{macro}{\BitSet@AfterFiFi}
%    \begin{macrocode}
\def\BitSet@AfterFiFi#1#2\BitSet@Fi{\fi\fi#1}%
%    \end{macrocode}
%    \end{macro}
%    \begin{macro}{\BitSet@AfterFiFiFi}
%    \begin{macrocode}
\def\BitSet@AfterFiFiFi#1#2\BitSet@Fi{\fi\fi\fi#1}%
%    \end{macrocode}
%    \end{macro}
%
% \subsubsection{Check macros}
%
%    \begin{macro}{\BitSet@IfUndefined}
%    \begin{macrocode}
\def\BitSet@IfUndefined#1{%
  \expandafter\ifx\csname BS@#1\endcsname\relax
    \expandafter\BitSet@FirstOfTwo
  \else
    \expandafter\BitSet@SecondOfTwo
  \fi
}
%    \end{macrocode}
%    \end{macro}
%
%    \begin{macro}{\BitSet@CheckIndex}
%    |#1|: continuation code\\
%    |#2|: BitSet\\
%    |#3|: Index
%    \begin{macrocode}
\def\BitSet@CheckIndex#1#2#3{%
  \BitSet@IfUndefined{#2}{\bitsetReset{#2}}{}%
  \expandafter\expandafter\expandafter\BitSet@@CheckIndex
  \intcalcNum{#3}!%
  {#2}{#1}%
}
%    \end{macrocode}
%    \end{macro}
%    \begin{macro}{\BitSet@@CheckIndex}
%    |#1|: plain Index\\
%    |#2|: BitSet\\
%    |#3|: continuation code
%    \begin{macrocode}
\def\BitSet@@CheckIndex#1!#2#3{%
  \ifnum#1<0 %
    \BitSet@AfterFi{%
      \@PackageError{bitset}{%
        Invalid negative index (#1)%
      }\@ehc
    }%
  \else
    \BitSet@AfterFi{%
      #3{#2}{#1}%
    }%
  \BitSet@Fi
}
%    \end{macrocode}
%    \end{macro}
%
% \subsection{Miscellaneous}\label{src:misc}
%
%    \begin{macro}{\bitsetReset}
%    \begin{macrocode}
\def\bitsetReset#1{%
  \expandafter\def\csname BS@#1\endcsname{0}%
}
%    \end{macrocode}
%    \end{macro}
%
%    \begin{macro}{\bitsetLet}
%    \begin{macrocode}
\def\bitsetLet#1#2{%
  \BitSet@IfUndefined{#2}{%
    \bitsetReset{#1}%
  }{%
    \expandafter\let\csname BS@#1\expandafter\endcsname
                    \csname BS@#2\endcsname
  }%
}
%    \end{macrocode}
%    \end{macro}
%
% \subsection{Import}\label{src:import}
%
% \subsubsection{From binary number}
%
%    \begin{macro}{\bitsetSetBin}
%    \begin{macrocode}
\def\bitsetSetBin#1#2{%
  \edef\BitSet@Temp{#2}%
  \edef\BitSet@Temp{%
    \expandafter\expandafter\expandafter\BitSet@ZapSpace
    \expandafter\BitSet@Temp\BitSet@Space\BitSet@Empty
  }%
  \edef\BitSet@Temp{%
    \expandafter\BitSet@KillZeros\BitSet@Temp\BitSet@Empty
  }%
  \ifx\BitSet@Temp\BitSet@Empty
    \expandafter\let\csname BS@#1\endcsname\BitSet@Zero
  \else
    \expandafter\edef\csname BS@#1\endcsname{%
      \expandafter\BitSet@Reverse\BitSet@Temp!%
    }%
  \fi
}
%    \end{macrocode}
%    \end{macro}
%    \begin{macro}{\BitSet@KillZeros}
%    \begin{macrocode}
\def\BitSet@KillZeros#1{%
  \ifx#10%
    \expandafter\BitSet@KillZeros
  \else
    #1%
  \fi
}
%    \end{macrocode}
%    \end{macro}
%    \begin{macro}{\BitSet@Reverse}
%    \begin{macrocode}
\def\BitSet@Reverse#1#2!{%
  \ifx\\#2\\%
    #1%
  \else
    \BitSet@AfterFi{%
      \BitSet@Reverse#2!#1%
    }%
  \BitSet@Fi
}
%    \end{macrocode}
%    \end{macro}
%
% \subsubsection{From octal/hex number}
%
%    \begin{macro}{\bitsetSetOct}
%    \begin{macrocode}
\def\bitsetSetOct{%
  \BitSet@SetOctHex\BitSet@FromFirstOct
}
%    \end{macrocode}
%    \end{macro}
%    \begin{macro}{\bitsetSetHex}
%    \begin{macrocode}
\def\bitsetSetHex{%
  \BitSet@SetOctHex\BitSet@FromFirstHex
}
%    \end{macrocode}
%    \end{macro}
%    \begin{macro}{\BitSet@SetOctHex}
%    \begin{macrocode}
\def\BitSet@SetOctHex#1#2#3{%
  \edef\BitSet@Temp{#3}%
  \edef\BitSet@Temp{%
    \expandafter\expandafter\expandafter\BitSet@ZapSpace
    \expandafter\BitSet@Temp\BitSet@Space\BitSet@Empty
  }%
  \edef\BitSet@Temp{%
    \expandafter\BitSet@KillZeros\BitSet@Temp\BitSet@Empty
  }%
  \ifx\BitSet@Temp\BitSet@Empty
    \expandafter\let\csname BS@#2\endcsname\BitSet@Zero
  \else
    \edef\BitSet@Temp{%
      \expandafter#1\BitSet@Temp!%
    }%
    \ifx\BitSet@Temp\BitSet@Empty
      \expandafter\let\csname BS@#2\endcsname\BitSet@Zero
    \else
      \expandafter\edef\csname BS@#2\endcsname{%
        \expandafter\BitSet@Reverse\BitSet@Temp!%
      }%
    \fi
  \fi
}
%    \end{macrocode}
%    \end{macro}
%    \begin{macro}{\BitSet@FromFirstOct}
%    \begin{macrocode}
\def\BitSet@FromFirstOct#1{%
  \ifx#1!%
  \else
    \ifcase#1 \BitSet@AfterFiFi\BitSet@FromFirstOct
    \or 1%
    \or 10%
    \or 11%
    \or 100%
    \or 101%
    \or 110%
    \or 111%
    \else \BitSetError:WrongOctalDigit%
    \fi
    \expandafter\BitSet@FromOct
  \BitSet@Fi
}
%    \end{macrocode}
%    \end{macro}
%    \begin{macro}{\BitSet@FromOct}
%    \begin{macrocode}
\def\BitSet@FromOct#1{%
  \ifx#1!%
  \else
    \ifcase#1 000%
    \or 001%
    \or 010%
    \or 011%
    \or 100%
    \or 101%
    \or 110%
    \or 111%
    \else \BitSetError:WrongOctalDigit%
    \fi
    \expandafter\BitSet@FromOct
  \fi
}
%    \end{macrocode}
%    \end{macro}
%
%    \begin{macro}{\BitSet@FromFirstHex}
%    \begin{macrocode}
\def\BitSet@FromFirstHex#1{%
  \ifx#1!%
  \else
    \ifx#10%
      \BitSet@AfterFiFi\BitSet@FromFirstHex
    \fi
    \expandafter\ifx\csname BitSet@Hex#1\endcsname\relax
      \BitSetError:InvalidHexDigit%
    \else
      \expandafter\expandafter\expandafter\BitSet@KillZeros
      \csname BitSet@Hex#1\endcsname
    \fi
    \expandafter\BitSet@FromHex
  \BitSet@Fi
}
%    \end{macrocode}
%    \end{macro}
%    \begin{macro}{\BitSet@FromHex}
%    \begin{macrocode}
\def\BitSet@FromHex#1{%
  \ifx#1!%
  \else
    \expandafter\ifx\csname BitSet@Hex#1\endcsname\relax
      \BitSetError:InvalidHexDigit%
    \else
      \csname BitSet@Hex#1\endcsname
    \fi
    \expandafter\BitSet@FromHex
  \fi
}
%    \end{macrocode}
%    \end{macro}
%    \begin{macro}{\BitSet@Hex[0..F]}
%    \begin{macrocode}
\def\BitSet@Temp#1{%
  \expandafter\def\csname BitSet@Hex#1\endcsname
}
\BitSet@Temp 0{0000}%
\BitSet@Temp 1{0001}%
\BitSet@Temp 2{0010}%
\BitSet@Temp 3{0011}%
\BitSet@Temp 4{0100}%
\BitSet@Temp 5{0101}%
\BitSet@Temp 6{0110}%
\BitSet@Temp 7{0111}%
\BitSet@Temp 8{1000}%
\BitSet@Temp 9{1001}%
\BitSet@Temp A{1010}%
\BitSet@Temp B{1011}%
\BitSet@Temp C{1100}%
\BitSet@Temp D{1101}%
\BitSet@Temp E{1110}%
\BitSet@Temp F{1111}%
\BitSet@Temp a{1010}%
\BitSet@Temp b{1011}%
\BitSet@Temp c{1100}%
\BitSet@Temp d{1101}%
\BitSet@Temp e{1110}%
\BitSet@Temp f{1111}%
%    \end{macrocode}
%    \end{macro}
%
% \subsubsection{From decimal number}
%
%    \begin{macro}{\bitsetSetDec}
%    \begin{macrocode}
\def\bitsetSetDec#1#2{%
  \edef\BitSet@Temp{#2}%
  \edef\BitSet@Temp{%
    \expandafter\expandafter\expandafter\BitSet@ZapSpace
    \expandafter\BitSet@Temp\BitSet@Space\BitSet@Empty
  }%
  \edef\BitSet@Temp{%
    \expandafter\BitSet@KillZeros\BitSet@Temp\BitSet@Empty
  }%
  \ifx\BitSet@Temp\BitSet@Empty
    \expandafter\let\csname BS@#1\endcsname\BitSet@Zero
  \else
    \ifcase\bigintcalcSgn{\BitSet@Temp} %
      \expandafter\let\csname BS@#1\endcsname\BitSet@Zero
    \or
      \ifnum\bigintcalcCmp\BitSet@Temp\BitSet@MaxSize>0 %
        \expandafter\edef\csname BS@#1\endcsname{%
          \expandafter\BitSet@SetDecBig\BitSet@Temp!%
        }%
      \else
        \expandafter\edef\csname BS@#1\endcsname{%
          \expandafter\BitSet@SetDec\BitSet@Temp!%
        }%
      \fi
    \else
      \@PackageError{bitset}{%
        Bit sets cannot be negative%
      }\@ehc
    \fi
  \fi
}
%    \end{macrocode}
%    \end{macro}
%    \begin{macro}{\BitSet@SetDecBig}
%    \begin{macrocode}
\def\BitSet@SetDecBig#1#2#3#4#5#6#7#8#9!{%
  \ifx\\#9\\%
    \BitSet@SetDec#1#2#3#4#5#6#7#8!%
  \else
    \ifcase\BigIntCalcOdd#1#2#4#5#6#7#8#9! %
      0%
    \or
      1%
?   \else\BitSetError:ThisCannotHappen%
    \fi
    \BitSet@AfterFi{%
      \expandafter\expandafter\expandafter\BitSet@SetDecBig
      \BigIntCalcShr#1#2#3#4#5#6#7#8#9!!%
    }%
  \BitSet@Fi
}
%    \end{macrocode}
%    \end{macro}
%    \begin{macro}{\BitSet@SetDec}
%    \begin{macrocode}
\def\BitSet@SetDec#1!{%
  \ifcase#1 %
  \or 1%
  \else
    \ifodd#1 %
      1%
    \else
      0%
    \fi
    \BitSet@AfterFi{%
      \expandafter\expandafter\expandafter\BitSet@SetDec
      \IntCalcShr#1!!%
    }%
  \BitSet@Fi
}
%    \end{macrocode}
%    \end{macro}
%
% \subsection{Export}\label{src:export}
%
% \subsubsection{To binary number}
%
%    \begin{macro}{\bitsetGetBin}
%    \begin{macrocode}
\def\bitsetGetBin#1#2{%
  \romannumeral0%
  \expandafter\expandafter\expandafter\BitSet@@GetBin
  \intcalcNum{#2}!{#1}%
}
%    \end{macrocode}
%    \end{macro}
%    \begin{macro}{\BitSet@@GetBin}
%    \begin{macrocode}
\def\BitSet@@GetBin#1!#2{%
  \BitSet@IfUndefined{#2}{%
    \ifnum#1>1 %
      \BitSet@AfterFi{%
        \expandafter\expandafter\expandafter\BitSet@Fill
        \IntCalcDec#1!!0%
      }%
    \else
      \BitSet@AfterFi{ 0}%
    \BitSet@Fi
  }{%
    \expandafter\expandafter\expandafter\BitSet@NumBinRev
    \expandafter\expandafter\expandafter1%
    \expandafter\expandafter\expandafter!%
    \csname BS@#2\endcsname!!#1!%
  }%
}
%    \end{macrocode}
%    \end{macro}
%    \begin{macro}{\BitSet@Fill}
%    |#1|: number of leading digits |0|\\
%    |#2|: result
%    \begin{macrocode}
\def\BitSet@Fill#1!{%
  \ifnum#1>0 %
    \BitSet@AfterFi{%
      \expandafter\expandafter\expandafter\BitSet@Fill
      \IntCalcDec#1!!0%
    }%
  \else
    \BitSet@AfterFi{ }%
  \BitSet@Fi
}
%    \end{macrocode}
%    \end{macro}
%
%    \begin{macro}{\BitSet@NumBinRev}
%    |#1|: bit counter (including |#2|)\\
%    |#2#3|: reverted number\\
%    |#4|: result\\
%    |#5|: min size
%    \begin{macrocode}
\def\BitSet@NumBinRev#1!#2#3!{%
  \ifx\\#3\\%
    \BitSet@AfterFi{%
      \BitSet@NumBinFill#1!#2%
    }%
  \else
    \BitSet@AfterFi{%
      \expandafter\expandafter\expandafter\BitSet@NumBinRev
      \IntCalcInc#1!!#3!#2%
    }%
  \BitSet@Fi
}
%    \end{macrocode}
%    \end{macro}
%    \begin{macro}{\BitSet@NumBinFill}
%    \begin{macrocode}
\def\BitSet@NumBinFill#1!#2!#3!{%
  \ifnum#3>#1 %
    \BitSet@AfterFi{%
      \expandafter\expandafter\expandafter\BitSet@Fill
      \IntCalcSub#3!#1!!#2%
    }%
  \else
    \BitSet@AfterFi{ #2}%
  \BitSet@Fi
}
%    \end{macrocode}
%    \end{macro}
%
% \subsubsection{To octal/hexadecimal number}
%
%    \begin{macro}{\bitsetGetOct}
%    \begin{macrocode}
\def\bitsetGetOct#1#2{%
  \romannumeral0%
  \bitsetIsEmpty{#1}{%
    \expandafter\expandafter\expandafter\BitSet@@GetOctHex
    \intcalcNum{#2}!3!230%
  }{%
    \expandafter\expandafter\expandafter\BitSet@@GetOct
    \expandafter\expandafter\expandafter1%
    \expandafter\expandafter\expandafter!%
    \expandafter\expandafter\expandafter!%
    \csname BS@#1\endcsname00%
    \BitSet@Empty\BitSet@Empty\BitSet@Empty!{#2}%
  }%
}
%    \end{macrocode}
%    \end{macro}
%    \begin{macro}{\bitsetGetHex}
%    \begin{macrocode}
\def\bitsetGetHex#1#2{%
  \romannumeral0%
  \bitsetIsEmpty{#1}{%
    \expandafter\expandafter\expandafter\BitSet@@GetOctHex
    \intcalcNum{#2}!4!340%
  }{%
    \expandafter\expandafter\expandafter\BitSet@@GetHex
    \expandafter\expandafter\expandafter1%
    \expandafter\expandafter\expandafter!%
    \expandafter\expandafter\expandafter!%
    \csname BS@#1\endcsname000%
    \BitSet@Empty\BitSet@Empty\BitSet@Empty\BitSet@Empty!{#2}%
  }%
}
%    \end{macrocode}
%    \end{macro}
%    \begin{macro}{\BitSet@@GetOct}
%    |#1|: number of digits\\
%    |#2|: result\\
%    |#3#4#5|: bits
%    \begin{macrocode}
\def\BitSet@@GetOct#1!#2!#3#4#5{%
  \ifx#5\BitSet@Empty
    \BitSet@AfterFi{%
      \expandafter\expandafter\expandafter\BitSet@GetOctHex
      \IntCalcDec#1!!#2!23%
    }%
  \else
    \BitSet@AfterFi{%
      \expandafter\expandafter\expandafter\BitSet@@GetOct
      \number\IntCalcInc#1!\expandafter\expandafter\expandafter!%
      \csname BitSet@Oct#5#4#3\endcsname#2!%
    }%
  \BitSet@Fi
}
%    \end{macrocode}
%    \end{macro}
%    \begin{macro}{\BitSet@Oct[000..111]}
%    \begin{macrocode}
\def\BitSet@Temp#1#2#3#4{%
  \expandafter\def\csname BitSet@Oct#1#2#3\endcsname{#4}%
}
\BitSet@Temp0000%
\BitSet@Temp0011%
\BitSet@Temp0102%
\BitSet@Temp0113%
\BitSet@Temp1004%
\BitSet@Temp1015%
\BitSet@Temp1106%
\BitSet@Temp1117%
%    \end{macrocode}
%    \end{macro}
%    \begin{macro}{\BitSet@@GetHex}
%    |#1|: number of digits\\
%    |#2|: result\\
%    |#3#4#5#6|: bits
%    \begin{macrocode}
\def\BitSet@@GetHex#1!#2!#3#4#5#6{%
  \ifx#6\BitSet@Empty
    \BitSet@AfterFi{%
      \expandafter\expandafter\expandafter\BitSet@GetOctHex
      \IntCalcDec#1!!#2!34%
    }%
  \else
    \BitSet@AfterFi{%
      \expandafter\expandafter\expandafter\BitSet@@GetHex
      \number\IntCalcInc#1!\expandafter\expandafter\expandafter!%
      \csname BitSet@Hex#6#5#4#3\endcsname#2!%
    }%
  \BitSet@Fi
}
%    \end{macrocode}
%    \end{macro}
%    \begin{macro}{\BitSet@Hex[0000..1111]}
%    \begin{macrocode}
\def\BitSet@Temp#1#2#3#4#5{%
  \expandafter\def\csname BitSet@Hex#1#2#3#4\endcsname{#5}%
}
\BitSet@Temp00000%
\BitSet@Temp00011%
\BitSet@Temp00102%
\BitSet@Temp00113%
\BitSet@Temp01004%
\BitSet@Temp01015%
\BitSet@Temp01106%
\BitSet@Temp01117%
\BitSet@Temp10008%
\BitSet@Temp10019%
\BitSet@Temp1010A%
\BitSet@Temp1011B%
\BitSet@Temp1100C%
\BitSet@Temp1101D%
\BitSet@Temp1110E%
\BitSet@Temp1111F%
%    \end{macrocode}
%    \end{macro}
%    \begin{macro}{\BitSet@GetOctHex}
%    Leading zeros $ (|#4| - |#1| * 3 + 2)/3 $ if $ |#4| > |#1| * 3$\\
%    |#1|: digit size\\
%    |#2|: result\\
%    |#3|: bits per digit - 1\\
%    |#4|: bits per digit
%    |#5|: garbage\\
%    |#6|: min size
%    \begin{macrocode}
\def\BitSet@GetOctHex#1!#2!#3#4#5!#6{%
  \expandafter\BitSet@@GetOctHex
  \number\intcalcNum{#6}\expandafter\expandafter\expandafter!%
  \IntCalcMul#1!#4!!#3#4#2%
}
%    \end{macrocode}
%    \end{macro}
%    \begin{macro}{\BitSet@@GetOctHex}
%    |#1|: plain min size\\
%    |#2|: digits * (bits per digit)\\
%    |#3|: bits per digit - 1\\
%    |#4|: bits per digit
%    \begin{macrocode}
\def\BitSet@@GetOctHex#1!#2!#3#4{%
  \ifnum#1>#2 %
    \BitSet@AfterFi{%
      \expandafter\expandafter\expandafter\expandafter
      \expandafter\expandafter\expandafter\BitSet@Fill
      \expandafter\IntCalcDiv\number
      \expandafter\expandafter\expandafter\IntCalcAdd
      \IntCalcSub#1!#2!!#3!!#4!!%
    }%
  \else
    \BitSet@AfterFi{ }%
  \BitSet@Fi
}
%    \end{macrocode}
%    \end{macro}
%
% \subsubsection{To decimal number}
%
%    \begin{macro}{\bitsetGetDec}
%    \begin{macrocode}
\def\bitsetGetDec#1{%
  \romannumeral0%
  \BitSet@IfUndefined{#1}{ 0}{%
    \expandafter\expandafter\expandafter\BitSet@GetDec
    \csname BS@#1\endcsname!%
  }%
}
%    \end{macrocode}
%    \end{macro}
%    \begin{macro}{\BitSet@GetDec}
%    \begin{macrocode}
\def\BitSet@GetDec#1#2!{%
  \ifx\\#2\\%
    \BitSet@AfterFi{ #1}%
  \else
    \BitSet@AfterFi{%
      \BitSet@@GetDec2!#1!#2!%
    }%
  \BitSet@Fi
}
%    \end{macrocode}
%    \end{macro}
%    \begin{macro}{\BitSet@@GetDec}
%    |#1|: power of two\\
%    |#2|: result\\
%    |#3#4|: number
%    \begin{macrocode}
\def\BitSet@@GetDec#1!#2!#3#4!{%
  \ifx\\#4\\%
    \ifx#31%
      \BitSet@AfterFiFi{%
        \expandafter\expandafter\expandafter\BitSet@Space
        \IntCalcAdd#1!#2!%
      }%
    \else
      \BitSet@AfterFiFi{ #2}%
    \fi
  \else
    \ifx#31%
      \BitSet@AfterFiFi{%
        \csname BitSet@N#1%
        \expandafter\expandafter\expandafter\endcsname
        \IntCalcAdd#1!#2!!#4!%
      }%
    \else
      \BitSet@AfterFiFi{%
        \csname BitSet@N#1\endcsname#2!#4!%
      }%
    \fi
  \BitSet@Fi
}
%    \end{macrocode}
%    \end{macro}
%    \begin{macro}{\BitSet@N[1,2,4,...]}
%    \begin{macrocode}
\def\BitSet@Temp#1#2{%
  \expandafter\def\csname BitSet@N#1\endcsname{%
    \BitSet@@GetDec#2!%
  }%
}
\BitSet@Temp{1}{2}
\BitSet@Temp{2}{4}
\BitSet@Temp{4}{8}
\BitSet@Temp{8}{16}
\BitSet@Temp{16}{32}
\BitSet@Temp{32}{64}
\BitSet@Temp{64}{128}
\BitSet@Temp{128}{256}
\BitSet@Temp{256}{512}
\BitSet@Temp{512}{1024}
\BitSet@Temp{1024}{2048}
\BitSet@Temp{2048}{4096}
\BitSet@Temp{4096}{8192}
\BitSet@Temp{8192}{16384}
\BitSet@Temp{16384}{32768}
\BitSet@Temp{32768}{65536}
\BitSet@Temp{65536}{131072}
\BitSet@Temp{131072}{262144}
\BitSet@Temp{262144}{524288}
\BitSet@Temp{524288}{1048576}
\BitSet@Temp{1048576}{2097152}
\BitSet@Temp{2097152}{4194304}
\BitSet@Temp{4194304}{8388608}
\BitSet@Temp{8388608}{16777216}
\BitSet@Temp{16777216}{33554432}
\BitSet@Temp{33554432}{67108864}
\BitSet@Temp{67108864}{134217728}
\BitSet@Temp{134217728}{268435456}
\BitSet@Temp{268435456}{536870912}
\BitSet@Temp{536870912}{1073741824}
%    \end{macrocode}
%    \end{macro}
%    \begin{macro}{\BitSet@N1073741824}
%    \begin{macrocode}
\expandafter\def\csname BitSet@N1073741824\endcsname{%
  \BitSet@GetDecBig2147483648!%
}%
%    \end{macrocode}
%    \end{macro}
%
%    \begin{macro}{\BitSet@GetDecBig}
%    |#1|: current power of two\\
%    |#2|: result\\
%    |#3#4|: number
%    \begin{macrocode}
\def\BitSet@GetDecBig#1!#2!#3#4!{%
  \ifx\\#4\\%
    \ifx#31%
      \BitSet@AfterFiFi{%
        \expandafter\expandafter\expandafter\BitSet@Space
        \BigIntCalcAdd#1!#2!%
      }%
    \else
      \BitSet@AfterFiFi{ #2}%
    \fi
  \else
    \ifx#31%
      \BitSet@AfterFiFi{%
        \expandafter\expandafter\expandafter\BitSet@@GetDecBig
        \BigIntCalcAdd#1!#2!!#1!#4!%
      }%
    \else
      \BitSet@AfterFiFi{%
        \expandafter\expandafter\expandafter\BitSet@GetDecBig
        \BigIntCalcShl#1!!#2!#4!%
      }%
    \fi
  \BitSet@Fi
}
%    \end{macrocode}
%    \end{macro}
%    \begin{macro}{\BitSet@@GetDecBig}
%    |#1|: result\\
%    |#2|: power of two\\
%    |#3#4|: number
%    \begin{macrocode}
\def\BitSet@@GetDecBig#1!#2!{%
  \expandafter\expandafter\expandafter\BitSet@GetDecBig
  \BigIntCalcShl#2!!#1!%
}
%    \end{macrocode}
%    \end{macro}
%
% \subsection{Logical operators}\label{src:log}
%
% \subsubsection{\cs{bitsetAnd}}
%
%    \begin{macro}{\bitsetAnd}
%    Decision table for \cs{bitsetAnd}:
%    \begin{quote}
%    \begin{tabular}{l|l|l|l|}
%                         & undef(B)   & empty(B)   & cardinality(B)$>$0\\
%      \hline
%      undef(A)           & A := empty & A := empty & A := empty\\
%      \hline
%      empty(A)           &            &            &\\
%      \hline
%      cardinality(A)$>$0 & A := empty & A := empty & A \&= B\\
%      \hline
%    \end{tabular}
%    \end{quote}
%    \begin{macrocode}
\def\bitsetAnd#1#2{%
  \bitsetIsEmpty{#1}{%
    \bitsetReset{#1}%
  }{%
    \bitsetIsEmpty{#2}{%
      \bitsetReset{#1}%
    }{%
      \expandafter\edef\csname BS@#1\endcsname{%
        \expandafter\expandafter\expandafter\BitSet@And
        \csname BS@#1\expandafter\expandafter\expandafter\endcsname
        \expandafter\expandafter\expandafter!%
        \csname BS@#2\endcsname!!%
      }%
      \expandafter\ifx\csname BS@#1\endcsname\BitSet@Empty
        \bitsetReset{#1}%
      \fi
    }%
  }%
}
%    \end{macrocode}
%    \end{macro}
%    \begin{macro}{\BitSet@And}
%    \begin{macrocode}
\def\BitSet@And#1#2!#3#4!#5!{%
  \ifx\\#2\\%
    \ifnum#1#3=11 #51\fi
  \else
    \ifx\\#4\\%
      \ifnum#1#3=11 #51\fi
    \else
      \ifnum#1#3=11 %
        #51%
        \BitSet@AfterFiFiFi{%
          \BitSet@And#2!#4!!%
        }%
      \else
        \BitSet@AfterFiFiFi{%
          \BitSet@And#2!#4!#50!%
        }%
      \fi
    \fi
  \BitSet@Fi
}
%    \end{macrocode}
%    \end{macro}
%
% \subsubsection{\cs{bitsetAndNot}}
%
%    \begin{macro}{\bitsetAndNot}
%    Decision table for \cs{bitsetAndNot}:
%    \begin{quote}
%    \begin{tabular}{l|l|l|l|}
%                         & undef(B)   & empty(B)   & cardinality(B)$>$0\\
%      \hline
%      undef(A)           & A := empty & A := empty & A := empty\\
%      \hline
%      empty(A)           &            &            &\\
%      \hline
%      cardinality(A)$>$0 &            &            & A \&= !B\\
%      \hline
%    \end{tabular}
%    \end{quote}
%    \begin{macrocode}
\def\bitsetAndNot#1#2{%
  \bitsetIsEmpty{#1}{%
    \bitsetReset{#1}%
  }{%
    \bitsetIsEmpty{#2}{%
    }{%
      \expandafter\edef\csname BS@#1\endcsname{%
        \expandafter\expandafter\expandafter\BitSet@AndNot
        \csname BS@#1\expandafter\expandafter\expandafter\endcsname
        \expandafter\expandafter\expandafter!%
        \csname BS@#2\endcsname!!%
      }%
      \expandafter\ifx\csname BS@#1\endcsname\BitSet@Empty
        \bitsetReset{#1}%
      \fi
    }%
  }%
}
%    \end{macrocode}
%    \end{macro}
%    \begin{macro}{\BitSet@AndNot}
%    \begin{macrocode}
\def\BitSet@AndNot#1#2!#3#4!#5!{%
  \ifx\\#2\\%
    \ifnum#1#3=10 #51\fi
  \else
    \ifx\\#4\\%
      #5%
      \ifnum#1#3=10 1\else 0\fi
      #2%
    \else
      \ifnum#1#3=10 %
        #51%
        \BitSet@AfterFiFiFi{%
          \BitSet@AndNot#2!#4!!%
        }%
      \else
        \BitSet@AfterFiFiFi{%
          \BitSet@AndNot#2!#4!#50!%
        }%
      \fi
    \fi
  \BitSet@Fi
}
%    \end{macrocode}
%    \end{macro}
%
% \subsubsection{\cs{bitsetOr}}
%
%    \begin{macro}{\bitsetOr}
%    Decision table for \cs{bitsetOr}:
%    \begin{quote}
%    \begin{tabular}{l|l|l|l|}
%                         & undef(B)   & empty(B)   & cardinality(B)$>$0\\
%      \hline
%      undef(A)           & A := empty & A := empty & A := B\\
%      \hline
%      empty(A)           &            &            & A := B\\
%      \hline
%      cardinality(A)$>$0 &            &            & A \string|= B\\
%      \hline
%    \end{tabular}
%    \end{quote}
%    \begin{macrocode}
\def\bitsetOr#1#2{%
  \bitsetIsEmpty{#2}{%
    \BitSet@IfUndefined{#1}{\bitsetReset{#1}}{}%
  }{%
    \bitsetIsEmpty{#1}{%
      \expandafter\let\csname BS@#1\expandafter\endcsname
                      \csname BS@#2\endcsname
    }{%
      \expandafter\edef\csname BS@#1\endcsname{%
        \expandafter\expandafter\expandafter\BitSet@Or
        \csname BS@#1\expandafter\expandafter\expandafter\endcsname
        \expandafter\expandafter\expandafter!%
        \csname BS@#2\endcsname!%
      }%
    }%
  }%
}
%    \end{macrocode}
%    \end{macro}
%    \begin{macro}{\BitSet@Or}
%    \begin{macrocode}
\def\BitSet@Or#1#2!#3#4!{%
  \ifnum#1#3>0 1\else 0\fi
  \ifx\\#2\\%
    #4%
  \else
    \ifx\\#4\\%
      #2%
    \else
      \BitSet@AfterFiFi{%
        \BitSet@Or#2!#4!%
      }%
    \fi
  \BitSet@Fi
}
%    \end{macrocode}
%    \end{macro}
%
% \subsubsection{\cs{bitsetXor}}
%
%    \begin{macro}{\bitsetXor}
%    Decision table for \cs{bitsetXor}:
%    \begin{quote}
%    \begin{tabular}{l|l|l|l|}
%                         & undef(B)   & empty(B)   & cardinality(B)$>$0\\
%      \hline
%      undef(A)           & A := empty & A := empty & A := B\\
%      \hline
%      empty(A)           &            &            & A := B\\
%      \hline
%      cardinality(A)$>$0 &            &            & A \string^= B\\
%      \hline
%    \end{tabular}
%    \end{quote}
%    \begin{macrocode}
\def\bitsetXor#1#2{%
  \bitsetIsEmpty{#2}{%
    \BitSet@IfUndefined{#1}{\bitsetReset{#1}}{}%
  }{%
    \bitsetIsEmpty{#1}{%
      \expandafter\let\csname BS@#1\expandafter\endcsname
                      \csname BS@#2\endcsname
    }{%
      \expandafter\edef\csname BS@#1\endcsname{%
        \expandafter\expandafter\expandafter\BitSet@Xor
        \csname BS@#1\expandafter\expandafter\expandafter\endcsname
        \expandafter\expandafter\expandafter!%
        \csname BS@#2\endcsname!!%
      }%
      \expandafter\ifx\csname BS@#1\endcsname\BitSet@Empty
        \bitsetReset{#1}%
      \fi
    }%
  }%
}
%    \end{macrocode}
%    \end{macro}
%    \begin{macro}{\BitSet@Xor}
%    \begin{macrocode}
\def\BitSet@Xor#1#2!#3#4!#5!{%
  \ifx\\#2\\%
    \ifx#1#3%
      \ifx\\#4\\%
      \else
        #50#4%
      \fi
    \else
      #51#4%
    \fi
  \else
    \ifx\\#4\\%
      #5%
      \ifx#1#30\else 1\fi
      #2%
    \else
      \ifx#1#3%
        \BitSet@AfterFiFiFi{%
          \BitSet@Xor#2!#4!#50!%
        }%
      \else
        #51%
        \BitSet@AfterFiFiFi{%
          \BitSet@Xor#2!#4!!%
        }%
      \fi
    \fi
  \BitSet@Fi
}
%    \end{macrocode}
%    \end{macro}
%
% \subsubsection{Shifting}\label{src:shift}
%
% \subsubsection{\cs{bitsetShiftLeft}}
%
%    \begin{macro}{\bitsetShiftLeft}
%    \begin{macrocode}
\def\bitsetShiftLeft#1#2{%
  \BitSet@IfUndefined{#1}{%
    \bitsetReset{#1}%
  }{%
    \bitsetIsEmpty{#1}{%
    }{%
      \expandafter\expandafter\expandafter\BitSet@ShiftLeft
      \intcalcNum{#2}!{#1}%
    }%
  }%
}
%    \end{macrocode}
%    \end{macro}
%    \begin{macro}{\BitSet@ShiftLeft}
%    \begin{macrocode}
\def\BitSet@ShiftLeft#1!#2{%
  \ifcase\intcalcSgn{#1} %
  \or
    \begingroup
      \uccode`m=`0 %
    \uppercase\expandafter{\expandafter\endgroup
      \expandafter\edef\csname BS@#2\expandafter\endcsname
      \expandafter{%
        \romannumeral#1000\expandafter\BitSet@Space
        \csname BS@#2\endcsname
      }%
    }%
  \else
    \expandafter\BitSet@ShiftRight\BitSet@Gobble#1!{#2}%
  \fi
}
%    \end{macrocode}
%    \end{macro}
%
% \subsubsection{\cs{bitsetShiftRight}}
%
%    \begin{macro}{\bitsetShiftRight}
%    \begin{macrocode}
\def\bitsetShiftRight#1#2{%
  \BitSet@IfUndefined{#1}{%
    \bitsetReset{#1}%
  }{%
    \bitsetIsEmpty{#1}{%
    }{%
      \expandafter\expandafter\expandafter\BitSet@ShiftRight
      \intcalcNum{#2}!{#1}%
    }%
  }%
}
%    \end{macrocode}
%    \end{macro}
%    \begin{macro}{\BitSet@ShiftRight}
%    \begin{macrocode}
\def\BitSet@ShiftRight#1!#2{%
  \ifcase\intcalcSgn{#1} %
  \or
    \expandafter\edef\csname BS@#2\endcsname{%
      \expandafter\expandafter\expandafter\BitSet@Kill
      \csname BS@#2\expandafter\endcsname\expandafter\BitSet@Empty
      \expandafter=%
      \expandafter{\expandafter}\expandafter{\expandafter}%
      \romannumeral#1000!%
    }%
  \else
    \expandafter\BitSet@ShiftLeft\BitSet@Gobble#1!{#2}%
  \fi
}
%    \end{macrocode}
%    \end{macro}
%    \begin{macro}{\BitSet@Kill}
%    \begin{macrocode}
\def\BitSet@Kill#1#2=#3#4#5{%
  #3#4%
  \ifx#5!%
    \ifx#1\BitSet@Empty
      0%
    \else
      #1#2%
    \fi
  \else
    \ifx#1\BitSet@Empty
      0%
      \BitSet@AfterFiFi\BitSet@Cleanup
    \else
      \BitSet@Kill#2=%
    \fi
  \BitSet@Fi
}
%    \end{macrocode}
%    \end{macro}
%
% \subsection{Bit manipulation}\label{src:bitman}
%
%    \begin{macro}{\bitsetClear}
%    \begin{macrocode}
\def\bitsetClear{%
  \BitSet@CheckIndex\BitSet@Clear
}
%    \end{macrocode}
%    \end{macro}
%    \begin{macro}{\bitsetSet}
%    \begin{macrocode}
\def\bitsetSet{%
  \BitSet@CheckIndex\BitSet@Set
}
%    \end{macrocode}
%    \end{macro}
%    \begin{macro}{\bitsetFlip}
%    \begin{macrocode}
\def\bitsetFlip{%
  \BitSet@CheckIndex\BitSet@Flip
}
%    \end{macrocode}
%    \end{macro}
%    \begin{macro}{\bitsetSetValue}
%    \begin{macrocode}
\def\bitsetSetValue#1#2#3{%
  \expandafter\expandafter\expandafter\BitSet@SetValue
  \intcalcNum{#3}!{#1}{#2}%
}
%    \end{macrocode}
%    \end{macro}
%    \begin{macro}{\BitSet@SetValue}
%    |#1|: plain value\\
%    |#2|: BitSet\\
%    |#3|: Index
%    \begin{macrocode}
\def\BitSet@SetValue#1!{%
  \BitSet@CheckIndex{%
    \ifcase#1 %
      \expandafter\BitSet@Clear
    \or
      \expandafter\BitSet@Set
    \else
      \BitSet@ErrorInvalidBitValue{#1}%
      \expandafter\expandafter\expandafter\BitSet@Gobble
      \expandafter\BitSet@Gobble
    \fi
  }%
}
%    \end{macrocode}
%    \end{macro}
%    \begin{macro}{\BitSet@ErrorInvalidBitValue}
%    |#1|: Wrong bit value
%    \begin{macrocode}
\def\BitSet@ErrorInvalidBitValue#1{%
  \@PackageError{bitset}{%
    Invalid bit value (#1) not in range 0..1%
  }\@ehc
}
%    \end{macrocode}
%    \end{macro}
%
% \subsubsection{Clear operation}
%
%    \begin{macro}{\BitSet@Clear}
%    |#1|: BitSet\\
%    |#2|: plain and checked index
%    \begin{macrocode}
\def\BitSet@Clear#1#2{%
  \edef\BitSet@Temp{%
    \expandafter\expandafter\expandafter\BitSet@@Clear
    \csname BS@#1\expandafter\endcsname
    \expandafter\BitSet@Empty\expandafter=\expandafter!%
    \romannumeral#2000!%
  }%
  \expandafter\let\csname BS@#1\expandafter\endcsname
  \ifx\BitSet@Temp\BitSet@Empty
    \BitSet@Zero
  \else
    \BitSet@Temp
  \fi
}
%    \end{macrocode}
%    \end{macro}
%    \begin{macro}{\BitSet@@Clear}
%    \begin{macrocode}
\def\BitSet@@Clear#1#2=#3!#4{%
  \ifx#4!%
    \ifx#1\BitSet@Empty
    \else
      \ifx\BitSet@Empty#2%
      \else
        #30#2%
      \fi
    \fi
  \else
    \ifx#1\BitSet@Empty
      \BitSet@AfterFiFi\BitSet@Cleanup
    \else
      \ifx#10%
        \BitSet@AfterFiFiFi{%
          \BitSet@@Clear#2=#30!%
        }%
      \else
        #31%
        \BitSet@AfterFiFiFi{%
          \BitSet@@Clear#2=!%
        }%
      \fi
    \fi
  \BitSet@Fi
}
%    \end{macrocode}
%    \end{macro}
%
% \subsubsection{Set operation}
%
%    \begin{macro}{\BitSet@Set}
%    |#1|: BitSet\\
%    |#2|: plain and checked Index
%    \begin{macrocode}
\def\BitSet@Set#1#2{%
  \expandafter\edef\csname BS@#1\endcsname{%
    \expandafter\expandafter\expandafter\BitSet@@Set
    \csname BS@#1\expandafter\endcsname
    \expandafter\BitSet@Empty\expandafter=%
    \expandafter{\expandafter}\expandafter{\expandafter}%
    \romannumeral#2000!%
  }%
}
%    \end{macrocode}
%    \end{macro}
%    \begin{macro}{\BitSet@@Set}
%    \begin{macrocode}
\def\BitSet@@Set#1#2=#3#4#5{%
  #3#4%
  \ifx#5!%
    1#2%
  \else
    \ifx#1\BitSet@Empty
      0%
      \BitSet@AfterFiFi\BitSet@@@Set
    \else
      #1%
      \BitSet@@Set#2=%
    \fi
  \BitSet@Fi
}
%    \end{macrocode}
%    \end{macro}
%    \begin{macro}{\BitSet@@@Set}
%    \begin{macrocode}
\def\BitSet@@@Set#1{%
  \ifx#1!%
    1%
  \else
    0%
    \expandafter\BitSet@@@Set
  \fi
}
%    \end{macrocode}
%    \end{macro}
%
% \subsubsection{Flip operation}
%
%    \begin{macro}{\BitSet@Flip}
%    |#1|: BitSet\\
%    |#2|: plain and checked Index
%    \begin{macrocode}
\def\BitSet@Flip#1#2{%
  \edef\BitSet@Temp{%
    \expandafter\expandafter\expandafter\BitSet@@Flip
    \csname BS@#1\expandafter\endcsname
    \expandafter\BitSet@Empty\expandafter=\expandafter!%
    \romannumeral#2000!%
  }%
  \expandafter\let\csname BS@#1\expandafter\endcsname
  \ifx\BitSet@Temp\BitSet@Empty
    \BitSet@Zero
  \else
    \BitSet@Temp
  \fi
}
%    \end{macrocode}
%    \end{macro}
%    \begin{macro}{\BitSet@@Flip}
%    \begin{macrocode}
\def\BitSet@@Flip#1#2=#3!#4{%
  \ifx#4!%
    \ifx#11%
      \ifx\BitSet@Empty#2%
      \else
        #30#2%
      \fi
    \else
      #31#2%
    \fi
  \else
    \ifx#1\BitSet@Empty
      #30%
      \BitSet@AfterFiFi\BitSet@@@Set
    \else
      \ifx#10%
        \BitSet@AfterFiFiFi{%
          \BitSet@@Flip#2=#30!%
        }%
      \else
        #31%
        \BitSet@AfterFiFiFi{%
          \BitSet@@Flip#2=!%
        }%
      \fi
    \fi
  \BitSet@Fi
}
%    \end{macrocode}
%    \end{macro}
%
% \subsubsection{Range operators}
%
%    \begin{macro}{\bitsetClearRange}
%    \begin{macrocode}
\def\bitsetClearRange{%
  \BitSet@Range\BitSet@Clear
}
%    \end{macrocode}
%    \end{macro}
%    \begin{macro}{\bitsetSetRange}
%    \begin{macrocode}
\def\bitsetSetRange{%
  \BitSet@Range\BitSet@Set
}
%    \end{macrocode}
%    \end{macro}
%    \begin{macro}{\bitsetFlipRange}
%    \begin{macrocode}
\def\bitsetFlipRange{%
  \BitSet@Range\BitSet@Flip
}
%    \end{macrocode}
%    \end{macro}
%    \begin{macro}{\bitsetSetValueRange}
%    \begin{macrocode}
\def\bitsetSetValueRange#1#2#3#4{%
  \expandafter\expandafter\expandafter\BitSet@SetValueRange
  \intcalcNum{#4}!{#1}{#2}{#3}%
}
%    \end{macrocode}
%    \end{macro}
%    \begin{macro}{\BitSet@SetValueRange}
%    \begin{macrocode}
\def\BitSet@SetValueRange#1!#2#3#4{%
  \ifcase#1 %
    \BitSet@Range\BitSet@Clear{#2}{#3}{#4}%
  \or
    \BitSet@Range\BitSet@Set{#2}{#3}{#4}%
  \else
    \BitSet@ErrorInvalidBitValue{#1}%
  \fi
}
%    \end{macrocode}
%    \end{macro}
%
%    \begin{macro}{\BitSet@Range}
%    |#1|: clear/set/flip macro\\
%    |#2|: BitSet\\
%    |#3|: Index from\\
%    |#4|: Index to
%    \begin{macrocode}
\def\BitSet@Range#1#2#3#4{%
  \edef\BitSet@Temp{%
    \noexpand\BitSet@@Range\noexpand#1{#2}%
    \intcalcNum{#3}!\intcalcNum{#4}!%
  }%
  \BitSet@Temp
}
%    \end{macrocode}
%    \end{macro}
%    \begin{macro}{\BitSet@@Range}
%    |#1|: clear/set/flip macro\\
%    |#2|: BitSet\\
%    |#3|: Index from\\
%    |#4|: Index to
%    \begin{macrocode}
\def\BitSet@@Range#1#2#3!#4!{%
  \ifnum#3<0 %
    \BitSet@NegativeIndex#1{#2}#3!#4!0!#4!%
  \else
    \ifnum#4<0 %
      \BitSet@NegativeIndex#1{#2}#3!#4!#3!0!%
    \else
      \ifcase\intcalcCmp{#3}{#4} %
      \or
        \@PackageError{bitset}{%
          Wrong index numbers in range [#3..#4]\MessageBreak% hash-ok
          for clear/set/flip on bit set `#2'.\MessageBreak
          The lower index exceeds the upper index.\MessageBreak
          Canceling the operation as error recovery%
        }\@ehc
      \else
        \BitSet@@@Range#3!#4!#1{#2}%
      \fi
    \fi
  \fi
}
%    \end{macrocode}
%    \end{macro}
%    \begin{macro}{\BitSet@NegativeIndex}
%    \begin{macrocode}
\def\BitSet@NegativeIndex#1#2#3!#4!#5!#6!{%
  \@PackageError{bitset}{%
    Negative index in range [#3..#4]\MessageBreak % hash-ok
    for \string\bitset
    \ifx#1\BitSet@Clear
      Clear%
    \else
      \ifx#1\BitSet@Set
        Set%
      \else
        Flip%
      \fi
    \fi
    Range on bit set `#2'.\MessageBreak
    Using [#5..#6] as error recovery% hash-ok
  }\@ehc
  \BitSet@@Range#1{#2}#5!#6!%
}
%    \end{macrocode}
%    \end{macro}
%    \begin{macro}{\BitSet@@Range}
%    \begin{macrocode}
\def\BitSet@@@Range#1!#2!#3#4{%
  \ifnum#1<#2 %
    #3{#4}{#1}%
    \BitSet@AfterFi{%
      \expandafter\expandafter\expandafter\BitSet@@@Range
      \IntCalcInc#1!!#2!#3{#4}%
    }%
  \BitSet@Fi
}
%    \end{macrocode}
%    \end{macro}
%
% \subsection{Bit retrieval}\label{src:bitret}
%
% \subsubsection{\cs{bitsetGet}}
%
%    \begin{macro}{\bitsetGet}
%    \begin{macrocode}
\def\bitsetGet#1#2{%
  \number
  \expandafter\expandafter\expandafter\BitSet@Get
  \intcalcNum{#2}!{#1}%
}
%    \end{macrocode}
%    \end{macro}
%    \begin{macro}{\BitSet@Get}
%    |#1|: plain index\\
%    |#2|: BitSet
%    \begin{macrocode}
\def\BitSet@Get#1!#2{%
  \ifnum#1<0 %
    \BitSet@AfterFi{%
      0 \BitSetError:NegativeIndex%
    }%
  \else
    \BitSet@IfUndefined{#2}{0}{%
      \expandafter\expandafter\expandafter\BitSet@@Get
      \csname BS@#2\expandafter\endcsname
      \expandafter!\expandafter=%
      \expandafter{\expandafter}\expandafter{\expandafter}%
      \romannumeral\intcalcNum{#1}000!%
    }%
    \expandafter\BitSet@Space
  \BitSet@Fi
}
%    \end{macrocode}
%    \end{macro}
%    \begin{macro}{\BitSet@@Get}
%    \begin{macrocode}
\def\BitSet@@Get#1#2=#3#4#5{%
  #3#4%
  \ifx#5!%
    \ifx#1!%
      0%
    \else
      #1%
    \fi
  \else
    \ifx#1!%
      0%
      \BitSet@AfterFiFi\BitSet@Cleanup
    \else
      \BitSet@@Get#2=%
    \fi
  \BitSet@Fi
}
%    \end{macrocode}
%    \end{macro}
%
% \subsubsection{\cs{bitsetNextClearBit}, \cs{bitsetNextSetBit}}
%
%    \begin{macro}{\bitsetNextClearBit}
%    \begin{macrocode}
\def\bitsetNextClearBit#1#2{%
  \number
  \expandafter\expandafter\expandafter\BitSet@NextClearBit
  \intcalcNum{#2}!{#1} %
}
%    \end{macrocode}
%    \end{macro}
%    \begin{macro}{\BitSet@NextClearBit}
%    |#1|: Index\\
%    |#2|: BitSet
%    \begin{macrocode}
\def\BitSet@NextClearBit#1!#2{%
  \ifnum#1<0 %
    \BitSet@NextClearBit0!{#2}%
    \BitSet@AfterFi{%
      \expandafter\BitSet@Space
      \expandafter\BitSetError:NegativeIndex\romannumeral0%
    }%
  \else
    \bitsetIsEmpty{#2}{#1}{%
      \expandafter\BitSet@Skip
      \number#1\expandafter\expandafter\expandafter!%
      \csname BS@#2\endcsname!!!!!!!!!=%
      {\BitSet@@NextClearBit#1!}%
    }%
  \BitSet@Fi
}
%    \end{macrocode}
%    \end{macro}
%    \begin{macro}{\BitSet@@NextClearBit}
%    |#1|: index for next bit in |#2|\\
%    |#2|: next bit
%    \begin{macrocode}
\def\BitSet@@NextClearBit#1!#2{%
  \ifx#2!%
    #1%
  \else
    \ifx#20%
      #1%
      \BitSet@AfterFiFi\BitSet@Cleanup
    \else
      \BitSet@AfterFiFi{%
        \expandafter\expandafter\expandafter\BitSet@@NextClearBit
        \IntCalcInc#1!!%
      }%
    \fi
  \BitSet@Fi
}
%    \end{macrocode}
%    \end{macro}
%
%    \begin{macro}{\bitsetNextSetBit}
%    \begin{macrocode}
\def\bitsetNextSetBit#1#2{%
  \number
  \expandafter\expandafter\expandafter\BitSet@NextSetBit
  \intcalcNum{#2}!{#1} %
}
%    \end{macrocode}
%    \end{macro}
%    \begin{macro}{\BitSet@NextSetBit}
%    |#1|: Index\\
%    |#2|: BitSet
%    \begin{macrocode}
\def\BitSet@NextSetBit#1!#2{%
  \ifnum#1<0 %
    \BitSet@NextSetBit0!{#2}%
    \BitSet@AfterFi{%
      \expandafter\BitSet@Space
      \expandafter\BitSetError:NegativeIndex\romannumeral0%
    }%
  \else
    \bitsetIsEmpty{#2}{-1}{%
      \expandafter\BitSet@Skip
      \number#1\expandafter\expandafter\expandafter!%
      \csname BS@#2\endcsname!!!!!!!!!=%
      {\BitSet@@NextSetBit#1!}%
    }%
  \BitSet@Fi
}
%    \end{macrocode}
%    \end{macro}
%    \begin{macro}{\BitSet@@NextSetBit}
%    |#1|: index for next bit in |#2|\\
%    |#2|: next bit
%    \begin{macrocode}
\def\BitSet@@NextSetBit#1!#2{%
  \ifx#2!%
    -1%
  \else
    \ifx#21%
      #1%
      \BitSet@AfterFiFi\BitSet@Cleanup
    \else
      \BitSet@AfterFiFi{%
        \expandafter\expandafter\expandafter\BitSet@@NextSetBit
        \IntCalcInc#1!!%
      }%
    \fi
  \BitSet@Fi
}
%    \end{macrocode}
%    \end{macro}
%    \begin{macro}{\BitSet@Cleanup}
%    \begin{macrocode}
\def\BitSet@Cleanup#1!{}
%    \end{macrocode}
%    \end{macro}
%
%    \begin{macro}{\BitSet@Skip}
%    |#1|: number of bits to skip\\
%    |#2|: bits\\
%    |#3|: continuation code
%    \begin{macrocode}
\def\BitSet@Skip#1!#2{%
  \ifx#2!%
    \BitSet@AfterFi{%
      \BitSet@SkipContinue%
    }%
  \else
    \ifcase#1 %
      \BitSet@AfterFiFi{%
        \BitSet@SkipContinue#2%
      }%
    \or
      \BitSet@AfterFiFi\BitSet@SkipContinue
    \or
      \BitSet@AfterFiFi{%
        \expandafter\BitSet@SkipContinue\BitSet@Gobble
      }%
    \else
      \ifnum#1>8 %
        \BitSet@AfterFiFiFi{%
          \expandafter\BitSet@Skip
          \number\IntCalcSub#1!8!\expandafter!%
          \BitSet@GobbleSeven
        }%
      \else
        \BitSet@AfterFiFiFi{%
          \expandafter\expandafter\expandafter\BitSet@Skip
          \IntCalcDec#1!!%
        }%
      \fi
    \fi
  \BitSet@Fi
}
%    \end{macrocode}
%    \end{macro}
%    \begin{macro}{\BitSet@SkipContinue}
%    |#1|: remaining bits\\
%    |#2|: continuation code
%    \begin{macrocode}
\def\BitSet@SkipContinue#1!#2=#3{%
  #3#1!%
}
%    \end{macrocode}
%    \end{macro}
%    \begin{macro}{\BitSet@GobbleSeven}
%    \begin{macrocode}
\def\BitSet@GobbleSeven#1#2#3#4#5#6#7{}
%    \end{macrocode}
%    \end{macro}
%
% \subsubsection{\cs{bitsetGetSetBitList}}
%
%    \begin{macro}{\bitsetGetSetBitList}
%    It's just a wrapper for \cs{bitsetNextSetBit}.
%    \begin{macrocode}
\def\bitsetGetSetBitList#1{%
  \romannumeral0%
  \bitsetIsEmpty{#1}{ }{%
    \expandafter\BitSet@GetSetBitList
    \number\BitSet@NextSetBit0!{#1}!{#1}{}!%
  }%
}
%    \end{macrocode}
%    \end{macro}
%    \begin{macro}{\BitSet@GetSetBitList}
%    |#1|: found index\\
%    |#2|: BitSet\\
%    |#3|: comma
%    |#4|: result
%    \begin{macrocode}
\def\BitSet@GetSetBitList#1!#2#3#4!{%
  \ifnum#1<0 %
    \BitSet@AfterFi{ #4}%
  \else
    \BitSet@AfterFi{%
      \expandafter\BitSet@GetSetBitList\number
      \expandafter\expandafter\expandafter\BitSet@NextSetBit
      \IntCalcInc#1!!{#2}!{#2},#4#3#1!%
    }%
  \BitSet@Fi
}
%    \end{macrocode}
%    \end{macro}
%
% \subsection{Bit set properties}\label{src:prop}
%
%    \begin{macro}{\bitsetSize}
%    \begin{macrocode}
\def\bitsetSize#1{%
  \number
  \BitSet@IfUndefined{#1}{0 }{%
    \expandafter\expandafter\expandafter\BitSet@Size
    \expandafter\expandafter\expandafter1%
    \expandafter\expandafter\expandafter!%
    \csname BS@#1\endcsname!0!%
  }%
}
%    \end{macrocode}
%    \end{macro}
%    \begin{macro}{\BitSet@Size}
%    |#1|: counter\\
%    |#2#3|: bits\\
%    |#4|: result
%    \begin{macrocode}
\def\BitSet@Size#1!#2#3!#4!{%
  \ifx#21%
    \ifx\\#3\\%
      \BitSet@AfterFiFi{#1 }%
    \else
      \BitSet@AfterFiFi{%
        \expandafter\expandafter\expandafter\BitSet@Size
        \IntCalcInc#1!!#3!#1!%
      }%
    \fi
  \else
    \ifx\\#3\\%
      \BitSet@AfterFiFi{#4 }%
    \else
      \BitSet@AfterFiFi{%
        \expandafter\expandafter\expandafter\BitSet@Size
        \IntCalcInc#1!!#3!#4!%
      }%
    \fi
  \fi
  \BitSet@Fi
}
%    \end{macrocode}
%    \end{macro}
%
%    \begin{macro}{\bitsetCardinality}
%    \begin{macrocode}
\def\bitsetCardinality#1{%
  \number
  \BitSet@IfUndefined{#1}{0 }{%
    \expandafter\expandafter\expandafter\BitSet@Cardinality
    \expandafter\expandafter\expandafter0%
    \expandafter\expandafter\expandafter!%
    \csname BS@#1\endcsname!%
  }%
}
%    \end{macrocode}
%    \end{macro}
%    \begin{macro}{\BitSet@Cardinality}
%    |#1|: result\\
%    |#2#3|: bits
%    \begin{macrocode}
\def\BitSet@Cardinality#1!#2#3!{%
  \ifx#21%
    \ifx\\#3\\%
      \BitSet@AfterFiFi{\IntCalcInc#1! }%
    \else
      \BitSet@AfterFiFi{%
        \expandafter\expandafter\expandafter\BitSet@Cardinality
        \IntCalcInc#1!!#3!%
      }%
    \fi
  \else
    \ifx\\#3\\%
      \BitSet@AfterFiFi{#1 }%
    \else
      \BitSet@AfterFiFi{%
        \BitSet@Cardinality#1!#3!%
      }%
    \fi
  \fi
  \BitSet@Fi
}
%    \end{macrocode}
%    \end{macro}
%
% \subsection{Queries}\label{src:query}
%
%    \begin{macro}{\bitsetIsDefined}
%    \begin{macrocode}
\def\bitsetIsDefined#1{%
  \BitSet@IfUndefined{#1}%
  \BitSet@SecondOfTwo
  \BitSet@FirstOfTwo
}
%    \end{macrocode}
%    \end{macro}
%    \begin{macro}{\bitsetIsEmpty}
%    \begin{macrocode}
\def\bitsetIsEmpty#1{%
  \BitSet@IfUndefined{#1}\BitSet@FirstOfTwo{%
    \expandafter\ifx\csname BS@#1\endcsname\BitSet@Zero
      \expandafter\BitSet@FirstOfTwo
    \else
      \expandafter\BitSet@SecondOfTwo
    \fi
  }%
}
%    \end{macrocode}
%    \end{macro}
%    \begin{macro}{\BitSet@Zero}
%    \begin{macrocode}
\def\BitSet@Zero{0}
%    \end{macrocode}
%    \end{macro}
%
%    \begin{macro}{\bitsetQuery}
%    \begin{macrocode}
\def\bitsetQuery#1#2{%
  \ifnum\bitsetGet{#1}{#2}=1 %
    \expandafter\BitSet@FirstOfTwo
  \else
    \expandafter\BitSet@SecondOfTwo
  \fi
}
%    \end{macrocode}
%    \end{macro}
%
%    \begin{macro}{\bitsetEquals}
%    \begin{macrocode}
\def\bitsetEquals#1#2{%
  \BitSet@IfUndefined{#1}{%
    \BitSet@IfUndefined{#2}\BitSet@FirstOfTwo\BitSet@SecondOfTwo
  }{%
    \BitSet@IfUndefined{#2}\BitSet@SecondOfTwo{%
      \expandafter\ifx\csname BS@#1\expandafter\endcsname
                      \csname BS@#2\endcsname
        \expandafter\BitSet@FirstOfTwo
      \else
        \expandafter\BitSet@SecondOfTwo
      \fi
    }%
  }%
}
%    \end{macrocode}
%    \end{macro}
%
%    \begin{macro}{\bitsetIntersects}
%    \begin{macrocode}
\def\bitsetIntersects#1#2{%
  \bitsetIsEmpty{#1}\BitSet@SecondOfTwo{%
    \bitsetIsEmpty{#2}\BitSet@SecondOfTwo{%
      \expandafter\expandafter\expandafter\BitSet@Intersects
      \csname BS@#1\expandafter\expandafter\expandafter\endcsname
      \expandafter\expandafter\expandafter!%
      \csname BS@#2\endcsname!%
    }%
  }%
}
%    \end{macrocode}
%    \end{macro}
%    \begin{macro}{\BitSet@Intersects}
%    \begin{macrocode}
\def\BitSet@Intersects#1#2!#3#4!{%
  \ifnum#1#3=11 %
    \BitSet@AfterFi\BitSet@FirstOfTwo
  \else
    \ifx\\#2\\%
      \BitSet@AfterFiFi\BitSet@SecondOfTwo
    \else
      \ifx\\#4\\%
        \BitSet@AfterFiFiFi\BitSet@SecondOfTwo
      \else
        \BitSet@AfterFiFiFi{%
          \BitSet@Intersects#2!#4!%
        }%
      \fi
    \fi
  \BitSet@Fi
}
%    \end{macrocode}
%    \end{macro}
%
%    \begin{macrocode}
\BitSet@AtEnd%
%</package>
%    \end{macrocode}
%
% \section{Test}
%
% \subsection{Catcode checks for loading}
%
%    \begin{macrocode}
%<*test1>
%    \end{macrocode}
%    \begin{macrocode}
\catcode`\{=1 %
\catcode`\}=2 %
\catcode`\#=6 %
\catcode`\@=11 %
\expandafter\ifx\csname count@\endcsname\relax
  \countdef\count@=255 %
\fi
\expandafter\ifx\csname @gobble\endcsname\relax
  \long\def\@gobble#1{}%
\fi
\expandafter\ifx\csname @firstofone\endcsname\relax
  \long\def\@firstofone#1{#1}%
\fi
\expandafter\ifx\csname loop\endcsname\relax
  \expandafter\@firstofone
\else
  \expandafter\@gobble
\fi
{%
  \def\loop#1\repeat{%
    \def\body{#1}%
    \iterate
  }%
  \def\iterate{%
    \body
      \let\next\iterate
    \else
      \let\next\relax
    \fi
    \next
  }%
  \let\repeat=\fi
}%
\def\RestoreCatcodes{}
\count@=0 %
\loop
  \edef\RestoreCatcodes{%
    \RestoreCatcodes
    \catcode\the\count@=\the\catcode\count@\relax
  }%
\ifnum\count@<255 %
  \advance\count@ 1 %
\repeat

\def\RangeCatcodeInvalid#1#2{%
  \count@=#1\relax
  \loop
    \catcode\count@=15 %
  \ifnum\count@<#2\relax
    \advance\count@ 1 %
  \repeat
}
\def\RangeCatcodeCheck#1#2#3{%
  \count@=#1\relax
  \loop
    \ifnum#3=\catcode\count@
    \else
      \errmessage{%
        Character \the\count@\space
        with wrong catcode \the\catcode\count@\space
        instead of \number#3%
      }%
    \fi
  \ifnum\count@<#2\relax
    \advance\count@ 1 %
  \repeat
}
\def\space{ }
\expandafter\ifx\csname LoadCommand\endcsname\relax
  \def\LoadCommand{\input bitset.sty\relax}%
\fi
\def\Test{%
  \RangeCatcodeInvalid{0}{47}%
  \RangeCatcodeInvalid{58}{64}%
  \RangeCatcodeInvalid{91}{96}%
  \RangeCatcodeInvalid{123}{255}%
  \catcode`\@=12 %
  \catcode`\\=0 %
  \catcode`\%=14 %
  \LoadCommand
  \RangeCatcodeCheck{0}{36}{15}%
  \RangeCatcodeCheck{37}{37}{14}%
  \RangeCatcodeCheck{38}{47}{15}%
  \RangeCatcodeCheck{48}{57}{12}%
  \RangeCatcodeCheck{58}{63}{15}%
  \RangeCatcodeCheck{64}{64}{12}%
  \RangeCatcodeCheck{65}{90}{11}%
  \RangeCatcodeCheck{91}{91}{15}%
  \RangeCatcodeCheck{92}{92}{0}%
  \RangeCatcodeCheck{93}{96}{15}%
  \RangeCatcodeCheck{97}{122}{11}%
  \RangeCatcodeCheck{123}{255}{15}%
  \RestoreCatcodes
}
\Test
\csname @@end\endcsname
\end
%    \end{macrocode}
%    \begin{macrocode}
%</test1>
%    \end{macrocode}
%
% \subsection{Macro tests}
%
% \subsubsection{Preamble}
%
%    \begin{macrocode}
%<*test2>
\NeedsTeXFormat{LaTeX2e}
\nofiles
\documentclass{article}
\makeatletter
%<*noetex>
\let\SavedNumexpr\numexpr
\let\SavedIfcsname\ifcsname
\let\SavedCurrentgrouplevel\currentgrouplevel
\def\ETeXDisable{%
  \let\ifcsname\@undefined
  \let\numexpr\@undefined
  \let\currentgrouplevel\@undefined
}
\ETeXDisable
%</noetex>
\makeatletter
\chardef\BitSet@TestMode=1 %
\makeatother
\usepackage{bitset}[2011/01/30]
%<*noetex>
\def\ETeXEnable{%
  \let\numexpr\SavedNumexpr
  \let\ifcsname\SavedIfcsname
  \let\currentgrouplevel\SavedCurrentgrouplevel
}
\ETeXEnable
%</noetex>
\usepackage{qstest}
\IncludeTests{*}
\LogTests{log}{*}{*}
\makeatletter
%    \end{macrocode}
%
% \subsubsection{Time}
%
%    \begin{macrocode}
\begingroup\expandafter\expandafter\expandafter\endgroup
\expandafter\ifx\csname pdfresettimer\endcsname\relax
\else
  \newcount\SummaryTime
  \newcount\TestTime
  \SummaryTime=\z@
  \newcommand*{\PrintTime}[2]{%
    \typeout{%
      [Time #1: \strip@pt\dimexpr\number#2sp\relax\space s]%
    }%
  }%
  \newcommand*{\StartTime}[1]{%
    \renewcommand*{\TimeDescription}{#1}%
    \pdfresettimer
  }%
  \newcommand*{\TimeDescription}{}%
  \newcommand*{\StopTime}{%
    \TestTime=\pdfelapsedtime
    \global\advance\SummaryTime\TestTime
    \PrintTime\TimeDescription\TestTime
  }%
  \let\saved@qstest\qstest
  \let\saved@endqstest\endqstest
  \def\qstest#1#2{%
    \saved@qstest{#1}{#2}%
    \StartTime{#1}%
  }%
  \def\endqstest{%
    \StopTime
    \saved@endqstest
  }%
  \AtEndDocument{%
    \PrintTime{summary}\SummaryTime
  }%
\fi
%    \end{macrocode}
%
% \subsubsection{Detection of unwanted space}
%
%    \begin{macrocode}
\let\orig@qstest\qstest
\let\orig@endqstest\endqstest
\def\qstest#1#2{%
  \orig@qstest{#1}{#2}%
  \setbox0\hbox\bgroup\begingroup\ignorespaces
}
\def\endqstest{%
  \endgroup\egroup
  \Expect*{\the\wd0}{0.0pt}%
  \orig@endqstest
}
%    \end{macrocode}
%
% \subsubsection{Test macros}
%
%    \begin{macrocode}
\newcounter{Test}

\def\TestError#1#2{%
  \begingroup
    \setcounter{Test}{0}%
    \sbox0{%
      \def\@PackageError##1##2##3{%
        \stepcounter{Test}%
        \begingroup
          \let\MessageBreak\relax
%<*noetex>
          \ETeXEnable
%</noetex>
          \Expect{##1}{bitset}%
          \Expect*{##2}*{#1}%
        \endgroup
      }%
%<*noetex>
      \ETeXDisable
%</noetex>
      #2%
    }%
    \Expect*{\theTest}{1}%
    \Expect*{\the\wd0}{0.0pt}%
  \endgroup
}

\def\TestErrorNegativeIndex#1#2{%
  \TestError{Invalid negative index (#1)}{#2}%
}

\def\TestGetterUndefined#1{%
  \CheckUndef{dummy}%
  \expandafter\expandafter\expandafter\Expect
  \expandafter\expandafter\expandafter{#1{dummy}}{0}%
}

\def\ExpectBitSet#1#2{%
  \expandafter\expandafter\expandafter\Expect
  \expandafter\expandafter\expandafter
  {\csname BS@#1\endcsname}*{#2}%
}
\def\Check#1#2{%
  \ExpectBitSet{#1}{#2}%
}
\def\CheckUndef#1{%
  \begingroup
    \Expect*{%
      \expandafter
      \ifx\csname BS@#1\endcsname\relax true\else false\fi
    }{true}%
  \endgroup
}
\def\RevCheck#1#2{%
  \ExpectBitSet{#1}{\Reverse#2!!}%
}
\def\Set#1#2{%
  \expandafter\def\csname BS@#1\endcsname{#2}%
}
\def\RevSet#1#2{%
  \expandafter\edef\csname BS@#1\endcsname{%
    \Reverse#2!!%
  }%
}
\def\Reverse#1#2!#3!{%
  \ifx\\#2\\%
    #1#3%
    \expandafter\@gobble
  \else
    \expandafter\@firstofone
  \fi
  {\Reverse#2!#1#3!}%
}
%    \end{macrocode}
%
% \subsubsection{Test sets}
%
%    \begin{macrocode}
\begin{qstest}{Let}{Let}
  \CheckUndef{abc}%
  \CheckUndef{xyz}%
  \bitsetLet{xyz}{abc}%
  \CheckUndef{abc}%
  \Check{xyz}{0}%
  \Set{abc}{1}%
  \Check{abc}{1}%
  \Check{xyz}{0}%
  \bitsetLet{xyz}{abc}%
  \Check{abc}{1}%
  \Check{xyz}{1}%
  \Set{xyz}{11}%
  \Check{abc}{1}%
  \Check{xyz}{11}%
\end{qstest}

\begin{qstest}{Reset}{Reset}
  \bitsetReset{xyz}%
  \Check{xyz}{0}%
  \bitsetReset{abc}%
  \Check{abc}{0}%
  \Set{abc}{10101}%
  \bitsetReset{abc}%
  \Check{abc}{0}%
\end{qstest}

\begin{qstest}{Get/Query}{Get/Query}
  \expandafter\expandafter\expandafter\Expect
  \expandafter\expandafter\expandafter{%
    \bitsetGet{dummy}{0}%
  }{0}%
  \begingroup
    \expandafter\def\csname BitSetError:NegativeIndex\endcsname{}%
    \Set{abc}{1}%
    \Expect*{\bitsetQuery{abc}{-1}{true}{false}}{false}%
  \endgroup
  \def\Test#1#2#3{%
    \Set{abc}{#1}%
    \expandafter\expandafter\expandafter\Expect
    \expandafter\expandafter\expandafter{\bitsetGet{abc}{#2}}{#3}%
    \Expect*{\bitsetQuery{abc}{#2}{true}{false}}%
           *{\ifcase#3 false\or true\else error\fi}%
  }%
  \Test{1}{100}{0}%
  \Test{0}{0}{0}%
  \Test{1}{0}{1}%
  \Test{11}{1}{1}%
  \Test{111}{1}{1}%
  \Test{101}{1}{0}%
  \Test{101}{2}{1}%
  \Test{10100110011}{10}{1}%
\end{qstest}

\begin{qstest}{Size}{Size}
  \TestGetterUndefined\bitsetSize
  \def\Test#1#2{%
    \Set{abc}{#1}%
    \expandafter\expandafter\expandafter\Expect
    \expandafter\expandafter\expandafter{\bitsetSize{abc}}{#2}%
  }%
  \Test{0}{0}%
  \Test{1}{1}%
  \Test{00}{0}%
  \Test{0000000}{0}%
  \Test{10}{1}%
  \Test{01}{2}%
  \Test{11}{2}%
  \Test{010}{2}%
  \Test{011}{3}%
  \Test{100110011}{9}%
  \Test{0000011111000001111100000}{20}%
  \Test{000000000000000000000000011111111111111111111}{45}%
\end{qstest}

\begin{qstest}{Cardinality}{Cardinality}
  \TestGetterUndefined\bitsetCardinality
  \def\Test#1#2{%
    \Set{abc}{#1}%
    \expandafter\expandafter\expandafter\Expect
    \expandafter\expandafter\expandafter{%
      \bitsetCardinality{abc}%
    }{#2}%
  }%
  \Test{0}{0}%
  \Test{1}{1}%
  \Test{00}{0}%
  \Test{0000000}{0}%
  \Test{10}{1}%
  \Test{01}{1}%
  \Test{11}{2}%
  \Test{010}{1}%
  \Test{011}{2}%
  \Test{100110011}{5}%
  \Test{0000011111000001111100000}{10}%
  \Test{000000000000000000000000011111111111111111111}{20}%
\end{qstest}

\begin{qstest}{NextClearBit/NextSetBit}{NextClearBit/NextSetBit}
  \def\Test#1#2{%
    \expandafter\expandafter\expandafter\Expect
    \expandafter\expandafter\expandafter{%
      \TestOp{abc}{#1}%
    }{#2}%
  }%
  \def\Clear{\let\TestOp\bitsetNextClearBit}%
  \def\Set{\let\TestOp\bitsetNextSetBit}%
  \begingroup
    \catcode`\:=11 %
    \bitsetSetBin{abc}{1}%
    \Clear
    \Test{-1}{1\BitSetError:NegativeIndex}%
    \Set
    \Test{-1}{0\BitSetError:NegativeIndex}%
  \endgroup
  \let\BS@abc\@undefined
  \Clear
  \Test{0}{0}%
  \Test{1}{1}%
  \Test{2}{2}%
  \Test{100}{100}%
  \Set
  \Test{0}{-1}%
  \Test{1}{-1}%
  \Test{100}{-1}%
  \bitsetReset{abc}%
  \Clear
  \Test{0}{0}%
  \Test{1}{1}%
  \Test{2}{2}%
  \Test{100}{100}%
  \Set
  \Test{0}{-1}%
  \Test{1}{-1}%
  \Test{100}{-1}%
  \bitsetSetBin{abc}{1}%
  \Clear
  \Test{0}{1}%
  \Test{1}{1}%
  \Test{2}{2}%
  \Test{100}{100}%
  \Set
  \Test{0}{0}%
  \Test{1}{-1}%
  \Test{100}{-1}%
  \bitsetSetBin{abc}{111000111000111000111}%
  \Clear
  \Test{0}{3}%
  \Test{1}{3}%
  \Test{2}{3}%
  \Test{3}{3}%
  \Test{4}{4}%
  \Test{5}{5}%
  \Test{6}{9}%
  \Test{7}{9}%
  \Test{8}{9}%
  \Test{9}{9}%
  \Test{10}{10}%
  \Test{11}{11}%
  \Test{12}{15}%
  \Test{13}{15}%
  \Test{14}{15}%
  \Test{15}{15}%
  \Test{16}{16}%
  \Test{17}{17}%
  \Test{18}{21}%
  \Test{19}{21}%
  \Test{20}{21}%
  \Test{21}{21}%
  \Test{22}{22}%
  \Test{100}{100}%
  \Set
  \Test{0}{0}%
  \Test{1}{1}%
  \Test{2}{2}%
  \Test{3}{6}%
  \Test{4}{6}%
  \Test{5}{6}%
  \Test{6}{6}%
  \Test{7}{7}%
  \Test{8}{8}%
  \Test{9}{12}%
  \Test{10}{12}%
  \Test{11}{12}%
  \Test{12}{12}%
  \Test{13}{13}%
  \Test{14}{14}%
  \Test{15}{18}%
  \Test{16}{18}%
  \Test{17}{18}%
  \Test{18}{18}%
  \Test{19}{19}%
  \Test{20}{20}%
  \Test{21}{-1}%
  \Test{22}{-1}%
  \Test{100}{-1}%
  \bitsetSetBin{abc}{1111111}%
  \Clear
  \Test{6}{7}%
  \Test{7}{7}%
  \Test{8}{8}%
  \Test{100}{100}%
  \Set
  \Test{6}{6}%
  \Test{7}{-1}%
  \Test{8}{-1}%
  \Test{100}{-1}%
  \bitsetSetBin{abc}{11111111}%
  \Clear
  \Test{7}{8}%
  \Test{8}{8}%
  \Test{9}{9}%
  \Test{100}{100}%
  \Set
  \Test{7}{7}%
  \Test{8}{-1}%
  \Test{9}{-1}%
  \Test{100}{-1}%
  \bitsetSetBin{abc}{111111111}%
  \Clear
  \Test{8}{9}%
  \Test{9}{9}%
  \Test{10}{10}%
  \Test{100}{100}%
  \Set
  \Test{8}{8}%
  \Test{9}{-1}%
  \Test{10}{-1}%
  \Test{100}{-1}%
  \bitsetSetBin{abc}{1111111111}%
  \Clear
  \Test{9}{10}%
  \Test{10}{10}%
  \Test{11}{11}%
  \Test{100}{100}%
  \Set
  \Test{9}{9}%
  \Test{10}{-1}%
  \Test{11}{-1}%
  \Test{100}{-1}%
\end{qstest}

\begin{qstest}{GetSetBitList}{GetSetBitList}
  \let\BS@abc\@undefined
  \expandafter\expandafter\expandafter\Expect
  \expandafter\expandafter\expandafter{%
    \bitsetGetSetBitList{abc}%
  }{}%
  \def\Test#1#2{%
    \bitsetSetBin{abc}{#1}%
    \expandafter\expandafter\expandafter\Expect
    \expandafter\expandafter\expandafter{%
      \bitsetGetSetBitList{abc}%
    }{#2}%
  }%
  \Test{0}{}%
  \Test{1}{0}%
  \Test{10}{1}%
  \Test{11}{0,1}%
  \Test{10110100}{2,4,5,7}%
  \Test{101101001010011}{0,1,4,6,9,11,12,14}%
\end{qstest}

\begin{qstest}{GetDec}{GetDec}
  \TestGetterUndefined\bitsetGetDec
  \def\Test#1#2{%
    \RevSet{abc}{#1}%
%<*noetex>
    \begingroup\expandafter\expandafter\expandafter\endgroup
%</noetex>
    \expandafter\expandafter\expandafter\Expect
    \expandafter\expandafter\expandafter{%
      \bitsetGetDec{abc}%
    }{#2}%
  }%
  \Test{0}{0}%
  \Test{1}{1}%
  \Test{10}{2}%
  \Test{11}{3}%
  \Test{100}{4}%
  \Test{101}{5}%
  \Test{110}{6}%
  \Test{111}{7}%
  \Test{1000}{8}%
  \Test{000111}{7}%
  \Test{1111111111111111%
        111111111111111}{2147483647}%
  \Test{0001111111111111111%
        111111111111111}{2147483647}%
  \Test{1000000000000000%
        0000000000000000}{2147483648}%
  \Test{1000000000000000%
        00000000000000000}{4294967296}%
  \Test{0001000000000000000%
        00000000000000000}{4294967296}%
  \Test{1100000000000000%
        00000000000000011}{6442450947}%
\end{qstest}

\begin{qstest}{Clear}{Clear}
  \def\Test#1#2#3{%
    \RevSet{abc}{#1}%
    \bitsetClear{abc}{#2}%
    \Expect*{\BS@abc}*{\Reverse#3!!}%
  }%
  \bitsetClear{abc}{2}%
  \RevCheck{abc}{0}%
  \TestErrorNegativeIndex{-1}{\bitsetClear{abc}{-1}}%
  \RevCheck{abc}{0}%
  \Test{0}{0}{0}%
  \Test{1}{0}{0}%
  \Test{111}{1}{101}%
  \Test{111}{30}{111}%
  \Test{0000111}{5}{0000111}% 111 would also be ok
  \Test{10000111}{5}{10000111}%
  \Test{1001001}{3}{1000001}%
\end{qstest}

\begin{qstest}{Set}{Set}
  \def\Test#1#2#3{%
    \RevSet{abc}{#1}%
    \bitsetSet{abc}{#2}%
    \Expect*{\BS@abc}*{\Reverse#3!!}%
  }%
  \bitsetSet{abc}{2}%
  \RevCheck{abc}{100}%
  \TestErrorNegativeIndex{-1}{\bitsetSet{abc}{-1}}%
  \RevCheck{abc}{100}%
  \Test{0}{0}{1}%
  \Test{1}{0}{1}%
  \Test{100}{1}{110}%
  \Test{111}{1}{111}%
  \Test{11}{1}{11}%
  \Test{11}{2}{111}%
  \Test{11}{3}{1011}%
  \Test{111}{10}{10000000111}%
  \Test{0000111}{5}{0100111}% 100111 would also be ok
  \Test{10000111}{5}{10100111}%
  \Test{1000001}{3}{1001001}%
  \Test{1001001}{3}{1001001}%
\end{qstest}

\begin{qstest}{Flip}{Flip}
  \def\Test#1#2#3{%
    \RevSet{abc}{#1}%
    \bitsetFlip{abc}{#2}%
    \Expect*{\BS@abc}*{\Reverse#3!!}%
  }%
  \bitsetFlip{abc}{2}%
  \RevCheck{abc}{100}%
  \TestErrorNegativeIndex{-1}{\bitsetFlip{abc}{-1}}%
  \RevCheck{abc}{100}%
  \Test{0}{0}{1}%
  \Test{1}{0}{0}%
  \Test{0}{2}{100}%
  \Test{100}{1}{110}%
  \Test{111}{1}{101}%
  \Test{11}{1}{1}%
  \Test{11}{2}{111}%
  \Test{11}{3}{1011}%
  \Test{111}{10}{10000000111}%
  \Test{0000111}{5}{0100111}% 100111 would also be ok
  \Test{10000111}{5}{10100111}%
  \Test{1000001}{3}{1001001}%
  \Test{1001001}{3}{1000001}%
  \Test{11111}{2}{11011}%
\end{qstest}

\begin{qstest}{SetValue}{SetValue}
  \def\Test#1#2{%
    \TestError{Invalid bit value (#2) not in range 0..1}{%
      \bitsetSetValue{abc}{#1}{#2}%
    }%
  }%
  \Test{0}{-1}%
  \Test{0}{2}%
  \Test{0}{10}%
  \def\Test#1#2#3{%
    \let\BS@abc\@undefined
    \bitsetSetValue{abc}{#1}{#2}%
    \bitsetSetBin{result}{#3}%
    \Expect*{\BS@abc}*{\BS@result}%
  }%
  \Test{0}{0}{0}%
  \Test{0}{1}{1}%
  \Test{1}{0}{0}%
  \Test{1}{1}{10}%
  \def\Test#1#2#3#4{%
    \bitsetSetBin{abc}{#1}%
    \bitsetSetBin{result}{#4}%
    \bitsetSetValue{abc}{#2}{#3}%
    \Expect*{\BS@abc}*{\BS@result}%
  }%
  \Test{0}{0}{0}{0}%
  \Test{0}{0}{0}{0}%
  \Test{0}{0}{1}{1}%
  \Test{0}{1}{0}{0}%
  \Test{0}{1}{1}{10}%
  \Test{1010}{2}{1}{1110}%
  \Test{1010}{4}{1}{11010}%
  \Test{1010}{6}{1}{1001010}%
  \Test{1010}{1}{0}{1000}%
  \Test{1010}{2}{0}{1010}%
  \Test{1010}{3}{0}{10}%
  \Test{1010}{4}{0}{1010}%
  \Test{1010}{6}{0}{1010}%
  \Test{1010}{2}{\csname iffalse\endcsname 0\else 1\fi}{1110}%
  \Test{1010}{1}{\csname iffalse\endcsname 1\else 0\fi}{1000}%
\end{qstest}

\begin{qstest}{IsDefined}{IsDefined}
  \let\BS@abc\@undefined
  \Expect*{\bitsetIsDefined{abc}{true}{false}}{false}%
  \bitsetReset{abc}%
  \Expect*{\bitsetIsDefined{abc}{true}{false}}{true}%
\end{qstest}

\begin{qstest}{IsEmpty}{IsEmpty}
  \let\BS@abc\@undefined
  \Expect*{\bitsetIsEmpty{abc}{true}{false}}{true}%
  \bitsetReset{abc}%
  \Expect*{\bitsetIsEmpty{abc}{true}{false}}{true}%
  \bitsetSet{abc}{1}%
  \Expect*{\bitsetIsEmpty{abc}{true}{false}}{false}%
\end{qstest}

\begin{qstest}{Equals}{Equals}
  \def\Test#1#2#3{%
    \Expect*{\bitsetEquals{#1}{#2}{true}{false}}{#3}%
  }%
  \let\BS@abc\@undefined
  \Test{abc}{abc}{true}%
  \Test{abc}{foo}{true}%
  \Test{foo}{abc}{true}%
  \bitsetReset{abc}%
  \Test{abc}{abc}{true}%
  \Test{abc}{foo}{false}%
  \Test{foo}{abc}{false}%
  \bitsetReset{foo}%
  \Test{abc}{foo}{true}%
  \Test{foo}{abc}{true}%
  \bitsetSet{abc}{4}%
  \Test{abc}{foo}{false}%
  \Test{foo}{abc}{false}%
  \bitsetFlip{foo}{4}%
  \Test{abc}{foo}{true}%
  \Test{foo}{abc}{true}%
\end{qstest}

\begin{qstest}{Intersects}{Intersects}
  \def\Test#1{%
    \Expect*{\bitsetIntersects{abc}{foo}{true}{false}}{#1}%
  }%
  \let\BS@abc\@undefined
  \let\BS@foo\@undefined
  \Test{false}%
  \Set{abc}{0}%
  \Test{false}%
  \Set{foo}{0}%
  \Test{false}%
  \let\BS@abc\@undefined
  \Test{false}%
  \Set{foo}{1}%
  \Test{false}%
  \Set{abc}{0}%
  \Test{false}%
  \Set{abc}{1}%
  \Test{true}%
  \let\BS@foo\@undefined
  \Test{false}%
  \Set{foo}{0}%
  \Test{false}%
  \def\Test#1#2#3{%
    \bitsetSetBin{abc}{#1}%
    \bitsetSetBin{foo}{#2}%
    \Expect*{\bitsetIntersects{abc}{foo}{true}{false}}{#3}%
  }%
  \Test{1010}{0101}{false}%
  \Test{0}{10}{false}%
  \Test{1}{11}{true}%
  \Test{11}{1}{true}%
  \Test{10}{1}{false}%
\end{qstest}

\begin{qstest}{And/AndNot/Or/Xor}{And/AndNot/Or/Xor}
  \def\@Test#1#2#3#4#5{%
    \begingroup
      #5%
      \begingroup
        \let\BS@foo\@undefined
        \csname bitset#1\endcsname{abc}{foo}%
        \CheckUndef{foo}%
        \Check{abc}{#2}%
      \endgroup
      \begingroup
        \bitsetReset{foo}%
        \csname bitset#1\endcsname{abc}{foo}%
        \Check{foo}{0}%
        \Check{abc}{#3}%
      \endgroup
      \begingroup
        \def\BS@foo{0101}%
        \csname bitset#1\endcsname{abc}{foo}%
        \Check{foo}{0101}%
        \Check{abc}{#4}%
      \endgroup
    \endgroup
  }%
  \def\Test#1{%
    \def\Op{#1}%
    \Test@
  }%
  \def\Test@#1#2#3#4#5#6#7#8#9{%
    \@Test\Op{#1}{#2}{#3}{%
      \let\BS@abc\@undefined
    }%
    \@Test\Op{#4}{#5}{#6}{%
      \bitsetReset{abc}%
    }%
    \@Test\Op{#7}{#8}{#9}{%
      \def\BS@abc{1001}%
    }%
  }%
  \Test{And}%
       {0}{0}{0}%
       {0}{0}{0}%
       {0}{0}{0001}%
  \Test{AndNot}%
       {0}{0}{0}%
       {0}{0}{0}%
       {1001}{1001}{1}%
  \Test{Or}%
       {0}{0}{0101}%
       {0}{0}{0101}%
       {1001}{1001}{1101}%
  \Test{Xor}%
       {0}{0}{0101}%
       {0}{0}{0101}%
       {1001}{1001}{11}%
  \def\Test#1#2#3{%
    \bitsetSetBin{abc}{#1}%
    \bitsetSetBin{foo}{#2}%
    \csname bitset\Op\endcsname{abc}{foo}%
    \RevCheck{foo}{#2}%
    \RevCheck{abc}{#3}%
  }%
  \def\Op{And}%
  \Test{1}{111}{1}%
  \Test{111}{1}{1}%
  \Test{10}{111}{10}%
  \Test{111}{10}{10}%
  \Test{111}{1000}{0}%
  \Test{1000}{111}{0}%
  \def\Op{AndNot}%
  \Test{1010}{11}{1000}%
  \Test{100}{100}{0}%
  \Test{111}{1111}{0}%
  \Test{100}{111}{0}%
  \def\Op{Or}%
  \Test{0}{0}{0}%
  \Test{1}{0}{1}%
  \Test{0}{1}{1}%
  \Test{1}{1}{1}%
  \Test{1000}{10}{1010}%
  \Test{10}{1000}{1010}%
  \def\Op{Xor}%
  \Test{0}{0}{0}%
  \Test{1}{0}{1}%
  \Test{0}{1}{1}%
  \Test{1}{1}{0}%
  \Test{1000}{10}{1010}%
  \Test{10}{1000}{1010}%
  \Test   {110011001100}%
       {111000111000111}%
       {111110100001011}%
  \Test{111000111000111}%
          {110011001100}%
       {111110100001011}%
\end{qstest}

\begin{qstest}{GetUndef}{GetUndef, GetBin, GetOct, GetHex}
  \def\TestUndef#1#2{%
    \let\BS@abc\@undefined
    \expandafter\expandafter\expandafter\Expect
    \expandafter\expandafter\expandafter{%
      \x{abc}{#1}%
    }{#2}%
  }%
  \let\x\bitsetGetBin
  \TestUndef{-1}{0}%
  \TestUndef{0}{0}%
  \TestUndef{1}{0}%
  \TestUndef{2}{00}%
  \TestUndef{8}{00000000}%
  \let\x\bitsetGetOct
  \TestUndef{-1}{0}%
  \TestUndef{0}{0}%
  \TestUndef{1}{0}%
  \TestUndef{2}{0}%
  \TestUndef{3}{0}%
  \TestUndef{4}{00}%
  \TestUndef{5}{00}%
  \TestUndef{6}{00}%
  \TestUndef{7}{000}%
  \TestUndef{8}{000}%
  \TestUndef{9}{000}%
  \TestUndef{10}{0000}%
  \let\x\bitsetGetHex
  \TestUndef{-1}{0}%
  \TestUndef{0}{0}%
  \TestUndef{1}{0}%
  \TestUndef{2}{0}%
  \TestUndef{3}{0}%
  \TestUndef{4}{0}%
  \TestUndef{5}{00}%
  \TestUndef{6}{00}%
  \TestUndef{7}{00}%
  \TestUndef{8}{00}%
  \TestUndef{9}{000}%
  \TestUndef{10}{000}%
  \TestUndef{12}{000}%
  \TestUndef{13}{0000}%
  \TestUndef{16}{0000}%
  \TestUndef{17}{00000}%
\end{qstest}

\begin{qstest}{SetBin}{SetBin}
  \def\Test#1#2{%
    \let\BS@abc\@undefined
    \bitsetSetBin{abc}{#1}%
    \expandafter\Expect\expandafter{\BS@abc}{#2}%
  }%
  \Test{}{0}%
  \Test{0}{0}%
  \Test{1}{1}%
  \Test{10}{01}%
  \Test{11}{11}%
  \Test{010}{01}%
  \Test{011}{11}%
  \Test{0010}{01}%
  \Test{1010}{0101}%
\end{qstest}

\begin{qstest}{SetOct}{SetOct}
  \def\Test#1#2{%
    \bitsetSetOct{abc}{#1}%
    \expandafter\Expect\expandafter{\BS@abc}{#2}%
  }%
  \Test{}{0}%
  \Test{0}{0}%
  \Test{000}{0}%
  \Test{1}{1}%
  \Test{001}{1}%
  \Test{010}{0001}%
  \Test{020}{00001}%
  \Test{42}{010001}%
  \Test{377}{11111111}%
  \Test{0377}{11111111}%
  \Test{76543210}{000100010110001101011111}%
  \Test{ 0 7 0 7 1 }{100111000111}%
\end{qstest}

\begin{qstest}{SetHex}{SetHex}
  \def\Test#1#2{%
    \bitsetSetHex{abc}{#1}%
    \expandafter\Expect\expandafter{\BS@abc}{#2}%
  }%
  \Test{}{0}%
  \Test{0}{0}%
  \Test{000}{0}%
  \Test{1}{1}%
  \Test{001}{1}%
  \Test{010}{00001}%
  \Test{020}{000001}%
  \Test{42}{0100001}%
  \Test{3F}{111111}%
  \Test{03F}{111111}%
  \Test{43210}{0000100001001100001}%
  \Test{98765}{10100110111000011001}%
  \Test{FEDCBA}{010111010011101101111111}%
  \Test{ 0 F 0 F 1 }{1000111100001111}%
\end{qstest}

\begin{qstest}{SetDec}{SetDec}
  \def\Test#1#2{%
    \bitsetSetDec{abc}{#1}%
    \expandafter\Expect\expandafter{\BS@abc}{#2}%
  }%
  \Test{}{0}%
  \Test{0}{0}%
  \Test{000}{0}%
  \Test{1}{1}%
  \Test{7}{111}%
  \Test{8}{0001}%
  \Test{001}{1}%
  \Test{010}{0101}%
  \Test{020}{00101}%
  \Test{53}{101011}%
  \Test{255}{11111111}%
  \Test{256}{000000001}%
  \Test{999999999}{111111111001001101011001110111}%
  \Test{1000000000}{000000000101001101011001110111}%
  \Test{4210987654}{01100001010010010111111101011111}%
  \Test{2147483647}{1111111111111111111111111111111}%
  \Test{2147483648}{00000000000000000000000000000001}%
\end{qstest}

\begin{qstest}{GetBin}{GetBin}
  \def\TestUndef#1#2{%
    \let\BS@abc\@undefined
    \expandafter\expandafter\expandafter\Expect
    \expandafter\expandafter\expandafter{%
      \bitsetGetBin{abc}{#1}%
    }{#2}%
  }%
  \TestUndef{-1}{0}%
  \TestUndef{0}{0}%
  \TestUndef{1}{0}%
  \TestUndef{2}{00}%
  \TestUndef{8}{00000000}%
  \def\Test#1#2{%
    \bitsetSetBin{abc}{#2}%
    \expandafter\expandafter\expandafter\Expect
    \expandafter\expandafter\expandafter{%
      \bitsetGetBin{abc}{#1}%
    }{#2}%
  }%
  \Test{-1}{0}%
  \Test{0}{0}%
  \Test{1}{0}%
  \Test{1}{1}%
  \Test{2}{01}%
  \Test{2}{10}%
  \Test{3}{010}%
  \Test{2}{00}%
  \Test{2}{01}%
  \Test{8}{00101100}%
  \Test{2}{10101}%
  \Test{-100}{11011}%
\end{qstest}

\begin{qstest}{GetOct}{GetOct}
  \def\Test#1#2#3{%
    \edef\x{\zap@space#1 \@empty}%
    \edef\x{\noexpand\bitsetSetBin{abc}{\x}}%
    \x
    \expandafter\expandafter\expandafter\Expect
    \expandafter\expandafter\expandafter{%
      \bitsetGetOct{abc}{#2}%
    }{#3}%
  }%
  \Test{111 110 101 100 011 010 001 000}{0}{76543210}%
  \Test{000 111}{0}{7}%
  \Test{101 000}{-1}{50}%
  \Test{111}{-1}{7}%
  \Test{111}{0}{7}%
  \Test{111}{1}{7}%
  \Test{111}{3}{7}%
  \Test{111}{4}{07}%
  \Test{111}{6}{07}%
  \Test{111}{7}{007}%
  \Test{111 010}{6}{72}%
  \Test{111 010}{7}{072}%
  \Test{011 111}{0}{37}%
  \Test{011 111}{6}{37}%
  \Test{011 111}{7}{037}%
  \Test{001 111}{0}{17}%
  \Test{001 111}{6}{17}%
  \Test{001 111}{7}{017}%
\end{qstest}

\begin{qstest}{GetHex}{GetHex}
  \def\Test#1#2#3{%
    \bitsetSetBin{abc}{#1}%
    \expandafter\expandafter\expandafter\Expect
    \expandafter\expandafter\expandafter{%
      \bitsetGetHex{abc}{#2}%
    }{#3}%
  }%
  \Test{1111 1110 1101 1100 1011 1010 1001 1000}{0}{FEDCBA98}%
  \Test{0111 0110 0101 0100 0011 0010 0001 0000}{0}{76543210}%
  \Test{0000 1111}{0}{F}%
  \Test{0101 0000}{-1}{50}%
  \Test{1111}{-1}{F}%
  \Test{1111}{0}{F}%
  \Test{1111}{1}{F}%
  \Test{1111}{4}{F}%
  \Test{1111}{5}{0F}%
  \Test{1111}{8}{0F}%
  \Test{1111}{9}{00F}%
  \Test{1111 0010}{8}{F2}%
  \Test{1111 0010}{9}{0F2}%
  \Test{0111 1111}{0}{7F}%
  \Test{0111 1111}{8}{7F}%
  \Test{0111 1111}{9}{07F}%
  \Test{0011 1111}{0}{3F}%
  \Test{0011 1111}{8}{3F}%
  \Test{0011 1111}{9}{03F}%
  \Test{0001 1111}{0}{1F}%
  \Test{0001 1111}{8}{1F}%
  \Test{0001 1111}{9}{01F}%
\end{qstest}

\begin{qstest}{Range}{Range}
  \TestError{%
    Wrong index numbers in range [9..8]\MessageBreak% hash-ok
    for clear/set/flip on bit set `abc'.\MessageBreak
    The lower index exceeds the upper index.\MessageBreak
    Canceling the operation as error recovery%
  }{%
    \bitsetSetRange{abc}{9}{8}%
  }%
  \def\TestErrorNegInd#1#2#3#4#5#6{%
    \TestError{%
      Negative index in range [#2..#3]\MessageBreak % hash-ok
      for \string\bitset #1Range on bit set `abc'.\MessageBreak
      Using [#4..#5] as error recovery% hash-ok
    }{%
      \csname bitset#1Range\endcsname{abc}{#2}{#3}%
      \global\let\BS@global\BS@abc
    }%
    \Check{global}{#6}%
  }%
  \Set{abc}{111}%
  \TestErrorNegInd{Clear}{-1}{0}{0}{0}{111}%
  \TestErrorNegInd{Clear}{0}{-1}{0}{0}{111}%
  \TestErrorNegInd{Clear}{-2}{2}{0}{2}{001}%
  \bitsetReset{abc}%
  \TestErrorNegInd{Set}{-1}{0}{0}{0}{0}%
  \TestErrorNegInd{Set}{0}{-1}{0}{0}{0}%
  \TestErrorNegInd{Set}{-2}{2}{0}{2}{11}%
  \Set{abc}{101}%
  \TestErrorNegInd{Flip}{-1}{0}{0}{0}{101}%
  \TestErrorNegInd{Flip}{0}{-1}{0}{0}{101}%
  \TestErrorNegInd{Flip}{-2}{2}{0}{2}{011}%
  \def\Test#1#2#3#4{%
    \bitsetSetBin{abc}{#1}%
    \csname bitset\TestOp Range\endcsname{abc}{#2}{#3}%
    \Expect*{\bitsetGetBin{abc}{0}}{#4}%
  }%
  \def\TestOp{Clear}%
  \Test{0}{0}{1}{0}%
  \Test{1111}{1}{2}{1101}%
  \Test{1111}{1}{3}{1001}%
  \Test{1111111100000000}{12}{14}{1100111100000000}%
  \def\TestOp{Set}%
  \Test{0}{0}{1}{1}%
  \Test{1000}{1}{2}{1010}%
  \Test{0}{1}{2}{10}%
  \Test{1}{12}{15}{111000000000001}%
  \Test{1111}{1}{3}{1111}%
  \Test{1000000000000000}{12}{14}{1011000000000000}%
  \def\TestOp{Flip}%
  \Test{0}{0}{1}{1}%
  \Test{1}{0}{1}{0}%
  \Test{10101010}{1}{5}{10110100}%
  \def\Test#1#2#3#4#5{%
    \bitsetSetBin{abc}{#1}%
    \bitsetSetValueRange{abc}{#2}{#3}{#4}%
    \Expect*{\bitsetGetBin{abc}{0}}{#5}%
  }%
  \Test{0}{0}{1}{0}{0}%
  \Test{0}{0}{1}{1}{1}%
  \Test{1010}{1}{3}{0}{1000}%
  \Test{1010}{1}{3}{1}{1110}%
\end{qstest}

\begin{qstest}{ShiftLeft/ShiftRight}{ShiftLeft/ShiftRight}
  \def\@Test#1#2{%
    \let\BS@abc\@undefined
    \csname bitsetShift#1\endcsname{abc}{#2}%
    \Expect*{\BS@abc}{0}%
  }%
  \def\Test#1{%
    \@Test{Left}{#1}%
    \@Test{Right}{#1}%
  }%
  \Test{-16}%
  \Test{-1}%
  \Test{0}%
  \Test{1}%
  \Test{16}%
  \def\Test#1#2#3{%
    \bitsetSetBin{abc}{#1}%
    \bitsetSetBin{result}{#3}%
    \csname bitsetShift\Op\endcsname{abc}{#2}%
    \Expect*{\bitsetGetBin{abc}{0}}*{\bitsetGetBin{result}{0}}%
  }%
  \def\Op{Left}%
  \Test{0}{0}{0}%
  \Test{0}{1}{0}%
  \Test{0}{-1}{0}%
  \Test{1}{0}{1}%
  \Test{1}{1}{10}%
  \Test{1}{-1}{0}%
  \Test{10}{1}{100}%
  \Test{10}{-1}{1}%
  \Test{1}{32}{100000000000000000000000000000000}%
  \Test{1}{-100}{0}%
  \def\Op{Right}%
  \Test{0}{0}{0}%
  \Test{0}{1}{0}%
  \Test{0}{-1}{0}%
  \Test{1}{0}{1}%
  \Test{1}{1}{0}%
  \Test{1}{-1}{10}%
  \Test{10}{1}{1}%
  \Test{10}{-1}{100}%
  \Test{1}{-32}{100000000000000000000000000000000}%
  \Test{1}{100}{0}%
  \Test{110110110110110}{10}{11011}%
  \Test{110110110110110}{100}{0}%
  \Test{1}{100000}{0}%
\end{qstest}

\begin{qstest}{Profile: Set}{Profile: Set}
  \bitsetSet{abc}{4095}%
  \global\let\BS@global\BS@abc
\end{qstest}

\begin{qstest}{Profile: Get}{Profile: Get}
  \edef\x{\bitsetGet{global}{4095}}%
\end{qstest}

\begin{document}
\end{document}
%</test2>
%    \end{macrocode}
%
% \section{Installation}
%
% \subsection{Download}
%
% \paragraph{Package.} This package is available on
% CTAN\footnote{\url{ftp://ftp.ctan.org/tex-archive/}}:
% \begin{description}
% \item[\CTAN{macros/latex/contrib/oberdiek/bitset.dtx}] The source file.
% \item[\CTAN{macros/latex/contrib/oberdiek/bitset.pdf}] Documentation.
% \end{description}
%
%
% \paragraph{Bundle.} All the packages of the bundle `oberdiek'
% are also available in a TDS compliant ZIP archive. There
% the packages are already unpacked and the documentation files
% are generated. The files and directories obey the TDS standard.
% \begin{description}
% \item[\CTAN{install/macros/latex/contrib/oberdiek.tds.zip}]
% \end{description}
% \emph{TDS} refers to the standard ``A Directory Structure
% for \TeX\ Files'' (\CTAN{tds/tds.pdf}). Directories
% with \xfile{texmf} in their name are usually organized this way.
%
% \subsection{Bundle installation}
%
% \paragraph{Unpacking.} Unpack the \xfile{oberdiek.tds.zip} in the
% TDS tree (also known as \xfile{texmf} tree) of your choice.
% Example (linux):
% \begin{quote}
%   |unzip oberdiek.tds.zip -d ~/texmf|
% \end{quote}
%
% \paragraph{Script installation.}
% Check the directory \xfile{TDS:scripts/oberdiek/} for
% scripts that need further installation steps.
% Package \xpackage{attachfile2} comes with the Perl script
% \xfile{pdfatfi.pl} that should be installed in such a way
% that it can be called as \texttt{pdfatfi}.
% Example (linux):
% \begin{quote}
%   |chmod +x scripts/oberdiek/pdfatfi.pl|\\
%   |cp scripts/oberdiek/pdfatfi.pl /usr/local/bin/|
% \end{quote}
%
% \subsection{Package installation}
%
% \paragraph{Unpacking.} The \xfile{.dtx} file is a self-extracting
% \docstrip\ archive. The files are extracted by running the
% \xfile{.dtx} through \plainTeX:
% \begin{quote}
%   \verb|tex bitset.dtx|
% \end{quote}
%
% \paragraph{TDS.} Now the different files must be moved into
% the different directories in your installation TDS tree
% (also known as \xfile{texmf} tree):
% \begin{quote}
% \def\t{^^A
% \begin{tabular}{@{}>{\ttfamily}l@{ $\rightarrow$ }>{\ttfamily}l@{}}
%   bitset.sty & tex/generic/oberdiek/bitset.sty\\
%   bitset.pdf & doc/latex/oberdiek/bitset.pdf\\
%   test/bitset-test1.tex & doc/latex/oberdiek/test/bitset-test1.tex\\
%   test/bitset-test2.tex & doc/latex/oberdiek/test/bitset-test2.tex\\
%   test/bitset-test3.tex & doc/latex/oberdiek/test/bitset-test3.tex\\
%   bitset.dtx & source/latex/oberdiek/bitset.dtx\\
% \end{tabular}^^A
% }^^A
% \sbox0{\t}^^A
% \ifdim\wd0>\linewidth
%   \begingroup
%     \advance\linewidth by\leftmargin
%     \advance\linewidth by\rightmargin
%   \edef\x{\endgroup
%     \def\noexpand\lw{\the\linewidth}^^A
%   }\x
%   \def\lwbox{^^A
%     \leavevmode
%     \hbox to \linewidth{^^A
%       \kern-\leftmargin\relax
%       \hss
%       \usebox0
%       \hss
%       \kern-\rightmargin\relax
%     }^^A
%   }^^A
%   \ifdim\wd0>\lw
%     \sbox0{\small\t}^^A
%     \ifdim\wd0>\linewidth
%       \ifdim\wd0>\lw
%         \sbox0{\footnotesize\t}^^A
%         \ifdim\wd0>\linewidth
%           \ifdim\wd0>\lw
%             \sbox0{\scriptsize\t}^^A
%             \ifdim\wd0>\linewidth
%               \ifdim\wd0>\lw
%                 \sbox0{\tiny\t}^^A
%                 \ifdim\wd0>\linewidth
%                   \lwbox
%                 \else
%                   \usebox0
%                 \fi
%               \else
%                 \lwbox
%               \fi
%             \else
%               \usebox0
%             \fi
%           \else
%             \lwbox
%           \fi
%         \else
%           \usebox0
%         \fi
%       \else
%         \lwbox
%       \fi
%     \else
%       \usebox0
%     \fi
%   \else
%     \lwbox
%   \fi
% \else
%   \usebox0
% \fi
% \end{quote}
% If you have a \xfile{docstrip.cfg} that configures and enables \docstrip's
% TDS installing feature, then some files can already be in the right
% place, see the documentation of \docstrip.
%
% \subsection{Refresh file name databases}
%
% If your \TeX~distribution
% (\teTeX, \mikTeX, \dots) relies on file name databases, you must refresh
% these. For example, \teTeX\ users run \verb|texhash| or
% \verb|mktexlsr|.
%
% \subsection{Some details for the interested}
%
% \paragraph{Attached source.}
%
% The PDF documentation on CTAN also includes the
% \xfile{.dtx} source file. It can be extracted by
% AcrobatReader 6 or higher. Another option is \textsf{pdftk},
% e.g. unpack the file into the current directory:
% \begin{quote}
%   \verb|pdftk bitset.pdf unpack_files output .|
% \end{quote}
%
% \paragraph{Unpacking with \LaTeX.}
% The \xfile{.dtx} chooses its action depending on the format:
% \begin{description}
% \item[\plainTeX:] Run \docstrip\ and extract the files.
% \item[\LaTeX:] Generate the documentation.
% \end{description}
% If you insist on using \LaTeX\ for \docstrip\ (really,
% \docstrip\ does not need \LaTeX), then inform the autodetect routine
% about your intention:
% \begin{quote}
%   \verb|latex \let\install=y\input{bitset.dtx}|
% \end{quote}
% Do not forget to quote the argument according to the demands
% of your shell.
%
% \paragraph{Generating the documentation.}
% You can use both the \xfile{.dtx} or the \xfile{.drv} to generate
% the documentation. The process can be configured by the
% configuration file \xfile{ltxdoc.cfg}. For instance, put this
% line into this file, if you want to have A4 as paper format:
% \begin{quote}
%   \verb|\PassOptionsToClass{a4paper}{article}|
% \end{quote}
% An example follows how to generate the
% documentation with pdf\LaTeX:
% \begin{quote}
%\begin{verbatim}
%pdflatex bitset.dtx
%makeindex -s gind.ist bitset.idx
%pdflatex bitset.dtx
%makeindex -s gind.ist bitset.idx
%pdflatex bitset.dtx
%\end{verbatim}
% \end{quote}
%
% \section{Catalogue}
%
% The following XML file can be used as source for the
% \href{http://mirror.ctan.org/help/Catalogue/catalogue.html}{\TeX\ Catalogue}.
% The elements \texttt{caption} and \texttt{description} are imported
% from the original XML file from the Catalogue.
% The name of the XML file in the Catalogue is \xfile{bitset.xml}.
%    \begin{macrocode}
%<*catalogue>
<?xml version='1.0' encoding='us-ascii'?>
<!DOCTYPE entry SYSTEM 'catalogue.dtd'>
<entry datestamp='$Date$' modifier='$Author$' id='bitset'>
  <name>bitset</name>
  <caption>Handle bit-vector datatype.</caption>
  <authorref id='auth:oberdiek'/>
  <copyright owner='Heiko Oberdiek' year='2007,2011'/>
  <license type='lppl1.3'/>
  <version number='1.1'/>
  <description>
    This package defines and implements the data type bit set,
    a vector of bits. The size of the vector may grow dynamically.
    Individual bits can be manipulated.
    <p/>
    The package is part of the <xref refid='oberdiek'>oberdiek</xref> bundle.
  </description>
  <documentation details='Package documentation'
      href='ctan:/macros/latex/contrib/oberdiek/bitset.pdf'/>
  <ctan file='true' path='/macros/latex/contrib/oberdiek/bitset.dtx'/>
  <miktex location='oberdiek'/>
  <texlive location='oberdiek'/>
  <install path='/macros/latex/contrib/oberdiek/oberdiek.tds.zip'/>
</entry>
%</catalogue>
%    \end{macrocode}
%
% \begin{History}
%   \begin{Version}{2007/09/28 v1.0}
%   \item
%     First version.
%   \end{Version}
%   \begin{Version}{2011/01/30 v1.1}
%   \item
%     Already loaded package files are not input in \hologo{plainTeX}.
%   \end{Version}
% \end{History}
%
% \PrintIndex
%
% \Finale
\endinput
|
% \end{quote}
% Do not forget to quote the argument according to the demands
% of your shell.
%
% \paragraph{Generating the documentation.}
% You can use both the \xfile{.dtx} or the \xfile{.drv} to generate
% the documentation. The process can be configured by the
% configuration file \xfile{ltxdoc.cfg}. For instance, put this
% line into this file, if you want to have A4 as paper format:
% \begin{quote}
%   \verb|\PassOptionsToClass{a4paper}{article}|
% \end{quote}
% An example follows how to generate the
% documentation with pdf\LaTeX:
% \begin{quote}
%\begin{verbatim}
%pdflatex bitset.dtx
%makeindex -s gind.ist bitset.idx
%pdflatex bitset.dtx
%makeindex -s gind.ist bitset.idx
%pdflatex bitset.dtx
%\end{verbatim}
% \end{quote}
%
% \section{Catalogue}
%
% The following XML file can be used as source for the
% \href{http://mirror.ctan.org/help/Catalogue/catalogue.html}{\TeX\ Catalogue}.
% The elements \texttt{caption} and \texttt{description} are imported
% from the original XML file from the Catalogue.
% The name of the XML file in the Catalogue is \xfile{bitset.xml}.
%    \begin{macrocode}
%<*catalogue>
<?xml version='1.0' encoding='us-ascii'?>
<!DOCTYPE entry SYSTEM 'catalogue.dtd'>
<entry datestamp='$Date$' modifier='$Author$' id='bitset'>
  <name>bitset</name>
  <caption>Handle bit-vector datatype.</caption>
  <authorref id='auth:oberdiek'/>
  <copyright owner='Heiko Oberdiek' year='2007,2011'/>
  <license type='lppl1.3'/>
  <version number='1.1'/>
  <description>
    This package defines and implements the data type bit set,
    a vector of bits. The size of the vector may grow dynamically.
    Individual bits can be manipulated.
    <p/>
    The package is part of the <xref refid='oberdiek'>oberdiek</xref> bundle.
  </description>
  <documentation details='Package documentation'
      href='ctan:/macros/latex/contrib/oberdiek/bitset.pdf'/>
  <ctan file='true' path='/macros/latex/contrib/oberdiek/bitset.dtx'/>
  <miktex location='oberdiek'/>
  <texlive location='oberdiek'/>
  <install path='/macros/latex/contrib/oberdiek/oberdiek.tds.zip'/>
</entry>
%</catalogue>
%    \end{macrocode}
%
% \begin{History}
%   \begin{Version}{2007/09/28 v1.0}
%   \item
%     First version.
%   \end{Version}
%   \begin{Version}{2011/01/30 v1.1}
%   \item
%     Already loaded package files are not input in \hologo{plainTeX}.
%   \end{Version}
% \end{History}
%
% \PrintIndex
%
% \Finale
\endinput
|
% \end{quote}
% Do not forget to quote the argument according to the demands
% of your shell.
%
% \paragraph{Generating the documentation.}
% You can use both the \xfile{.dtx} or the \xfile{.drv} to generate
% the documentation. The process can be configured by the
% configuration file \xfile{ltxdoc.cfg}. For instance, put this
% line into this file, if you want to have A4 as paper format:
% \begin{quote}
%   \verb|\PassOptionsToClass{a4paper}{article}|
% \end{quote}
% An example follows how to generate the
% documentation with pdf\LaTeX:
% \begin{quote}
%\begin{verbatim}
%pdflatex bitset.dtx
%makeindex -s gind.ist bitset.idx
%pdflatex bitset.dtx
%makeindex -s gind.ist bitset.idx
%pdflatex bitset.dtx
%\end{verbatim}
% \end{quote}
%
% \section{Catalogue}
%
% The following XML file can be used as source for the
% \href{http://mirror.ctan.org/help/Catalogue/catalogue.html}{\TeX\ Catalogue}.
% The elements \texttt{caption} and \texttt{description} are imported
% from the original XML file from the Catalogue.
% The name of the XML file in the Catalogue is \xfile{bitset.xml}.
%    \begin{macrocode}
%<*catalogue>
<?xml version='1.0' encoding='us-ascii'?>
<!DOCTYPE entry SYSTEM 'catalogue.dtd'>
<entry datestamp='$Date$' modifier='$Author$' id='bitset'>
  <name>bitset</name>
  <caption>Handle bit-vector datatype.</caption>
  <authorref id='auth:oberdiek'/>
  <copyright owner='Heiko Oberdiek' year='2007,2011'/>
  <license type='lppl1.3'/>
  <version number='1.1'/>
  <description>
    This package defines and implements the data type bit set,
    a vector of bits. The size of the vector may grow dynamically.
    Individual bits can be manipulated.
    <p/>
    The package is part of the <xref refid='oberdiek'>oberdiek</xref> bundle.
  </description>
  <documentation details='Package documentation'
      href='ctan:/macros/latex/contrib/oberdiek/bitset.pdf'/>
  <ctan file='true' path='/macros/latex/contrib/oberdiek/bitset.dtx'/>
  <miktex location='oberdiek'/>
  <texlive location='oberdiek'/>
  <install path='/macros/latex/contrib/oberdiek/oberdiek.tds.zip'/>
</entry>
%</catalogue>
%    \end{macrocode}
%
% \begin{History}
%   \begin{Version}{2007/09/28 v1.0}
%   \item
%     First version.
%   \end{Version}
%   \begin{Version}{2011/01/30 v1.1}
%   \item
%     Already loaded package files are not input in \hologo{plainTeX}.
%   \end{Version}
% \end{History}
%
% \PrintIndex
%
% \Finale
\endinput
|
% \end{quote}
% Do not forget to quote the argument according to the demands
% of your shell.
%
% \paragraph{Generating the documentation.}
% You can use both the \xfile{.dtx} or the \xfile{.drv} to generate
% the documentation. The process can be configured by the
% configuration file \xfile{ltxdoc.cfg}. For instance, put this
% line into this file, if you want to have A4 as paper format:
% \begin{quote}
%   \verb|\PassOptionsToClass{a4paper}{article}|
% \end{quote}
% An example follows how to generate the
% documentation with pdf\LaTeX:
% \begin{quote}
%\begin{verbatim}
%pdflatex bitset.dtx
%makeindex -s gind.ist bitset.idx
%pdflatex bitset.dtx
%makeindex -s gind.ist bitset.idx
%pdflatex bitset.dtx
%\end{verbatim}
% \end{quote}
%
% \section{Catalogue}
%
% The following XML file can be used as source for the
% \href{http://mirror.ctan.org/help/Catalogue/catalogue.html}{\TeX\ Catalogue}.
% The elements \texttt{caption} and \texttt{description} are imported
% from the original XML file from the Catalogue.
% The name of the XML file in the Catalogue is \xfile{bitset.xml}.
%    \begin{macrocode}
%<*catalogue>
<?xml version='1.0' encoding='us-ascii'?>
<!DOCTYPE entry SYSTEM 'catalogue.dtd'>
<entry datestamp='$Date$' modifier='$Author$' id='bitset'>
  <name>bitset</name>
  <caption>Handle bit-vector datatype.</caption>
  <authorref id='auth:oberdiek'/>
  <copyright owner='Heiko Oberdiek' year='2007,2011'/>
  <license type='lppl1.3'/>
  <version number='1.1'/>
  <description>
    This package defines and implements the data type bit set,
    a vector of bits. The size of the vector may grow dynamically.
    Individual bits can be manipulated.
    <p/>
    The package is part of the <xref refid='oberdiek'>oberdiek</xref> bundle.
  </description>
  <documentation details='Package documentation'
      href='ctan:/macros/latex/contrib/oberdiek/bitset.pdf'/>
  <ctan file='true' path='/macros/latex/contrib/oberdiek/bitset.dtx'/>
  <miktex location='oberdiek'/>
  <texlive location='oberdiek'/>
  <install path='/macros/latex/contrib/oberdiek/oberdiek.tds.zip'/>
</entry>
%</catalogue>
%    \end{macrocode}
%
% \begin{History}
%   \begin{Version}{2007/09/28 v1.0}
%   \item
%     First version.
%   \end{Version}
%   \begin{Version}{2011/01/30 v1.1}
%   \item
%     Already loaded package files are not input in \hologo{plainTeX}.
%   \end{Version}
% \end{History}
%
% \PrintIndex
%
% \Finale
\endinput
